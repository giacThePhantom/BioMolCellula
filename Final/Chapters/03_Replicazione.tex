\chapter{Replicazione del DNA e il mantenimento dei telomeri}
Nella duplicazione del DNA si identifica un filamento stampo e uno di nuova sintesi. I due filamenti si identificano anche per la loro polarit\`a $3'-5'$ e viceversa. I filamenti stampo
hanno polarit\`a diverse, importante per il funzionamento della DNA polimerasi. Utilizzando il filamento stampo e l'appaiamento delle basi. Utilizzando un filamento primer la DNA
polimerasi introduce le basi complementari creando un nuovo filamento utilizzando quello parentale come stampo e va a promuovere la formazione del legame fosfodiesterico tra il gruppo
fosfato del nucleotide libero e quello OH presente al $3'$ viene scisso ATP con la liberazione di pirofosfato. 
\section{Mutazioni}
Il processo di replicazione \`e accurato ma non perfetto. Senza mutazioni non ci potrebbe essere evoluzione. Il processo pu\`o funzionare in maniera non del tutto precisa. Nei batteri
si possono introdurre errori ogni $10^{10}$ nucleotidici di $3-4$ cambi. Le mutazioni possono essere geniche (singoli basi), cromosomico e genomiche. Mutazione genica che normalmente 
avvengono a carico dei singoli nucleotidi, possono essere sinonima o di sostituzione: cambiamento di una base che non comporta un cambio amminoacidico, mutazioni di senso errato: il
codone viene sostituito con uno che codifica per un altro amminoacido, mutazioni non senso: si forma un codone di stop prematuro, mutazioni frameshift: si sposta l'ordine di lettura, 
mutazioni per frequenze ripetute che presenti nelle zone promotrici e la duplicazione della DNA polimerasi e aumenta il numero delle sequenze ripetute e pu\`o portare problemi in quanto
pu\`o rompere i meccanismi di replicazione (causare Corea di Hintington e la sindrome dell'X fragile). La mutazione cromosomica si riarrangiano parti di cromosoma con traslocazioni 
Inter cromosomiche, tipico di tumori e leucemie (poco tollerata), nelle mutazioni genomiche varia il numero di cromosomi (sindrome di Down, trisomia 21, 18). 
\section{Modello di duplicazione}
\subsection{Modello semi-conservativo}
La cellula figlia riceve una catena neosintetizzata e una parentale.
\subsection{Modello conservativo}
La cellula madre duplica l'intera catena del DNA e da origine a due cellule figlie una con il filamento parentale e una con il filamento neosintetizzato. 
\subsection{Modello dispersivo}
Ciascun filamento \`e un mosaico tra il DNA parentale e quello neosintetizzato. 
Per la conferma sperimentale si usano colonie batteriche contenenti azoto pesante \ce{^14N} i batteri venivano cresciuti in presenza di \ce{^15N} per una generazione in modo che tutte le
cellule lo introdussero, dopo di che vengono messe in un terreno con azoto leggero. Alla  duplicazione si incorpora azoto leggero. Utilizzando tecniche di centrifugazione si possono 
discriminare quelle contenenti azoto leggero e pesante. Si osserva che nella prima generazione si ha una miscela costituita da azoto leggero e pesante e nella generazione successiva 
scompare quella pesante e compare quella leggera e la miscela, risultato che conferma il modello semi-conservativo. 

La duplicazione del DNA avviene in un origine nei batteri nell'origine di replicazione dove vengono reclutata la macchina per la duplicazione, si forma la bolla di replicazione: 
le due catene si separano, viene reclutata la macchina di duplicazione e la duplicazione inizia e si forma la struttura theta che si allarga sempre di pi\`u fino a creare due catene
di DNA costituite da una catena parentale e una neosintetizzata. A livello di microscopia elettronica si pu\`o dare un analogo radioattivo (timina triziata) incorporata nel DNA di nuova
sintesi che rende visibile dove avviene la sintesi e incorporazione. L'origine di replicazione \`e caratterizzata attraverso esperimenti che comportano la digestione da un enzima di 
restrizione che taglia e i frammenti vengono legati e si va a vedere se il frammento del plasmide va incontro a duplicazione oppure no. Isolando il frammento e sequenziandolo \`e 
possibile capire la struttura del DNA responsabile dell'origine di replicazione. Negli eucarioti ci sono diverse origine di replicazione, alcune precoci nel ciclo cellulare, altre
attivate pi\`u tardivamente in  base alla struttura in cui si trova il DNA nel tempo di duplicazione: livello di compattazione. Ci sono diverse origine in quanto il DNA \`e molto lungo.
Attivando pi\`u origini di replicazione si riduce il tempo che la polimerasi impiega a completare la replicazione. Le bolle di replicazione si fonderanno successivamente tra di loro.
Le origini di replicazione o fabbriche di replicazione. L'apparato deputato alla replicazione \`e complesso ed \`e detto di cui l'attore principale \`e la DNA polimerasi a forma di mano
in cui il DNA si lega al palmo con strutture che permettono l'entrata dei nucleotidi incorporati e permette alle due catene di rimanere separate e domini di controllo della qualit\`a 
della duplicazione permettendo di notare incorporazioni errate. La DNA polimerasi lavora in modo processivo; ina volta che si attacca al DNA va a replicare lo stampo in maniera 
continuativa. UN enzima non processivo si attacca, incorpora, si stacca e si riattacca. La DNA polimerasi per la duplicazione del DNA lavora in maniera processiva. La DNA polimerasi 
\`e un enzima complesso e si dice oloenzima, costituito da subunit\`a con ruoli diversi. Aggiunge nucleotidi attivit\`a polimerasica corrispondenti a quelli presenti sulla catena stampo, 
ha attivit\`a esonucleasica $3'- 5'$ ed esonucleasica $5'=3'$ che permettono alla polimerasi di riconoscere errori. Esonucleasica: quando viene introdotto un nucleotide non complementare
viene rimosso e questa avviene in entrambe le direzioni. Il ruolo maggiore di duplicazione \`e data dalla DNA polimearsi I mentre le altre hanno ruolo in altri processi, in particolare 
nella riparazione del DNA rotture a singolo o doppio strand, cosa vera anche negli euocarioti. Considerando la bolla di replicazione la duplicazione del DNA non pu\`o avvenire in entrambe
le direzioni in quanto pu\`o incorporare nucleotidi solo in direzione $5'-3'$ solo un filamento viene duplicato. Mentre un filamento viene duplicato in maniera duplicativa sull'altro 
si formano dei piccoli frammenti di piccola dimensione detti frammenti di Okazaki, frammenti di DNA duplicato in direzione $5'-3'$. Si identificano uno strand leading o duplicato in 
maniera continuativa e uno in maniera discontinuativa con i frammenti o lagging. Questi frammenti sono formati preceduti dalla fomrazione di primer a innesco cosituiti da RNA importanti
in quanto la DNA polimerasi \`e in gradi di introdurre i nucleotidi solo se trova un gruppo Oh libero a livello del primo nucleotide. NOn \`e in grado di iniziare la sintesi da sola 
ma necessita di un filamento primer che permette l'inizio del processo di replicazione (a differenza dell'RNA polimerasi). La DNA polimerasi necessita di un innesco fornito da primer 
prodotti da una DNA primasi che si lega =, forma un innesco, brevi frammenti di RNA che forniscono il gruppo OH alla DNA polimerasi che si lega e inizia a polimerizzare in direzione 
$3'-5'$. La DNA polimerasi produce un filamento di Okazaki, poi viene rimosso da un enzima dall'una primer e dopo al gruppo OH libero la polimerasi pu\`o riempire il buco. Il complesso
\`e unico, uno per la duplicazione del filamento continuo e uno per il filamento discontinuo. In quello discontinuo forma una ansa che forma il modello a trombone. Ci sono nel complesso
pertanto due DNA polimerasi, una primasi per il primer e il filamento discontinuo, le proteine che si legano al DNA a filamento discontinuo lo proteggono e impediscono la formazione di
strutture secondarie che bloccano la polimerasi. Ci sono le elicasi che svolgono il DNA e le topoisomerasi che risolvono i superavvolgimenti e vengono reclutate tutte le proteine 
coinvolte nella formazione dei nucleosomi e dell'ottamero istonico. 

\section{Lezione 7}
La maggior parte delle DNA polimerasi sono coinvolte nelle riparazione dei danni del DNA che possono essere di diversi tipi. Esistono dei danni legati alla rottura delle catene dei
DNA e la cellula fa intervenire dei meccanismi specifiche per questi tipi di danno, meccanismo fondamentale in quanto pu\`o portare a mutazioni che se non riparate possono in alcuni 
casi portare all'apoptosi della cellula. Questo sta alla base dell'azione di molti farmaci tumorali che creano multiple rotture che possono portare alla morte cellulare e alla morte della
cellula tumorale. Il modello utilizzato che corrisponde a quello semi conservativo in cui la catena parentale si divide e funzionano da stampo, le cellule figlie ricevono un filamento
parentale e uno stampo. L'inizio della duplicazione del DNA avviene durante la fase S dove il DNA viene duplicato, si trova la preparazione che avviene nella fase G1 in cui si preparano
i siti di replicazione, i punti in cui le due catene iniziano a separarsi e viene reclutata la macchina di replicazione. Nel caso dei procarioti inizia nell'origine di replicazione, 
mentre negli eucarioti ci sono pi\`u origini di replicazione. Le forcelle di replicazione non sono distribuite omogeneamente ma sembrano raggrupparsi in foci di replicazione che
contengono ad alte concentrazione i fattori coinvolti nella replicazione del DNA. Il bromodiossiuridina \`e un analogo della timina che viene accoppiato con una molecola fluorescente 
attraverso un anticorpo. Ci sono pi\`u forcelle di replicazione distanziate da circa 50-70 kilobasi e gli essere umani possono avere dalle 100000 ai 100000 origini di replicazione. Non
sono determinate da una sequenza specifica ma sono ricche di A e T in quanto possiedono meno legami a idrogeno e sono pertanto pi\`u facili da separare. L'enzima principale coinvolto
nella duplicazione \`e la DNA polimerasi un enzima composto da una componente enzimatica (attivit\`a polimerasica) accompagnato da altri cofattori che lo aiutano a duplicare il DNA
e conferiscno altre caratteristiche: correggere gli errori un nucleotide incorporato in maniera errata l'enzima lo riconosce, rompe il legame fosfodiesterico e rimuovere la base 
scorretta. \`E molto efficiente: $10^9 - 10^{10}$ basi per un errore. La DNA polimerasi comune anche all'RNA polimerasi che produce RNA \`e che sono processive: una volta che si attacca
al DNA copia la catena stampo in maniera continuativa. La DNA polimerasi non ha una grande affinit\`a: tenderebbe a staccarsi dopo poche basi. Questa viene aumentata dalla pinza 
scorrevole che scorre sul DNA durante la duplicazione consentendo alla DNA polimerasi di rimanere attaccata al DNA. La funzione della DNA polimerasi \`e legata alla sua struttura a mano
con l'alloggiamento del DNA, un punto in cui gli ATP entrano e un punto in cui le catene escono. Il compito \`e di polimerizzare e aggiungere nucleotidi in $5'-3'$ usando un gruppo OH
libero in $3'$ di un filamento primer di innesco usa un DNTP con un trifosfato con un legame fosfodiesterico e il pirofosfato. La formazione di dlegame \`e spinta dalla scissione del
pirososfato con la formazione di due fosfati. La DNA polimerasi \`e in grado di sintetizzare in una sola direzione: pertanto per sintetizzare il filamento in direzione opposta si 
formano i frammenti di Okazaki con dimesioni variabili: da 1000-2000 per eucarioti e 100-200 per procarioti, creati attraverso la primasi che aggiunge degli inneschi formati da RNA, 
primer di dimensioni ridotte 10-20-30 che forniscono il gruppo $3'$ OH su cui la polimerasi si inserisce. Si formano frammenti con il DNA sintetizzato e una sequenza formata da RNA che
viene rimossa grazie all'RNAasi H che degrada l'RNA e riconosce l'ibrido e degrada rimuovendo i primer. A questo punto si trova un buco dove c'era il primer colmato dalla DNA polimerasi
e i frammenti vengono legati tra di loro da una DNA ligasi. In questo modo il filamento lagging produce un filamento duplicato continuo. Ci sono altri componenti reclutati durante la 
formazione dell'origine della forcella: nel caso di escherichia coli la DNA polimerasi III a cui si aggiungono altre DNA polimerasi per ruoli accessori. Il riconoscimento dell'escherichia
coli dell'origine di replicazione che \`e caratterizzata e riconosciuta da parte della DNAa, una proteina che contiene quattro ripetizioni di nove nucleotidi e tre ripetizioni di tredici
paia di basi a monte delle ripetizioni. La DNAa si attacca alle ripetizioni di nove e si lega a quelle di 13 facilitando la denaturazione del DNA e permettendo l'assemblaggio delle
proteine replicative: la DNAb e la DNAc che formano il complesso di preinnesco. La DNAb \`e un'elicasi che legando l'ATP ha il ruolo di svolgere e aprire le due catene di DNA che permette
il reclutamento della DNA primasi che crea i primer a RNA sui filamenti. Infine viene reclutata la DNA polimerasi. 

Negli eucarioti le cose sono pi\`u complicate: si ha una DNA polimerasi, fattori elicasi, primasi eccetera. Il processo di duplicazione \`e differente dalla tempistica in quanto il DNA
presenta un compattamento della cromatina accentuato e le zone molto compatte devono essere duplicate alla fine della fase S, molto tardive in quanto si devono decompattare le zone e 
intervengono fattori che decompattano la cromatina, rimuovono le modifiche post traduzionali e gli istoni stessi. I complessi che vengono nella forcella di replicazione: la DNA 
polimerasi, con subunit\`a $\alpha$ che polimerizza, due componenti con attivit\`a di proofreading a cui si aggiunge il complesso $\gamma$ o pinza di caricamento e il complesso $\beta$
che \`e la pinza scorrevole. L'apparato che duplica il DNA \`e un dimero, con due DNA polimerasi, due ................................... *audio corrotto* che vengono duplicati in maniera
asincrona. Il legame delle due DNA assicura che la velocit\`a sia singola. Queste subunit\`a $\tau$ rende dimerica la macchina deputata alla replicazione. 

La pinza scorrevole \`e la componente che permette alla DNA polimerasi di aumentare l'affinit\`a con il DNA. \`E fondamentalmente un anello che in condizioni di non interazione con DNA
si trova aperto: due subunit\`a. Quando si lega al DNA si chiude e forma un anello attorno al DNA che permette l'interazione e il mantenimento della DNA polimerasi al DNA e la segue
durante la replicazione. La pinza si carica sul DNA grazie al caricatore della pinza, che interagisce con le due DNA polimerasi. un ATPasi utilizza ATP, prende la pinza scorrevole, la
carica sul DNA, cambia conformazione, si stacca e lascia la pinza scorrevole favorendo il mantenimento  della DNA polimerasi sul DNA. L'attivit\`a della pinza scorrevole mantiene la 
DNA polimerasi e promuove l'attivit\`a della DNA polimerasi stessa che ha una sintesi molto rapida e viene mantenuta dalla presenza del fattore. Due filamento che si duplicano in 
maniera diversa ha prodotto il modello a trombone in cui si formano anse in vir\`u del fatto in cui uno dei due filamenti viene replicato diversamente. L'elicasi \`e un esamero di 
sei subunit\`a che legano l'ATP in maniera diversa: due opposte legano ATP, altre ADP e altre sono vuote: le elicasi separano i filamenti idrolizzando ATP si trovano sei subunit\`a 
identiche ma schiacciato l'anello: delle sei due opposte legano ATP, due ADP e due sono vuote. Questi tre stati si interscambiano cambiando la schiacciatura dell'anello e gli anelli
che legano il DNA oscillano e possono portare DNA nell'anello centrale svolgendolo. L'elicasi \`e uno dei primi fattori in quanto la sua funzione \`e quella di separare i due filamenti
permettendo il reclutamendo della macchina di replicazione e deve svolgere le due catene a monte. \`E precoce ma rimane associata. L'altra proteina \`e una proteina SSB che lega il 
DNA a singolo strand e impedisce che forme delle strutture secondarie stabili in modo che nel filamento discontinuo con la fomrazione dell'innesco a primer e quello a valle reso 
a singolo grazie all'elicasi che potrebbe assumere delle strutture secondiarie stabili che renderebbe difficoltosa l'attivit\`a della polimerasi. Questo viene pertanto lasciato lineare
e dopo le proteine si staccano e le strutture secondarie del DNA sono molto stabili e possono bloccare la macchina di duplicazione. 

Nel caso degli eucarioti arrivati in fondo al filamento la duplicazione termina, nel caso dei procarioti la situzione \`e semplice: la terminazione  dell'escherichia coli \` e terminata
da sequenze Ter molto brevi che sono poste a 180 gradi rispetto all'origine di replicazione. QUando le forcelle raggiungono questa zona la DNA polimerasi viene separata e vengono  
separate le due catene di DNA che si sono formate e sono legate a proteine che favoriscono la separazione. Nel caso degli eucarioti la situazione \`e pi\`u complessa in quanto \`e 
presente cromatina compatta: occorre rimuovere gli istoni dalle origini di replicazione, formare un complesso di pre replicazione presente alle origini di replicazione e tutto questo 
viene orchestrato da una macchina pi\`u complessa rispetto ai procarioti e viene coordinata da fattori dette cicline, che attivano la duplicazione del DNA durante la fase del ciclo 
cellulare specifico. Queste proteine controllano che l'attivazione avvenga nel momento corretto. Il reclutamento della macchina pu\`o avvenire prima della fase S: le cicline impediscono
che questo reclutamento dia origine immediata alla duplicazione del DNA. Il reclutamento non pu\`o avvenire prima della fase S e agisce sulle proteine che fosforilano elementi 
appartenenti alla macchina di replicazione. 
\subsection{Formazione del complesso di prereplicazione alle origini di replicazione}
Alcune proteine hanno nomi diversi con funzioni uguali: PCNA pinza scorrevole, polimerasi epsilon - polimerasi III, MCM 2-7 elicasi, polimerasi $\alpha$, primasi e SSB RPA. Il complesso
\`e comunque conservato. Esistono un numero maggiori di polimerasi con un ruolo importante nella riparazione del DNA. La duplicazione inizia negli eucarioti: un ruolo importante nel
processo di controllo di inizio \`e svolto dalle CDK che fosforilano chinasi - dipendenti dalle cicline, che vengono attivate dalle cicline che quando vengono prodotte attivano chinasi
e danno il via al processo di duplicazione. All'origine di replicazione vengono reclutati i primi attori come Orc 1 - 6 formato da sei subunit\`a e recluta altri due fattori Cdt1 r 
Cdc6 e questo complesso permette il reclutamento delle due elicasi MCM 2 - 7. Quest'ultimo complesso viene detto complesso di pre replicazione che non \`e in grado di duplicare il DNA
e non \`e ancora attivo. Viene reso attivo grazie al reclutamento di chinasi CDK che fosforila Cdc6 e Cdt1 che vengono rilasciate favorendo il reclutamento del caricatore della pinza, 
della pinza scorrevole, della DNA polimerasi, DNA primasi e cos\`i via. Senza l'attivit\`a della chinasi il complesso di pre replicazione non inizierebbe a reclutare fattori e non 
inizia la replicazione del DNA. L'attivit\`a delle cicline \`e importante per attivare la chinasi. Senza la sintesi delle cicline la duplicazione non ha inizio.  *audio corrotto* nei 
complessi di prereplicazione non funzionanti in quanto attivit\`a CDK \`e bassa. Andando a fosforilare i complessi dei siti di prereplicazione li vanno ad attivare. L'attivit\`a delle
cicline sono quattro famiglie di cicline che interagiscono con quattro CDK e sono attivate in punti strategici durante il ciclo cellulare e contribuiscono a controllare le varie fasi del
ciclo cellulare. Altri fattori associati alla macchina di replicazione sono le topoisomerasi, la replicazione del DNA con differenze legate a tempi, attivazione e componenti \`e 
simile tra eucarioti e batteri. 
\subsection{Istoni}
La maggior parte delle proteine istoniche viene sintetizzata nella fase G1 e la produzione degli istoni \`e massiccia, facilitata dal fatto che gli RNA sono molto stabili e che la 
maggior parte degli eucariote possiede geni multipli per ciascun istone. Gli istoni vengono anche sintetizzati durante la fase S. Alcuni degli istoni rimangono associati e vengono 
distribuiti in maniera randomica tra i due filamenti. 
\subsection{Telomeri}
I telomeri sono le estremit\`a dei cromosomi, caratterizzati da una sequenza ripetuta di GGGTTA. La problematica non \`e presente nei batteri in quanto hanno DNA circolare. Mentre
la loro problematica \`e presente dove non \`e presente DNA circolare ma lineare. Fu scoperta da Barbara McClinton nel mais e i cromosomi rotti tendono ad unirsi tra di loro formando
strutture non canonoiche. I telomeri \`e il ristultato della duplicazione non perfetta alle estremit\`a dei cromosomi. I telomeri sono originati e mantenuti, permettono di mantenere 
la stabilit\`a del cromosoma e li stabilizzano rendendoli incapaci di interagire con altre estremit\`a. Sono originati e mantenuto dalla telomerasi che scoprirono questa attivit\`a detta
transferasi terminale telomerasica. La telomerasi \`e un complesso con una componente proteica detto TerT e una componente a RNA detta TerC. Di fatto questo complesso si posizione 
nelle RNP. Ha una componente proteica a mano semi aperta e una a RNA all'interno del palmo. \`E necessaria in quanto l'ultima parte del cromosoma nel filamento lagging non possiede 
primer e non pu\`o pertanto essere sintetizzata in quanto non si trova l'OH necessario. La telomerasi usa a componente a RNA per allungare il filamento parentale e grazie alla sua 
attivit\`a di retrotrascrizione lo allunga con le sequenze ripetute $n$ volte aumentando lo spazio affinch\`e la primasi possa creare un primer e la polimerasi possa copiare la sequenza. 
Se non si aggiungessero si perderebbero informazioni. Questo processo perde efficienza con il numero di divisioni in quanto la lunghezza dei telomeri \`e inferiori. Correlazione tra 
invecchiamento cellulare e numero di telomeri. L'attivit\`a delle telomerasi \`e alta in cellule caratterizzata da un'elevata duplicazione come cellule germinali, cellule epiteliali, 
linfociti. Meno attiva nelle cellule staminali adulte e fibroblasti. La telomerasi aggiunge sequenze telomeriche e la polimerasi copia il segmento non copiato, si riforma la struttura
a doppio filamento riconosciuta da TRF1 e proteina POT (protection of telomers) che si associano alle porzioni terminali neosintetizzate. Il meccanismo non \`e ancora chiaro ma 
l'attivit\`a della telomerasi \`e bloccata quando questi segmenti diventano abbastanza grandi inibendo l'attivit\`a della telomerasi. Un attivit\`a della telomerasi alta permette
alle cellule di sopravvivere in maniera continua. Prendendo le cellule e mettendole in coltura mano a mano che si dividono vanno incontro a una fase esponenziale di crescita fino 
alla raggiunta di un plateau detto fase di senescenza, alterata potenziando l'attivit\`a delle telomerasi, questo limite prende il nome di limite di Hayflick, in cui le colture cellulari
vanno in un processo di senescenza. \`E possibile modificando l'attivit\`a delle telomerasi modificare questo limite. Nel caso dei fibroblasti con il gene della telomerasi non si ha
perdita dei telomeri e la proliferazione continua in maniera indefinita e le cellule non vanno incontro a senescenza. Si \`e scoperto che non \`e possibile andare incontro a immortalit\`a
in quanto overesprimendo la telomerasi le cellule durante la proliferazione accumulano errori che continuando a proliferare si aggiungono e aumenta la probabilit\`a del destino tumorale 
e che ci sono cellule senza attivit\`a telomerasica come le cellule del sistema nervoso che non vanno incontro a mitosi. Il ripristino dell'attivit\`a telomerasica \`e importante 
per l'aumento delle capacit\`a rigenerative. La discheratosi congenita per perdita delle funzione della telomerasi causa pigmentazione anomala, distrofia unguela, ingrigimento precoce, 
cirrosi epatica e disordini intestinali, il gene \`e localizzato sul cromosoma X e il gene mutato \`e la discherina, una chinasi che si lega a piccoli RNA nucleolari coinvolta nei 
processi di modifica dell'RNA ribosomiale. 

\section{Lezione 8}

\subsection{Mitocondri}

* Non ha registrato *

La struttura del sistema di creste del mitoconrio e la prossimit\`a con il sistema vescicolare del reticolo endoplasmatico. Non c'\`e continuit\`a di membrana e contenuti ma sono in
stretta prossimit\`a. Questo avviene in quanto i mitocondri sono un buffer per lo ione calcio contenuto all'interno dell'ER liscio e rilasciato nell'ambiente intracellulare e svolge il
ruolo di messaggero secondario. Quando non \`e pi\`u necessario rientra nel ER o nei mitocondri. I mitocondri possono dividersi (duplicazione del DNA) per fissione e possono fondersi 
tra di loro, regolato da una serie di meccanismi che risponde alle esigenze della cellula. I mitocondri sono strutture molto mobili, vengono trasportati utilizzando i microtubili come
binari. Le molecole del trasporto sono chinesine e dineine, molecole motrici che si attaccano ai mitocondri e microtubili permettendo il trasporto dalla periferia alla parte centrale e
viceversa. Il DNA mitocondriale \`e costituito da una moleocla a doppio filamento circolare. Non porta molti geni in quanto la maggior parte sono stati trasferiti all'interno  del genoma
della cellula ospita. Contiene 37 geni all'interno degli introni per l'rRNA, per il tRNA e 13 per gli mRNA che vengono tradotti in proteine. Si distingue una catena H heavy e 
complementare L low con asimmetria dei geni localizzati sulle due catene. Ci sono 5 - 10 molecole di DNA per mitocondrio con un codice genetico ridondante interpretato in maniera diversa
rispetto alla traduzione citoplasmatica. Vi sono patologie associate alla funzione dei mitocondri, un'eredit\`a materno che non segue le leggi di Mendel. L'insorgenza della malattia
dipende dalla percentuale di mitocondri che contengono quella mutazione. Recntemenente \`e stato visto che aluni RNA prodotti dal nucleo della cellula per i mitocondri vengono trasportati
in prossimit\`a dei mitocodnri, circondati da ribosomi della cellula e si localizzano in prossomit\`a dei mitocondri, esistono delle sequenze negli mRNA che trasportano l'RNA dal 
citoplasma in prossimit\`a dei mitocondri e quando lo raggiungono vengono tradotti e producono proteine e direttamente incorporate nei mitocondri e usano il sistema di trasporto TOM e 
TIM. La sequenza di segnale \`e poi rimossa dalle proteine da una peptidasi.  Senza il peptide segnale la proteina non interagisce e non entra nei mitocondri. I chaperons che si 
legano alla proteina entrante e impediscono che assuma una struttura secondaria durante il trasporto in modo da facilitare il passaggio durante il trasporto. Quando questo \`e finito 
si separano. Conoscendo la sequenza del peptide segnale si possono inserire proteine nel mitocondrio artificialmente. 
