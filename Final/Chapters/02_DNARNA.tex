\chapter{DNA e RNA}
VEDI CAPITOLO 1 DELL'ALBERTS\\
Si nota come organismi diversi dal punto di vista fenotipico possono avere cellule molto simili. La differenza si trova nelle informazioni contenute nel patrimonio genetico e come
queste vengono elaborate producendo proteine diverse. Vi sono a livello cellulare caratteristiche comuni presenti in tutte le cellule: \`e sempre presente una membrana cellulare che 
le isola tra di loro e tra l'ambiente. Possono avere composizioni diverse in base alla specie e al tipo della cellula. All'interno della membrana si trova il nucleo che contiene le 
informazioni comuni tra tutti in quanto \`e caratterizzata dalla presenza dalle quattro basi azotate che formano nucleotidi. I nucleotidi sono caratterizzati da una base azotata e 
zucchero (nucleoside) e dal fosfato. Si trovano quattro nucleotidi (adenina, citosina, timina e guanina), sono legati tra di loro da un legame fosfodiesterico che libera una molecola
d'acqua nel momento in cui si forma. \`E legato alla presenza di un gruppo \ce{OH} libero su un nucleotide e un gruppo \ce{H} su quello adiacente. Il DNA presente in tutte le cellule si
trova a doppio filamento. All'interno della catena si distingue una catena stampo e una neo sintetizzata durante il processo di duplicazione. La struttura a doppia elica \`e 
caratterizzata dalle basi azotate all'interno e lo scheletro di zuccheri e fosfati all'esterno in quanto le basi azotate sono idrofobiche e tendono ad interagire tra di loro. Il DNA 
\`e pertanto molto stabile e pu\`o essere preservato nel tempo. Le basi sono molto vicine. La stabilit\`a rende difficile il trasferimento dell'informazione in quanto richiede 
il distaccamento dei due filamenti al fine di permettere l'introduzione di proteine dedicate alla sintesi degli RNA messaggeri. Quando una cellula va incontro a divisione deve duplicare
il proprio DNA e questo avviene nel meccanismo di duplicazione del DNA semi conservativa: i due filamenti di DNA si staccano e fungono da stampo per il filamento neo-sintetizzato. Questo
avviene in quanto le basi tendono a interagire tra di loro specificatamente (C-G, tre legami a idrogeno, A-T, due legami a idrogeno). Ogni cellula figlia possiede un filamento 
proveniente dalla cellula madre e uno di nuova sintesi. Le mutazioni possono avvenire in una qualsiasi delle cellule figlie o avvenire nella cellula madre. La replicazione permette la
trasmissione delle informazioni, mentre la trascrizione permette la trascrizione di DNA in RNA. Esistono diversi tipi di RNA: messaggeri, quelli usati per la produzione delle proteine e
altri RNA con una funzione regolatoria importante come i microRNA, i long-non-coding-RNA, gli RNA circolari con un ruolo nell'espressione genica e RNA coinvolti in altri processi come i 
tRNA che contengono attaccato un amminoacido e lo portano sui ribosomi per produrre le proteine. Esistono inoltre gli rRNA (RNA ribosomiali) che formano i ribosomi, la macchina 
traduttrice della cellula che unisce le proteine e funzione nella sintesi proteica (RNA molecola catalitica in grado di scindere e creare legami). Il processo trascrizionale \`e 
complesso e pu\`o andare in contro a variazioni (trascritti alternativi). Le modifiche cui l'RNA pu\`o andare incontro dopo la trascrizione sono alla base della variabilit\`a genetica. 
Si possono avere diversi tipi di RNA messaggeri dalla stessa informazione nel DNA attraverso questi meccanismi e creare molti trascritti. L'RNA \`e una molecola a filamento singolo ma 
pu\`o assumere delle strutture secondarie importanti a causa della complementarit\`a delle basi che possono interagire tra di loro e formare strutture a doppia elica accompagnate da 
elementi a elica singola, importante in quanto conferiscono stabilit\`a e possono essere riconosciute da proteine che legano queste strutture per la protezione o importanti per il 
controllo della traduzione dell'RNA o il trasporto. L'acquisizione della struttura secondaria \`e importante in quanto permette la formazione del sito catalitico in cui svolgono la 
propria porzione. Il sito del lisozima (degrada catene polisaccaridi) in particolare lega una catena di polisaccaride che causa un cambio conformazionale che facilita la rottura e 
l'attivit\`a enzimatica del lisozima. 
Il codice genetico nel nucleo delle cellule eucariotiche: tre basi azotate ATG danno l'informazione che corrisponde alla metionina, mentre nel mitocondrio ATG pu\`o essere riconosciuto 
come un altro amminoacido. Il codice delle quattro basi \`e pertanto diverso tra diverse specie: per esprimere una proteina al mitocondrio devo capire come la macchina traduttrice 
interpreta l'RNA. Le quattro componenti principali di una cellula sono acidi nucleici (DNA e RNA), amminoacidi (20, proteine), lipidi e zuccheri. Assieme a questi si trovano elementi 
comuni: idrogeno, ossigeno, carbonio, zolfo, azoto e fosforo, che di fatto sono presenti in tutti gli organismi. Non tutte le mutazioni comportano cambiamenti a livello delle proteine
in quanto triplette diverse possono riconoscere lo stesso amminoacido in quanto il codice genetico \`e ridondante. Il DNA \`e molto grande e la porzione trascritta in mRNA \`e lo $0.1\%$
e ha pi\`u probabilit\`a di avvenire in luoghi non critico per la produzione di una proteina. Inoltre la mutazione pu\`o non avere effetto in quanto per ciascun gene si possiedono due 
alleli, ognuno da un genitore e una mutazione pu\`o non dare problemi in quanto l'altro allele produce la proteina normale. Le mutazioni possono inoltre produrre una proteina che 
funziona meglio rispetto a una proteina normale. Ci sono mutazioni non puntiformi, come quelle cromosomiche in cui si traslocano o perdono parti di cromosoma e sono poco tollerate e 
produrre gravi danni. Le mutazioni genomiche sono perdite e acquisizioni di cromosomi in pi\`u che possono essere tollerati. Una delle caratteristiche comuni della cellula \`e la 
membrana. Ne esistono di diversi tipi con composizioni proprie: la plasmatica contiene colesterolo, la nucleare no e questo costituisce una certa rigidit\`a. Sono costruite da molecole
anfipatiche, costituite da una testa polare e code idrofobiche: polimeri di idrocarburi di catene pi\`u o meno lunghe di \ce{CH2} le membrane si distribuiscono in ambiente polare in un
doppio strato fosfolipidico con le teste all'esterno spontaneamente in modo che le code apolari siano lontane dall'acqua. Questo avviene in quanto nelle cellule sia l'ambiente esterno
che quello interno \`e acquoso. Tutte le membrane cellulari hanno questa struttura. Cambia la composizione dei lipidi (colesterolo, sfingolipidi) e la composizione proteica che 
contengono in quanto permettono comunicazione tra i due lati della membrana. Tali proteine possono essere trasportatori (permettono l'entrata e uscita contro un gradiente), per molecole
grandi. Altre molecole rimangono sulla membrana e permettono il riconoscimento di caratteristiche dell'ambiente e con altri elementi. Permette si l'isolamento ma deve permettere la 
comunicazione i trasportatori contro gradiente lo fanno utilizzando ATP (energia) o accoppiandolo con il trasporto accoppiato da un gradiente opposto. La differenza fondamentale tra 
procarioti ed eucarioti \`e che i secondi contengono un nucleo, mentre nei primi il DNA si trova nel citoplasma attaccato alla parete cellulare e sono provvisti di una parete cellulare 
oltre alla membrana e si muovono usando flagelli. Gli archei si trovano a met\`a tra procarioti ed eucarioti. 
\section{Omologia tra geni}
In molti casi a livello di DNA vi sono degli elementi (geni), conservati durante l'evoluzione, ovvero l'informazione \`e rimasta tale per tutta la durata del processo evolutivo. 
Sottolinea la pressione selettiva affinch\`e questi geni non andassero incontro a mutazioni e sottintendono un'importanza strategica all'interno della cellula. Si supponga di avere un
organismo ancestrale che presenta un gene G questo pu\`o evolversi e pu\`o dare origine a due specie separate. Il G origine muta e d\`a origine a due geni, $G_a$ e $G_b$ nelle due specie
diversi. Si dice che $G_a$ e $G_b$ sono ortologhi. Si supponga un organismo ancestrale comune e il gene $G$ va incontro a duplicazione genica e i geni $G_1$ e $G_2$ vanno incontro a 
divergenza e vengono detti paraloghi. Il processo di duplicazione genica \`e avvenuto abbastanza spesso e infatti i geni che sono andati incontro a duplicazione e creato delle famiglie. 
Quelli pi\`u comuni sono quelli degli rRNA, degli istoni, danno origine a proteine con funzione pi\`u o meno simile. I geni omologhi hanno somiglianze di sequenza o funzione. Il processo
di duplicazione pu\`o dare origine a pseudogeni, ovvero modifiche che eliminano il ruolo funzionale del DNA che non viene pi\`u tradotto come una proteina. Si intende per gene una 
sequenza, un segmento di DNA che corrisponde a una proteina o a una serie di varianti di trascritti che hanno funzione regolatoria o strutturale. In molti casi vi sono mutazioni in cui il
fenotipo si varia: nel lievito ci sono delle mutazioni che gli fanno assumere una forma a T. 
\section{Struttura della cellula eucariotica}
\`E definita da membrana plasmatica e presente un nucleo delimitato da membrana nucleare che contiene proteine del poro nucleare, che formano pori nella membrana. Sono importanti per il
trasporto di molecole, proteine e acidi nucleici dal nucleo al citoplasma e viceversa. Sono importanti in quanto il nucleo necessita di proteine per la cromatina o fattori trascrizionali
che sono prodotti nel citoplasma. Il poro nucleare \`e uno scolapasta con tanti buchi che permettono la comunicazione tra i due ambienti. All'interno del nucleo si trova il nucleolo 
il cui numero dipende dalla fase del ciclo cellulare, dallo stato trascrizionale: quanti RNA produce la cellula. \`E la zona dove avviene l'assemblamento e la trascrizione degli rRNA. 
\subsection{ER}
In contatto con la membrana nucleare, un sistema di cisterne a membrana che pu\`o essere liscio o rugoso. Il secondo si chiama cos\`i in quanto sono presenti ribosomi. Qui avviene la 
sintesi delle proteine di membrana e secrete dalla cellula. Altre proteine vengono tradotte dai ribosomi liberi nel citoplasma.
\subsection{Apparato di Golgi}
Un altro sistema di cisterne in continuit\`a funzionale ma non fisica con il reticolo endoplasmatico da cui riceve vescicole che permette alle proteine in cui nel Golgi subiscono 
modifiche come glicosilazione (aggiunta di zuccheri). Si trova un trafficking continuo che porta alla formazione di vescicole tra le membrane. 
\subsection{Mitocondri}
Possiedono un proprio DNA e apparato di traduzione con pochi geni in quanto molti dei geni sono presenti nel nucleo cellulare. Sono degli antichi batteri che una protocellula ha inglobato
e gli ha comportato un vantaggio evolutivo in quanto producono ATP in presenza di ossigeno molto pi\`u efficiente che in assenza. Si sono pertanto trasferite funzione dalla cellula 
mitocondriale alla cellula ospite e le proteine vengono trasportate dalla cellula al mitocondrio, mentre altre prodotte dal mitocondrio stesso. I mitocondri si dividono (fishing) e 
fusione (fusion) regolato da meccanismi: il numero di mitocondri dipende dal metabolismo della cellula e da altri fattori. Viaggiano all'interno della cellula grazie a proteine motrici
che usano i microtubuli come rotaie. Il mitocondrio \`e costituito da una membrana esterna con due strati con uno spazio bianco interno. \`E caratterizzato da una serie di invaginazione
e una serie di trasportatori per sintesi e traduzione di ATP. In maniera analoga le cellule vegetali hanno cellule cloroplasti per la produzione di glucosio grazie alla sintesi 
clorofilliana. Contengono clorofilla e fissano l'azoto atmosferico producendo zuccheri utilizzati da altri organelli. I funghi sono organismi particolari senza cloroplasti e non si 
muovono e sono degli spazzini utilizzando sostanze prodotte dall'ambiente circostante che utilizzano per la propria sopravvivenza. 
\subsection{Lisosomi}
Sono la sede della degradazione delle proteine.
\subsection{Citoscheletro}
Formato da microtubuli, filamenti di actina che conferiscono motilit\`a e si trovano principalmente nel leading edge che permettono alla cellula di dirigersi e sentire l'ambiente. I 
filamenti intermedi si trovano in particolare associati con la membrana nucleare. \`E importante per l'adesione cellulare e il movimento. Sostanze che destabilizzano il citoscheletro le 
impediscono di dividersi e sono antitumolari (taxolo). 

\section{Struttura del DNA}
Le basi del DNA vengono divise in purine (Adenina e Guanina) con due anelli e le pirimidine (Citosina, Timina e uracile) con un anello. Queste sono le basi canoniche, ma ne esistono
altre nell'RNA che possono modificarsi in modo che possa assumere funzioni particolari. Alcune basi del tRNA possono subire modifiche post-trascrizionali e nel mRNA che comportano cambio
di propriet\`a e funzioni. Questo meccanismo di modifica si dice RNA editing. CRISPR Cars modifica di una o pi\`u basi all'interno del DNA. Esiste un complesso detto editosoma nei 
parameci che modifica l'RNA e avviene anche a livello dei mammiferi negli RNA che producono recettori nel sistema nervoso. L'editing dell'RNA non va contro il dogma centrale della
biologia molecolare in quanto si modifica l'RNA dopo che \`e stato tradotto e non si modifica il DNA. Gli zuccheri sono diversi DNA: $2'$-deossiribosio e RNA ribosio in modo che sia in 
grado di assumere strutture secondarie. \`E presente il gruppo fosfato. Le basi come Adenina e Guanina possono essere usate anche come molecole energetiche: ATP e GTP che portano avanti
reazioni della cellula e sono precursori dei messaggeri secondari: cAMP e cGMP. Si tratta di molecole prodotte in particolari condizioni e aumentano il segnale. Chargaff riusc\`i ad 
osservare che il DNA aveva una composizione uguale all'interno delle cellule di uno stesso organismo anche di tessuti diversi, mentre organismi diversi avevano una composizione del DNA
diversa. Il numero di residui A e T era uguale cos\`i come il numero di C e G all'interno della stessa molecola, permettendo a Watson e Crick di formare il modello e definire
l'interazione tra le basi. La composizione del DNA di un organismo: percentuale di basi pirimidiniche o puriniche differisce: organismi che vivono in temperature elevate hanno un 
contenuto elevato hanno una percentuale maggiore di C e G in quanto la loro interazione coinvolge una formazione di un maggior numero di legami a idrogeno rispetto a A e T, pertanto il 
DNA \`e pi\`u stabile. Le regole di Chargaff non valgono per l'RNA in quanto non \`e a doppia elica ma a singolo filamento, ma pu\`o comunque formare strutture secondarie nella presenza
di regioni con interazioni tra le basi complementari. 
\subsection{Modifiche dell'RNA}
L'inosina \`e una base azotata che viene riconosciuta come G e pu\`o appaiarsi con la C. Un'altra modifica \`e la 7-metil guanosina, la metilazione della guanosina che avviene in tutti 
gli mRNA una volta sintetizzati nella prima base e viene riconosciuta dalla cap-binding protein che legandosi all'RNA messaggero lo stabilizza e aiuta il processo di traduzione. Altre 
modifiche sono la pseudouiridina e la tiouridina. Si dice nucleoside la base legata allo zucchero, con un fosfato vengono detti nucleotidi. La trasformazione dell'ATP in cAMP, una 
molecola importante in quanto messaggero secondario portato avanti dall'andenilato ciclasi che favorisce la formazione di un anello con la liberazione di pirofosfato e l'anello d\`a il 
nome alla molecola. Il pirofosfato viene poi scisso. Un altro enzima detto fosfodiesterasi che utilizzando acqua rompe l'anello e trasforma il cAMP in AMP che non \`e un messaggero 
secondario. Un discorso analogo con guanil ciclasi forma il cGMP. Le basi dei nucleotidi vengono unite da un legame fosfodiesterico che consiste nell'interazione tra un gruppo OH e quello
fosfato tra il nucleotide adiacente e libera pirofosfato. \`E un legame molto stabile. Si forma pertanto il legame fosfodiesterico che conferisce la polarit\`a al filamento $5'-3'$ che 
\`e importante in quanto il meccanismo di duplicazione avviene in direzione $5'-3'$ con la presenza di un gruppo OH libero per avvenire. La polarit\`a \`e importante anche per i fenomeni
di trascrizione.
\section{Struttura del DNA}
Il DNA ha una struttura stabile, stabilit\`a legata alla natura idrofobica tra le basi e le interazioni tra di loro grazie alla formazione di legami a idrogeno. Nel caso tra A e T si 
formano due legami a idrogeno, mentre tra C e G se ne formano tre. Ci sono degli accoppiamenti non canonici (coppie di basi di Hoogsteen) che interagiscono tra di loro ma non sono 
complanari con una rotazione dell'anello pirimidinico che forma un numero diverso di legami a idrogeno (2) per C e G che rende il DNA meno stabile. Le basi non sono fisse ma tendono a 
muoversi tra di loro: twist: basi adiacenti roteano l'una sull'altra, roll, si inclinano lungo l'asse maggiore e tilt inclinano lungo l'asse minore. Possono avere ruolo fondamentale nei
meccanismi di trascrizione.  Il giro d'elica \`e di circa $10$ paia di basi e $34$ armstrong con $10$ armstrong di diametro, con un solco minore di $12$ e uno maggiore di $22$ armstrong. 
Molti fattori trascrizionali si inseriscono nel solco maggiore che facilita l'accesso alle proteine. Queste proteine sono caratterizzate da domini ad alfa elica, responsabile delle 
interazioni con il solco maggiore. Il DNA assume questa struttura in condizioni fisiologiche, forma B, struttura cristallizzata dalla Franklin. Vi sono altre due strutture, la struttura A
pi\`u tarchiata e si forma nella riduzione della quantit\`a di acqua e aumento della salinit\`a, le coppie di basi sono pi\`u angolate rispetto al piano della doppia elica e ci sono 11
basi per giro, la scanalatura principale \`e pi\`u profonda, \`e poco presente nel DNA, pi\`u presente nell'RNA. La struttura Z \`e peculiare, identificata nelle zone con sequenze ricche 
in C e G che tendono a formarla che presenta 12 paia di basi per giro d'elica, molto pi\`u stretta e allungata rispetto alla B. La scalanatura principale \`e poco profonda. In queste zone
che si trovano nelle zone promotrici dei geni, in quanto possono essere modificate con metilazioni. Ogni fosfato ha una carica negativa rendendolo polare. La specificit\`a della relazione
tra le basi garantisce la complementarit\`a tra le basi. 
\section{Struttura dell'RNA}
L'RNA assume una struttura A in quanto si trova un maggiore ingombro sterico legato al gruppo OH del ribosio che facilita l'idrolisi dei legami fosfodiesterici e si rompe pi\`u facilmente
rispetto al DNA, un bene in quanto la funzione dell'RNA \`e la produzione di proteine e pertanto devono essere degradati. L'RNA pu\`o essere stabilizzato dal legame con proteine che 
impediscono la rottura del legame fosfodiesterico. 
\section{Interazioni tra basi}
\subsection{Temperatura di melting}
Aumentando la temperatura si rompono i legami a idrogeno per rompere le basi. Per stabilire se due DNA sono pi\`u o meno stabili: si usa uno spettrofotometro: il DNA a doppia elica 
assorbe a $260 nm$, mano a mano che si denatura l'assorbimento tende ad aumentare. Prendendo il DNA, mettendolo in una provetta e osservando l'assorbanza e aumentando la temperatura si 
osserva come questo valore cambia. La temperatura in cui met\`a del DNA \`e denaturato viene detta temperatura di melting. Attraverso questo tipo di analisi si pu\`o determinare la 
percentuale di legami di guanina e citosina determinando quale DNA \`e pi\`u stabile. 
\section{Conformazioni particolari}
\subsection{Tripla elica}
Il DNA pu\`o assumere strutture a tripla elica che si formano dove esiste una sequenza particolare. Queste strutture influiscono sul processo di duplicazione e trascrizione del DNA e 
molto spesso rendono il DNA ancora pi\`u stabile e la capacit\`a delle due catene di separarsi \`e compromessa, impedendo l'arrivo dei fattori necessari ai processi. \`E ricca di A e T  
ed \`e resa possibile dalle interazioni di Hogsteen. 
\subsection{La struttura cruciforme}
Stem-loop o a forcina. Si formano dove ci sono sequenze palindromiche: sequenza di basi ripetuta e invertita nella sequenze invertite. Deleterie per i processi in quanto molto stabili. 
Se rendono l'mRNA e il DNA stabili possono creare problemi durante i meccanismi di duplicazione e trascrizione che comportano la dissociazione delle due catene.
\section{DNA superavvolto}
Il DNA tende a formare superavvolgimenti che sono caratteristiche del DNA e possono essere descritte in termini matematici. Questa tematica va a impattare sui meccanismi di trascrizione
e duplicazione. Si dice twist (frequenza), quante volte un filamento si avvolge sull'altro. Si dice writhe l'indice del numero di superavvolgimenti che si vengono a formare. Questi due 
parametri stanno alla base dello studio dei meccanismi di superavvolgimento. Un DNA lineare ha un writhe pari a $0$ e un twist $n$, ovvero quante volte un filamento si avvolge sull'altro.
Twist e writhe sono legati dal linking number, $Lk = Tw + Wr$. Il DNA superavvolto ha un write diverso da zero. Il superavvolgimento pu\`o essere positivo (destrorso) o negativo 
(sinistrorso). Il DNA tende a mantenere il linking number costante quando si varia il twist introducendo del writhe. Quando aumento il linking number il DNA diventa superavvolto e si 
deve introdurre un writhe number negativo e viceversa. Pu\`o capitare che un DNA che si trova in condizioni stabili possa andare incontro a fenomeni che alterino il twist number e 
pertanto per tornare in una condizione stabile tende a superavvolgersi (aperture di bolle di trascrizione e duplicazione) in modo che ritorni ad una struttura stabile ottimale. Quando
questo avviene i superavvolgimenti possono bloccare il meccanismo. Pertanto a questo punto intervengono le topoisomerasi.
\subsection{Topoisomerasi}
Sono proteine che intervengono per risolvere i superavvolgimenti indotti nel twist. Sono famiglie di proteine con il compito di tagliare il DNA. Sono di due famiglie: di tipo I presenti
sia nei batteri che eucarioti che archei e il tipo II presenti negli eucarioti e hanno la capacit\`a di rilassare il DNA che \`e andato incontro al superavvolgimento. Quelle di tipo I 
non richiedono ATP quelle di tipo II lo richiedono. Sono diverse nel modo in cui lo tagliano. Quelle di tipo I hanno un dominio di interazione del DNA (alfa elica) a pinze che si legano 
nel punto con il superavvolgimento, tagliano uno dei due filamenti, favoriscono il passaggio di un'estremtit\`a da una parte e poi lo rilegano. Non usano ATP in quanto la risoluzione 
del superavvolgimento in questo caso \`e favorita dal punto di vista energetico. Nel tipo II viene tagliato il doppio filamento e sono dei dimeri con due cancelli uno N e uno C e un 
cancello per il DNA. Una volta che \`e stato tagliato il doppio filamento un altro viene alloggiato nel cancello N che si chiude e causa un cambio conformazionale che fa passare il 
filamento sopra nel cancello C sotto, rilasciato e il doppio filamento rotto viene risaldato aumentano o diminuiscono il linking number di $2$. Nelle topoisomerasi di tipo I contiene nel
sito catalitico una tirosina con un gruppo OH che altamente riattivo attacca e rompe il legame fosfodiesterico ma la topoisomerasi rimane attaccata al filamento. Il superavvolgimento 
viene risolto e si riforma il legame fosfodiesterico usando il gruppo OH e si ha la rottura della tirosina. Alcuni dei farmaci hanno come target le topoisomerasi come la camptotecina 
la topoisomerasi di tipo I ed \`e utilizzata per il trattamento al cancro delle ovaio e del colon. Il DNA viene tagliato e si blocca la fase di cucitura e pertanto il taglio risulta 
permanente e le cellule muoiono. Sensibile per le cellule tumorali in quanto hanno alti livelli di telomerasi. Possono agire anche sulle cellule normali e hanno effetti su quelle cellule
con un elevato tasso di mitosi come globuli bianchi, follicoli piliferi tratto gastro intestinali. 
\section{Modifiche post-traduzionali}
Sono modifiche che avvengono sulle proteine e ne modificano l'attivit\`a e funzione, localizzazione, inibire o aumentarne l'attivit\`a o il turnover. Le proteine vengono degradate con il
tempo pi\`u o meno velocemente.
\subsection{Fosforilazione}
La fosforilazione \`e l'aggiunta di un gruppo fosfato da enzimi detti chinasi. \`E una modifica reversibile. L'enzima che la toglie \`e la fosfatasi.
\subsection{Acetilazione}
Aggiunta di un gruppo acetilico attraverso acetiltransferasi che viene rimosso da deacetilasi. 
\subsection{Metilazione}
Aggiunta di un gruppo metilico attraverso metiltransferasi e una demetilasi che lo rimuove. 

Normalmente queste modifiche avvengono a carico di amminoacidi particolari a carico di una lisina, fosforilazione serina, treonina, metilazione avviene alla lisina. Vi sono dei programmi
che permettono di identificare dove queste modifiche possono avvenire. Queste modifiche avvengono a tutte le proteine citoplasmatiche e nucleari. Ci sono delle chinasi e acetiltransferasi
nucleari e altre che si trovano nel citoplasma. 
\subsection{Ubiquitinazione}
Aggiunta di un gruppo di ubiquitina, un polipeptide che viene aggiunto grazie a degli enzimi particolari e che cambia il turnover della proteina. Una proteina che va incontro a 
ubiquitinazione va incontro a degradazione. Questa modifica avviene grazie all'attivazione dell'ubiquitina ad opera dell'enzima E1 a cui segue l'attivazione e reclutamento di E2 e E3 con
funzioni diverse. La proteina bersaglio contiene la lisina, l'unico amminoacido che va incontro a ubiquitinazione. Gli enzimi prendono l'ubiquitina e la legano alla proteina bersaglio
alla lisina. Possono essere legate $n$ molecole alla proteina (poliubiquitinata). Questa modifica cambia di molto il peso molecolare della proteina ed \`e possibile vedere sul gel la
modifica di questo tipo. La proteina poliubiquitinata viene portata al proteasoma, un complesso $26s$ costituito da un nucleo (cilindro cavo), un cappuccio d'entrata e uno d'uscita. Nel
cappuccio d'entrata si trova il recettore Rpn10 che riconosce l'ubiquitina e permette e forza la proteina ad entrare all'interno del proteasoma. La coda rimane al di fuori. L'ubiquitina 
pertanto non viene degradata e rimarr\`a disponibile. Gli amminoacidi escono alla fine dal cappuccio d'uscita. La isopeptidasi Ppn11 ha lo scopo di staccare la coda poliubiquitina. 
Si noti come l'ubiquitina \`e irreversibile. Non tutte le proteine contenenti lisina possono essere degradate in quanto ci devono essere a valle e a monte altri amminoacidi specifici. 
\subsection{Sumoilazione}
Meccanismo analogo e simile all'ubiquitina. Si tratta dell'aggiunta del polipeptide con l'attivazione di enzimi che lo vanno attaccare alla proteina e come nel caso dell'ubiquitinazione
ci pu\`o essere polisumoilazione. Questo processo non comporta necessariamente degradazione della proteina. Normalmente viene utilizzata per modificare attivit\`a o localizzazione di una
proteina specifica. Si lega sempre su una lisina. I substrati di SUMO si trovano nei pori nucleari per modificare il traffico nucleare cambiando permeabilit\`a o specificit\`a. Nei 
fattori di trascrizione influisce nell'espressione genica, nelle proteine nella riparazione e replicazione del DNA influendo sulla stabilit\`a genomica e nel citocore e centromeri 
influisce sull'integrit\`a cromosoma. La sumoilaizone \`e reversibile e quando deregolata pu\`o essere alla base di malattie neurodegenerative, tumori, diabete, eccetera. \`E importante
per progressione del ciclo cellulare, sopravvivenza, divisione e proliferazione cellulare, differenziamento e invecchiamento. Avvengono su domini critici: per fattori di trascrizione
nei domini di legame del DNA o la sequenza che permette alle proteine di entrare nel nucleo. La sumoilazione pu\`o agire sulla funzione di una proteina coinvolta nella riparazione del
DNA timina-DNA glicosilasi che scansiona il DNA e quando trova una U lo rimuove e va incontro a sumoilazione che determina il distacco con il DNA e favorisce l'introduzione di un altro
fattore che va a chiudere il gap mettendo la base corretta. Sumo pu\`o essere rimosso e TDG torna a funzionare. 

\section{Come il DNA viene organizzato nel nucleo: cromatina}
A differenza dei batteri con DNA circolare il DNA eucariotico \`e molto lungo e deve stare in uno spazio molto ristretto. Il compattamento del DNA nel nucleo \`e complesso dal punto di
vista strutturale in quanto un DNA troppo compatto rende difficile l'accesso per i fattori trascrizionali o di replicaizone. Il compattamento del DNA coinvolge una serie di eventi e 
proteine. Gli istoni sono una famiglia di proteine molto conservata. Quattro istoni H2-H4, H2A e H2B formano la core particle istonica attorno a cui si avvolge il filamento di DNA (due 
giri e mezzo). Vanno incontro a fosforilazione, acetilazione, ubiquitinazione e metilazione. Normalmente durante le prime fasi di assemblamento le code istoniche sono acetilate che 
impedisce a due core particles di associarsi in maniera molto stretta. Quando si vuole aumentare il grado di compattamento si deve rimuovere l'acetilazione attraverso la deacetilasi che
favorisce l'avvicinamento e il compattamento dei core istonici. La struttura base istone pi\`u DNA si chiama nucleosoma. Pi\`u nucleosomi deacetilati si uniscono formando strutture
che si avvolgono tra di loro riuscendo ad entrare e trovare posto all'interno del nucleo. Tutte le cellule contengono gli istoni e nel caso degli spermatozoi invece degli istoni si 
trovano le protammine che hanno la funzione di ultracompattare il DNA in quanto la loro funzione \`e quella di viaggiare per lunghe distanze in modo da facilitare questa funzione. Ci 
pu\`o essere compatatmento a zig-zag e a solenoide. Per caratterizzare il DNA avvolto attorno al nucleosoma si tratta di una tecnica che consiste di digerire il DNA in maniera blanda 
preservando il DNA attaccato agli istoni. La parte proteica dei nucleosomi viene rimossa e viene sequenziato per determinare se esiste una sequenza precisa a cui si legano i nucleosomi.
Gli istoni hanno legame non specifico anche se dove ci sono pezzi di DNA trascritti in maniera elevata si assiste ad una specificit\`a di legame degli istoni con quelle regioni di DNA. 
Il compattamento influisce si replicazione, ricombinazione, trascrizione, riparazione e segregazione. Le modifiche post traduzionali possono avere conseguenze per questi processi, 
attivandoli o inibendoli. 
\subsection{Cromatina}
In base al compattamento della cromatina si distingue in eucromatina e eterocromatina divisa in costitutiva o facoltativa. L'eucromatina si trova in uno stato di scarsa compattezza a 
differenza dell'eterocromatina in uno stato pi\`u elevato di condensazione. Si dice costitutiva l'eterocromatina sempre compatta e rappresenta la funzione di genoma senza capacit\`a
codificante che non viene mai espressa e si parla di cromatina con ruolo strutturale nei cromosomi. L'eterocromatina facoltativa \`e in uno stato molto condensato e in determinate 
condizioni ha capacit\`a condensanti. Vi sono regioni che durante lo sviluppo passano da eucromatina ad eterocromatina facoltativa. Il cromosma X in una delle coppie viene in parte 
condensato in eterocromatina. Uno dei due cromosomi X viene parzialmente inattivato e si forma il corpo di Bar nelle cellule femminili. L'eterocromatina appare molto pi\`u scura alla
microscopia elettronica. L'eterocromatina \`e pi\`u presente nella periferia attaccata alla membrana nucleare mentre l'eucromatina nella parte pi\`u interna. Il genoma dei procarioti
\`e costituito da un genoma circolare che assume strutture secondarie. Forma strutture detti nucleoidi. Non si trova libero nel citoplasma ma ancorato su un punto della membrana detto
mesosoma. Si trova associato a delle proteine con funzione simile agli istoni (HU, IHF, H1, P) e tendono a fargli assumere una struttura maggiormente compatta. Si nota come il cromosoma
\`e costituito da anse (sottodomini) indipendenti in modo che \`e possibile compattare una regione lasciando decompattata un'altra riflettendo la capacit\`a di trascrizione di un'ansa
rispetto ad un'altra. Ci sono delle topoisomerasi o girasi che risolvono i ripiegamenti favorendo la separazione delle due catene quando necessario. Si nota come il DNA di organismi 
evoluti contiene una certa quantit\`a di informazione con una certa lunghezza. Si trova grande differenza tra genoma umani e di altri organismi pi\`u semplici si trova una relazione
inversa tra complessit\`a dell'organismo e densit\`a genica: $DG = \frac{1}{\alpha}$, con $\alpha$ la complessit\`a o il rapproto tra geni e numero di basi. Questa diminuzione \`e dovuta
in due modi: aumentando le dimensioni dei geni tramite sequenze intrageniche (sequenze introniche spesso) o aumentando la quantit\`a di DNA esistente tra i geni (sequenze intergeniche) 
pi\`u  del $60\%$ del genoma umano \`e costituito da sequenze intergeniche. Questa porzione \`e stata considerata come DNA spazzatura ma in realt\`a \`e una sequenza che in parte 
trascrive con importante ruolo strutturale. Le sequenze uniche si trovano i pseudogeni, geni che nel corso dell'evoluzione hanno perso la capacit\`a di essere trascritti in proteine e 
il secondo gruppo di sequenze intergeniche \`e il DNA ripetuto: DNA microsatellite ripetizioni di nucleotidi (analisi di paternit\`a), durante la replicazione si possono allungare e 
quando questo avviene in zone geniche o promotrici allora possono nascere problemi e pu\`o causare il blocco della trascrizione. Il DNA altamente ripetuto come le sequenze ALU che 
si ripetono tante volte all'interno del genoma umano con funzione sconosciuta. Tra le sequenze intergeniche ci sono anche le porzioni di DNA che codificano per microRNA. Il DNA degli 
eucarioti contiene sequenze con ruolo regolatorio e vengono trascritti in molti casi. Quando il DNA si replica deve replicare tutte queste sequenze. Tra le regioni ripetute senza ruolo
trascrizionale con caratteristiche di compattamento e struttura si trova la regione centromerica che si tratta di una regione presente a met\`a dei cromosomi circa ed \`e costituita da
DNA e proteine che formano il cinetocore, la struttura importante durante il processo di divisione cellulare in quanto qua si legano i microtubuli e durante il processo di mitosi permette
la separazione dei due cromatidi nelle cellule figlie nei lati opposti. La sequenza centromerica ha determinate caratteristiche di sequenza e compattamento. Il lievito si presta a questi
studi in quanto facile da crescere e semplice. Nel lievito \`e di circa $100-120$ paia di basi (vertebrati anche milioni) e prende il nome di CEN. \`E costituita da tre domini CDE-I, 
CDE-II e CDE-III, il secondo \`e centrale ($90$ nucleotidi), non ha sequenza specifica ma molto ricca in T e A. Questa sequenza \`e fiancheggiata dalle altre due sequenze. La sequenza
viene riconosciuta da proteine che riconoscono la sequenza e permettono la costruzione del cinetocore. Sono riconosciuti da una che si lega a CDE-II (CSE4P), mentre le due periferiche
vengono riconosciuti dal fattore CBF3 e CBF1 che va a reclutare un complesso trimerico Ctf9, Mcm21 e un altro. La sequenza \`e importante in quanto in presenza di mutazioni si pu\`o 
compromettere il riconoscimento causando ripercussioni durante il processo di divisione cellulare e ci possono essere delle partizioni asimmetriche dei cromosomi durante il processo di 
divisione. Queste regioni sono regioni in cui la cromatina assume una struttura particolare: eterocromatico costitutivo. Il maggior grado di compattazione avviene durante il processo di 
mitosi, quando appare sotto forma di cromosomi, la forma pi\`u estrema di impacchettamento. La mitosi \`e preceduta da altre fasi G1, S e G2 che costituiscono l'interfase, la parte del 
ciclo cellulare tra una mitosi e l'altra. In questa fase le proteine che si legano al DNA e ne permettono il compattamento vengono sintetizzate. Le cellule che non si dividono come i 
neuroni rimangono nella fase G0. La cromatina nella fase M appare come cromosomi (22 somatici e 2 sessuali), hanno diverse forme e dimensioni. I cromosomi possono essere colorati tramite
bandeggio di gemsa e si caratterizzano da due braccia P e Q, un centromero e una serie di bandeggi che definiscono la posizione dei geni all'interno del cromosoma. Non sempre la 
duplicazione avviene in maniera canonica: in alcuni organismi pu\`o avvenire una divisione incompleta: politenia, le replicazioni non vengono seguiti da segregazione e i cromosomi 
rimangono attaccati tra di loro e formano strutture molto grandi. \`E tollerata in alcune specie e cellule come le cellule salivari della Drosophila, per trascrivere elevata quantit\`a di
RNA. 
\subsection{Istoni e prime fasi di compattamento del DNA}
Il primo step di avvolgimento di DNA prende il nome di nucleosoma, pi\`u nucleosomi si compattano tra di loro formando fibre con conformazione a zigzag o solenoide. Esiste un modo per
misurare il grado di compattamento: il rapporto di impacchettamento: il rapporto tra la lunhgezza del DNA e la lunghezza della struttura che lo contiene. \`E possibile capire quanto 
compatta \`e la cromatina in determinate condizioni. Il compattamento \`e cruciale in quanto svolge un ruolo importante nei processi di replicazione, ricombinazione, trascrizione, 
riparazione e segregazione. Il processo di compattamento \`e dinamico: gli istoni si muovono lungo il DNA stesso. Le proteine che si legano al DNA e hanno un ruolo nella formazione della
cromatina sono le proteine istoniche e non istoniche (facilitazione del compattamento e nel reclutare istoni sul DNA o nel processo di trascrizione o duplicazione), associano con il DNA
e con la cromatina ma non necessariamente hanno ruolo nel suo compattamento. La struttura del nucleo base del primo livello di interazione DNA-istoni \`e il nucleosoma, \`e costituito 
dagli istoni H2A H2B, H3 e H4 che si associano tra di loro costruendo il nucleo ottamerico, il DNA si lega intorno al nucleo (147 nucleotidi) formando il nucleosoma. H3 e H4 hanno code
istoniche che hanno ruolo cruciale nella regolazione del compattamento. Gli istoni sono altamente conservati in quanto svolgono un ruolo molto importante nei processi. Sono proteine 
non eccessivamente grandi e sono caratterizzate da una coda N terminale e un dominio di ripiegamento istonico. Le code sono diverse in lunghezza e struttura ma il dominio di ripiegamento
istonico \`e conservato, costituito da tre $\alpha$-eliche che sono uniti da delle anse. La formazione dell'ottamero: si formano i dimeri H3 e H4 che si combinano tra di loro formando 
un tetramero H3-H4 che inizia il primo contatto con il DNA. Successivamente il complesso si unisce a due dimeri H2A-H2B formando il core istonico e il nucleosoma. L'assemblamento non
richiede energia e proteine che aiutano il folding e l'interazione tra i monomeri (chaperons). Tutte le code N terminali sono esposte all'esterno, cosa importante in quanto sono le 
modifiche agli amminoacidi in questa regione hanno ruolo nel regolare il compattamento della cromatina in quanto permettono ai nucleosomi di interagire diversamente tra di loro. Ci 
sono metilazioni, acetilazioni (facilita l'avvicinamento tra due nucleosomi). L'interazione tra ottamero e il DNA per formare il nucleosoma avviene tra l'ossatura degli amminoacidi 
istonici e l'ossatura dei fosfati e dagli zuccheri del DNA, in particolare amminoacidi degli istoni come lisina e arginina (basici) facilitano l'interazione con le cariche negative 
presenti nello scheletro del DNA facilitando la formazione del legame. Pertanto l'interazione dell'ottamero \`e aspecifica anche se in molti casi c'\`e una forte presenza di AA, TT e TA
in quanto presenti nella maggior parte del solco minore con una maggiore compressione del solco minore all'interno. Sul DNA sono presenti proteine che facilitano il reclutamento in zone
dell'ottamero istonico. Ci sono proteine che impediscono la formazione di nucleosomi tra di loro legandosi in un numero minore di basi. La formazione non dipende da sequenze specifiche
ma da fattori che possono determinare la formazione di nucleosomi in determinati punti del DNA. Questo pu\`o essere valutato usando enzimi come DNAasi che degrada il DNA: in una cromatina
poco condensata la DNAasi pu\`o digerire tra un nucleosma e l'altro, in una cromatina pi\`u condensata l'accesso all'enzima del DNA \`e bloccato dall'elevato compattamento dei nucleosomi.
Questo permette di capire se un gene (o il promotore che attiva la trascrizione del gene) verr\`a trascritto oppure no. Si pensava che la struttura del nucleosoma che si venisse a 
formare fosse un qualcosa di fisso a causa della forte interazione DNA-proteine. Si \`e visto come questa sia una struttura dinamica: il nucleosoma si svolge 4 volte al secondo lasciando
esposto il DNA per 10 millisecondi prima che si riformi a monte o a valle di dove si fosse formato prima il nucleosoma. Questo \`e importante in quanto rende pi\`u dinamico e muovendosi
il nucleosoma permette al DNA di separarsi e dare accessibilit\`a a fattori proteici trascrizionali o replicazionali. Il nucleosoma \`e reso dinamico dal complesso di rimodellamento della
cromatina, costituito da un ATPasi che lega e idrolizza ATP, un dominio che lega DNA e uno che lega l'ottamero istonico. Si trova un dominio elicasico che \`e quello che aiuta il DNA a 
staccarsi e a rimuoversi dall'ottamero e a muoversi a monte o a valle. Questi complessi sono importanti in quanto svolgono questo ruolo in diversi momenti: processo trascrizionale o 
di duplicazione. Sono fattori importanti e oltre a questi complessi esistono altri fattori detti chaperons degli istoni che cooperano con il rimodellatore della cromatina. La funzione di 
questi fattori \`e quella di facilitare la dissociazione o associazione dei monomeri o dimeri di istoni durante il rimodellamento della cromatina. Sono complessi proteici grandi e 
riconoscono delle sequenze specifiche presenti all'interno dei singoli istoni. Quando il DNA si duplica si duplicano gli istoni: le catene neosintetizzate contengono in parte istoni 
provenienti dal vecchio e altri nuovi. Alcuni di questi rimangono associati durante la replicazione e devono andare incontro a un deposizionamento. Questi chaperoni facilitano la 
rimozione e uno scambio tra le subunit\`a di istoni neosintetizzati inseriti nella catena neosintetizzata. I due nucleosomi adiacenti tendono a compattarsi dove le code istoniche sono
andate incontro a modifiche post traduzionali a capo di amminoacidi, quella che facilita la compattazione \`e la deacetilazione che avviene a carico di DACH deacetilase histon e la 
proteina che acetila l'istone \`e HAC histon acetylase. Queste sono modifiche post traduzionali reversibili, la seconda facilita il compattamento la prima il compattamento. Vi sono 
altri istoni che contribuiscono alla formazione della compattazione della cromatina come l'istone H1, il pi\`u grande degli istoni e detto istone linker e si posiziona tra il DNA a 
doppio strand non legato al nucleosoma e interagisce ul DNA presente sull'istone. Fa assumere al DNA non nel nucleosoma un'angolazione che favorisce il compattamento tra un nucleosoma e 
l'altro. \`E quello meno conservato dal punto di vista evolutivo. Volendo vedere se un gene \`e espresso in diversi tessuti un controllo per vedere se ci sono proteine nei campioni si 
usano gli istoni in quando sono espressi da tutte le cellule in qualsiasi condizione. Un altro istone importante ma coinvolto nella riparazione del DNA \`e l'H2AX si tratta di un istone 
reclutato nelle zone dove c'\`e danno al DNA il reclutamento di questi istoni permette la riparazione del DNA danneggiato a carico di entrambe le catene. Il danno pu\`o essere fatto 
tramite sostanze usate come antitumorali o un laser. Ci sono pertanto altri istoni coinvolti in altri processi come la riparazione. La dinamicit\`a del nucleosoma ne esiste una una anche
nell'eterocromatina e nell'eucromatina. Il processo di formazione della prima se non bloccato tende a prendere il sopravvento e a trasformare zone eucromatiche in eterocromatiche. Ci sono
pertanto delle barriere costituite o da elementi proteici che bloccano il richiamo di modifica degli istoni o zone di DNA che fanno da barriera. Questo fenomeno sta alla base del 
silenziamento di geni durante lo sviluppo. Vi sono delle mutazioni a carico dei fattori proteici che possono far prendere il sopravvento e va a silenziare zone eucromatiche di geni
che devono essere espressi. I geni per le globine vengono prodotti in diverse fasi dello sviluppo: la gamma alla nascita e poi il beta prende il sopravvento. Questi geni si trovano in 
cluster e vanno incontro a un silenziamento dovuto all'eucromatina e poi all'eterocromatina per i geni per la globulina gamma. Le regolazioni avvengono in quanto le code di istoni 
vanno incontro a modifiche che creano un codice detto codice istonico che viene riconosciuto dal complesso che lo legge e ne interpreta il significato. Ci sono diverse modifiche e a
carico dello stesso amminoacido ci possono essere pi\`u modifiche che possono dare un significato diverso da un punto di vista funzionale. L'interpretazione di queste modiche dipende da 
chi le legge: per l'istone H3 una trimetilazione della lisina determina la formazione di eterocromatina e silenziamento, una acetilazione e la trimetilazione di un'altra determina 
l'espressione genica. Il codice istonico viene riconosciuto da proteine in grado di leggerlo e possono apportare modifiche successive agli istoni promuovendo il compattamento o il 
decompattamento della cromatina. Questi meccanismi di modifica stanno alla base dell'epigenetica, che consiste nelle modifiche post traduzionali a carico di proteine istoniche che possono
comportare delle modifiche a livello di espressione genica che vengono resettate con la formazione della cellula uovo e dello spermatozoo in quanto questo richiede l'attivazione di 
queste modifiche. Una modifica post trascrizionale pu\`o essere ereditata anche dalla progenie. 
\subsection{Sindrome di Rett}
La malattia si manifesta precocemente dopo un periodo apparente di sviluppo normale dopo di che iniziano a comparire segni di sviluppo alterato come una riduzione delle capacit\`a 
motorie, lo sviluppo del cervello viene rallentato. Le caratteristiche a livello morfologico: una riduzione della grandezza del cervello legata a neuroni iposviluppati. Si tratta di una
patologia non legata alla morte cellulare ma si tratta di una malattia dello sviluppo. La mutazione avviene a carico di MeCP2, membro di una famiglia che riconoscono le regioni CG 
metilate e serve come repressore della trascrizione genica in regioni del DNA metilate. Questa proteina controlla la trascrizione: la cromatina attiva presenta acetilazione delle code
istoniche e queste code non sono metilate. La metilazione a livello di regioni ricche di C e G \`e legata alla presenza di metil trasferasi che vengono reclutate nelle zone promotrici
di questi geni e metilano le regioni del DNA ricche di questi elementi. Le code istoniche sono ancora acetilate: la cromatina \`e in uno stato eucromatico. Si reclutano il complesso NUrd 
e Sin3A che fanno parte di un complesso che contiene le acetilasi, cosa che comporta il compattamento interiore della cromatina e la metilazione da parte delle isole ricche di sequenze 
C e G il compito di MeCP2 favorisce la trasformazione da stato eucromatico ad eterocromatico, lo fa quando riconosce zone metilate nelle regioni ricche di C e G. Siccome MeCP2 favorisce 
la trasformazione da uno stadio eucromatico a eterocromatico quando non \`e presente non si forma eterocromatina silente ma uno stadio eucromatico con una certa capacit\`a trascrizionale.
Mecp2 \`e pertanto un inibitore trascrizionale. Questo vuol dire che se si blocca un inibitore si attiva e i pazienti che soffrono della sindrome di Rett hanno cellule che continuano a
trascrivere geni normalmente disattivati. La zona del DNA non \`e bloccata e si esprimono geni che non devono essere pi\`u espressi. Questi geni sono coinvolti nella crescita delle 
cellule nervose (NGF nerve grow factor) che causa problemi in quanto altri fattori che vengono over espressi comportano l'alterata morfologia e sono fattori che impediscono il 
differenziamento delle cellule nervose e i neuroni funzionano male: non formano sinapsi e dendriti, pertanto funzionano meno rispetto a cellule nervose normali. La modifica da eucromatica
a eterocromatica avviene grazie al reclutamento di enzimi che modificano gli istoni a livello post trascrizionali. Questa modifica viene letta da proteine che reclutano proteine che
modificano gli istoni. Questa onda si propaga con le modiche dei cromosomi che causa la formazione di eterocromatina. L'onda viene bloccata per zone eucromatiche attraverso proteine 
barriera ancorate a pori nucleari. Le zone eucromatiche ed eterocromatiche sono distribuite con le seconde localizzate verso la periferia, un modo per ancorare porzioni di DNA vicino
alla membrana. Un altro meccansimo sono proteine che svolgono ruolo di barriera verso modifiche a monte e infine proteine che vanno a bloccare la modifica a monte e impediscono la 
propagazione dell'onda. Queste proteine che leggono il codice e interpretano il codice istonico contengono dei domini particolari: si trova il bromodominio caratteristico delle proteine
che aggiungono gruppi acetile agli istoni, tipico delle HAT, un altro dominio presente in proteine che riconoscono e modificano per l'aggiunta dei gruppi metilici \`e il cromodominio. 
Questi domini riconoscono e amplificano le modifiche a carico degli istoni. Si passa da una condizione poco compatta collana a perle -solenoidi - anse cromosomiche - cromosomi mitotici. 
Si trova un'altra situazione osservata nei cromosomi degli oociti dello xenopus, rana utilizzata per esperimenti dello sviluppo con oociti molto grandi. Queste anse sono organizzazioni
ultrastrutturali che fuoriescono dal cromosoma e si tratta di cromatina in grado di trascrivere. Si tratta di un'organizzazione che permette alla cromatina di formare anse in grado di
trascrivere. I cromosomi nell'interfase non si trovano sparsi all'interno del nucleo ma presentano domini distinti. I cromosomi presentano anche nella fase di non compattamento dei 
domini discreti. L'ibridazione in situ \`e una tecnica in cui si prende una sonda di DNA a filamento singolo specifica per un determinato cromosoma e la si rende radioattiva o 
fluorescente. A questo punto si ibrida, ovvero si permette alla sonda complementare di ibridare all'interno del nucleo e va a ibridare dove va a trovare la sequenza complementare. I 
cromosomi normalmente sono mobili e possono modificare la posizione all'interno del nucleo in base allo stato trascrizionale. Geni non attivi localizzano in periferia e mentre si attivano
queste regioni tendono a muoversi verso zone pi\`u interne nel nucleo. Si ipotizza che questo movimento avvicini i geni in trascrizione alle strutture per la modifica degli mRNA come 
per lo splicing (rimozione intrioni) o il nucleolo dove avviene di fatto il controllo qualit\`a dell'RNA che viene processato ma viene legato da proteine RNA binding protein che lo 
stabilizzano e facilitano il trasporto dal nucleo al citoplasma. L'incontro \`e mediato dal nucleolo. Questo spiega il cambio posizionale delle regioni del DNA che da eterocromatiche si
muovono verso il centro nel momento in cui diventano attive dal punto di vista tascrizionale. Un altro aspetto \`e legato agli istoni durante il processo di replicazione che deve essere
coordinato in quanto si devono far scivolare gli istoni o rimuoverli e quando la catena si \`e sintetizzata si deve fare in modo che gli istoni tornino a interagire con il DNA. Il 
processo di duplicazione durante la fase S \`e preceduta dalla sintesi degli istoni. Il meccanismo della distribuzione degli istoni. La modifica degli istoni presenti pu\`o avere 
conseguenze sull'accessibilit\`a dei geni. Si sono marcati in maniera selettiva le subunit\`a istoniche per capire cosa succede loro durante il processo di replicazione. Il tetramero 
H3-H4 rimane legato casualmente a uno dei due filamenti senza  mai staccarsi e i dimeri H2A e H2B vengono rilasciati e vanno a far parte del pool di dimeri presenti nella cellula. Si 
mischiano e vengono ripartiti in maniera random tra le due catene. Questo \`e mediato da proteine che facilitano l'assemblamento dei nucleosomi e fattori di modifiche degli istoni che 
promuovono le modifiche della struttura della cromatina. Questa macchina \`e associata e interagisce con la macchina coinvolta nella duplicazione del DNA. 
\section{Struttura dell'RNA}
L'RNA non interagisce con istoni ma normalmente \`e associato a proteine la maggior parte delle quali appartengno alle RNA binding protein che lo stabilizzano, controllano il trasporto
e controllano la traduzione. L'RNA fa da tramite tra l'informazione del DNA e le proteine trasportandola fuori dal nucleo. L'RNA differisce dal DNA in quanto contiene il ribosio che 
possiede un gruppo OH che ha un ingombro sterico che gli impedisce di assumere determinate forme e assume la forma A. A differenza del DNA le basi possono andare in contro a maggiori 
modifiche che hanno ruoli diversi: maggiore stabilit\`a, propriet\`a biologiche diverse: la modifica di un microRNA che interagiscono con l'mRNA si modifica la capacit\`a di interagire
con la base complementare e una modifica post trascrizionale pu\`o cambiare la loro funzione biologica. L'RNA pu\`o assumere strutture a doppia elica, la struttura ad ansa a forcina con 
stelo a doppia elica e un loop a singolo strand ci possono essere appaiamenti non canonici in queste strutture. Sono importanti in quanto conferiscono stabilit\`a all'RNA e possono essere
riconosciute da proteine che svolgono diversi ruoli. La struttura secondaria a stem loop dell'RNA pu\`o avere un ruolo importante nel metabolismo dello ione ferro. Lo ione ferro \`e 
molto reattivo e tende a reagire con l'ossigeno (reazione di fempton) e produce radicali che possono danneggare il DNA. Pertanto lo ione ferro nonostante sia utile (importante fattore
nei globuli rossi), un eccesso di ioni ferro a causa di prodotti legati al metabolismo cellulare pu\`o dare origine a specie reattive pericolose. Il meccanismo viene regolato pertanto. 
Ci sono dei recettori sulla membrana detti recettori per la transferrina che lega la transferrina, la molecola che si lega al ferro presente nel torrente circolatorio che si lega al
suo recettore sulla membrana. Una volta che il recettore si lega alla transferina viene internalizzato si forma una vescicola endocitotica che quando presente nel citplasma diventa un
endosoma che al suo interno grazie a determinati trasportatori va incontro a una riduzione del pH che causa il distacco dello ione ferro dal suo recettore. Questo determina il rilascio
dello ione nel citoplasma e il recupero dell'endosoma e della transferrina scarica e torna in membrana dove si dissocia e la transferrina vuota pu\`o tornare nel circolo sanguigno. 
A questo punto allo ione ferro rilasciato nel citoplasma in parte viene utilizzato dove necessario come attivatore o coattivatore di enzimi e proteine. Quello in eccesso viene legato 
dalla ferritina che di fatto blocca lo ione ferro libero non necessario per reazioni e lo mantiene silente e ne impedisce che venga coinvolto in reazioni pericolose. Quando si ha poco
ferro si devono aumentare i recettori di membrana e ridurre i livelli di ferritina presenti nel citoplasma. In carenza di ferro si aumentano i recettori e diminuiscono i livelli di 
ferritina, in condizioni di eccesso di ferro avviene il processo inverso. Questo meccanismo di regolazione  \`e post trascrizionale e avviene a carico degli RNA per il recettore e la
ferritina una volta che sono stati generati. Una proteina \`e un RNA binding protein aconitasi che lega l'RNA e lo lega riconoscendo la struttura a stem loop presenta nel $5'$ non 
codificante per la ferritina e $3'$ non codificante (UTR untraslated region)per il recettore. Un mRNA possiede a monte e a valle sequenze pi\`u o meno lunghe che non vengono tradotte in
proteine e servono per il controllo e regolazione. Possono trovarsi in $5'$ o $3'$. Quando l'aconitasi riconosce questa struttura a monte per il gene della ferritina e a valle per la
transferrina. L'aconitasi si lega alle strutture in condizioni di bassi livelli di ferro. Quando non c'\`e ferro l'aconitasi si lega agli RNA e per la ferritina va a bloccare il 
reclutamento dei ribosomi sull'mRNA bloccando la traduzione della ferritina. Contemporaneamente si lega Alla $3'$ UTR a valle del gene del recettore e lo stabilizza e pu\`o essere 
tradotto in maniera pi\`u efficiente e avviene la produzione del recettore, cosa che ci serve per internalizzare pi\`u ferro. La stessa struttura a monte o a valle pu\`o dare origine a 
due effetti biologici diversi. Con elevati livelli di ferro si vuole inibire il recettore e aumentare la ferritina. Il ferro si lega all'aconitasi che cambia conformaizione e non \`e
pi\`u legata alla struttura a forcina permettendo la traduzione dell'RNA per la ferritina e dall'altra parte destabilizza l'RNA che codifica l'RNA per il recettore che pu\`o andare in
contro a una pi\`u rapida degradazione. L'aconitasi \`e un sensore per il ferro. L'elemento a forcina prende il nome di iron responsive element IRE. Questo meccanismo viene sfruttato per
creare un sistema fluorescente atto a misurare i livelli di ferro all'interno della cellula questo si svolge utilizzando la GFP: prendendo la GFP come reporter si mette l'RNA a valle
della sequenza dell'IRE, che trascrive per l'RNA ibrido con IRE e il reporter. Nella condizione senza ferro la ferritina non deve essere prodotta e l'aconitasi si lega all'IRE e blocca
la traduzione e non si dovrebbe osservare espressione di GFP. In condizioni di presenza di ferro questo si lega all'aconitasi questa si stacca dalla sequenza a forcina e ne permette la
traduzione. Le strutture secondarie si possono formare in maniera spontanea e la stabilit\`a dipende dalla sequenza e dall'appaiamento delle basi. Pu\`o essere quantificata in base 
alla quantit\`a di energia usata per formare gli accoppiamenti delle basi. Tanto pi\`u negativo il valore di $\Delta G$ tanto pi\`u stabili le strutture. In alcuni casi possono 
intervenire proteine che permettono all'RNA di assumere strutture proteine. Questo \`e uno dei problemi della previsione delle strutture secondarie. In molti casi infatti le strutture 
predette possono non essere veritiere a causa delle proteine presenti. I tRNA hanno una struttura a trifoglio importante per la funzione: il tRNA che porta l'amminoacido nella zona 
$3'$ con una struttura a forcina che forma l'anticodone che riconosce il codone sull'mRNA che determina l'introduzione dell'amminoacido specifico sulla proteina.
