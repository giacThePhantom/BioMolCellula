\chapter{Citoscheletro}
Il citoscheletro \`e importante per diverse funzioni. Il citoscheletro \`e importante per il mantenimento della forma della cellula, fornisce resistenza agli stress, mantiene la forma
del nucleo e permette il movimento alla cellula di aderire al substrato e di muoversi, se \`e alterato e impedisce alla cellula di aderire al substrato, estremamente importante per il
movimento cellulare e per il differenziamento della cellula. Il citoscheletro \`e importante in quanto fornisce i binari per il movimento di organelli e proteine. Permettono la 
segregazione dei cromosomi nelle cellule figlie. \`E costituito da tre elementi: i filamenti di actina, i microtubili e i filamenti intermedi con funzioni diverse e disparate. I 
filamenti di actina o microfilamenti sono costituiti da delle subunit\`a globulari di actina G e i monomeri si uniscono tra di loro fomrando filamenti flessibili dal diametro di 8 
nanometri che tendono ad avvolgersi l'uno sull'altro. La si trova in tutte le cellule dove \`e organizzata in maniera diversa. Forma le stress fibers e forma la struttura ordinata nel 
muscolo che ne garantisce la funzionalit\`a e ne permette la contrazione. I microtubili formano dei tubi cavi costituiti da monomeri di due forme di tubulina, hanno una lunghezza 
notevole rispetto alla dimensione della cellula e sono liberi ad un estremit\`a mentre all'altra sno attaccata al microtubule organizing center o centrosoma (MTOC) da cui si orginiano
microtubuli con cui rimane in contatto. UNa partitcolare organizzazione si trova nelle ciglia che hanno un ruolo importante. I filamenti intermedi che hanno dimensioni e funzionalit\`a
diverse, resistenza agli stress e si trovano presenti alle cellule epitaliali e formano l'impalcatura al di sotto della membrana nucleare (lamina) vanno incontro a dei rimaneggiamenti
duranti particolari fasi della embmrana nucleare: la lamina durante la divisione deve essere disassemblata. La plasticit\`a degli elementi del citoscheletro \`e notevole. Il citoscheletro
di filamenti intermedio, i microtubili e i filamenti di actina, i microtubuli permettono le adesioni cellula cellula e con la lamina basale (ancoramento al substrato). 
\section{Actina}
I filamenti di actina sono costituiti da actina globulare una proteina di 375 amminoacidi molto conservata presente in tutte le cellula e pu\`o essere considerata come gene housekeeping
e permette la cura e ilmantenimento della casa, permette alla cellula di funzionare in maniera corretta. I monomeeri di actina sono in grado di legare ATP e ADP, importante nel 
processo di formaizone e disassemblamento di actina, molto conservata nell'evoluzione ed esistno tre isoforme di actina $\alpha$, $\beta$ e $\gamma$, la prima solo nelle cellule 
muscolari. L'actina ha una strututra globulare con una polarit\`a, una porzione pi\`u e una meno e il dominio che lega ATP o ADP si trova verso la porzione meno, questo nel momento in
cui si assemblano fomrano il filamento si distinuge una terminazione pi\`u e una meno. L'assemblamento dei monomeri avviene nel momento in cui il monomero lega AATP e si assemblano
in entrambe le direzioni, la crescita in direzione pi\`u e meno pu\`o avvenire solo che la crescita in direzione pi\`u \`e pi\`u rapida rispetto all'altra direzione. Questo \`e 
importante nel contesto della cinetica e la formazione dei filamenti di actina. I monomeri di actina G si possono identificare due polarit\`a, una a spina (pi\`u) e una a barbiglio (
meno). I monomeri si associano secondo una costante $k_{on}$ e si disassociano secondo una costante $k_{off}$. I filamenti di actina, se si prendono i monomeri in un tubo di reazione
l'actina pu\`o formaare in maniera spontanea dei filamenti se si formano oligomeri tre subunit\`a di actina si uniscono tra di loro e creano un nucleo. Importante per dare origine al
processo di polimerizzazione e aggiunta di successive subunit\`a di actina per la formaizone del filamento. La reaizone \`e molto lenta e quando si forma il complesso di nucleazione la
velocit\`a con cui il filamento polimerizza aumenta con velocit\`a esponenziale fino a quando si ragginge un equilibrio e le subunit\`a si riducono in concentraizone e si raggiunge uno
stato di equilibrio in cui la velocit\`a con cui vengon introdotte nuove subunit\`a \`e uguale alla velocit\`a con cui le subunit\`a lasciano il filamento e la lunghezza rimane costante. 
Diventa limitante quando il processo rallenta. Per l'importanza della nucleazine e della formazione del trimero fornendo un trimero all'inizio e promuovendo la sua formaizone di fatto non
si ha la fase iniziale lenta ma parte il meccanismo di polimerizzazione. ALl'interno della cellula ci sono porteine accessorie dell'actina che favoriscono la formazione di nuclei e
sono importanti in quanto velocizzano molto la fase di inizio. Un fenomeno legato alle propriet\`a dei monomeri di legare e idrolizzare ATP in ADP \`e il treadmilling che consiste nel
meccanismo in cui monomeri di che legano ATP si uniscono promuovendo polimerizzaizone e i monomeri gi\`a introdotti idrolizzano ADP e a un estremit\`a tendono a staccarsi e trovano ATO
si legano ad esso e possono essere riutilizzati e reintrodotti, se il neointrodotto mano a mano che si aggiungono altri monomeri questo tende a spostarsi a idrolizzare ATP e a trovarsi 
in fondo dove si stacca e pu\`o essere utilizzato che consente all'equilibrio di mantenere costante la lunghezza e i monomeri vanno incontro a un processo di distacco e reniserimento. 
Viene utilizzato per le propriet\`a dei filamneti di actina. Ci sono fattori cellulari che bloccano un'estremit\`a, altri fattori possono legarsi ad una g protein e promuovono 
l'allungmanteo in un altra diezione creando una rete di filamenti di actina che si intrecciano tra di loro. Si trovano proteine che alterano queste propriet\`a. Il citoscheletro
va incontro a profondi riarrangiamenti durante la fase di divisione cellulare. Questo \`e importante per gli organelli che vengono mantenuti tali grazie alla presenza dei microtubili
che permettono la compattazione di ER e Golgi e forniscono il substrato per i movimenti delle proteine e delle vescicole. Duratne il processo mitotico gli organelli si disgregano insieme
al citoscheletro sia a livello nuclerare fino a tutti i microtubuli, non tutti disassemblati in questa fase in quanto il fuso mitotico viene costruito in modo tale da poter connettersi
con i cromosomi e permettere la loro segregazione. L'actina \`e costituita da monomeri di proteina actina G globulare che vanno incontro a un processo di polimerizzazione e 
depolimerizzaione diversi in base alla polarit\`a. Sono costituite da due estremit\`a una a punta e una a barbigli che si uniscono tra di loro accrescendo il filamento in maniera diversa
rispetto alla direzione della polimerizzaizone, la struttura asimmetrica determina una polarit\`a pi\`u all'estremit\`a a punta e una polarit\`a meno per i barbigli. La quale contiene
il sito di legame per l'ATP. Le portien G leganti ATP vengono incorporate in direzione Pi\`u e meno ma l'accrescimento nella direzione pi\`u rispetto a quella della polimerizzazione meno
legato alla struttura dei monomeri. Una volta che \`e stata incorporata avviene l'idrolisi di ATP e ADP che determina una minor stabilit\`a dei monomeri che sono pi\`u propensi a lasciare
il filamento che possono rilegare ATP e reintrodotti in una delle due polarit\`a, la velocit\`a di associazione e dissociazione \`e legata adelle costanti la cui velocit\`a \`e diversa. 
Il processo di nucleazione \`e importante in modo che possa esserci polimerizzazione occorre che si crei un nucleo costituito da oligomeri, tre subunit\`a di actina che dopo la formaizone
l'incorporazione delle subunit\`a delle portiere globulari nei filamenti aumenta in maniera esponenziale fino alla condizione di equilibrio. Se si forniscono gli oligomeri in vitro 
l'organismo di polimerizzaizone avviene istantaneamente, la velocit\`a in vitro \`e diversa da quella in vivo in quanto in vivo sono presenti delle proteine che aiutano promuovendo 
la formaizone dei centri di nucleazione dei trimeri di actina che danno il via alla polimerizzazione e molecole che si legano ai filamenti di actina e altri che si legano ai filamenti
promuovendo la polimerizzazione e altre che promuovono la depolimerizzaizone dei filamenti di actina. Il cambiamento di percorso \`e determinato dalla formazione di citoscheletro in 
un'altra parte e occcorrre che ci sia depolimerizzaione da un altra in modo che l'asimmetria crei un cambiamento di direzione. Il processo di polimerizzazione e depolimerizzaiozne \`e 
controllato da fattori presenti nella cellula. Vi sono delle sostanze che vengono prodotti da organismi vegetali che producono molti alcaloidi, sostanze che agiscono sul citoscheletro,
sui filamenti di actina e microtubuli e vengono utilizzate come agenti chemioterapici in quanto possono destabilizzare il citoscheletro in momenti chiave come la divisione. Queste 
sostanze come nel caso dell'actina possono essere per l'actina la latruncolina che depolimerizza i filamenti di actina con il rilascio di monomeri che si lega alle subunit\`a e 
destabilizza le suvunit\`a che vengono rilasciate, la citocalasina B con attivazione depolarizzante e tende a legarsi alle estremit\`a pi\`u dei iflamenti la falloidina invece estratta
dall'amanite falloide stabilizza i filamenti di actina e pu\`o essere deleterio per la cellula. Sono state utilizzate per studuare le propriet\`a del citosceletro dell'actina la 
falloidinda resa fluorescente pu\`o essere usata per colorare il citoscheletro di actina epu\`o essere utilizzata per marcare il citoscheletro.Per i meccansimi di rtasporto legati 
all'actina si pu\`o utilizzare la latruncolare o la citocalasina B per destabilizzare i filamenti e verificare se il trasporto avviene o no. Il tassolo estratto dal tasso stabilizza
i microtubuli, il nocodazolo e la colchicina favoriscno la depolimerizzazione dei microtubuli, sono utilizzati come antitumorali in quanto vanno a destabilizzare i microtubuli del 
fuso mitotico durante il processo di divisione cellulare. Moleocle che svolgono ruoli nel promuovere la polimerizzazione o la depolimerizzazione dei filamenti di actina. La timosina e 
la profilina si tratta di proteine che legano i monomeri di actina con funzione opposta: la timosina lega il monomero di actina e lo rende indisponibile per la formaizone del filamento
la profilna favorisce l'incorporazione dei filamenti favorento la polimerizzazione e competono con i monomeri di actina, se la cellula esprime la timosina il pool di proteine G
disponibile per l'incorporazione \`e minore per la cellula con espressione di profilina. La profilina pu\`o essere modificata con fosforilazione e dopo l'affinit\`a della profilina per 
i monomeri si riduce e la sua attivit\`a di promotoer viene inibita. In condizioni del ciclo cellulare o in altri condiizoni fisiologiche perde affinit\`a per i monomeri favorendo il 
legame della timosina con i monomeri riducendo l'attivit\`a di plimerizzazione. Altri fattori sono Arp 2/3 (actine related protein) e la formina. Questi due complessi favoriscono la 
formazione dei nuclei per promuovere la sintesi dei filamenti di actina. Lavorano in maniera diversa in quanto Arp 2/3 \`e costituito da due molecole Arp2 e Arp3 con struttura simile a
quella dell'actina, la funzione del complesso \`e quella di favorire la crescita di filamenti di actina partendo dall'estremit\`a meno, si legano grazie all'interazione con altre 
proteine accessorio li rendono capaci di interagire con i monomeri di actina e promuovere la formazione dei nuclei da cui prende il via la formazione dei filamenti di actina. Si noti
come Arp2/3 permettono la formazione di ramificazioni di branching e la formazione di filamnti partendo da uno preesistenti, questi fattori possono legare anche delle subunit\`a di actina
gi\`a incorporate nel filamento e interagendo con essi creano un centro di nucleazione in una posizione del filmaento preesistente con la formazione di filamenti laterali con un angolo
di 70 gradi con una rete di filamenti di actina. La formina invece promuove la formazione di centri di nucleazione ma la formina lo fa ma non \`e in grado di produrre e di attaccarsi a
monomeri all'interno di un filamento preesistente e creare le strutture a rete, la formina \`e un dimero che favorisce l'interazione di monomeri di actina tra di loro dando il via alla
polimerizzazione e alla formazione dei filamenti di actina. Alcuni membri delle formine presentano dei domini baffi che hanno la funzione di reclutare altri fattori che promuovono 
l'incorporazione di monomeri di actina all'interno del filmamento di actina come profilina che favorisce il reclutamento dei monomeri di actina. Vi sono poi altre molecole che vanno 
a regolare le propriet\`a dei filamenti di actina. Vi sono molecole che hanno un ruolo nella stabilizzazione e nell'inibizione nella formazione dei filamenti di actina come 
la tropomiosina nel muscolo che si posiziona sopra i siti di attacco della'ctina conla misina andando a bloccare l'interazione tra le teste di miosina e actina impedendo la loro 
interazione e la contrazione del muscolo, rimossa dalla troponina. La tropomiosina oltre a inibire i siti di rezione e rende rigido il filamento di actina. Un altra proteina nel
muscolo \`e CapZ che localizza ed \`e la responsabile della formazione della banda Z del muscolo. Un'altra proteine accessoria legata ai filamenti di actina che va a stabilizzare il 
filamento di actina in un'estremit\`a rendendola inattiva. Di fatto La presenza di CapZ a un'estremit\`a impedisce la crescita e la polimerizzazione dei filamenti di actina in quanto
blocca un'estremit\`a che viene bloccata. La polimerizzazione pu\`o avvenire in un'unica dirzione e si localizza nelle zone che delimitano il sarcomero. La tropomodulina ha la funzione
di rivestire e incapsulare le cellule muscolari e i filamenti di actina legandoli tra di loro importante per mantenere la funzionalit\`a, struttura e resistenza. Vi sono molecole che
regolano e promuovono la depolimerizzaizone dei filamenti di actina come la gelsolina, membro della superfamiglia che viene attivata a seguito del rilascio di calcio all'interno della
cellula che la attivano e va a destabilizzare il filamento di actina promuovendo la depolimerizzazione in maniera analoga funziona la cofilina che va a forzare il filametno di actina a 
compattarsi e il compattametno ulteriore lo va a stressare e lo stress rende instabile il filamento stesso promuovendo il distacco dei monomeri in particolare di quelli che legano ADP 
che tendono gi\`a a staccarsi e la presenza della cofilina favorisce ulteriormente il distacco di questi monomeri e di fatto sembra essere importante per rinnovare i filamenti di
actina per cambiare i momomeri che legano ADP iini quelli che legano ATP. Queste proteine sono importanti per permettere ai filamenti di actina per formare complessi di ordine superiore:
le reti dendritiche, i fasci e le strutture a rete. Le strutture parallele pi\`u o meno compattte nei filopodi, estroflessioni sensoriali della membrana che permette alla cellula di 
muoversi in una direzione, ci sono strutture a rete caratteristiche e fungono da impalcatura per i lamellipodi, la pelle tra le dita sono ricchi espansioni della membrana plasamtica molto
mobili. Le stress fibers sno importanti per l'interazione della cellula con il substrato dosve si trova iuna conformazione parallela simile a quella dei muscoli anche ocon orientamento
diverso e la compattezza di queste strutture \`e inferiori di quella dei filopodi, si trova una struttura pi\`u disordinata che crea una rete a gel. Tutte queste strutture si fomrano
grazie alla presenza delle proteine viste in precedenza. Si trovano altri fattori che influenzano stabilit\`a e dinamicit\`a delle strutture come la fimbrina che lavora come monomero e 
si lega a due monomeri di actina, l'$\alpha$ actinina lavora come dimero e ciascuna lega un unico monomero di actina. La spectrina \`e la componente importante del citoscheletro al di
sotto della membrana dei globuli rossi, cellule prive di nucleo e la spectrina conferisce la resistenza alla membrana  in quanto soggetto a stress notevole e una certa elasticit\`a alla
membrana in quando deve poter deformarsi per passare nei capillari per poi tornare nella forma iniziale. Proteina molto lunga costituita da due subunit\`a $\alpha$ e due $\beta$ e solo
la prima lega l'actina. Fungono inoltre da ponte tra citoscheletro e altre porteine che interagiscono con la membrana o citoplasmatiche che vengono ancorate al citoscheletro. La filamina
responsabile della formazione di srtutture a croce, un dimero che presenta in cui ciascun monomero lega una subunit\`a di acitna \`e molto elastica e la strututra a V permette di rendere
stabile le strutture a X a rete caratteristiche di alcuni tipi cellulari e di alcune zone della cellula. Mutaizoni della filamina sono responsabili di patologie che danno ripercussioni 
sulla migrazione di neuroni dalle zone ventricolari dove ci sono i precursori dei neuroni che migrano verso le zone apicali differenziandosi, questo movimento dalla parte ventricolare 
viene compromesso a seguito di mutazioni a carico della filamina in cui i neuroni non si muovono e rimangono in blocco. Creando la rete favorisce la migrazione delle cellule precursori 
dei neuroni probabilmente. La fimbrina funziona come monomero e la si trova dove i filamenti di actina forkmano strutture a fasci legate in maniera stretta e legando diversi fasci di 
filamenti di actina li lega in maniera pi\`u stretta rispetto all'$\alpha$ actinina con ingombro maggiore rispetto alla fimbrina e si trova nei muscoli dove permette contrazioni 
mantenendoli ordinati e stabili strutturalmente paralleli. La miosina \`e una famiglia di proteina costituite da due catene con un dominio elicoidale importante per la formaizone del 
dimero, un collo e una tescta che interagisce con i monomeri di actina, ci sono due catene pesatni e due leggere, la parte N terminale interagisce con l'actina o altri interattori, 
non esiste solo questa proteina quella muscolare \`e la miosina due, ci sono altri tipi di miosina che sono caratterizzata da un dominio motore e dette proteine motrici insieme a 
chinesine e dineine e hanno altri domini per interazioni per formare dimeri o assumere strutture peculiari, altri tipi di miosina come I, III che lavorano come monomeri, V dove possono
interagire da un lato con citoscheletro e regioni che possono interagire con vescicole o organelli ed essere coinvolto nel trasporto di altri filamenti del citoscheletro come filmaenti 
intermedi, fungere da ponte tra componenti citoscheletriche diverse o il trasporto di vescicole, il dominio motrice con ilfamento e altre porzione interagisce specificatamente con un
cargo come vescicola o altro filamento. La miosina V usa i filamenti di actina per muovere mitocondri dalla cellula madre alla figlia durante il processo di gemmazione del lievito. 
\section{Microtubuli}
I microtubuli sono dei polimeri costituiti da una proteina detta tubulina. La tubulina \`e di due tipi si trova l'$\alpha$ e $\beta$ tubulina che si associano non covalentemente per 
formare dimeri, come l'actina legano un GTP di questi l'attivit\`a GTPasica avviene a carico della $\beta$ tubulina. Si uniscno insieme formando il protofilamento. Il microtubulo \`e 
formato dall'interazione di 13 protofilamenti che formano un microtubulo con una struttura cava che garantisce una certa robustezza e una certa elasticit\`a. Si possono piegare 
permettendo il movimento di ciglia e flagelli la cui anima \`e costituita da microtubuli, sono strutture resistenti ma flessibili. I protofilamenti interagiscono tra i monomeri della
stessa specie. Questo crea un andamento e struttura elicoidale del microtubulo. Inoltre si pu\`o distinugere una polarit\`a negativa rivolta alla base che presenta la subnit\`a $\alpha$
e una polarit\`a positiva rivolta verso la subunit\`a $\beta$. La formazione dei microtubuli si foramno in maniera analoga ai filamenti di actina con una rapida crescita verso pi\`u 
el aformazione del protofilamento avviene on l;incoroporazione dei dimeri, poi la subunit\`a $\beta$ idrolizza il GTP favorendo il processo di depolarizzazione ora il monomero $\beta$ 
pu\`o tornare a legare GTP e essere riutilizzata. Fino a che sit rva il cappuccio e viene mantenuta una estremit\`a contenente iun certo numero di dimeri contenenti leganti GTP il 
processo di polimerizzazione pu\`o avvenire, quando la maggior parte dei dimeri \`e legata a GDP avviene una destaiblizzazione con il disassemblamento delle subunit\`a, processo 
catastrofico che porta un'instabilit\`a del microtubulo con il rilascio delle varie subunit\`a. Ci sono sostanze agenti antitumorali che staiblizzano o destabilizzano i microtubuli come
taxolo, nocodazolo o la colchicina. I microtubuli sono strutture altamente dinamiche e si possono formare in maniera rapida. La dinamicit\`a \`e garatnita da fattori specifici e 
sbolgono il ruolo analogo dei filamenti di actina. I microtubuli si formano grazie alla creazione di un nucleo ad opera di un terzo tipo di tubulina, la tubulina $\gamma$ \`e un 
componente importante del centro organizzatore dei microtubuli, il centro dove si creano (MTOC, centro organizzatore dei microtubuli). Questo centro \`e costituito da tubulina $\gamma$ 
e proteine accessorie che formano il complesso $\gamma$-TUSC, struttura importante in quanto contribuisce a creare il centro di nucleazione da cui vengono prodotti i microtubuli. Il 
centro organizzatore dei microtubuli in molte cellule animali viene detto centrosoma, costituito da siti di nucleazione dove parte la polimerizzaizone con $\gamma$-Tusc e al suo interno
presenta due centrioli. I centrioli sono costituiti da microtubuli particolari che formano dei trimeri e sono uniti tra di loro formando un centriolo formato da 9 trimeri. Questa
struttrua \`e importante e viene duplicata poco prima della divisione ed \`e importante in quanto da l\`i si origina il fuso mitotico da dove si formano i microtubuli che si attaccano ai
centromeri che permettono la segregazione dei cromosomi. Il centrosoma con i centrioli e i microtubuli con la polarit\`a positiva verso la periferia della cellula e quella negativa verso 
il centrosoma, vale in generale e nel caso delle cellule neuronali la polarit\`a pu\`o essere diversa. Questa polarit\`a \`e imoprtante in quanto pu\`o essere riconosciuta da dinesine 
(trasporto (pi\`u meno) e chinesine (deputate al trasporto meno pi\`u). Per trasporto verso periferia usano dineine e quello in senso opposto chinesine. Vi sono proteine e fattori che
permettono e stabilizzano e destabilizzano i microtubuli consentendo un certo turnover come le proteine associate ai microtubuli o MAP. DUe sono importanti nel contesto delle cellule
nervose e sono MAP2 e la proteina tau, interessanti in quanto MAP2 la si trova nei dendriti delle cellule neuronali, mentre tau localizza nell'assone. La funzione di queste proteine 
\`e di associarsi ai microtubuli interagendo con i protofilamenti e legano tra di loro protofilamenti, MAP2 ha un dominio molto pi\`u lungo e consente un compattamento pi\`u lasso 
rispetto a tau dove il dominio \`e corto e permette un compattamento maggiore. Il cimpattamento \`e importante per garantire una certa rigidit\`a alla struttura e mutazioni di tau 
possono avere ripercussioni importanti nei trasporti, molte taupatie comportano delle ripercussioni sul funzionamento della cellula nervosa con gravi compromisssioni, MAP2 e tau possono
essere utilizzati come marcatori specifici di un compartimento. CI sono altre componenti importanti dei microtubuli ci sono altri fattori che promiovono la stabilit\`a e la formazione di 
microtubuli e altri che destaiblizano la struttura e promuvoono la depolarizzaione, XMAP11 promuove la stabilit\`a e la chimnesina13 \`e un fattore catastrofico in quanto destabilizza
il cappuccio e rimuovendolo e destabilizzanodlo promiove l'evento catastrofico, XMAP11 compete con la chinesina13 stabilizzando il CAP promuovendo la fomrazione dei microtubuli, ce ne
sono altre che si attaccano a un'estremit\`a come  + TIP che si legano in maniera specifica all'estremit\`a pi\`u del microtubulo e altre proteine come EB1 che viaggia lungo il 
microtubuolo e si posizione sulla polarit\`a pi\`u e recluta altri fattori che poi possono essere coinvolti nella stabilizzazione o destabilizzazione. Ci sono fattori che destabilizzano
i microtubuli come sostanze alcaloidi estratte da funghi, in questo caso prodotte dalla cellula e attivate in condizioni come la  statmina che legandosi a due dimeri di tubulina impedisce
che questi vengano incorporati nel protofilamento. Due proteine interagiscono con i microtubuli e hanno ruolo di trasporto: sono due tipi di proteine e utilizzano i microtubuli come 
binari e sono importanti per il trasporto di vescicole e protiene, sono le chinesine e le dineine, proteine motrici con un dominio che interagisce con i microtubuli, il dominio motore
che \`e presente nella porzione N terminale e in altri pu\`o essere presente nella porizone C terminale nel caso della chinesina13 o della chinesina14, hanno tutte un dominio motore
e sono similli alla miosina e formano dimeri caratterizzati da catene pesanti e catene leggere, quello delle catene leggere \`e poco chiaro ma sembra siano importanti per interagire con
il cargo specifico. Il dominio motore \`e un dominio ATPasico e l'ATP ha un ruolo importante in quanto il legame di ATP con il dominio determia dei cambi conformazionali che permette
alla proteina motrice di muoversi lungo il microtubuli, ne presentano due in quanto permette loro di camminare lungo il microtubulo alternando lo stato di legame ATP e ADP. Una
subunit\`a legata a ADO e l'altra da ATP, l'idrolisi determina il cambio conformazionale che porta la subunit\`a che ha subito l'idrolisi di avanzare e porsi di fianco rispetto alla
subunit\`a che in precedenza legava ADP. La velocit\`a di movimento di alcuni cargo \`e notevole. Si mette in risalto oltre alle teste anche il ruolo della struttura di collo che 
permettono alle teste di spostarsi e alla chinesina di muoversi lungo il microtubulo. L'altra famiglia di proteine motrici soono le dineine, pi\`u complesse e molto pi\`u grandi, oltre
alle catene pesanti e leggere presentano delle catene intermedie, hanno un dominio motore con attivit\`a ATPasica e un dominio che riconosce i microtubuli, le dineine possono 
essere citoplasamtiche e associate all'assonema e sono nel contesto di ciglia e flagelli con ruolo nel movimento di microtubuli che determina il movimento ondulatorio dei flagelli e 
quello a frusta delle ciglia. Un cargo viene mosso in una direzione pu\`o essere legato non da un'unica chinesina o da un'unica dineina ma da multiple e il rapporto da chinesine a 
dineine d\`a la direzionalit\`a del trasporto. Se prevale chinesine verso pi\`u (centro verso periferia) se prevale dineina verso meno (da periferia verso il centro), le dineine sono
importanti per riscostituire l'apparato del Golgi e ER dopo la mitosi. Il centrosoma \`e il centro di organizzazione dei microtubuli con i centrioli perpendicolari, sono costituiti da 
9 triplette di microtubuli di cui uno \`e completo mentre gli altri due hanno forma simile a mezzaluna, questi tre uniti tra di loro da porteine accessorie costituiscono i centrioli
che hanno un ruolo importante nel mantenimento del ciglio. La katanina il cui nome deriva dalla katana che va a tagliare e a scindere i microtubuli, se la statmina impedisce il
recltumento la katanina va a rompere il legame tra monomeri nei microtubuli. La portiene EB1 va a localizzare nell'estremit\`a in fase di accrescimento e i microtubuli sono etremamente
dinamici, si lega al cap ricco di dimeri legati a GTP. Il citoscheletro oltre a garantire certa rigidit\`a \`e un componente estermamente dinamico. IL movimento avviene utilizzando 
microtubuli e proteine motrici con movimento bidirezionale che avviene grazie a chinesine e dineine. Poich\`e i microtubuli hanno polarit\`a le proteine motrici la riconoscono e median
il trasproto in una direzione, chinesine estremit\`a pi\`u e dineine estremit\`a meno. Le chinesine sono quelle che mediano il trasporto da centro a periferia mentre dineine opposto. 
Il movimento netto \`e datto dalla quantit\`a di chinesine e dineine attaccate in rapporto tra di loro. Le dineine si trattano di portiene motrici molto grandi con catene pesanti e 
leggere ma presentano altri componenti che costituiscono le catene itnermedi, hanno domini motori che tramite idrolisi dell'ATP di cambiare conformazione alla base della capacit\`a di 
muoversi lungo i microtubuli. Lo spostamento di una dineina \`e notevole, sono in grado di effetturare spostamenti notevoli ($8$ nm) e sono quelle coinvolte nel funzionamento e nel 
movimento di due strutture specializzate che sono ciglia e flagelli. Il flagello caratteristico di batteri e di spermatozoi \`e un movimento ondulatorio metnre le ciglia a frusta, sono
coinvolte dineine assonemiche mentre quelle coinvolte nel trasporto sono dineine citoplasmatiche. Nel caso di cellule che hanno una struttura particolare come i neuroni il sistema 
costituito dai microtubuli \`e complesso: nell'assone si trova una polarit\`a negativa vevrso il corpo cellulare e positiva verso la periferia, nei dendriti ci sono microtubuli pi\`u 
corti con polarit\`a mista. Nei dendriti i microtubuli sono pi\`u corti e con polarit\`a diversa rispetto all'assone, tendono a sovrapporsi tra di loro e il trasporto nei dendriti \`e di
tipo saltatorio in quanto ilcargo viene trasportato, si blocca quando arriva all'estremit\`a e va a saltare nel microtubulo adiacente. Andando a vedere alcune strutture come ciglia e 
flagelli sono strutture costituite e presentano al loro interno microtubuli arrangiati in un particolare modo. Hanno struttura simile caratterizzato da 9 dimeri di microtubuli, uno 
completo mentre l'altro a forma di mezzaluna come nel centriolo, costituiscono l'anello pi\`u esterno e all'interno si trova una coppia di microtbuubli completi uni mediante proteine
accessorie circondati da una guaina. I vari microtubuli sono uniti tra di loro mediante nessina che unisce tra di loro i dimeri di microtubuli. L'altra componente importante \`e
costituita dalla dineina con due estremit\`a con un braccio interno e uno esterno, l'interazione di dineine con microtubuli adiacenti permette il movimento di flagello e ciglio, il 
sistema funziona in queanto i microtubuli sono uniti tra di loro e la dineina non pu\`o spostare i microtubuli e l'idrolisi dell'ATP si traduce in un spostamento a frusta o ondulatorio 
dei microtubuli, alla base del funzionamento \`e le dineine e come interagiscono con i microtubuli e come questi interagiscono tra di loro, rompendo i legami tra i microtubuli questi 
sono in grado di scorrere tra di loro grazie alle dineine, quando rimangono legati la dineina causa una flessione alla base dell'attivit\`a ondulatoria o a frusta. Il ciglio \`e simile 
al flagello ma alla base presenta un centriolo che permette di interagire con il citoscheletro circostante. 
\section{Filamenti intermedi}
I filamenti intermedi a differenza di actina e tubulina sono assenti negli animali con scheletri rigidi costituita da chitina ma si trovano nelle cellule di organismi senza questa
struttura. La funzione dei filamenti \`e di conferire resistenza agli stress meccanici alle cellule in particolare epiteliali. Che sono unite tra di loro utilizzando desmosomi giunzioni
molto strette che permette alle cellule di unirsi tra di loro e i filamenti intermedi sono componente importanti dei desmosomi in modo che a stress meccanico non si dissocino tra di loro.
Sono ancorate alla base con emidesmosomi e i filamenti intermedi permettono alle cellule di ancorarsi. La funzione dei filamenti \`e una funzione legata a rendere le cellule resistenti
agli stress meccanici. Si trovano anche nel nucleo dove formano la lamina nucleare, la rete sotto la membrana nucleare e sono costituiti da lamina A, B e C. Alcine proteine che formani
i filamenti intermedi si trovano nel muscolo come desmine, origine mesenchimale come vimentina e in particolare la componente espresssa dalle cellule epiteliali sono le cheratine che 
resistono e vengono mantenute quando le cellule muoiono e costituiscono unghie e capelli. I microfilamenti sono otto tetrameri che sono costituiti a loro volta dal monomeri di una 
proteine filamentosa con strittura ad alpha elica nella struttura centrale, una porzione N e C terminale e all'interno dell'alpha elica ci possono essere ripetizioni fino a 40 volte di
amminoacidi che permette di formare un dimero, i dimeri si uniscono tra di loro, interazione favorita da caratteri idrofobici e a differenza di microtubuli e filamenti di actina non
hanno polarit\`a, la porizone N e C terminale si associa in maniera antiparallela rispetto ad un alttro dimero. I filamenti intermedi non hanno polarit\`a. La funzione dei filamenti 
intermedi \`e importante e mutazioni a carico dei geni delle cheratine possono dare origine a patologia come l'epidermolisi bollosa semplice in cui mutazione a carico della cheratina
determina la formazione di vesciche a causa della scarsa resistenza a stress meccanici con ulcerazioni a carico di epidermide e mucose. Come nel caso dei filamenti di actina e 
microtubuli ci sono delle proteine associate ai filamenti intermedi che svolgono un ruolo di unire tra di loro i filametni intermedi e metterli a contatto con altri componenti come i 
membri della famiglia delle placchine che si tratta di proteine che interagiscono e legano i filamenti intermedi con i microtubuli. Membri di questa famiglia possono anche permettere
l'interazione tra elementi citoscheletrici e componenti nucleari attraverrso l'interazione con proteine nella memebrana esterna e interna nel nucleo le porteine KASH e SUN transmembrana 
in cui KASH verso il citoplasma e SUN verso il lume del nucleo, queste proteine permettono di comunicare tra l'ambiente nucleare e l'ambiente esterno, comunicazione mediata dai filamenti
intermedi della lamina nucleare e dalle placchine che interagiscono con i componenti citoscheletrici del mitoplasma. Anche il nucleo pu\`o essere deformato e informazioni nel nucleo 
vengono tradotte e inviate alle componenti del citoplasma. La lamina nucleare ha funzione meccanica ma permette l'aggancio e inserimento tra DNA e altre componenti con la membrana 
nucleare (ancoraggio). Nelle cellule nervose si trovano i neurofilamenti N, L e H e sono importanti in quanto forniscono un substrato, un sostegno alle strutture citoscheletriche 
all'assone e permette il mantenimento struttruale e le malattie di lugherig o SLA (sclerosi laterale neutrofica), associata con accumulo e assemblaggio anomalo di neurofilamenti e 
l'accumulo di queste proteine ha ripercussioni sul trasporto di sostanze e organelli che porta alla degenerazione del neurone e non si attua pi\`u la trasmissione e si perdono le 
capacit\`a motorie. 
\section{Movimento cellulare}
Il movimento cellulare avviene in tre passi: protrusione di una parte della membrana plasmatica che viene estroflessa, attacco e interazione tra membrna plasmatica e citoscheletro con
substrato e la trazione in cui la parte citoplasmatica che si trova dietro viene portata in avanti. Questo meccanismo comporta riarrangiamenti del citoscheletro, in particolare la
protrusione pu\`o avvenire grazie alla foramzione di filpodi, lamellipodi o invadopodi, i filopodi \`e costituito da strutture simili a dita proddotti dalla polimerizzazione di acitna
che forma fasci molto stretti paralleli. I lamellipodi hanno alla base uno scheletro di actina che si dispone a rete a maglie e gli invadopodi o podosomi che costituiscono un terzo tipo
di protrusione ricchi di actina. Gli invadopodi o podosomi avvengono nelle tre direzioni e sta alla base in cellule cancerogene che invadono altri tessuti con alta mobilit\`a e invadere
tessuti legate alla produzuone di metallo proteasi e altre sostanze che degradano la matrice extracelluare permettendo alla cellula di muoversi. Le proteine appartenenti alla famiglia 
Rho, Rac e Cdc42 sono GTPasi monomeriche proteine che legano GTP e passano da attive in inattive se legano GTP o GDP. Sono state viste nella trasduzione del segnale, la funzione di 
questi fattore \`e di agire al livello del citoscheletro, attivazione di Cdc42 comporta la formazione di tanti filopodi, in quanto viene attivata la formazinoe di filaemtni di actina che
di fatto va ad attivare la proteina WASP (Wiskott-Aldrich sindrome legata a sue mutazioni e va ad agire a carico della popolaizone del sisitema immunitario con elevata mobilit\`a e 
cellule hanno scarsa mobilit\`a e in cellule che cercano agenti estranei causa problemi) che promuove la nucleazione dell'actina promuovendo la formazione di filamenti di actina. 
L'attivazione di Rac, una GTPasi monomerica porta alla formazione di strutture lamellipoidiali. Rho porta alla formazione di filamenti di actina che permettono l'ancoraggio molto forte
tra cellula e substrato (fibre di stress) importanti per una forte interazione e ancoraggio con il substrato. Importante in quanto l'attivazione di filopodi o lamellipodi pu\`o creare
un movimento in una direzione. Specialmente negli anticorpi il riconoscimento di batteri l'arrivo del segnale si ha attivazione di molecole che permettono la formazione di lamellipodi
o filopodi in una parte e disattivazione nell'altra. Le sostanze chemiorepellenti invece tendono a depolimerizzare l'actina e promuovere la formazione di filopodi in un'altra zona della
cellula permettendole di allontanarsi. Le cellule molto mobili sono molto ricche in filopodi e lamellipodi, strutture molto sensibili che promuovono la crescita e il movimento in un'altra
direzione quando trovano sostanze chemiorepellenti, regolate da membri di famiglia Rac, Rho e Cdc42. Dopo l'attivazione di Rho vengono attivate altre vie dei filamenti di actina su 
cui miosina va ad agire portando alla contrazione della cellula verso i filopodi.
