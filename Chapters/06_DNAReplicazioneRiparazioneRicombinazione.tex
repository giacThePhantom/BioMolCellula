\chapter{Replicazione, riparazione e ricombinazione del DNA}
L'abilit\`a della cellula di mantenere un alto grado di ordine dipende dall'accurata duplicazione di grandi quantit\`a di informazioni genetiche trasportate in forma chimica come DNA.
Il processo della replicazione del DNA deve avvenire prima che una cellula possa produrre due cellule figlie geneticamente indipendenti. Il mantenimento dell'ordine richiede anche una
continua sorveglianza e riparaizone dell'informazione genetica, continaumente danneggiata dall'ambiente chimicamente, attraverso radiazioni, calore e molecole attive generate nella 
cellula. Questi processi vengono svolti da proteine che catalizzano processi rapidi e accurati che avvengono all'interno della cellula. Se la sopravvivenza a breve termine di una cellula
dipende dalla prevenzione dei cambi nella sequenza, quella a lungo termine richiede che una sequenza di DNA cambi lungo le generazioni in modo da permettere adattamento evolutivo a un
ambiente dinamico. Nonostante gli sforzi della cellula avvengono dei cambi nel DNA che possono mettere a disposizioni varianti che la selelzione spinge durante l'evoluzione.
\section{Il mantenimento della sequenza di DNA}
La sopravviveza dell'individuo richiede un alto grado di stabilit\`a genetica. Solo raramente il processo di mantenimento del DNA fallisce causando cambi permenenti nel DNA o mutazioni, 
che possono distruggere un organismo se avvengono se in posizioni vitali della sequenza del DNA.
\subsection{I tassi di mutazione sono estremamenti bassi}
Il tasso di mutaziine pu\`o essere derminato direttamente attraverso esperimenti con batteri come l'Escherichia coli che si divide ogni $30$ minuti e una cellula singola pu\`o generare
una grande popolazione in meno di un giorno. In tale di popolazione \`e possibile individuare una piccola frazione di batteri con una mutazione dannosa in un gene particolare se on
\`e necessario alla sopravvivenza. Tale frazione \`e una sottostima delle mutazioni in quanto ne esistono silenti. Dopo aver corretto per queste mutazioni silenti si trova che un singolo
gene per una proteina di dimensione media ($10^3$ paia di nucleotidi) accumla una mutazione una volta ogni $10^6$ generazioni: il tasso di mutazione \`e pertanto di tre cambi 
nucleotidici per $10^{10}$ nucleotidi per generazione di cellule. Recentemente \`e stato possbile misurare il tasso di mutazione direttamente in organismi pi\`u complessi e la 
riproduzione sessuale. In questo caso la sequenza genomica di una famiglia si sequenzia e una comparazione determina circa $70$ mutazionidi singoli nucleotidi in ogni discendente. 
Normalizzando alla dimensione del genoma umano il tasso di mutazione di un nucleotide cambia per $10^8$ nucleotidi per generazione. Questa \`e una sottostima in quanto non considera le 
mutazioni letali che non sarebbero presenti nella progenie. Circa $100$ divisioni accadono tra il concepimento e il tempo di produzione di uova e sperma che producono una nuova 
generazione, pertanto il tasso umano di mutazione \`e di $1$ mutazione per $10^{10}$ divisioni cellulari. In entrambi gli esperimenti si nota come i tassi di mutazioni sono estremamente
bassi e con un fattore di tre l'uno dall'altro: sono infatti preservati i meccanismi base che garantiscono questi tassi bassi e sono stati conservati da cellule ancestrali molto antiche.
\subsection{I tassi di mutazione bassi sono necessari per la vita come la conosciamo}

