\chapter{Compartimenti intracellulari e ordinamento delle proteine}
Una cellula eucariotica \`e difica in compartimenti racchiusi da membrana e funzionalmente distinti. Ogni compartimento o organello contiene il proprio insieme di enzimi e altre molecole
caratteristico e sistemi di trasporto specializzti distribuiscono prodotti specifici tra di essi. Le proteine danno a ogni compartimeto le proprie propriet\`a funzionali e strutturali.
Catalizzano le reazioni che avvengono in esso e trasportano selettivamente piccole molecole dentro e fuori. Per organelli racchiusi da membrana servono anche da marcatori di superficie
che direzionano la consegna di nuove proteine e lipidi. 
\section{La compartimentalizzazione della cellula}
\subsection{Tutte le cellule hanno lo stesso insieme base di organelli racchiusi da membrana}
Molti processi biochimici vitali avvengono nelle memrane o sulle loro superfici. Le membrane intracellulari, oltre a fornire un aumento di superficie di membrana per svolgerle forma
un sistema di comrpartimenti chiusi separati dal citosol, creando spazi acquosi funzionalmente specializzati dove sottoinsiemi di molecole sono concentrati per ottimizzare le reazioni
biochimiche in cui partecipano. Essendo il bistrato impermeabile alla maggior parte delle molecole idrofiliche si trovano proteine di trasporto per importare ed esportare metaboliti
specifici. Ogni membrana dell'organello ha anche un meccanismo per importare e incorporare le proteine che lo rendono unico. Il nucleo della molecola contiene il genoma ed \`e il sito
principale della sintesi di DNA e RNA. Il citoplasma che lo circonda consiste del citosol e organelli citoplasmatici sospesi in esso. Il citosol \`e il luogo principale della sintesi
e degradazione delle proteine oltre a svolgere la maggior parte del metabolismo intermediario (sintesi e degradazione di piccole molecole per fornire componenti delle macromoleocle). 
Circa met\`a dell'area di membrana racchiude lo spazio del reticolo endoplasmatico. L'ER ruvido possiede molti ribosomi legati alla superficie citosilica, organelli non racchiusi di 
membrana che sintetizzano le proteine di membrana desinate per la secrezione o per altri organelli. Ogni porteine viene trasportata nell'ER mentre viene sintetizzata. L'ER produce anche
la maggir parte dei lipidi e conserva ioni \ce{Ca^{2+}}. Le regioni dell'ER senza ribosomi sono detter ER liscio. Molte delle proteine e dei lipidi vengono mandati dall'ER nell'apparato
di Golgi, consistente di uno stack di strutture a dico detti cisterne di Golgi. L'apparato riceve lipidi e proteine e li invia verso varie destinazioni solitamente modificandole 
covalentemente nel frattempo. I mitocondri e i cloroplasto generano la maggior parte dell'ATP che la cellula usa per guidare le reazioni che richiedono un input di energia libera. I 
cloroplasti sono versioni specializzate di plastidi e possono anche avere altre funzioni come conservazione di nutrienti o pigmenti. I lisosomi contengono enzimi digestivi che degradano
organelli defunti, macromolecole e particelle incorporate attraverso endocitosi. Quando si muove verso i lisosomi il materiale endocitato deve prima passare attraverso gli endosomi. 
I perossisomi sono piccoli compartimenti vescicolari che contengono enzimi utilizzati in reazioni ossidative. Ogni organello svolge lo stesso insieme di funzioni base in tutti i tipi
cellulari, ma variano in abbondanza e possono avere funzioni aggiuntive specifiche al tipo. La presenza e la forma degli organelli sono regolate in modo da soddisfare le necessit\`a 
della cellula. Si trovano spesso in posizioni caratteristiche. La dimensione, forma, composizione e locazione sono importanti e regolate in modo da contribuire alla funzione 
dell'organello stesso.
\subsection{Origini evolutive spiegano le reazioni topologiche degli organelli}
Per capire le relazioni tra i compartimenti cellulari aiuta considerare come potrebbero essersi evoluti. I precursori degli eucarioti erano cellule senza membrane intraellulari. La
membrana plasmatica forniva tutte le funzioni dipendenti da essa. La profusione di membrane interne pu\`o essere considerata come un adattamento per l'aumento in dimensione della
cellula in modo da fornire un sufficiente rapporto volume/superficie. L'evoluzione di membrane interne \`e accoppiata con la specializzazione della loro funzione. Uno schema ipotetico
per l'evoluzione del nucleo e dell'ER viene dall'invaginazione e separazione della membrana plasmatica di una cellula ancestrale che crea un organello con un interno o lume 
topologicamente equivalente all'essterno della cellula. Questo viene mantenuto per gli organelli coinvolti nei cammini secretori e endociti come ER, apparato di Golgi, endosomi, 
lisosomi e perissosomi, assimilabili a membri dello stesso compartimento topologicamente equivalente. I loro lumi comunicano estensivamente tra di loro e con l'esterno attraverso
vescicoli di trasporto. i mitocondri e i plastidi potrebbero essersi evoluti da batteri incorporati da altre cellule con cui avrebbero vissuto in simbiosi. Si possono pertanto 
raggruppare i compartimenti intracellulari in quattro famiglie:
\begin{itemize}
	\item Il nucleo e il sitosol che comunicano tra di loro attraverso complessi di pori nucleari e sono continui topologicamente.
	\item Gli organelli che funzionano nei cammini secretori e endociti come ER, apparato di Golgi, endosomi, lisosomi e altri intermediari di trasporto.
	\item Mitocondri.
	\item Plastidi (solo nelle piante).
\end{itemize}
\subsection{Le proteine si possono muovere tra i compartimenti in modi diversi}
La sintesi delle proteine inizia sui ribosomi nel citosol con l'eccezione di quelle sintetizzate sui ribosomi di mitocondri e platidi. Il loro destino seguente dipende alla sequenza
di amminacidi che contiene segnali di ordinamento che direzionano la loro consegna alle locazioni al di fuori del citosol o a superfici di organelli. Alcune non la possiedono e 
rimangono nel citosol. Tali segnali possono anche direzionarli dall'ER ad altre destinazioni nella cellula. Ci sono tre modi in cui le proteine si possono muovere da un compartimento 
all'altro.
\begin{itemize}
	\item Nel trasporto gated proteine e RNA si muovono tra il citosol e il nucleo attraverso complessi di pori nucleari nella membrana nucleare. Funzionano come 
		gate selettivi che supportano il trasporto attivo di macromolecole specifiche e assemblaggi macromolecolari tra i due spazi topologicamente equivalenti, permettendo la
		diffusione di molecole pi\`u piccole.
	\item Nella traslocazione di proteine proteine transmembrana traslocatrici direzionano il trasporto di proteine specifiche attraverso una membrana dal citosol in uno spazio 
		topologicamente distinto. Le proteine devono solitamente spiegarsi per attraversare il traslocvatore. Avviene tra il citosol nel lume dell'ER o nei mitocondri. Lo 
		stesso meccanismo viene utilizzato parzialmente per le proteine integrali di membrana.
	\item Nel trasporto vescicolare intermediari di trasporto racchiusi da membrana, vescicole sferiche poccole o grandi frammenti di organelli irregolari traportano proteine tra
		compartimenti topologicamente equivalenti. Le vescicole si caricano nel lume di un compartimento e si separano dalla sua membrana e si scaricano in un secondo diventando
		parte della sua membrana. Questo avviene per le proteine solubili tra l'ER e l'apparato di Golgi. Non attraversando una membrana pu\`o avvenire unicamente tra 
		compartimenti topologicamente equivalenti.
\end{itemize}
Ogni modo di trasferimento \`e guidato da segnali di ordinamento nella proteina trasportata riconosciuti da recettori di ordinamento. 
\subsection{Sequenze di segnali e recettori di orginamento direzionano le proteine al corretto indirizzo nella cellula}
La maggior parte dei segnali di ordinamento si trova in una lunghezza di amminoacidi lunga tra i $15$ e i $60$. Tali sequenze si trovano spesso verso la terminazione \ce{N} della catena
e in molti casi peptidasi di segnale la rimuovono una volta che il processo \`e completo. Posson oanche essere sequenze interne che rimangono parte della proteina, utilizzati nel 
trasporto gated nel nucleo. Possono anche essere composti da multiple sequenze di amminoacidi interne che formano una conformazione tridimensionale delgi atomi di supeficie. Tali 
aree di segnale sono utilizzate per l'importazione nucleare e nel trasporto vescicolare. Ogni sequenza specifica una destinazione particolare. Proteine destinate ad un trasferimento 
iniziale verso l'ER hanno una sequenza alla terminazione \ce{N} che include una sequenza di $5$-$10$ amminoacidi idrofobici. Molte di queste proteine sono continamente passate tra gli
apparati di Golgi e l'ER e alcune con una sequenza specifica alla terminazione \ce{C} sono riconosciute come residenti nell'ER e ritornate ad esso. Le proteine destinate per i mitocondri
hanno una sequenza di segnale in cui amminoacidi carichi positivamente si alternano con altri idrofobici. Molte proteine destinate al perissosoma hanno una sequenza di tre amminoacidi
alla terminazione \ce{C}. Tali sequenze sono necessarie e sufficienti per l'indirizzamento della proteina, pur essendo variabili le sequenze di proteine con la stessa destinazione sono
funzionalmente intercambiabili. Tali sequenze possono essere riconosicute da recettori complementari che le guidano alla destinazione dove le scaricano. I recettori funzionano 
cataliticamente: dopo aver terminato un trasporto ritornano nel punto iniziale. La maggior parte riconoscono classi di proteine. 
\subsection{La maggior parte degli organelli non pu\`o essere costruita da nuovo e richiedono informazioni nell'organello stesso}
Quando una cellula si riproduce per divisione deve duplicare i propri organelli oltre ai suoi cromosomi. QUesto avviene incorporando nuove molecole negli organelli esistenti 
ingrandendoli e causando la loro divisione e distribuzione nelle cellule figlie. Ogni cellula figlia eredita un insieme completo di membrane cellulari ereditate dal genitore. Questo \`e
essenziale in quanto una cellula non pu\`o sintetizzarle da zero. Le informazioni richieste per costruire un organello non si trovano esclusivamente nel DNA che ne specifica le proteine.
\`E necessaria informazione nella forma di proteine nella membrana preesistente. Inoltre alcuni organelli ne possono formare altri che non devono essere ereditati alla difisione 
cellulare. 
\section{Il trasporto di molecole tra il nucleo e il citosol}
La membrana nucleare racchiude il DNA e definisce il compartimento nucleare e consiste di due membrane concentriche penetrate da complessi di pori nucleari. Nonostante le membrane
esterne e interne siano continue mantengono distinte composizioni proteineche. La membrana nucleare interna contiene proteine che sono siti di legame per cromosomi e per la lamina
nucleare, una rete di proteine che forniscono supporto strutturale oltre ad agire come ancora per cromosomi e il citoscheletro citoplasmatico. La membrana esterna circonda quella interna
ed \`e continua con la membrana dell'ER. \`E costellata di ribosomi che sintetizzano proteine traportate nello spaizo tra le due membrane (spazio perinucleare), continuo con il lume ER.
Avviene continuamente un traffico bidirezionale tra nucleo e citosol: molte proteine con funzione nel primo sono importate dal secondo e tutti gli RNA sintetizzati sono esportati in
direzione opposta. Entrambi i processi sono selettivi. 
\subsection{Complessi di pori nucleari perforano la membrana nuclare}
