\chapter{Compartimenti intracellulari e ordinamento delle proteine}
Una cellula eucariotica \`e difica in compartimenti racchiusi da membrana e funzionalmente distinti. Ogni compartimento o organello contiene il proprio insieme di enzimi e altre molecole
caratteristico e sistemi di trasporto specializzti distribuiscono prodotti specifici tra di essi. Le proteine danno a ogni compartimeto le proprie propriet\`a funzionali e strutturali.
Catalizzano le reazioni che avvengono in esso e trasportano selettivamente piccole molecole dentro e fuori. Per organelli racchiusi da membrana servono anche da marcatori di superficie
che direzionano la consegna di nuove proteine e lipidi. 
\section{La compartimentalizzazione della cellula}
\subsection{Tutte le cellule hanno lo stesso insieme base di organelli racchiusi da membrana}
Molti processi biochimici vitali avvengono nelle memrane o sulle loro superfici. Le membrane intracellulari, oltre a fornire un aumento di superficie di membrana per svolgerle forma
un sistema di comrpartimenti chiusi separati dal citosol, creando spazi acquosi funzionalmente specializzati dove sottoinsiemi di molecole sono concentrati per ottimizzare le reazioni
biochimiche in cui partecipano. Essendo il bistrato impermeabile alla maggior parte delle molecole idrofiliche si trovano proteine di trasporto per importare ed esportare metaboliti
specifici. Ogni membrana dell'organello ha anche un meccanismo per importare e incorporare le proteine che lo rendono unico. Il nucleo della molecola contiene il genoma ed \`e il sito
principale della sintesi di DNA e RNA. Il citoplasma che lo circonda consiste del citosol e organelli citoplasmatici sospesi in esso. Il citosol \`e il luogo principale della sintesi
e degradazione delle proteine oltre a svolgere la maggior parte del metabolismo intermediario (sintesi e degradazione di piccole molecole per fornire componenti delle macromoleocle). 
Circa met\`a dell'area di membrana racchiude lo spazio del reticolo endoplasmatico. L'ER ruvido possiede molti ribosomi legati alla superficie citosilica, organelli non racchiusi di 
membrana che sintetizzano le proteine di membrana desinate per la secrezione o per altri organelli. Ogni porteine viene trasportata nell'ER mentre viene sintetizzata. L'ER produce anche
la maggir parte dei lipidi e conserva ioni \ce{Ca^{2+}}. Le regioni dell'ER senza ribosomi sono detter ER liscio. Molte delle proteine e dei lipidi vengono mandati dall'ER nell'apparato
di Golgi, consistente di uno stack di strutture a dico detti cisterne di Golgi. L'apparato riceve lipidi e proteine e li invia verso varie destinazioni solitamente modificandole 
covalentemente nel frattempo. I mitocondri e i cloroplasto generano la maggior parte dell'ATP che la cellula usa per guidare le reazioni che richiedono un input di energia libera. I 
cloroplasti sono versioni specializzate di plastidi e possono anche avere altre funzioni come conservazione di nutrienti o pigmenti. I lisosomi contengono enzimi digestivi che degradano
organelli defunti, macromolecole e particelle incorporate attraverso endocitosi. Quando si muove verso i lisosomi il materiale endocitato deve prima passare attraverso gli endosomi. 
I perossisomi sono piccoli compartimenti vescicolari che contengono enzimi utilizzati in reazioni ossidative. Ogni organello svolge lo stesso insieme di funzioni base in tutti i tipi
cellulari, ma variano in abbondanza e possono avere funzioni aggiuntive specifiche al tipo. La presenza e la forma degli organelli sono regolate in modo da soddisfare le necessit\`a 
della cellula. Si trovano spesso in posizioni caratteristiche. La dimensione, forma, composizione e locazione sono importanti e regolate in modo da contribuire alla funzione 
dell'organello stesso.
\subsection{Origini evolutive spiegano le reazioni topologiche degli organelli}
Per capire le relazioni tra i compartimenti cellulari aiuta considerare come potrebbero essersi evoluti. I precursori degli eucarioti erano cellule senza membrane intraellulari. La
membrana plasmatica forniva tutte le funzioni dipendenti da essa. La profusione di membrane interne pu\`o essere considerata come un adattamento per l'aumento in dimensione della
cellula in modo da fornire un sufficiente rapporto volume/superficie. L'evoluzione di membrane interne \`e accoppiata con la specializzazione della loro funzione. Uno schema ipotetico
per l'evoluzione del nucleo e dell'ER viene dall'invaginazione e separazione della membrana plasmatica di una cellula ancestrale che crea un organello con un interno o lume 
topologicamente equivalente all'essterno della cellula. Questo viene mantenuto per gli organelli coinvolti nei cammini secretori e endociti come ER, apparato di Golgi, endosomi, 
lisosomi e perissosomi, assimilabili a membri dello stesso compartimento topologicamente equivalente. I loro lumi comunicano estensivamente tra di loro e con l'esterno attraverso
vescicoli di trasporto. i mitocondri e i plastidi potrebbero essersi evoluti da batteri incorporati da altre cellule con cui avrebbero vissuto in simbiosi. Si possono pertanto 
raggruppare i compartimenti intracellulari in quattro famiglie:
\begin{itemize}
	\item Il nucleo e il sitosol che comunicano tra di loro attraverso complessi di pori nucleari e sono continui topologicamente.
	\item Gli organelli che funzionano nei cammini secretori e endociti come ER, apparato di Golgi, endosomi, lisosomi e altri intermediari di trasporto.
	\item Mitocondri.
	\item Plastidi (solo nelle piante).
\end{itemize}
\subsection{Le proteine si possono muovere tra i compartimenti in modi diversi}
La sintesi delle proteine inizia sui ribosomi nel citosol con l'eccezione di quelle sintetizzate sui ribosomi di mitocondri e platidi. Il loro destino seguente dipende alla sequenza
di amminacidi che contiene segnali di ordinamento che direzionano la loro consegna alle locazioni al di fuori del citosol o a superfici di organelli. Alcune non la possiedono e 
rimangono nel citosol. Tali segnali possono anche direzionarli dall'ER ad altre destinazioni nella cellula. Ci sono tre modi in cui le proteine si possono muovere da un compartimento 
all'altro.
\begin{itemize}
	\item Nel trasporto gated proteine e RNA si muovono tra il citosol e il nucleo attraverso complessi di pori nucleari nella membrana nucleare. Funzionano come 
		gate selettivi che supportano il trasporto attivo di macromolecole specifiche e assemblaggi macromolecolari tra i due spazi topologicamente equivalenti, permettendo la
		diffusione di molecole pi\`u piccole.
	\item Nella traslocazione di proteine proteine transmembrana traslocatrici direzionano il trasporto di proteine specifiche attraverso una membrana dal citosol in uno spazio 
		topologicamente distinto. Le proteine devono solitamente spiegarsi per attraversare il traslocvatore. Avviene tra il citosol nel lume dell'ER o nei mitocondri. Lo 
		stesso meccanismo viene utilizzato parzialmente per le proteine integrali di membrana.
	\item Nel trasporto vescicolare intermediari di trasporto racchiusi da membrana, vescicole sferiche poccole o grandi frammenti di organelli irregolari traportano proteine tra
		compartimenti topologicamente equivalenti. Le vescicole si caricano nel lume di un compartimento e si separano dalla sua membrana e si scaricano in un secondo diventando
		parte della sua membrana. Questo avviene per le proteine solubili tra l'ER e l'apparato di Golgi. Non attraversando una membrana pu\`o avvenire unicamente tra 
		compartimenti topologicamente equivalenti.
\end{itemize}
Ogni modo di trasferimento \`e guidato da segnali di ordinamento nella proteina trasportata riconosciuti da recettori di ordinamento. 
\subsection{Sequenze di segnali e recettori di orginamento direzionano le proteine al corretto indirizzo nella cellula}
La maggior parte dei segnali di ordinamento si trova in una lunghezza di amminoacidi lunga tra i $15$ e i $60$. Tali sequenze si trovano spesso verso la terminazione \ce{N} della catena
e in molti casi peptidasi di segnale la rimuovono una volta che il processo \`e completo. Posson oanche essere sequenze interne che rimangono parte della proteina, utilizzati nel 
trasporto gated nel nucleo. Possono anche essere composti da multiple sequenze di amminoacidi interne che formano una conformazione tridimensionale delgi atomi di supeficie. Tali 
aree di segnale sono utilizzate per l'importazione nucleare e nel trasporto vescicolare. Ogni sequenza specifica una destinazione particolare. Proteine destinate ad un trasferimento 
iniziale verso l'ER hanno una sequenza alla terminazione \ce{N} che include una sequenza di $5$-$10$ amminoacidi idrofobici. Molte di queste proteine sono continamente passate tra gli
apparati di Golgi e l'ER e alcune con una sequenza specifica alla terminazione \ce{C} sono riconosciute come residenti nell'ER e ritornate ad esso. Le proteine destinate per i mitocondri
hanno una sequenza di segnale in cui amminoacidi carichi positivamente si alternano con altri idrofobici. Molte proteine destinate al perissosoma hanno una sequenza di tre amminoacidi
alla terminazione \ce{C}. Tali sequenze sono necessarie e sufficienti per l'indirizzamento della proteina, pur essendo variabili le sequenze di proteine con la stessa destinazione sono
funzionalmente intercambiabili. Tali sequenze possono essere riconosicute da recettori complementari che le guidano alla destinazione dove le scaricano. I recettori funzionano 
cataliticamente: dopo aver terminato un trasporto ritornano nel punto iniziale. La maggior parte riconoscono classi di proteine. 
\subsection{La maggior parte degli organelli non pu\`o essere costruita da nuovo e richiedono informazioni nell'organello stesso}
Quando una cellula si riproduce per divisione deve duplicare i propri organelli oltre ai suoi cromosomi. QUesto avviene incorporando nuove molecole negli organelli esistenti 
ingrandendoli e causando la loro divisione e distribuzione nelle cellule figlie. Ogni cellula figlia eredita un insieme completo di membrane cellulari ereditate dal genitore. Questo \`e
essenziale in quanto una cellula non pu\`o sintetizzarle da zero. Le informazioni richieste per costruire un organello non si trovano esclusivamente nel DNA che ne specifica le proteine.
\`E necessaria informazione nella forma di proteine nella membrana preesistente. Inoltre alcuni organelli ne possono formare altri che non devono essere ereditati alla difisione 
cellulare. 
\section{Il trasporto di molecole tra il nucleo e il citosol}
La membrana nucleare racchiude il DNA e definisce il compartimento nucleare e consiste di due membrane concentriche penetrate da complessi di pori nucleari. Nonostante le membrane
esterne e interne siano continue mantengono distinte composizioni proteineche. La membrana nucleare interna contiene proteine che sono siti di legame per cromosomi e per la lamina
nucleare, una rete di proteine che forniscono supporto strutturale oltre ad agire come ancora per cromosomi e il citoscheletro citoplasmatico. La membrana esterna circonda quella interna
ed \`e continua con la membrana dell'ER. \`E costellata di ribosomi che sintetizzano proteine traportate nello spaizo tra le due membrane (spazio perinucleare), continuo con il lume ER.
Avviene continuamente un traffico bidirezionale tra nucleo e citosol: molte proteine con funzione nel primo sono importate dal secondo e tutti gli RNA sintetizzati sono esportati in
direzione opposta. Entrambi i processi sono selettivi. 
\subsection{Complessi di pori nucleari perforano la membrana nuclare}
Grandi ed elaborati complessi di pori nucleari (NPC) perforano la membrana nucleare in tutti gli eucarioti. Ogni NPC \`e composto da un insieme di $30$ nucleoporine. Ogni nucleoporina
\`e presente in copie multiple. La maggior parte delle nucleoproteine sono composte da domini proteici ripetitivi. Alcune delle nucleoporine di impalcatura sono imparentate ai complessi
di proteina dell'involucro delle vescicole che gli danno forma. La membrana contiene tra le $3000$ e le $4000$ NPC che possono trasportare ognuna fino a $1000$ macromolecole per secondo
in entrambe le direzioni contemporaneamente. Ogni NPC contiene passaggi acquosi che permettono il passaggio di piccole molecole solubili in acqua a un tasso pari alla diffusione libera.
Le moleocle pi\`u grandi devono invece essere attivamente trasportate. Il canale della nucleoporina ha regioni non strutturate che formano un ingarbuglio disordinato che restringe la
diffusione delle molecole pi\`u grandi. Grazie a questo il compartimento nucleare e il citosol possono mantenere diverse composizioni proteiche. L'esportazione o importazione di 
grandi molecole avviene attraverso recettori specifici. 
\subsection{Segnali di localizzazione direzionano le proteine nucleari al nucleo}
Segnali di ordinamento detti segnali di localizzazione nucleare (NLS) sono responsabili per la selettivit\`a del processo di importazione nucelare. In molte proteine consiste di fino a
due corte sequenze richhe di amminoacidi carichi positivamente come lisina e arginina. Altre proteine nucleari contengono diversi sengali. Tali segnali si possono trovare ovunque nella
catena e formano anelli o sucperfici sulla superficie proteica. QUando deve avvenire l'importazione le particelle si legano a fibrille che si estendono dalle nucleoporine di impalcatura
al bordo dell'NPC nel citosole poi procedono attraverso il suo centro. Le regioni non strutturate formano una barriera di diffusione e sono spostate per permettere il passaggio delle
proteine. Il trasporto nucleare differisce dagli altri in quanto permette il passaggio di proteine senza modificare la loro struttura.
\subsection{Recettori di importazione nucleare si legano sia ai segnali di localizzazione nucleare e alle proteine NPC}
Per iniziare l'importo lineare la maggior parte dei segnali di localizzazione nucleare devono essere riconosciuti da recettori di import nucleari o importine. Ogni membro della famiglia
dei geni codifica una proteina recettrice che pu\`o legare e trasportare il sottoinsieme contenente il segnale appropriato. Non legano sempre le proteine direttamente ma proteine 
adattatrici possono creare un ponte tra i recettori e i segnali di localizzazione nucleare. Alcune proteine adattatrici sono strutturalmente imparentate ai recettori. Utilizzando una
grande variet\`a di adattatori e recettori le cellule sono capaci di riconoscere tutti i vari segnali mostrati sulle proteine nucleari. I recettori sono proteine solubili al citosol che
si legano al segnale di localizzazione sul carbo e alle ripetizioni fenilanila-glicina (FG) nei domini non strutturati delle nucleoporine canale che confinano con il poro centrale. Tali
ripetizioni si trovano anche nelle fibrille citoplasmatiche e nucleari. Nel ingarbuglio non strutturato al poro tali ripetizioni interagiscono debolmente imponendo una barriera 
impermeabile alle grandi macromolecole e servono come siti di attracco per i recettori di import. Le ripetizioni allineano il cammino attraverso gli NPC preso dalle importine e le 
proteine legate. Secondo un modello di trasporto nucleare il complesso importina-cargo si muove lungo il cammino di trasporto legando, disassociandosi e riassociandosi a sequenze di
ripetizioni FG. Tali legami dissolvono localmente la fase di gel dell'ingarbuglio della nucleoporina  permettendo il passaggio del complesso che una volta dentro il nucleo si dissocia
(cosa che non avviene nel citosol, garantendo direzionalit\`a) una volta dissociata l'importina ritorna al citosol.
\subsection{L'esportazione nucelare lavora come l'importazione ma in senso opposto}
L'esportazione nucleare di grandi molecole avviene attraverso NPC e dipende da un sistema di trasporto selettivo che dipende da segnali di esportazione nucleaare sulle macromolecole da
esportare oltre a recettori di esportazione nucleare o esportine che si legano ai segnali di esportazione e a proteine NPC per guidare il loro cargo attraverso l'NPC al citosol.
Sono imparentati strutturalmente alle importine e codificati dalla stessa famiglia di recettori di trasporto nucleare o karyopherins. Si nota pertanto come che i sitemi di importazione
ed esportazione lavano secondo gli stessi meccanismi anche se in direzioni diverse.
\subsection{La Ran GTPasi impone direzionalit\`a al trasporto attraverso NPC}
L'importazione di proteine nucleari attravero NPC concentra specifiche proteine nel nucleo aumentando l'ordine della cellula. Deve pertanto essere utilizzata l'energia conservata in
gradienti di concentrazione o nella forma legata al GTP monomerica Ran, richiesta sia per l'esportazione che l'importazione. \`E uno switch molecolare che esiste in due conformazioni
in base al fatto che sia legato GDP o GTP. Due proteine regolatorie specifiche al Ran causano la conversione tra le due: una GTPasi attivatrice (GAP) citosilica che causa l'idrolisi 
del GTP e un fattore di scambio di guanina (GEF) nucleare che promuove lo scambio di GDP per GTP e converte Ran-GDP in Ran-GTP. La localizzazione di queste due proteine crea un gradiente
che guida il trasporto nucleare nella direziona appropiata. Le importine si legano alle ripetizioni FG anche se non legate al cargo ed entrano il canale. Quando raggiungono il lato 
nucleare Ran-GTP si lega ad esse e causa il rilascio del cargo che avviene solo sul lato nucleare. L'importina, ora legata alla Ran-GTP \`e trasportato nel citosol dove Ran-GAP causa
l'idrolizzazione del GTP convertendola a Ran-GDP che si dissocia dal recettore che \`e di uovo utilizzabile. L'esporto avviene con un meccanismo simile tranne per il fatto che Ran-GTP
nel nucleo promuove il legame del cargo con l'esportina. Una votla che si muove nella nucleoporina viene idrolizzata e l'esportina la rilascia insieme al cargo. Esportine libere sono
successivamente riportate nel nucleo.
\subsection{Il trasporto attraverso NPC pu\`o essere regolato controllando l'accesso ai macchinari di trasporto}
Alcune proteine contengono sia segnali per l'importazione che per l'esportazione e si spostano continuamente fuori e dentro il nucleo. Il tasso di importazione ed esportazione determina
lo stato di localizzazione delle proteine shuttling. Cambiandolo si cambia la locazione di una proteina. Alcune proteine si muovono continuamente ma in altri casi il trasporto \`e
altamente controllato. Questo avviene regolando i sengali di localizzazione nucleare ed esportazione accendendoli o spegnendoli attraverso fosforilazione di amminoacidi ficini alle
sequenze di segnale. Altri regolatori di trascrizione sono legati a proteine citosiliche che le ancorano nel citosol o mascano i segnali in modo che non possano interagire con i 
recettori di importazione. Uno stimolo appropriato rilascia la proteina dall'ancora citosilica o dalla maschera e permette il trasporto nel nucleo. Un esempio \`e una proteina che 
controlla l'espressione di proteine coinvolte nel metabolismo del colesterolo. La proteina viene conservata in una forma inattiva come proteina transmembrana nell'ER. Quando una 
cellula viene deprivata di colesterolo la proteina viene trasportata all'apparato di Golgi dove incontra proteasi che separano il dominio citosolico che viene importato nel nucleo dove
attiva la trascrizione dei geni per il recupero e sintesi di colesterolo. L'esportazione di mRNA avviene come grandi assemblaggi che possono cntenere centinaia di proteine. I 
complesssi ribonucleoproteina-mRNA (mRNP) attraccano al lato nucleare degli NPC dove sono rimodellati ed esportati attraverso idrolisi dell'ATP. 
\subsection{Durante la mitosi la membrana nucleare si disassembla}
La lamina nucleare locata sul lato nucleare della membrana interna \`e una rete di subunit\`a proteiche connesse dette lamine nucleari. Proteine filamentose che si polimerizzano in
un lattice bidimensionale che d\`a forma e stabilit\`a alla membrana nucleare a cui \`e ancorata attraverso gli NPC e proteine transmembrana. La lamina interagisce con la cromatina e
insieme a proteine della membrana interna crea collegamenti strutturali tra DNA e membrana nucleare. QUando il nucleo si disassembla durante la mitosi gli NPC e la lamina nucleare si 
disassemblano e la membrana si frammenta. Durante questo processo NPC si legano a recettori di import che svolgono un ruiolo nel riassemblaggio alla fine della mitosi. Le proteine della
membrana nucleare si disperdono attraverso la membrana ER. Pi\`u tardi durante la mitosi la membrana nucleare si riassembla sulla superficie dei cromosomi figli. La Ran GTPasi 
agisce come marcatore posizionale per la cromatina e rilascia le proteien NPC in prossimit\`a dei cromosomi dalle importine. 
\section{Il trasporto di proteine in mitocondri e cloroplasti}
I mitocondri e cloroplasti sono organelli racchiusi da doppia membrana che si specializzano nella sintesi dell'ATP utilizzando energia derivante da trasporto di elettroni e
fosforilazione ossidativa nei mitocondri e dalla fotosintesi nei cloroplasti. Entrambi gli organelli contengono il proprio DNA ribosomi e altri componenti per la sintesi di proteine
la maggior parte di esse sono sintetizzate nel nucleo cellulare e importate dal citosol. Ogni proteina deve raggiungere un sottocompartimento particolare in cui funziona. Esistono 
diversi compartimenti nei mitocondri: lo spazio di matrice e spazio intermembrana continuo con lo spazio cristae sono formati da due membrane mitocondrial concentriche. Quella 
interna racchiude lo spazio di matrice e forma invaginazioni dette cristae e la membrana esterna che \`e in contatto con il citosol. Complessi proteici danno confini alle 
giunzioni dove la cristae si incagina e divide la membrana interna in due domini: uno che circonda lo spazio cristae e l'altro dominio lungo la membrana esterna. I cloroplasti posseggono
anch'essi una doppia membrana che racciude uno spazio intermembrana e lo stroma, l'equivalente dello spaizo di matrice. Hanno anche gli spazi thylakoidi circondati dalla membvrana
thylacoide che deriva dalla membrana interna durante lo sviluppo plastide ed \`e pizzicata diventando discontinua. Ogni sottocompartimento contiene le proprie proteine distinte. 
Nuovi miticondri e cloroplasti sono prodotti dalla crescita di organelli preesistenti seguti da fissione. La crescita dipende dall'importazione di proteine dal citosol che devono 
essere trasportate attraverso diverse membrane in successione nel processo di traslocazione proteica. 
\subsection{La traslocazione nei mitocondri dipende da sequenze di segnale e traslocatori di proteine}
