\chapter{Compartimenti intracellulari e ordinamento delle proteine}
Una cellula eucariotica \`e divisa in compartimenti racchiusi da membrana e funzionalmente distinti. Ogni compartimento o organello contiene il proprio insieme di enzimi e altre molecole
caratteristico e sistemi di trasporto specializzati distribuiscono prodotti specifici tra di essi. Le proteine danno a ogni compartimeto le proprie propriet\`a funzionali e strutturali.
Catalizzano le reazioni che avvengono in esso e trasportano selettivamente piccole molecole dentro e fuori. Per organelli racchiusi da membrana servono anche da marcatori di superficie
che direzionano la consegna di nuove proteine e lipidi. 
\section{La compartimentalizzazione della cellula}
\subsection{Tutte le cellule hanno lo stesso insieme base di organelli racchiusi da membrana}
Molti processi biochimici vitali avvengono nelle membrane o sulle loro superfici. Le membrane intracellulari, oltre a fornire un aumento di superficie di membrana per svolgerle forma
un sistema di compartimenti chiusi separati dal citosol, creando spazi acquosi funzionalmente specializzati dove sottoinsiemi di molecole sono concentrati per ottimizzare le reazioni
biochimiche in cui partecipano. Essendo il bistrato impermeabile alla maggior parte delle molecole idrofiliche si trovano proteine di trasporto per importare ed esportare metaboliti
specifici. Ogni membrana dell'organello ha anche un meccanismo per importare e incorporare le proteine che lo rendono unico. Il nucleo della molecola contiene il genoma ed \`e il sito
principale della sintesi di DNA e RNA. Il citoplasma che lo circonda consiste del citosol e organelli citoplasmatici sospesi in esso. Il citosol \`e il luogo principale della sintesi
e degradazione delle proteine oltre a svolgere la maggior parte del metabolismo intermediario (sintesi e degradazione di piccole molecole per fornire componenti delle macromoleocle). 
Circa met\`a dell'area di membrana racchiude lo spazio del reticolo endoplasmatico. L'ER ruvido possiede molti ribosomi legati alla superficie citosilica, organelli non racchiusi di 
membrana che sintetizzano le proteine di membrana desinate per la secrezione o per altri organelli. Ogni proteine viene trasportata nell'ER mentre viene sintetizzata. L'ER produce anche
la maggir parte dei lipidi e conserva ioni \ce{Ca^{2+}}. Le regioni dell'ER senza ribosomi sono detter ER liscio. Molte delle proteine e dei lipidi vengono mandati dall'ER nell'apparato
di Golgi, consistente di uno stack di strutture a dico detti cisterne di Golgi. L'apparato riceve lipidi e proteine e li invia verso varie destinazioni solitamente modificandole 
covalentemente nel frattempo. I mitocondri e i cloroplasto generano la maggior parte dell'ATP che la cellula usa per guidare le reazioni che richiedono un input di energia libera. I 
cloroplasti sono versioni specializzate di plastidi e possono anche avere altre funzioni come conservazione di nutrienti o pigmenti. I lisosomi contengono enzimi digestivi che degradano
organelli defunti, macromolecole e particelle incorporate attraverso endocitosi. Quando si muove verso i lisosomi il materiale endocitato deve prima passare attraverso gli endosomi. 
I perossisomi sono piccoli compartimenti vescicolari che contengono enzimi utilizzati in reazioni ossidative. Ogni organello svolge lo stesso insieme di funzioni base in tutti i tipi
cellulari, ma variano in abbondanza e possono avere funzioni aggiuntive specifiche al tipo. La presenza e la forma degli organelli sono regolate in modo da soddisfare le necessit\`a 
della cellula. Si trovano spesso in posizioni caratteristiche. La dimensione, forma, composizione e locazione sono importanti e regolate in modo da contribuire alla funzione 
dell'organello stesso.
\subsection{Origini evolutive spiegano le reazioni topologiche degli organelli}
Per capire le relazioni tra i compartimenti cellulari aiuta considerare come potrebbero essersi evoluti. I precursori degli eucarioti erano cellule senza membrane intracellulari. La
membrana plasmatica forniva tutte le funzioni dipendenti da essa. La profusione di membrane interne pu\`o essere considerata come un adattamento per l'aumento in dimensione della
cellula in modo da fornire un sufficiente rapporto volume/superficie. L'evoluzione di membrane interne \`e accoppiata con la specializzazione della loro funzione. Uno schema ipotetico
per l'evoluzione del nucleo e dell'ER viene dall'invaginazione e separazione della membrana plasmatica di una cellula ancestrale che crea un organello con un interno o lume 
topologicamente equivalente all'esterno della cellula. Questo viene mantenuto per gli organelli coinvolti nei cammini secretori e endociti come ER, apparato di Golgi, endosomi, 
lisosomi e perissosomi, assimilabili a membri dello stesso compartimento topologicamente equivalente. I loro lumi comunicano estensivamente tra di loro e con l'esterno attraverso
vescicoli di trasporto. i mitocondri e i plastidi potrebbero essersi evoluti da batteri incorporati da altre cellule con cui avrebbero vissuto in simbiosi. Si possono pertanto 
raggruppare i compartimenti intracellulari in quattro famiglie:
\begin{itemize}
	\item Il nucleo e il citosol che comunicano tra di loro attraverso complessi di pori nucleari e sono continui topologicamente.
	\item Gli organelli che funzionano nei cammini secretori e endociti come ER, apparato di Golgi, endosomi, lisosomi e altri intermediari di trasporto.
	\item Mitocondri.
	\item Plastidi (solo nelle piante).
\end{itemize}
\subsection{Le proteine si possono muovere tra i compartimenti in modi diversi}
La sintesi delle proteine inizia sui ribosomi nel citosol con l'eccezione di quelle sintetizzate sui ribosomi di mitocondri e plasmidi. Il loro destino seguente dipende alla sequenza
di amminoacidi che contiene segnali di ordinamento che direzionano la loro consegna alle locazioni al di fuori del citosol o a superfici di organelli. Alcune non la possiedono e 
rimangono nel citosol. Tali segnali possono anche direzionarli dall'ER ad altre destinazioni nella cellula. Ci sono tre modi in cui le proteine si possono muovere da un compartimento 
all'altro.
\begin{itemize}
	\item Nel trasporto gated proteine e RNA si muovono tra il citosol e il nucleo attraverso complessi di pori nucleari nella membrana nucleare. Funzionano come 
		gate selettivi che supportano il trasporto attivo di macromolecole specifiche e assemblaggi macromolecolari tra i due spazi topologicamente equivalenti, permettendo la
		diffusione di molecole pi\`u piccole.
	\item Nella traslocazione di proteine proteine transmembrana traslocatrici direzionano il trasporto di proteine specifiche attraverso una membrana dal citosol in uno spazio 
		topologicamente distinto. Le proteine devono solitamente spiegarsi per attraversare il traslocatore. Avviene tra il citosol nel lume dell'ER o nei mitocondri. Lo 
		stesso meccanismo viene utilizzato parzialmente per le proteine integrali di membrana.
	\item Nel trasporto vescicolare intermediari di trasporto racchiusi da membrana, vescicole sferiche piccole o grandi frammenti di organelli irregolari trasportano proteine tra
		compartimenti topologicamente equivalenti. Le vescicole si caricano nel lume di un compartimento e si separano dalla sua membrana e si scaricano in un secondo diventando
		parte della sua membrana. Questo avviene per le proteine solubili tra l'ER e l'apparato di Golgi. Non attraversando una membrana pu\`o avvenire unicamente tra 
		compartimenti topologicamente equivalenti.
\end{itemize}
Ogni modo di trasferimento \`e guidato da segnali di ordinamento nella proteina trasportata riconosciuti da recettori di ordinamento. 
\subsection{Sequenze di segnali e recettori di ordinamento direzionano le proteine al corretto indirizzo nella cellula}
La maggior parte dei segnali di ordinamento si trova in una lunghezza di amminoacidi lunga tra i $15$ e i $60$. Tali sequenze si trovano spesso verso la terminazione \ce{N} della catena
e in molti casi peptidasi di segnale la rimuovono una volta che il processo \`e completo. Possono anche essere sequenze interne che rimangono parte della proteina, utilizzati nel 
trasporto gated nel nucleo. Possono anche essere composti da multiple sequenze di amminoacidi interne che formano una conformazione tridimensionale degli atomi di superficie. Tali 
aree di segnale sono utilizzate per l'importazione nucleare e nel trasporto vescicolare. Ogni sequenza specifica una destinazione particolare. Proteine destinate ad un trasferimento 
iniziale verso l'ER hanno una sequenza alla terminazione \ce{N} che include una sequenza di $5$-$10$ amminoacidi idrofobici. Molte di queste proteine sono continuamente passate tra gli
apparati di Golgi e l'ER e alcune con una sequenza specifica alla terminazione \ce{C} sono riconosciute come residenti nell'ER e ritornate ad esso. Le proteine destinate per i mitocondri
hanno una sequenza di segnale in cui amminoacidi carichi positivamente si alternano con altri idrofobici. Molte proteine destinate al perissosoma hanno una sequenza di tre amminoacidi
alla terminazione \ce{C}. Tali sequenze sono necessarie e sufficienti per l'indirizzamento della proteina, pur essendo variabili le sequenze di proteine con la stessa destinazione sono
funzionalmente intercambiabili. Tali sequenze possono essere riconosciute da recettori complementari che le guidano alla destinazione dove le scaricano. I recettori funzionano 
cataliticamente: dopo aver terminato un trasporto ritornano nel punto iniziale. La maggior parte riconoscono classi di proteine. 
\subsection{La maggior parte degli organelli non pu\`o essere costruita da nuovo e richiedono informazioni nell'organello stesso}
Quando una cellula si riproduce per divisione deve duplicare i propri organelli oltre ai suoi cromosomi. Questo avviene incorporando nuove molecole negli organelli esistenti 
ingrandendoli e causando la loro divisione e distribuzione nelle cellule figlie. Ogni cellula figlia eredita un insieme completo di membrane cellulari ereditate dal genitore. Questo \`e
essenziale in quanto una cellula non pu\`o sintetizzarle da zero. Le informazioni richieste per costruire un organello non si trovano esclusivamente nel DNA che ne specifica le proteine.
\`E necessaria informazione nella forma di proteine nella membrana preesistente. Inoltre alcuni organelli ne possono formare altri che non devono essere ereditati alla difisione 
cellulare. 
\section{Il trasporto di molecole tra il nucleo e il citosol}
La membrana nucleare racchiude il DNA e definisce il compartimento nucleare e consiste di due membrane concentriche penetrate da complessi di pori nucleari. Nonostante le membrane
esterne e interne siano continue mantengono distinte composizioni proteiche. La membrana nucleare interna contiene proteine che sono siti di legame per cromosomi e per la lamina
nucleare, una rete di proteine che forniscono supporto strutturale oltre ad agire come ancora per cromosomi e il citoscheletro citoplasmatico. La membrana esterna circonda quella interna
ed \`e continua con la membrana dell'ER. \`E costellata di ribosomi che sintetizzano proteine trasportate nello spazio tra le due membrane (spazio perinucleare), continuo con il lume ER.
Avviene continuamente un traffico bidirezionale tra nucleo e citosol: molte proteine con funzione nel primo sono importate dal secondo e tutti gli RNA sintetizzati sono esportati in
direzione opposta. Entrambi i processi sono selettivi. 
\subsection{Complessi di pori nucleari perforano la membrana nucleare}
Grandi ed elaborati complessi di pori nucleari (NPC) perforano la membrana nucleare in tutti gli eucarioti. Ogni NPC \`e composto da un insieme di $30$ nucleoporine. Ogni nucleoporina
\`e presente in copie multiple. La maggior parte delle nucleoproteine sono composte da domini proteici ripetitivi. Alcune delle nucleoporine di impalcatura sono imparentate ai complessi
di proteina dell'involucro delle vescicole che gli danno forma. La membrana contiene tra le $3000$ e le $4000$ NPC che possono trasportare ognuna fino a $1000$ macromolecole per secondo
in entrambe le direzioni contemporaneamente. Ogni NPC contiene passaggi acquosi che permettono il passaggio di piccole molecole solubili in acqua a un tasso pari alla diffusione libera.
Le moleocle pi\`u grandi devono invece essere attivamente trasportate. Il canale della nucleoporina ha regioni non strutturate che formano un ingarbuglio disordinato che restringe la
diffusione delle molecole pi\`u grandi. Grazie a questo il compartimento nucleare e il citosol possono mantenere diverse composizioni proteiche. L'esportazione o importazione di 
grandi molecole avviene attraverso recettori specifici. 
\subsection{Segnali di localizzazione direzionano le proteine nucleari al nucleo}
Segnali di ordinamento detti segnali di localizzazione nucleare (NLS) sono responsabili per la selettivit\`a del processo di importazione nucleare. In molte proteine consiste di fino a
due corte sequenze ricche di amminoacidi carichi positivamente come lisina e arginina. Altre proteine nucleari contengono diversi segnali. Tali segnali si possono trovare ovunque nella
catena e formano anelli o superfici sulla superficie proteica. Quando deve avvenire l'importazione le particelle si legano a fibrille che si estendono dalle nucleoporine di impalcatura
al bordo dell'NPC nel citosol e poi procedono attraverso il suo centro. Le regioni non strutturate formano una barriera di diffusione e sono spostate per permettere il passaggio delle
proteine. Il trasporto nucleare differisce dagli altri in quanto permette il passaggio di proteine senza modificare la loro struttura.
\subsection{Recettori di importazione nucleare si legano sia ai segnali di localizzazione nucleare e alle proteine NPC}
Per iniziare l'importo lineare la maggior parte dei segnali di localizzazione nucleare devono essere riconosciuti da recettori di import nucleari o importine. Ogni membro della famiglia
dei geni codifica una proteina recettrice che pu\`o legare e trasportare il sottoinsieme contenente il segnale appropriato. Non legano sempre le proteine direttamente ma proteine 
adattatrici possono creare un ponte tra i recettori e i segnali di localizzazione nucleare. Alcune proteine adattatrici sono strutturalmente imparentate ai recettori. Utilizzando una
grande variet\`a di adattatori e recettori le cellule sono capaci di riconoscere tutti i vari segnali mostrati sulle proteine nucleari. I recettori sono proteine solubili al citosol che
si legano al segnale di localizzazione sul carbo e alle ripetizioni fenilanila-glicina (FG) nei domini non strutturati delle nucleoporine canale che confinano con il poro centrale. Tali
ripetizioni si trovano anche nelle fibrille citoplasmatiche e nucleari. Nel ingarbuglio non strutturato al poro tali ripetizioni interagiscono debolmente imponendo una barriera 
impermeabile alle grandi macromolecole e servono come siti di attracco per i recettori di import. Le ripetizioni allineano il cammino attraverso gli NPC preso dalle importine e le 
proteine legate. Secondo un modello di trasporto nucleare il complesso importina-cargo si muove lungo il cammino di trasporto legando, dissociandosi e riassociandosi a sequenze di
ripetizioni FG. Tali legami dissolvono localmente la fase di gel dell'ingarbuglio della nucleoporina  permettendo il passaggio del complesso che una volta dentro il nucleo si dissocia
(cosa che non avviene nel citosol, garantendo direzionalit\`a) una volta dissociata l'importina ritorna al citosol.
\subsection{L'esportazione nucleare lavora come l'importazione ma in senso opposto}
L'esportazione nucleare di grandi molecole avviene attraverso NPC e dipende da un sistema di trasporto selettivo che dipende da segnali di esportazione nucleare sulle macromolecole da
esportare oltre a recettori di esportazione nucleare o esportine che si legano ai segnali di esportazione e a proteine NPC per guidare il loro cargo attraverso l'NPC al citosol.
Sono imparentati strutturalmente alle importine e codificati dalla stessa famiglia di recettori di trasporto nucleare o karyopherins. Si nota pertanto come che i sistemi di importazione
ed esportazione lavano secondo gli stessi meccanismi anche se in direzioni diverse.
\subsection{La Ran GTPasi impone direzionalit\`a al trasporto attraverso NPC}
L'importazione di proteine nucleari attraverso NPC concentra specifiche proteine nel nucleo aumentando l'ordine della cellula. Deve pertanto essere utilizzata l'energia conservata in
gradienti di concentrazione o nella forma legata al GTP monomerica Ran, richiesta sia per l'esportazione che l'importazione. \`E uno switch molecolare che esiste in due conformazioni
in base al fatto che sia legato GDP o GTP. Due proteine regolatorie specifiche al Ran causano la conversione tra le due: una GTPasi attivatrice (GAP) citosilica che causa l'idrolisi 
del GTP e un fattore di scambio di guanina (GEF) nucleare che promuove lo scambio di GDP per GTP e converte Ran-GDP in Ran-GTP. La localizzazione di queste due proteine crea un gradiente
che guida il trasporto nucleare nella direzione appropriata. Le importine si legano alle ripetizioni FG anche se non legate al cargo ed entrano il canale. Quando raggiungono il lato 
nucleare Ran-GTP si lega ad esse e causa il rilascio del cargo che avviene solo sul lato nucleare. L'importina, ora legata alla Ran-GTP \`e trasportato nel citosol dove Ran-GAP causa
l'idrolizzazione del GTP convertendola a Ran-GDP che si dissocia dal recettore che \`e di uovo utilizzabile. L'esporto avviene con un meccanismo simile tranne per il fatto che Ran-GTP
nel nucleo promuove il legame del cargo con l'esportina. Una volta che si muove nella nucleoporina viene idrolizzata e l'esportina la rilascia insieme al cargo. Esportine libere sono
successivamente riportate nel nucleo.
\subsection{Il trasporto attraverso NPC pu\`o essere regolato controllando l'accesso ai macchinari di trasporto}
Alcune proteine contengono sia segnali per l'importazione che per l'esportazione e si spostano continuamente fuori e dentro il nucleo. Il tasso di importazione ed esportazione determina
lo stato di localizzazione delle proteine shuttling. Cambiandolo si cambia la locazione di una proteina. Alcune proteine si muovono continuamente ma in altri casi il trasporto \`e
altamente controllato. Questo avviene regolando i segnali di localizzazione nucleare ed esportazione accendendoli o spegnendoli attraverso fosforilazione di amminoacidi vicini alle
sequenze di segnale. Altri regolatori di trascrizione sono legati a proteine citosiliche che le ancorano nel citosol o mascherano i segnali in modo che non possano interagire con i 
recettori di importazione. Uno stimolo appropriato rilascia la proteina dall'ancora citosilica o dalla maschera e permette il trasporto nel nucleo. Un esempio \`e una proteina che 
controlla l'espressione di proteine coinvolte nel metabolismo del colesterolo. La proteina viene conservata in una forma inattiva come proteina transmembrana nell'ER. Quando una 
cellula viene deprivata di colesterolo la proteina viene trasportata all'apparato di Golgi dove incontra proteasi che separano il dominio citosolico che viene importato nel nucleo dove
attiva la trascrizione dei geni per il recupero e sintesi di colesterolo. L'esportazione di mRNA avviene come grandi assemblaggi che possono contenere centinaia di proteine. I 
complessi ribonucleoproteina-mRNA (mRNP) attraccano al lato nucleare degli NPC dove sono rimodellati ed esportati attraverso idrolisi dell'ATP. 
\subsection{Durante la mitosi la membrana nucleare si disassembla}
La lamina nucleare locata sul lato nucleare della membrana interna \`e una rete di subunit\`a proteiche connesse dette lamine nucleari. Proteine filamentose che si polimerizzano in
un lattice bidimensionale che d\`a forma e stabilit\`a alla membrana nucleare a cui \`e ancorata attraverso gli NPC e proteine transmembrana. La lamina interagisce con la cromatina e
insieme a proteine della membrana interna crea collegamenti strutturali tra DNA e membrana nucleare. Quando il nucleo si disassembla durante la mitosi gli NPC e la lamina nucleare si 
disassemblano e la membrana si frammenta. Durante questo processo NPC si legano a recettori di import che svolgono un ruolo nel riassemblamento alla fine della mitosi. Le proteine della
membrana nucleare si disperdono attraverso la membrana ER. Pi\`u tardi durante la mitosi la membrana nucleare si riassembla sulla superficie dei cromosomi figli. La Ran GTPasi 
agisce come marcatore posizionale per la cromatina e rilascia le proteien NPC in prossimit\`a dei cromosomi dalle importine. 
\section{Il trasporto di proteine in mitocondri e cloroplasti}
I mitocondri e cloroplasti sono organelli racchiusi da doppia membrana che si specializzano nella sintesi dell'ATP utilizzando energia derivante da trasporto di elettroni e
fosforilazione ossidativa nei mitocondri e dalla fotosintesi nei cloroplasti. Entrambi gli organelli contengono il proprio DNA ribosomi e altri componenti per la sintesi di proteine
la maggior parte di esse sono sintetizzate nel nucleo cellulare e importate dal citosol. Ogni proteina deve raggiungere un sottocompartimento particolare in cui funziona. Esistono 
diversi compartimenti nei mitocondri: lo spazio di matrice e spazio intermembrana continuo con lo spazio cristae sono formati da due membrane mitocondriali concentriche. Quella 
interna racchiude lo spazio di matrice e forma invaginazioni dette cristae e la membrana esterna che \`e in contatto con il citosol. Complessi proteici danno confini alle 
giunzioni dove la cristae si invagina e divide la membrana interna in due domini: uno che circonda lo spazio cristae e l'altro dominio lungo la membrana esterna. I cloroplasti posseggono
anch'essi una doppia membrana che racchiude uno spazio intermembrana e lo stroma, l'equivalente dello spazio di matrice. Hanno anche gli spazi tilacoidi circondati dalla membrana
tilacoidi che deriva dalla membrana interna durante lo sviluppo plastide ed \`e pizzicata diventando discontinua. Ogni sottocompartimento contiene le proprie proteine distinte. 
Nuovi mitocondri e cloroplasti sono prodotti dalla crescita di organelli preesistenti seguiti da fissione. La crescita dipende dall'importazione di proteine dal citosol che devono 
essere trasportate attraverso diverse membrane in successione nel processo di traslocazione proteica. 
\subsection{La traslocazione nei mitocondri dipende da sequenze di segnale e traslocatori di proteine}
Le proteine mitocondriali sono solitamente sintetizzate interamente come proteine mitocondriali precursori nel citosol e traslocate da un meccanismo post-traduzionale. Sequenze di 
segnale direzionano i precursori nei sottocompartimenti appropriati. Molte proteine che entrano nello spazio di matrice contengono una sequenza di segnale alla terminazione \ce{N-} 
che una segnal-peptidasi rimuove dopo l'importazione. Altre proteine importate hanno sequenze di segnale interne che non sono rimosse. Le sequenze sono necessarie e sufficienti per
l'importazione e corretta localizzazione delle proteine. Le sequenze di segnale che direzionano la proteine precursore nello spazio di matrice mitocondriale formano $\alpha$-eliche
anfipatiche in cui residui carichi positivamente si raggruppano su un lato dell'elica, mentre quelli non carichi sul lato opposto. Recettori specifici riconoscono tale configurazione
e non la sequenza precisa. Complessi proteici a subunit\`a multiple mediano il movimento delle proteine attraverso le membrane mitocondriali come traslocatori proteici. Il complesso TOM 
trasferisce le proteine attraverso la membrana esterna e due complessi TIM (22/23) attraverso la membrana interna. Questi complessi contengono recettori per i precursori e le 
componenti dei canali di traslocazione. Il complesso TOM \`e richiesto per l'importazione di tutte le proteine codificate dal nucleo. Trasporta la loro sequenza di segnale nello 
spazio di membrana e aiuta a inserire proteine transmembrana nella membrana esterna. $\beta$-barili sono poi passati al complesso SAM, un traslocatore che le posiziona correttamente. 
Il complesso TIM23 trasporta proteine solubili nello spazio di matrice e aiuta a inserire proteine transmembrana nella membrana interna. TIM22 media l'inserimento di una sottoclasse
di proteine per la membrana interna come il trasportatore che muove ATP, ADP e fosfato fuori e dentro il mitocondrio. Il complesso OXA media l'inserimento delle proteine di membrana 
interna sintetizzate dai mitocondri.
\subsection{I precursori mitocondriali sono importati come catene polipeptidiche non piegate}
Le proteine mitocondriali rimangono non piegate nella loro conformazione nel citosol attraverso l'interazione con altre proteine come accompagnatrici della famiglia hsp70 con elementi
dedicati a tali precursori che si legano direttamente alla sequenza di segnale. Le interazioni aiutano a prevenire l'aggregazione o piegamento spontaneo dei precursori prima che si
leghino al complesso TOM. I recettori di TOM legano la sequenza di segnale del precursore, le proteine interagenti sono separate e la catena non piegata \`e portata attraverso il 
canale di traslocazione. Una volta che la proteina si trova nello spazio intermembrana viene separata la sequenza di segnale e si lega al complesso TIM che trasloca la proteina nello
spazio di matrice o la inserisce nella membrana interna.
\subsection{L'importazione \`e guidata da idrolisi dell'ATP e da potenziale di membrana}
L'idrolisi dell'ATP da potenza al sistema di importazione a due siti discreti, uno fuori dal mitocondrio e uno nello spazio di matrice. \`E anche necessario il potenziale di membrana.
La prima richiesta di energia avviene quando la proteina si lega a TOM e si devono separare le proteine accompagnatrici attraverso idrolisi dell'ATP. Una volta che la sequenza di segnale
viene passata al complesso TIM, ulteriore traslocazione richiede potenziale di membrana, il componente elettrico del gradiente di \ce{H+} attraverso la membrana interna, mantenuto
da sistemi di pompe di \ce{H+} dallo spazio di matrice alla membrana interna. L'energia del gradiente oltre a guidare la traslocazione di tali sequenze permette la sintesi dell'ATP.
hsp70 mitocondriale si lega al lato di matrice del complesso TIM23 e agisce come motore che tira il precursore nello spazio di matrice, ha una grande affinit\`a con le proteine non 
piegate e si lega strettamente al precursore appena emerge nello spazio di matrice. Successivamente subisce un cambio conformazionale e si separa dal precursore in un passo dipendente
dall'ATP. 
\subsection{Batteri e mitocondri utilizzano meccanismi simili per inserire porine nella membrana esterna}
La membrana mitocondriale esterna contiene proteine formanti $\beta$-barili dette porine che la rendono permeabile a ioni e a metaboliti inorganici. Il complesso TOM non pu\`o 
integrare  le porine nel bistrato lipidico: sono prima trasportate nello spazio intermembrana non piegate dove si legano transientemente con proteine specializzate che impediscono 
la loro aggregazione. Successivamente si legano al complesso SAM che le inserisce nella membrana esterna e le piega. 
\subsection{Trasoprto nella membrana interna mitocondriale e nello spazio intermembrana avviene secondo diverse strade}
Inizialmente lo stesso meccanismo che trasporta proteine nello spazio di matrice utilizzando i traslocatori TOM e TIM 23 media la traslocazione iniziale di molte proteine destinate
alla membrana interna o allo spazio intermembrana. Nella strada pi\`u comune solo la terminazione \ce{N-} con la sequenza di segnale entra nello spazio di matrice. Una sequenza di
amminoacidi idrofobici agisce come sequenza di stop per il trasferimento, impedendo ulteriore traslocazione. Il rimanente della proteina attraversa la membrana esterna, la sequenza
di segnale \`e rotta, la sequenza idrofobica \`e rilasciata da TIM23 e rimane ancorata nella membrana interna. In un'altra strada il complesso TIM 23 trasloca l'intera proteina nello
spazio di matrice dove una peptidasi di segnale rimuove la sequenza di segnale nella terminazione \ce{N} che espone una sequenza idrofobica che guida la proteina al complesso OXA che
la inserisce nella membrana interna. Molte proteine che usano questi cammini rimangono ancorate alla membrana attraverso la loro sequenza idrofobica, mentre altre sono rilasciate 
nello spazio intermembrana da una proteasi che rimuove l'ancora di membrana. MOlte di queste proteine rimangono attaccate alla superficie esterna della membrana interna come subunit\`a
di complessi proteici che contengono proteine transmembrana. Alcune proteine dello spazio intermembrana sono importate attraverso un altro cammino: contengono motivi di cisteina e 
formano un legame disulfide transiente con la proteina Mia40. Le proteine importate sono poi rilasciate in una forma ossidata con legami disulfiti nella catena. Mia40 si riduce e 
reossidata. I mitocondri sono i siti di sintesi dell'ATP e contengono anche molti enzimi metabolici e devono trasportare piccoli metaboliti attraverso la membrana. La membrana interna
non contiene le porine di quella esterna ma una famiglia di trasportatori specifici al metabolita. 
\subsection{Due sequenze di segnale direzionano le proteine a una membrana tilacoide nei cloroplasti}
L'importazione di proteine nei cloroplasti \`e simile a quello nei mitocondri, con processi post-traduzionali con complessi separati per ogni membrana e usano segnali anfipatici alle
terminazione \ce{N}. Le molecole che formano i complessi differiscono e i cloroplasti non hanno un gradiente \ce{H+} attraverso la membrana interna e pertanto usano idrolisi di GTP e
ATP. I recettori di import sui cloroplasti e mitocondri di piante devono riconoscere le proprie proteine. I cloroplasti possiedono inoltre un altro compartimento il tilacoide nella cui
membrana si trovano molte proteine. Molte proteine sono importate in un processo a due passaggi: prima passano nella attraverso la doppia membrana e poi nello stroma. Successivamente
si traslocano nella membrana o spazio tilacoide. 
\section{Perossisomi}
I perossisomi sono circondati da una singola membrana e non contengono DNA o ribosomi e tutte le proteine sono codificate nel nucleo e le acquisiscono attraverso importazione selettiva
dal citosol. I perossisomi sono comuni a tutte le cellule eucariotiche, contengono enzimi ossidativi ad alte concentrazioni. Sono un sito di utilizzo di ossigeno e si pensa siano
vestigia di antichi organelli che svolgevano il metabolismo dell'ossigeno. I mitocondri li hanno resi obsoleti e si occupano ora solo delle funzioni che il mitocondrio non svolge.
\subsection{I perossisomi usano ossigeno molecolare e perossido di idrogeno per svolgere reazioni di ossidazione}
Contengono degli enzimi che usano ossigeno molecolare per rimuovere idrogeno da substrati specifici in una reazione di ossidazione che produce perossido di idrogeno: \ce{RH2 + O2 ->
R + H2O2}. La catalasi usa il perossido generato per ossidare altri substrati: \ce{H2O2 + RH2 -> R + 2H2O}. Quando \ce{H2O2} si accumula in eccesso viene convertito in acqua:
\ce{2H2O2 -> 2H2O + O2}. Una funzione principale della reazione di ossidazione \`e la rottura delle molecole di acidi grassi nella $\beta$ ossidazione che accorcia le catene alchili
sequenzialmente convertendolo in acetil CoA che viene esportato nel citosol per le reazioni biosintetiche. I perossisomi animali si occupando i catalizzare la prima reazione nella 
formazione di plasmalogeni, la pi\`u abbondante classe di fosfolipidi nella mielina. Sono altamente responsivi verso l'ambiente esterno e nelle piante si occupano della 
fotorespirazione. 
\subsection{Una corta sequenza di segnale direziona l'importazione di proteine nei perossisomi}
La sequenza di amminoacidi Ser-Lys-Leu alla terminazione \ce{C} di molte proteine perossisomali funziona come un segnale di import, mentre altre ne contengono una verso la terminazione
\ce{N}. Una di queste causa l'importazione nei perossisomi. Sono riconosciuti da recettori solubili nel citosol e le perossine partecipano nel processo di import guidato dall'idrolisi
dell'ATP. Un complesso di $6$ proteine forma un complesso alla membrana che permette il passaggio alle proteine piegate. Il poro si pensa sia dinamico nella dimensione. Il recettore 
Pex5 riconosce il segnale di importazione perossisomale \ce{C}-terminale, accompagna il cargo fino nel perossisoma, lo rilascia e ritorna nel citosol. Dopo che il cargo viene rilasciato
Pex5 subisce ubiquitilazione in modo da rilasciarlo nel citosol, dove l'ubiquitina \`e rimossa. Un ATPasi composta da Pex1 e Pex6 cattura l;energia dell'idrolisi per il rilascio di Pex5
dal perossisoma. Molte proteine di membrana persossisomale sono create nel citosol e inserite nella membrana di perossisomi gi\`a esistenti, ma altre sono prima integrate nella membrana
ER dove sono poi impacchettate in vescicole precursori dei perossisomi. 
\subsection{Il reticolo endoplasmatico}
Tutte le cellule eucariotiche possiedono un reticolo endoplasmatico, la cui membrana costituisce met\`a della membrana totale della cellula. \`E organizzato come un labirinto simile a 
una rete di tubuli ramificati e sacche appiattite interconnessi che si estendono attraverso il citosol con la membrana continua con la membrana nucleare esterna. Le membrane di ER e 
quella nucleare formano un foglio continuo che racchiude il lume ER o lo spazio di cisterna ER. L'ER ha un ruolo centrale nella sintesi di lipidi e proteine, come conservatore di
\ce{Ca^{2+}} intracellulare come segnale ed \`e il sito di produzione di organelli come ER, apparato di Golgi, lisosomi, endosomi, vescicole secretorie e membrane plasmatiche. 
La sua membrana \`e i sito in cui i lipidi per le membrane mitocondriali e perossisomiche sono sintetizzate. Tutte le cellule che saranno secrete all'esterno, all'apparato di Golgi o ai
lisosomi sono inizialmente trasportate nel lume ER.
\subsection{L'ER \`e strutturalmente e funzionalmente diverso}
Se tutte le funzioni dell'ER sono fondamentali ak funzionamento della cellula la loro importanza relativa dipende dal tipo cellulare e diverse regioni dell'ER sono altamente 
specializzate come avviene per l'ER ruvido. Le cellule mammifere iniziano l'importazione delle proteine attraverso l'ER prima della completa sintesi della catena polipeptidica (\`e un
processo co-traduzionale). L'importo di proteine nei mitocondri, cloroplasti, nuclei e perissosomi \`e post-traduzionale. Nel primo tipo di trasporto il ribosoma si trova attaccato alla
membrana dell'ER permettendo a una terminazione della proteina di essere traslocata nell'ER. Questi ribosomi si trovano sulla regione dell'ER ruvido. Le regioni senza ribosomi sono 
dette ER liscio. Aree di ER liscio in cui proteine trasportano vescicole contenenti proteine verso l'apparato di Golgi sono dette ER transizionale. In alcuni tipi di cellule l'ER liscio
\`e abbondante e ha altre funzioni: \`e prominente nelle cellule che si specializzano nel metabolismo lipidico o nell'epatocita del fegato dove \`e il sito di produzione delle 
particelle lipoproteiche. Gli enzimi che sintetizzano le componenti lipidiche delle particelle si trovano nella membrana dell'ER liscio insieme ad altri che catalizzano reazioni per
detossificare droge liposolubili e altri composti creati dal metabolismo. Un altra funzione dell'ER \`e il sequestro di \ce{Ca^{2+}} dal citosol. Il rilascio e la ripresa avviene in
risposta a segnali extracellulari. Si nota come quando le cellule si rompono per omogenizzazione l'ER si rompe in fragmenti che si richiudono in piccole vescicole dette microsomi 
facilmente purificabili e mantengono la propria funzione. 
\subsection{Sequenze di segnale sono state scoperte prima in proteine importate nell'ER ruvido}
L'ER cattura selettivamente due tipi di proteine sintetizzate: transmembrana che si incorporano nella membrana dell'ER e membrane solubili in acqua rilasciate nel lume ER. Alcune 
proteine transmembrana hanno funzione nell'ER ma la maggior parte sono destinate alla membrana plasmatica o di un altro organello, mentre le solubili sono destinate a secrezione o 
residenza nel lume di un organello cellulare. Tutte le proteine sono direzionate alla membrana ER da una sequenza di segnale ER che inizia la traslocazione. Le sequenze segnale
dopo il trasporto sono separate da una peptidasi di segnale nella membrana. 
\subsection{Una particella di riconoscimento del segale (SRP) direziona la sequenza di segnale ER a un recettore specifico nella membrana dell'ER ruvido}
La sequenza di segnale ER \`e guidata all'ER da una particella di riconoscimento del segnale (SNP) che cicla tra la membrana ER e il citosol e si lega alla sequenza di segnale e un 
recettore SRP nella membrana ER. L'SRP \`e un grande complesso di diverse catene polipeptidiche legate a una singola molecola di RNA. Le sequenze di segnale sono varie ma possiedono 
almeno otto o pi\`u amminoacidi non polari al centro. Il sito di legame nell'SRP consiste di una tasca idrofobica abbastanza plastica da accomodare sequenze idrofobiche di diversa
forma e dimensione. L'SRP \`e una struttura a bastoncello che si avvolge intorno alla maggiore subunit\`a ribosomiale con una fune che si lega alla sequenza di segnale appena emerge 
dal ribosoma. L'altra terminazione blocca il sito di legame per i fattori di allungamento all'interfaccia tra le subunit\`a ribosomiali in modo da bloccare la sintesi che
da il tempo necessario al ribosoma per legarsi alla membrana ER prima del completamento di una catena polipeptidica assicurando che la proteina non sia rilasciata nel citosol. La pausa
impedisce anche il piegamento della proteina rendendo inutili le proteine accompagnatrici. Al legame con la sequenza SRP espone un sito di legame per il recettore SRP, un complesso
transmembrana nella membrana ER ruvida. Il legame dell'SRP con il recettore porta il complesso SRP ribosoma a un traslocatore proteico libero nella membrana. SRP e il recettore sono
rilasciati e il traslocatore trasferisce la catena polipeptidica attraverso la membrana. Si creano in questo modo i ribosomi legati a membrana sull'ER ruvido dalla parte citosilica e
ribosomi liberi nel citosol. Essendo che molti ribosomi possono legarsi a una singola molecola di mRNA si forma un poliribosoma. Se questo codifica una proteina con una sequenza di 
segnale ER il poliribosoma si attacca alla membrana ER. I ribosomi individuali associati possono ritornare al citosol dopo la traduzione, mentre l'mRNA rimane attaccato alla membrana
da ribosomi diversi.
\subsection{La catena polipeptidica passa attraverso un canale acquoso nel traslocatore}
Il traslocatore forma un canale acquoso nella membrana attraverso cui passa la catena. Il nucleo del traslocatore, il complesso Sec61 \`e formato da tre subunit\`a e suggerisce che 
$\alpha$-eliche dalla subunit\`a maggiore circondano un canale centrale, gated da una corta $\alpha$-elica che lo tiene chiuso a riposo e si muove quando sta lavorando. Il canale si
apre transientemente quando una catena lo attraversa. IL canale rimane chiuso per rendere la membrana impermeabile a ioni. Quando lo attraversa la catena un anello di amminoacidi 
idrofobici forma un isolamento flessibile. Il poro pu\`o aprirsi lungo una fessura sul lato che permette accesso laterale in un nucleo idrofobico della membrana in modo 
da rilasciare peptidi di segnale separati nella membrana e per integrare le proteine transmembrana nel bistrato. Questi insieme a enzimi che modificano la catena crescente come
trasferasi oligosaccarida e la peptidasi di segnale formano il translocon.
\subsection{Traslocazione attraverso la membrana ER non richiede sempre un allungamento della catena polipeptidica in corso}
