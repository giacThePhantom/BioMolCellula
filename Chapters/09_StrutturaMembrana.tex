\chapter{Struttura di membrana}
Le membrane cellulari sono vitali per la vita della cellula. La membrana plasmatica definisce i limiti della cellula e mantiene le differenze essenziali
tra il citosol e l'ambiente extracellulare. All'interno delle cellule eucariotiche le membrane del nucleo, il reticolo endoplasmatico, l'apparato di 
Golgi, i mitocondri e altri organelli racchiusi da membrana mantengono differenze caratteristiche per quanto riguarda i loro ocntenuti. I gradienti ionici
tra le membrane generati dall'attivit\`a di proteine di membrana possono essere utilizzati per la sintesi dell'ATP, per guidare il tasporto di soluti
specifici o per produrre e trasmettere segnali elettrici. Nella membrana plasmatica sono anche presenti proteine sensori che permettono alla cellula di 
cambiare il comportamento in base a segnali esterni. Questi recettori trasferiscono informazinoi e non molecole attraverso la membrana. Tutte le membrane
possiedono una struttura generale comune: sono un film di lipidi e proteine, mantenuti insieme da interazioni non covalenti. Sono strutture dinamiche e 
fluide e la maggior parte delle loro molecole sono libere di muoversi nel piano della membrana. Le molecole lipidiche sono ordinate in un doppio strato 
che serve come membrana impermeabile all'acqua. Le proteine di membrana attraversano il bistrato e mediano le funzioni della membrana. Nella membrana 
plasmatica alcune connettono il citoscheletro a una matrice extracellulare su cellule adiacenti. 
\section{Il bistrato lipidico}
Il bistrato lipidico fornisce la struttura base per tutte le membrane cellulari e la sua struttura \`e attribuibile alle propriet\`a delle molecole 
lipidiche che assumono tale conformazione spontaneamente. 
\subsection{Fosfogliceridi, sfingolipidi e steroli sono i lipidi pi\`u comuni nelle membrane cellulari}
Le molecole lipidiche costituiscono il $50\%$ della massa delle membrane cellulari. Tutte le molecole lipidiche nella cellula sono anfipatiche, ovvero
hanno una parte idrofilica o polare e una idrofobica o non polare. I lipidi di membrana pi\`u abbondanti sono i fosfolipidi che possiedono una testa
polare con un gruppo fosfato e due code idrocarburiche idrofobiche. Le code sono tipicamente acidi grassi e possono cambiare in lunghezza. Una coda
contiene tipicamente uno o pi\`u legami doppi cis, mentre l'altra no ognuno dei quali forma una curva nella coda. Il fosfolipide principale negli animali
\`e il fosfogliceride con un backbone a tre glicerli. Due lunghe catene di acidi grassi sono legate a atomi adiacenti del carbonio attraverso legami
estere e il terzo atomo di carbonio del glicerolo \`e legato a un gruppo fosfato che viene legato a diversi tipi di gruppo di testa. Combinando diversi
acidi grassi e gruppi di testa la cellula crea diversi fosfogliceridi come fosfatidiletanolammina, fosfatidilserina e fosfatidilcolina. Un altra classe
importante di fosfolipidi sono gli sfingolipidi che sono costriuiti dalla sfingosina, una catena acile lunga con un gruppo ammino  e due gruppo idrossili
a una fine. Nella sfingomielina la tadena a acido grasso \`e legata al gruppo ammino, il gruppo fosfocolina al gruppo idrossile terminale. Oltre ai 
fosfolipidi nel bistrato sono contenuti glicolipidi e colisterolo, i primi sono come gli sfingolipidi ma con uno zucchero invege del gruppo di testa. 
La membrana plasmatica degli eucarioti contiene grandi quantit\`a di colesterolo, uno sterolo con una struttura ad anello rigida, un gruppo idrossile 
polare e una corta catena idrocarburica e si orientano nel bistrato con il gruppo idrossilico vicino alle teste polari. 
\subsection{I fosfolipidi formano bistrati spontaneamente}
La natura anfipatica e la forma dei fosfolipidi causa la formazione spontanea di bistrati in ambiente acquoso. Quando moleocle anfipatiche sono esposte in
ambiente acqueoso si aggregano spontaneamente per seppellire le loro code idrofobiche all'interno dove sono separate dall'acqua e espongono le teste
idrofiliche a essa. In base alla loro forma possono creare micelle sferiche o foglietti a doppio strato o bistrati con le catene idrofobiche all'interno
delle teste idrofiliche. Una rottura nel bistrato causa un lato a contatto con l'acqua che essendo sfavorevole energeticamente causa i lipidi di 
riordinarsi in modo da eliminarlo. Il solo modo per bisrtati di avere questi lati \`e di formare un compartimento chiuso chiudendosi su s\`e stesso. 
\subsection{Il bistrato lipidico \`e un fluido bidimensionale}
Molecole individuali di lipidi si possono diffondere liberamente nel piano di un bistrato lipidico. Si \`e notato come le molecole lipiche migrano 
raramente tra un monostrato all'altro nel processo di \``flip flop" che accade su tempi di ore (con il colesterolo come eccezione) ma si scambiano
veolocemente tra uno stesso monolivello dando origine a una diffusione laterale rapida. I lipidi ruotano inoltre velocemente sul loro asse e hanno 
catene idrocarburiche flessibili. Il confinamento in un monolivello causa problemi per la sintesi che avviene unicamente in uno strato, risolto da
proteine di membrana dette traslocatori di fosfolipidi o flippasi che catalizzano il flip-flop rapido di fosfolipidi da un monolivello all'altro. I 
diversi bistrati non si fondono facilmente quando sospesi in acqua in quanto dovrebbero spostare le molecole legate ai gruppi di testa. In questo modo
viene mantenuta l'integrit\`a di molte membrane interne e prevenuta una loro fusione incontrollata. 
\subsection{La fluidit\`a del bistrato lipidico dipende dalla sua composizione}
