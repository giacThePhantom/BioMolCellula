\chapter{Struttura di membrana}
Le membrane cellulari sono vitali per la vita della cellula. La membrana plasmatica definisce i limiti della cellula e mantiene le differenze essenziali
tra il citosol e l'ambiente extracellulare. All'interno delle cellule eucariotiche le membrane del nucleo, il reticolo endoplasmatico, l'apparato di 
Golgi, i mitocondri e altri organelli racchiusi da membrana mantengono differenze caratteristiche per quanto riguarda i loro ocntenuti. I gradienti ionici
tra le membrane generati dall'attivit\`a di proteine di membrana possono essere utilizzati per la sintesi dell'ATP, per guidare il tasporto di soluti
specifici o per produrre e trasmettere segnali elettrici. Nella membrana plasmatica sono anche presenti proteine sensori che permettono alla cellula di 
cambiare il comportamento in base a segnali esterni. Questi recettori trasferiscono informazinoi e non molecole attraverso la membrana. Tutte le membrane
possiedono una struttura generale comune: sono un film di lipidi e proteine, mantenuti insieme da interazioni non covalenti. Sono strutture dinamiche e 
fluide e la maggior parte delle loro molecole sono libere di muoversi nel piano della membrana. Le molecole lipidiche sono ordinate in un doppio strato 
che serve come membrana impermeabile all'acqua. Le proteine di membrana attraversano il bistrato e mediano le funzioni della membrana. Nella membrana 
plasmatica alcune connettono il citoscheletro a una matrice extracellulare su cellule adiacenti. 
\section{Il bistrato lipidico}
Il bistrato lipidico fornisce la struttura base per tutte le membrane cellulari e la sua struttura \`e attribuibile alle propriet\`a delle molecole 
lipidiche che assumono tale conformazione spontaneamente. 
\subsection{Fosfogliceridi, sfingolipidi e steroli sono i lipidi pi\`u comuni nelle membrane cellulari}
Le molecole lipidiche costituiscono il $50\%$ della massa delle membrane cellulari. Tutte le molecole lipidiche nella cellula sono anfipatiche, ovvero
hanno una parte idrofilica o polare e una idrofobica o non polare. I lipidi di membrana pi\`u abbondanti sono i fosfolipidi che possiedono una testa
polare con un gruppo fosfato e due code idrocarburiche idrofobiche. Le code sono tipicamente acidi grassi e possono cambiare in lunghezza. Una coda
contiene tipicamente uno o pi\`u legami doppi cis, mentre l'altra no ognuno dei quali forma una curva nella coda. Il fosfolipide principale negli animali
\`e il fosfogliceride con un backbone a tre glicerli. Due lunghe catene di acidi grassi sono legate a atomi adiacenti del carbonio attraverso legami
estere e il terzo atomo di carbonio del glicerolo \`e legato a un gruppo fosfato che viene legato a diversi tipi di gruppo di testa. Combinando diversi
acidi grassi e gruppi di testa la cellula crea diversi fosfogliceridi come fosfatidiletanolammina, fosfatidilserina e fosfatidilcolina. Un altra classe
importante di fosfolipidi sono gli sfingolipidi che sono costriuiti dalla sfingosina, una catena acile lunga con un gruppo ammino  e due gruppo idrossili
a una fine. Nella sfingomielina la tadena a acido grasso \`e legata al gruppo ammino, il gruppo fosfocolina al gruppo idrossile terminale. Oltre ai 
fosfolipidi nel bistrato sono contenuti glicolipidi e colisterolo, i primi sono come gli sfingolipidi ma con uno zucchero invege del gruppo di testa. 
La membrana plasmatica degli eucarioti contiene grandi quantit\`a di colesterolo, uno sterolo con una struttura ad anello rigida, un gruppo idrossile 
polare e una corta catena idrocarburica e si orientano nel bistrato con il gruppo idrossilico vicino alle teste polari. 
\subsection{I fosfolipidi formano bistrati spontaneamente}
La natura anfipatica e la forma dei fosfolipidi causa la formazione spontanea di bistrati in ambiente acquoso. Quando moleocle anfipatiche sono esposte in
ambiente acqueoso si aggregano spontaneamente per seppellire le loro code idrofobiche all'interno dove sono separate dall'acqua e espongono le teste
idrofiliche a essa. In base alla loro forma possono creare micelle sferiche o foglietti a doppio strato o bistrati con le catene idrofobiche all'interno
delle teste idrofiliche. Una rottura nel bistrato causa un lato a contatto con l'acqua che essendo sfavorevole energeticamente causa i lipidi di 
riordinarsi in modo da eliminarlo. Il solo modo per bisrtati di avere questi lati \`e di formare un compartimento chiuso chiudendosi su s\`e stesso. 
\subsection{Il bistrato lipidico \`e un fluido bidimensionale}
Molecole individuali di lipidi si possono diffondere liberamente nel piano di un bistrato lipidico. Si \`e notato come le molecole lipiche migrano 
raramente tra un monostrato all'altro nel processo di \``flip flop" che accade su tempi di ore (con il colesterolo come eccezione) ma si scambiano
veolocemente tra uno stesso monolivello dando origine a una diffusione laterale rapida. I lipidi ruotano inoltre velocemente sul loro asse e hanno 
catene idrocarburiche flessibili. Il confinamento in un monolivello causa problemi per la sintesi che avviene unicamente in uno strato, risolto da
proteine di membrana dette traslocatori di fosfolipidi o flippasi che catalizzano il flip-flop rapido di fosfolipidi da un monolivello all'altro. I 
diversi bistrati non si fondono facilmente quando sospesi in acqua in quanto dovrebbero spostare le molecole legate ai gruppi di testa. In questo modo
viene mantenuta l'integrit\`a di molte membrane interne e prevenuta una loro fusione incontrollata. 
\subsection{La fluidit\`a del bistrato lipidico dipende dalla sua composizione}
La fluidit\`a della membrana cellulare deve essere precisamente regolata in quanto dalla sua viscosit\`a dipendono alcuni processi di traporto e attivit\`a enzimatiche. La fluidit\`a del
bistrato dipende dalla sua composizione etemperatura.  Esiste una temperatura detta transizione di fase in cui assume una struttura cristallina rigida o gel, pi\`u bassa rispetto pi\`u 
corte le catene idrocarburiche o pi\`u presenti i doppi legami. La lunghezza di coda minore riduce la tendenza delle catene idrocarburiche di interagire tra di loro e i doppi legami
cis producono curve in esse che rendono la loro condensazione pi\`u difficile e la membrana rimane fluida a temperature pi\`u basse. Il colesterolo modula le propriet\`a del bistrato:
quando mescolato con fosfolipidi aumenta le propriet\`a di barriera permeabile. Si inserisce in esso con il gruppo idrossile vicino alle teste polari dei fosfolipidi in 
modo che l'anello steroide rigido interagisca e immobilizzi parzialmente le regioni delle catene idrocarboniche pi\`u vicine alle teste polari. Diminuisce la motilit\`a dei primi gruppi
\ce{CH2} della catena e lo rende meno deformabile diminuiendo la permealibilit\`a di piccole molecole solubili in acqua. Non rende la membrana meno fluida in quanto ad alte 
concentraizoni previene le catene idrocarburiche di interagire e cristallizzarsi. 
\subsection{Nonostante la loro fluidit\`a i bistrati lipidici possono formare domini di composizioni diverse}
Nonostante molti lipidi e proteine di membrana siano distribuiti uniformemente e segregazione di fase a larga scala si trovi raramente specifici lipidi e proteine di membrana si 
concentrano in maniera temporanea e dinamica facilitate da interazioni da proteine che permettono la formazione di regioni di membrana specializzata. Questi cluster aiutano a creare
zattere in membrane organizzando e concentrando le proteine per trasporto o per lavorare nell'assemblaggio di proteine.
\subsection{Goccioline di lipidi sono circondate da un monostrato fosfolipidico}
La maggior parte delle cellule conservano lipidi in eccesso in goccioline di pipidi da dove possono essere recuperate. Gli adipociti sono cellule specializzate per la conservazione
di lipidi. Contengono una gocciolina gigantesca di lipidi che occupa la maggior parte del citoplasma. Gli acidi grassi possono essere liberati dalle goccioline a richiesta e esportati
ad altre cellule attraverso il flusso sanguigno. Tali goccioline conservano lipidi neutri come trigliceroli e colesteroli esteri che sono sintetizzti da enzimi nella membrana del
reticolo endoplasmatico. Sono molecole idrofobiche e si aggregano spontaneamente in goccioline tridimensionali. Sono organelli unici in quanto sono circondati da un singolo monostrato di
fosfolipidi con una grande variet\`a di proteine, enzimi coinvolti nel metabolismo dei lipidi. 
\subsection{L'asimmetria del bistrato lipidico \`e funzionalmente importante}
La composizione lipidica dei due monostrati \`e diversa: per esempio nelle cellule eritrocite del sangue rosso tutte le molecole fosfolipidiche hanno colina nel gruppo di testa nel 
monostrato esterno, mentre tutte contengono un amminoacido primario terminale nel monostrato interno. In quanto il fosfoatidilserina \`e locato nel monostrato interno si trova una
differenza in carica tra le due met\`a del bistrato. QUesta asimmetria \`e specialmente importante in convertire segnali extracellulari in intracellulari: molte proteine citosoliche
si legano a specifici gruppi di testa lipidici e vengono attivati dai segnali che cambiano la carica del monostrato interno. In altri casi gruppi di testa devono essere modificati per 
creare siti di legame delle proteinea un tempo e luogo specifico. Varie chinasi lipidiche possono aggiungere un gruppo fosfato a posizione distinte sull'anello inositolo creando siti
di legame che reclutano proteine specifiche dal citosol. SOno usati anche per convertire i segnali extracellulari in intracellulari attraverso fosfolipasi attivate da segnali 
extracellulari che rompono specifiche molecole di fosfolipidi generando frammenti che agiscoono come brevi mediatori intracellulari. Questa asimmetria viene utilizzata per riconoscere
le cellule vive da quelle morte in quanto alla morte la fosfoatidilserina si trasloca al monostrato esterno attraverso il fofsfolipide traslocatore che normalmente trasporta questo
lipide al monostrato interno \`e disattivato e una scramblasi che trasferisce i fosfolipidi non specificatamente si attiva.
\subsection{I glicolipidi si trovano sulla superficie di tutte le membrane plasmatiche eucariotiche}
I glicolipidi sono lipidi che contengono zucchero e hanno una asimmetria estrema: si trovano esclusivamente nel monostrato dalla parte opposta del citosol. Tendono ad autoassociarsi
attraverso legami idrogeno tra gli zuccheri e forze di Van der Waals rtra le catene idrocarburiche che li partiziona in fasi di zattere lipidiche. Questa disrtibuzione assimmetrica
risulta dall'addizione dello zucchero nel lumen degli apparati di GOlfi. Il compartimento in cui sono creati \`e topologicamente equivalente all'estenro della cellula. Quando sono 
portati alla membrana plasmatica gli zuccheri sono esposti alla superficie cellulare dove hanno importanti ruoli in interazioni. Si trovano anche in alcune membrane intracellulari. Il
glicolipide pi\`u complesso \`e il ganglioside che contiene oligosaccaride con pi\`u frazioni di acidi sialici che gli danno una carica negativa. La loro funzione \`e quella di 
proteggere contro condizioni dure. Quelli carichi sono importanti a causa del loro effetto elettrico che altera la concentrazione di ioni. Hanno un ruolo nei processi di riconoscimento
cellulare in cui le lectine, proteine che si legano al carbonio legate alla membrana legano ai gruppi zucchero sui glicolipidi e glicoproteine. Alcuni glicolipidi forniscono punti di 
entrata per tossine e virus.
\section{Proteine di membrana}
Le proteine di membrana svolgono la maggior parte delle funzioni specifiche a essa dandole le propriet\`a funzionali caratteristiche. La quantit\`a e il tipo di proteine sono altamente
variabili: nelle membrane mieline, che servono come isolamento elettrico per gli assoni delle cellule nervose meno del $25\%$ della massa \`e proteine. Nelle membrane coinvolte nella
produzione di ATP circa il $75\%$ \`e composto da proteine. Essendo comunque le molecole lipidiche pi\`u piccole rispetto alle proteine sono sempre pi\`u numerose rispetto a esse. 
\subsection{Le proteine di membrana si possono associare con il bistrato lipidico in vari modi}
Le proteine di membrana sono anfipatiche e molte si estendono attraverso il bistrato e sono dette proteine transmembrana, con parte della propria massa su ogni parte. La loro regione
irdofobica passa attraverso le molecole lipidiche dove sono separate dall'acqua, la regione idrofilica \`e esterna alla membrana e esposta all'acqua. L'attacco covalente di una catena
acido grasso che si inserisce nel monostrato citosilico aumenta l'idrofobicit\`a delle proteine transmembrana. Altre proteine si trovano interamente nel citosol e sono attaccate al
monostrato citosilico da un $\alpha$-elica anfipatica esposta sulla superficie della proteina o da catene lipidiche attaccate covalentemente. Altre proteine di membrana sono 
completamente esposte alla superficie esterna della cellula attaccate al bistrato da un legame covalente con un oligosaccaride specifico a un'ancora lipidica nel monostrato esterno
della membrana plasmatica. Le proteine legate ai lipidi sono fatte come proteine solubili nel citosol e sono poi ancorate alla membrana da legami covalenti con il gruppo lipide, altre
sono create come proteine di membrana nel reticolo endoplasmatico. Quando ancora nell'ER i segmenti transmembrana della proteina sono rotti e si aggiunge una ancora 
glicosilfosfatidilinositolo (GPI) lasciando la proteina legata alla superficie non citosilica della membrana ER da questa ancora. Vescicole di trasporto eventualmente trasportano
la proteina alla membrana plasmatica. Le proteina associate alla membrana non si estendono all'interno idrofobico ma sono legate ad una faccia del bistrato da interazioni non covalenti
con altre proteine di membrana e possono essere rilasciate da essa con interazioni gentili come esposizione di soluzioni di piccola forza ionica o pH estremo e sono dette proteine di 
membrana periferiche. Le proteine transmembrana e quelle mantenute nel bistrato da gruppi lipidici o regioni di polipeptide idrofobico non possono essere rimosse in questo modo.
\subsection{Ancore lipidice controllano la localizzazione nella membrana di alcune proteine segnalatrici}
Solo le proteine transmembrana possono funzionare su tutte e due le parti del bistrato o trasportare molecole attraverso essa. I recettori della superficie di membrana sono di solito
proteine di transmembrana che legano molecole di sengale nello spazio extracellulare e generano diversi segnali intracellulare sulla parte opposta della membrana. Una proteina di
trasporto deve fornire un cammino per le molecole che devono attraversare la barriera di permeabilit\`a idrofobica del bistrato, cosa ottenuta dall'architettura di proteine transmembrana
multipasso. Le proteine che funzionano su solo un lato del bistrato lipidico sono associate esclusivamente con il monostrato lipidico o un dominio proteico su tale lato. Alcune proteine
di segnale intracelulare che inoltrano dei messaggi sono legati alla met\`a citosilica della membrana plasmatica da gruppi lipidi attaccati covalentemente come acidi grassi o gruppi 
prenile. In alcuni casi l'acido miristico \`e aggiunto al gruppo \ce{N-} terminale della proteina durante la sintesi su un ribosoma. L'attacco attraverso un'ancora lipidica non \`e molto
forte, ma pu\`o essere rafforzato da una seconda ancora. Molte proteine si attaccano alla proteine transientemente: alcune sono proteine periferiche che si associano da interazioni
tra proteine regolate, altre si trasformano da solubili a proteine di membrana grazie a un cambio conformazionale che espone un peptide idrofobico o attacca covalentemente un'ancora 
lipidica.
\subsection{Nella maggior parte delle proteine transmembrana la catena polipeptidica attraversa il bistrato lipidico in una conformazione a $\mathbf{\alpha}$-elica}
Una proteina transmembrana ha un'orientamento unico nella membrana che riflette il modo asimmetrico in cui \`e inferita nel bistrato lipidico nell'ER durante la biosintesi e le 
funzioni diverse del dominio citosilico e noncitosilico. Questi due sono separati da segmenti della catena polipeptidica che attraversano la membrana che contattano l'ambiente 
idrofobico del bistrato e sono composti da amminoacidi con catene laterali non polari. Essendo i legami peptidici polari tutti i legami peptidici tendono a formare legami a idrogeno tra
di loro. Questo \`e massimizzato quando la catena forma una $\alpha$-elica quando attraversa il bistrato. In proteine transmembrana a singolo passaggio la catena polipeptidica attraversa
solo una volta, mentre nelle proteine transmembrana a multiplo passaggio viene attraverseata muolte volte. Un altro modo affinch\`e i legami peptidici soddisfino i requisiti di 
legami a idrogeno \`e di formare un $\beta$-foglietto girato in un cilindro o $\beta$-barrel, presnte nelle proteine porina. La necessit\`a di massimizzare i legami a idrogeno vuol dire
che una catena polipetidica non cambia direzione mentre attaversa la membrana, ma proteine a passaggio multiplo possono contenere regioni che si piegano nella membrana a entrambe le 
estremit\`a in spazi tra $\alpha$-eliche senza toccare il nucleo idrofobico del doppio strato ed essendo che non devono interagire con il bistrato possono avere una variet\`a di 
strutture secondarie. Tali regioni sono importanti per la funzione di tali proteine. 
\subsection{$\mathbf{\alpha}$-eliche transmembrana interagiscono tra di loro}
Le $\alpha$-eliche transmembrana di molte proteine a singolo passaggio non contribuiscono al piegamento dei domini proteici sui lati della membrana. Nonostatnte questo non hanno solo
la funzione di ancora e formano omo o eterodimeri che sono tenuti insemi da interazioni non covalenti ma specifiche tra tue $\alpha$-eliche grazue alle interazioni  tra le sequenze di
amminoacidi idrofobici. Tali $\alpha$-eliche nelle proteine transmembrana a multipassaggio occupano posizioni specifiche nella struttrua piegata della proteina determinate dalle 
interazioni con le loro vicine, cruciali per la struttura e funzione di molti canali e trasportatori. Eliche vicine fanno da scudo per altre eliche dai lipidi di membrana che nonosotante
questo sono composte principalmente da amminoacidi idrofobiche in quanto le $\alpha$-eliche transmembrana sono inserite nel bistrato sequenzialmente da un traslocatore proteico. Dopo
che ogni elica lo lascia \`e circondata temporaneamente di lipidi: \`e solo nella struttura finale che i contatti vengono fatti tra le diverse eliche e interazioni proteina-proteina
sostituiscono alcuni contatti proteina-lipide.
\subsection{Alcuni barili $\beta$ formano grandi canali}
Proteine di membrana a multipassaggio con i segmenti transmembrana ordinati in $\beta$-barili sono rigidi e tengono a cristallizzarsi quando isolati. Sono abbondanti nella membrana 
esterna dei batteri, mitocondri e cloroplasti. Alcuni formano pori con canali pieni d'acqua che permettono il passaggio di piccole molecole idrofiliche selezionate. Molti barili di tali
porine sono formate da un $\beta$-foglietto a $16$-filamenti antiapralleli in una struttura cilindrica. Le catene amminoacide polari determinano il canale acquoso interno, mentre quelle
esterne non polari interagiscono con i lipidi. Anelli della catena polipeptidicha protrudono spesso nel lume del canale stringendolo e dandogli specificit\`a. Non tutte queste sono
proteine di trasporto: alcune funzionano come recettori o enzimi con il barile come ancora rigida. 
\subsection{Molte proteine di membrana sono glicolisate}
La maggior parte delle proteine transmembrana nelle cellule animali sono glicolisati: i residui di zucchero sono aggiunti nel lume dell'ER e degli apparati di Golgi. Per questa ragione
le catene oligosaccaridi sono presenti sempre sul lato noncitosilico della membrana. Inoltre l'ambiene del citosol riduce la probabilit\`a che legami intra o inter catena disulfidi 
(\ce{S-S}) si formano tra cisteina su tale lato. Si formano invece sul lato noncitosilico dove aiutano a stabilizzare la struttura piegata o la sua associazione con altre catene. A 
causa della glicosilizzazione delle proteine di membrana sul lato extracellulare i carboidrati coprono la superficie di tutte le cellule eucariotiche nella forma di catene oligosaccaridi
legate covalentemente a glicoproteine e glicolipidi. Si trovano anche come catene di membrane integrali proteoglicani. I proteoglicani consistono di lunghe catene polisaccaridi legate
covalentemente a un nucleo proteico e si trovano al di fuori della cellula come parte della matrice cellulare e in alcuni casi il nucleo proteico si estende attraverso il bistrato o vi
\`e attaccato attraverso un'ancora GPI. Questo trato di carboidrati forma zone ricche di essi dette copertura della cellula o glicocalice. Tale strato pu\`o contenere anche glicoproteine
e proteoglicani e si occupa di proteggere la cellula da danni meccanici o chimici, oltre a distanziarla da altre cellule. Le grandi possibilit\`a combinatorie degli zuccheri li rende
ottimali per i processi di riconoscimento delle cellule. Molte proteine di membrana funzionano come parte di complessi a multicomponenti. La maggior parte delle proteine non fanno 
flip-flop attraverso il bistrato ma ruotano lungo un asse perpendicolare al piano del bistrato e sono capaci di muoversi lateralmente.
\subsection{Le cellule possono confinare proteine e lipidi a domini specifici nella membrana}
La maggior parte delle cellule confina le proteine di membrana a regioni specifiche nel bistrato continuo. Questa distribuzione asimmetrica delle proteine di membrana \`e essenziale per
la funzione di certe molecole, con anche composizione lipidica diversa. Ci sono barriere create da specifici tipi di giunzioni intercellulari che mantengono la separazione delle 
molecole. Chiaramente le proteine che formano tali giunzioni non possono diffondersi lateralmente nelle membrane interagenti. I domini possono essere creati attraverso interazioni tra
proteine regolati che creano domini zattera sulla nanoscala che funzionano come segnalatori e nel trafficking della membrana. 	
\subsection{Il citoscheletro corticale d\`a alle membrane forza meccanica e vincola la diffusione delle proteine di membrana}
Un modo comune in cui una cellula vincola la mobilit\`a laterale di proteine specifiche \`e legandole insieme in assemblaggi macromolecolari. Nei globili rossi la forma biconcava \`e 
data da interazioni delle proteine derivanti da un citoscheletro, un insieme della proteina filamentosa spectrina, un bastoncello lungo, sottile e flessibile che mantiene l'integrit\`a
strutturale e la forma della membrana plasmatica. Si uniscce alla m embrana attraverso specifiche proteine con risultato una rete che copre l'intera superficie citosilica della 
membrana e che le permette di sopportare certi livelli di stress. Una rete di citoscheletro esiste nella maggior parte delle cellule. Questa rete costituisce la corteccia della cellula,
\`e ricca di filamenti di attina attaccati alla membrana. Il rimodellamento dinamico dell'actina corticale d\`a forza a molte funzioni essenziali come il movimento, endocitosi e la
formazione di strutture transienti come filopodi e lamellopodi. La corceccia contiene anche omologhi della spettrina e altre componenti. Tale rete vincola la diffusione delle proteine
in quanto essendo apposti vicini alla superficie citosilica della membrana plasmatica possono formare barriere meccaniche che ostruiscono la diffusione libere delle proteine di membrna.
Queste partizioni o corral possono essere fissi o transienti. L'impatto del confinamento dipende dall'associazione della proteina con altre e la dimensione del suo dominio 
citoplasmatico. Si pensa aiuti a concentrare i complessi di segnalazione aumentandone velocit\`a ed efficienza.
\subsection{Proteine piegatrici di membrana deformano i bistrati}
Le membrane cellulari possono assumere forme diverse controllate dinamicamente con elaborate deformazioni transienti. In molti casi la forma viene influenzata da spinte dinamiche da
strutture citoscheletriche o extracellulari. Una parte cruciale viene svolta da proteine piegatrici della membrana che controllano la curvatura locale e agiscono in tre possibili modi:
\begin{itemize}
	\item Alcune inseriscono domini proteici idrofobici o attccano ancorre lipidiche aumentando l'area di leaflets del bistato causando il piegamento. 
	\item Alcune formano impalcature rigide che deformano o stabilizzano una membrana.
	\item Alcune causano la formazione di cluster di lipidi particolare inducendo la curvatura determinata dall'area della sezione trasversale del gruppo di testa e della coda
		idrocarburica. 
\end{itemize}
Spesso diverse proteine collaborano per arrivare a una particolare curva.
