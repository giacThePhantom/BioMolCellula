\chapter{Struttura di membrana}
Le membrane cellulari sono vitali per la vita della cellula. La membrana plasmatica definisce i limiti della cellula e mantiene le differenze essenziali
tra il citosol e l'ambiente extracellulare. All'interno delle cellule eucariotiche le membrane del nucleo, il reticolo endoplasmatico, l'apparato di 
Golgi, i mitocondri e altri organelli racchiusi da membrana mantengono differenze caratteristiche per quanto riguarda i loro ocntenuti. I gradienti ionici
tra le membrane generati dall'attivit\`a di proteine di membrana possono essere utilizzati per la sintesi dell'ATP, per guidare il tasporto di soluti
specifici o per produrre e trasmettere segnali elettrici. Nella membrana plasmatica sono anche presenti proteine sensori che permettono alla cellula di 
cambiare il comportamento in base a segnali esterni. Questi recettori trasferiscono informazinoi e non molecole attraverso la membrana. Tutte le membrane
possiedono una struttura generale comune: sono un film di lipidi e proteine, mantenuti insieme da interazioni non covalenti. Sono strutture dinamiche e 
fluide e la maggior parte delle loro molecole sono libere di muoversi nel piano della membrana. Le molecole lipidiche sono ordinate in un doppio strato 
che serve come membrana impermeabile all'acqua. Le proteine di membrana attraversano il bistrato e mediano le funzioni della membrana. Nella membrana 
plasmatica alcune connettono il citoscheletro a una matrice extracellulare su cellule adiacenti. 
\section{Il bistrato lipidico}
Il bistrato lipidico fornisce la struttura base per tutte le membrane cellulari e la sua struttura \`e attribuibile alle propriet\`a delle molecole 
lipidiche che assumono tale conformazione spontaneamente. 
\subsection{Fosfogliceridi, sfingolipidi e steroli sono i lipidi pi\`u comuni nelle membrane cellulari}
Le molecole lipidiche costituiscono il $50\%$ della massa delle membrane cellulari. Tutte le molecole lipidiche nella cellula sono anfipatiche, ovvero
hanno una parte idrofilica o polare e una idrofobica o non polare. I lipidi di membrana pi\`u abbondanti sono i fosfolipidi che possiedono una testa
polare con un gruppo fosfato e due code idrocarburiche idrofobiche. Le code sono tipicamente acidi grassi e possono cambiare in lunghezza. Una coda
contiene tipicamente uno o pi\`u legami doppi cis, mentre l'altra no ognuno dei quali forma una curva nella coda. Il fosfolipide principale negli animali
\`e il fosfogliceride con un backbone a tre glicerli. Due lunghe catene di acidi grassi sono legate a atomi adiacenti del carbonio attraverso legami
estere e il terzo atomo di carbonio del glicerolo \`e legato a un gruppo fosfato che viene legato a diversi tipi di gruppo di testa. Combinando diversi
acidi grassi e gruppi di testa la cellula crea diversi fosfogliceridi come fosfatidiletanolammina, fosfatidilserina e fosfatidilcolina. Un altra classe
importante di fosfolipidi sono gli sfingolipidi che sono costriuiti dalla sfingosina, una catena acile lunga con un gruppo ammino  e due gruppo idrossili
a una fine. Nella sfingomielina la tadena a acido grasso \`e legata al gruppo ammino, il gruppo fosfocolina al gruppo idrossile terminale. Oltre ai 
fosfolipidi nel bistrato sono contenuti glicolipidi e colisterolo, i primi sono come gli sfingolipidi ma con uno zucchero invege del gruppo di testa. 
La membrana plasmatica degli eucarioti contiene grandi quantit\`a di colesterolo, uno sterolo con una struttura ad anello rigida, un gruppo idrossile 
polare e una corta catena idrocarburica e si orientano nel bistrato con il gruppo idrossilico vicino alle teste polari. 
\subsection{I fosfolipidi formano bistrati spontaneamente}
La natura anfipatica e la forma dei fosfolipidi causa la formazione spontanea di bistrati in ambiente acquoso. Quando moleocle anfipatiche sono esposte in
ambiente acqueoso si aggregano spontaneamente per seppellire le loro code idrofobiche all'interno dove sono separate dall'acqua e espongono le teste
idrofiliche a essa. In base alla loro forma possono creare micelle sferiche o foglietti a doppio strato o bistrati con le catene idrofobiche all'interno
delle teste idrofiliche. Una rottura nel bistrato causa un lato a contatto con l'acqua che essendo sfavorevole energeticamente causa i lipidi di 
riordinarsi in modo da eliminarlo. Il solo modo per bisrtati di avere questi lati \`e di formare un compartimento chiuso chiudendosi su s\`e stesso. 
\subsection{Il bistrato lipidico \`e un fluido bidimensionale}
Molecole individuali di lipidi si possono diffondere liberamente nel piano di un bistrato lipidico. Si \`e notato come le molecole lipiche migrano 
raramente tra un monostrato all'altro nel processo di \``flip flop" che accade su tempi di ore (con il colesterolo come eccezione) ma si scambiano
veolocemente tra uno stesso monolivello dando origine a una diffusione laterale rapida. I lipidi ruotano inoltre velocemente sul loro asse e hanno 
catene idrocarburiche flessibili. Il confinamento in un monolivello causa problemi per la sintesi che avviene unicamente in uno strato, risolto da
proteine di membrana dette traslocatori di fosfolipidi o flippasi che catalizzano il flip-flop rapido di fosfolipidi da un monolivello all'altro. I 
diversi bistrati non si fondono facilmente quando sospesi in acqua in quanto dovrebbero spostare le molecole legate ai gruppi di testa. In questo modo
viene mantenuta l'integrit\`a di molte membrane interne e prevenuta una loro fusione incontrollata. 
\subsection{La fluidit\`a del bistrato lipidico dipende dalla sua composizione}
La fluidit\`a della membrana cellulare deve essere precisamente regolata in quanto dalla sua viscosit\`a dipendono alcuni processi di traporto e attivit\`a enzimatiche. La fluidit\`a del
bistrato dipende dalla sua composizione etemperatura.  Esiste una temperatura detta transizione di fase in cui assume una struttura cristallina rigida o gel, pi\`u bassa rispetto pi\`u 
corte le catene idrocarburiche o pi\`u presenti i doppi legami. La lunghezza di coda minore riduce la tendenza delle catene idrocarburiche di interagire tra di loro e i doppi legami
cis producono curve in esse che rendono la loro condensazione pi\`u difficile e la membrana rimane fluida a temperature pi\`u basse. Il colesterolo modula le propriet\`a del bistrato:
quando mescolato con fosfolipidi aumenta le propriet\`a di barriera permeabile. Si inserisce in esso con il gruppo idrossile vicino alle teste polari dei fosfolipidi in 
modo che l'anello steroide rigido interagisca e immobilizzi parzialmente le regioni delle catene idrocarboniche pi\`u vicine alle teste polari. Diminuisce la motilit\`a dei primi gruppi
\ce{CH2} della catena e lo rende meno deformabile diminuiendo la permealibilit\`a di piccole molecole solubili in acqua. Non rende la membrana meno fluida in quanto ad alte 
concentraizoni previene le catene idrocarburiche di interagire e cristallizzarsi. 
\subsection{Nonostante la loro fluidit\`a i bistrati lipidici possono formare domini di composizioni diverse}
Nonostante molti lipidi e proteine di membrana siano distribuiti uniformemente e segregazione di fase a larga scala si trovi raramente specifici lipidi e proteine di membrana si 
concentrano in maniera temporanea e dinamica facilitate da interazioni da proteine che permettono la formazione di regioni di membrana specializzata. Questi cluster aiutano a creare
zattere in membrane organizzando e concentrando le proteine per trasporto o per lavorare nell'assemblaggio di proteine.
\subsection{Goccioline di lipidi sono circondate da un monostrato fosfolipidico}
La maggior parte delle cellule conservano lipidi in eccesso in goccioline di pipidi da dove possono essere recuperate. Gli adipociti sono cellule specializzate per la conservazione
di lipidi. Contengono una gocciolina gigantesca di lipidi che occupa la maggior parte del citoplasma. Gli acidi grassi possono essere liberati dalle goccioline a richiesta e esportati
ad altre cellule attraverso il flusso sanguigno. Tali goccioline conservano lipidi neutri come trigliceroli e colesteroli esteri che sono sintetizzti da enzimi nella membrana del
reticolo endoplasmatico. Sono molecole idrofobiche e si aggregano spontaneamente in goccioline tridimensionali. Sono organelli unici in quanto sono circondati da un singolo monostrato di
fosfolipidi con una grande variet\`a di proteine, enzimi coinvolti nel metabolismo dei lipidi. 
\subsection{L'asimmetria del bistrato lipidico \`e funzionalmente importante}
La composizione lipidica dei due monostrati \`e diversa: per esempio nelle cellule eritrocite del sangue rosso tutte le molecole fosfolipidiche hanno colina nel gruppo di testa nel 
monostrato esterno, mentre tutte contengono un amminoacido primario terminale nel monostrato interno. In quanto il fosfoatidilserina \`e locato nel monostrato interno si trova una
differenza in carica tra le due met\`a del bistrato. QUesta asimmetria \`e specialmente importante in convertire segnali extracellulari in intracellulari: molte proteine citosoliche
si legano a specifici gruppi di testa lipidici e vengono attivati dai segnali che cambiano la carica del monostrato interno. In altri casi gruppi di testa devono essere modificati per 
creare siti di legame delle proteinea un tempo e luogo specifico. Varie chinasi lipidiche possono aggiungere un gruppo fosfato a posizione distinte sull'anello inositolo creando siti
di legame che reclutano proteine specifiche dal citosol. SOno usati anche per convertire i segnali extracellulari in intracellulari attraverso fosfolipasi attivate da segnali 
extracellulari che rompono specifiche molecole di fosfolipidi generando frammenti che agiscoono come brevi mediatori intracellulari. Questa asimmetria viene utilizzata per riconoscere
le cellule vive da quelle morte in quanto alla morte la fosfoatidilserina si trasloca al monostrato esterno attraverso il fofsfolipide traslocatore che normalmente trasporta questo
lipide al monostrato interno \`e disattivato e una scramblasi che trasferisce i fosfolipidi non specificatamente si attiva.
\subsection{I glicolipidi si trovano sulla superficie di tutte le membrane plasmatiche eucariotiche}
I glicolipidi sono lipidi che contengono zucchero e hanno una asimmetria estrema: si trovano esclusivamente nel monostrato dalla parte opposta del citosol. Tendono ad autoassociarsi
attraverso legami idrogeno tra gli zuccheri e forze di Van der Waals rtra le catene idrocarburiche che li partiziona in fasi di zattere lipidiche. Questa disrtibuzione assimmetrica
risulta dall'addizione dello zucchero nel lumen degli apparati di GOlfi. Il compartimento in cui sono creati \`e topologicamente equivalente all'estenro della cellula. Quando sono 
portati alla membrana plasmatica gli zuccheri sono esposti alla superficie cellulare dove hanno importanti ruoli in interazioni. Si trovano anche in alcune membrane intracellulari. Il
glicolipide pi\`u complesso \`e il ganglioside che contiene oligosaccaride con pi\`u frazioni di acidi sialici che gli danno una carica negativa. La loro funzione \`e quella di 
proteggere contro condizioni dure. Quelli carichi sono importanti a causa del loro effetto elettrico che altera la concentrazione di ioni. Hanno un ruolo nei processi di riconoscimento
cellulare in cui le lectine, proteine che si legano al carbonio legate alla membrana legano ai gruppi zucchero sui glicolipidi e glicoproteine. Alcuni glicolipidi forniscono punti di 
entrata per tossine e virus.
\section{Proteine di membrana}
