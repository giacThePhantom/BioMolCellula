\chapter{Splicing}
Negli eucarioti la sequenza del DNA non corrisponde a quella dell'mRNA in quanto sono contenute sequenze rimosse dette introni. L'RNA trascritto detto trascritto primario che viene 
fatto dall'RNA polimerasi deve andare incontro a una serie di mofiche per la rimozione di introni: splicing. Ci sono altre modifiche il capping, aggiunta di un cappuccio nel primo
nucleotide nella zona $5'$ che nromalemnte \`e un adenina, l'RNA splicinge e la poliadenilazione, l'aggiunta di una coda nella zona $3'$ una votla che sono avvenute l'mRNA pu\`o 
essere tradotto o andare incontro ad altri destini. Il meccanismo di processamento dell'RNA pu\`o avvenire a livello trascrizionale o post-trascrizionale. Tutti questi meccanismi 
che vanno alla formazione della proteina attiva sta alla base dell'espressione genica. Pu\`o andare incontro a ina seie di modifiche che regolano la prudzione della proteina. Si pu\`o 
controllare la trascrizione dell'RNA, regolare le sue modifiche splicing o splicing alternativo producendo mRNA messaggero diverso che lascia il nucleo che pu\`o essere tradotto o 
degradato, post trascrizionale controllando la sua traduzione in proteina degradandolo oppure pu\`o produrreuna proteina gi\`a attiva o pu\`o essere inattiva fosforilandola o 
acetilandola o modificandola post traduzionalmente. Una volta che l'RNA messaggero \`e presente nel citoplamsa pu\`o ssere tradotto, degradato o trasportato in una zona della cellula
un meccanismo di controllo post trascrizionale che sta alla base del controllo dell'espressione di determinati geni. UN esempio di trasporto \`e quello dei mitocondri. Un altra modifica
post trascrizionale \`e l'editing che consiste in una modifica di basi dell'RNA, modifica alla base che contribuisce la variabilit\`a genetica senza andare contro il postulato della 
biologia moleoclare. In quanto l'informaizone nel DNA non viene modificata. L'editing \`e presente anche negli eucarioti e contribuisce a variare l'espressione genica. 
\section{Capping}
Il capping $5'$ \`e una delle prime modifiche tutti gli mRNA vanno incontro a capping ma anche alcuni RNA non codificanti possono andare incontro a questa modifica. Avviene subito
durante l'RNA polimerasi sta trascrivendo ed \`e attivata da fosforilazioni di specifici amminoacidi della coda dell'RNA polimerasi II, la fosforilaizone a seguito di chinasi e altri
fattori reclutati durante la trascrizione determina il richiamo di fattori legati al capping che consiste nell'aggiunta di una 7 metil guanosina al primo nucleotide normalmente A nel
trascritto primario. Questo avviene in maniera precoce durante la trascrizione (20 nucleotidi) e \`e importante per un trasporto efficiente e per la traduzione in quanto i fattori 
coinvolti nella traduzione riconoscono la struttura a CAP e promuove il processo traduzionale, importante per la stabilit\`a dell'RNA in quanto pu\`o essere degradata da enzimi che lo 
degradano da testa o coda che se protetti gli enizmi hanno meno accesso all'RNA e importante per i meccanismi di splicing. A livello chimico il cap consiste nell'aggiunta di una 7 metil
guanosina al $5'$ del primo nucleotide. Sono coinvolti tre passaggi e tre enzimi. Il primo \`e una RNA trifosfatasi che va a tagliare il fosfato $5'$ dell'RNA in crescita lasciando un
gruppo bifosfato. Successivamente la guanil transferasi che va ad attaccare un GMP guanosina monofosfato al l'estremit\`a dellRNA formando un legame $5'$ $5'$ infine interviene la 
guanina 7 metil trasnferasi che trasfersice il gruppo CH3 alla guanosina. Questo viene detto CAP $0$ in quanto l'unico metile aggiunto viene aggiunto alla guansuna. CI sono 
altre modifiche al primo nucleotide detto Cap 1, poi i Cap n che avviene negli RNA virali. Il capping non consiste solo nell'aggiunta della 7 metil guanosina ma viene aggiunta in quanto
viene riconosciuta da un complesso Cap binding complex costituito da una CBP20 e CBP80 (pesi molecolari) che si legano ad esso in quanto riconoscono la modifica e proteggono l'RNA che 
risulta pi\`u stabile in quanto protetto da proteine e viene degradato difficilmente e svolgono un ruolo importante in quanto interagiscono con i meccanismi di esporto e sono coinvolti
nel processo traduzionale. Queste proteine si legano al Cap nel nucleo e una volta che arriva nel citoplasma vengono sostituiti da fattori che prendono ruolo nel processo traduzinale
detti eIF4E, eucariotic initiation factor 4E, coinvolti nell'inizio della traduzione, imoprtanti per un'efficiente traduzione. Il Cap svolge anche un ruolo sia nel nucleo che nel 
citoplasma dove pu\`o essere riconosciuto e legato a fattori che promuovono la traduzione.
\section{Poliadenilazione}
La poliadenilazione avviene nella coda e consiste nell'aggiunta di una coda di A ripetute alla fine del trascritto il cui numero varia a seconda del destino dell'RNA per un'intensta 
traduzione intensa poliadenilazione. Avviene durante la trascrizione alla fine della $3'$ UTR una sequenza ricca in A presente a tutti gli RNA incontro a questo meccanismo detta sequenza
di poli adenilazione caratterizzata AAUAA seguita da sequenza ricca in GU. Nella trascrizione delgi eucarioti un modell prevede che un enzima taglia alla regione ricca in GU e il 
reclitamento di altre RNAasi che lo degradano l'RNA e determinano il distacco dell'RNA polimerasi. Questo \`e quello che avviene: gli enzimi riconoscono la sequenza GU. QUando 'RNA
viene tagliato in questa zona si ha la posssibilit\`a di aggiungere la coda di poliA e avviene grazie a fatori come CPSF (fattore di specificit\`a e riconosce la zona da tagliare) che 
permette l'attacco della poliA polimerasi che attacca la coda di A. I fattori CPSF (cleavege and polyadenylation specificiy facto) e CstF (stimola il taglio) e CFIm CFIIm (fattori di
taglio). CPSF rionosce AAUAA e recluta la PAP (polimerasi) che aggiunge le code di A e vengono riconosiute da una proteina RNA binding proteine PAPB PolA binding protein. La modifica
determina poi l'associazione con proteine e questa protezione rende pi\`u stabile l'RNA e ne impedisce la degradaizone. SONo importanti per il riconoscimento di proteine specifiche.
Le polyA binding proteine sono la PABPN nucleare e la PABPC citoplasmatica. Come nel caso del capping proteine che si legano nella modifica a livello nucleare e altre che si legano 
all'RNA nel citoplasma in cui l'RNA pu\`o andare incontro a degradazione e pertanto queste proteine si legano all'RNA nella coda del citoplasma. La PAP poliA polimerasi contiene 
due NLS, un dominio che lega l'RNA e due NLS. Sono state identificate una patologia legata a problemi legati alla produzione della polyA binding protein nucleare detta distrofia muscolare
oculo-faringea determinata da abbassamento delle palpebre, difficolt\`a di deglutizione, debolezza muscolare e paralisi legata all'accumolo di RNA nel nucleo e un accumulo in foci 
specifiche della poliA polimerasi legata ad una mutazione a carico della proteina che lega la coda di poliA. I foci intrappolano la PAP e l'RNA non viene pi\`u esportato in maniera 
corretta e tende ad accumulare nel nucleo. compare intorno ai 40 - 60 anni e va a colpire il sistema muscolare e neuronale (motoneuroni), interessante che una patologia colpisca questo
apparato. 
\section{Splicing}
Il fatto che l'informazione scritta nel DNA non veniva portata tal quale negli mRNA maturi si deve a Roberts e Sharp che dimostrarono che i geni contengono delle infomrazioni e sequenze
che non vengono trovate negli mRNA che codificano le proteine. Queste sequenze che vengono trascritte ma rimosse sono gli introni e si distinguono da quelle trascritte e mantenute 
nell'mRNA e vengono detti estroni. Sono stati considerati come parte del junk RNA ma all'unterno degli introni sono contenuti altri geni che vengono trascritti con ruolo strutturale e
regolatorio per la produzione dei microRNA o dei lncRNA. Gli introni se presentano un reading frame, sequenza codifcata da specifici amminoacidi possono essere fonti della variabilit\`a
genica. Molte delle mutaizoni che avvengono nel genoma possono andare a colpire gli introni che non vanno a interferire con la funzionalit\`a della proteina. La reazione di splicing 
\`e favorita in quanto si ha un guadagno netto di entropia in quanto si creano diverse moleocle di RNA da una e gli introni rimossi vengono degradati e lo spostamento verso la formazione
di un RNA messaggero splicato \`e favorita. I meccanismi molecolari che la rendono possibili sono due reazioni di transesterificazione: nella prima si ha un attacco nucleofilo del gruppo
OH di un'adenina presente in un sito particolare dell'introne che \`e diretto al gruppo fosfato $5'$ della guanina del nucleotide GU e questo attacco nucleofilo determina la rottura 
del legame fosfodiesterico e la formazione di una struttura a cappio. Il secondo attacco nucleofilo \`e portato dalla base G nei confronti di un'altra G presente nella doppietta AG e 
l'attacco nucleofilo determina la rottura del legame fosfodiesterico. IN alcunni splicing l'attacco nucleofilo \`e portato da un'adenina esterna. Questa reazione di transesterifacione
che portano la rottura del legame fosfodiesterico in due punto. Le regioni di GU e AG sono le sequenze conservate nei punti di passaggio tra esone e introne. La sequnza AG \`e la sequenza
accettore e la GU \`e la sequenza donatore. Lo studio dello splicing \`e ocmplesso in quanto ci sono tantissimi fattori per la regolaizone dello splicing, fattori che lo promuovono lo 
splicing e la rimpaiozni degli introni e fattori che inibisocono lo splicing e mantengono le sequenze introniche all'interno dell'mRNA maturao. QUesti fattori son regolati e lo splicing
ha una componente legata al tipo cellulare in cui avviene. Ci sono proteine che legano l'RNA e svolgono ruolo regolatorio in questo contesto. Lo splicing avviene nel nucleo in zone 
dette regioni detti nuclear speckles dove avviene il meccanismo di splicing. Lo splicing pu\`o creare un enorme variabilit\`a l'RNA primario che contiene le sequenze introniche possiede
tanti esoni i quali possono andare incontro a splicing alternativi. Alcune sequenze isoniche e altre mantenute. Alla fine da un unico premRNA si generano un gran numero di trascritti
alternativi. Da un'unica molecola si ottiene un grande aumento del disordine legato al numero di molecole. Circa il $90\%$ dei trascrittti della cellula vanno incontro a splicing 
alternatico che consiste nel mantenimento o rimozione di sequenze introniche o esoniche. QUesto meccanismo molto regolato presenta sequenze com GU e AG mantenute non modificate in quanto
altrimenti lo splicing pu\`o avvenire in maniera aberrante creando problemi in alcune patologie in cui modifiche degli esoni causano la formazione di uno splicing aberrante con la 
formazione di un'RNA messaggero e proteina non funzionale. La mutazione nella Distrofia muscolare di DUchenne avviene nell'esone 31 che causa uno splicing aberrante e la rimozione 
dell'esone 31 con la formazione di una proteina tronca. Altre mutazioni coinvolgono i siti GA o GU causando un non splicing della regione e inseirmento di un introne con al creazione
di una proteina tronca o aberrante. Ci sono cinque classi di introni: autocatalitici di gruupo I , di gruppo II introni dei tRNA, introni degli archei, introni spliceosomali del pre-mRNA 
nucleare. La scoperta dell'attivit\`a autocatalita degli introni di gruppo I la si deve a Chech e Sidney Altman che stabano studiando la tetrahymena thermophila e videro che la capacit\`a
di questi introni di andare incontro a splicing rompendo e formando legami in presenza di determinati fattori, in particolare magnesio e ATP un'RNA era in grado di produrre versioni
pi\`u corte e di tagliarsi riformando legami. Fino ad allora sembrava che l'RNA sembrasse una molecola di passaggio e grazie a questi esperimenti si \`e vsito che l'RNA possiedeva 
un'attivit\`a catalitica.
\section{Introni catalitici di gruppo 1}
Questi introni effettuano l'autosplicing grazie ad una serie di passaggi che comportano alla cui base si trova l'attacco nucleofilo e reazione di transesterificazione. L'attacco 
nucleofilo viene portato da una guanosina esterna usata come fattore. Il tutto determina un ripiegamento della sequenza intronica e la guanosina porta un attacco nucleofilo alla prima
nucloetide dell'introne che determina la rottura del legame fosfodiesterico, l'altra guanina presente nella base dell'altro introne porta il secondo attacco nucleofileo, si forma un 
legame, i due esoni si legano e si libera l'introne che pu\`o andare incontro a degrdazione. La struttura secondaria e primaria sono indispensabili per l'attivit\`a catalitica di questi
introni. Alla base si trova una reazione di transesterificazione con l'attacco nucleofilo portato da una guanosina esterna.
\section{Introni autocatalici del gruppo 2}
Parzialmente presenti nei genomi di alcuni mitocondri e plasmidi delle pante. Lo splicing avviene in maniera simile solo che l'attacco nucleofilo \`e portato da un'adenina presente 
all'interno dell'introne, simile al gruppo I. L'attacco nucleofilo viene portato dalla A al sito accettore e si ha la formazione di una struttura a cappio e il secondo attacco nucleofilo
simile a prima, ligeazione dei due esoni, rilascio dell'introne a cappio e degradazione. i tRNA vanno incontro a splicing e vengono trascritti come pretRNA e la rimozione di un introne.
\section{Introni spliceosomali del pre-mRNA nucleare}
IL meccanismo con cui la maggior parte delle nostre cellule regola il processo di splicing. Questo meccanismo \`e portato avanti dallo spliceosoma composto da proteine che riconoscono
la coppia di siti di splicing tra siti simili i siti di splicing sono regioni conservate GU e AG e la macchina \`e in grado di riconoscere i siti donatori e accettori all'interno della
sequenza e permette l'avvvicinamento e il posizionamento nel sito catalitico per far avvenire in modo effficiente le due reazioni di transterifcazione. Facilita l'avvicinamento dei 
due esoni, ripiega l'introne avvicinando l'introne dell'attacco nucleofilo ai siti donatori e accettori. Nel caso degli eucarioti gli introni hanno lunghezze notevoli (2000 paia di basi)
lo spliceosoma facilita l'avvicinamento dei nucleotidi per portare l'attacco nucleofilo in maniera efficiente. Si veda nel dettaglio cosa compone le componenti dello spliceosoma, sono 
proteine e RNA, in particolare quelli snRNA, piccola con ruolo particolare di regolare e permettere uno splicing corretto, formano dei complessi ribonucloproteici detti SNRNP questi small
nuclear RNA sono di 5 tipi: U1, U2, U4, U5 e U6 in quanto U3 non \`e componente dello splicing. Questi formano complessi con proteine specifiche e tra queste quella presente \`e detta
proteine SN che formano un anello importante in quanto permette l'assunzione di strutture secondarie stabili importanti durante il processo di splicing. Si deve considerare due siti
importanti: il sito GU o donatore e il sito AG o sito accettore. Il primo \`e quello che si trova nell'introne $5'$ mentre l'accettore $3'$ un altro elemento importante \`e il nucleotide
A che si posiziona in una regione specifica detta branching point o punto di ramificazione seguita da una serie di nucleotidi conservati importanti in quanto riconosciuti dai fattori
coinvolti nello splicing. Lo splicing mediato dallo spliceosoma inizia con appaiamento di U1 con il sito donatore GU, contemporaneamente si lega BBP branching point binding protein che
si lega con il sito di branchin con l'adenina origine dell'attacco nucleofilo, poi attacco del dimero di U2 una a 65  l'altra a 35 kilodalton e una si lega nel punto di passaggio esone
sito accettore e l'altra si lega al tratto a valle del branching point. L'interazione di U1 con il sito donatore \`e mediata dalla componente a RNA presente nella SNRNP e sottoplina 
l'importanza della conservazione dei nucleotidi donatore e di alcuni nucleotidi a valle, mutazioni di questi siti determina un allinamento non corretto che pu\`o bloccare lo splcign. 
La formazione di questo complesso nell'introne si chiama complesso precoce E (early), il passaggio dal complesso early al complesso A prevede l'inserimento di U2 che va a rimuovere
BBP e comporta un'appaiamento specifico di basi tra U2 e la sequenza nel branching point. Complesso A. Il complesso B1 si forma al reclutamento degli altri tre fattori U4 U6 e U5 che 
\`e legato alla rimozione U2AF65-35 e si forma il complesso B1 in cui l'introne viene piegato avvicinando branching point del punto donatore. Si passa al complesso B2 che si forma
quando viene scalzato U1, U6 interagisce con l'esone $5'$ con una sequenza complementare con la regione al $5'$. Si passa a B* con la rimozione di U4 e la formazione del complesso C1 
\`e il complesso in cui avviene l'attacco nucleofilo portato da A nei confronti del sito donatore e la rottura del legame fosfodiesterico con la formazione del complesso C1, l'introne
forma il cappio e rimane associato grazie a U2, U6 e U5 a questo punto il secondo attacco nucleofilo a carico del sito accettore, legame dei due esoni, rilascio dell'introne a cappio e 
sua degradazione. La componente U4 di fatto \`e una componente che impedisce il legame e si interpone tra U5 e U2 e fintanto che si trova U4 kl primo attacco nucleofilo non avviene, 
questo avviene solo quando U4 viene rimosso. 
\section{Minigeni}
\`E una tecnica utilizzata per studiare alterazioni nello splicing. Data la complessit\`a del fenomeno \`e utile. Si utilizza per studiare se si trovano alterazioni nello splicing si 
coinvolgono minigeni, si fa esprimere alla cellula un plasmide che trascrive per un minigene, un gene artificiale, una parte di un gene presente nella cellula semplificato che in questo
caso \`e costituito da una serie alternata di 3 isoni e introni. COn questo meccanismo \`e possibile studiare se lo splciing del minigene avviene in maniera corretta attraverso la PCR. 
Contando poi il numero di nucleotidi dei frammenti amplificati. La proteina Smn (survival motor neuron) si tratta di una proteina la cui mutazione causa l'atrofia muscolare spinale e
si tratta di una proteine di 40 kilodalton e quando \`e mutata causa questa patologia frequente e presente in 1 su 10000 nati dovuta a mutazioni dovute al gene Smn1 e se ne trova Smn2
di cui si trovano due o tre copie, importante ai fini della patologia. Modifiche di Smn1 causa patologia che causa una degenerazione dei motoneuroni. Sezionando il midollo spinali si 
trovano le corna dorsali e quelle ventrali che contengono il corpo cellulare dei motoneuroni che comandano la contrazione muscolare. Questa patologia colpisce i motoneuroni nella zona
alta del midollo. Come conseguenza degenerano dei neuroni e determinano l'atrofia dei muscoli in quanto permettono al muscolo di avere una certa quantit\`a di fibre e la mancanza di 
segnali causa una paralisi conseguenza vita tragica. La proteina Smn fa parte del complesso che si lega all'importina SN delle SNRNP e la formazione del complesso con la proteina SMN \`e
importante nella maturazione dello spliceosoma con ricadute sullo splicing. Questa proteina localizza nel nucleo all'interno di alcune strutture e anche nel citoplasma. \`E una proteina
che va avanti e indietro dal citoplasma con NLS e NES. Oltre a questo contiene domini che legano RNA e altri importanti con le interazioni con altre proteine per formare il complesso 
dello spliceosoma o fattori importanti per il traporto da nucleo a citoplasma e viceversa. Sue mutazioni vanno a colpire i motoneuroni probabilmente in quanto sono molto pi\`u sensibili
ad alterazioni nello splicing. Fatto sta che uno si aspetterebbe che mutaizoni in una proteina cosi\`i critica dovrebbe avere ricadute generali ma in realt\`a va a colpire soprattuto i 
motoneuroni. Tra le funzioni ci sono l'esporto nucleare degli snRNA, assemblaggio nucleare del complesso Sm, metilazione del cappuccio $5'$ e la formazione delle gemins, complessi 
che vengono reclutate nel complesso dello spliceosoma costituite da diverse proteine indicate con Gemin1, 2, 3, 4, 5, 6, 7 interferisce con queste in quanto l'assenza di Smn determina
l'assenza di gemin2, 4, 5 presente, 6, 7 assenza. L'assenza di Smn interferisce con  l'espressione di alcune queste proteine, pu\`o farlo in due modi: o agendo a livello trascrizionale 
bloccando la trascrizione per i geni che codificano queste proteine o a livello post traduzionale interferendo con la stabilit\`a di queste proteine (che sembra il motivo) in quanto 
molto probabilmente non vengono reclutate in un solo complesso e sono degradate pi\`u velocemente. Pazienti con mmutazione di Smn1 ma con pi\`u copie del gene Smn2 hanno sintomi della
malattia pi\`u lievi rispetto a pazienti con bassi libelli di espressione di Smn2 si pensa che avere un gene che trascrive ad alti livelli aiuta questi pazienti, si cerca di utilizzare
questa strada per rallentare la patologia oversepreimento il gene per compensare difetti nel gene Smn1, che disolito cadono nell'esone 7 che in molti casi comporrtano la formazine di un
RNA aberrante o la proteina non viene fatta. 
La formazione degli snRNA che vengono trascritti come precursori e pur non venendo 
tradotti e possono essere modificati (capping) e ha attivit\`a regolatoria, questo in quanto \`e importante per reclutare fattori per mantenerne la stabilit\`a e per interagire con i 
fattori di esporto. Una volta che lo snRNA assembla con fattori nella stabilit\`a dell'esporto, viene esportato nel citoplasma dove lega con Sm e le gemins e si forma un complesso che
va incontro a metilazione, interagisce con le importine, viene importato nel nucleo nei corpi di Caja dove forma le SNRNP matura. Le gemins e le proteine SN permettono l'assemblamento 
corretto dei fattori coinvolti dello splicing che avviene in parte nel nucleo ma per la maggior parte nel citoplasma. Va incontro a modifiche per gli RNA sottolineando l'importanza dello
regolazione dello splicing che pu\`o avvenire in molti modi: splicing alternativo: da un'unico RNA alternativo si possono formare n RNA messaggeri maturi che contengono introni, possono
perdere esoni. Pu\`o avvenire un exon skipping in cui si rimuove un esone. Ci sono casi di intron retention, retenzione dell'introne che viene mantenuto nell'mRNA maturo. Lo splicing 
pu\`o creare un RNA che pu\`o rimuovere porzioni di RNA ed andare incontro ad una sintesi proteica diversa creando proteine pi\`u corte usando siti di traduzione diversi. La complessit\`a
\`e legata a fattori che si legano a RNA che possono promuovere l'eliminazione di un esone o il mantenimento di un introne o viceversa. Queste RNA binding protein alcune sono nella
famiglia delle hnRNP, altre sono SM. Lo splicing alternativo pu\`o dare origine a proteine con funzioni diverse. Uno di questi \`e quello della chinasi CamKIIdelta, una chinasi che 
dipende dai livelli di calcio un messaggero secondario che aumenta la concentrazione nella cellula in risposta a un segnale esterno che arriva ad aprire dei canali di calcio sul
reticolo endoplasmatico liscio e in questo caso attiva altre proteine come la CaMKII$\delta$ che fosforila nel momento in cui viene legata al calcio e va a fosforilare altre proteine. 
A carico di questa chinasi la rimozione dell'esone 14 che contiene un NLS che rimangono citoplasmatiche la rimozinoe dell'esone che contiene una sequenza che porta la proteina ad entrare
nel nucleo in questo caso da origine ad una proteina citoplasmatica e in questi due casi sono chinasi che fosforilano proteine nucleari o citoplasmatiche. Lo splicing non modifica
l'attivit\`a chinasica ma la localizzazione intracellulare e una sua funzione. Vi sono altri esempi di splicing alternativi che influiscono sul comportamento, geni che vengono modificati
e vanno incontro a splicing nel sistema nervoso con cnseguenze comportamentali come mutanti fruitless della drosophila la cui mutaizone a carico dello splicing di un gene comporta un
comportamento aberrante: nella riproduzione nromail il maschio agita l'ala facendo un suono che indica la femmina che vuole accoppiarsi, questo deriva da un gene presente nel maschio e
quando questo va incontro a uno splicing aberrante causa il comportamento aberrante durante la fase riproduttiva: in questi mutanti fruitless i maschi si incatenano tra di loro cercando 
di accoppiarsi. Questi meccanismi stanno alla base della diversit\`a dell'espressione genica: da un unico RNA eterogeneo che va incontro a splicing alternativo si creano proteine 
diverse. 
\section{RNA editing}
Studiato per la prima volta nei tripanosoma brucei e consiste in modifiche di basi a carico dell'RNA, modifica post trascrizionale che consistono o in una conversione di una base di
un'altra o inserzione e delezione di nucletotidi. \`E un meccanismo utilizzato da organismi dove la variabilit\`a genetica non \`e portata avanti dallo splicing e modificando una 
base in un'altra si pu\`o modificare un amminoacido e creare una proteina che funziona in un modo diverso. Si p\`o introdurre un cambio che determina una proteina tronca dando 
variabilit\`a alla sua funzione. L'editing non va contro il paradigma della biologia molecolare perch\`e \'e una modifica post trascrizionale e non viene cambiata l'informazione nel DNA. 
\`E stato identificato nel tripanosoma brucei andando che produceva degli RNA la cui sequenza era diversa dalla sequenza presente nel loro DNA che riguardava aggiunta o rimozione di 
nucleotidi, in particolare nucleotidi. Il complesso coinvolto \`e l'editosoma. L'editing avviene grazie alla presenza di RNA che vengono trascritti dall'organismo detti RNA guida con
una sequenza parzialmente complementare all'mRNA e si lega ad esso con delle zone con un perfetto allineamento e vi sono regioni in cui questo appaiamento non avviene. Nel primo 
nucleotide spaiato avviene il primo passo del processo di editing iniziato da un'endonucleasi che taglia in corrispondenza della regione non appaiata. A questo punto interviene un'enzima
che aggiunge paia di basi complementar, dopo la sequenz asi appaia attraverso una ligasi. L'enzima che aggiunge le basi \`e una TUTasi. Alla fine si trova un RNA messaggero con nucleotidi
in pi\`u che modifica la proteina corrispondente. L'aggiunta di queste basi va a modificare la proteina risultante. I tripanosomi sono eucarioti e possono infettare l'uomo, parassiti 
ma a livello biologico sono eucarioti protozoi come l'ameba, organismi unicellulari con flagelli e motilit\`a. 
\subsection{Editosoma}
Nei mammiferi l'editing avviene in maniera diversa e non comporta inserimento e delezine ma modifiche di basi: sono due i meccanismi la trasformazione di adenosina in inosina (A -> I
editing), e dalla citosina si trasfomra in uracile (C -> U editing) e non comporta l'inserimento o delezione che possono causare proteine aberranti e pi\`u corte e nei mammiferi 
l'editing si basa su una modifica  acarico di un nucleotide che comporta un riconscimento a livello di codone diverso. La trasformazione da adenosina a inosina viene riconosciuto come
C complementare anche se l'inosina pu\`o avere appaiamenti anche con G (inosina maggiore variaiblit\`a) pu\`o comportare l'inserimento di un amminoacido diverso. La modifica A -> I si
trova negli mRNA che codificano per dei recettori dalle cellule nervose. C -> U invece spesso porta alla formazione di un codone di stop e una proteina tronca. La modifica pu\`o 
portare una diversa permeabilit\`a ad un recettore. Modificando gli amminoacidi si pu\`o modificare il diametro del passaggio e lo ione che prima veniva bloccato pu\`o passare, variazini
di permeabilit\`a e selettivit\`a o il canale si apre o si chiude pi\`u o meno velocemente. La struttura dell'inosina \`e uguale all'adenosina con un gruppo amminico in meno. L'enzima
coinvolto in A -> I editing \`e una proteina ADAR che lega l'RNA e lega l'RNA a doppio strand riconoscendo strutture a forcina a doppia elica dell'mRNA e dopo aver riconosciuto 
queste strutture riconosce una loro adenosina e la modifica andando ad attuare una deamminazione rimuovendo il gruppo amminico trasformando l'adenosina in inosina. Un esempio che avviene
all'interno di un recettore per il glutammato la modifica ad opera di ADAR determian un cambio amminoacidico e propriet\`a diverse. L'editing pu\`o avvenire a pi\`u adenine con 
conseguente modifica dell'amminoacido (AUA che diventa AUI da lisina a metionina, AUU a IUU da isoleucina a valina). Per l'editing C -> U si prende come esempio l'editing per 
l'apolipoproteina B o ApoB, una proteina plasmatica con ruolo crucial nel trasporto, assemblaggio e metabolismo delle lipoproteine plasmatiche che sequestrano il colesterolo. Esistono
due forme di ApoB una di 100 kilodalton (ApoB100) prodottta dalle cellule epatiche e componente delle lipoproteine a bassa densit\`a (LDL) che sono quelle che misurano il colesterolo 
cattivo le altre sono le lipoproteine a bassissima densit\`a per il colesterolo buono. L'isoforma da ApoB48 viene prodotta dalle cellule dell'intestino. La 100 viene riconociuta da
un recettore con cui si lega e viene internalizzata, mentre la 48 viene internalizzata da altra via. Vengono prodotte a partire dallo stesso RNA che va incontro ad editing che 
quando avviene si forma l'ApoB48. Questo perch\`e il gene da 43 kilobasi con 29 esoni e l'isoforma viene pordotta a seguito di un editing di una citosina in CAA che diventa pertanto UAA
che causa dalla codifica di un amminoacido a una tripletta di stop. Nel fegato senza editing l'RNA non viene modificato e genera ApoB100, nell'interstino avviene l'editing e si 
trasforma in UAA e lo stop codon prematuro genera ApoB48. Il meccanismo di editing avviene graie all'APOBEC1, il fattore che compie e porta avanti la deamminazione o la trasformazione
da C a U accoppiato al fattore ACF fattore di complementazione. Il primo compie la deamminazione riconoscendo la tripletta ma non in maniera specifica che viene data dal fattore di 
complementazione. Non necessariamente tutti gli stop codon prematuri danno origine alle proteine tronche in quanto non \`e detto che l'introduzione di uno stop codon prematuro dia
origine a una proteina trocna in quanto le tronche possono essere pi\`u instabili e prone ad essere degradate in quanto non interagiscono con il complesso e una proteina che noo fa 
parte di un complesso pu\`o essere libera e subire maggiore degradazione. Le cellule hanno inoltre in meccansimo di controllo che quando trova uno stop codon prematuro non canonico l'RNA
viene degradato per eliminare gli RNA con stop codon prematuri. Quello visto non viene degradato da NMD  (non sens medated decay) che viene attivata in condizioni particolari, quando 
lo stop codon si trova all'interno di un esone. Sono presenti nell'RNA delle proteine legate all'RNA fino alla sua traduzione e costituiscno l'exon junction complex (EJC) e un RNA 
maturo \`e fatto da RNA di esoni separati dagli EJC che marcano le zone che sono andate incontro a splicing. Normalemnte uno stop codon normale si trova alla fine prima della $3'$ UTR
che non va incontro a splicing uno stop codon canonico ha a monte un EJC e a valle nulla. Uno prematuro si trova invece tra EJC. Il meccanismo di controllo che la celliul ha per 
riconoscere stop codon prematuri riconosce come prematuro uno stop codon compreso tra due EJC. In questo caso la macchina lo riconosce e lo degrada. La macchina non lo riconosce 
se si trova molto vicino all'EJC. Le proteine ADAR che deamminano l'adenosina in inosina, sono una famiglia di quattro componenti (tre, una uno splicing alternativo) e legano l'RNA a
doppio strand dsRBM identifica un dominio che lega un RNA a doppio strand (double strand RNA binding motif / domain) ADAR ne ha due e un dominio deamminasico. Negli umani si esprimono
tre ADAR1, 2, 3 e la prima va incontro a uno splicing alternatico con due isoforme a 150 e 110 kilodalton possono avere un dominio deamminasico, due o tre dsRBM e sono distinte dal 
numero di domini dsRBD e dai domini terminali che possono avere per legare RNA e altri con funzione non nota. I due domini interagiscono con strututra a doppio strand permettendo alla
proteina di legarsi all'RNA e di avvicinare il dominio catalitico alla base che va incontro ad editing. Le ADAR possono fare cambi da polare a non polare e portare delle modifiche
che vanno a colpire i siti critici nello splicing. Eliminando il branching point o creando i siti di splice a $5'$ o a $3'$. Possono determinare pertanto ritenzione degli introni, nel
sito accettore lo distrugge il sito di splicing al $3'$. Le conseguenze possono essere diverse. a livello di splicing o di amminaocidi. POssono anche modificare i microRNA che non 
codificano ma si legano agli mRNA e ne bloccano la traduzione, modificandoli si pu\`o modificarne la sequenza e l'RNA messaggero a cui questo si lega.
\section{Essay per editing}
Questi sistema serve per studiare fattori e proteine coinvolti nell'editing dell'RNA \`e analogo a quello dello splicing con minigene, plasmide all'interno del gene che viene spresso
nella cellula. Si tratta di un costrutto trascritto grazie a un promotore e questo costrutto trascrive un RNA costituito da tre proteine di fusione e il trascritto contiene due
potenizali siti di trduzione e tra i due si trova un uag che consiste una seuqneza di stop e la A \`e quella che va incontro a editing. Se non c'\`e editing il trascritto viene 
fatto. la traduzione inizia e si blocca. Come secondo si to Aug traduce una parte del trascritto che viene tradotta come fosfatasi alcalica e una GFP mettendo queste cellule si 
vede verde e con un anticorpo specifico avviene una reazione e non si vede la regione prima non si vede. I blocci HA sono dei tago come la EGFP, sequenze amminoacidice artificiali 
visualizzabili da un anticorpo. 
