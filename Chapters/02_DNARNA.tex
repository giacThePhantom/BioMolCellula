\chapter{DNA e RNA}
\section{Struttura del DNA}
Il DNA si trova nel nucleo delle cellule eucarioti e ed ancorato alla membrana delle cellule procarioti.
\`E costituita da due catene polinucleotide disposte antiparallelamente. 
\`E pertanto un polimero costituito da nucleotidi.

	\subsection{Componenti}

		\subsubsection{Monomeri}

			\paragraph{Nucleosidi}
			I nucleosidi sono formati dalla base azotata e uno zucchero legate da un legame $\beta$-$N$-glicosidico tra il carbonio anomerico $C1$ del ribosio e $N9$ di purine o $N1$ delle pirimidine.
			Sono pertanto costituiti da uno zucchero come $2'$-deossiribosio.

				\subparagraph{Orientamento del legame}
				Il legame pu\`o essere orientato in:
				\begin{multicols}{2}
					\begin{itemize}
						\item \emph{Syn}.
						\item \emph{Anti}, l'orientamento principale.
					\end{itemize}
				\end{multicols}

			\paragraph{Nucleotidi}
			I nucletidi sono esteri fosfato dei nucleosidi: il sito di fosforilazione \`e spesso l'ossidrile al $5'$ del ribosio.
			Possono essere mono, di e trifosfati.
			I gruppi fosfati sono legati tra di loro da legame anidridico stabile e ad alta energia.
			
				\subparagraph{Funzioni}
				I nucleotidi hanno funzione di:
				\begin{multicols}{2}
					\begin{itemize}
						\item Unit\`a strutturale degli acidi nucleici.
						\item Deposito di energia delle reazioni di trasferimento di fosfato come \emph{ATP} e \emph{GTP}.
						\item Mediatore di processi cellulari \emph{cAMP}.
						\item Parte di coenzimi: come \emph{NADH}, \emph{FAD}, \emph{CoA}.
					\end{itemize}
				\end{multicols}

				\subparagraph{Gruppo idrossile}
				Il gruppo idrossile in \emph{$3'$-OH} \`e importante per la polimerizzazione.
				
				\subparagraph{Base azotata}
				I nucleotidi nel DNA contengono quattro basi diverse:
				\begin{multicols}{2}
					\begin{itemize}
						\item Timina, una pirimidina con un unico anello.
						\item Citosina, una pirimidina con un unico anello.
						\item Guanina, una purina con due anelli.
						\item Adenina, una purina con due anelli.
					\end{itemize}
				\end{multicols}

		\subsubsection{Polimerizzazione}
		La posizione del gruppo idrossile in \emph{$3'$-OH} \`e importante in quanto \`e il gruppo coinvolte nella creazione di legami fosfodiesterici per la polimerizzazione.
		Questi legami infatti nascono tra il \emph{$3'$-OH} e il gruppo fosfato in $5'$.
		Si forma pertanto una catena costituita da una successione di nucleotidi in cui variano le basi azotate.
		
	\subsection{Polarit\`a}
	Essendo il polimero lineare possiede un inizio e una fine:
	\begin{multicols}{2}
		\begin{itemize}
			\item Estremit\`a $5'$ con fosfato libero viene tenuta per convenzione in alto a sinistra.
			\item Estremit\`a $3'$ con \emph{-OH} libero in basso a sinistra.
		\end{itemize}
	\end{multicols}
	La polarit\`a di questa direzionalit\`a e nella doppia elica si nota un autoparallelismo con due filamenti con polarit\`a opposta.

	\subsection{Interazioni tra basi}
	Le basi azotate nelle due catene della doppia elica interagiscono tra di loro attraverso legami a idrogeno:
	\begin{multicols}{2}
		\begin{itemize}
			\item $A$-$T$ con due legami a idrogeno.
			\item $G$-$C$ con tre legami a idrogeno.
		\end{itemize}
	\end{multicols}

		\subsubsection{Movimento tra basi appaiate}

			\paragraph{Twist}
			Nel twist le basi possono roteare l'una sull'altra.

			\paragraph{Roll}
			Nel roll le basi ruotano lungo l'asse maggiore.

			\paragraph{Tilt}
			Nel tilt le basi ruotano lungo l'asse minore.
			
		\subsubsection{Chargaff}
		Chargaff osservando DNA proveniente da vari organismi e nota come i residui di $A$ e $T$ e $C$ e $G$ sono in numero uguale a due a due, provando la complementariet\`a tra le basi. 
		Nota inoltre come il rapporto tra le basi non complementari rimane uguale in ogni tessuto di ogni organismo ma \`e un numero univoco che differisce tra gli organismi.
		Si nota come organismi che sopravvivono a temperature pi\`u elevate possiedono una percentuale maggiore di $C$ e $G$. 
		Questo avviene in quanto $C$ e $G$ formano un legame a idrogeno in pi\`u rispetto ad $A$ e $T$ rendendo la macromolecola pi\`u resistente alla denaturazione. 

		\subsubsection{Temperatura di melting}
		La temperatura di melting viene utilizzata per determinare la percentuale di $C$ e $G$ all'interno del DNA in quanto pi\`u sono presenti pi\`u \`e difficile denaturarlo. 
		Si usa lo spettrofotometro per osservare una provetta annotando la sua assorbanza che cambia in base allo stato di denaturazione del DNA (a doppia elica assorbe a $260\si{\nano\metre}$). 
		Grazie alla variazione del valore di assorbanza si riesce a determinare la temperatura in cui met\`a del DNA viene denaturato o temperatura di melting. 

	\subsection{Caratteristiche}
	Il diametro dell'elica \`e di $2\si{nm}$ e avviene un giro ogni $10$ nuceotidi.
	Si nota un solco maggiore di $22\si{\angstrom}$ e uno minore di $12\si{\angstrom}$.
	Questa differenza \`e dovuta ai diversi angoli tra i due legami glicosidici.
	L'angolo meno ampio genera il solco minore.
	Il solco maggiore fornisce pi\`u informazioni di tipo chimico in quanto contiene pi\`u gruppi donatori ed accettori di protoni.

		\subsubsection{Lettura del codice genetico}
		La lettura del codice genetico da parte delle proteine si basa pertanto su:
		\begin{multicols}{2}
			\begin{itemize}
				\item Accettore di legame \emph{HB} ($A$).
				\item Donatore di legame \emph{HB} ($D$).
				\item Gruppo Metilico ($M$).
				\item Idrogeno non polare ($H$).
			\end{itemize}
		\end{multicols}
		Queste propriet\`a permettono alle proteine di riconoscere senza ambiguit\`a specifiche sequenze di DNA senza che sia necessario leggere la doppia elica.
	\subsection{Tipologie di strutture}

		\subsubsection{Struttura primaria}
		Si intende per struttura primaria la sequenza di nucleotidi che costituiscono la molecola.

		\subsubsection{Struttura secondaria}
		Si intende per struttura secondaria la configurazione tridimensionale stabile di una molecola di DNA.
		La molecola di DNA \`e formata da $2$ catene di nucleotidi appaiate secondo la regola della complementariet\`a.
		L'elica \`e destrorsa con due catene antiparallele.
		Le basi si trovano all'interno dell'elica allineate ad angolo retto.

			\paragraph{Impalamento delle basi}
			La disposizione delle basi all'interno della doppia elica \`e generalmente favorita.
			Molecole contenenti anelli aromatici tendono a disporsi impilate spontaneamente in quanto idrofobiche.
			Le basi si impilano disponendosi parallelamente riducendo in questo modo il contatto con l'acqua e stabilizzando la struttura.

		\subsubsection{Basi di Hoogsteen}
		Le basi di Hoogsteen si formano tra basi non complanari con rotazione dell'anello pirimidinico e meno stabili.

		\subsubsection{Conformazioni particolari}
			
			\paragraph{Forma $\mathbf{A}$}
			La forma $A$ si forma a causa della riduzione della quantit\`a d'acqua, presenta una struttura tarchiata con $11$ basi per giro e una scanalatura principale pi\`u profonda.

			\paragraph{Forma $\mathbf{B}$}
			La forma $B$ \`e la forma canonica cristallizzata da Franklin con $10$ paia di basi per giro.

			\paragraph{Forma $\mathbf{Z}$}
			La forma $Z$ si forma in zone particolari ricche di $C$ e $G$ con $12$ paia di basi per giro.
			La scanalatura principale \`e poco profonda e questo rende le basi pi\`u prone alla metilazione e pertanto presente nelle zone promotrici dei geni.

			\paragraph{Tripla elica}
			Il DNA pu\`o assumere strutture a tripla elica che si formano dove esiste una sequenza particolare. 
			Queste strutture influiscono sul processo di duplicazione e trascrizione del DNA e molto spesso rendono il DNA ancora pi\`u stabile e la capacit\`a delle due catene di separarsi \`e compromessa, impedendo l'arrivo dei fattori necessari ai processi. 
			\`E ricca di A e T ed \`e resa possibile dalle interazioni di Hogsteen. 

			\paragraph{Struttura cruciforme}
			La struttura cruciforme, a stem loop o a forcina si forma dove si trovano sequenze palindromiche. 
			\`E caratteristica sia di DNA ed RNA dove sono presenti sequenze ripetute seguite da una sequenza invertita. 
			Possono essere deleterie per i processi di duplicazione e trascrizione in quanto conferiscono alla catena un'elevata stabilit\`a impedendo la separazione delle due catene. 

			\paragraph{Struttura a forcina}
			La struttura a forcina o hairpin \`e costituita da ripiegamenti a doppia elica con basi accoppiate e zone a loop dove non si trova complementariet\`a.

			\paragraph{DNA curvo}
			Coppie di vasi adiacenti non perfettamente parallele portano a piccoli ripiegamenti.
			Se le perturbazioni sono casuali la struttura \`e irrevolare ma dritta: le coppie diverse hanno un angolo opposto che si bilancia.
			Se le perturbazioni sono regolari gli angoli si sommano portando a una curvatura del DNA.

	\subsection{Stabilizzazione della doppia elica}
	La doppia elica \`e stabilizzata da interazioni non covalenti:
	\begin{multicols}{2}
		\begin{itemize}
			\item Forze di Van der Waals: interazioni additive di impacchettamento tra le basi impilate.
			\item Legami idrogeno: interazioni tra le basi azotate appaiate di filamenti opposti.
			\item Effetti idrofobici: l'elusione di molecole di acqua dalle coppie rafforza i legami a idrogeno.
			\item Interazioni carica-carica: la repulsione elettrostatica tra le cariche negative dei gruppi fosfato \`e schermata da interazioni con ioni \emph{$Mg^{2+}$} e da proteine ricche di residui basici come lisina e arginina.
		\end{itemize}
	\end{multicols}

	\subsection{Sintesi}
	La sintesi del DNA avviene attraverso la formazione del legame fosfodiesterico.
	La formazione di questo legame libera una molecola di acqua.
	Il backbone \`e formato da residui di zucchero e fosfato alternati.
	La posizione $3'$ viene legata alla $5'$ attraverso un gruppo fosfato.
	Le basi azotate sporgono al di fuori del backbone.
	I gruppi fosfato conferiscono una carica negativa al backbone.
	L'ordine delle basi lungo la catena \`e casuale.

	\subsection{DNA superavvolto o struttura terziaria}
	Il DNA tende a formare dei superavvolgimenti, ovvero ad assumere strutture spaziali particolari che pongono stress sulla doppia elica. 
	I superavvolgimenti vengono descritti in termini matematici e lo studio dello stato di avvolgimento del DNA \`e di interesse in quanto va a influire sui meccanismi di trascrizione e replicazione. 
	Lo stato di superavvolgimento si descrive attraverso due parametri: twist (frequenza) che indica quante volte un filamento si avvolge sull'altro e il writhe, il numero di superavvolgimenti che si vengono a formare. 
	Un DNA lineare presenta un writhe pari a $0$ e un twist $n$ dipendente dalla sua lunghezza. 
	Le misure di twist e writhe vengono combinate nel linking number, dove $Lk=Tw+Wr$. 
	Il writhe pu\`o assumere un valore positivo o negativo a seconda che il superavvolgimento sia rispettivamente destrorso o sinistrorso. 
	Il linking number tende a rimanere costante e pertanto variazioni di uno dei due parametri portano ad un bilanciamento nell'altro. 
	Molti meccanismi (come replicazione e trascrizione) vanno a modificare il twist del DNA andando pertanto ad introdurre writhe e a stressare la molecola che tender\`a a superavvolgersi per tornare ad una struttura stabile ottimale. 
	Un aumento incontrollato del writhe rende sempre pi\`u difficoltoso il procedere di questi meccanismi che pertanto richiederanno l'aiuto di enzimi che diminuiscono i superavvolgimenti in modo da poter continuare.

		\subsubsection{Topoisomerasi}
		Le topoisomerasi sono proteine che intervengono per risolvere i super avvolgimenti introdotti dalle variazioni di twist rilassando la molecola. 
		Si trovano due famiglie di topoisomerasi: la famiglia di tipo I presente in batteri, eucarioti ed archei e la famiglia di tipo II presente solo negli eucarioti. 
		Le due famiglie differiscono per come agiscono sul DNA.

			\paragraph{Topoisomerasi di tipo I} 
			Le topoisomerasi di tipo I non richiedono ATP per funzionare, possiedono un dominio di interazione con il DNA formato da $\alpha$-eliche a pinze che si lega nel punto dove si trova il superavvolgimento. 
			Uno dei due filamenti viene tagliato e viene favorito il suo passaggio da un'estremit\`a all'altra in quanto la molecola cos\`i riduce il proprio stress. 
			Alla fine  la topoisomerasi riunisce il filamento e si stacca dal DNA. 
			Il linking number varia di $1$. 
			Il funzionamento di queste proteine \`e dovuto alla presenza di una tirosina con un gruppo \ce{OH} altamente reattivo che rompe il legame fosfodiesterico in un primo momento e viene poi riutilizzato per riformarlo. 
			Molti farmaci come la camptotecina impediscono la riformazione del legame fosfodiesterico rendendo permanente il taglio. 
			Viene utilizzata come antitumorale in quanto le cellule tumorali hanno elevati livelli di telomerasi. 
			Questo farmaco ha effetto anche su cellule con elevato tasso di mitosi.

			\paragraph{Topoisomerasi di tipo II}
			Le topoisomerasi di tipo II rompono il doppio filamento. 
			Sono dei dimeri con tre cancelli: $N$, $C$ e uno per l'ingresso del DNA. 
			Una volta che il doppio filamento viene tagliato un altro viene alloggiato nel cancello $N$ che si chiude. 
			L'idrolisi dell'ATP causa un cambio conformazionale che fa passare il filamento dal cancello $N$ al $C$ che lo rilascia una volta che il DNA viene risaldato. 
			Il linking number varia di $2$. 

		\subsubsection{Compattazione della struttura del DNA}
		Mediamente il DNA si trova in uno stato di supercoiling negativo.
		Una funzione del superavvolgimento \`e la compattazione della struttura del DNA.
		La torsione negativa favorisce la separazione delle catene e l'azione della polimerasi, mentre la positiva la impedisce.
		Si deve pertanto fare un compromesso tra la stabilit\`a della doppia elica e la sua permissivit\`a delle proteine che agiscono su essa.
		Nel DNA dei cromosomi eucarioti la compattazione \`e ottenuta con un superavvolgimento di tipo solenoidale intorno alle proteine istoniche.

\section{Struttura del RNA}
L'RNA \`e una macromolecola con diverse funzioni: se ne classificano pertanto diversi tipi in base alla funzione:
\begin{itemize}
	\item RNA messaggeri (mRNA): utilizzati come intermedi per la produzione delle proteine a partire dalle informazioni contenute nel DNA.
	\item RNA ribosomial (rRNA): fanno parte del ribosoma e vengono utilizzati per la produzione di proteine.
	\item RNA di trasporto (tRNA): si occupano di trasportare gli amminoacidi necessari alla sintesi di proteine.
	\item Micro RNA (miRNA), short interfering RNA (siRNA) e piwi-interacting RNA (piRNA): hanno funzione regolatoria.
	\item Piccoli RNA nucleari (snRNA): modificano tipicamente i nucleotidi e altri RNA.
	\item Long non coding RNA: ruolo regolatorio. 

\end{itemize}
L'RNA \`e un polimero a catena singola formato da nucleotidi. 
Presenta lo zucchero ribosio con in posizione $3'$ un gruppo fosfato su cui avviene il legame fosfodiesterico in $5'$ con il
nucleotide successivo. 
La base \`e collegata in $1'$. 
Il gruppo fosfato conferisce una carica negativa alla catena mentre il gruppo idrossile in $2'$ causa un ingombro sterico 
che gli fa assumere una forma $A$ con un solco maggiore molto profondo. 
Le basi sono $4$: due pirimidine citosina e uracile e due purine adenina e guanina. 

Pur essendo tipicamente a catena singola pu\`o assumere strutture secondarie particolari grazie alle interazioni tra le basi. 

\begin{itemize}
	\item A doppia elica.
	\item Ansa o forcina: con stelo doppia elica e un loop in testa.
\end{itemize}
Queste strutture oltre a fornire ulteriore stabilit\`a alla molecola impedendo la rottura del legame fosfodiesterico e le permettono di essere riconosciuta da proteine importanti per la 
sua funzione. 

\subsection{Modifiche dell'RNA}
L'RNA pu\`o andare incontro a diverse modifiche che permettono una regolazione della sua attivit\`a.
\begin{itemize}
	\item Inosina, pseudouridina e tiouridina: basi azotate che vanno a sostituirsi alla guanina.
	\item Metilazione della guanina: viene formata la $7$-metil guanina, avviene nel primo nucleotide di tutti gli mRNA e viene riconosciuta dalla cap-binding protein che aiuta il processo di traduzione.
\end{itemize}
\section{Organizzazione del DNA nel nucleo}
Se i procarioti presentano un DNA circolare negli eucarioti il DNA rimane lineare separato in cromosomi. 
Le lunghe catene devono compattarsi per potersi trovare nel nucleo. 
Lo stato
di compattamento del DNA deve essere necessariamente dinamico per permettere l'accesso a fattori di trascrizione e replicazione. 

	
	\subsection{Nucleosomi}
	Il primo livello di impacchettamento del DNA sono i nucleosomi.
	Sono formati da un ottamero istonico a cui si avvolgono $127$ basi di DNA con due giri e mezzo.
	Ulteriori livelli di compattamento portano alla formazione di solenoidi e a zig-zag.

	\subsection{Istoni}
	Gli istoni sono proteine basiche (la carica positiva bilancia la carica negativa dei gruppi fosfato) che condensano il DNA.
	Ne esistono diversi tipi:
	\begin{itemize}
		\item \emph{H2A, H2B, H3, H4} formano un ottamero e sono la parte proteica del nucleosoma.
		\item \emph{H1} \`e il DNA linker, permette ulteriore compattamento tra nucleosomi.
	\end{itemize}
	Ci sono varianti di questi istoni come \emph{H2AZ}, introdotto nei nucleosomi quando viene indotto danno al DNA.
		
		\subsubsection{Code istoniche}
		Ogni istone contiene delle code che vengono modificate in modo da modificare l'impacchettamento della cromatina.
		Le code sono $N$-terminale e in base alla combinazione di diverse modifiche post-traduzionali creano un codice che permette di modificare la struttura della cromatina e l'attivit\`a di un gene.
		Sono rivolte verso l'esterno.
		I gruppi acetili allentano l'impacchettamento per esempio.

		\subsubsection{Istone linker}
		L'istone linker \emph{H1} \`e un istone che si posiziona sul DNA non nucleosomico e lo piega avvicinando tra di loro i nucleosomi.

		\subsubsection{Epigenetica}
		Si intende per epigenetica le informazioni esterne al DNA che vengono aggiunte tramite legame con gruppi senza variare la sequenza nucleotidica della cellula e possono essere ereditate.

		\subsubsection{Complessi di rimodellamento della cromatina \emph{ATP} dipendenti}
		Questi complessi hanno una subunit\`a che si lega sia al nucleo proteico del nucleosoma che al DNA intorno.
		Attraverso idrolisi di \emph{ATP} causano un cambio conformazionale del nucleosoma liberando il DNA.
		I complesso possono catalizzare lo scorrimento dei nucleosomi, loro rimozione o scambio di istoni.
		Sono importanti in quanto rendono dinamico lo stato cromatico permettendo alla cellula di rispondere a eventi diversi.

		\subsubsection{Modifiche delle code istoniche}
		Le modifiche a livello delle code istoniche vengono riconosciute da domini delle proteine.
		\begin{multicols}{2}
			\begin{itemize}
				\item Bromodominio: riconosce acetilazione.
				\item Cromodominio: riconosce metilazione
			\end{itemize}
		\end{multicols}
		Le proteine possono contenere diverse combinazioni dei vari domini.

		\subsubsection{Sintesi degli istoni}
		Gli istoni principali vengono sintetizzati durante la fase $S$ del ciclo cellulare e vengono inseriti nella cromatina attraverso lo scambio degli istoni.
		Questo processo viene catalizzato da complessi di rimodellamento della cromatina.

			\paragraph{Facilitates chromatine transcription}
			\emph{FACT} \`e una proteina che destabilizza gli istoni, lavorano a monte della RNA polimerasi e rimuovono i dimeri \emph{H2A-H2B}.

			\paragraph{Chaperonine}
			Le chaperonine sono proteine in grado di piegare le strutture delle altre proteine in modo da renderle pi\`u funzionali.

		\subsubsection{Formazione dell'ottamero}
		Si formano due dimeri \emph{H2A-H2B} e due \emph{H3-H4} che si legano prima in tetrameri e poi nell'ottamero.
		Questo si posiziona sul DNA aspecificamente (scoperto tramite digestioni con \emph{DNAasi}) anche se preferisce regioni ricche di $AT$ e avvolge $127$ basi formando un nucleosoma.

		\subsubsection{Ereditare i nucleosomi}
		Durante la replicazione l'ottamero si dissocia: \emph{H3-H4} rimane legato al DNA, mentre \emph{H2A-H2B} torna a far parte del pool di dimeri libero nella cellula, da dove viene mischiato
		con quelli di nuova sintesi che si ripartiscono casualmente nei due filamenti.

	\subsection{Stati della cromatina}
	Osservando la microscopio elettronico il DNA si notano zone pi\`u chiare e altre pi\`u scure della membrana nucleare.
	Queste zone sono formate da cromatina in stati diversi.

		\subsubsection{Eterocromatina}
		L'eterocromatina \`e altamente condensata e si presenta come zone pi\`u scure e dense.
		Si trova principalmente sulla periferia del nucleo. 

		\`E la parte dei cromosomi non attiva.
		\`E molto metilata.
		Si divide in:
		\begin{multicols}{2}
			\begin{itemize}
				\item Costitutiva: lo stato eterocromatico \`e persistente, ha ruolo strutturale e non codificante.
				\item Facoltativa: lo stato eterocromatico pu\`o essere rilassato in eucromatico.
			\end{itemize}
		\end{multicols}

		\subsubsection{Eucromatina}
		L'eucromatina \`e meno condensata e si presenta come zone meno scure.
		Si trova principalmente sulle braccia dei cromosomi ed \`e trascritta attivamente.
		\`E molto acetilata e si ricombina durante la meiosi.

		\subsubsection{Nucleolo}
		Il nucleolo \`e una parte densa molto attiva: vengono sintetizzati gli rRNA.

		\subsubsection{Effetto di posizione}
		L'eterocromatina tende a essere trasmessa nelle zone vicine causando il silenziamento di geni normalmente attivi.
		Barriere proteiche impediscono la diffusione incontrollata dell'eterocromatina.
		Una volta stabilita una situazione eterocromatica questa viene ereditata stabilmente dalla progenie della cellula.

		\subsubsection{Effetto della compattazione}
		La compattazione del DNA \`e importante in:
		\begin{multicols}{3}
			\begin{itemize}
				\item Replicazione.
				\item Ricombinazione.
				\item Trascrizione.
				\item Riparazione.
				\item Segregazione dei cromosomi.
			\end{itemize}
		\end{multicols}

	\subsection{Sindrome di Rett}
	
		\subsubsection{Caratteristiche cliniche}
		Le persone affette da sindrome di Rett presentano, dopo un periodo di apparente sviluppo normale:
		\begin{multicols}{3}
			\begin{itemize}
				\item Ridotte capacit\`a motorie.
				\item Riduzione della grandezza del cervello.
				\item Ritardi nello sviluppo.
			\end{itemize}
		\end{multicols}

		\subsubsection{Motivazioni biologiche}
		La malattia \`e dovuta a una mutazione a carico di \emph{MeCP2}, un repressore della trascrizione genica in regioni del DNA metilata.
		La mancata metilazione del gene per \emph{NGF} nerve growth factor impedisce la differenziazione dei neuroni e una loro corretta morfologia: non formano sinapsi e dendriti in numero sufficiente.

\section{Modifiche post-traduzionali}
Le modifiche post-traduzionali \`e un meccanismo di regolazione rapido che produce modifiche reversibili o irreversibili, fisiologiche o patologiche.
Sono modifiche che avvengono sulle proteine e ne modificano l'attivit\`a e funzione, localizzazione, inibire o aumentarne l'attivit\`a o il turnover. 
Le proteine vengono degradate con il tempo pi\`u o meno velocemente.
Le modifiche avvengono sia nel citoplasma che nel nucleo.

	\subsection{Fosforilazione}
	La fosforilazione \`e l'aggiunta di un gruppo fosfato da enzimi detti chinasi che trasferiscono un gruppo fosfato da \emph{ATP} a residui di serina, treonina e tirosina.
	\`E una modifica reversibile. 
	L'enzima che toglie il gruppo fosfato \`e detto fosfatasi e sono meno specifici
	Pu\`o determinare cambiamenti strutturali profondi ed \`e determinante in quasi tutte le vie di trasduzione di segnali, permettendo il riconoscimento tra proteine.

	\subsection{Acetilazione}
	L'acetilazione consiste nell'aggiunta di un gruppo acetilico attraverso acetiltransferasi che viene rimosso da deacetilasi. 
	Il donatore del gruppo acetile \`e l'\emph{acetil-CoA}.
	Si formano pertanto legami covalenti con residui di lisina o N-terminali.
	L'acetilazione intramolecolare permette la creazione di modifiche epigenetiche ereditabili a livello degli istoni.


	\subsection{Metilazione}
	Aggiunta di un gruppo metilico attraverso metiltransferasi e una demetilasi che lo rimuove. 

	Avviene tipicamente a carico di una lisina.

	\subsection{Ubiquitinazione}
	Aggiunta di un gruppo di ubiquitina, un polipeptide che viene aggiunto grazie a degli enzimi particolari e che cambia il turnover della proteina. 
Una proteina che va incontro a 
	ubiquitinazione va incontro a degradazione. 
Questa modifica avviene grazie all'attivazione dell'ubiquitina ad opera dell'enzima E1 a cui segue l'attivazione e reclutamento di E2 e E3 con
	funzioni diverse. 
La proteina bersaglio contiene la lisina, l'unico amminoacido che va incontro a ubiquitinazione. 
Gli enzimi prendono l'ubiquitina e la legano alla proteina bersaglio
	alla lisina. 
Possono essere legate $n$ molecole alla proteina (poliubiquitinata). 
Questa modifica cambia di molto il peso molecolare della proteina ed \`e possibile vedere sul gel la
	modifica di questo tipo. 
La proteina poliubiquitinata viene portata al proteasoma, un complesso $26s$ costituito da un nucleo (cilindro cavo), un cappuccio d'entrata e uno d'uscita. 
Nel
	cappuccio d'entrata si trova il recettore Rpn10 che riconosce l'ubiquitina e permette e forza la proteina ad entrare all'interno del proteasoma. 
La coda rimane al di fuori. 
L'ubiquitina 
	pertanto non viene degradata e rimarr\`a disponibile. 
Gli amminoacidi escono alla fine dal cappuccio d'uscita. 
La isopeptidasi Ppn11 ha lo scopo di staccare la coda poliubiquitina. 

	Si noti come l'ubiquitina \`e irreversibile. 
Non tutte le proteine contenenti lisina possono essere degradate in quanto ci devono essere a valle e a monte altri amminoacidi specifici. 


	\subsection{Sumoilazione}
	Meccanismo analogo e simile all'ubiquitina. 
Si tratta dell'aggiunta del polipeptide con l'attivazione di enzimi che lo vanno attaccare alla proteina e come nel caso dell'ubiquitinazione
	ci pu\`o essere polisumoilazione. 
Questo processo non comporta necessariamente degradazione della proteina. 
Normalmente viene utilizzata per modificare attivit\`a o localizzazione di una
	proteina specifica. 
Si lega sempre su una lisina. 
I substrati di SUMO si trovano nei pori nucleari per modificare il traffico nucleare cambiando permeabilit\`a o specificit\`a. 
Nei 
	fattori di trascrizione influisce nell'espressione genica, nelle proteine nella riparazione e replicazione del DNA influendo sulla stabilit\`a genomica e nel citocore e centromeri 
	influisce sull'integrit\`a cromosoma. 
La sumoilaizone \`e reversibile e quando deregolata pu\`o essere alla base di malattie neurodegenerative, tumori, diabete, eccetera. 
\`E importante
	per progressione del ciclo cellulare, sopravvivenza, divisione e proliferazione cellulare, differenziamento e invecchiamento. 
Avvengono su domini critici: per fattori di trascrizione
	nei domini di legame del DNA o la sequenza che permette alle proteine di entrare nel nucleo. 
La sumoilazione pu\`o agire sulla funzione di una proteina coinvolta nella riparazione del
	DNA timina-DNA glicosilasi che scansiona il DNA e quando trova una U lo rimuove e va incontro a sumoilazione che determina il distacco con il DNA e favorisce l'introduzione di un altro
	fattore che va a chiudere il gap mettendo la base corretta. 
Sumo pu\`o essere rimosso e TDG torna a funzionare. 

		
\section{Metabolismo del ferro}
Lo ione ferro \`e in grado di reagire con l'ossigeno e creare radicali liberi che possono danneggiare la molecola di DNA.
Si rende pertanto necessario regolare il livello del ferro.

	\subsection{Introduzione del ferro nella cellula}
	Il ferro viene introdotto nella cellula grazie a recettori transferrina presenti sulla membrana citoplasmatica della cellula, che lo legano quando presente nel torrente circolatorio.
	Dopo il legame il complesso ha un $pH$ neutro $7.4
$ che causa la sua internalizzazione.
	Si forma cos\`i una vescicola endocitotica.
	Questa matura in un endosoma che si acidifica permettendo il distacco dello ione dalla transferrina.
	La transferrina torna in membrana.

	\subsubsection{Il ferro nella cellula}
	Il ferro nella cellula viene utilizzato come co-attivatore di enzimi.
	Il ferro in eccesso che si trova nella cellula viene disattivato dalla ferritina.
	Questa proteina lo lega impedendo che venga coinvolto in reazioni dannose per la cellula.


	\subsection{Meccanismo di regolazione}
	Si nota pertanto come per regolare i livelli di ferro si devono modulare le concentrazioni di transferrina e di ferritina.
	Questo avviene a livello post-trascrizionale.

		\subsubsection{Aconitasi}
		L'aconitasi \`e una proteina che lega RNA e funge da sensore per il ferro, fa parte della famiglia delle \emph{IRP} (iron regulatory proteins).
		Si lega a strutture a forcina che si forma nella \emph{$5'$-UTR} del RNA per la ferritina e \emph{$3'$-UTR} del RNA per la transferrina.

		\subsubsection{Carenza di ferro}
		In carenza di ferro la cellula deve aumentare i livelli di transferrina e diminuire quelli di ferritina.
		L'aconitasi, non legata al ferro si lega alle struttura a forcina degli mRNA o \emph{IRE} (iron responsive elements):
		\begin{itemize}
			\item Transferrina: stabilizza il suo RNA impedendone la sua degradazione.
			\item Ferritina: impedisce il legame dei ribosomi con il mRNA.
		\end{itemize}

		\subsubsection{Eccesso di ferro}
		In eccesso di ferro la cellula deve diminuire i livelli di transferrina ed aumentare quelli di ferritina.
		L'aconitasi legata al ferro si dissocia dagli \emph{IRE} degli mRNA di:
		\begin{itemize}
			\item Transferrina: destabilizza il mRNA causandone la degradazione.
			\item Ferritina: permette ai ribosomi di tradurre il mRNA.
		\end{itemize}
	
	\subsection{Metodo di osservazione}
	Per osservare il processo si ingegnerizza il gene della ferritina in modo che esprima una proteina reporter \emph{GFP}.
	In questo modo quando il ferro non \`e presente l'aconitasi si lega al RNA e blocca la traduzione impedendo alla cellula di esprimere \emph{GFP}.
	Quando invece \`e presente avviene la traduzione e la cellula esprime \emph{GFP}, che essendo fluorescente \`e facilmente osservabile.
