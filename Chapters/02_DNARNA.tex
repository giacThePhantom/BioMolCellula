\chapter{DNA e RNA}
\section{Struttura del DNA}
Il DNA si trova nel nucleo delle cellule eucarioti e ed ancorato alla membrana delle cellule procarioti.
\`E costituita da due catene polinucleotide disposte antiparallelamente. 
\`E pertanto un polimero costituito da nucleotidi.

	\subsection{Componenti}

		\subsubsection{Monomeri}

			\paragraph{Nucleosidi}
			I nucleosidi sono formati dalla base azotata e uno zucchero legate da un legame $\beta$-$N$-glicosidico tra il carbonio anomerico $C1$ del ribosio e $N9$ di purine o $N1$ delle pirimidine.
			Sono pertanto costituiti da uno zucchero come $2'$-deossiribosio.

				\subparagraph{Orientamento del legame}
				Il legame pu\`o essere orientato in:
				\begin{multicols}{2}
					\begin{itemize}
						\item \emph{Syn}.
						\item \emph{Anti}, l'orientamento principale.
					\end{itemize}
				\end{multicols}

			\paragraph{Nucleotidi}
			I nucletidi sono esteri fosfato dei nucleosidi: il sito di fosforilazione \`e spesso l'ossidrile al $5'$ del ribosio.
			Possono essere mono, di e trifosfati.
			I gruppi fosfati sono legati tra di loro da legame anidridico stabile e ad alta energia.
			
				\subparagraph{Funzioni}
				I nucleotidi hanno funzione di:
				\begin{multicols}{2}
					\begin{itemize}
						\item Unit\`a strutturale degli acidi nucleici.
						\item Deposito di energia delle reazioni di trasferimento di fosfato come \emph{ATP} e \emph{GTP}.
						\item Mediatore di processi cellulari \emph{cAMP}, formato grazie alla adenilciclasi da \emph{ATP}, processo invertito dalla fosfodiesterasi.
						\item Parte di coenzimi: come \emph{NADH}, \emph{FAD}, \emph{CoA}.
					\end{itemize}
				\end{multicols}

				\subparagraph{Gruppo idrossile}
				Il gruppo idrossile in \emph{$3'$-OH} \`e importante per la polimerizzazione.
				
				\subparagraph{Base azotata}
				I nucleotidi nel DNA contengono quattro basi diverse:
				\begin{multicols}{2}
					\begin{itemize}
						\item Timina, una pirimidina con un unico anello.
						\item Citosina, una pirimidina con un unico anello.
						\item Guanina, una purina con due anelli.
						\item Adenina, una purina con due anelli.
					\end{itemize}
				\end{multicols}

		\subsubsection{Polimerizzazione}
		La posizione del gruppo idrossile in \emph{$3'$-OH} \`e importante in quanto \`e il gruppo coinvolte nella creazione di legami fosfodiesterici per la polimerizzazione.
		Questi legami infatti nascono tra il \emph{$3'$-OH} e il gruppo fosfato in $5'$.
		Si forma pertanto una catena costituita da una successione di nucleotidi in cui variano le basi azotate.
		
	\subsection{Polarit\`a}
	Essendo il polimero lineare possiede un inizio e una fine:
	\begin{multicols}{2}
		\begin{itemize}
			\item Estremit\`a $5'$ con fosfato libero viene tenuta per convenzione in alto a sinistra.
			\item Estremit\`a $3'$ con \emph{-OH} libero in basso a sinistra.
		\end{itemize}
	\end{multicols}
	La polarit\`a di questa direzionalit\`a e nella doppia elica si nota un autoparallelismo con due filamenti con polarit\`a opposta.

	\subsection{Interazioni tra basi}
	Le basi azotate nelle due catene della doppia elica interagiscono tra di loro attraverso legami a idrogeno:
	\begin{multicols}{2}
		\begin{itemize}
			\item $A$-$T$ con due legami a idrogeno.
			\item $G$-$C$ con tre legami a idrogeno.
		\end{itemize}
	\end{multicols}

		\subsubsection{Movimento tra basi appaiate}

			\paragraph{Twist}
			Nel twist le basi possono roteare l'una sull'altra.

			\paragraph{Roll}
			Nel roll le basi ruotano lungo l'asse maggiore.

			\paragraph{Tilt}
			Nel tilt le basi ruotano lungo l'asse minore.
			
		\subsubsection{Chargaff}
		Chargaff osservando DNA proveniente da vari organismi e nota come i residui di $A$ e $T$ e $C$ e $G$ sono in numero uguale a due a due, provando la complementariet\`a tra le basi. 
		Nota inoltre come il rapporto tra le basi non complementari rimane uguale in ogni tessuto di ogni organismo ma \`e un numero univoco che differisce tra gli organismi.
		Si nota come organismi che sopravvivono a temperature pi\`u elevate possiedono una percentuale maggiore di $C$ e $G$. 
		Questo avviene in quanto $C$ e $G$ formano un legame a idrogeno in pi\`u rispetto ad $A$ e $T$ rendendo la macromolecola pi\`u resistente alla denaturazione. 

		\subsubsection{Temperatura di melting}
		La temperatura di melting viene utilizzata per determinare la percentuale di $C$ e $G$ all'interno del DNA in quanto pi\`u sono presenti pi\`u \`e difficile denaturarlo. 
		Si usa lo spettrofotometro per osservare una provetta annotando la sua assorbanza che cambia in base allo stato di denaturazione del DNA (a doppia elica assorbe a $260\si{\nano\metre}$). 
		Grazie alla variazione del valore di assorbanza si riesce a determinare la temperatura in cui met\`a del DNA viene denaturato o temperatura di melting. 

	\subsection{Caratteristiche}
	Il diametro dell'elica \`e di $2\si{nm}$ e avviene un giro ogni $10$ nucleotidi.
	Si nota un solco maggiore di $22\si{\angstrom}$ e uno minore di $12\si{\angstrom}$.
	Questa differenza \`e dovuta ai diversi angoli tra i due legami glicosidici.
	L'angolo meno ampio genera il solco minore.
	Il solco maggiore fornisce pi\`u informazioni di tipo chimico in quanto contiene pi\`u gruppi donatori ed accettori di protoni.

		\subsubsection{Lettura del codice genetico}
		La lettura del codice genetico da parte delle proteine si basa pertanto su:
		\begin{multicols}{2}
			\begin{itemize}
				\item Accettore di legame \emph{HB} ($A$).
				\item Donatore di legame \emph{HB} ($D$).
				\item Gruppo Metilico ($M$).
				\item Idrogeno non polare ($H$).
			\end{itemize}
		\end{multicols}
		Queste propriet\`a permettono alle proteine di riconoscere senza ambiguit\`a specifiche sequenze di DNA senza che sia necessario leggere la doppia elica.
	\subsection{Tipologie di strutture}

		\subsubsection{Struttura primaria}
		Si intende per struttura primaria la sequenza di nucleotidi che costituisce la molecola.

		\subsubsection{Struttura secondaria}
		Si intende per struttura secondaria la configurazione tridimensionale stabile di una molecola di DNA.
		La molecola di DNA \`e formata da $2$ catene di nucleotidi appaiate secondo la regola della complementariet\`a.
		L'elica \`e destrorsa con due catene antiparallele.
		Le basi si trovano all'interno dell'elica allineate ad angolo retto.

			\paragraph{Impilamento delle basi}
			La disposizione delle basi all'interno della doppia elica \`e generalmente favorita.
			Molecole contenenti anelli aromatici tendono a disporsi impilate spontaneamente in quanto idrofobiche.
			Le basi si impilano disponendosi parallelamente riducendo in questo modo il contatto con l'acqua e stabilizzando la struttura.

		\subsubsection{Basi di Hoogsteen}
		Le basi di Hoogsteen si formano tra basi non complanari con rotazione dell'anello pirimidinico e meno stabili.

		\subsubsection{Conformazioni particolari}
			
			\paragraph{Forma $\mathbf{A}$}
			La forma $A$ si forma a causa della riduzione della quantit\`a d'acqua, presenta una struttura tarchiata con $11$ basi per giro e una scanalatura principale pi\`u profonda.

			\paragraph{Forma $\mathbf{B}$}
			La forma $B$ \`e la forma canonica cristallizzata da Franklin con $10$ paia di basi per giro.

			\paragraph{Forma $\mathbf{Z}$}
			La forma $Z$ si forma in zone particolari ricche di $C$ e $G$ con $12$ paia di basi per giro.
			La scanalatura principale \`e poco profonda e questo rende le basi pi\`u prone alla metilazione e pertanto presente nelle zone promotrici dei geni.

			\paragraph{Tripla elica}
			Il DNA pu\`o assumere strutture a tripla elica che si formano dove esiste una sequenza particolare. 
			Queste strutture influiscono sul processo di duplicazione e trascrizione del DNA e molto spesso rendono il DNA ancora pi\`u stabile e la capacit\`a delle due catene di separarsi \`e compromessa, impedendo l'arrivo dei fattori necessari ai processi. 
			\`E ricca di A e T ed \`e resa possibile dalle interazioni di Hogsteen. 

			\paragraph{Struttura cruciforme}
			La struttura cruciforme, a stem loop o a forcina si forma dove si trovano sequenze palindromiche. 
			\`E caratteristica sia di DNA ed RNA dove sono presenti sequenze ripetute seguite da una sequenza invertita. 
			Possono essere deleterie per i processi di duplicazione e trascrizione in quanto conferiscono alla catena un'elevata stabilit\`a impedendo la separazione delle due catene. 

			\paragraph{Struttura a forcina}
			La struttura a forcina o hairpin \`e costituita da ripiegamenti a doppia elica con basi accoppiate e zone a loop dove non si trova complementariet\`a.

			\paragraph{DNA curvo}
			Coppie di basi adiacenti non perfettamente parallele portano a piccoli ripiegamenti.
			Se le perturbazioni sono casuali la struttura \`e irregolare ma dritta: le coppie diverse hanno un angolo opposto che si bilancia.
			Se le perturbazioni sono regolari gli angoli si sommano portando a una curvatura del DNA.

	\subsection{Stabilizzazione della doppia elica}
	La doppia elica \`e stabilizzata da interazioni non covalenti:
	\begin{multicols}{2}
		\begin{itemize}
			\item Forze di Van der Waals: interazioni additive di impacchettamento tra le basi impilate.
			\item Legami idrogeno: interazioni tra le basi azotate appaiate di filamenti opposti.
			\item Effetti idrofobici: l'esclusione di molecole di acqua dalle coppie rafforza i legami a idrogeno.
			\item Interazioni carica-carica: la repulsione elettrostatica tra le cariche negative dei gruppi fosfato \`e schermata da interazioni con ioni \emph{$Mg^{2+}$} e da proteine ricche di residui basici come lisina e arginina.
		\end{itemize}
	\end{multicols}

	\subsection{Sintesi}
	La sintesi del DNA avviene attraverso la formazione del legame fosfodiesterico.
	La formazione di questo legame libera una molecola di acqua.
	Il backbone \`e formato da residui di zucchero e fosfato alternati.
	La posizione $3'$ viene legata alla $5'$ attraverso un gruppo fosfato.
	Le basi azotate sporgono al di fuori del backbone.
	I gruppi fosfato conferiscono una carica negativa al backbone.
	L'ordine delle basi lungo la catena \`e casuale.

	\subsection{DNA superavvolto o struttura terziaria}
	Il DNA tende a formare dei superavvolgimenti, ovvero ad assumere strutture spaziali particolari che pongono stress sulla doppia elica. 
	I superavvolgimenti vengono descritti in termini matematici e lo studio dello stato di avvolgimento del DNA \`e di interesse in quanto va a influire sui meccanismi di trascrizione e replicazione. 
	Lo stato di superavvolgimento si descrive attraverso due parametri: twist (frequenza) che indica quante volte un filamento si avvolge sull'altro e il writhe, il numero di superavvolgimenti che si vengono a formare. 
	Un DNA lineare presenta un writhe pari a $0$ e un twist $n$ dipendente dalla sua lunghezza. 
	Le misure di twist e writhe vengono combinate nel linking number, dove $Lk=Tw+Wr$. 
	Il writhe pu\`o assumere un valore positivo o negativo a seconda che il superavvolgimento sia rispettivamente destrorso o sinistrorso. 
	Il linking number tende a rimanere costante e pertanto variazioni di uno dei due parametri portano ad un bilanciamento nell'altro. 
	Molti meccanismi (come replicazione e trascrizione) vanno a modificare il twist del DNA andando pertanto ad introdurre writhe e a stressare la molecola che tender\`a a superavvolgersi per tornare ad una struttura stabile ottimale. 
	Un aumento incontrollato del writhe rende sempre pi\`u difficoltoso il procedere di questi meccanismi che pertanto richiederanno l'aiuto di enzimi che diminuiscono i superavvolgimenti in modo da poter continuare.

		\subsubsection{Topoisomerasi}
		Le topoisomerasi sono proteine che intervengono per risolvere i super avvolgimenti introdotti dalle variazioni di twist rilassando la molecola. 
		Si trovano due famiglie di topoisomerasi: la famiglia di tipo I presente in batteri, eucarioti ed archei e la famiglia di tipo II presente solo negli eucarioti. 
		Le due famiglie differiscono per come agiscono sul DNA.

			\paragraph{Topoisomerasi di tipo I} 
			Le topoisomerasi di tipo I non richiedono ATP per funzionare, possiedono un dominio di interazione con il DNA formato da $\alpha$-eliche a pinze che si lega nel punto dove si trova il superavvolgimento. 
			Uno dei due filamenti viene tagliato e viene favorito il suo passaggio da un'estremit\`a all'altra in quanto la molecola cos\`i riduce il proprio stress. 
			Alla fine  la topoisomerasi riunisce il filamento e si stacca dal DNA. 
			Il linking number varia di $1$. 
			Il funzionamento di queste proteine \`e dovuto alla presenza di una tirosina con un gruppo \ce{OH} altamente reattivo che rompe il legame fosfodiesterico in un primo momento e viene poi riutilizzato per riformarlo. 
			Molti farmaci come la camptotecina impediscono la riformazione del legame fosfodiesterico rendendo permanente il taglio. 
			Viene utilizzata come antitumorale in quanto le cellule tumorali hanno elevati livelli di telomerasi. 
			Questo farmaco ha effetto anche su cellule con elevato tasso di mitosi.

			\paragraph{Topoisomerasi di tipo II}
			Le topoisomerasi di tipo II rompono il doppio filamento. 
			Sono dei dimeri con tre cancelli: $N$, $C$ e uno per l'ingresso del DNA. 
			Una volta che il doppio filamento viene tagliato un altro viene alloggiato nel cancello $N$ che si chiude. 
			L'idrolisi dell'ATP causa un cambio conformazionale che fa passare il filamento dal cancello $N$ al $C$ che lo rilascia una volta che il DNA viene risaldato. 
			Il linking number varia di $2$. 

		\subsubsection{Compattazione della struttura del DNA}
		Mediamente il DNA si trova in uno stato di supercoiling negativo.
		Una funzione del superavvolgimento \`e la compattazione della struttura del DNA.
		La torsione negativa favorisce la separazione delle catene e l'azione della polimerasi, mentre la positiva la impedisce.
		Si deve pertanto fare un compromesso tra la stabilit\`a della doppia elica e la sua permissivit\`a delle proteine che agiscono su essa.
		Nel DNA dei cromosomi eucarioti la compattazione \`e ottenuta con un superavvolgimento di tipo solenoidale intorno alle proteine istoniche.

\section{Struttura del RNA}
L'RNA \`e una macromolecola con diverse funzioni: se ne classificano pertanto diversi tipi in base alla funzione.

	\subsection{Funzioni del RNA}
	\begin{multicols}{2}
		\begin{itemize}
			\item RNA messaggeri (mRNA): utilizzati come intermedi per la produzione delle proteine a partire dalle informazioni contenute nel DNA.
			\item RNA ribosomiale (rRNA): fanno parte del ribosoma e vengono utilizzati per la produzione di proteine.
			\item RNA di trasporto (tRNA): si occupano di trasportare gli amminoacidi necessari alla sintesi di proteine.
			\item Micro RNA (miRNA), short interfering RNA (siRNA) e piwi-interacting RNA (piRNA): hanno funzione regolatoria.
			\item Piccoli RNA nucleari (snRNA): modificano tipicamente i nucleotidi e altri RNA.
			\item Long non coding RNA: ruolo regolatorio. 
		\end{itemize}
	\end{multicols}

	\subsection{Struttura}
	L'RNA \`e un polimero a catena singola formato da nucleotidi. 
	Presenta lo zucchero ribosio con in posizione $3'$ un gruppo fosfato su cui avviene il legame fosfodiesterico in $5'$ con il
	nucleotide successivo. 
	La base \`e collegata in $1'$. 
	Il gruppo fosfato conferisce una carica negativa alla catena mentre il gruppo idrossile in $2'$ causa un ingombro sterico che gli fa assumere una forma $A$ con un solco maggiore molto profondo. 
	Le basi sono $4$: due pirimidine citosina e uracile e due purine adenina e guanina. 

	\subsection{Strutture secondarie}
	Pur essendo tipicamente a catena singola pu\`o assumere strutture secondarie particolari grazie alle interazioni tra le basi. 
	\begin{multicols}{2}
		\begin{itemize}
		\item A doppia elica.
		\item Ansa o forcina: con stelo doppia elica e un loop in testa.
	\end{itemize}
	\end{multicols}
	Queste strutture oltre a fornire ulteriore stabilit\`a alla molecola impedendo la rottura del legame fosfodiesterico e le permettono di essere riconosciuta da proteine importanti per la sua funzione. 

\subsection{Modifiche dell'RNA}
L'RNA pu\`o andare incontro a diverse modifiche che permettono una regolazione della sua attivit\`a.
\begin{itemize}
	\item Inosina, pseudouridina e tiouridina: basi azotate che vanno a sostituirsi alla guanina.
	\item Metilazione della guanina: viene formata la $7$-metil guanina, avviene nel primo nucleotide di tutti gli mRNA e viene riconosciuta dalla cap-binding protein che aiuta il processo di traduzione.
\end{itemize}
\section{Organizzazione del DNA nel nucleo}
Le molecole di DNA dei vari cromosomi di una cellula sono troppo lunghe per essere contenute nel nucleo.
Devono pertanto essere compattate, sia negli eucarioti che nei procarioti in modo da poter riuscire ad entrare nel nucleo.
Questo compattamento va ad influenzare la capacit\`a della sequenza di essere letta dalle proteine coinvolte in trascrizione e replicazione e deve pertanto essere necessariamente dinamica.

	\subsection{Procarioti}
	I procarioti contengono un cromosoma di forma circolare nel nucleoide, una regione della cellula non isolata da membrana.
	Il DNA \`e ancorato ad un punto della membrana detto mesosoma, associato a proteine che formano sotto-domini che rendono indipendenti le anse.
	Le proteine sono \emph{HU}, \emph{IHF}, \emph{H1}, \emph{P}.

	\subsection{Eucarioti}
	Negli eucarioti i cromosomi sono lineari e si trovano nel nucleo, un organello deputato alla protezione del materiale genetico.
	Le proteine che legano e organizzano il DNA negli eucarioti sono degli analoghi di quelle nei procarioti e sono chiamate istoni.
	
	\subsection{Cromatina}
	La cromatina \`e la struttura che compone i cromosomi e determina la forma che assumono durante l'interfase.
	\`E formata da nucleosomi uniti tra di loro da DNA linker lungo fino a $80$ nucleotidi che conferisce al cromosoma una struttura a collana di perle.
	La collana di perle dei nucleosomi pu\`o essere successivamente compattata in una struttura a solenoide o a zig-zag.

		\subsubsection{Tipologie di cromatina}
		La cromatina si distingue in due tipologie in base al grado di impacchettamento del DNA.
		Queste si osservano attraverso microscopio elettronico: il DNA si presenta come zone chiare e pi\`u scure, dove l'intensit\`a del colore \`e lineare con il grado di compattamento.

			\paragraph{Eucromatina}
			L'eucromatina \`e una forma di cromatina rilassata e si presenta chiara al microscopio elettronico.
			Si trova in stato eucromatico la porzione di DNA che \`e in stato attivo di trascrizione.

			\paragraph{Eterocromatina}
			L'eterocromatina \`e una forma di cromatina altamente compattata e si presenta scura al microscopio elettronico.
			Si trova in stato eterocromatico la porzione di DNA non attivamente trascritta.
			Si trova tipicamente nella periferia del nucleo ed \`e divisa in:
			\begin{multicols}{2}
				\begin{itemize}
					\item Costitutiva: \`e sempre compatta, l'informazione genetica al suo interno non viene mai espressa e la regione si DNA assume una funzione strutturale.
					\item Facoltativa: \`e una regione di DNA che pu\`o essere decompattata e in determinate condizioni pu\`o avere propriet\`a codificanti.
				\end{itemize}
			\end{multicols}

			\paragraph{Nucleolo}
			Il nucleolo \`e una parte del nucleo molto densa e attiva in cui vengono sintetizzati gli rRNA.
	
		\subsubsection{Nucleosomi}
		Il primo livello di impacchettamento del DNA sono i nucleosomi.
		Sono formati da un ottamero istonico a cui si avvolgono $127$ basi di DNA con due giri e mezzo.
		Le proteine che formano un nucleosoma si distinguono in due categorie.
		
			\paragraph{Non core proteins}
			Le non core proteins sono l'istone \emph{H1}, che si occupa di collegare tra di loro diversi nucleosomi aumentando il grado di impacchettamento.
			Si posiziona pertanto sul DNA nucleosomico e lo piega avvicinando tra di loro i nucleosomi.
			Si nota come questa \`e la proteina istonica meno conservata.

			\paragraph{Core proteins}
			Le core proteins formano il nucleo di un nucleosoma e sono cariche positivamenti in modo da attirare la carica negativa dello scheletro a deossiribosio-fosfato del DNA.
			Non sono proteine grandi con conservati $3$ domini ad $\alpha$-elica e una coda $N$-terminale.
			\begin{multicols}{4}
				\begin{itemize}
					\item \emph{H2A}.
					\item \emph{H2B}.
					\item \emph{H3}.
					\item \emph{H4}.
				\end{itemize}
			\end{multicols}
			
				\subparagraph{Legame con il DNA}
				Le proteine del core istonico sono formate da residui di arginina o lisina, che le conferiscono una carica positiva che neutralizza il backbone del DNA.
				In questo modo il loro legame con il DNA non dipende dalla specifica sequenza ma dalla presenza del backbone.
				Gli istoni legano tipiche regioni del DNA.
				Il contatto tra tetramero e DNA avviene nel solco minore e le interazioni tramite legami a idrogeno tra proteina ed ossigeno del legame fosfodiesterico.
				Il grande numero di legami a idrogeno fornisce l'energia necessaria alla curvatura del DNA.

				\subparagraph{Struttura}
				Gli istoni sono proteine eterogenee che variano da tessuto a tessuto.
				Sono proteine piccole con un motivo strutturale comune detto ripiegamento istonico formato da $3$ $\alpha$-eliche connesse da due anse.

				\subparagraph{Sintesi}
				Gli istoni vengono sintetizzati durante la fase $S$ del ciclo cellulare e vengono inseriti nella cromatina attraverso lo scambio degli istoni.
				Questo processo viene catalizzato da complessi di romodellamento della cromatina.
				
				\subparagraph{Assemblaggio degli istoni}
				I ripiegamenti istonici si legano tra di loro formando dimeri \emph{H3-H4} e \emph{H2A-H2B}.
				Due dimeri \emph{H3-H4} si legano tra di loro formando un tetramero che recluta a sua volta due dimeri \emph{H2A-H2B} per formare il core proteico del nucleosoma.
				All'ottamero si avvolge il DNA.
				Il processo \`e spontaeo.
				Le code $N$-terminali stabilizzano il complesso avvolgendosi intorno all'ottamero proteico.

				\subparagraph{Code istoniche}
				Le code istoniche subiscono modifiche post-traduzionali in modo da modificare il grado di impacchettamento della cromatina.
				Le diverse modifiche creano un codice che viene letto e recluta altre proteine che influiscono lo stato cromatinico.
				Ogni coda \`e rivolta verso l'esterno del nucleosoma.

				\subparagraph{Ereditariet\`a}
				Durante la replicazione l'ottamero si dissocia: un dimero \emph{H3-H4} rimane legato al DNA, mentre \emph{H2A-H2B} tornano a far parte del pool di dimeri libero nella cellula.
				Il pool contiene pertanto un mix di dimeri di nuova sintesi e di quelli dissociati dal DNA, che vengono casualmente ripartiti nei due filamenti.

				\subparagraph{Dinamicit\`a}
				Il DNA in un nucleosoma isolato si svolge $4$ volte al secondo rimanendo esposto in modo da essere disponibile per il legame con altre proteine.
				Ulteriori allentamenti possono essere ottenuti attraverso i complessi di rimodellamento della cromatina.

		\subsubsection{Ulteriori livelli di impacchettamento}
		I nucleosomi sono disposti a zig-zag nel filo di perle a causa dell'istone \emph{H1} che crea angolature definite con le quali il DNA entra in contatto col nucleo istonico.
		Stabilizza inoltre la fibra cromatinica formata dalle modifiche delle code istoniche.
		Proteine chaperones aiutano il processo.
			
			\paragraph{Gradi di impacchettamento}
			\begin{multicols}{2}
				\begin{itemize}
					\item DNA a doppia elica: $2\si{nm}$.
					\item Perline su fili: $11\si{nm}$.
					\item Zig-zag o solenoide: $30\si{nm}$.
					\item Fibra di cromatina ripiegata in anse: $700\si{nm}$.
					\item Cromosoma mitotico intero: $1400\si{nm}$.
				\end{itemize}
			\end{multicols}
			
			\paragraph{Anse radiali}
			Le anse radiali impediscono la formazione di una tensione che permetterebbe il superavvolgimento del DNA.
			Sono bloccate alla base da una struttura proteica detta scaffold nucleare.

		\subsubsection{Collegamenti tra nuclesomi}
		I collegamenti nucleosoma-nucleosoma sono formati dalle code istoniche, principalmente da \emph{H4-H1}.

		\subsubsection{Effetto di posizione}
		L'eterocromatina tende ad espandersi autonomamente venendo trasmessa nelle regioni adiacenti.
		Questo causerebbe il silenziamento di geni normalmente attivi e deve pertanto essere controllato.
		Barriere proteiche impediscono la diffusione incontrollata della cromatina.
		Si nota come una situazione eterocromatica viene ereditata stabilmente dalla progenie della cellula.

		\subsubsection{Effetto della compattazione}
		La compattazione del DNA \`e importante in:
		\begin{multicols}{3}
			\begin{itemize}
				\item Replicazione.
				\item Ricombinazione.
				\item Trascrizione.
				\item Riparazione.
				\item Segregazione dei cromosomi.
			\end{itemize}
		\end{multicols}

	\subsection{Regolazione della struttura cromatinica}
	L'interazione del DNA con l'ottamero istonico \`e dinamica: le interazioni deboli con gli istoni possono spezzarsi nei momenti di necessit\`a.
	La stabilit\`a dell'interazione \`e influenzata da grossi complessi nucleici detti complessi di rimodellamento  della cromatina.

		\subsubsection{Complessi di rimodellamento della cromatina}
		I complessi di rimodellamento della cromatina sono complessi proteici dipendenti da \emph{ATP}.
		Possiedono una subunit\`a che si lega al nucleo proteico del nucleosoma sia al DNA inteorno.
		Idrolizzano \emph{ATP} causando un cambio conformazionale del nucleosoma liberando il DNA.
		Rendono pertanto dinamico lo stato cromatinico della cellula permettendole di rispondere a diversi eventi.
		
			\paragraph{Attivit\`a}
			\begin{multicols}{2}
				\begin{itemize}
					\item Scorrimento dei nucleosomi.
					\item Rimozione dei nucleosomi.
					\item Scambio di istoni.
				\end{itemize}
			\end{multicols}

			\paragraph{Facilitates chromatine transcription}
			\emph{FACT} \`e una proteina che destabilizza gli istoni, lavorano a monte della RNA polimerasi e rimuovono i dimeri \emph{H2A-H2B}.

			\paragraph{Chaperonine}
			Le chaperonine sono proteine in grado di piegare le strutture delle altre proteine in modo da renderle pi\`u funzionali.

			\paragraph{Subunit\`a di idrolisi}
			La subunit\`a di idrolisi di \emph{ATP} di questi complessi \`e correlata alle DNA elicasi e si lega sia al nucleo proteico che al DNA.
			Utilizza l'energia per spostare il DNA rispetto agli istoni allentando il legame.

		\subsubsection{Modifiche covalenti degli istoni}
		Le catene laterali degli amminoacidi degli istoni sono soggette a diverse modifiche che avvengono sulle code $N$-terminali.
		Sono create da enzimi specifici.

			\paragraph{Gruppi acetile}
			I gruppi acetile sono aggiunti da una serie di istone acetil transferasi \emph{HAT} e rimossi da complessi di istone deacetilasi \emph{DACH}.
			L'acetilazione di lisine tende ad allentare la struttura della cromatina rimuovendo la carica positiva degli istoni.

			\paragraph{Gruppi metile}
			La lisina pu\`o essere mono-, di- o tri-metilata.
			Ognuna di queste modifiche pu\`o avere un significato diverso per la cellula.

		\subsubsection{Varianti istoniche}
		La cellula pu\`o contenere varianti di istoni che vengono assemblate nei nucleosomi.
		Sono sintetizzate durante l'interfase e inserite in cromatine gi\`a formate attraverso cambio istonico.
		Questo processo \`e altamente specifico.
		\begin{multicols}{2}
			\begin{itemize}
				\item \emph{H2A.Z}: nel lievito limita l'espansione dell'eterocromatina, mentre nei mammiferi e in Drosophila promuove il silenziamento genico.
				\item \emph{H2A.X}: \`e coinvolto nella riparazione del DNA, nella forma fosforilata appare nei nucleosomi fiancheggianti le rotture a doppio filamento.
				\item \emph{macroH2A}: arricchito nei corpi di Barr e coinvolto nell'attivazione del cromosoma $X$.
				\item \emph{H2ABbd}: riduce la componente acidica.
			\end{itemize}
		\end{multicols}

		\subsubsection{Sequenze barriera}
		Una sequenza \emph{HS4} separa il dominio di cromatina attiva della $\beta$-globina da una regione adiacente di cromatina condensata.
		Viene aggiunta spesso alle estremit\`a di un gene inserito per proteggerlo dal silenziamento dovuto a diffusione di eterocromatina.

	\subsection{Epigenetica}
	Si intende per epigenetica le informazioni esterne al DNA che vengono aggiunte tramite legame con gruppi senza variare la sequenza nucleotidica della cellula e che possono essere ereditate.

		\subsubsection{Modifiche delle code istoniche}
		Le modifiche a livello delle code istoniche vengono riconosciute da domini delle proteine.
		\begin{multicols}{2}
			\begin{itemize}
				\item Bromodominio: riconosce acetilazione.
				\item Cromodominio: riconosce metilazione
			\end{itemize}
		\end{multicols}
		Le proteine possono contenere diverse combinazioni dei vari domini.

		\subsubsection{Effetti delle modifiche delle code istoniche}
		Le modifiche post-trascrizionali influenzano l'accessibilit\`a della cromatina e creano pertanto un codice epigenetico che viene letto dai complessi di rimodellamento della cromatina.
			
			\paragraph{Code metilate}
			La metilazione delle code causa uno spegnimento dell'espressione della regione interessata in quanto complessi di rimodellamento si legano sulle posizioni metilate compattando ulteriormente la struttura del cromosoma.

			\paragraph{Code acetilate}
			L'acetilazione delle code causa un'attivazione dell'espressione della regione interessata in quanto complessi di rimodellamento si legano sulle posizione rilassando il nucleosoma.

	\subsection{Cromatina dei centromeri}
	I centromeri sono costituiti da cromatina costitutiva che permane per tutta l'interfase e contiene una variante di \emph{H3} \emph{CENP-A} (centromere protein-A).
	Altre proteine compattano i nucleosomi in disposizioni dense e formano il cinetocore.

		\subsubsection{\emph{Saccharomyces cerevisiae}}
		Pi\`u di una dozzina di proteine diverse si assemblano sul centromero.
		Si forma il nucleosoma centromero-specifico formato da \emph{CENP-A-H2A-H2B-H4}.
		Le proteine addizionali lo attaccano a un microtubulo del fuso mitotico e le sequenze \emph{CDE-1,2,3} permettono il riconoscimento delle proteine necessarie: \emph{Cse4p}, \emph{CBF3}, \emph{Ctf19}, \emph{Mem21}, \emph{Okp1}.
		Mutazioni delle sequenze portano a segregazioni errate.

		\subsubsection{Umano}
		I centromeri degli organismi complessi sono definiti da un complesso di proteine e non da una sequenza specifica di DNA.
		Sono comunque presenti brevi ripetizioni DNA satellite $\alpha$ non sufficienti per la formazione del centromero.

		\subsubsection{Formazione del centromero}
		La formazione di un nuovo centromero richiede un evento di semina che comporta la formazione di una struttura DNA-proteina contenente i nucleosomi specifici.
		I tetrameri \emph{H3-H4} sono ereditati direttamente dalle eliche di DNA figlie a livello della forcella.
		L'intero centromero si forma in maniera binaria, suggerendo un'aggiunta altamente cooperativa di proteine dopo un evento iniziale.

		\subsubsection{Proteine centromeriche}

			\paragraph{\emph{CENP-A}}
			\emph{CENP-A} \`e in grado di assemblare in vitro le particelle nucleoproteine in assenza di \emph{H3}.

			\paragraph{\emph{Cse4}}
			\emph{Cse4} in S. cerevisiae \`e presente nel nucleosoma che forma il centromero puntiforme.

			\paragraph{\emph{HCP3}}
			\emph{HCP3} \`e presente in C. elegans lungo il cromosoma.

			\paragraph{\emph{Cid}}
			\emph{Cid} \`e presente in Drosophila melanogaster.

			\paragraph{\emph{CEN-H3}}
			\emph{CEN-H3} svolge un ruolo da impalcatura per l'assemblaggio corretto del cinetocore nella parte esterna del DNA centromerico oltre che per l'appaiamento dei due cromatidi fratelli nella regione centromerica.

			\paragraph{\emph{CENP-B}}
			\emph{CENP-B} localizza su centromeri umani e murini attraverso alle regioni $\alpha$-satellite.
			Organizza il DNA satellite centromerico facendogli assumere una configurazione pi\`u stabile.

			\paragraph{\emph{CENP-C}}
			\emph{CENP-C} connette la cromatina centromerica e il cinetocore ed \`e pertanto essenziale per la segregazione.

			\paragraph{\emph{CENP-H}}
			\emph{CENP-H} \`e un chaperone per la corretta associazione di \emph{CENP-A} con la regione centromerica.



		
	\subsection{Sindrome di Rett}
	
		\subsubsection{Caratteristiche cliniche}
		Le persone affette da sindrome di Rett presentano, dopo un periodo di apparente sviluppo normale:
		\begin{multicols}{3}
			\begin{itemize}
				\item Ridotte capacit\`a motorie.
				\item Riduzione della grandezza del cervello.
				\item Ritardi nello sviluppo.
			\end{itemize}
		\end{multicols}

		\subsubsection{Motivazioni biologiche}
		La malattia \`e dovuta a una mutazione a carico di \emph{MeCP2}, un repressore della trascrizione genica in regioni del DNA metilata.
		La mancata metilazione del gene per \emph{NGF} nerve growth factor impedisce la differenziazione dei neuroni e una loro corretta morfologia: non formano sinapsi e dendriti in numero sufficiente.

\section{Modifiche post-traduzionali}
Le modifiche post-traduzionali sono un meccanismo di regolazione rapido che produce modifiche reversibili o irreversibili, fisiologiche o patologiche.
Sono modifiche che avvengono sulle proteine e ne modificano l'attivit\`a e funzione, localizzazione, inibiscono o aumentano l'attivit\`a o il turnover. 
Le proteine vengono degradate con il tempo pi\`u o meno velocemente.
Le modifiche avvengono sia nel citoplasma che nel nucleo.

	\subsection{Fosforilazione}
	La fosforilazione \`e l'aggiunta di un gruppo fosfato da enzimi detti chinasi che trasferiscono un gruppo fosfato da \emph{ATP} a residui di serina, treonina e tirosina.
	\`E una modifica reversibile. 
	L'enzima che toglie il gruppo fosfato \`e detto fosfatasi e sono meno specifici.
	Pu\`o determinare cambiamenti strutturali profondi ed \`e determinante in quasi tutte le vie di trasduzione di segnali, permettendo il riconoscimento tra proteine.

	\subsection{Acetilazione}
	L'acetilazione consiste nell'aggiunta di un gruppo acetilico attraverso acetiltransferasi che viene rimosso da deacetilasi. 
	Il donatore del gruppo acetile \`e l'\emph{acetil-CoA}.
	Si formano pertanto legami covalenti con residui di lisina o N-terminali.
	L'acetilazione intramolecolare permette la creazione di modifiche epigenetiche ereditabili a livello degli istoni.


	\subsection{Metilazione}
	La metilazione consiste nell'aggiunta di un gruppo metilico attraverso metiltransferasi.
	La \emph{S-adenosilmetionina} \emph{SAM} \`e il donatore di gruppi metilici.
	Avviene tipicamente a carico di una lisina.
	L'aggiunta di metile determina un aumento dell'idrofobicit\`a.
	Il bersaglio molecolare sono gli istoni.
	La rimozione del gruppo metile o demetilazione avviene da parte di una demetilasi.

	\subsection{Glicosilazione}
	La glicosilazione \`e una modifica post-traduzionale che avviene sia durante la traduzione che successivamente.
	Ha un ruolo fondamentale nella risposta immunitaria, nella localizazione intracellulare, nella stabilit\`a e nella struttura.

		\subsubsection{N-glicosilazione}
		La $N$-glicosilazione consiste nell'aggiunta dello zucchero ad un atomo di $N$ dell'aspargina.

		\subsubsection{O-glicosilazione}
		La $O$-glicosilazione consiste dell'aggiunta dello zucchero a un atomo di ossigeno a serina, treonina, prolina e lisina.

	\subsection{Ubiquitinazione}
	L'ubiquitinazione consiste mell'aggiunta di ubiquitina, una piccola proteina conservata negli eucarioti.
	La modifica \`e irreversibile e etichetta le proteine da degradare.
	Le proteine che non sono in grado di essere ubiquitinate sono degradate attraverso degradazione del loro mRNA.
	Avviene unicamente su molecole contenenti lisina circondata da una sequenza consenso specifica.

		\subsubsection{Proteosoma}
		Il proteosoma \`e un complesso proteico di grandi dimensioni $26S$ formato da un core, $2$ cap e $2$ lid.
		Degrada le proteine ubiquitinate.
		Le proteine del lid legano i substrati da distruggere.
		Le proteine del cap le svolgono e nel core vengono degradate.
		Tutta l'operazione richiede \emph{ATP}.
		L'isopeptidasi \emph{Ppn11} rimuove la coda di poliubiquitina.

		\subsubsection{Poliubiquitina}
		Una sola molecola di ubiquitina non \`e sufficiente per indirizzare le proteine al proteosoma, ne vengono pertanto aggiunge a catena pi\`u di $4$.


	\subsection{Sumoilazione}
	Meccanismo analogo e simile all'ubiquitina. 
	Regola le funzioni della proteina, il suo attacco richiede \emph{ATP} ma il processo \`e reversibile.
	Intuisce su:
	\begin{multicols}{2}
		\begin{itemize}
			\item Traffico nucleare.
			\item Espressione genica.
			\item Stabilit\`a genomica.
			\item Integrit\`a cromosomica.
		\end{itemize}
	\end{multicols}

		\subsubsection{Sumoilazione bilanciata}
		Una sumoilazione bilanciata porta a funzioni cellulari normali.

		\subsubsection{Sumoilazione deregolata}
		Una sumoilazione deregolata porta a tumori, malattie neurodegenerarive, diabete, infezioni virali e difetti nello sviluppo.
\section{Metabolismo del ferro}
Lo ione ferro \`e in grado di reagire con l'ossigeno e creare radicali liberi che possono danneggiare la molecola di DNA.
Si rende pertanto necessario regolare il livello del ferro.

	\subsection{Introduzione del ferro nella cellula}
	Il ferro viene introdotto nella cellula grazie a recettori transferrina presenti sulla membrana citoplasmatica della cellula, che lo legano quando presente nel torrente circolatorio.
	Dopo il legame il complesso ha un $pH$ neutro $7.4
$ che causa la sua internalizzazione.
	Si forma cos\`i una vescicola endocitotica.
	Questa matura in un endosoma che si acidifica permettendo il distacco dello ione dalla transferrina.
	La transferrina torna in membrana.

	\subsubsection{Il ferro nella cellula}
	Il ferro nella cellula viene utilizzato come co-attivatore di enzimi.
	Il ferro in eccesso che si trova nella cellula viene disattivato dalla ferritina.
	Questa proteina lo lega impedendo che venga coinvolto in reazioni dannose per la cellula.


	\subsection{Meccanismo di regolazione}
	Si nota pertanto come per regolare i livelli di ferro si devono modulare le concentrazioni di transferrina e di ferritina.
	Questo avviene a livello post-trascrizionale.

		\subsubsection{Aconitasi}
		L'aconitasi \`e una proteina che lega RNA e funge da sensore per il ferro, fa parte della famiglia delle \emph{IRP} (iron regulatory proteins).
		Si lega a strutture a forcina che si forma nella \emph{$5'$-UTR} del RNA per la ferritina e \emph{$3'$-UTR} del RNA per la transferrina.

		\subsubsection{Carenza di ferro}
		In carenza di ferro la cellula deve aumentare i livelli di transferrina e diminuire quelli di ferritina.
		L'aconitasi, non legata al ferro si lega alle struttura a forcina degli mRNA o \emph{IRE} (iron responsive elements):
		\begin{itemize}
			\item Transferrina: stabilizza il suo RNA impedendone la sua degradazione.
			\item Ferritina: impedisce il legame dei ribosomi con il mRNA.
		\end{itemize}

		\subsubsection{Eccesso di ferro}
		In eccesso di ferro la cellula deve diminuire i livelli di transferrina ed aumentare quelli di ferritina.
		L'aconitasi legata al ferro si dissocia dagli \emph{IRE} degli mRNA di:
		\begin{itemize}
			\item Transferrina: destabilizza il mRNA causandone la degradazione.
			\item Ferritina: permette ai ribosomi di tradurre il mRNA.
		\end{itemize}
	
	\subsection{Metodo di osservazione}
	Per osservare il processo si ingegnerizza il gene della ferritina in modo che esprima una proteina reporter \emph{GFP}.
	In questo modo quando il ferro non \`e presente l'aconitasi si lega al RNA e blocca la traduzione impedendo alla cellula di esprimere \emph{GFP}.
	Quando invece \`e presente avviene la traduzione e la cellula esprime \emph{GFP}, che essendo fluorescente \`e facilmente osservabile.
