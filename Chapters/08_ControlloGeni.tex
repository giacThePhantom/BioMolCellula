\chapter{Controllo dell'espressione genica}
\section{Una panoramica del controllo dei geni}
I differenti tipi di cellula in un organismo cellulare differiscono per struttura e funzione a causa delle diverse proteine che sintetizzano. Non sono differenze nella loro sequenza
genica a determinarne la differenziazione ma i cambi nell'espressione dei geni.
\subsection{Diversi tipi di cellule sintetizzano diversi insiemi di RNA e proteine}
Molti processi sono comuni a tutte le cellule e due di esse in un singolo organismo condividono molti prodoti dei geni come proteine strutturali e cromosomi, RNA e DNA polimerasi, enzimi
di riparazione del DNA, proteine ribosomiali e RNA, gli enzimi che catalizzano le reazioni centrali del metabolismo e molte proteine che formano il citoscheletro. Alcuni RNA e proteine
sono abbondanti in cellule specializzate e non sono individuati in altri luoghi. Studi dei numeri degli RNA diversi suggeriscono che una tipica cellula umana ad ogni momento esprime
tra il $30$ e il $60\%$ dei suoi geni. Il livello di espressione di quasi tutti i geni varia tra un tipo di cellula e l'altro. Le differenze tra gli mRNA sottostimano le differenze 
finali nelle proteine in quanto ci sono diversi passaggi dopo la produzione dell'RNA con cui l'espressione viene regolata. 
\subsection{Segnali esterni possono causare il cambio dell'espressione dei geni di una cellula}
Ogni cellula \`e capace di alterare l'espressione dei geni in risposta a indizi extracellulari. Se una cellula del fegato \`e esposta all'ormone glucocorticoide si aumenta la produzione
di energia dagli amminoacidi e altre piccole molcole inducendo l'enzima tirosina amminotrasferasi. Altre cellule rispondono diversamente o non rispondono affatto. Altre caratteristiche
dell'espressione genica non cambiano e danno alla cellula le sue caratteristiche distintive.
\subsection{L'espressione dei geni pu\`o essere regolata a molti dei passaggi nel cammino da DNA a RNA a proteine}
La cellula pu\`o controllare la proteina che produce controllando quanto e quanto spesso viene trascritto il gene (controllo trascrizionale), controllando lo splicing e il processamento
dei trascritti di RNA (controllo del processamento dell'RNA), selezionando quale mRNA completo viene esportato al citisol e determinando dove viene localizzato (controllo di trasporto e 
localizzazione dell'RNA), selezionando quale mRNA nel citoplasma viene tradotto dal ribosoma (controllo traduzionale), destabilizzando certe molecole di mRNA nel citoplasma (controllo 
della degradazione dell'mRNA) e attivando, disattivando, degradando o localizzando selettivamente proteine dopo che sono state create (controllo dell'attivit\`a proteica).
\section{Controllo della trascrizione da parte di proteine leganti a specifiche sequenze}
Un gruppo di proteine detti regolatori di trascrizione riconoscono specifiche sequenze di DNA dette sequenze cis-regolatorie in quanto devono essere sullo stesso cromosoma del gene che
regolano. I regolatori di trascrizione si legano a queste sequenze che sono disperse attraverso il genoma e il legame \`e messo in movimento da una serie di reazioni che specificano 
i geni da trascrivere e il tasso di trascrizione. La trascrizione di ogni gene \`e controllata da una collezione di sequenze ci-regolatorie che si trovano tipicamente vicino al gen, 
tipicamente a monte del punto di inizio di trascrizione. La maggior parte ha un complesso ordinamento di tali sequenze, ognuna riconosciuta da una proteina diversa che determinano il 
tempo e il luogo di trascrizione di ogni gene. 
\subsection{La sequenza dei nucleotidi nella doppia elica di DNA pu\`o essere letta da proteine}
I regolatori di trascrizione riconoscono sequenze cis-regolatorie corte e specifiche nella doppia elica attraverso informazioni nella sua parte esterna: il lato di ogni coppia presenta
un pattern di donatori e accettori di legami a idrogeno e parti idrofobiche sia nella scanalatura maggiore che minore. Tutti i regolatori di trascrizione fanno contatto con la 
scanalatura maggiore.
\subsection{I regolatori di trascrizione contengono motivi strutturali che possono leggere sequenze di DNA}
Il riconoscimento moleoclare dipende da un adattamento esatto tra le superfici di due molecole: un regolatore di trascrizione riconosce una sequenza cis-regolatoria specifica in quanto 
la superficie della proteina \`e estensivamente complementare alle caratteristiche della doppia elica che presenta tale sequenza. Ogni regolatore di trascrizione fa una serie di contatti
con il DNA attraverso legami ionici, a idrogeno e interazioni idrofibiche che in grande numero legano le due parti in una relazione stretta e specifica. Molti regolatori di trascrizione
possiedono motivi strutturali che usano $\alpha$-eliche o $\beta$-foglietti per legarsi alla scanalatura maggiore del DNA. Le catene laterali amminoacide che si estendono fanno contatti
specifici con il DNA.
\subsection{La dimerizzazione dei regolatori di transizione aumenta la loro affinit\`a e specificit\`a per il DNA}
Un monomeri di un regolatore di trascrizione tipico riconosce circa $6$-$8$ coppie di nucleotidi, ma riconoscono un intervallo di sequenze strettamente imparentate con l'affinit\`a della
proteina variando in base a quanto la corrispondenza \`e corretta. Le sequenze cis-regolatorie mostrano l'intervallo di sequenze riconosciute da un regolatore di trascrizione 
particolare. La sequenza di DNA riconosciuta da un monomero non ha informazioni sufficienti per essere scelta da tutti le sequenze randomiche nel genoma e si rendono necessarie altre
contribuzioni per aumentare la specificit\`a. Molti regolatori di trascrizione formano dimeri con entrambi i monomeri facendo contatti molto simili con il DNA. In questo modo si 
raddoppia la lunghezza della sequenza riconosciuta e si aumenta l'affinit\`a e la specificit\`a del legame dei regolatori. Spesso si formano eterodimeri tra due regolatori diversi con 
pi\`u di una proteina in modo da creare diverse specificit\`a diverse.
\subsection{I regolatori di trascrizione si legano cooperativamente al DNA}
Nel caso pi\`u semplice l'insieme dei legami covalenti che tiene insieme i dimeri \`e sufficiente per formare queste strutture obbligatoriamente e impedendo la loro separazione, caso in
cui l'unit\`a di legament del dimero e la curva di legame per il regolatoer di trascirizione ha una forma esponenziale standard. In molti casi i dimeri e gli eterodimeri sono tenuti
insieme molto debolmente. le proteine legano il DNA cooperativamente con una curva sigmoidale. Questo vuol dire che per un intervallo di concentraizoni del regolatore di trascrizione
il legame \`e pi\`u un fenomeno binario che per legami non cooperativi, ovvero alla maggior parte delle concentrazioni la sequenza cis-regolatoria \`e o quasi vuota o quasi completamente
occupata.
\subsection{La struttura del nucelosoma promuove legami cooperativi di regolatori di trascrizione}
Esiste un meccanismo secondario e indiretto per faborire i legami cooperativi che nasce dalla struttura nucleosomica dei cromosomi eucariotici. In generale i regolatori di trascrizione
si legano al DNA nei nucleosomi con affinit\`a minore rispetto al DNA da solo in quanto la superficie potrebbe essere contro il nucleosoma non disponibile o i cambi che i regolatori 
compiono al DNA sono contrastati dal legame stretto del DNA lungo il nucleo istonico. Anche senza rimodellazione i regoaltori possono avere accesso limitato nel DNA che alla fine di 
un nucleosma si espone temporaneamente e permette il legame dei rgeolatori. Queste propriet\`a del nuclesoma promuovono legami cooperativi in quanto se un regolatore entra nel DNA di 
un nucleosoma previene il suo restringimento aumentando l'affinit\`a per un secondo reoglatore. Se i due interagiscono tra di loro l'effetto cooperativo \`e maggiore arrivando in 
alcuni casi a separare l'istone. La cooperazione aumenta quando sono coinvolti complessi di rimodellazione del nucleosoma. Se un regolatore di trascrizione si lega alla sequenza 
cis-regolatoria e attrae un complesso di rimodellazione della cromatina, l'azione localizzata di quest ultimo permette a un secondo regoaltore di legarsi efficientemente vicino. 
\section{I regolatori di trascrizione attivano e disattivano i geni}
\subsection{Il repressore triptofano disattiva i geni}
Il genoma nel batterio E. coli consiste di una molecola di DNA di $4.6\cdot 10^6$ coppie di nucleotidi e codifica $4300$ proteine. L'espressione dei geni viene regolata in base alla
disponibilit\`a di nutrienti nell'ambiente. Ci sono $5$ geni che codificano per l'amminoacido triptofano in un cluster sul cromosoma e sono trascritti da un singolo promotore su una 
molecola di mRNA dettta operone. Gli operoni sono rari negli eucarioti, dove i geni sono trascritti e regolati individualmente. Quando la concentrazione di triptofano \`e bassa l'operone
\`e trascritto producendo un insieme di enzimi biosintetici che lavorano insieme per produrre triptofano da molecole pi\`u semplici. Quando il triptofano \`e abbondante l'amminoacido \`e
\`e importato nella cellula e la produzione di questi enzimi viene bloccata. Nel promotore dell'operone \`e presente una sequenza cis-regolatrice che \`e riconosciuta da un regolatore 
di trascrizione che quando si lega alla sequenza blocca l'accesso della RNA polimerasi al promotore impedendo la trascrizione. Questo regolatore \`e detto repressore del triptofano
e la sequenza cis-regolatrice triptofano operatore. Il repressore pu\`o legarsi al DNA solo se ha anche legate molte molecole di triptofano. Il repressore \`e una proteina allosterica e 
il legame con il triptofano causa un cambio conformazionale in modo che possa legarsi alla sequenza operatore. Quando la concentrazione di triptofano diminuisce questo si disassocia dal
repressore che non pu\`o pi\`u rimanere legato al DNA.
\subsection{I repressori disattivano i geni e gli attivatori li attivano}
Le proteine repressori trascrizionali disattivano i geni o li reprimono. Alcuni regolatori attivano i geni. Questi attivatori trascrizionali lavorano con il promotore che \`e solo 
marginalmente capace di legarsi a sequenze cis-regolatorie e contatta la RNA polimerasi per aiutarla a iniziare la trascrizione. Queste proteine attivatrici legate al DNA possono 
aumentare il tasso anche di $1000$ volte. Queste proteine devono interagire con una seconda molecola per riuscire a legarsi al DNA: l'attivatore batterico CAP deve legare AMP ciclici 
prima che possa legarsi al DNA. I geni attivati da esso vengono trascritti in risposta a un aumento della concentrazione di cAMP che aumenta quando il glucosio non \`e pi\`u disponibile,
guidando la produzione di enzimi che permettono al batterio di digerire altri zuccheri. 
\subsection{Un attivatore e un repressore controllano l'operone Lac}
In molte istanze l'attivit\`a di un singolo promotore \`e controllata da diversi regolatori di trascrizione. L'operone Lac \`e controllato sia dai repressori Lac e dagli attivatori CAP. 
L'operone Lac codifica le proteine richieste per l'importazione e la digesione del lattosio. In assenza di glucosio il batterio crea cAMP che attiva CAP attivando i geni che permettono
alla cellula di utilizzare alternative fonti di carbonio come il lattosio. Il repressore Lac disattiva l'operone in assenza di lattosio. Questo permette al controllo della regione 
dell'operone Lac di integrare due segnali diversi in modo che il gene \`e altamente espresso solo quando il glucosio \`e assente e il lattosio \`e presente. Tutti i regolatori di 
trascrizione devono legarsi al DNA per sortire un effetto. In questo modo ogni proteina regolatoria agisce selettivamente, controllando solo i geni che portano una sequenza 
cis-regolatoria riconosciuta da essa.
\subsection{L'inanellamento del DNA pu\`o avvenire durante la regolaizone genica dei batteri}
Gli attivatori e i repressori sono molto simili in quanto devono riconoscere la sequenza cis-regolatorio attraverso lo stesso motivo strutturale. Alcune protein possono pertanto agire
sia come repressori che attivatori in base al loro posizionamento esatto nella sequenza. La maggior parte dei batteri hanno un genoma piccolo e compatto e le sequenze cis-regolatorie si
trovano molto vicino al punto di inizio della trascrizione, con delle eccezioni in cui \`e distante centinaia o migliaia di coppie. In questi casi il DNA \`e inanellato permettendo a un
legame proteico ad un sito distante di contattare la RNA polimerasi e il DNA si comporta come un legame aumentando la probabilit\`a che le proteine collidano. 
\subsection{Interruttori complessi controllano la trascrizione genica negli eucarioti}
I regolatori di trascrizione negli eucarioti coinvolgono molte proteine lunghe sequenze di DNA. Come nei batteri il tempo e luogo della trascrizione di un gene \`e regolato da sequenze
cis-regolatorie che sono lette dai regolatori di trascrizione che si legano a esse. Gli attivatori aiutano a legare la RNA polimerasi ad iniziare la trascrizione e i repressori bloccano
il processo. Queste interazioni sono per la maggior parte indirette con molte proteine intermediarie come gli istoni. Negli organismi multicellulari dozzine di regolatori di trascrizione
controllano un singolo gene con sequenze cis-regolatorie diffuse in decine di migliaia di paia di nucleotidi. L'inallenamento del DNA permette le proteine regolatorie di interagire tra 
di loro e con la polimerasi al promotore. Infine l'iniziazione della trascrizione deve superare il blocco imposto dai nucleosomi e strutture di livello superiore.
\subsection{Una regione di controllo di un gene eucariotico consiste di un promotore e molte sequenze cis-regolatorie}
Negli eucarioti la RNA polimerasi II trascrive tutti i geni codificatori delle proteine e molti non codificatori e richiede $5$ fattori di trascrizione generali il cui assemblaggio mette
a disposizione multipli passi per il regolamento della trascrizione. Si usa il termine regione del controllo del gene per descrivere l'intera lunghezza di DNA coinvolta nella regolazione
e iniziazione della trascrizione che include il promotore, dove i fattori generali di trascrizione e la polimerasi si assemblano e tutte le sequenze cis-regolatorie in cui i regolatori
si legano per controllare il tasso di assemblaggio al promotore. Alcune parti delle regioni regolatorie sono trascritte come lncRNA e si possono considerare come sequenze spaziatrici
che i regolatori di trascrizione non riconoscono direttamente. In contrasto con il piccolo numero di fattori generali di trascrizione ci sono migliaia di diversi regolatori di 
trascrizione che attivano o disattivano un singolo gene. Negli eucarioti gli operoni sono rari e ogni gene \`e regolato individualmente. 
\subsection{I regolatori di trascrizione eucariotici lavorano in gruppi}
I regolatori di trascrizione eucariotici si assemblano in gruppi alle loro sequenze cis-regolatorie con spesso interazioni cooperative. Oltre a questi sono presenti proteine 
coattivatrici o corepressori che si assemblano sul DNA con essi e non riconoscono specifiche sequenze di DNA e sono portati in queste posizioni dai regolatori di trascrizione. Queste
interazioni proteina-proteina sono troppo deboli per avvenire in soluzione ma possono cristallizzarsi con l'appropriata combinazione di sequenze cis-regolatrici. I coattivatori e i 
corepressori influenzano la trascrizione dopo che sono stati localizzati sul genoma dai regolatori. Un regolatore pu\`o partecipare in pi\`u di un tipo di complesso regolatorio e 
funzionano pertanto come parti di un complesso che reprime la trascrizione. Ogni gene eucariotico \`e regolato da un insieme di proteine che devono essere tutte presenti per esprimere
il gene al livello appropriato.
\subsection{Le proteine attivatrici promuovono l'assemblaggio di RNA polimerasi al punto di inizio della trascrizione}
Le sequenze cis-regolatorie in cui le proteine attivatrici si legano aumentano il tasso di iniziazione della trascrizione attraendo e posizionando RNA polimerasi II al promotore e 
rilasciarla in modo che il processo inizi. Alcune proteine attivatrici si legano direttamente a fattori di trascrizione generali velocizzando il loro assemblaggio al promotore che \`e
stato portato in prossimit\`a grazie all'inallenamento del DNA. La maggior parte degli attivatori attraggono coattivatori che poi svolgono il compito biochimico per iniziare la 
trascrizione. Uno di questi \`e il complesso di proteine mediatore composto da pi\`u di $30$ subunit\`a e serve come ponte tra gli attivatori legati al DNA, la RNA polimearsi e i 
fattori di trascrizione generali facilitando il loro assemblaggio al promotore.
\subsection{Gli attivatori di trascrizione eucariotici direazionano la modifica di strutture cromatiniche locali}
I fattori generali di trascrizione e l'RNA polimerasi sono incapaci di assemblarsi su un promotore che \`e condensato in un nucleosoma. Gli attivatori devono pertanto svolgere questo 
compito causando cambi alla struttura cromatinica rendendo il DNA pi\`u accessibile. Il modo pi\`u importante per compiere questo \`e attraverso cambi locali con modifiche covalenti 
degli istoni, rimodellamento dei nucleosomi, la loro rimozione e sostituzione. Gli attivatori usano tutti questi meccanismi e attraggono coattivatori che includono enzimi di modifica 
degli istoni, complessi di rimodellamento della cromatina dipendenti dall'ATP e accompagnatori di istoni, onguno dei quali pu\`o alterare la struttura cromatinica del promotore. 
L'alterazione della struttura cromatinica pu\`o persistere per diverso tempo: in alcuni casi le modifiche sono eliminate appena i regolatori si disassociano dal DNA, cosa importante per
geni che la cellula deve rapidamente attivare o disattivare in risposta a segnali esterni. In altri casi la struttura persiste in modo da estendere questa informazione alle generazioni
successive e modificare la memoria dei pattern di espressione genica. Un tipo speciale di modifica cromatinica accade quando l'RNA polimerasi trascrive un gene: gli istoni dopo la 
polimerasi possono essere acetilati da enzimi, rimossi dagli accompagnatori degli istoni e depositati dietro la polimerasi. Questi istoni sono poi deacilati e metilati rapidamente da
complessi trasportati dalla polimerasi lasciando nucleosomi resistenti a trascrizione. Questo processo sembra prevenisca reiniziazioni spurie dietro a una polimerasi che deve crearsi
una via attraverso la cromatina.
\subsection{Gli attivatori di trascrizione possono promuoverla rilasciando la RNA polimerasi dai promotori}
In alcuni casi l'inizazione della trascrizione richiede che un attivatore di trascrizione legato al DNA rilasci la RNA polimerasi dal promotore in modo da permettere l'inizio della
trascrizione. IN altri casi la polimerasi si blocca dopo aver trascritto $50$ nucelotidi e ulteriore allungamento richiede un attivatore legato dietro a essa. Queste polimerasi bloccate
sono comuni negli esseri umani. Il rilascio della RNA polimerasi pu\`o avvenire attraverso un complesso di rimodellamento della cromatina che rimuove un blocco nucleosomico o attraverso
un attivatore che comunica con la RNA polimerasi tipicamente attraverso un coattivatore segnalandole di procedere. Infine i fattori di allungamento permettono alla polimerasi di 
trascrivere attraverso la cromatina che in alcuni casi \`e il passo chiave per il controllo. Una volta che i fattori di allungamento sono caricati permettono alla polimerasi di muoversi
attraverso i blocchi cromatinici. Avere una RNA polimerasi gi\`a posizionata ad un promotore permette di bypassare il passaggio di assemblaggio e di rispondere pi\`u velocemente a 
segnali esterni.
\subsection{Gli attivatori di trascrizione lavorano sinergeticamente}
I complessi di attivatori e coattivatori si assemblano cooperativamente sul DNA e possono promuovere diversi passi dell'iniziazione di trascrizione. L'aumento del tasso di reazione 
finale dovuto a un insieme di attivatori \`e il prodotto dei contributi dei singoli. Si nota pertanto come gli attivatori esibiscano sinergia trascrizionale. 
\subsection{Repressori di trascrizione eucariotici possono inibire la trascrizione in molti modi}
I repressori di trascrizione possono diminuire il tasso di trascrizione e disattivare geni rapidamente. Grandi regioni di genoma possono essere disattivate dalla condensazione del DNA
in forme cromatiniche specialmente resistenti, ma la maggior parte non sono organizzati per funzione e pertanto non \`e una strategia sempre disponibile. La maggior parte dei repressori
lavorano gene per gene. Non competono direttamente con la RNA polimerasi per l'accesso al DNA ma usano molti meccanismi che bloccano la trascrizione. Tipicamente agiscono portando un
corepressore al DNA. I repressori possono agire su pi\`u di un meccanismo ad un gene. La repressione \`e specialmente importante per organismi la cui crescita dipende da programm di 
sviluppo elaborati e complessi. 
\subsection{Sequenze di DNA isolanti impediscono a regolatori di influenzare geni distanti}
Per evitare che regolatori di geni diversi si influiscano tra di loro elementi del DNA compartimentalizzano il DNA in domini regolatori discreti. SOno presenti sequenze barriera che
impediscono la diffusione di eterocromatina in geni che devono essere espressi. Un isolatore impedisce a sequenze cis-regolatorie di attivare geni inappropriati. FUnzionano formando 
anelli di cromatina attraverso proteine specializzate che si legano a loro. Gli anelli tengono un gene e la sua regione di controllo in prossimit\`a e aiutano a prevenire l'uscita di 
una regione di controllo in geni adiacenti. La distribuzione degli isolatori e sequenze barriera si pensa divida il genoma in domini indipendenti di regolazione genica e struttura 
cromatinica. 
\subsection{Meccanismi genetici molecolari che creano e mantengono tipi di cellula specializzati}
Le cellule di organismi multicellulari mantengono la scelta della differenziazione in un tipo di cellula specifico, ricordando i cambi nell'espressione genica coinvolti nella scelta.
Questa memoria \`e un prerequisito per la creazione di tessuti organizzati e per il mantenimento di tipi cellulari stabilmente differenziati. 
\subsection{Complessi interruttori genici che regolano lo sviluppo della Drosophila sono creati da molecole pi\`u piccole}
Considerando il gene Even-skipped (Eve) della Drosophila la cui espressione \`e fondamentale per lo sviluppo dell'embrione si nota come se il gene \`e disattivato per mutazione, molte
parti dell'embrione non si formano e questo muore. Quando Eve comincia ad essere espressio l'embrione \`e una singola cellula gigante contenente nuclei multipli in un citoplasma comune 
che contiene anche un insieme di regolatori di trascrizione distribuiiti lungo la lunghezza dell'embrione fornendo informazioni posizionali che distinguono una parte dell'embrione 
dall'altra. I nuclei cominciano rapidamente ad esprimere geni diversi in quanto sono esposti a differenti regolatori. Le sequenze di DNA regolatorie che controllano il gene Eve si 
sono evolute per leggere le concentrazioni di regolatori di trascrizione ad ogni posizione e causano l'espressione del gene in sette fascie precisamente posizionate, ognuna larga tra
$5$ e $6$ nuclei. La regione reoglatoria del gene Eve \`e molto lunga ed \`e formata da una serie di moduli regolatori con una semplice sequenza cis-regolatoria responsabile per la
specificazione di una particolare fascia di espressione lungo l'embrione.
\subsection{Il gene Eve della Drosophila \`e regolato da controlli combinatori}
Il modello del modulo regolatorio della seconda facia contiene sequenze di riconoscimento per due regolatori di trascrizione (Bicoid e Hunchback) che attivano la trascrizione di Eve e 
per altri due (K\"uppel e Giant) che la reprimono. La concentrazione relativa di queste quattro proteine determina se il complesso presente alla fascia attiva la trascizionie. Tale 
elemento, come quelli di tutte le altre fasce sono autonomi. L'intera regione di controllo lega pi\`u di $20$ regolatori di trascrizione. $7$ combinazioni di regolatori ne specificano
l'espressione, mentre altre combinazioni lo mantengono silente. La regione di controllo \`e pertanto costituita da una serie di moduli pi\`u piccoli, ognuno dei quali consiste di un
unico ordinamento di corte sequenze cis-regolatorie riconosciute da regolatori di trascrizione specifici. Il gene Eve codifica un regolatore di trascrizione che controlla l'espressione
di altri geni. L'embrione viene divisio in regioni sempre pi\`u specifiche fino a che costituisce le parti del corpo della Drosophila. 
\subsection{I regolatori di trascrizione sono messi ingioco da segnali extracellulari}
Nella maggior parte degli embrioni e nelle cellule adulte i nuclei si trovano in cellule separate e si necessita un passaggio di informazioni extracellulari attraverso la membrana
plasmatica per generare segnali nel citosol che causano l'attivazione di diversi regolatori. 
\subsection{Il controllo dei geni combinatorio diversi tipi di cellule}
Ogni regolatore di trascrizione in un organismo contribuisce al controllo di molti geni. A causa di questo controllo combinatorio un regolatore di trascrizione non ha una funzione 
definita come un comandante di uno specifico insieme di geni o di un tipo cellulare, ma \`e la sua combinazione relativa con altri a stabilire quest'informazione. Il controllo genico
combinatorio causa l'addizione di un nuoveo regolatore di trascrizione che dipende dalla storia precedente. Durante lo sviluppo una cellula accumula una serie di regolatori che
ne alterano l'espressione genica solo quando l'ultimo membro presente viene aggiunto. Questo meccanismo permette ad alcune cellule di convertire da un tipo all'altro. 
\subsection{Combinazioni di regolatori di trascrizione master specificano il tempo della cellula controllando l'espressione di molti geni}
I pattern di espressione del tipo cellulare sono determinati da una combinazione di regolatori di trascrizione master che in molti casi si legano direttamente a sequenze cis-regolatoire
del gene. MyiD si lega direttamente alla sequenza cis-regolatoria delle regioni di controllo sui geni specifici ai muscoli. In altri casi i regolatori master controllano l'espressione
di regolatori a valle che si legano alle regioni di controllo di altri geni specifici e controllano la loro sintesi. La specifica di un particolare tipo di cellula coinvolge cambi 
nell'espressione di migliaia di geni: quelli richiesti dal particolare tipo sono prodotti, gli altri no. 
\subsection{Cellule specializzate devono rapidamente attivare e disattivare insiemi di geni}
Le cellule specializzate devono rispondere ai cambi nell'ambiente come segnali da altre cellule che coordinano il comportamento dell'intero organismo. Molti di questi segnali inducono
cambi temporanei alla trascrizione dei geni. L'effetto di un singolo regolatore pu\`o avere effetto, completando la combinazione necessaria per attivare o reprimere il gene. Un esempio 
\`e la proteina glucocorticoide-recettrice umana. Per elgarsi alla sua sequenza cis-regolatoria deve prima formare un complesso con un ormone glucocorticoide steriode come il cortisolo
che viene rilasciato dal corpo durante periodi di fame e intensa attivit\`a ficia, stimolando le cellule del fegato ad aumentare la produzione di glucosio dagli amminoacidi. Tali 
cellule aumentano l'espressione di molti geni che codificano gli enzimi metabolici. Nonostante questi geni abbiano regioni di controllo diverse e complesse la loro espressione massimale
dipende dal legame del complesso ormone-glucocorticoide alla sua sequenza cis-regolatoria, presente nella regione di controllo di ogni gene. 
\subsection{Cellule differenziate mantengono la loro identit\`a}
Una volta che una cellula si \`e differenziata generalmente rimarr\`a differenziata e tutta la sua progenie rimarr\`a dello stesso tipo. Alcune cellule altamente specializzate non si
dividono pi\`u, ovvero si differenziano terminalmente. Molte altre cellule differenziate si dividono molte volte nella vita di un individuo generando progenie dello stesso tipo. 
Affinch\`e il tipo venga mantenuto (memoria cellulare) i pattern di espressione genica responsabili devono essere mantenuti e passati alle figlie durante le divisioni. Questo 
viene ottenuto attraverso un loop a feedback positivo, dove un regolatore di trascrizione master attiva la trascrizione per il proprio gene e a tutti gli altri specifici al sito. Ogni
volta che la cellula si divide questo viene passato alle figlie. 
\subsection{Circuiti di trascrizione permettono alla cellula di compiere operazioni logiche}
Semplici regolatori possono essere combinati per creare dispositivi di controllo che se ordinati in motivi di rete possono essere trovati continuamente in cellule di specie diverse. 
Sono comuni anelli di feedback positivi e negativi. Con pi\`u regolatori i comportamenti dei circuiti possono diventari pi\`u complessi come flip-flops, loop a feed-forward. 
\section{Meccanismi che rinforzano la memoria cellulare in piante e animali}
