\chapter{Controllo dell'espressione genica}
\section{Una panoramica del controllo dei geni}
I differenti tipi di cellula in un organismo cellulare differiscono per struttura e funzione a causa delle diverse proteine che sintetizzano. Non sono differenze nella loro sequenza
genica a determinarne la differenziazione ma i cambi nell'espressione dei geni.
\subsection{Diversi tipi di cellule sintetizzano diversi insiemi di RNA e proteine}
Molti processi sono comuni a tutte le cellule e due di esse in un singolo organismo condividono molti prodoti dei geni come proteine strutturali e cromosomi, RNA e DNA polimerasi, enzimi
di riparazione del DNA, proteine ribosomiali e RNA, gli enzimi che catalizzano le reazioni centrali del metabolismo e molte proteine che formano il citoscheletro. Alcuni RNA e proteine
sono abbondanti in cellule specializzate e non sono individuati in altri luoghi. Studi dei numeri degli RNA diversi suggeriscono che una tipica cellula umana ad ogni momento esprime
tra il $30$ e il $60\%$ dei suoi geni. Il livello di espressione di quasi tutti i geni varia tra un tipo di cellula e l'altro. Le differenze tra gli mRNA sottostimano le differenze 
finali nelle proteine in quanto ci sono diversi passaggi dopo la produzione dell'RNA con cui l'espressione viene regolata. 
\subsection{Segnali esterni possono causare il cambio dell'espressione dei geni di una cellula}
Ogni cellula \`e capace di alterare l'espressione dei geni in risposta a indizi extracellulari. Se una cellula del fegato \`e esposta all'ormone glucocorticoide si aumenta la produzione
di energia dagli amminoacidi e altre piccole molcole inducendo l'enzima tirosina amminotrasferasi. Altre cellule rispondono diversamente o non rispondono affatto. Altre caratteristiche
dell'espressione genica non cambiano e danno alla cellula le sue caratteristiche distintive.
\subsection{L'espressione dei geni pu\`o essere regolata a molti dei passaggi nel cammino da DNA a RNA a proteine}
La cellula pu\`o controllare la proteina che produce controllando quanto e quanto spesso viene trascritto il gene (controllo trascrizionale), controllando lo splicing e il processamento
dei trascritti di RNA (controllo del processamento dell'RNA), selezionando quale mRNA completo viene esportato al citisol e determinando dove viene localizzato (controllo di trasporto e 
localizzazione dell'RNA), selezionando quale mRNA nel citoplasma viene tradotto dal ribosoma (controllo traduzionale), destabilizzando certe molecole di mRNA nel citoplasma (controllo 
della degradazione dell'mRNA) e attivando, disattivando, degradando o localizzando selettivamente proteine dopo che sono state create (controllo dell'attivit\`a proteica).
\section{Controllo della trascrizione da parte di proteine leganti a specifiche sequenze}
Un gruppo di proteine detti regolatori di trascrizione riconoscono specifiche sequenze di DNA dette sequenze cis-regolatorie in quanto devono essere sullo stesso cromosoma del gene che
regolano. I regolatori di trascrizione si legano a queste sequenze che sono disperse attraverso il genoma e il legame \`e messo in movimento da una serie di reazioni che specificano 
i geni da trascrivere e il tasso di trascrizione. La trascrizione di ogni gene \`e controllata da una collezione di sequenze ci-regolatorie che si trovano tipicamente vicino al gen, 
tipicamente a monte del punto di inizio di trascrizione. La maggior parte ha un complesso ordinamento di tali sequenze, ognuna riconosciuta da una proteina diversa che determinano il 
tempo e il luogo di trascrizione di ogni gene. 
\subsection{La sequenza dei nucleotidi nella doppia elica di DNA pu\`o essere letta da proteine}
I regolatori di trascrizione riconoscono sequenze cis-regolatorie corte e specifiche nella doppia elica attraverso informazioni nella sua parte esterna: il lato di ogni coppia presenta
un pattern di donatori e accettori di legami a idrogeno e parti idrofobiche sia nella scanalatura maggiore che minore. Tutti i regolatori di trascrizione fanno contatto con la 
scanalatura maggiore.
\subsection{I regolatori di trascrizione contengono motivi strutturali che possono leggere sequenze di DNA}
Il riconoscimento moleoclare dipende da un adattamento esatto tra le superfici di due molecole: un regolatore di trascrizione riconosce una sequenza cis-regolatoria specifica in quanto 
la superficie della proteina \`e estensivamente complementare alle caratteristiche della doppia elica che presenta tale sequenza. Ogni regolatore di trascrizione fa una serie di contatti
con il DNA attraverso legami ionici, a idrogeno e interazioni idrofibiche che in grande numero legano le due parti in una relazione stretta e specifica. Molti regolatori di trascrizione
possiedono motivi strutturali che usano $\alpha$-eliche o $\beta$-foglietti per legarsi alla scanalatura maggiore del DNA. Le catene laterali amminoacide che si estendono fanno contatti
specifici con il DNA.
\subsection{La dimerizzazione dei regolatori di transizione aumenta la loro affinit\`a e specificit\`a per il DNA}
Un monomeri di un regolatore di trascrizione tipico riconosce circa $6$-$8$ coppie di nucleotidi, ma riconoscono un intervallo di sequenze strettamente imparentate con l'affinit\`a della
proteina variando in base a quanto la corrispondenza \`e corretta. Le sequenze cis-regolatorie mostrano l'intervallo di sequenze riconosciute da un regolatore di trascrizione 
particolare. La sequenza di DNA riconosciuta da un monomero non ha informazioni sufficienti per essere scelta da tutti le sequenze randomiche nel genoma e si rendono necessarie altre
contribuzioni per aumentare la specificit\`a. Molti regolatori di trascrizione formano dimeri con entrambi i monomeri facendo contatti molto simili con il DNA. In questo modo si 
raddoppia la lunghezza della sequenza riconosciuta e si aumenta l'affinit\`a e la specificit\`a del legame dei regolatori. Spesso si formano eterodimeri tra due regolatori diversi con 
pi\`u di una proteina in modo da creare diverse specificit\`a diverse.
\subsection{I regolatori di trascrizione si legano cooperativamente al DNA}
Nel caso pi\`u semplice l'insieme dei legami covalenti che tiene insieme i dimeri \`e sufficiente per formare queste strutture obbligatoriamente e impedendo la loro separazione, caso in
cui l'unit\`a di legament del dimero e la curva di legame per il regolatoer di trascirizione ha una forma esponenziale standard. In molti casi i dimeri e gli eterodimeri sono tenuti
insieme molto debolmente. le proteine legano il DNA cooperativamente con una curva sigmoidale. Questo vuol dire che per un intervallo di concentraizoni del regolatore di trascrizione
il legame \`e pi\`u un fenomeno binario che per legami non cooperativi, ovvero alla maggior parte delle concentrazioni la sequenza cis-regolatoria \`e o quasi vuota o quasi completamente
occupata.
\subsection{La struttura del nucelosoma promuove legami cooperativi di regolatori di trascrizione}
Esiste un meccanismo secondario e indiretto per faborire i legami cooperativi che nasce dalla struttura nucleosomica dei cromosomi eucariotici. In generale i regolatori di trascrizione
si legano al DNA nei nucleosomi con affinit\`a minore rispetto al DNA da solo in quanto la superficie potrebbe essere contro il nucleosoma non disponibile o i cambi che i regolatori 
compiono al DNA sono contrastati dal legame stretto del DNA lungo il nucleo istonico. Anche senza rimodellazione i regoaltori possono avere accesso limitato nel DNA che alla fine di 
un nucleosma si espone temporaneamente e permette il legame dei rgeolatori. Queste propriet\`a del nuclesoma promuovono legami cooperativi in quanto se un regolatore entra nel DNA di 
un nucleosoma previene il suo restringimento aumentando l'affinit\`a per un secondo reoglatore. Se i due interagiscono tra di loro l'effetto cooperativo \`e maggiore arrivando in 
alcuni casi a separare l'istone. La cooperazione aumenta quando sono coinvolti complessi di rimodellazione del nucleosoma. Se un regolatore di trascrizione si lega alla sequenza 
cis-regolatoria e attrae un complesso di rimodellazione della cromatina, l'azione localizzata di quest ultimo permette a un secondo regoaltore di legarsi efficientemente vicino. 
\section{I regolatori di trascrizione attivano e disattivano i geni}
