\chapter{Controllo dell'espressione genica}
\section{Una panoramica del controllo dei geni}
I differenti tipi di cellula in un organismo cellulare differiscono per struttura e funzione a causa delle diverse proteine che sintetizzano. Non sono differenze nella loro sequenza
genica a determinarne la differenziazione ma i cambi nell'espressione dei geni.
\subsection{Diversi tipi di cellule sintetizzano diversi insiemi di RNA e proteine}
Molti processi sono comuni a tutte le cellule e due di esse in un singolo organismo condividono molti prodotti dei geni come proteine strutturali e cromosomi, RNA e DNA polimerasi, enzimi
di riparazione del DNA, proteine ribosomiali e RNA, gli enzimi che catalizzano le reazioni centrali del metabolismo e molte proteine che formano il citoscheletro. Alcuni RNA e proteine
sono abbondanti in cellule specializzate e non sono individuati in altri luoghi. Studi dei numeri degli RNA diversi suggeriscono che una tipica cellula umana ad ogni momento esprime
tra il $30$ e il $60\%$ dei suoi geni. Il livello di espressione di quasi tutti i geni varia tra un tipo di cellula e l'altro. Le differenze tra gli mRNA sottostimano le differenze 
finali nelle proteine in quanto ci sono diversi passaggi dopo la produzione dell'RNA con cui l'espressione viene regolata. 
\subsection{Segnali esterni possono causare il cambio dell'espressione dei geni di una cellula}
Ogni cellula \`e capace di alterare l'espressione dei geni in risposta a indizi extracellulari. Se una cellula del fegato \`e esposta all'ormone glucocorticoide si aumenta la produzione
di energia dagli amminoacidi e altre piccole molecole inducendo l'enzima tirosina amminotrasferasi. Altre cellule rispondono diversamente o non rispondono affatto. Altre caratteristiche
dell'espressione genica non cambiano e danno alla cellula le sue caratteristiche distintive.
\subsection{L'espressione dei geni pu\`o essere regolata a molti dei passaggi nel cammino da DNA a RNA a proteine}
La cellula pu\`o controllare la proteina che produce controllando quanto e quanto spesso viene trascritto il gene (controllo trascrizionale), controllando lo splicing e il processamento
dei trascritti di RNA (controllo del processamento dell'RNA), selezionando quale mRNA completo viene esportato al citosol e determinando dove viene localizzato (controllo di trasporto e 
localizzazione dell'RNA), selezionando quale mRNA nel citoplasma viene tradotto dal ribosoma (controllo traduzionale), destabilizzando certe molecole di mRNA nel citoplasma (controllo 
della degradazione dell'mRNA) e attivando, disattivando, degradando o localizzando selettivamente proteine dopo che sono state create (controllo dell'attivit\`a proteica).
\section{Controllo della trascrizione da parte di proteine leganti a specifiche sequenze}
Un gruppo di proteine detti regolatori di trascrizione riconoscono specifiche sequenze di DNA dette sequenze cis-regolatorie in quanto devono essere sullo stesso cromosoma del gene che
regolano. I regolatori di trascrizione si legano a queste sequenze che sono disperse attraverso il genoma e il legame \`e messo in movimento da una serie di reazioni che specificano 
i geni da trascrivere e il tasso di trascrizione. La trascrizione di ogni gene \`e controllata da una collezione di sequenze cis-regolatorie che si trovano tipicamente vicino al gene, 
tipicamente a monte del punto di inizio di trascrizione. La maggior parte ha un complesso ordinato di tali sequenze, ognuna riconosciuta da una proteina diversa che determinano il 
tempo e il luogo di trascrizione di ogni gene. 
\subsection{La sequenza dei nucleotidi nella doppia elica di DNA pu\`o essere letta da proteine}
I regolatori di trascrizione riconoscono sequenze cis-regolatorie corte e specifiche nella doppia elica attraverso informazioni nella sua parte esterna: il lato di ogni coppia presenta
un pattern di donatori e accettori di legami a idrogeno e parti idrofobiche sia nella scanalatura maggiore che minore. Tutti i regolatori di trascrizione fanno contatto con la 
scanalatura maggiore.
\subsection{I regolatori di trascrizione contengono motivi strutturali che possono leggere sequenze di DNA}
Il riconoscimento molecolare dipende da un adattamento esatto tra le superfici di due molecole: un regolatore di trascrizione riconosce una sequenza cis-regolatoria specifica in quanto 
la superficie della proteina \`e estensivamente complementare alle caratteristiche della doppia elica che presenta tale sequenza. Ogni regolatore di trascrizione fa una serie di contatti
con il DNA attraverso legami ionici, a idrogeno e interazioni idrofobiche che in grande numero legano le due parti in una relazione stretta e specifica. Molti regolatori di trascrizione
possiedono motivi strutturali che usano $\alpha$-eliche o $\beta$-foglietti per legarsi alla scanalatura maggiore del DNA. Le catene laterali amminoacide che si estendono fanno contatti
specifici con il DNA.
\subsection{La dimerizzazione dei regolatori di transizione aumenta la loro affinit\`a e specificit\`a per il DNA}
Un monomero di un regolatore di trascrizione tipico riconosce circa $6$-$8$ coppie di nucleotidi, ma riconoscono un intervallo di sequenze strettamente imparentate con l'affinit\`a della
proteina variando in base a quanto la corrispondenza \`e corretta. Le sequenze cis-regolatorie mostrano l'intervallo di sequenze riconosciute da un regolatore di trascrizione 
particolare. La sequenza di DNA riconosciuta da un monomero non ha informazioni sufficienti per essere scelta da tutti le sequenze randomiche nel genoma e si rendono necessarie altre
contribuzioni per aumentare la specificit\`a. Molti regolatori di trascrizione formano dimeri con entrambi i monomeri facendo contatti molto simili con il DNA. In questo modo si 
raddoppia la lunghezza della sequenza riconosciuta e si aumenta l'affinit\`a e la specificit\`a del legame dei regolatori. Spesso si formano eterodimeri tra due regolatori diversi con 
pi\`u di una proteina in modo da creare diverse specificit\`a diverse.
\subsection{I regolatori di trascrizione si legano cooperativamente al DNA}
Nel caso pi\`u semplice l'insieme dei legami covalenti che tiene insieme i dimeri \`e sufficiente per formare queste strutture obbligatoriamente e impedendo la loro separazione, caso in
cui l'unit\`a di legame del dimero e la curva di legame per il regolatore di trascrizione ha una forma esponenziale standard. In molti casi i dimeri e gli eterodimeri sono tenuti
insieme molto debolmente. le proteine legano il DNA cooperativamente con una curva sigmoidale. Questo vuol dire che per un intervallo di concentrazioni del regolatore di trascrizione
il legame \`e pi\`u un fenomeno binario che per legami non cooperativi, ovvero alla maggior parte delle concentrazioni la sequenza cis-regolatoria \`e o quasi vuota o quasi completamente
occupata.
\subsection{La struttura del nucelosoma promuove legami cooperativi di regolatori di trascrizione}
Esiste un meccanismo secondario e indiretto per favorire i legami cooperativi che nasce dalla struttura nucleosomica dei cromosomi eucariotici. In generale i regolatori di trascrizione
si legano al DNA nei nucleosomi con affinit\`a minore rispetto al DNA da solo in quanto la superficie potrebbe essere contro il nucleosoma non disponibile o i cambi che i regolatori 
compiono al DNA sono contrastati dal legame stretto del DNA lungo il nucleo istonico. Anche senza rimodellazione i regolatori possono avere accesso limitato nel DNA che alla fine di 
un nucleosma si espone temporaneamente e permette il legame dei regolatori. Queste propriet\`a del nuclesoma promuovono legami cooperativi in quanto se un regolatore entra nel DNA di 
un nucleosoma previene il suo restringimento aumentando l'affinit\`a per un secondo regolatore. Se i due interagiscono tra di loro l'effetto cooperativo \`e maggiore arrivando in 
alcuni casi a separare l'istone. La cooperazione aumenta quando sono coinvolti complessi di rimodellazione del nucleosoma. Se un regolatore di trascrizione si lega alla sequenza 
cis-regolatoria e attrae un complesso di rimodellazione della cromatina, l'azione localizzata di quest ultimo permette a un secondo regolatore di legarsi efficientemente vicino. 
\section{I regolatori di trascrizione attivano e disattivano i geni}
\subsection{Il repressore triptofano disattiva i geni}
Il genoma nel batterio E. coli consiste di una molecola di DNA di $4.6\cdot 10^6$ coppie di nucleotidi e codifica $4300$ proteine. L'espressione dei geni viene regolata in base alla
disponibilit\`a di nutrienti nell'ambiente. Ci sono $5$ geni che codificano per l'amminoacido triptofano in un cluster sul cromosoma e sono trascritti da un singolo promotore su una 
molecola di mRNA detta operone. Gli operoni sono rari negli eucarioti, dove i geni sono trascritti e regolati individualmente. Quando la concentrazione di triptofano \`e bassa l'operone
\`e trascritto producendo un insieme di enzimi biosintetici che lavorano insieme per produrre triptofano da molecole pi\`u semplici. Quando il triptofano \`e abbondante l'amminoacido \`e
importato nella cellula e la produzione di questi enzimi viene bloccata. Nel promotore dell'operone \`e presente una sequenza cis-regolatrice che \`e riconosciuta da un regolatore 
di trascrizione che quando si lega alla sequenza blocca l'accesso della RNA polimerasi al promotore impedendo la trascrizione. Questo regolatore \`e detto repressore del triptofano
e la sequenza cis-regolatrice triptofano operatore. Il repressore pu\`o legarsi al DNA solo se ha anche legate molte molecole di triptofano. Il repressore \`e una proteina allosterica e 
il legame con il triptofano causa un cambio conformazionale in modo che possa legarsi alla sequenza operatore. Quando la concentrazione di triptofano diminuisce questo si disassocia dal
repressore che non pu\`o pi\`u rimanere legato al DNA.
\subsection{I repressori disattivano i geni e gli attivatori li attivano}
Le proteine repressori trascrizionali disattivano i geni o li reprimono. Alcuni regolatori attivano i geni. Questi attivatori trascrizionali lavorano con il promotore che \`e solo 
marginalmente capace di legarsi a sequenze cis-regolatorie e contatta la RNA polimerasi per aiutarla a iniziare la trascrizione. Queste proteine attivatrici legate al DNA possono 
aumentare il tasso anche di $1000$ volte. Queste proteine devono interagire con una seconda molecola per riuscire a legarsi al DNA: l'attivatore batterico CAP deve legare AMP ciclici 
prima che possa legarsi al DNA. I geni attivati da esso vengono trascritti in risposta a un aumento della concentrazione di cAMP che aumenta quando il glucosio non \`e pi\`u disponibile,
guidando la produzione di enzimi che permettono al batterio di digerire altri zuccheri. 
\subsection{Un attivatore e un repressore controllano l'operone Lac}
In molte istanze l'attivit\`a di un singolo promotore \`e controllata da diversi regolatori di trascrizione. L'operone Lac \`e controllato sia dai repressori Lac e dagli attivatori CAP. 
L'operone Lac codifica le proteine richieste per l'importazione e la digestione del lattosio. In assenza di glucosio il batterio crea cAMP che attiva CAP attivando i geni che permettono
alla cellula di utilizzare alternative fonti di carbonio come il lattosio. Il repressore Lac disattiva l'operone in assenza di lattosio. Questo permette al controllo della regione 
dell'operone Lac di integrare due segnali diversi in modo che il gene \`e altamente espresso solo quando il glucosio \`e assente e il lattosio \`e presente. Tutti i regolatori di 
trascrizione devono legarsi al DNA per sortire un effetto. In questo modo ogni proteina regolatoria agisce selettivamente, controllando solo i geni che portano una sequenza 
cis-regolatoria riconosciuta da essa.
\subsection{L'inanellamento del DNA pu\`o avvenire durante la regolazione genica dei batteri}
Gli attivatori e i repressori sono molto simili in quanto devono riconoscere la sequenza cis-regolatorio attraverso lo stesso motivo strutturale. Alcune protein possono pertanto agire
sia come repressori che attivatori in base al loro posizionamento esatto nella sequenza. La maggior parte dei batteri hanno un genoma piccolo e compatto e le sequenze cis-regolatorie si
trovano molto vicino al punto di inizio della trascrizione, con delle eccezioni in cui \`e distante centinaia o migliaia di coppie. In questi casi il DNA \`e inanellato permettendo a un
legame proteico ad un sito distante di contattare la RNA polimerasi e il DNA si comporta come un legame aumentando la probabilit\`a che le proteine collidano. 
\subsection{Interruttori complessi controllano la trascrizione genica negli eucarioti}
I regolatori di trascrizione negli eucarioti coinvolgono molte proteine lunghe sequenze di DNA. Come nei batteri il tempo e luogo della trascrizione di un gene \`e regolato da sequenze
cis-regolatorie che sono lette dai regolatori di trascrizione che si legano a esse. Gli attivatori aiutano a legare la RNA polimerasi ad iniziare la trascrizione e i repressori bloccano
il processo. Queste interazioni sono per la maggior parte indirette con molte proteine intermediarie come gli istoni. Negli organismi multicellulari dozzine di regolatori di trascrizione
controllano un singolo gene con sequenze cis-regolatorie diffuse in decine di migliaia di paia di nucleotidi. L'inallenamento del DNA permette le proteine regolatorie di interagire tra 
di loro e con la polimerasi al promotore. Infine l'iniziazione della trascrizione deve superare il blocco imposto dai nucleosomi e strutture di livello superiore.
\subsection{Una regione di controllo di un gene eucariotico consiste di un promotore e molte sequenze cis-regolatorie}
Negli eucarioti la RNA polimerasi II trascrive tutti i geni codificatori delle proteine e molti non codificatori e richiede $5$ fattori di trascrizione generali il cui assemblaggio mette
a disposizione multipli passi per il regolamento della trascrizione. Si usa il termine regione del controllo del gene per descrivere l'intera lunghezza di DNA coinvolta nella regolazione
e iniziazione della trascrizione che include il promotore, dove i fattori generali di trascrizione e la polimerasi si assemblano e tutte le sequenze cis-regolatorie in cui i regolatori
si legano per controllare il tasso di assemblaggio al promotore. Alcune parti delle regioni regolatorie sono trascritte come lncRNA e si possono considerare come sequenze spaziatrici
che i regolatori di trascrizione non riconoscono direttamente. In contrasto con il piccolo numero di fattori generali di trascrizione ci sono migliaia di diversi regolatori di 
trascrizione che attivano o disattivano un singolo gene. Negli eucarioti gli operoni sono rari e ogni gene \`e regolato individualmente. 
\subsection{I regolatori di trascrizione eucariotici lavorano in gruppi}
I regolatori di trascrizione eucariotici si assemblano in gruppi alle loro sequenze cis-regolatorie con spesso interazioni cooperative. Oltre a questi sono presenti proteine 
coattivatrici o corepressori che si assemblano sul DNA con essi e non riconoscono specifiche sequenze di DNA e sono portati in queste posizioni dai regolatori di trascrizione. Queste
interazioni proteina-proteina sono troppo deboli per avvenire in soluzione ma possono cristallizzarsi con l'appropriata combinazione di sequenze cis-regolatrici. I coattivatori e i 
corepressori influenzano la trascrizione dopo che sono stati localizzati sul genoma dai regolatori. Un regolatore pu\`o partecipare in pi\`u di un tipo di complesso regolatorio e 
funzionano pertanto come parti di un complesso che reprime la trascrizione. Ogni gene eucariotico \`e regolato da un insieme di proteine che devono essere tutte presenti per esprimere
il gene al livello appropriato.
\subsection{Le proteine attivatrici promuovono l'assemblaggio di RNA polimerasi al punto di inizio della trascrizione}
Le sequenze cis-regolatorie in cui le proteine attivatrici si legano aumentano il tasso di iniziazione della trascrizione attraendo e posizionando RNA polimerasi II al promotore e 
rilasciarla in modo che il processo inizi. Alcune proteine attivatrici si legano direttamente a fattori di trascrizione generali velocizzando il loro assemblaggio al promotore che \`e
stato portato in prossimit\`a grazie all'inallenamento del DNA. La maggior parte degli attivatori attraggono coattivatori che poi svolgono il compito biochimico per iniziare la 
trascrizione. Uno di questi \`e il complesso di proteine mediatore composto da pi\`u di $30$ subunit\`a e serve come ponte tra gli attivatori legati al DNA, la RNA polimearsi e i 
fattori di trascrizione generali facilitando il loro assemblaggio al promotore.
\subsection{Gli attivatori di trascrizione eucariotici direazionano la modifica di strutture cromatiniche locali}
I fattori generali di trascrizione e l'RNA polimerasi sono incapaci di assemblarsi su un promotore che \`e condensato in un nucleosoma. Gli attivatori devono pertanto svolgere questo 
compito causando cambi alla struttura cromatinica rendendo il DNA pi\`u accessibile. Il modo pi\`u importante per compiere questo \`e attraverso cambi locali con modifiche covalenti 
degli istoni, rimodellamento dei nucleosomi, la loro rimozione e sostituzione. Gli attivatori usano tutti questi meccanismi e attraggono coattivatori che includono enzimi di modifica 
degli istoni, complessi di rimodellamento della cromatina dipendenti dall'ATP e accompagnatori di istoni, onguno dei quali pu\`o alterare la struttura cromatinica del promotore. 
L'alterazione della struttura cromatinica pu\`o persistere per diverso tempo: in alcuni casi le modifiche sono eliminate appena i regolatori si disassociano dal DNA, cosa importante per
geni che la cellula deve rapidamente attivare o disattivare in risposta a segnali esterni. In altri casi la struttura persiste in modo da estendere questa informazione alle generazioni
successive e modificare la memoria dei pattern di espressione genica. Un tipo speciale di modifica cromatinica accade quando l'RNA polimerasi trascrive un gene: gli istoni dopo la 
polimerasi possono essere acetilati da enzimi, rimossi dagli accompagnatori degli istoni e depositati dietro la polimerasi. Questi istoni sono poi deacilati e metilati rapidamente da
complessi trasportati dalla polimerasi lasciando nucleosomi resistenti a trascrizione. Questo processo sembra prevenisca reiniziazioni spurie dietro a una polimerasi che deve crearsi
una via attraverso la cromatina.
\subsection{Gli attivatori di trascrizione possono promuoverla rilasciando la RNA polimerasi dai promotori}
In alcuni casi l'inizazione della trascrizione richiede che un attivatore di trascrizione legato al DNA rilasci la RNA polimerasi dal promotore in modo da permettere l'inizio della
trascrizione. In altri casi la polimerasi si blocca dopo aver trascritto $50$ nucelotidi e ulteriore allungamento richiede un attivatore legato dietro a essa. Queste polimerasi bloccate
sono comuni negli esseri umani. Il rilascio della RNA polimerasi pu\`o avvenire attraverso un complesso di rimodellamento della cromatina che rimuove un blocco nucleosomico o attraverso
un attivatore che comunica con la RNA polimerasi tipicamente attraverso un coattivatore segnalandole di procedere. Infine i fattori di allungamento permettono alla polimerasi di 
trascrivere attraverso la cromatina che in alcuni casi \`e il passo chiave per il controllo. Una volta che i fattori di allungamento sono caricati permettono alla polimerasi di muoversi
attraverso i blocchi cromatinici. Avere una RNA polimerasi gi\`a posizionata ad un promotore permette di bypassare il passaggio di assemblaggio e di rispondere pi\`u velocemente a 
segnali esterni.
\subsection{Gli attivatori di trascrizione lavorano sinergeticamente}
I complessi di attivatori e coattivatori si assemblano cooperativamente sul DNA e possono promuovere diversi passi dell'iniziazione di trascrizione. L'aumento del tasso di reazione 
finale dovuto a un insieme di attivatori \`e il prodotto dei contributi dei singoli. Si nota pertanto come gli attivatori esibiscano sinergia trascrizionale. 
\subsection{Repressori di trascrizione eucariotici possono inibire la trascrizione in molti modi}
I repressori di trascrizione possono diminuire il tasso di trascrizione e disattivare geni rapidamente. Grandi regioni di genoma possono essere disattivate dalla condensazione del DNA
in forme cromatiniche specialmente resistenti, ma la maggior parte non sono organizzati per funzione e pertanto non \`e una strategia sempre disponibile. La maggior parte dei repressori
lavorano gene per gene. Non competono direttamente con la RNA polimerasi per l'accesso al DNA ma usano molti meccanismi che bloccano la trascrizione. Tipicamente agiscono portando un
corepressore al DNA. I repressori possono agire su pi\`u di un meccanismo ad un gene. La repressione \`e specialmente importante per organismi la cui crescita dipende da programma di 
sviluppo elaborati e complessi. 
\subsection{Sequenze di DNA isolanti impediscono a regolatori di influenzare geni distanti}
Per evitare che regolatori di geni diversi si influiscano tra di loro elementi del DNA compartimentalizzano il DNA in domini regolatori discreti. Sono presenti sequenze barriera che
impediscono la diffusione di eterocromatina in geni che devono essere espressi. Un isolatore impedisce a sequenze cis-regolatorie di attivare geni inappropriati. Funzionano formando 
anelli di cromatina attraverso proteine specializzate che si legano a loro. Gli anelli tengono un gene e la sua regione di controllo in prossimit\`a e aiutano a prevenire l'uscita di 
una regione di controllo in geni adiacenti. La distribuzione degli isolatori e sequenze barriera si pensa divida il genoma in domini indipendenti di regolazione genica e struttura 
cromatinica. 
\subsection{Meccanismi genetici molecolari che creano e mantengono tipi di cellula specializzati}
Le cellule di organismi multicellulari mantengono la scelta della differenziazione in un tipo di cellula specifico, ricordando i cambi nell'espressione genica coinvolti nella scelta.
Questa memoria \`e un prerequisito per la creazione di tessuti organizzati e per il mantenimento di tipi cellulari stabilmente differenziati. 
\subsection{Complessi interruttori genici che regolano lo sviluppo della Drosophila sono creati da molecole pi\`u piccole}
Considerando il gene Even-skipped (Eve) della Drosophila la cui espressione \`e fondamentale per lo sviluppo dell'embrione si nota come se il gene \`e disattivato per mutazione, molte
parti dell'embrione non si formano e questo muore. Quando Eve comincia ad essere espresso l'embrione \`e una singola cellula gigante contenente nuclei multipli in un citoplasma comune 
che contiene anche un insieme di regolatori di trascrizione distribuiti lungo la lunghezza dell'embrione fornendo informazioni posizionali che distinguono una parte dell'embrione 
dall'altra. I nuclei cominciano rapidamente ad esprimere geni diversi in quanto sono esposti a differenti regolatori. Le sequenze di DNA regolatorie che controllano il gene Eve si 
sono evolute per leggere le concentrazioni di regolatori di trascrizione ad ogni posizione e causano l'espressione del gene in sette fasce precisamente posizionate, ognuna larga tra
$5$ e $6$ nuclei. La regione regolatoria del gene Eve \`e molto lunga ed \`e formata da una serie di moduli regolatori con una semplice sequenza cis-regolatoria responsabile per la
specificazione di una particolare fascia di espressione lungo l'embrione.
\subsection{Il gene Eve della Drosophila \`e regolato da controlli combinatori}
Il modello del modulo regolatorio della seconda fascia contiene sequenze di riconoscimento per due regolatori di trascrizione (Bicoid e Hunchback) che attivano la trascrizione di Eve e 
per altri due (K\"uppel e Giant) che la reprimono. La concentrazione relativa di queste quattro proteine determina se il complesso presente alla fascia attiva la trascrizione. Tale 
elemento, come quelli di tutte le altre fasce sono autonomi. L'intera regione di controllo lega pi\`u di $20$ regolatori di trascrizione. $7$ combinazioni di regolatori ne specificano
l'espressione, mentre altre combinazioni lo mantengono silente. La regione di controllo \`e pertanto costituita da una serie di moduli pi\`u piccoli, ognuno dei quali consiste di un
unico ordinamento di corte sequenze cis-regolatorie riconosciute da regolatori di trascrizione specifici. Il gene Eve codifica un regolatore di trascrizione che controlla l'espressione
di altri geni. L'embrione viene diviso in regioni sempre pi\`u specifiche fino a che costituisce le parti del corpo della Drosophila. 
\subsection{I regolatori di trascrizione sono messi in gioco da segnali extracellulari}
Nella maggior parte degli embrioni e nelle cellule adulte i nuclei si trovano in cellule separate e si necessita un passaggio di informazioni extracellulari attraverso la membrana
plasmatica per generare segnali nel citosol che causano l'attivazione di diversi regolatori. 
\subsection{Il controllo dei geni combinatorio diversi tipi di cellule}
Ogni regolatore di trascrizione in un organismo contribuisce al controllo di molti geni. A causa di questo controllo combinatorio un regolatore di trascrizione non ha una funzione 
definita come un comandante di uno specifico insieme di geni o di un tipo cellulare, ma \`e la sua combinazione relativa con altri a stabilire quest'informazione. Il controllo genico
combinatorio causa l'addizione di un nuovo regolatore di trascrizione che dipende dalla storia precedente. Durante lo sviluppo una cellula accumula una serie di regolatori che
ne alterano l'espressione genica solo quando l'ultimo membro presente viene aggiunto. Questo meccanismo permette ad alcune cellule di convertire da un tipo all'altro. 
\subsection{Combinazioni di regolatori di trascrizione master specificano il tempo della cellula controllando l'espressione di molti geni}
I pattern di espressione del tipo cellulare sono determinati da una combinazione di regolatori di trascrizione master che in molti casi si legano direttamente a sequenze cis-regolatorie
del gene. MyiD si lega direttamente alla sequenza cis-regolatoria delle regioni di controllo sui geni specifici ai muscoli. In altri casi i regolatori master controllano l'espressione
di regolatori a valle che si legano alle regioni di controllo di altri geni specifici e controllano la loro sintesi. La specifica di un particolare tipo di cellula coinvolge cambi 
nell'espressione di migliaia di geni: quelli richiesti dal particolare tipo sono prodotti, gli altri no. 
\subsection{Cellule specializzate devono rapidamente attivare e disattivare insiemi di geni}
Le cellule specializzate devono rispondere ai cambi nell'ambiente come segnali da altre cellule che coordinano il comportamento dell'intero organismo. Molti di questi segnali inducono
cambi temporanei alla trascrizione dei geni. L'effetto di un singolo regolatore pu\`o avere effetto, completando la combinazione necessaria per attivare o reprimere il gene. Un esempio 
\`e la proteina glucocorticoide-recettrice umana. Per allungarsi nella sua sequenza cis-regolatoria deve prima formare un complesso con un ormone glucocorticoide steroide come il cortisolo
che viene rilasciato dal corpo durante periodi di fame e intensa attivit\`a fisica, stimolando le cellule del fegato ad aumentare la produzione di glucosio dagli amminoacidi. Tali 
cellule aumentano l'espressione di molti geni che codificano gli enzimi metabolici. Nonostante questi geni abbiano regioni di controllo diverse e complesse la loro espressione massimale
dipende dal legame del complesso ormone-glucocorticoide alla sua sequenza cis-regolatoria, presente nella regione di controllo di ogni gene. 
\subsection{Cellule differenziate mantengono la loro identit\`a}
Una volta che una cellula si \`e differenziata generalmente rimarr\`a differenziata e tutta la sua progenie rimarr\`a dello stesso tipo. Alcune cellule altamente specializzate non si
dividono pi\`u, ovvero si differenziano terminalmente. Molte altre cellule differenziate si dividono molte volte nella vita di un individuo generando progenie dello stesso tipo. 
Affinch\`e il tipo venga mantenuto (memoria cellulare) i pattern di espressione genica responsabili devono essere mantenuti e passati alle figlie durante le divisioni. Questo 
viene ottenuto attraverso un loop a feedback positivo, dove un regolatore di trascrizione master attiva la trascrizione per il proprio gene e a tutti gli altri specifici al sito. Ogni
volta che la cellula si divide questo viene passato alle figlie. 
\subsection{Circuiti di trascrizione permettono alla cellula di compiere operazioni logiche}
Semplici regolatori possono essere combinati per creare dispositivi di controllo che se ordinati in motivi di rete possono essere trovati continuamente in cellule di specie diverse. 
Sono comuni anelli di feedback positivi e negativi. Con pi\`u regolatori i comportamenti dei circuiti possono diventari pi\`u complessi come flip-flops, loop a feed-forward. 
\section{Meccanismi che rinforzano la memoria cellulare in piante e animali}
\subsection{Pattern di metilazione del DNA possono essere ereditati quando le cellule dei vertebrati si dividono}
Nelle cellule dei vertebrati la metilazione della citosina fornisce un meccanismo attraverso cui i pattern di espressione genica possono essere passati alla progenie della cellula. La
$5$-metil citosina ha la stessa relazione con la citosina che la timina ha con l'uracile e la modifica non ha effetto sull'accoppiamento delle basi. La metilazione del DNA avviene 
principalmente in sequenze CG accoppiate alla stessa sequenza opposta all'altro filamento. Un semplice meccanismo permette a questo pattern di essere ereditato. L'enzima metil trasferasi
di manutenzione agisce sulle sequenze gi\`a metilate e il pattern di metilazione serve come stampo per la metilazione del filamento figlio. Tali pattern sono dinamici durante lo 
sviluppo. Dopo la fertilizzazione c'\`e un'onda di metliazione dove la maggior parte dei gruppi metili sono persi dal DNA attraverso soppressione dell'attivit\`a della metil trasferasi o
grazie a enizimi di demitelazione. Successivamente nello sviluppo nuovi pattern sono stabiliti da nuove DNA metil trasferasi direzionate al DNA da proteine specifiche alla sequenza.
Una volta che i nuovi pattern sono stabiliti possono essere propagati attraverso la replicazione. La metilazione de DNA lavora in congiunzione con altri meccanismi di controllo genici
per stabilire una forma efficiente di repressione genica. In questo modo i geni eucariotici possono essere altamente repressi. I gruppi metili sulle citosine si trovano nella scanalatura
maggiore del DNA e interferiscono direttamente con le proteine di legame richieste per l'iniziazione della trascrizione e la cellula contiene un insieme di proteine che si legano a tali
citosine come enzimi di modifica degli istoni che portano a stati di cromatina repressivi dove la metilazione e la struttura cromatinica agiscono sinergisticamente.
\subsection{Isole ricche di CG sono associate con molti geni nei mammiferi}
La citosina metilata tende ad essere eliminata nei genomi dei vertebrati in quanto deamminazione accidentali di un non metilato C genera U che viene riconosciuto dalla DNA glicolasi, 
esportato e sostituito con un C. Deamminazione accidentale di una citosina metilata non pu\`o essere riparata in quanto il prodotto \`e una T. Esiste pertanto uno speciale sistema di
riparazione per rimuovere queste T mutanti che non sono precisi. Le sequenze CG rimanenti sono distribuite non uniformemente nel genoma e maggiormente nelle isole CG e tipicamente 
includono promotori dei geni. Sono maggiormente presenti nei geni che codificano proteine necessarie alla vita della cellula. Isole CG rimangono non metilate nella maggior parte delle
cellule somatiche e tale stato \`e mantenuto da proteine specifiche alla sequenza. Legandosi a queste sequenze proteggono il DNA dalla metil trasferasi e reclutano DNA demitilasi che
converte la $5$-metil C in idrossi-metil C che \`e successivamente sostituita da C attraverso riparazione del DNA o passivamente. Tali isole permettono ad alcune proteine che si legano
alle isole di CG e le proteggono dall'enzima modificatore di istoni che recluta la metilazione e rendono l'isola amichevole ai promotori. La RNA polimerasi \`e pertanto spesso
legata a promotori nelle isole di CG. La polimerasi si trova pertanto maggiormente rispetto ai nucleosomi. Altri passi sono necessari per iniziare la trascrizione e sono diretti da
regolatori che si legano alle sequenze cis-regolatorie del DNA che rilasciano la polimerasi con i fattori di allungamento.
\subsection{L'imprinting genomico si basa sulla metilazione del DNA}
Le cellule dei mammiferi sono diploidi e l'espressione di una minoranza dei geni dipende dall'origine materna o paterna: quando la copia del gene paterno \`e
attiva quello materno \`e inattiva o viceversa. Questo fenomeno \`e detto imprinting genomico. L'imprinting pu\`o smascherare mutazioni coperte dall'altra coppia funzionale. Nel primo 
embrione i geni soggetti a imprinting sono marcati da metilazione secondo il fatto che siano derivati da un cromosoma dello sperma o dell'uovo. In questo modo la metilazione del DNA 
viene usata per distinguere due geni altrimenti identici. In qualche modo sono protetti dall'onda di demitelazione le cellule somatiche si ricordano l'origine parentale di ogni gene e 
regolano la loro espressione secondo questo. Nella maggior parte dei casi vengono silenziati geni vicini, ma in alcuni casi possono essere attivati. In altri casi l'imprinting coinvolge
long noncoding RNA definiti come molecole di RNA pi\`u lunghe di $200$ nucleotidi che non codificano le proteine. Nel caso del gene Kcnq1 che codifica per un canale di calcio legato
al voltaggio necessario per la funzione del cuore l'lncRNA \`e fatto dall'allele paterno non metilato ma non \`e rilasciato dalla RNA polimerasi e rimane nel sito della sintesi e recluta
enzimi modificatori degli istoni e per la metilazione del DNA che direzionano la formazione di cromatina repressiva che silenza il gene associato sul cromosoma derivato dal padre. Il 
gene materno \`e immune a questi effetti. 
\subsection{Alterazioni globali al cromosoma nella struttura cromatinica possono essere ereditate}
Nei mammiferi alterazioni nella struttura cromatinica di un cromosoma possono modulare il livello di espressione della maggior parte dei geni su quel cromosoma. Maschi e femmine 
differiscono nei loro cromosomi sessuali: le femmine possiedono due cromosomi X mentre i maschi un X e un Y. La cellula femmina contiene pertanto il doppio delle coppie dei geni del 
cromosoma X, diverso dall'Y: il primo \`e grande e contiene pi\`u di mille geni, mentre il secondo \`e piccolo e ne contiene meno di $100$. Si \`e evoluto un meccanismo di compensazione
del dosaggio per equalizzare il dosaggio di prodotti dei geni del cromosoma X tra maschi e femmine. Il tasso corretto dei geni del cromosoma X a autosoma \`e controllato e mutazioni che
lo coinvolgono sono generalmente letali. I mammiferi ottengono questa compensazione disattivano uno dei due cromosomi X nelle cellule femminili somatiche nella disattivazione-X: due
cromosomi X possono coesistere nello stesso nucleo e differire completamente nella loro espressione. In un primo momento durante lo sviluppo dell'embrione femminile uno dei due 
cromosomi X in ogni cellula diventa altamente condensato in un tipo di eterocromatina. La scelta iniziale \`e casuale e una volta che viene disattivato rimane silente attraverso tutte
le cellule seguenti. A causa della casualit\`a ogni femmina \`e un mosaico di gruppi clonati di cellule in cui uno dei due \`e silenziato. Questi gruppi si trovano in piccoli cluster.
La disattivazione del cromosoma X inizia e si diffonde da un sito vicino al centro del cromosoma, il centro di inattivazione X (XIC) dove \`e trascritto un lncRNA (Xist) che \`e 
espresso unicamente dal cromosoma inattivo. L'RNA Xist si diffonde dal XIC all'intero cromosoma e guida il silenziamento. Lo Xist che si diffonde \`e prima passato attraverso la base 
degli anelli che creano il cromosoma. Imprinting e disattivazione del cromosoma X sono esempi di espressione genetica monoallelica. 
\subsection{Meccanismi epigenetici assicurano che pattern stabili di espressione genica possono essere trasmessi a cellule figlie}
Il modo pi\`u semplice per una cellula di ricordare la sua identit\`a \`e attraverso loop a feedback positivo in cui un regolatore di trascrizione chiave attiva la trascrizione del
proprio gene. Tali loop concatenati forniscono grande stabilit\`a bufferando il circuito contro fluttazioni nel livello di qualcuno dei regolatori di trascrizione. Siccome i regolatori
di trascrizione sono sintetizzati nel citosol e si diffondono attraverso il nucleo i loop hanno effetto su entrambe le coppie del gene nella cellula diploide, ma anche le differenze 
di livello di espressione tra lo stesso gene viene ereditata. L'abilit\`a della cellula di mantenere la memoria dei pattern di espressione genica \`e un esempio di ereditariet\`a
epigenetica: un'alterazione nel fenotipo ereditabile che non risulta da cambi nella sequenza del DNA. 
\section{Controlli post trascrizionali}
I controlli post trascrizionali operano dopo che la RNA polimerasi si \`e legata al promotore del gene e ha iniziato la sintesi sono cruciali per la sintesi di molti geni. 
\subsection{L'attenuazione della trascrizione causa la terminazione prematura di alcune molecole di RNA}
L'espressione di alcuni geni \`e inibita da terminazione prematura della trascrizione, fenomeno detto attenuazione della trascrizione. In alcuni di questi casi la catena di RNA adotta 
una struttura che causa la sua interazione con la polimerasi in modo da abortire la sua trascrizione. Quando il gene \`e richiesto proteine regolatorie si legano alla catena e rimuovono
l'attenuazione. Un esempio di questo avviene durante il ciclo dell'HIV in cui il DNA virale \`e trascritto dalla polimerasi II che di solito termina prima di averne sintetizzato l'intero
genoma. Quando le condizioni per la crescita virale sono ottime una proteina codificata dal virus Tat si lega a una struttura specifica dell'RNA che contiene una base rigonfia impedendo
la terminazione prematura. 
\subsection{Riboswitches rappresentano probabilmente antiche forme di controllo genico}
I riboswitches dimostrano come anche l'RNA pu\`o formare dispositivi di controllo: piccole sequenze di RNA che cambiano la loro conformazione quando si legano a piccole molecole come i 
metaboliti. Ogni riboswitch riconosce una piccola molecola e il cambio conformazionale viene utilizzato per regolare l'espressione dei geni. Sono spesso localizzati vicino alla 
terminazione $5'$ dell'mRNA e si piegano mentre viene sintetizzato bloccando o permettendo il progresso della RNA polimerasi. Sono comuni nei batteri, altamente specifici. Non 
necessitano proteine regolatorie. Controllano l'allungamento della trascrizione.
\subsection{RNA splicing alternativo pu\`o produrre diverse forme di proteine dallo stesso gene}
L'RNA splicing accorcia il trascritto di molti geni rimuovendo sequenze di introni dal precursore. Una cellula pu\`o fare splicing sullo stesso precursore in maniera diversa creando 
diverse catene polipeptidiche durante RNA splicing alternativo. Quando esistono diverse possibilit\`a di splicing un singolo gene pu\`o produrre dozzine di diverse proteine. In alcuni
casi accade a causa di ambiguit\`a nella sequenza dell'introne: lo spliceosoma standard \`e incapace di distinguere tra due o pi\`u coppie alternative di accoppiamenti $5'$ e $3'$ di 
siti di splicing e possono essere fatte scelte diverse. In molti casi viene regolato per passare dalla produzione di una proteina non funzionale  a una funzionale o viceversa. Oltre a
questo pu\`o produrre diverse versioni di una proteina secondo la necessit\`a della cellula. Lo splicing pu\`o essere regolato negativamente da una molecola regolatoria che impedisce 
al macchinario di accedere a un sito di splice particolare all'RNA o positivamente da una che aiuta a direzionare lo splicing. Si pu\`o pensare allo splicing di una molecola di pre-mRNA
come a un equilibrio tra siti di splicing in competizione. 
\subsection{La definizione di un gene \`e stata modificata}
Fino alla scoperta che i geni ecuariotici contengono introni e che le loro sequenze codificanti possono essere assemblate in vari modi il gene era definito operazionalmente come una
regione del genoma che segrega una singola unit\`a durante la meiosi e d\`a origine a un tratto fenotipico definibile. Successivamente si \`e passati a una definizione secondo cui
la maggior parte dei geni corrispondono a una regione del genoma che direziona la sintesi di un singolo enzima. La giorno d'oggi un gene viene definito come una sequenza di DNA che \`e
trascritta come una singola unit\`a e codifica un insieme di catene polipeptidiche strettamente imparentate (proteine isoformiche). 
\subsection{Un cambio nel sito della rottura del trascritto a RNA e dell'addizione Poli-A pu\`o cambiare la terminazione C di una proteina}
La terminazione di una molecola di mRNA non \`e formata dalla sintesi dell'RNA da parte della polimerasi, ma risulta da una reazione di rottura catalizzata da altre proteine mentre il 
trascritto si allunga. Una cellula pu\`o controllare il sito di questa rottura cambiando la terminazione C della proteina risultante. In alcuni casi si produce una versione troncata, 
in altri rottura alternativa e siti di poliadenilazione si trovano tra le sequenze di introni e viene alterato il pattern di splicing. Questo processo pu\`o produrre due proteine che
differiscono solo nella sequenza alla terminazione C. Un esempio di questo \`e il passaggio dalla forma legata a membrana a quella di antibiotico secreto dalla durante lo sviluppo dei 
linfociti B. Presto nella vita del linfocita B l'anticorpo che produce \`e ancorato alla membrana plasmatica dove \`e un recettore per antigeni la cui presenza causa una sua 
moltiplicazione e l'inizio della secrezione di anticorpi. La forma secreta \`e identica a quella legata a membrana ad eccezione della terminazione C. In questa parte si trova una
lunga stringa di amminoacidi idrofobici che attraversano il bistrato lipidico della membrana, mentre la forma secreta \`e corta e contiene amminoacidi idrofilici. Questo passaggio
avviene a causa del cambio del sito di rottura e poliadenilazzione causato da un aumento di concentrazione della proteina CstF che promuove la rottura dell'RNA: il primo sito di 
rottura e addizione poliA viene ignorato in quanto subottimo solitamente, ma quando viene attivata per produrre anticorpi viene considerato a causa di un aumento di concentrazione di 
proteina CstF.
\subsection{Modifiche dell'RNA possono cambiare il significato del messaggio RNA}
Il processo di RNA editing modifica la sequenza nucleotidica dei trascritti dopo che sono stati sintetizzati. Negli animali avvengono principalmente la deamminazione dell'adenina per
produrre inosina (A-to-I editing) e la deamminazione della citosina per produrre uracile (C-to-U editing). Essendo che queste modifiche cambiano le propriet\`a di accoppiamento delle
basi hanno profondi effetti sul significato dell'RNA. Se questo avviene in una regione codificante pu\`o cambiare la sequenza di amminoacidi o produrre una proteina troncata, nella 
sequenza intronica modificare i pattern di splicing, il trasporto dell'mRNA, l'efficienza della traduzione o l'accoppiamento di basi tra microRNA (miRNA) e i loro obiettivi. Il processo
A-to-I \`e prevalente negli umani e enzimi ADAR (adenina deaminasi agente sull'RNA) lo svolge riconoscendo una struttura di RNA a doppio filamento che \`e formata dall'accoppiamento tra 
le basi e una sequenza complementare in altro luogo, tipicamente un introne. La struttura a doppio filamento specifica se l'mRNA debba essere editato e dove debba essere fatto. 
\subsection{Il trasporto di RNA dal nucleo pu\`o essere regolato}
Nei mammiferi solo circa un ventesimo della massa totale dell'RNA lascia il nucleo in quanto i residui del processamento e RNA danneggiati sono degradati nel nucleo. L'esportazione
dal nucleo \`e ritardata fino a che il processamento \`e terminato e meccanismi che possono annullare questo passaggio possono essere usati per regolare l'espressione genica. Questa
strategia forma le basi del trasporto nucleare controllato dell'mRNA.
\subsection{Alcuni mRNA sono localizzati in regioni specifiche del citosol}
Una volta che una molecola di mRNA esce nel citosol incontra un ribosoma che la traduce in una catena polipeptidica. Una volta che passa il test di correttezza viene rapidamente 
tradotta. Se la proteina \`e destinata alla secrezione o espressa nella superficie cellulare una sequenza di segnali alla terminazione N la direziona al reticolo endoplasmatico ER in 
cui componenti dell'apparato di ordinamento delle proteine riconoscono la sequenza di segnale appena emerge dal ribosoma e direziona l'intero complesso all'ER dove viene sintetizzata
la catena rimanente. In altri casi ribosomi liberi nel citosol sintetizzano la proteina e segnali possono poi direzionarla in altri siti della cellula. Molti mRNA sono direzionati a
specifiche locazioni intracellulari prima che inizi la traduzione permettendo alla cellula di posizionarli vicino al sito dove la proteina \`e necessitata. Altri vantaggi includono
stabilimenti di asimmetrie nel citosol e mRNA localizzati con controlli di traduzinoe permettono alla cellula di regolare l'espressione genica indipendentemente in regioni diverse.
Altri meccanismi di localizzazione dell'mRNA richiedono specifici segnali sull'mRNA, concentrati nella regione non tradotta $3'$ o UTR, che si estende dal codone di stop. La 
localizzazione \`e tipicamente accoppiata con controlli trascrizionali per garantire che l'mRNA rimane quiescente fino a che \`e stato mosso nel luogo corretto. Le molecole di mRNA 
portano numerose marcature quando escono dal nucleo che segnalano il completamento del processamento e l'UTR $3'$ pu\`o essere pensato come un codice postale che direziona l'mRNA in
diversi luoghi. 
\subsection{Le regioni non tradotte $\mathbf{5'}$ e $\mathbf{3'}$ controllano la traduzione dei propri mRNA}
Una volta che l'mRNA viene sintetizzato si controlla il livello di prodotto proteico controllando l'iniziazione della traduzione. Nei mRNA batterici una lunghezza di nucleotidi 
conservata (sequenza Shine-Dalgarno) si trova pochi nucleotide a monte del codone AUG iniziante. I controlli traduzionali sono svolti da proteine o moleocle di RNA e coinvolgono
esposizione o blocco della sequenza Shine-Dalgarno. Gli mRNA eucariotici non contengono tale sequenza ma la selazione di un codone AUG dipende dalla sua prossomit\`a al cappuccio alla
terminazione $5'$, il sito in cui la piccola subunit\`a ribosomica si lega all'mRANe inizia lo scano. I repressori si possono legare a tale reminazione e inibire l'iniziazione della
traduzione. Altri repressori riconoscono sequenze nucelotidiche nella $3'$ UTR e interferiscono con la comunicaizone tra il cappuccio $5'$ e la coda poli-A $3'$ o da miRNA che si legano
all'mRNA e ridocono l'output di proteine.
\subsection{La fosforilazione di un fattore di iniziazione regola globalmente la sintesi proteica}
Le cellule eucariotiche diminuiscono il tasso di sintesi delle proteine in risposta a molte situazioni come deprivazione di fattori di crescita o nutrienti, infezioni da virus e aumenti
di temperatura. La maggior parte di questa diminuizione \`e causata dalla fosforilazione del fattore di iniziazione di trafuzione eIF2 da una proteina chinasi. eIF2 normalmente forma
un complesso con CTP e media il legame dell'inizatore metionile tRNA alla piccola subunit\`a ribosomica che prima si lega alla terminazione $5'$ dell'mRNA e inizia la scansione. 
Quando viene riconosciuto un codone AUG eIF2 idrolizza GTP in GDP causando un cambio conformazionale nella proteina e rilasciandola dalla piccola subunit\`a ribosomica. La grande allora
si unisce formando il ribosoma completo che inizia la sintesi delle proteine. Un fattore di scambio guanina nucleotire eIF2B \`e necessario per rilasciare il GDP in modo che una nuova
GTP si possa legare e eIF2 possa essere riusata. Il riutilizzo \`e inibito quando \`e fosforilata in quanto si lega a eIF2B strettamente disattivando quest'ultima.
\subsection{Iniziazione al codone AUG a monte dell'inizio di traduzione pu\`o regolare l'iniziazione della traduzione eucariotica}
I nucleotidi che circondano il primo codone AUG a valle della terminazione $5'$ influenzano l'efficienza dell'iniziazione della traduzione. Se il sito di riconoscimento \`e poco 
efficiente la piccola subunit\`a potrebbe ignorarlo e passare al secondo o terzo in una strategia per produrre due o pi\`u proteine strettamente imparentate. Questo processo viene 
utilizzato per la produzione della stessa proteina senza una sequenza di segnale che le permette di essere potrata in due posizioni diverse. Le cellule possono regolare le concentrazioni
relative. Un altri tipo di controllo usa open reading frames, corte lunghezze di DNA che cominciano con un codone di inizio ATG e finiscono con un codone di stop che si trovano tra
la terminazione $5'$ dell'mRNA e l'inizio del gene. Spesso la sequenza di amminaocidi codificata non \`e importante e gli uORF servono solo per la loro funzione regolatoria. Un uORF 
diminuisce la traduzione del gene intrappolando il complesso della scansione di iniziazione e causando la sua traduzione e la disassociazione del ribosoma prima che raggiunga la 
proteina.
\subsection{Siti di entrata interni al ribosoma forniscono opportunit\`a per il controllo traduzionale}
Una sequenza di RNA detta internal ribosome entry site (IRES) pu\`o essere utilizzata per iniziare la traduzione a posizioni distanti dalla terminazione $5'$ dell'mRNA. In alcuni casi
due sequenze codificanti sono portate in tandem sullo stesso mRNA: la traduzione della prima accade dal meccanismo di scansione solito e la traduzione della seconda attraverso un IRES, 
tipicamente centinaia di nucleotidi di lunghezza e sono piegati in strutture specifiche che legano molte delle stesse proteine utilizzate per iniziare una traduzione normale dipendente
dal cappuccio. IRES diversi richiedono diversi sottoinsiemi di fattori di iniziazione e possono saltare il riconoscimento del cappuccio da parte di eIF4E. Vengono utilizzati da 
alcuni virus in una strategia che blocca la sintesi del gene normale: durante un'infezione producono una proteasi che rompe il fattore di traduzione eIF4G che diventa inabile di 
legarsi a eiF4E il complesso di legame al cappuccio. Questo disattiva la maggior parte delle traduzioni e le sposta verso i macchinari delle sequenze IRES presenti sul mRNA virale. 
\subsection{Cambi nella stabilit\`a dell'mRNA possono regolare l'espressione genica}
Gli mRNA nelle cellule eucariotiche sono pi\`u stabili di quelli batterici. Esistono due meccanismi generali per distruggere ogni mRNA creato dalla cellula. Entrambi iniziano con un
accorciamento graduale della coda poli-A da un esonucleasi, processo che inizia appena l'mRNA raggiunge il citosol che agisce come timer per il tempo di vita dell'mRNA. Una volta che 
la coda \`e ridotta ad una lunghezza critica i due cammini divergono. In uno il cappuccio $5'$ \`e rimosso e l'mRNA esposto \`e degradato da tale terminazione. Nell'altro l'mRNA continua
ad essere degradato dalla terminazione $3'$. Tutti gli mRNA sono soggetti ad entrambi i tipi di decadimento e la sequenza determina quanto velocemente questi passi avvengono. Sono 
specialmente importanti le sequenze UTR $3'$ in quanto hanno siti di legame per proteine che aumentano o diminuiscono il tasso del decadimento. Accorciamento poli-A e deincappucciamento
competono con il macchinario che traduce l'mRNA. Alcuni mRNA possono essere degradati da meccanismi specializzati che non necessitano dell'accorciamento poli-A. In questi casi 
rotture da nucleasi specifiche rompono l'mRNA internamente, deincappucciando una fine e rimuovendo la coda poli-A dall'altra. Tali mRNA hanno sequenze specifiche, tipicamente nei $3'$
UTR che servono come sequenze di riconoscimento per le endonucleasi. Questa strategia rende semplice regolare la stabilit\`a degli mRNA bloccando o esponendo il sito per l'endonucleasi
in risposta a segnali extranucleari. 
\subsection{La regolazione della stabilit\`a dell'mRNA coinvolge corpi P e granuli di stress}
Grandi aggregati di proteine e acidi nucleici che lavorano insieme sono tenuti in prossimit\`a da connessioni a bassa affinit\`a formando organelli. Gli eventi discussi in questa sezione
avvengono in aggregati detti corpi di processamento o corpi P presenti nel citosol. Alcuni mRNA rimangono intatti e ritornati nell'insieme di mRNA che possono essere tradotti. Per
essere recuperati gli mRNA si muovono dai corpi P a un granulo di stress che contiene un fattore di iniziazione della traduzione, proteine leganti al poli-A e la piccola subunit\`a
ribosomica.
\section{Regolazione dell'espressione genica da RNA non codificanti}
\subsection{I trascritti di piccoli RNA non codificanti regolano molti geni animali e di piante attraverso interferenza RNA}
Un gruppo di RNA corti svolgono interferenza RNA o RNAi dove RNA a singolo filamento guidano RNA che riorganizzano e legano selettivamente altri RNA nella cellula. Quando un obiettivo 
\`e un mRNA maturo la sua traduzione pu\`o essere inibita o la sua distruzione catalizzata. Se la molecola di RNA sta venendo trascritta si possono legare ad essa e guidare la formazione
di certi tipi di strutture cromatiniche repressive sullo stampo di DNA. Tre classi lavorano in questa maniera: microRNA (miRNA), small interfering RNA (siRNA) e piwi-interacting RNA
(piRNA) che riconoscono l'obiettivo attraverso l'accoppiamento di basi e generalmente causano una riduzione nell'espressione genica.
\subsection{Il miRNA regola la traduzione e stabilit\`a dell'mRNA}
I microRNA regolano almeno un terzo dei geni codificatori di proteine. Una volta creati accoppiano le basi con mRNA specifici e modificano finemente la loro stabilit\`a e traduzione.
I precursori di miRNA sono sintetizzati dalla RNA polimerasi II e incappucciati e poliadenilati. Subiscono un processo al termine del quale il miRNA \`e assemblato con un insieme di
proteine per formare un complesso di silenziamento indotto da RNA o RISC che una volta formato cerca il proprio mRNA obiettivo cercando per la sequenza nucleotidica complemetare,
ricerca facilitata dalla proteina Argonauta che mantiene la regione $5'$ del miRNA in modo che sia posizionata ottimamente per l'accoppiamento di basi. Questo accoppiamento di basi 
avviene principalmente nella regione $3'$ dell'mRNA. Una volta che l'mRNA viene legato da un miRNA se l'accoppiamento di basi \`e estensivo l'mRNA \`e rotto (sliced) da una proteina
Argonauta, rimuovendo  la coda poli-A ed esponendolo all'esonucleasi. Successive rotture dell'mRNA causano il rilascio del RISC e dell'miRNA che pu\`o cercare altri mRNA. Se 
l'accoppiamento \`e meno estensivo la traduzione dell'mRNA viene repressa e l'mRNA viene trasportato nei corpi P dove avviene l'accorciamento della coda poli-A, deincappucciamento e 
degradazione. Un singolo miRNA pu\`o regolare un insieme di diversi mRNA se portano una sequenza corta comune negli UTR e la regolazione pu\`o essere combinatorica: pi\`u miRNA possono
legarsi allo stesso mRNA riducendone la traduzione e occupano meno spazio genomico rispetto alle proteine.
\subsection{L'interferenza a RNA \`e anche usata come un meccanismo di difesa per la cellula}
Le proteine che partecipano nei meccanismi regolatori del miRNA si occupano anche di orchestrare la degradazione di molecole di RNA straniere, specialmente quelle nella forma a doppio
filamento. Molti elementi trasponibili e virus producono quei filamenti almeno transientemente e l'interferenza a RNA aiuta a tenerli sotto controllo. La presenza di RNA a doppio 
filamento causa RNAi attraendo il complesso di proteine Dicer, la stessa nucleasi che processa miRNA e rompe l'RNA in piccoli frammenti  detti small interfering RNA (siRNA), legati da
Argonauti e altre componenti del RISC. Le siRNA che non vengono scartate guidano il RISC verso molecole di RNA complementari prodotte dal virus o dagli elementi trasponibili. Ogni volta
che RISC rompe una nuova molecola viene rilasciato. RNA dipendente RNA polimearsi usa siRNA come primer per produrre coppie addizionali di RNA a doppio filamento che sono rotti in 
siRNA, amplificazione che garantisce che una volta iniziata l'interferenza a RNA pu\`o continuare anche dopo che tutti gli RNA a doppio filamento iniziatori sono stati degradati. In
alcuni organismi questa attivit\`a pu\`o essere diffusa dal trasferimento di frammenti di RNA da cellula a cellula. 
\subsection{L'interferenza a RNA pu\`o direzionare la formazione di eterocromatina}
Il cammino di interferenza a siRNA pu\`o in alcuni casi selettivamente impedire la sintesi di RNA obiettivi. Affinch\`e questo accada i corti siRNA prodotti dal Dicer sono assemblati 
con un gruppo di proteine per formare i complessi RITS (RNA induced transcriptional silencing). Utilizzando siRNA a singolo filamento come guida il complesso si lega a RNA trascritti
complementare mentre emergono da una RNA polimerasi II e attraggono proteine che modificano covalentemente istoni vicini e direzionano la formazione di eterocromatina prevenendo 
ulteriori iniziazioni di trascrizione. In alcuni casi una RNA dipendente RNA polimerasi e un enzima Dicer sono reclutati dal complesso RITS in modo da generare siRNA addizionali in sito.
Questo feedback loop positivo assicura una repressione continua del gene obiettivo. Questo tipo di formazione cromatinica \`e un importante meccanismo di difesa in quanto limita la
diffusione di elementi trasponibili mantenendo la loro sequenza in forma silente. Per esempio mantengono l'eterocromatina formata intorno ai centromeri. 
\subsection{I piRNA proteggono la linea germinale da elementi trasponibili}
Un terzo sistema di interferenza a RNA si basa sui piRNA (piwi-interacting RNA, dove Piwi \`e una classe di proteine imparentata con l'Argonauta). Sono creati specificatamente nella
linea germinale dove bloccano il movimento di elementi trasponibili. I geni che codificano essi consistono di sequenze di frammenti di elementi trasponibili. Questi cluster sono 
trascritti e rotti in piRNA corti e a filamento singolo. Sono pi\`u lunghi di miRNA e siRNA e non coinvolgono una proteina Argonauta ma una Piwi. Una volta formati ricercano RNA obiettivo
attraverso l'accoppiamento di basi e silenziano trascrizionalmente geni di trasposoni e distruggono ogni RNA da loro prodotto. 
\subsection{I batteri usano piccoli RNA non codificanti per proteggersi dai virus}
I virus dei batteri hanno generalmente genomi a DNA. Molte specie di batteri usano un repertorio di piccole molecole di RNA non codificante per cercare e distruggere il DNA dei virus
invadenti. Il sistema detto CRISPR simile a quelle di miRNA e siRNA ma quando i batteri e archea sono infettati da un virus hanno un meccanismo che causa piccoli frammenti del
DNA virale di integrarsi nel genoma batterico. Serve come vaccinazione: diventano stampi per produrre piccoli RNA non codificanti detti crRNA che distruggono il virus se reinfetter\`a
i discendenti della cellula originale. \`E simile all'immunit\`a adattiva nei mammiferi. Un altra caratteristca \`e che i crRNA si associano con proteine speciali che permettono loro di
cercare e distruggere DNA a doppio filamento. Il processo CRISPR avviene in tre fasi. Nella prima sequenze di DNA virali sono integrate in speciali regioni del genoma come CRISPR
(clustered regularly interspersed short palindromic repeat) loci, un CRISPR locus consiste di centinaia di ripetizioni di una sequenza ospite interspersa con una collezione di sequenze
che \`e stata derivata da esposizioni a virus precedenti. La nuova sequenza virale \`e sempre integrata nella terminazione $5'$ del locus che trasporta un record temporale di 
infezioni precedenti. Nel secondo passo il locus CRISPR \`e trascritto per produrre un lungo RNA che \`e processato in corti crRNA. Nel terzo passo crRNA viene unito a proteina Cas che
cercano sequenze complementari di DNA virali e guidano la loro distruzione da parte della nucleasi. Le proteine Cas sono analoghe agli Argonauti e ai Piwi: mantengono corti RNA in 
una configurazione estesa ottimizzata per cercare e formare accoppiamenti con il DNA. 
\subsection{Lunghi RNA non codificanti hanno funzioni diverse nella cellula}
Molte delle funzioni non conosciute dei long noncoding RNA (lncRNA). La maggior parte di questi sono trascritti da RNA polimerasi II e hanno cappucci $5'$ e code poli-A. A causa del
fatto che la maggior parte di questi sono pensati il risultato dal rumore di fondo di trascrizione e processamento dell'RNA e non forniscono nessun cambiamento al fitness e sono pertanto
tollerati dall'organismo come prodotti secondari dei pattern dell'espressione genica. Per queste ragioni \`e difficile stimare quali lncRNA hanno funzione nella cellula e discriminarli. 
Esempi di lncRNA sono gli RNA nelle telomerasi, Xist RNA e quelli coinvolti nell'imprinting. Altri sono stati implicati come controllatori dell'attivit\`a enzimatica delle proteine, 
inattivando regolatori di trascrizione, avendo effetto sui pattern di splicing e bloccando traduzione di certi mRNA. Tutti gli lncRNA possono funzionare come molecole di RNA impalcatura
che legano insieme gruppi di proteine per coordinare la loro azione. Un'altra \`e l'abilit\`a di servire da sequenze guida, legandosi a molecole obiettivo di RNA o DNA con specificit\`a.
In alcuni casi lavora senza proteine e si trovano nella direzione inversa in geni codificanti di proteine. Questi RNA antigene possono accoppiarsi con l'mRNA e bloccare la sua traduzione
in proteina. Altri di questi alterano il pattern di slicing. Alcuni lncRNA possono agire soltanto i cis. 
