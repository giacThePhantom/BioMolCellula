\chapter{Organelli}

\section{Panoramica}
Si intende per organello un qualsiasi compartimento della cellula eucariote circondato da membrana.
Il sistema di membrane si trova alla base del loro sviluppo evolutivo.

\section{Trasporto di proteine}
I diversi organelli rendono necessari diversi sistemi per permettere il trasporto delle proteine prodotte per la maggior parte nel citoplasma negli organelli in cui dovranno svolgere la loro funzione.

	\subsection{Trafficking nucleo-citoplasma}
	
		\subsubsection{Membrane del nucleo}
		
			\paragraph{Involucro nucleare}
			Il nucleo \`e circondato da un involucro nucleare formato da due membrane.
			La membrana esterna \`e in continuit\`a con il reticolo endoplasmatico pur avendo una composizione diversa.

			\paragraph{Lamina nucleare}
			La lamina nucleare \`e un reticolo fibroso con funzione strutturale.
			Mette in contatto componenti che si associano al DNA e proteine associate alla membrana interna.

			\paragraph{Funzione}
			La membrana isola e protegge il materiale genetico, ma \`e sede di numerosi trasporti.

				\subparagraph{Pori nucleari}
				I pori nucleari \emph{NPC} sono formati da $30$ nucleoporine.
				Il numero varia in base al tipo cellulare.
				Permettono un trasporto molto veloce.
				La parte interna presenta una struttura a cono, mentre l'esterna ha una struttura filiforme.
				Possiede proteine che la ancorano alla membrana, proteine strutturali e che delimitano il canale.
				La parte interna funge da filtro con una struttura ben definita.
		
		\subsubsection{Trasporto}
		Affinch\`e avvenga il trasporto una proteina deve possedere la sequenza segnale.
		La proteina interagisce con la parte fibrillare, viene inserita nel centro del poro dove si completa il trasporto.

			\paragraph{Importine}
			Le importine mediano l'entrata nel nucleo grazie al \emph{NLS}.

			\paragraph{Proteina cargo}
			Si intende per proteina cargo una proteina adattatrice che espone \emph{NLS} quando cambia conformazione e viene riconosciuta dall'importina.

			\paragraph{Importo della proteina}
			Per importare una proteina con \emph{NLS} forma un complesso con l'importina.
			Questo interagisce con la parte fibrillare del poro nucleare.
			L'importina interagisce con la matrice creando uno spazio di passaggio spezzando legami: l'importina is lega a \emph{Ran-GTP} cambiando conformazione.
			Il complesso entra nel nucleo.
			Avviene un idrolisi in \emph{Ran-GDP} che causa il rilascio dell'importina che viene esportata dal nucleo.

			\paragraph{Esporto}
			Per esportare una porteina dal nucleo l'esportina lega \emph{Ran-GTP} e il cargo causando l'uscita del complesso.
			L'idrolisi in \emph{Ran-GDP} causa un cambio conformazionale dell'esportina rilasciando la proteina nel cargo.

			\paragraph{Gradiente di \emph{Ran}}
			Si nota un gradiente di \emph{Ran-GTP}: questa \`e determinata da \emph{Ran-GAP} che interagiscono con la parte fibrillare delle nucleoporine.
			Questo garantisce una direzionalit\`a del trasporto in concomitanza con \emph{Ran-GEF}.

			\paragraph{Mascheramento di \emph{NLS}}
			Una proteina con \emph{NLS} che non localizza nel nucleo possiede modifiche alla sequenza segnale come fosforilazioni.
	
				\subparagraph{Sintesi del coleserolo}
				La molecola \emph{NF-AT} fosforilata \`e disattiva e maschera il \emph{NLS}.
				Un aumento di calcio attiva la calcineurina, una fosfatasi che rende di nuovo accessibile il \emph{NLS}.
				Viene portata nel nucleo una \emph{SREBP}, una proteina coinvolta nella sintesi del colesterolo come fattore trascrizionale.
				Il colesterolo quando presente interagisc con \emph{SCAP}, una proteina legata a \emph{SREBP}.
				Il complesso in assenza di colesterolo si localizza al Golgi.
				Qui proteasi tagliano \emph{SREB} liberando un frammento che va ad attivare la trascrizione di geni coinvolti nella sintesi del colesterolo.



	\subsection{Trasporto in plastidi e mitocondri}
	Affinch\`e le proteine possano essere riconosciute dai recettori ed importate nei mitocondri devono essere completamente tradotti.

		\subsubsection{Peptide segnale}
		Il peptide segnale \`e formato da $18$ amminocidi polari e non polari che formano una struttura ad $\alpha$-elica asimmetrica.

		\subsubsection{Traslocazione nel mitocondrio}
		La traslocazione di una proteina nel mitocondrio richiede due complessi: il complesso \emph{TOM} che media il trasporto dal citoplasma nello spazio tra le membrane e il complesso \emph{TIM} che completa il trasporto nella membrana interna.
		La proteina possiede il peptide segnale e viene riconosciuta dalla porzione recettoriale del complesso \emph{TOM} e viene forzata ad entrare.
		Qui incontra il complesso \emph{TIM} che la porta attraverso la membrana interna assieme alla chaperonina \emph{HSP70} (impedisce la formazione di strutture secondarie).
		In questo secondo passaggio avviene il taglio del peptide segnale.
		I due copessi possono lavorare in modo indipendente.

			\paragraph{Energia necessaria}
			L'energia necessaria alla traslocazione viene fornita da:
			\begin{multicols}{2}
				\begin{itemize}
					\item Idrolisi di \emph{ATP} da parte di \emph{TOM}.
					\item Gradiente elettrochimico su \emph{TIM}.
				\end{itemize}
			\end{multicols}

		\subsubsection{Inserimento di una proteina nella matrice tra le membrane}
		Una proteina che deve rimanere nello spazio tra le due membrane possiede il peptide segnale e una sequenza di stop del trasferimento che blocca il passaggio della proteina verso il citoplasma.

		\subsubsection{Complesso \emph{Oxa}}
		Il complesso \emph{Oxa} media l'inserzione di proteine nella membrana interna.
		La proteina attraversa \emph{TIM} dove le viene rimosso il peptide segnale, ma ne contiene un secondo che la indirizza verso il complesso \emph{Oxa}.
		Questo permette l'ancoraggio della proteina all'interno della membrana interna del mitocondrio.

\section{Perossisomi}

	\subsection{Panoramica}
	I perossisomi sono vestigia di antichi organelli che svolgevano funzioni nel metabolismo dell'ossigeno.

		\subsubsection{Struttura}
		I perossisomi sono delimitati da membrana, presentano una forma tondeggiante e inclusioni paracristalline.
		Contengono diversi enzimi.

		\subsubsection{Funzione}
		Gli enzimi nei perossisomi permettono:
		\begin{multicols}{2}
			\begin{itemize}
				\item Rimozione di \emph{H} da substrati organici specifici con liberazione di acqua ossigenata trasformata in acqua dalla catalasi.
				\item Demolizione degli acidi grassi attraverso $\beta$-ossidazione in acetil \emph{CoA}.
				\item Catalizzano le prime reazioni nella formazione dei plasmalogeni, che formano una guaina che riveste gli assoni.
				\item Sono presenti nelle cellule vegetali e coinvolti nel ciclo del glicosilato per la conversione degli acidi grassi in zuccheri.
			\end{itemize}
		\end{multicols}

	\subsection{Trasporto}
	I perossisomi non presentano DNA e pertanto tutte le proteine necessarie al loro comportamento devono essere importate.
	Il meccanismo di trasporto \`e simile a quello dei mitocondri.

		\subsubsection{Sequenza target}
		La sequenza target si dice \emph{PLS} ed \`e costituita da \emph{Ser-Lys-Leu} N o C terminale.

		\subsubsection{Riconoscimento della sequenza target}
		La sequenza target viene riconosciuta dalle perossine o \emph{Pex}.
		Queste partecipano al processo di importazione.
		\begin{multicols}{2}
			\begin{itemize}
				\item \emph{Pex5}: riconosce una \emph{PLS} C terminale insieme a \emph{Pex1} e la \emph{ATPasi} \emph{Pex6}.
				\item \emph{Pex7}: riconosce una \emph{PLS} N terminale.
			\end{itemize}
		\end{multicols}

	\subsection{Genesi dei perossisomi}
	I perossisomi si formano da gemmazioni del reticolo endoplasmatico.
	Questo forma una struttura vescicolare con \emph{Pex3} come proteina transmembrana e \emph{Pex19} ad essa associato.
	Questi vengono detti protoperossisomi e vengono arricchiti da componenti che causano un loro accrescimento.
	Possono inoltre andare incontro a fissione.

	\subsection{Sindrome di Zellweger}
	La sindrome di Zellweger presenta anomalie celebrali, del fegato e reni.
	Queste sono causate da mutazioni a carico di \emph{Pex5}, non si trova un importo di enzimi con funzioni ossidative.

\section{Reticolo endoplasmatico}
Il reticolo endoplasmatico \`e un insieme di membrane in continuit\`a fisica ma non di composizione con la membrana nucleare.

	\subsection{Funzioni}
	Il reticolo endoplasmatico ha diverse funzioni:
	\begin{multicols}{2}
		\begin{itemize}
			\item Traduzione e modifica post-traduzionale di proteine.
			\item Biosintesi e traslocazione, in particolare nell'apparato del Golgi.
			\item Trasduzione del segnale: \`e il principale magazzino del calcio intracellulare.
				Ha pertanto un ruolo chiave nella contrazione muscolare.
		\end{itemize}
	\end{multicols}

	\subsection{Tipologie}
	Il reticolo endoplasmatico pu\`o essere diviso in tre regioni principali:
	\begin{multicols}{2}
		\begin{itemize}
			\item \emph{RE rugoso}: si presenta rugoso a causa della presenza dei ribosomi.
			\item \emph{RE intermedio} o di transiizone: consente il passaggio delle proteine al Golgi.
			\item \emph{RE liscio}: sistema di membrane per l'accumulo dello ione calcio e la sintesi dei lipidi.
				Questi vengono poi trasportati nei distretti della cellula attraverso vescicole.
		\end{itemize}
	\end{multicols}
	Questa organizzazione non \`e statica ma va incontro ad un notevole riarrangiamento grazie al citoscheletro e proteine specifiche.

	\subsection{Traduzione e traslocazione delle proteine}
	Le proteine che devono essere traslocate nel reticolo endoplasmatico presentano una sequenza segnale rimosso poi da una peptidasi quando traslocata all'interno.
	Gli mRNA per queste proteine possono essere tradotti da ribosomi legati alla membrana del reticolo endoplasmatico per proteine che rimangono all'interno di vescicole, proteine di membrana o che avanno in organelli.
	Un altro modo di traduzione \`e da ribosomi liberi nel citoplasma per proteine che vanno nel nucleo, nei mitocondri, cloroplasti o perossisomi.
	Gli enzimi deputati alla modifica elle proteine si trovano all'interno del reticolo endoplasmatico.

		\subsubsection{Traslocazione co-traduzionale}
		La traslocazione co-traduzionale consiste nel passaggio della proteina nel reticolo endoplasmatico mentre viene tradotta.
		I ribosomi sono legati alla membrana del reticolo endoplasmatico e alla fine la proteina si trova associata ad esso.
		La traslocazione co-traduzionale avviene grazie alla presenza di una sequenza segnale lungo il peptide nascente.
		Il ribosoma e la catena peptidica vengono portati sul reticolo endoplasmatico.
		Qui la traduzione continua quando i ribosomi aderiscono al reticolo endoplasmatico.

			\paragraph{Sequenza segnale}
			La sequenza segnale localizza nella regione ammino-terminale della catena polipeptidica.
			\`E costituita da una sequenza di circa $20$ amminoacidi idrofobici preceduti da un'arginina.
			
			\paragraph{Riconoscimento della sequenza segnale}
			Il peptide segnale viene riconosciuto da un complesso \emph{SRP} o signal recognition particle che dirige le proteine con una specifica sequenza segnale ad un recettore sul reticolo endoplasmatico.
			Blocca pertanto la traduzione e porta il ribosoma sul reticolo endoplasmatico.

				\subparagraph{Composizione \emph{SRP}}
				\emph{SRP} \`e composto da $6$ proteine e $1$ RNA.
				Una parte nteragisce con proteine ed RNA ribosomiali bloccando la traduzione.
				Un'altra interagisce con la sequenza segnale.
				Una terza interagisce con il recettore sul reticolo endoplasmatico.
				Si trova un hinge che permette un cambio conformazionale.

			\paragraph{Processo di traslocazione}
			\begin{multicols}{2}
				\begin{enumerate}
					\item Il mRNA viene tradotto dai ribosomi presenti nel citoplasma.
					\item Viene riconosciuto il peptide segnale da \emph{SRP}.
					\item Il ribosoma si ferma e il complesso viene reclutato sulla membrana del reticolo endoplasmatico.
					\item Il complesso \emph{ribosoma-RNA-SRP} viene portato sulla membrana del reticolo endoplasmatico dove si lega a un recettore specifico.
					\item Il legame con il recettore permette l'associazione del ribosoma trasducente al reticolo endoplasmatico rugoso.
					\item Si distacca \emph{SRB} dal peptide segnale.
				\end{enumerate}
			\end{multicols}

			\paragraph{Ripresa della traduzione}
			Il distacco di \emph{SRP} da ribosomi e sequenza segnale viene mediato dall'idrolisi del \emph{GTP}.
			La proteina passa attraverso un canale trans-membrana traslocatore ed entra nel lume del reticolo endoplasmatico.

				\subparagraph{Traslocone}
				Il traslocone \`e il canale che mette in comunicazione il lume del reticolo endoplasmatico con la membrana permettendo di continuare la traduzione.
				\`e composto da $3$ proteine trans-membrana \emph{Sec61} $\alpha$, $\beta$ e $\gamma$.
				I domini che formano il poro formano $\alpha$-eliche, come quelli che formano il tappo e lo chiudono.
				Si trova un hinge che permette l'apertura e la chiusura del poro e un plug che permette l'ingresso della proteina in fase di sintesi.

			\paragraph{Ingresso della proteina}
			La proteina che entra nel reticolo endoplasmatico viene legata dalla chaperonina \emph{Bip} per facilitarne il passaggio.
			Una peptidasi rimuove il peptide segnale.

			\paragraph{Proteine con domini transmembrana}
			Le proteine con domini transmembrana contengono una sequenza di stop.
			Questa quando entra nel traslocone determina la sua apertura e il rilascio della proteina.
			Questa pertanto rimane nella membrana con la parte $N$ termminale rivolta nel lume.
			La sequenza di terminazione interna pu\`o inoltre invertire l'orientamento.
			
				\subparagraph{Multipli domini transmembrana}
				Le proteine con multipli domini transmembrana possiedono una sequenza di trasferimento interna seguita da una sequenza di stop.
				Queste si alternano discriminando i domini transmembrana da quelli citoplasmatici.

			\paragraph{\emph{Sec61}}
			\emph{Sec61} \`e un esempio di traslocone e pu\`o essere utilizzato come marcatore del reticolo endoplasmatico.
			Associato ad esso si trova una proteasi con il compito di rimuovere il peptide segnale.
			La peptidasi rimuovendo il peptide segnale permette alla catena polipeptidica di essere rilasciata nel lume del reticolo endoplasmatico.

			\paragraph{Trasferimento di proteine inter-membrana}

				\subparagraph{Processo}
				\begin{multicols}{2}
					\begin{enumerate}
						\item Grazie alla sequenza segnale il ribosoma viene portato sul traslocone.
						\item Inizia la sintesi della proteina.
						\item Avviene un taglio ad opera della peptidasi.
						\item La traduzione continua e la proteina entra nel reticolo endoplasmatico.
						\item Viene tradotta la regione idrofobica di $\alpha$-elica come blocco di trasferimento.
						\item La sequenza introdotta blocca il trasferimento causando l'apertura del traslocone e l'interazione della proteina neosintetizzata con la membrana.
						\item La proteina rimane in membrana mentre il ribosoma traduce la parte rimanente fino al completamento della traduzione.
					\end{enumerate}
				\end{multicols}

				\subparagraph{Domini C terminali all'interno del lume del reticolo endoplasmatico}
				Affinch\`e la porzione $C$-terminale si trovi all'interno del lume la sintesi proteica avviene nel citoplasma fino alla produzione di una sequenza segnale.
				Questa, presente in una regione interna del peptide causa un suo reclutamento al traslocone, dove viene portata all'interno da quella posizione.
				In questo caso non avviene il taglio peptidico ad opera della peptidasi.
				Un dominio trans-membrana ancora la proteina sulla membrana.

				\subparagraph{Multipli domini trans-membrana}
				In caso di multipli domini trans-membrana si trovano multiple sequenze segnale e di stop.
				La traslocazione viene interrotta alla prima sequenza di stop, ma una successiva traduzione di una sequenza segnale causa una ripresa del processo dal punto interno.

			\paragraph{Chaperonine}
			Mentre viene tradotta la proteina si associa a chaperonine, proteine che impediscono il ripiegamento della proteina nelle strutture secondarie e terziarie.

			\paragraph{Ponti disolfurp}
			Nel lume del reticolo endoplasmatico avvengono reazioni di formazione dei ponti disolfuro che non avvengono nell'ambiente citoplasamtico a causa della sua natura riducente.
			Questo infatti li mantiene nella forma \emph{SH-HS}.
			Nell'ambiente ossidante del reticolo endoplasmatico i gruppi diventano \emph{S-S} formando legami covalenti.
			La formazione dei ponti disolfuro determina l'assunzione di strutture secondarie e terziarie alle proteine.

		\subsubsection{Traslocazione post-traduzionale}
		Nella traslocazione post-traduzionale la proteina viene tradotta da ribosomi liberi nel citoplasma ed entra nel reticolo endoplasmatico successivamente.
		
			\paragraph{Fattori coinvolti}
			\begin{multicols}{2}
				\begin{itemize}
					\item Complesso \emph{Sec 62/63/71/72}.
					\item Chaperoni.
				\end{itemize}
			\end{multicols}

				\subparagraph{Chaperoni}
				I chaperoni fanno parte della famiglia delle \emph{Hsp}, si legano alla catena poli peptidica completa impedendole di assumere strutture secondarie.
				Questo avviene per permettere il passaggio attraverso il traslocone.

			\paragraph{Processo}
			Non viene coinvolto il complesso \emph{SRP}, ma uno composto di \emph{Sec63/63}.
			Queste sono proteine associate al traslocone nella membrana del reticolo endoplasmatico.
			La porzione terminale del peptide interagisce con esso portando la proteina in prossimit\`a del traslocone.
			\emph{Bip1} facilita l'ingresso della catena polipeptidica, \`e un \emph{ATPasi}.

		\subsubsection{Inserimento delle proteine nel reticolo endoplasmatico}
		Nel reticolo endoplasmatico vengono inserite proteine che possono essere secrete nell'ambiente extra-cellulare o proteine di membrana.
		Nel secondo caso la traslocazione non deve essere completa.
		Le proteine trasportate sel reticolo endoplasmatico non entrano nel lume ma rimangono associate alla sua membrana e raggiungono la destinazione finale seguendo lo stesso percorso delle proteine secrete.

			\paragraph{Meccanismi di inserimento delle proteine di membrana nel doppio strato fosfolipidico}
			L'orientamento delle proteine in membrana \`e determinato dall'orientamento del peptide durante la traslocazione nel reticolo endoplasmatico.
			Questo determina che dominio espongono nell'ambiente citoplasmatico.

				\subparagraph{Inserimento in membrana}
				Nel reticolo endoplasmatico si forma una vescicola con la proteina in questione che fonde con la membrana.
				La porzione che si trovava all'interno viene esposta all'esterno della cellula.

				\subparagraph{Proteine che attraversano la membrana pi\`u volte}
				La porzione trans-membrana di queste proteine ha tipicamente una struttura ad $\alpha$-elica, \`e idrofobica all'esterno e idrofilica all'interno.
			La struttura trans-membrana ha pi\`u porzioni idrofobiche che interagiscono con la membrana plasmatica.

	\subsection{Modifiche post-traduzionali}

