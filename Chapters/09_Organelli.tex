\chapter{Organelli}

\section{Panoramica}
Si intende per organello un qualsiasi compartimento della cellula eucariote circondato da membrana.
Il sistema di membrane si trova alla base del loro sviluppo evolutivo.

\section{Trasporto di proteine}
I diversi organelli rendono necessari diversi sistemi per permettere il trasporto delle proteine prodotte per la maggior parte nel citoplasma negli organelli in cui dovranno svolgere la loro funzione.

	\subsection{Trafficking nucleo-citoplasma}
	
		\subsubsection{Membrane del nucleo}
		
			\paragraph{Involucro nucleare}
			Il nucleo \`e circondato da un involucro nucleare formato da due membrane.
			La membrana esterna \`e in continuit\`a con il reticolo endoplasmatico pur avendo una composizione diversa.

			\paragraph{Lamina nucleare}
			La lamina nucleare \`e un reticolo fibroso con funzione strutturale.
			Mette in contatto componenti che si associano al DNA e proteine associate alla membrana interna.

			\paragraph{Funzione}
			La membrana isola e protegge il materiale genetico, ma \`e sede di numerosi trasporti.

				\subparagraph{Pori nucleari}
				I pori nucleari \emph{NPC} sono formati da $30$ nucleoporine.
				Il numero varia in base al tipo cellulare.
				Permettono un trasporto molto veloce.
				La parte interna presenta una struttura a cono, mentre l'esterna ha una struttura filiforme.
				Possiede proteine che la ancorano alla membrana, proteine strutturali e che delimitano il canale.
				La parte interna funge da filtro con una struttura ben definita.
		
		\subsubsection{Trasporto}
		Affinch\`e avvenga il trasporto una proteina deve possedere la sequenza segnale.
		La proteina interagisce con la parte fibrillare, viene inserita nel centro del poro dove si completa il trasporto.

			\paragraph{Importine}
			Le importine mediano l'entrata nel nucleo grazie al \emph{NLS}.

			\paragraph{Proteina cargo}
			Si intende per proteina cargo una proteina adattatrice che espone \emph{NLS} quando cambia conformazione e viene riconosciuta dall'importina.



	\subsection{Trasporto in plastidi e mitocondri}

\section{Perossisomi}

\section{Reticolo endoplasmatico}

	\subsection{Funzioni}

	\subsection{Tipologie}

	\subsection{Traduzione e traslocazione delle proteine}
