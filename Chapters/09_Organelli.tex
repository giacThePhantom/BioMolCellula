\chapter{Organelli}

\section{Panoramica}
Si intende per organello un qualsiasi compartimento della cellula eucariote circondato da membrana.
Il sistema di membrane si trova alla base del loro sviluppo evolutivo.

\section{Trasporto di proteine}
I diversi organelli rendono necessari diversi sistemi per permettere il trasporto delle proteine prodotte per la maggior parte nel citoplasma negli organelli in cui dovranno svolgere la loro funzione.

	\subsection{Trafficking nucleo-citoplasma}
	
		\subsubsection{Membrane del nucleo}
		
			\paragraph{Involucro nucleare}
			Il nucleo \`e circondato da un involucro nucleare formato da due membrane.
			La membrana esterna \`e in continuit\`a con il reticolo endoplasmatico pur avendo una composizione diversa.

			\paragraph{Lamina nucleare}
			La lamina nucleare \`e un reticolo fibroso con funzione strutturale.
			Mette in contatto componenti che si associano al DNA e proteine associate alla membrana interna.

			\paragraph{Funzione}
			La membrana isola e protegge il materiale genetico, ma \`e sede di numerosi trasporti.

				\subparagraph{Pori nucleari}
				I pori nucleari \emph{NPC} sono formati da $30$ nucleoporine.
				Il numero varia in base al tipo cellulare.
				Permettono un trasporto molto veloce.
				La parte interna presenta una struttura a cono, mentre l'esterna ha una struttura filiforme.
				Possiede proteine che la ancorano alla membrana, proteine strutturali e che delimitano il canale.
				La parte interna funge da filtro con una struttura ben definita.
		
		\subsubsection{Trasporto}
		Affinch\`e avvenga il trasporto una proteina deve possedere la sequenza segnale.
		La proteina interagisce con la parte fibrillare, viene inserita nel centro del poro dove si completa il trasporto.

			\paragraph{Importine}
			Le importine mediano l'entrata nel nucleo grazie al \emph{NLS}.

			\paragraph{Proteina cargo}
			Si intende per proteina cargo una proteina adattatrice che espone \emph{NLS} quando cambia conformazione e viene riconosciuta dall'importina.

			\paragraph{Importo della proteina}
			Per importare una proteina con \emph{NLS} forma un complesso con l'importina.
			Questo interagisce con la parte fibrillare del poro nucleare.
			L'importina interagisce con la matrice creando uno spazio di passaggio spezzando legami: l'importina is lega a \emph{Ran-GTP} cambiando conformazione.
			Il complesso entra nel nucleo.
			Avviene un idrolisi in \emph{Ran-GDP} che causa il rilascio dell'importina che viene esportata dal nucleo.

			\paragraph{Esporto}
			Per esportare una porteina dal nucleo l'esportina lega \emph{Ran-GTP} e il cargo causando l'uscita del complesso.
			L'idrolisi in \emph{Ran-GDP} causa un cambio conformazionale dell'esportina rilasciando la proteina nel cargo.

			\paragraph{Gradiente di \emph{Ran}}
			Si nota un gradiente di \emph{Ran-GTP}: questa \`e determinata da \emph{Ran-GAP} che interagiscono con la parte fibrillare delle nucleoporine.
			Questo garantisce una direzionalit\`a del trasporto in concomitanza con \emph{Ran-GEF}.

			\paragraph{Mascheramento di \emph{NLS}}
			Una proteina con \emph{NLS} che non localizza nel nucleo possiede modifiche alla sequenza segnale come fosforilazioni.
	
				\subparagraph{Sintesi del coleserolo}
				La molecola \emph{NF-AT} fosforilata \`e disattiva e maschera il \emph{NLS}.
				Un aumento di calcio attiva la calcineurina, una fosfatasi che rende di nuovo accessibile il \emph{NLS}.
				Viene portata nel nucleo una \emph{SREBP}, una proteina coinvolta nella sintesi del colesterolo come fattore trascrizionale.
				Il colesterolo quando presente interagisc con \emph{SCAP}, una proteina legata a \emph{SREBP}.
				Il complesso in assenza di colesterolo si localizza al Golgi.
				Qui proteasi tagliano \emph{SREB} liberando un frammento che va ad attivare la trascrizione di geni coinvolti nella sintesi del colesterolo.



	\subsection{Trasporto in plastidi e mitocondri}
	Affinch\`e le proteine possano essere riconosciute dai recettori ed importate nei mitocondri devono essere completamente tradotti.

		\subsubsection{Peptide segnale}
		Il peptide segnale \`e formato da $18$ amminocidi polari e non polari che formano una struttura ad $\alpha$-elica asimmetrica.

		\subsubsection{Traslocazione nel mitocondrio}
		La traslocazione di una proteina nel mitocondrio richiede due complessi: il complesso \emph{TOM} che media il trasporto dal citoplasma nello spazio tra le membrane e il complesso \emph{TIM} che completa il trasporto nella membrana interna.
		La proteina possiede il peptide segnale e viene riconosciuta dalla porzione recettoriale del complesso \emph{TOM} e viene forzata ad entrare.
		Qui incontra il complesso \emph{TIM} che la porta attraverso la membrana interna assieme alla chaperonina \emph{HSP70} (impedisce la formazione di strutture secondarie).
		In questo secondo passaggio avviene il taglio del peptide segnale.
		I due copessi possono lavorare in modo indipendente.

			\paragraph{Energia necessaria}
			L'energia necessaria alla traslocazione viene fornita da:
			\begin{multicols}{2}
				\begin{itemize}
					\item Idrolisi di \emph{ATP} da parte di \emph{TOM}.
					\item Gradiente elettrochimico su \emph{TIM}.
				\end{itemize}
			\end{multicols}

		\subsubsection{Inserimento di una proteina nella matrice tra le membrane}
		Una proteina che deve rimanere nello spazio tra le due membrane possiede il peptide segnale e una sequenza di stop del trasferimento che blocca il passaggio della proteina verso il citoplasma.

		\subsubsection{Complesso \emph{Oxa}}
		Il complesso \emph{Oxa} media l'inserzione di proteine nella membrana interna.
		La proteina attraversa \emph{TIM} dove le viene rimosso il peptide segnale, ma ne contiene un secondo che la indirizza verso il complesso \emph{Oxa}.
		Questo permette l'ancoraggio della proteina all'interno della membrana interna del mitocondrio.

\section{Perossisomi}

	\subsection{Panoramica}
	I perossisomi sono vestigia di antichi organelli che svolgevano funzioni nel metabolismo dell'ossigeno.

		\subsubsection{Struttura}
		I perossisomi sono delimitati da membrana, presentano una forma tondeggiante e inclusioni paracristalline.
		Contengono diversi enzimi.

		\subsubsection{Funzione}
		Gli enzimi nei perossisomi permettono:
		\begin{multicols}{2}
			\begin{itemize}
				\item Rimozione di \emph{H} da substrati organici specifici con liberazione di acqua ossigenata trasformata in acqua dalla catalasi.
				\item Demolizione degli acidi grassi attraverso $\beta$-ossidazione in acetil \emph{CoA}.
				\item Catalizzano le prime reazioni nella formazione dei plasmalogeni, che formano una guaina che riveste gli assoni.
				\item Sono presenti nelle cellule vegetali e coinvolti nel ciclo del glicosilato per la conversione degli acidi grassi in zuccheri.
			\end{itemize}
		\end{multicols}

	\subsection{Trasporto}
	I perossisomi non presentano DNA e pertanto tutte le proteine necessarie al loro comportamento devono essere importate.
	Il meccanismo di trasporto \`e simile a quello dei mitocondri.

		\subsubsection{Sequenza target}
		La sequenza target si dice \emph{PLS} ed \`e costituita da \emph{Ser-Lys-Leu} N o C terminale.

		\subsubsection{Riconoscimento della sequenza target}
		La sequenza target viene riconosciuta dalle perossine o \emph{Pex}.
		Queste partecipano al processo di importazione.
		\begin{multicols}{2}
			\begin{itemize}
				\item \emph{Pex5}: riconosce una \emph{PLS} C terminale insieme a \emph{Pex1} e la \emph{ATPasi} \emph{Pex6}.
				\item \emph{Pex7}: riconosce una \emph{PLS} N terminale.
			\end{itemize}
		\end{multicols}

	\subsection{Genesi dei perossisomi}
	I perossisomi si formano da gemmazioni del reticolo endoplasmatico.
	Questo forma una struttura vescicolare con \emph{Pex3} come proteina transmembrana e \emph{Pex19} ad essa associato.
	Questi vengono detti protoperossisomi e vengono arricchiti da componenti che causano un loro accrescimento.
	Possono inoltre andare incontro a fissione.

	\subsection{Sindrome di Zellweger}
	La sindrome di Zellweger presenta anomalie celebrali, del fegato e reni.
	Queste sono causate da mutazioni a carico di \emph{Pex5}, non si trova un importo di enzimi con funzioni ossidative.

\section{Reticolo endoplasmatico}
Il reticolo endoplasmatico \`e un insieme di membrane in continuit\`a fisica ma non di composizione con la membrana nucleare.

	\subsection{Funzioni}
	Il reticolo endoplasmatico ha diverse funzioni:
	\begin{multicols}{2}
		\begin{itemize}
			\item Traduzione e modifica post-traduzionale di proteine.
			\item Biosintesi e traslocazione, in particolare nell'apparato del Golgi.
			\item Trasduzione del segnale: \`e il principale magazzino del calcio intracellulare.
				Ha pertanto un ruolo chiave nella contrazione muscolare.
		\end{itemize}
	\end{multicols}

	\subsection{Tipologie}
	Il reticolo endoplasmatico pu\`o essere diviso in tre regioni principali:
	\begin{multicols}{2}
		\begin{itemize}
			\item \emph{RE rugoso}: si presenta rugoso a causa della presenza dei ribosomi.
			\item \emph{RE intermedio} o di transiizone: consente il passaggio delle proteine al Golgi.
			\item \emph{RE liscio}: sistema di membrane per l'accumulo dello ione calcio e la sintesi dei lipidi.
				Questi vengono poi trasportati nei distretti della cellula attraverso vescicole.
		\end{itemize}
	\end{multicols}
	Questa organizzazione non \`e statica ma va incontro ad un notevole riarrangiamento grazie al citoscheletro e proteine specifiche.

	\subsection{Traduzione e traslocazione delle proteine}
	Le proteine che devono essere traslocate nel reticolo endoplasmatico presentano una sequenza segnale rimosso poi da una peptidasi quando traslocata all'interno.
	Gli mRNA per queste proteine possono essere tradotti da ribosomi legati alla membrana del reticolo endoplasmatico per proteine che rimangono all'interno di vescicole, proteine di membrana o che avanno in organelli.
	Un altro modo di traduzione \`e da ribosomi liberi nel citoplasma per proteine che vanno nel nucleo, nei mitocondri, cloroplasti o perossisomi.
	Gli enzimi deputati alla modifica elle proteine si trovano all'interno del reticolo endoplasmatico.

		\subsubsection{Traslocazione co-traduzionale}
		La traslocazione co-traduzionale consiste nel passaggio della proteina nel reticolo endoplasmatico mentre viene tradotta.
		I ribosomi sono legati alla membrana del reticolo endoplasmatico e alla fine la proteina si trova associata ad esso.

		\subsubsection{Traslocazione post-traduzionale}
		Nella traslocazione post-traduzionale la proteina viene tradotta da ribosomi liberi nel citoplasma ed entra nel reticolo endoplasmatico successivamente.
