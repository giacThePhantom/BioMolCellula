\chapter{Organelli}

\section{Panoramica}
Si intende per organello un qualsiasi compartimento della cellula eucariote circondato da membrana.
Il sistema di membrane si trova alla base del loro sviluppo evolutivo.

\section{Trasporto di proteine}
I diversi organelli rendono necessari diversi sistemi per permettere il trasporto delle proteine prodotte per la maggior parte nel citoplasma negli organelli in cui dovranno svolgere la loro funzione.

	\subsection{Trafficking nucleo-citoplasma}
	
		\subsubsection{Membrane del nucleo}
		
			\paragraph{Involucro nucleare}
			Il nucleo \`e circondato da un involucro nucleare formato da due membrane.
			La membrana esterna \`e in continuit\`a con il reticolo endoplasmatico pur avendo una composizione diversa.

			\paragraph{Lamina nucleare}
			La lamina nucleare \`e un reticolo fibroso con funzione strutturale.
			Mette in contatto componenti che si associano al DNA e proteine associate alla membrana interna.

			\paragraph{Funzione}
			La membrana isola e protegge il materiale genetico, ma \`e sede di numerosi trasporti.

				\subparagraph{Pori nucleari}
				I pori nucleari \emph{NPC} sono formati da $30$ nucleoporine.
				Il numero varia in base al tipo cellulare.
				Permettono un trasporto molto veloce.
				La parte interna presenta una struttura a cono, mentre l'esterna ha una struttura filiforme.
				Possiede proteine che la ancorano alla membrana, proteine strutturali e che delimitano il canale.
				La parte interna funge da filtro con una struttura ben definita.
				La struttura filiforme \`e indefinita e formata da glicina e finilanalina.
				Queste strutture vengono separate all'arrivo di una molecola da importare.
				Queste strutture vengono separate all'arrivo di una molecola da importare.
		
		\subsubsection{Trasporto}
		Affinch\`e avvenga il trasporto una proteina deve possedere la sequenza segnale.
		La proteina interagisce con la parte fibrillare, viene inserita nel centro del poro dove si completa il trasporto.

			\paragraph{Importine}
			Le importine mediano l'entrata nel nucleo grazie al \emph{NLS}.

			\paragraph{Proteina cargo}
			Si intende per proteina cargo una proteina adattatrice che espone \emph{NLS} quando cambia conformazione e viene riconosciuta dall'importina.

			\paragraph{Importo della proteina}
			Per importare una proteina con \emph{NLS} forma un complesso con l'importina.
			Questo interagisce con la parte fibrillare del poro nucleare.
			L'importina interagisce con la matrice creando uno spazio di passaggio spezzando legami: l'importina is lega a \emph{Ran-GTP} cambiando conformazione.
			Il complesso entra nel nucleo.
			Avviene un idrolisi in \emph{Ran-GDP} che causa il rilascio dell'importina che viene esportata dal nucleo.

			\paragraph{Esporto}
			Per esportare una proteina dal nucleo l'esportina lega \emph{Ran-GTP} e il cargo causando l'uscita del complesso.
			L'idrolisi in \emph{Ran-GDP} causa un cambio conformazionale dell'esportina rilasciando la proteina nel cargo.

			\paragraph{Gradiente di \emph{Ran}}
			Si nota un gradiente di \emph{Ran-GTP}: questa \`e determinata da \emph{Ran-GAP} che interagiscono con la parte fibrillare delle nucleoporine.
			Questo garantisce una direzionalit\`a del trasporto in concomitanza con \emph{Ran-GEF}.

			\paragraph{Mascheramento di \emph{NLS}}
			Una proteina con \emph{NLS} che non localizza nel nucleo possiede modifiche alla sequenza segnale come fosforilazioni.
	
				\subparagraph{Sintesi del coleserolo}
				La molecola \emph{NF-AT} fosforilata \`e disattiva e maschera il \emph{NLS}.
				Un aumento di calcio attiva la calcineurina, una fosfatasi che rende di nuovo accessibile il \emph{NLS}.
				Viene portata nel nucleo una \emph{SREBP}, una proteina coinvolta nella sintesi del colesterolo come fattore trascrizionale.
				Il colesterolo quando presente interagisc con \emph{SCAP}, una proteina legata a \emph{SREBP}.
				Il complesso in assenza di colesterolo si localizza al Golgi.
				Qui proteasi tagliano \emph{SREB} liberando un frammento che va ad attivare la trascrizione di geni coinvolti nella sintesi del colesterolo.



	\subsection{Trasporto in plastidi e mitocondri}
	Affinch\`e le proteine possano essere riconosciute dai recettori ed importate nei mitocondri devono essere completamente tradotti.

		\subsubsection{Peptide segnale}
		Il peptide segnale \`e formato da $18$ amminocidi polari e non polari che formano una struttura ad $\alpha$-elica asimmetrica.

		\subsubsection{Traslocazione nel mitocondrio}
		La traslocazione di una proteina nel mitocondrio richiede due complessi: il complesso \emph{TOM} che media il trasporto dal citoplasma nello spazio tra le membrane e il complesso \emph{TIM} che completa il trasporto nella membrana interna.
		La proteina possiede il peptide segnale e viene riconosciuta dalla porzione recettoriale del complesso \emph{TOM} e viene forzata ad entrare.
		Qui incontra il complesso \emph{TIM} che la porta attraverso la membrana interna assieme alla chaperonina \emph{HSP70} (impedisce la formazione di strutture secondarie).
		In questo secondo passaggio avviene il taglio del peptide segnale.
		I due copessi possono lavorare in modo indipendente.

			\paragraph{Energia necessaria}
			L'energia necessaria alla traslocazione viene fornita da:
			\begin{multicols}{2}
				\begin{itemize}
					\item Idrolisi di \emph{ATP} da parte di \emph{TOM}.
					\item Gradiente elettrochimico su \emph{TIM}.
				\end{itemize}
			\end{multicols}

		\subsubsection{Inserimento di una proteina nella matrice tra le membrane}
		Una proteina che deve rimanere nello spazio tra le due membrane possiede il peptide segnale e una sequenza di stop del trasferimento che blocca il passaggio della proteina verso il citoplasma.

		\subsubsection{Complesso \emph{Oxa}}
		Il complesso \emph{Oxa} media l'inserzione di proteine nella membrana interna.
		La proteina attraversa \emph{TIM} dove le viene rimosso il peptide segnale, ma ne contiene un secondo che la indirizza verso il complesso \emph{Oxa}.
		Questo permette l'ancoraggio della proteina all'interno della membrana interna del mitocondrio.

\section{Perossisomi}

	\subsection{Panoramica}
	I perossisomi sono vestigia di antichi organelli che svolgevano funzioni nel metabolismo dell'ossigeno.

		\subsubsection{Struttura}
		I perossisomi sono delimitati da membrana, presentano una forma tondeggiante e inclusioni paracristalline.
		Contengono diversi enzimi.

		\subsubsection{Funzione}
		Gli enzimi nei perossisomi permettono:
		\begin{multicols}{2}
			\begin{itemize}
				\item Rimozione di \emph{H} da substrati organici specifici con liberazione di acqua ossigenata trasformata in acqua dalla catalasi.
				\item Demolizione degli acidi grassi attraverso $\beta$-ossidazione in acetil \emph{CoA}.
				\item Catalizzano le prime reazioni nella formazione dei plasmalogeni, che formano una guaina che riveste gli assoni.
				\item Sono presenti nelle cellule vegetali e coinvolti nel ciclo del glicosilato per la conversione degli acidi grassi in zuccheri.
			\end{itemize}
		\end{multicols}

	\subsection{Trasporto}
	I perossisomi non presentano DNA e pertanto tutte le proteine necessarie al loro comportamento devono essere importate.
	Il meccanismo di trasporto \`e simile a quello dei mitocondri.

		\subsubsection{Sequenza target}
		La sequenza target si dice \emph{PLS} ed \`e costituita da \emph{Ser-Lys-Leu} N o C terminale.

		\subsubsection{Riconoscimento della sequenza target}
		La sequenza target viene riconosciuta dalle perossine o \emph{Pex}.
		Queste partecipano al processo di importazione.
		\begin{multicols}{2}
			\begin{itemize}
				\item \emph{Pex5}: riconosce una \emph{PLS} C terminale insieme a \emph{Pex1} e la \emph{ATPasi} \emph{Pex6}.
				\item \emph{Pex7}: riconosce una \emph{PLS} N terminale.
			\end{itemize}
		\end{multicols}

	\subsection{Genesi dei perossisomi}
	I perossisomi si formano da gemmazioni del reticolo endoplasmatico.
	Questo forma una struttura vescicolare con \emph{Pex3} come proteina transmembrana e \emph{Pex19} ad essa associato.
	Questi vengono detti protoperossisomi e vengono arricchiti da componenti che causano un loro accrescimento.
	Possono inoltre andare incontro a fissione.

	\subsection{Sindrome di Zellweger}
	La sindrome di Zellweger presenta anomalie celebrali, del fegato e reni.
	Queste sono causate da mutazioni a carico di \emph{Pex5}, non si trova un importo di enzimi con funzioni ossidative.

\section{Reticolo endoplasmatico}
Il reticolo endoplasmatico \`e un insieme di membrane in continuit\`a fisica ma non di composizione con la membrana nucleare.

	\subsection{Funzioni}
	Il reticolo endoplasmatico ha diverse funzioni:
	\begin{multicols}{2}
		\begin{itemize}
			\item Traduzione e modifica post-traduzionale di proteine.
			\item Biosintesi e traslocazione, in particolare nell'apparato del Golgi.
			\item Trasduzione del segnale: \`e il principale magazzino del calcio intracellulare.
				Ha pertanto un ruolo chiave nella contrazione muscolare.
		\end{itemize}
	\end{multicols}

	\subsection{Tipologie}
	Il reticolo endoplasmatico pu\`o essere diviso in tre regioni principali:
	\begin{multicols}{2}
		\begin{itemize}
			\item \emph{RE rugoso}: si presenta rugoso a causa della presenza dei ribosomi.
			\item \emph{RE intermedio} o di transizione: consente il passaggio delle proteine al Golgi.
			\item \emph{RE liscio}: sistema di membrane per l'accumulo dello ione calcio e la sintesi dei lipidi.
				Questi vengono poi trasportati nei distretti della cellula attraverso vescicole.
		\end{itemize}
	\end{multicols}
	Questa organizzazione non \`e statica ma va incontro ad un notevole riarrangiamento grazie al citoscheletro e proteine specifiche.

	\subsection{Traduzione e traslocazione delle proteine}
	Le proteine che devono essere traslocate nel reticolo endoplasmatico presentano una sequenza segnale rimosso poi da una peptidasi quando traslocata all'interno.
	Gli mRNA per queste proteine possono essere tradotti da ribosomi legati alla membrana del reticolo endoplasmatico per proteine che rimangono all'interno di vescicole, proteine di membrana o che avanno in organelli.
	Un altro modo di traduzione \`e da ribosomi liberi nel citoplasma per proteine che vanno nel nucleo, nei mitocondri, cloroplasti o perossisomi.
	Gli enzimi deputati alla modifica elle proteine si trovano all'interno del reticolo endoplasmatico.

		\subsubsection{Traslocazione co-traduzionale}
		La traslocazione co-traduzionale consiste nel passaggio della proteina nel reticolo endoplasmatico mentre viene tradotta.
		I ribosomi sono legati alla membrana del reticolo endoplasmatico e alla fine la proteina si trova associata ad esso.
		La traslocazione co-traduzionale avviene grazie alla presenza di una sequenza segnale lungo il peptide nascente.
		Il ribosoma e la catena peptidica vengono portati sul reticolo endoplasmatico.
		Qui la traduzione continua quando i ribosomi aderiscono al reticolo endoplasmatico.

			\paragraph{Sequenza segnale}
			La sequenza segnale localizza nella regione ammino-terminale della catena polipeptidica.
			\`E costituita da una sequenza di circa $20$ amminoacidi idrofobici preceduti da un'arginina.
			
			\paragraph{Riconoscimento della sequenza segnale}
			Il peptide segnale viene riconosciuto da un complesso \emph{SRP} o signal recognition particle che dirige le proteine con una specifica sequenza segnale ad un recettore sul reticolo endoplasmatico.
			Blocca pertanto la traduzione e porta il ribosoma sul reticolo endoplasmatico.

				\subparagraph{Composizione \emph{SRP}}
				\emph{SRP} \`e composto da $6$ proteine e $1$ RNA\@.
				Una parte interagisce con proteine ed RNA ribosomiali bloccando la traduzione.
				Un'altra interagisce con la sequenza segnale.
				Una terza interagisce con il recettore sul reticolo endoplasmatico.
				Si trova un hinge che permette un cambio conformazionale.

			\paragraph{Processo di traslocazione}
			\begin{multicols}{2}
				\begin{enumerate}
					\item Il mRNA viene tradotto dai ribosomi presenti nel citoplasma.
					\item Viene riconosciuto il peptide segnale da \emph{SRP}.
					\item Il ribosoma si ferma e il complesso viene reclutato sulla membrana del reticolo endoplasmatico.
					\item Il complesso \emph{ribosoma-RNA-SRP} viene portato sulla membrana del reticolo endoplasmatico dove si lega a un recettore specifico.
					\item Il legame con il recettore permette l'associazione del ribosoma trasducente al reticolo endoplasmatico rugoso.
					\item Si distacca \emph{SRB} dal peptide segnale.
				\end{enumerate}
			\end{multicols}

			\paragraph{Ripresa della traduzione}
			Il distacco di \emph{SRP} da ribosomi e sequenza segnale viene mediato dall'idrolisi del \emph{GTP}.
			La proteina passa attraverso un canale trans-membrana traslocatore ed entra nel lume del reticolo endoplasmatico.

				\subparagraph{Traslocone}
				Il traslocone \`e il canale che mette in comunicazione il lume del reticolo endoplasmatico con la membrana permettendo di continuare la traduzione.
				\`e composto da $3$ proteine trans-membrana \emph{Sec61} $\alpha$, $\beta$ e $\gamma$.
				I domini che formano il poro formano $\alpha$-eliche, come quelli che formano il tappo e lo chiudono.
				Si trova un hinge che permette l'apertura e la chiusura del poro e un plug che permette l'ingresso della proteina in fase di sintesi.

			\paragraph{Ingresso della proteina}
			La proteina che entra nel reticolo endoplasmatico viene legata dalla chaperonina \emph{Bip} per facilitarne il passaggio.
			Una peptidasi rimuove il peptide segnale.

			\paragraph{Proteine con domini transmembrana}
			Le proteine con domini transmembrana contengono una sequenza di stop.
			Questa quando entra nel traslocone determina la sua apertura e il rilascio della proteina.
			Questa pertanto rimane nella membrana con la parte $N$ termminale rivolta nel lume.
			La sequenza di terminazione interna pu\`o inoltre invertire l'orientamento.
			
				\subparagraph{Multipli domini transmembrana}
				Le proteine con multipli domini transmembrana possiedono una sequenza di trasferimento interna seguita da una sequenza di stop.
				Queste si alternano discriminando i domini transmembrana da quelli citoplasmatici.

			\paragraph{\emph{Sec61}}
			\emph{Sec61} \`e un esempio di traslocone e pu\`o essere utilizzato come marcatore del reticolo endoplasmatico.
			Associato ad esso si trova una proteasi con il compito di rimuovere il peptide segnale.
			La peptidasi rimuovendo il peptide segnale permette alla catena polipeptidica di essere rilasciata nel lume del reticolo endoplasmatico.

			\paragraph{Trasferimento di proteine inter-membrana}

				\subparagraph{Processo}
				\begin{multicols}{2}
					\begin{enumerate}
						\item Grazie alla sequenza segnale il ribosoma viene portato sul traslocone.
						\item Inizia la sintesi della proteina.
						\item Avviene un taglio ad opera della peptidasi.
						\item La traduzione continua e la proteina entra nel reticolo endoplasmatico.
						\item Viene tradotta la regione idrofobica di $\alpha$-elica come blocco di trasferimento.
						\item La sequenza introdotta blocca il trasferimento causando l'apertura del traslocone e l'interazione della proteina neosintetizzata con la membrana.
						\item La proteina rimane in membrana mentre il ribosoma traduce la parte rimanente fino al completamento della traduzione.
					\end{enumerate}
				\end{multicols}

				\subparagraph{Domini C terminali all'interno del lume del reticolo endoplasmatico}
				Affinch\`e la porzione $C$-terminale si trovi all'interno del lume la sintesi proteica avviene nel citoplasma fino alla produzione di una sequenza segnale.
				Questa, presente in una regione interna del peptide causa un suo reclutamento al traslocone, dove viene portata all'interno da quella posizione.
				In questo caso non avviene il taglio peptidico ad opera della peptidasi.
				Un dominio trans-membrana ancora la proteina sulla membrana.

				\subparagraph{Multipli domini trans-membrana}
				In caso di multipli domini trans-membrana si trovano multiple sequenze segnale e di stop.
				La traslocazione viene interrotta alla prima sequenza di stop, ma una successiva traduzione di una sequenza segnale causa una ripresa del processo dal punto interno.

			\paragraph{Chaperonine}
			Mentre viene tradotta la proteina si associa a chaperonine, proteine che impediscono il ripiegamento della proteina nelle strutture secondarie e terziarie.

			\paragraph{Ponti disolfuro}
			Nel lume del reticolo endoplasmatico avvengono reazioni di formazione dei ponti disolfuro che non avvengono nell'ambiente citoplasamtico a causa della sua natura riducente.
			Questo infatti li mantiene nella forma \emph{SH-HS}.
			Nell'ambiente ossidante del reticolo endoplasmatico i gruppi diventano \emph{S-S} formando legami covalenti.
			La formazione dei ponti disolfuro determina l'assunzione di strutture secondarie e terziarie alle proteine.

		\subsubsection{Traslocazione post-traduzionale}
		Nella traslocazione post-traduzionale la proteina viene tradotta da ribosomi liberi nel citoplasma ed entra nel reticolo endoplasmatico successivamente.
		
			\paragraph{Fattori coinvolti}
			\begin{multicols}{2}
				\begin{itemize}
					\item Complesso \emph{Sec 62/63/71/72}.
					\item Chaperoni.
				\end{itemize}
			\end{multicols}

				\subparagraph{Chaperoni}
				I chaperoni fanno parte della famiglia delle \emph{Hsp}, si legano alla catena poli peptidica completa impedendole di assumere strutture secondarie.
				Questo avviene per permettere il passaggio attraverso il traslocone.

			\paragraph{Processo}
			Non viene coinvolto il complesso \emph{SRP}, ma uno composto di \emph{Sec63/63}.
			Queste sono proteine associate al traslocone nella membrana del reticolo endoplasmatico.
			La porzione terminale del peptide interagisce con esso portando la proteina in prossimit\`a del traslocone.
			\emph{Bip1} facilita l'ingresso della catena polipeptidica, \`e un \emph{ATPasi}.

		\subsubsection{Inserimento delle proteine nel reticolo endoplasmatico}
		Nel reticolo endoplasmatico vengono inserite proteine che possono essere secrete nell'ambiente extra-cellulare o proteine di membrana.
		Nel secondo caso la traslocazione non deve essere completa.
		Le proteine trasportate sel reticolo endoplasmatico non entrano nel lume ma rimangono associate alla sua membrana e raggiungono la destinazione finale seguendo lo stesso percorso delle proteine secrete.

			\paragraph{Meccanismi di inserimento delle proteine di membrana nel doppio strato fosfolipidico}
			L'orientamento delle proteine in membrana \`e determinato dall'orientamento del peptide durante la traslocazione nel reticolo endoplasmatico.
			Questo determina che dominio espongono nell'ambiente citoplasmatico.

				\subparagraph{Inserimento in membrana}
				Nel reticolo endoplasmatico si forma una vescicola con la proteina in questione che fonde con la membrana.
				La porzione che si trovava all'interno viene esposta all'esterno della cellula.

				\subparagraph{Proteine che attraversano la membrana pi\`u volte}
				La porzione trans-membrana di queste proteine ha tipicamente una struttura ad $\alpha$-elica, \`e idrofobica all'esterno e idrofilica all'interno.
			La struttura trans-membrana ha pi\`u porzioni idrofobiche che interagiscono con la membrana plasmatica.

	\subsection{Modifiche post-traduzionali}
	Una volta che le proteine sono introdotte nel lume o associate al ER vanno incontro a un primo processo di modifica post-traduzionale, la glicosilazione.

		\subsubsection{Dolicolo}
		Il dolicolo agisce come donatore di molecole di zuccheri.
		\`E una molecola lipidica di membrana.
		Dona una catena formata da glucosio, mannosio e N-acetilglucosammina al peptide tradotto.
		Il trasferimento dell'oligosaccaride alla proteina avviene all'interno del lume grazie all'oligosaccariltasferasi.

		\subsubsection{Precursore}
		
			\paragraph{Citosol}
			Nel citosol avvengono i primi passaggi:
			\begin{multicols}{2}
				\begin{enumerate}
					\item Attivazione del dolicolo tramite fosforilazione da \emph{CTP} a \emph{CDP}.
						I monomeri di zucchero sono legati a \emph{UDP} o \emph{GDP}.
					\item Dopo l'aggiunta di \emph{2 N-acetilglucosammina} e dei $5$ mannosi avviene il flipping.
				\end{enumerate}
			\end{multicols}

			\paragraph{Lume del reticolo endoplasmatico}
			Nel lume del reticolo endoplasmatico:
			\begin{multicols}{2}
				\begin{enumerate}
					\item Viene aggiunto il glicosile in corrispondenza dell'aspargina.
					\item Si forma il peptide glicosilato: $14$ zuccheri a livello dell'aspargina con $2$ \emph{N-acetilglucosammina}, $9$ mannosi e $3$ glucosi.
					\item Si rimuovono i $3$ residui del glucosio marcando la proteina con una conformazione corretta.
					\item I $3$ residui vengono rimossi in $2$ passaggi, il primo coinvolge la rimozione di $2$ unit\`a di glucosio, il secondo della rimanente.
				\end{enumerate}
			\end{multicols}

		\subsubsection{Meccanismo di controllo}
		La rimozione dei residui di glucosio \`e collegata all'attivit\`a degli chaperoni che controllano il ripiegamento delle proteine.

			\paragraph{Calnessina}
			La calnessina \`e coinvolta in uno dei meccanismi di controllo principali.
			\`E associata alla membrana e riconosce la proteina a cui \`e stato legato il precursore con $1$ glucosio.
			La glicosidasi rimuove l'ultimo glucosio e se la struttura \`e corretta lascia il reticolo endoplasmatico.
			Se sbagliata interviene la glucosil trasferasi che aggiunge un nuovo glucosio passando per la calnessina.
			
			\paragraph{Calreticulina}
			La calreticulina o marcatore del reticolo endoplasmatico riconosce e lega la proteina glicosilata.
			
			\paragraph{Misfolding persistente}
			In caso di misfolding persistente la calreticulina invia la proteina all'esterno dove viene ubiquitinata e degradata nel proteosoma.
			La proteina viene riconosciuta da una disolfuroisomerasi sui ponti disolfuro, la lectina riconosce la componente glucidica e il complesso chaperone-proteina-disolfuroisomerasi-lectina lascia il reticolo endoplasmatico.
			La fuorisuscita avviene grazie a un \emph{ATPasi}, mentre una $N$-glucanasi rimuove il complesso glucidico.

			\paragraph{Meccanismi che riducono il numero di proteine con struttura errata}
			Questi meccanismi coinvolgono una maggiore produzione di chaperoni.
			\begin{multicols}{2}
				\begin{enumerate}
					\item Viene attivato \emph{IRE1} sulla membrana del reticolo endoplasmatico, presenta una chinasi citoplasmatica e un sensore per il misfolding nel lume.
					\item \emph{IRE1} attivato va a causare uno splicing di \emph{XBP1} che attiva la trascrizione dei geni necessari per far fronte al misfolding.
					\item \emph{ATF6} attivata viene importata nel Golgi e tagliata da un enzima.
						La subunit\`a tagliata viene rilasciata nel nucleo.
					\item \emph{PERK} causa un blocco traduzionale di alcuni mRNA aumentando la traduzione dei fattori regolatori della trascrizione per i chaperoni.
				\end{enumerate}
			\end{multicols}

	\subsection{Proteine legate alla membrana attraverso \emph{GPI}}
	Alcune proteine di membrana presentano una porzione C-terminale ancorata in membrana attraverso \emph{GPI} (glicosilfosfatidilinositolo).
	Questo legame pu\`o essere rotto dalla fosfolipasi $C$ extracellulare determinando una secrezione dellla proteina.
	La porzione C-terminale di queste proteine \`e estremamente idrofobica.

	\subsection{Sintesi dei fosfolipidi}
	Nel reticolo endoplasmatico liscio vengono prodotti:
	\begin{multicols}{2}
		\begin{itemize}
			\item Fosfolipidi: costituenti di base della membrana plasmatica.
			\item Glicerolo: molecola precursore per la sintesi dei fosfolipidi
			\item Sfingomielina: lipide di membrana che non contiene il glicerolo, precursore della ceramide.
				Forma una guaina intorno ai nervi e d\`a una resistenza alla membrana al potenziale elettrico.
		\end{itemize}
	\end{multicols}

		\subsubsection{Processo}
		La sintesi avviene nella membrana del reticolo endoplasmatico: gli enzimi si trovano nel citosol e la produzione avviene sullo strato esterno.

			\paragraph{Fosfatidilcolina}
			La fosfatidilcolina viene sintetizzata a partire da glicerolo-3-fosfato e acidi grassi legati al coenzima $A$, il prodotto viene modificato da enzimi e la molecola mantiene le catene di acidi grassi all'interno della membrana mentre gli enzimi svolgono l'attivit\`a catalitica nel citosol.
			\begin{multicols}{2}
				\begin{enumerate}
					\item L'acido grasso ancorato ad una proteina viene ancorato nella porzione citosilica del reticolo endoplasmatico.
					\item Vengono aggiunti $2$ \emph{CoA}.
					\item Un enzima aggiunge il glicerolo-3-fosfato, formando acido fosfatidico.
					\item Si attacca la colina formando la fosfatidilcolina.
				\end{enumerate}
			\end{multicols}

			\paragraph{Ceramide}
			La ceramide viene prodotta partendo da una serina attaccata ad un acido grasso formando la sfingosina, mentre la seconda catena di acidi grassi viene aggiunta all'apparato di Golgi.
			\`E un importante componente delle membrane del sistema nervoso.

		\subsubsection{Distribuzione dei lipidi}
		Si nota come i lipidi si trovano da entrambe le parti della mebmrana.
		I lipidi neosintetizzati rimangono nel foglietto rivolto verso il citoplasma, mentre altri vengono trasferiti sul lato luminale.
		La sintesi asimmetrica rende necessarie scramblasi, proteine di membrana che trasportano i lipidi prodotti da una parte all'altra della membrana.
		Avviene una crescita simmetrica dei due monolayer.

	\subsection{Esportazione delle proteine}
	Le proteine che vanno incontro ad un folding corretto passano dal reticolo endoplasmatico all'apparato di Golgi.
	Passano attraverso il reticolo endoplasmatico di transizione.
	Il trasporto \`e di tipo vescicolare: le proteine nel lume e sulla membrana vengono incorporate in vescicole che passano nell'apparato successivo e poi nel Golgi.

\section{Trasporto vescicolare}

	\subsection{Direzionalit\`a del trasporto}

		\subsubsection{Escitosi}
		Si intende per esocitosi il processo di secrezione di molecole attraverso vescicole

		\subsubsection{Endocitosi}
		Si intende per endocitosi il processo di internalizzazione di parti della matrice extracellulare attraverso entroflessione della membrana plasmatica.
		\`E attivato da recettori.
	
	\subsection{Traffico vescicolare}
	Esiste traffico bidirezionale tra:
	\begin{multicols}{2}
		\begin{itemize}
			\item reticolo endoplasmatico-Golgi.
			\item Golgi-vescicole secretore.
			\item Golgi-late endosome.
			\item Golgi-early-endosome.
		\end{itemize}
	\end{multicols}
	Esiste in oltre un traffico dal Golgi alla membrana esterna.
	
	\subsection{Trasporto secretorio costitutivo}
	Si intende per trasporto secretorio costitutivo un processo in cui le vescicole si formano, sono portate in membrana dove si fondono ad essa e rilasciano il loro contenuto nella matrice extracellulare.

	\subsection{Trasporto secretorio regolatorio}
	Si intende per trasporto secretorio regolatorio un meccanismo di trasporto regolato da segnali esterni.
	Questi quando arrivano creano una cascata interna di segnale che causa una fusione della vescicola e il rilascio all'esterno.
	La maturazione delle vescicole nel trans-Golgi permette un accumulo della sostanza da rilasciare al loro interno, permettendo una secrezione efficiente.

	\subsection{Contenuti tipici delle vescicole secretorie}
	Le vescicole secretorie contengono spesso precursori.
	Questi sono enzimi che vengono attivati quando si trovano in ambente favorevoli.
	Possono inoltre essere processati nelle vescicole e svolgere la loro attivit\`a quando rilasciati.

	\subsection{Esempi}
		
		\subsubsection{Trasporto di acetilcolina}
		L'acetilcolina \`e un neurotrasmettitore che viene secreto attraverso vescicole.
		Le vescicole che lo contengono si fondono con la membrana in seguito all'ingresso di \emph{$Ca^{2+}$} all'interno della terminazione.

			\paragraph{Regolazione}
			La fusione viene permessa da un complesso \emph{SNARE}.
			Si trovano sulla vescicola \emph{v-SNARE}, mentre sulla membrana \emph{t-SNARE}.
			Le \emph{t-SNARE} si complessano con sintaxina e \emph{SNAP25} formando un trimero.
			\emph{SNAP25} \`e formata da due domini: uno che facilita il ripiegamento e uno per legarsi alle altre.
			La sinaptogamina presenta $4$ siti di legame per il calcio e impedisce la fusione tra le \emph{t-v-SNARE} se non in presenza di calcio.
			Il calcio pertanto causa un cambio conformazionale nella complessina che lega sia le \emph{t} che le \emph{v-SNARE}.
			Il cambio conformazionale le fa interagire facilitando la fusione delle membrane e la liberazione del neurotrasmettitore.

	\subsection{Accrescimento della membrana plasmatica}
	Alcuni eventi di esocitosi accrescono la membrana plasmatica:
	\begin{multicols}{2}
		\begin{itemize}
			\item Citochinesi.
			\item Fagocitosi.
			\item Riparazione in seguito a danneggiamento.
			\item Cellularizzazione.
		\end{itemize}
	\end{multicols}

	\subsection{Proteine che caratterizzano le vescicole}
	Le vescicole sono caratterizzate da diverse proteine in base alla provenienza:
	\begin{multicols}{2}
		\begin{itemize}
			\item Clatrine: vescicole che si originano nel trans Golgi.
			\item \emph{COP1}: vescicole che si originano nel Golgi intermedio.
			\item \emph{COP2}: vescicole che si originano nel reticolo endoplasmatico.
		\end{itemize}
	\end{multicols}
	Queste proteine si possono usare come marcatori delle diverse zone.

	\subsection{Endocitosi}
	Vicino alla membrana destinata all'endocitosi si accumula clatrina.
	La regione di membrana viene detta fossa rivestite da clatrina e presenta un introflessione.

		\subsubsection{Caveole}
		Si intende per caveole cavit\`a rivestite da caveoline.
		Associate alla membrana formano dei domini all'interno della membrana contenenti colesterolo, glicosfingolipidi, proteine di membrana.
		Delimitano le vescicole endocitotiche.

		\subsubsection{Endocitosi mediata da recettori}
		Si intende per \emph{LDL} un monostrato di fosfolipidi intervallati da colesterolo all'interno dei quali si accumula il colesterolo.
		Attorno si trova una componente proteica con un dominio riconosciuto da un recettore.
		Quando la cellula necessita colesterolo avviene un endocitosi.
		\emph{AP2} \`e una proteina adattatrice coinvolta nel meccanismo di internalizzazione dei \emph{LDL}.

	\subsection{Formazione e fusione di una vescicola di trasporto}
	La formazione delle vescicole \`e regolata da due fattori:
	
		\subsubsection{\emph{GTPasi} monomeriche}
		Le \emph{GTPasi} monomeriche sono controllate da \emph{GEF} e \emph{GAP}.
		\begin{multicols}{2}
			\begin{itemize}
				\item \emph{ARG1-3} sono localizzate sull'apparato del Golgi e formano le vescicole a clatrina e \emph{COP1}.
				\item \emph{Sar1} localizza sul reticolo endoplasmatico e forma le vescicole a \emph{COP2}.
			\end{itemize}
		\end{multicols}

		\subsubsection{Famiglia delle proteine \emph{Rab}}
		Le proteine \emph{Rab} legano il \emph{GTP}.
		Sono precisamente localizzate:
		\begin{multicols}{3}
			\begin{itemize}
				\item \emph{Rab1} nel reticolo endoplasmatico e Golgi.
				\item \emph{Rab2} nel cis-Golgi.
				\item \emph{Rab3A} nelle vescicole sinaptiche.
			\end{itemize}
		\end{multicols}
		Lavorano con \emph{Rab} effettori come proteine motrici e di attracco.

			\paragraph{Cascata di \emph{Rab}}
			Una proteina \emph{Rab} pu\`o attaccarne altre ed effettori creando domini importanti per il riconoscimento specifico delle visciole.
			La loro attivazione \`e ad opera di \emph{Rab-GEF}.
			\emph{Rab5} localizza in membrana e una chinasi fosforila il lipide di membrana formando \emph{PI3P} che recluta \emph{Rab-GEF} formando una zattera sulla membrana.
			Passano a uno stato attivo quando legano il \emph{GTP}.
			Le proteine \emph{Ras} e \emph{RAN} regolano il processo.
			Questi fattori reclutano e regolano le proteine adattatrici che interagiscono con le proteine di rivestimento delle vescicole.

	\subsection{Proteine di rivestimento vescicolare}
	Le proteine di rivestimento molecolare identificate nell'apparato di Golgi e presenti sulle vescicole ne determinano la destinazione.
	\begin{multicols}{2}
		\begin{itemize}
			\item Clatrina: le vescicole rivestite da clatrina trasportano in entrambe le direzioni tra trans-Golgi e membrana plasmatica o lisosomi o endosomi.
			\item \emph{COP1} le vescicole rivestite da \emph{COP1} trasportano dal trans al cis-Golgi: \`e la proteina che riveste le vescicole di recupero con movimento retrogrado.
			\item \emph{COP2}: le vescicole rivestite da \emph{COP2} sono le vescicole che trasportano da reticolo endoplasmatico al cis-Golgi e dal cis-Golgi al trans-Golgi con movimento normale.
		\end{itemize}
	\end{multicols}

	\subsection{Formazione delle vescicole}
	
		\subsubsection{Vescicole circondate da \emph{COP2}}
		\begin{multicols}{2}
			\begin{enumerate}
				\item \emph{Sar1-GDP} presenta un'elica con alta affinit\`a per la membrana mascherata nello stato inattivo.
				\item \emph{Sar1-GEF} sulla membrana attiva \emph{Sar1} in \emph{Sar1-GTP}.
				\item \emph{Sar1-GTP} si lega alla membrana.
				\item Reclutamento di \emph{Sec23} che recluta \emph{Sec24}, una proteina adattatrice.
				\item \emph{Sec24} lega il recettore e recluta \emph{Sec13-31}.
				\item \emph{Sec13-31} formano il cesto di \emph{COP2}.
				\item \emph{Sar1-GAP} idrolizza il \emph{GTP} in \emph{Sar1-GTP} rendendola solubile e causando il disassemblamento della struttura.
				\item La presenza di protocollagene conferisce una forma alla vescicola che pu\`o non essere sferica.
			\end{enumerate}
		\end{multicols}

		\subsubsection{Vescicole circondate da clatrina}
		\begin{multicols}{2}
			\begin{enumerate}
				\item \emph{Ras} legano \emph{GTP} grazie a \emph{Ras-GEF}.
				\item \emph{ARF} lega \emph{GTP} e recluta proteine adattatrici.
				\item Il complesso \emph{ARF/GTP} recluta la clatrina.
				\item La clatrina presenta una struttura a triskelion con $3$ catene pesanti e $3$ leggere in modo che riesce ad interagire con altri triskelion.
					Una regione $N$-terminale interagisce con altre proteine adattatrici.
					\`E l'unit\`a funzionale minima per costruire la struttura della vescicola.
				\item Forma strutture esameriche o pentameriche determinando l'inizio della gemmazione.
			\end{enumerate}
		\end{multicols}

	\subsection{Proteine adattatrici}
	Le proteine adattatrici formano un ponte tra la clatrina e i recettori sulla membrana che riconoscono la proteina cargo.

		\subsubsection{\emph{AP2}}
		\emph{AP2} si divide in $\alpha$, $\beta2$, $miu2$, $\gamma2$.
		Riconosce i recettori e i fosfolipidi nella membrana.
		La presenza di entrambi i fattori impedisce la formazione di vescicole prive di cargo.
		Quando vengono legati avviene un cambio conformazionale che apre la proteina adattatrice e l'interazione con la membrana da cui si origina la vescicola.
		I fosfoinositoli vengono modificati tramite fosforilazione da parte di chinasi che permette il riconoscimento da parte di proteina adattatrici in base alla posizione e quantit\`a fosfato.
		In particolare si trova \emph{PI(4)P} nel trans-Golgi.

		\subsubsection{Proteine che deformano la membrana}
		Certe proteine presentano domini \emph{BAR} che interagiscono con la membrana ripiegandola facilitando l'interazione dei trishceli.
		I domini sono ad $\alpha$ elica con domini idrofobici

		\subsubsection{Dinamina}
		La dinamina ha la funzione di rompere il legame con la membrana permettendo la formazione della vescicola matura.
		Presenta domini ad $\alpha$ elica ed \`e una \emph{GTPasi}.

	\subsection{Fusione delle vescicole}
	Dopo il rilascio della vescicola si perde il rivestimento di clatrina.
	Avviene una modifica degli fosfoinositoli che permettono il distacco delle proteine adattatrici e avviene la fusione della vescicola con l'organello target.

		\subsubsection{Fattori coinvolti}
		Durante il trasporto il \emph{GTP} legato ad \emph{ARF1} viene idrolizzato e il complesso \emph{ARF1-GDP} si distacca dalla membrana e rientra nel ciclo di formazione di altre vescicole.
		Il processo di fusione delle vescicole comprende:
		\begin{multicols}{2}
			\begin{itemize}
				\item \emph{SNARE}: proteine trans-membrana presenti sulle vescicole con un corrispettivo sulla membrana plasmatica (\emph{v-SNARE} sulla vescicola, monomeri e \emph{t-SNARE} sulla membrana, trimeri).
				\item Proteine della famiglia \emph{Rab} con funzione endocitotica.
				\item \emph{GTPasi}.
				\item \emph{GEF}: fattori che aumentano l'interazione della proteina con \emph{GTP}.
			\end{itemize}
		\end{multicols}

		\subsubsection{Processo}

			\paragraph{Riconoscimento della membrana bersaglio}
			Per permettere il riconoscimento della membrana bersaglio \emph{t-SNARE} interagiscono con \emph{V-SNARE} formando un tetramero con liberazione di acqua.
			Questo facilita la fusione delle membrane e il rilascio del contenuto della vescicola.
			Dopo la fusione le \emph{SNARE} vengono rimosse attraverso proteine accessorie come \emph{NFS} che rimuove il tetramero ricostruendo le \emph{v-SNARE} e le \emph{t-SNARE} grazie a \emph{Rab}.
			L'interazione tra le due \emph{SNARE} fornisce l'energia necessaria per l'avvicinamento dei due strati lipidici.
			Le \emph{Rab} regolano l'interazione tra le \emph{SNARE}.
			Raggiungono la membrana e interagiscono con le molecole presenti.
			\emph{GEF} attivano la \emph{RAB} in modo che possa rimanere associata alla membrana.

			\paragraph{Fusione della vescicola con la membrana berasglio}

			L'associazione di \emph{Rab} alla membrana e l'interazione tra \emph{SNARE} facilitano l'avvicinamento delle membrane.
			Questo processo stimola l'idrolisi di \emph{GTP}.
			L'idrolisi aumenta l'interazione portando a un contatto quasi diretto.
			Questo causa instabilit\`a nei due doppi strati lipidici causando una fusione.
			I fattori coinvolti si staccano e tornano disponibili per un altro ciclo.
			Si nota come il distacco degli \emph{SNARE} sia \emph{ATP} dipendente.

\section{Apparato del Golgi}
Il Golgi \`e una fabbrica in cui le proteine provenienti dal reticolo endoplasmatico sono sottoposte a modifica per essere poi direzionate alla loro destinazione definitiva.
Vengono inoltre sintetizzati sfingomieline e glicolipidi.

	\subsection{Struttura}
	Il Golgi \`e una struttura composta da tante cisterne e vescicole appiattite.
	\`E una struttura polarizzata in cui si distinguono due facce.

		\subsubsection{Facce del Golgi}
		\begin{multicols}{2}
			\begin{itemize}
				\item Faccia cis: \`e il versante di entrata rivolto verso il nucleo.
					Riceve le vescicole provenienti dal reticolo endoplasmatico di transizione.
				\item Faccia trans: \`e il versante di uscita rivolto verso la membrana plasmatica.
					Da qui escono le vescicole secrete e vengono mandate alla loro destinazione definitiva.
			\end{itemize}
		\end{multicols}

		\subsubsection{Compartimenti}
		Il Golgi pu\`o essere diviso in $3$ compartimenti:
		\begin{multicols}{2}
			\begin{itemize}
				\item Reticolo cis o cis-Golgi: comunica con il reticolo endoplasmatico.
				\item Pile del Golgi: il compartimento intermedio, caratterizzato da una porzione mediana e una trans.
				\item Reticolo trans o trans-Golgi: comunica con membrana, endosomi e lisosomi.
			\end{itemize}
		\end{multicols}

	\subsection{Flusso di vescicole}
	Le vescicole viaggiano principalmente dal nucleo verso il reticolo endoplasmatico, da cui arrivano al reticolo endoplasmatico intermedio, passano al cis-Golgi, alle pile del Golgi e al trans-Golgi.

		\subsubsection{Bidirezionalit\`a del flusso}
		Il flusso di vescicole pu\`o essere bidirezionale: il flusso all'indietro avviene con la presenza nella vescicola di proteine che devono essere localizzate nel punto precedente nel flusso.
		Le proteine che devono rimanere nel reticolo endoplasmatico sono identificate da una sequenza segnale caratterizzata da $2$ lisine.
		Le proteine solubili hanno una sequenza segnale \emph{KDEL} (lisina, aspartato, glutammato e leucina).
		Se si trovano nel Golgi questo le riconosce attraverso recettori e le lega grazie al $pH$ pi\`u basso.
		Questo causa la formazione di una vescicola che torna nel reticolo endoplasmatico, dove il $pH$ pi\`u alto favorisce la dissociazione tra recettore e proteina.

	\subsection{Funzionalit\`a del Golgi}
	All'interno dell'apparato del Golgi avvengono:
	\begin{multicols}{2}
		\begin{itemize}
			\item Glicosilazione: elaborazione e sintesi degli zuccheri che caratterizzano le glicoproteine.
			\item Modifica degli $N$-oligosaccaridi: che inizia nel reticolo endoplasmatico e subisce un ulteriore arricchimento.
			\item O-glicosilazione a carico della treonina.
		\end{itemize}
	\end{multicols}
	Gli enzimi nel Golgi non sono liberi ma si trovano associati alla membrana.

		\subsubsection{modifiche alle proteine}
		Dopo le modifiche nel reticolo endoplasmatico le proteine sono arricchite con ulteriori zuccheri e altre componenti.
		Ogni glicoproteina viene modificata in modo specifico in modo che assuma una sequenza oligosaccaridica specifica.
		Gli zuccheri vengono aggiunti a:
		\begin{multicols}{2}
			\begin{itemize}
				\item Proteine di membrana: gli zuccheri sono importanti per il riconoscimento e la produzione di anticorpi.
				\item Proteine secrete: importante nelle piante in quanto gli zuccheri sono alla base di processi vitali come la sintesi della cellulosa e della parete cellulare.
			\end{itemize}
		\end{multicols}

			\paragraph{Enzimi coinvolti}
			\begin{multicols}{2}
				\begin{itemize}
					\item Glicosiltransferasi: aggiunge i residui degli zuccheri.
					\item Glicosidasi: rimuove i residui di zuccheri.
				\end{itemize}
			\end{multicols}

			\paragraph{Direzionamento nei lisosomi}
			Alle proteine che sono direzionate nei lisosomi viene aggiunto un mannosio-6-fosfato.

	\subsection{Compartimentazione degli enzimi}
	Molti degli enzimi coinvolti nelle modifiche post traduzionali delle proteine si trovano associati alla membrana del Golgi.
	Qui sono divisi in diversi compartimenti.
	Questa caratteristica determina una regolarit\`a del Golgi.

	\subsection{Metagolismo dei lipidi e dei polisaccaridi}
	Il Golgi \`e importante per la sintesi dei lipidi come glicolipidi e sfingomielina che vengono prodotti a partire dalla ceramide.
	Se a questa viene aggiunta un gruppo di fosfatidilcolina diventa sfingomielina, se le vengono aggiunti residui di zuccheri diventa un glicolipide.
	Questi si trovano nel foglietto luminale del doppio strato fosfolipidico in quanto non possono traslocare da un lato all'altro delle membrana.
	Raggiungono la membrana plasmatica attraverso il trasporto vescicolare e vi si dispongono.

	\subsection{Mantenimento della polarit\`a cellulare}
	Le cellule dell'epitelio presentano una porzione apicale morfologicamente diversa, con funzioni e composizione diverse rispetto alla porzione basolaterale.
	Questo \`e permesso da un trasporto specifico di proteine verso le due porzioni.

		\subsubsection{Sorting diretto dal trans-Golgi network}
		I domini del trans-Golgi producono vescicole specifiche per le porzioni della membrana.

		\subsubsection{Sorting indiretto}
		Il trans-Golgi produce proteine indirizzate in entrambe le porzioni che si fondono con una membrana.
		Avviene un processo di riciclo in cui le proteine destinate all'altra porzione vengono reintrodotte mediante endocitosi negli endosomi precoci e indirizzate nella porzione giusta.


	\subsection{Tipologie di secrezione}
	
		\subsubsection{Secrezione non regolata}
		La secrezione non regolata avviene per le proteine costitutive sempre presenti sulla membrana plasmatica.

		\subsubsection{Secrezione regolata}
		La secrezione regolata avviene per le proteine che devono essere presenti sulla membrana o secrete solo in seguito a specifici eventi di segnalazione.
		Esempi sono ormoni e neurotrasmettitori.

	\subsection{Maturazione e mantenimento}

		\subsubsection{Modello di maturazione delle cisterne}
		In questo modello di maturazione dell'apparato del Golgi si considera come dal reticolo endoplasmatico si creano vescicole che si fondono tra di loro formando un cluster tubulare.
		Una successiva maturazione forma dal cluster il cis-Golgi network.
		Altri passi di maturazione lo trasformano nella zona intermedia e nel trans-Golgi network, con un continuo riciclo dalle diverse parti.

		\subsubsection{Modello di vescicola}
		Secondo il modello di vescicola non avviene maturazione delle componenti del Golgi ma il trafficking di vescicole garantisce funzionalit\`a e specificit\`a dei vari compartimenti del Golgi.

\section{Lisosomi}
I lisosomi sono strutture vescicolari che contengono un insieme di enzimi idrolitici che lavorano a $pH$ acidi in grado di degradare le molecole biologiche.

	\subsection{Mantenimento dell'ambiente}
	L'ambiente acido viene mantenuto nei lisosomi attraverso pompe \emph{$H^+$}.
	Queste utilizzano \emph{ATP} per mantenere un gradiente elettrochimico attraverso la membrana dei lisosomi.

	\subsection{Morfologia}
	Il numero e la grandezza dei lisosomi varia in base al tipo cellulare e allo stato fisiologico della cellula.

	\subsection{Strutture}
	I lisosomi sono strutture eterogenee che possono fondersi con un endosoma formando un endolisosoma.

	\subsection{Funzione}
	I lisosomi sono gli spazzini delle cellule e svolgono la digestione per endocitosi.

	\subsection{Formazione}
	I lisosomi si formano quando vescicole gemmate dal trans-Golgi si fondono con ensosomi che congengono molecole trasportate all'interno della membrana plasmatica per endocitosi.

		\subsubsection{Trasporto delle proteine}
		Le proteine prodotte nel reticolo endoplasmatico e dall'apparato del Golgi vanno nei lisosomi grazie al segnale del mannosio-6-fosfato.
		Si formano vescicole specifiche.
		Recettori riconoscono la modifica e la portano nei lisosomi.
		

			\paragraph{Proteine secrete}
			Pu\`o capitare che proteine che devono andare nei lisosomi vengano incluse in vescicole secretorie.
			In questo caso quando raggiungono l'ambiente extracellulare sono riconosciute da recettori che le riportano nella cellula e nei lisosomi attraverso endocitosi.

	\subsection{Attivazione degli enzimi lisosomiali}
	La maggior parte degli enzimi lisosomiali sono idrolasi acide e pertanto possono essere attivate unicamente a valori di $pH$ tra il $5$ e il $5.5$.
	Non possono pertanto funzionare nelle fasi precedenti a causa del $pH$ neutro.
	Questo \`e un meccanismo di controllo contro una digestione incontrollata delle molecole nel citosol.
	In caso di rottura gli enzimi rimangono inattivi.

	\subsection{Digestione del materiale}
	I lisosomi oltre alla digestione del materiale proveniente dall'endocitosi sono responsabili della degradazione del materiale proveniente da varie fonti.

		\subsubsection{Fagocitosi}
		La fagocitosi consiste nell'internalizzazione dei batteri nelle cellule e la loro successiva degradazione ad opera del lisosoma.
		\`E un meccanismo ristretto tipico dei macrofagi.

		\subsubsection{Autofagia}
		L'autofagia \`e un meccanismo che permette alle cellule di fare turn-over delle proprie componenti.
		Questi autofagosomi vengono creati in condizioni di stress o carenza.
		La proteina \emph{PINK1} per esempio viene riconosciuta da \emph{PARKIN} quando rimane nel citoplasma e induce l'autofagia.
		Nelle cellule mammifere \emph{ATG9} \`e responsabile della formazione dell'autofagosoma.

		\subsubsection{Pinocitosi}
		La pinocitosi \`e l'introduzione di liquidi o particelle ancorate alla membrana.

		\subsubsection{Macropinocitosi}
		La macropinocitosi \`e un riarrangiamento del citoscheletro attivato da un ligando come fattore di crescita, virus o ligandi per integrine che si lega ad un recettore.
		Si formano protrusioni di membrana che si richiudono su loro stesse.

\section{Endosomi}
Gli endosomi sono organelli della cellula che si formano in seguito a endocitosi.
L'endocitosi \`e il processo che porta materiali presenti nell'ambiente extracellulare verso l'interno con la formazione di vescicole endocitiche.
Pu\`o essere mediata da recettori, ma non \`e necessario.
Negli endosomi vengono trasportate alcune proteine prodotte dall'apparato del Golgi.

	\subsection{Tipologie}
	Gli endosomi sono pertanto vescicole che si formano sulla membrana e vengono utilizzate dalla cellula per introdurre sostanze grazie ad endocitosi.
	I recettori sono responsabili di riconoscere e legare le sostanze da introdurre.

		\subsubsection{Endosomi precoci}
		Gli endosomi precoci sono localizzati vicino alla membrana plasmatica.
		Ricevono vescicole endocitotiche e si formano dalla loro fusione.
		Separano le molecole destinate al riciclo verso la membrana da quelle destinate alla degradazione.
		
			\paragraph{Vescicole endocitotiche}
			Le vescicole endocitotiche pertanto convergono nell'enzima precoce.
			In alcuni casi i recettori possono essere riportati in membrana attraverso il ligando come per la ferritina.
			In altri sia il recettore che il ligando devono essere degradati.
			Il recettore viene ubiquitinato e indirizzato verso il lisosoma.
			\emph{ESCRT} media il riconoscimento del recettore ubiquitinato e dei fosfolipidi di membrana in tre passaggi: \emph{ESCRT-1-2-3}.
			Si forma una vescicola all'interno dell'endosoma che viene degradata nel lisosomi.
			Un esempio di questi recettori sono le \emph{EGF} (epidermal growth factor).
		

		\subsubsection{Endosomi di riciclo}
		Gli endosomi di riciclo si formano dopo il rilascio del ligando dal recettore.
		Contiene pertanto i recettori vuoti e li riporta in membrana permettendo cos\`i il riciclo del recettore.

		\subsubsection{Endosomi tardivi}
		Gli endosomi tardivi si formano dai precoci e sono deputati alla degradazione.
		Ricevono enzimi lisosomiali dal trans-Golgi e si fondono con i lisosomi.
		Dopo la fusione il $pH$ si abbassa attivando gli enzimi lisosomiali che determinano la degradazione dei ligandi.
	



