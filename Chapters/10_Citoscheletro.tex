\chapter{Citoscheletro}

\section{Panoramica}
Il citoscheletro \`e una rete di proteine filamentose che si estendono attraverso il citoplasma delle cellule eucariote.

	\subsection{Componenti principali}
	Il citoscheletro \`e composto da tre tipi principali di proteine:
	\begin{multicols}{3}
		\begin{itemize}
			\item Actina.
			\item Tubulina.
			\item Filamenti intermedi.
		\end{itemize}
	\end{multicols}
	I filamenti intermedi si pongono a met\`a tra i filamenti costituiti da actina e i microtubuli costituiti da tubulina.

	\subsection{Funzioni principali}
	Il citoscheletro determina lo scheletro della cellula.
	\begin{multicols}{2}
		\begin{itemize}
			\item Mantene la forma e la struttura della cellula.
			\item Determina la posizione degli organelli e l'organizzazione generale del citoplasma.
			\item Impedisce il collasso del nucleo.
			\item Conferisce rigidit\`a e resistenza a stress.
			\item Conferisce capacit\`a di movimento: l'actina gli permette di interagire con il substrato.
			\item Mantiene in posizione gli organelli.
		\end{itemize}
	\end{multicols}

	\subsection{Fattori coinvolti nell'assemblaggio del citoscheletro}
	Nell'assemblaggio del citoscheletro sono coinvolti diversi fattori:
	\begin{multicols}{2}
		\begin{itemize}
			\item Fattori molecolari: proteine che inibiscono o aiutano la polimerizzazione.
			\item Fattori esterni come la temperatura: ad esempio basse temperature causano la depolimerizzazione dei microtubuli.
		\end{itemize}
	\end{multicols}

	\subsection{Miosina}
	La miosina permette la contrazione muscolare e il trasporto vescicolare.
	Il citoscheletro \`e in grado di fungere da substrato per il trasporto cellulare.
	Questi trasporti avvengono grazie a proteine motrici e miosine.
	
		\subsubsection{Struttura}
		La miosina \`e una delle molecole pi\`u grandi del nostro organismo a $500kDa$.
		\`E un eterodimeri:
		\begin{multicols}{2}
			\begin{itemize}
				\item Due catene passanti che terminano con una testa globulare e una struttura ad $\alpha$-elica con cui interagiscono formando il dimero.
				\item Quattro catene leggere sede di regolazione.
			\end{itemize}
		\end{multicols}

		\subsubsection{Tipologie}
		Diversi tipi di miosine hanno ruoli diversi e interagiscono in modo diverso con il citoscheletro e i filamenti di actina.

\section{Actina}

	\subsection{Panoramica}
	L'actina \`e la proteina citoscheletrica pi\`u diffusa nelle nostre cellule.
	Viene organizzata in modo diverso e polimerizza a formare de filamenti sottili del diametro di $7nm$ detti microfilamenti.
	In alcune cellule, come nel muscolo i filamenti possono essere disposti in modo ordinato, mentre in altri casi formano una maglia tridimensionale con diversi gradi di organizzazione.
	Si trova al disotto della membrana plasmatica dove forma una rete che fornisce supporto meccanico.
	Si pensa abbia un ruolo nel trasporto delle proteine nel nucleo, mentre nel fibroblasto permette un movimento attivo.

	\subsection{Actina monomerica}
	I filamenti di actina sono caratterizzati da strutture globulari di $375aa$ e dal peso di $40kDa$.
	Questi sono detti monomeri di actina o actina globulare $G$.

		\subsubsection{Tipologie di actina globulare}
		\begin{multicols}{3}
			\begin{itemize}
				\item $\alpha$ nei muscoli.
				\item $\beta$.
				\item $\gamma$.
			\end{itemize}
		\end{multicols}

		\subsubsection{Polarit\`a}
		I monomeri di actina possiedono due estremit\`a polari:
		\begin{multicols}{2}
			\begin{itemize}
				\item Estremit\`a appuntita: polarit\`a positiva.
				\item Estremit\`a a barbigli: polarit\`a negativa, presenta il sito di legame per \emph{ATP}.
			\end{itemize}
		\end{multicols}
		La polarit\`a \`e importante in quanto alle diverse estremit\`a si possono legare diversi fattori coinvolti nella polimerizzazione o depolimerizzazione del filamento o al suo ancoraggio nella membrana.
		La polarit\`a definisce pertanto la direzione in cui si muovono le proteine motrici.

		\subsubsection{Legame con \emph{ATP}}
		Un monomero di actina lega \emph{ATP} o \emph{ADP}, rispettivamente \emph{T-actina} e \emph{D-actina}.
		I monomeri liberi in grado di essere reclutati all'interno del filamento legano \emph{ATP} e una volta che sono stati incorporati lo idrolizzano in \emph{ADP}.
		I monomeri alle estremit\`a che legano \emph{ADP} vengono rimssi dal filamento.

	\subsection{Actina filamentosa}
	L'actina filamentosa \`e il polimero formato da actina globulare.
	I monomeri interagiscono polarmente: un'estremit\`a appuntita si lega a una a barbigli dando una direzione al filamento.
	Il filamento non \`e completamente lineare ma si presenta leggermente arrotolato.


		\subsubsection{Centro di nucleazione}
		Si intende per centro di nucleazione o trimero la struttura minima che d\`a inizio al processo di polimerizzazione.
		Pu\`o formarsi in maniera spontanea o essere stimolato da fattori.
		Il processo \`e favorito dalla presenza di \emph{ATP}.
		Dal momento in cui si forma il nucleo la velocit\`a di polimerizzazione aumenta notevolmente fino ad un equilibrio.
		Questo \`e determinato dalla disponibilit\`a delle subunit\`a.

		\paragraph{Fattori che favoriscono la nucleazione}

			\subparagraph{Formine}
			Le formine sono proteine dotate di terminazioni a barbigli che permettono la nucleazione dei primi monomeri.
			Forma un dimero e recluta monomeri di actina favorendo la formazione di un trimero.
			Il processo \`e velocizzato.
			Sono coinvolte nelle formazione di filamenti non ramificati, fibre di stress e fibre muscolari.

			\subparagraph{\emph{ARP2/3}}
			\emph{ARP2/3} (actine related protein) favorisce la formazione di filamenti ramificati.
			Interagisce con l'estremit\`a a barbigli di un filamento lineare e promuove la formazione di un centro di nucleazione che si allunga andando a creare una ramificiazione.
			Si trova al di sotto della membrana plasmatica.
			\`E importante nei meccanismi di mobilit\`a della cellula.
		
		\subsubsection{Polimerizzazione}
		\begin{multicols}{2}
			\begin{enumerate}
				\item Il monomero di actina $G$ legato ad \emph{ATP} viene inserito all'interno della catena.
					L'inserimento pu\`o avvenire a qualsiasi delle due estremit\`a, ma la crescita in direzione positiva \`e pi\`u veloce.
				\item Una volta inserito viene idrolizzato \emph{ATP} e il monomero rimane legato a \emph{ADP}.
				\item Il legame con \emph{ADP} favorisce il distacco del monomero dal filamento.
			\end{enumerate}
		\end{multicols}
		Si nota come a un'estremit\`a un monomero idrolizza \emph{ATP}, mentre all'altra il monomero idrolizzato viene rilasciato.
		L'assemblaggio \`e reversibile, pertanto in caso di necessit\`a i filamenti possono depolimerizzare per dissociazione dei monomeri.
		
		\subsubsection{Equilibrio}
		Si nota come alla base della formazione dei filamenti di actina si trova una continua aggiunta e rimozione di monomeri fino ad arrivare ad una condizione di equilibrio in cui i monomeri aggiunti sono uguali ai monomeri rimossi.
		L'equilibrio \`e determinato dalla concentrazione dei monomeri.
			
			\paragraph{Concentrazione critica dei monomeri}
			Si intende per concentrazione critica dei monomeri la concentrazione per cui la velocit\`a di assemblaggio equivale quella di sisassemblaggio.

			\paragraph{Fattori coinvolti}
			L'equilibrio pu\`o essere modificato da fattori che depolarizzano la membrana aumentando la costante di dissociazione del filamento aumentando il pool di monomeri liberi.
			I fattori che faclitano la polimerizzazione spostano l'equilibrio velocizzando il processo di polimerizzazione favorendo la formazione dei filamenti di actina.

		\subsubsection{Velocit\`a di allungamento}
		I monomeri possono entrare da entrambe le direzioni, ma la velocit\`a di aggiunta dei monomeri \`e diversa in base alla polarit\`a: \`e infatti maggiore nella parte dei barbigli.
		Questo \`e legato al fatto che il complesso \emph{ATP}-actina si dissocia meno velocemente del complesso \emph{ADP}-actina.
		La concentrazione critica \`e pertanto diversa alle estremit\`a, dando origine al fenomeno di treadmilling o mulinello.

			\paragraph{Fenomeno mulinello}
			Il fenomeno mulinello riguarda il comportamento dinamico dell'actina.
			\`E importante per il  turnover e per la polimerizzazione di filamenti.
			I monomeri entrano a un'estremit\`a e si spostano lungo il filamento fino ad arrivare fino all'estremit\`a opposta dove vengono rilasciati.
			La perdita dei monomeri all'estremit\`a viene controbilanciata dall'aggiunta dei monomeri all'estremit\`a opposta.

				\subparagraph{Sostanze che bloccano il fenomeno}
				Sostanze che bloccano il fenomeno mulinello possono essere usate come antitumorali.
				\begin{multicols}{2}
					\begin{itemize}
						\item Latruncolina: depolimerizza i filamenti di actina destabilizzando le subunit\`a.
						\item Citocalasine: si legano nell'estremit\`a $+$ dei filamenti e ne impediscono l'allungamento.
						\item Fallodidina: impedisce la depolimerizzazione dei filamenti e li stabilizza.
					\end{itemize}
				\end{multicols}

		\subsubsection{Regolazione della polimerizzazione}
		Diverse proteine che legano l'actina vanno a controllare l'assemblaggio e disassemblaggio dei filamenti.
		\begin{multicols}{2}
			\begin{itemize}
				\item Timosina: lega il monomero rendendolo indisponibile.
				\item Profilina: lega il monomero promuovendo la sua polimerizzazione.
					Viene inibita attraverso fosforilazione.
				\item \emph{ARP2/3}: favorisce la formazione dei nuclei.
					\emph{Arp2/3} presentano una struttura simile a quella dell'actina.
					Si legano all'estremit\`a $+$ dopo che sono state attivate permettendo un rapido allungamento all'estremit\`a $+$.
					Permettono la formazione di ramificazioni partendo da un filamento preesistente.
				\item Formine: promuovono la formazione dei centri di nucleazione ma non \`e in grado di formare ramificazioni.
					\`E un dimero che si lega ai monomeri.
					Possono presentare domini capaci di reclutare altri fattori che promuovono il reclutamento di monomeri di actina all'interno del filamento nascente come profilina.
				\item Tropomiosina: una proteina allungata che si posiziona sopra i siti di attacco dell'actina con la miosina.
					Blocca l'interazione delle teste di miosina con i siti dell'actina.
					Rende il filamento rigido.
				\item \emph{CapZ}: si localizza nella banda $Z$ del muscolo e stabilizza un'estremit\`a del filamento di actina impedendo la crescita in quella direzione.
				\item Tropomodilina: riveste i filamenti di actina e li lega tra di loro.
				\item Gelsolina: viene attivata dal rilascio di \emph{$Ca^{2+}$} e destabilizza il filamento promuovendo la depolimerizzazione.
				\item Cofilina: \`e un fattore di depolimerizzazione che forza il filamento a compattarsi rendendolo instabile.
					Facilita il distacco dei monomeri \emph{D-actina}.
					Utilizzata per il rinnovo dei filamenti.
			\end{itemize}
		\end{multicols}

			\paragraph{Proteine che favoriscono la depolimerizzazione}
			Le proteine che favoriscono la deplimerizzazione dei filamenti di actina sono:
			\begin{multicols}{2}
				\begin{itemize}
					\item \emph{ADF}: active depolarization factor.
					\item \emph{ADF/cofilina}: rimuove i monomeri dai filamenti di actina.
						Si lega a filamenti impedendo ai monomeri di legarsi ad \emph{ATP}.
						I monomeri non sono pertanto pi\`u disponibili per creare un nuovo filamento.
					\item Profilina: la profilina contrasta l'effetto della cofilina rendendo i monomeri disponibili per la sintesi dei filamenti.
						Stimola lo scambio tra \emph{ADP} e \emph{ATP} favorendo la formazione di monomeri legaati ad \emph{ATP} che si dissociano dalla cofilina e si attaccano ad altri filamenti.
				\end{itemize}
			\end{multicols}
			L'attivit\`a di queste proteine \`e controllata da meccanismi segnale che consentono di regolare la polimerizzazione dell'actina in risposta a stimoli ambientali.
		
	\subsection{Complessi di ordine superiore}
	I filamenti di actina possono comporsi in modo da formare strutture di ordine superiore come:
	\begin{multicols}{2}
		\begin{itemize}
			\item Reti dendritiche grazie a \emph{ARP2/3} come i lamellipodi.
			\item Fasci grazie a formine come i filopodi e le stress fiber.
			\item Strutture a rete come per il cell cortex.
		\end{itemize}
	\end{multicols}

		\subsubsection{Filopodi}
		I filopodi sono strutture parallele.
		Si trovano in estroflessioni della membrana con recettori sensori dell'ambiente esterno e danno direzione al movimento.

		\subsubsection{Lamellipodi}
		I lamellipodi sono strutture a rete con il compito di espandere la membrana.

		\subsubsection{Stress fibers}
		Le stress fibers sono importanti per l'ancoraggio e l'interazione della cellula con il substrato.
		Presentano una conformazione parallela con orientamento opposto dei filamenti di actina.

		\subsubsection{Fattori coinvolti}
		In queste strutture si trovano fattori che influenzano la stabilit\`a e la dinamicit\`a di queste strutture.
		\begin{multicols}{2}
			\begin{itemize}
				\item Fimbrina: un monomero di fimbrina si lega a $2$ monomeri di actina.
					Forma dei fasci legati tra di loro in maniera stretta.
				\item $\alpha$-actinina: il dimero lega due monomeri di actina.
					Forma dei fasci di actina pi\`u distanti tra di loro come nel muscolo.
				\item Filamena: il dimero lega due subunit\`a di actina.
					L'elasticit\`a e la struttura a $V$ rende stabili croci dei filamenti caratteristiche di alcune zone della cellula.
				\item Spectina: importante nel citoscheletro dei globuli rossi conferisce resistenza e plasticit\`a ai filamenti.
					\`E molto lunga e costituita da $2$ subunit\`a $\alpha$ e $2$ $\beta$.
					Solo le prime legan l'actina e creano un ponte tra citoscheletro e proteine che interagiscono con esso o proteine ancorate alla membrana.
				\item Distrofina: lega l'actina ad altre proteine che possono interagire con i filamenti in maniera diversa: a livello delle terminazioni, facendo interagire filamenti diversi o controllare l'assemblaggio dei monomeri regolando l'irdolisi dell'\emph{APT}.
			\end{itemize}
		\end{multicols}

	\subsection{Organizzazione dei filamenti di actina}
	I filamenti possono essere organizzati secondo due modalit\`a.
	Questa \`e determinata da quali proteine legano l'actina.
	\begin{multicols}{2}
		\begin{itemize}
			\item Fasci: sono coinvolte proteine rigide e globulari che legano i filamenti ortogonali tra di loro.
				Tengono uniti i fasci in maniera stretta e ordinata.
			\item Reti: sono coinvolte proteine filamentosi e di grandi dimensioni che lavorano in dimeri forando strutture \emph{Y/D}.
				Legano i filamenti perpendicolarmente creando una struttura pi\`u lassa.
			\item Struttura intermedia: i muscoli presentano fasci non troppo stretti per permettere le interazioni con l'actina.
				Si trovano proteine che separano in maniera maggiore i filamenti.
		\end{itemize}
	\end{multicols}

		\subsubsection{Punti in comune}
		Le proteine coinvolte nell'organizzazione dei filamenti presentano:
		\begin{multicols}{2}
			\begin{itemize}
				\item Dominio che lega actina.
				\item Dominio che lega il calcio, messaggero secondario alla base delle contrazioni dei filamenti.
					\`E inoltre in grado di attivare chinasi che fosforavano e determinano una diversa interazione con l'actina.
			\end{itemize}
		\end{multicols}
		Questi domini sono separati da sequenze spaziatrici che variano in base alla lunghezza e alla flessibilit\`a e sono responsabili delle diverse propriet\`a di legame delle proteine.

		\subsubsection{Organizzazione dei fasci}
		
			\paragraph{Fasci molto compatti}
			I fasci molto compatti si trovano soprattutto nei microvilli, estroflessioni di membrana dell'epitelio intestinale.
			La loro forma viene mantenuta grazie alla presenza di fasci di actina permettendo cos\`i di aumentare la superficie di assorbimento dei nutrienti.
			Tutti i filamenti hanno la stessa polarit\`a, con la terminazione a barbigli in corrispondenza della membrana.
			I fasci sono a una distanza compresa tra i $14$ e i $15nm$.
			La distanza dipende dalla fimbrina.

				\subparagraph{Fimbrina}
				La fimbrina possiede due domini di legame per l'actina e due per il legame del calcio.
				Lavora monomericamene legando i filamenti e mantenendoli paralleli e vicini.

			\paragraph{Fasci meno compatti}
			I fasci meno compatti sono tipici delle cellule muscolari in quanto permettono la contrazione.
			La distanza \`e tipicamente di $40nm$ e permette alla miosina di interagire con l'actina.
			La distanza dipende dalla $\alpha$-actidina.

				\subparagraph{$\mathbf{\alpha}$-actidina}
				La $\alpha$-actidina contiene due domini, uno per il legame con l'actina e uno che lega il calcio.
				La proteina dimerizza legando due filamenti.

		\subsubsection{Organizzazione delle reti}
		La proteina coinvolta nelle formazioni della struttura a rete \`e la filamina.
		
			\paragraph{Filamina}
			La filamina \`e una proteina di grandi dimensioni che tiene insieme i filamenti nella struttura a rete.
			Possiede $1$ dominio che lega l'actina e uno che promuove la formazione del dimero e lo stabilizza.
			Tra i due domini si trova una struttura a $\beta$-foglietto che garantisce stabilit\`a e flessibilit\`a.
			Lavora come dimero.

			\paragraph{Localizzazione}
			La rete di actina si trova al di sotto della membrana plasmatica.
			La rete non \`e completamente dissociata ma interagisce con proteine presenti sulla membrana associata alla membrana formando il cortex cellulare.
			Determina la forma della cellula ed \`e coinvolta in varie funzioni come il movimento.

			\paragraph{Cortex cellulare}
			I globuli rossi sono utilizzati come modello per lo studio della struttura del cortex cellulare.
			Non presentano nucleo e organelli e la membrana \`e omogenea dal punto di vista della componente citoscheletrica.
			Il citoscheletro corticale di membrana \`e il principale responsabile della forma biconcava.
			La componente principale della corteccia \`e la spectrina.

				\subparagraph{Spectrina}
				La spectrina \`e una camponina, molto grande e lega actina.
				\`E un tentramero di due catene $\alpha$ e due $\beta$.
				La catena $\alpha$ contiene un dominio di interazione con il calcio.
				La catena $\beta$ contiene un dominio di interazione con l'actina.
				Le due catene formano un dimero e due si uniscono formando un tetramero.
				Le terminazioni dei tetrameri di spectrina si associano ai filamenti di actina.

				\subparagraph{Anchirina}
				L'anchirina \`e uno dei fattori che mediano il legame tra spectrina-actina e membrana plasmatica.
				Nella banda $3$ si trova una proteina trasnmembrana: la glicoforina.

				\subparagraph{Distrofina}
				La distrofina \`e una proteina filamentosa coinvolta nell'interazione tra citoscheletro e la membrana.
				Presenta un dominio di legame con l'actina e uno con la membrana plasamatica.
				Lega il citoscheletro con la matrice extracellulare.
				Mantiene la stabilit\`a cellulare durante la contrazione muscolare.

	\subsection{Strutture cellulari}

		\subsubsection{Fibroblasti}
		I fibroblasti secernono proteine della matrice extracellulare.
		Questa matrice \`e costituita da componenti prodotti dalla cellula.
		I fibroblasti adesicono alla superficie di membrana mediante proteine trans-membrana, le integrine.
		L'attivit\`a delle integrine \`e regolata dal calcio che permette la loro dimerizzazione e interazione con l'esterno.
		All'interno interagiscono con le proteine che legano tra di loro i filamenti di actina che formano le stress fiber.

			\paragraph{Adesioni focali}
			Si dicono adesioni focali aree specializzate di membrana che consentono alla cellula di aderire al substrato.

			\paragraph{Stress fibers}
			Si dicono stress fibers fasci contrattili di actina che ancorano le cellule ed esercitanno tensione in corrispondenza del substrato.
			Sono uniti alla membrana mediante integrine in punti focali di adesione.
			Sono sottoposte a tensione tra membrana e substrato.
			Sono organizzate in modo particolare per il movimento.
			La talina e la vinculina ancorano l'integrina al citoscheletro.
			La talina media l'interazione tra vinculina e integrina che pu\`o poi legarsi all'actina.
			Queste interazioni permettono il legame dei filamenti di actina alla membrana plasmatica.

			\paragraph{Giunzioni aderenti}
			Le giunzioni aderenti sono aree di contatto intercellulari a cui si lega il citoscheletro di actina.
			Sono presenti tra cellule epiteliali, dove assumono una struttura ad anello.
			Questo permette il legame tra le cellule e le permette di contrarsi determinando la variazione nella forma della cellula e dell'epitelio.
			Questa adesione \`e mediata da proteine specifiche.
			Si identificano giunzioni tra cellule non forti che permettono l'adesione o occludenti che saldano le cellule in maniera forte impedendo alle sostanze di passare tra una cellula all'altra.

				\subparagraph{Caderine}
				Le caderine sono molecole di membrana con grossa componente extracellulare.
				Si trovano nello spazio tra una cellula all'altra e ancorate allo scheletro tramite catenine.
				Le catenine sno proetine citoplasmatiche che ancorano i filamenti di actina alla membrana plasmatica, formate da una subunit\`a $\alpha$ e una $\beta$.

		\subsubsection{Microvilli}
		I microvilli sono estroflessioni specializzate con supporto di filamenti di actina.
		Sono numerosi sulle cellule dell'epitelio intestinale e coinvolti nell'assorbimento dei nutrienti.
		All'interno si trovano fasci di actina organizzati a compatti tenuti insieme da fimbrina o villina.
		Queste permettono un'adesione molto stretta tra i filamenti.
		Sono legati alla membrana plasmatica da prolungamenti laterali formati da calmodulina associata a miosina $I$.
		Questa struttura ancora il microvillo alla membrana.
		Si nota come nel microvillo avviene un trasporto di vescicole verso la parte terminale.
		I filamenti di actina fungono da binari per il trasporto.
		La calmoduina \`e una chiasi che lega il calcio e si fosforila in sua presenza.

			\paragraph{Ciglia}
			Le ciglia sono strutture molto simili ai microvilli.
			Sono una specializzazione del citoscheletro e si trovano nell'orecchio.
			Si tratta di cellule ciliate.
			Sollecitando le ciglia si trasmettono informazioni sul posizionamento del corpo rispetto alla gravit\`a.

		\subsubsection{Pseudopodi}
		Gli pseudopodi sono prolungamenti lunghi sorretti da filamenti di actina organizzati in una struttura tridimensionale.
		Sono responsabili della fagocitosi e dei movimenti ameboidi.

		\subsubsection{Filopodi}
		I filopodi sono estroflessioni di membrana ricche di filamenti di actina che servono alla cellula per sentire l'ambiente circostante e indirizzare il movimento.

		\subsubsection{Lamellipodi}
		I lamellipodi sono estensioni appiattite brevi che si formano ai margini dei fibroblasti.
		Testano l'ambiente circostante e ancorano la cellula al substrato.

	\subsection{Trasporto}

		\subsubsection{Miosine non muscolari}
		Le miosine contengono un dominio motore e altri che possono interagire con altri monomeri per formare dimeri.

			\paragraph{Miosina $\mathbf{I}$}

				\subparagraph{Struttura}
				La miosina $I$ ha una testa globulare con cui interagisce con i filamenti di actina.
				La coda interagisce con vescicole di membrana che vengono trasportate lungo il filamento.
				Pu\`o lavorare come monomero.
				Presenta inoltre catene leggere soggette a modifiche con un ruolo importante nella regolazione del trasporto.

				\subparagraph{Funzioni}
				La miosina $I$ interviene in:
				\begin{multicols}{2}
					\begin{itemize}
						\item Trasporto di vescicole e organelli lungo i filamenti di actina.
						\item Allungamento di pseudopodi.
						\item Spostare la membrana plasmatica lungo i fasci di actina verso le estremit\`a dei microvilli.
					\end{itemize}
				\end{multicols}

		\subsubsection{Filamenti di actina}
		Il trasporto sui filamenti di actina \`e polarizzato: le miosine assumono la direzione in base alla polarit\`a del filamento.
		Questo pu\`o determinare un movimento bidirezionale del cargo.
		
			\paragraph{Adesione al substrato}
			I filamenti di actina hanno un ruolo importante nell'adesione del substrato.
			L'interazione \`e importante per i meccanismi di differenziazione della cellula e la sua motilit\`a.
			
			\paragraph{Processo di movimento}
			\begin{multicols}{2}
				\begin{enumerate}
					\item Le cellule sviluppano una polarit\`a iniziale.
					\item La cellula aderisce al substrato.
					\item Si estende un'estroflessione allungando una porzione del citoscheletro.
					\item Questo recluta componenti della membrana che formano adesioni focali.
					\item Avviene una retrazione al margine posteriore nel corpo cellulare.
					\item La cellula avanza in una determinata direzione.
				\end{enumerate}
			\end{multicols}

				\subparagraph{Allungamento del margine anteriore}
				Il margine anteriore si allunga attraverso la costrizione di nuovi filamenti di actina in una direzione.
				Si crea una via di trasporto verso il margine in forma di accrescimento.
				L'estensione dei prolungamenti deriva dalla polimerizzazione dei filamenti di actina che spingono contro la membmrana della cellula.

				\subparagraph{Polimerizzazione dell'actina}
				La polimerizzazione dell'actina avviene grazie al complesso \emph{ARP2/3} e a \emph{WASP/SCAR}.
				La seconda attiva la prima.
				Vengono reclutati sulla membrana nel momento in cui i recettori interagiscono con un segnale esterno che stimola il movimento.
				\begin{multicols}{2}
					\begin{enumerate}
						\item Il recettore recluta \emph{WASP/SCAR} e \emph{ARP2/3}.
						\item I due complessi favoriscono la creazione di nuovi filamenti.
						\item \emph{ARP2/3} inizia l'intreccio dei filamenti di actina in prossimit\`a delle terminazioni a barbigli in modo da aumentare il numero delle terminazioni a barbigli in allungamento.
						\item Viene favorito il reclutamento dei monomeri in una certa zona.
					\end{enumerate}
				\end{multicols}

				\subparagraph{Fattori aggiuntivi}
				\begin{multicols}{2}
					\begin{itemize}
						\item \emph{ADF/Cofilina}: fattore di depolimerizzazione dell'actina che rende i monomeri non disponibili per formare nuovi filamenti.
						\item Twinfilina: rimuove il blocco della cofilina.
						\item Profilina: rende i monomeri in grado di legare \emph{ATP}.
					\end{itemize}
				\end{multicols}

				\subparagraph{Regolazione del meccanismo}
				Le proteine \emph{Rho/Rac} modificano il citoscheletro.
				Vengono attivate in particolari condizioni e modificano l'attivit\`a di altri fattori.
				Questo causa una maggiore polimerizzazione in una direzione.

\section{Filamenti intermedi}

	\subsection{Panoramica}
	I filamenti intermedi conferiscono resistenza agli stress.
	Sono presenti nella lamina nucleare e vanno incontro a rimaneggiamenti durante il ciclo nucleare.
	Sono assenti negli animali con scheletri rigidi.

		\subsubsection{Funzioni}
		I filamenti intermedi offrono resistenza a stress meccanici alle cellule epiteliali.
		Queste sono unite da desmosomi, giunzioni molto strette.
		Sono alla base degli emidesmosomi che permettono alla cellula di ancorarsi alla matrice extracellulare.
		Nel nucleo sono costituiti da lamina \emph{A, B, C}.
		Nei muscoli si trova la desmina, in altre cellule si trova la vimentina, mentre nelle cellule epitaliali le cheratine.

		\subsubsection{Confronto con actina e tubulina}
		I filamenti intermedi hanno un diametro compreso tra gli $8$ e i $25nm$.
		Non sono direttamente coinvolti nei movimenti cellulari e non possiedono polarit\`a: i monomeri non presentano terminazioni particolari.

		\subsubsection{Struttura}
		I filamenti sono costituiti da $8$ tetrameri.
		Il monomero \`e costituito da una proteina filamentosa caratterizzata da una porzione ad $\alpha$-elica nella parte centrale, $C$ ed $N$ etrminale.
		In tale porzione vengono ripetuti $7$ amminoacidi creando una struttura capace di formare dimeri che si uniscono tra di loro in maniera antiparallela grazie a interazioni idrofobiche.

	\subsection{Proteine accessorie}
	Proteine accessorie uniscono i filamenti intermedi mettendoli in contatto con altre componenti.
	\begin{multicols}{2}
		\begin{itemize}
			\item Plachina: proteine che uniscono i filamenti intermedi ai microtubuli.
				Permette l'interazione delle parti citoscheletriche con componenti del nucleo.
				In particolare \emph{KASH} guarda verso il citoplasma e \emph{SUN} verso il nucleo.
				Permette pertanto la comunicazione tra ambiente cellulare e esterno.
				Questo \`e mediato da componenti dei filamenti nucleari e proteine transmembrana che interagiscono con compoennti citoscheletriche nel citoplasam.
				Agganciano il DNA o latri componenti.
			\item Neurofilamenti \emph{L, M, H} forniscono un sostegno alle strutture citoscheletriche dell'assone.
		\end{itemize}
	\end{multicols}

	\subsection{Componenti}
	Le componenti dei filamenti intermedi variano in base al tipo cellulare.

		\subsubsection{Cheratine}
		Le cheratine si dividonoin aceide e basiche.
		Sono prodotte nella parte superiore dell'epidermide della cute.

		\subsubsection{Desmina}
		La desmina viene espressa nelle cellule muscolari dove unisce le linee $Z$ dei sarcomeri.
		Si trova inoltre in cellule gliali e neuroni.

		\subsubsection{Proteine dei neurofilamenti}
		Le proteine dei neurofilamenti formano la maggior parte dei filamenti intermedi dei neuroni fornendo sostegno agli assoni dei neuroni motori.

		\subsubsection{lamine nucleari}
		Le lamine nucleari forma una membrana al di sotto della membrana nucleare.

		\subsubsection{Nestine}
		Le nestine sono espresse durante lo sviluppo embrionale nelle cellule staminali.

	\subsection{Polimerizzazione}
	I filamenti intermedi polimerizzano solo in presenza di altri filamenti intermedi all'interno della cellua.
	Presentano un'organizzazione strutturale comunune con una testa, una coda e un dominio centrale formato da una struttura ad $\alpha$-elica e una a bastoncino.

	\subsection{Dimensioni}
	I filamenti intermedi possono raggiungere lunghezze notevoli.
	Permettono alla cellula di resistere agli stress meccanici e mantengono il citoscheletro delle cellule nei tessuti.

	\subsection{Asseblaggio}
	L'assemblaggio dei filamenti intermedi richiede interazioni tra specifici tipi di proteine dei filamenti intermedi.
	Le proteine sono spesso modificate attraverso fosforilazione per regolare il processo.

	\subsection{Organizzazione intracellulare}
	I filamenti intermedi possono agganciarsi alla membrana plasmatica o ad altri componenti del  citoscheletro.
	I filamenti di cheratina sono ancorati alla membrana plasmatica in corrispondenza di aree specifiche di giunzione detti desmosomi ed emidesmosomi.

		\subsubsection{Contatto}
		Il contatto \`e mediato da proteine trans-membrana dette caderine.
		Sulla faccia citoplasmatica sono costruite da placche dense, accumuli sui quali sono reclutati i filamenti id cheratina.
		Le adesioni sono mediate da componenti di membrana che legano i filamenti intermedi come desmplachina.

		\subsubsection{Emodesmosoma}
		L'emodesmosoma \`e la giunzione tra cellula e substrato.
		Permette l'ancoraggio delle cellule epiteliali con il tessuto connettivo sottostante.
		In quest caso i filamenti intermedi sono legati da altre componenti di membrana come le integrine.

	\subsection{Citodieresi e citocinesi}
	La citodieresi e la cithochinesi \`e il meccanismo che le cellule utilizzano per strozzare la membrana e formare due cellule indipendenti.
	Actina e miosina $II$ formano un anello contrattile alla fine della mitosi.
	Questo \`e in grado di contrarsi ma mantiene uno spessore costante.
	Dopo la contrazione actina e miosina vengono depolimerizzate dopo la contrazione per mantenere lo spessore.

\section{Microtubuli}

	\subsection{Panoramica}
	I microtubuli sono tubi cavi di lunghezza notevole liberi ad un'estremit\`a e attaccati all'altra ad un centrosoma \emph{MTCO}.
	Possiedono una particolare composizione nelle ciglia.
	Sono la componente del citoscheletro con diametro maggiore a $25nm$.

	\subsection{Struttura}
	I microtubuli sono polimeri della tubulina, una proteina globulare.
	$\alpha$-tubulina e $\beta$-tubulina si associano tra di loro formando dimeri.
	La seconda lega \emph{GTP}.
	I dimeri si uniscono tra di loro formando un protofilamento.
	La $\gamma$-tubulina \`e un tipo particolare di tubulina che localizza nel centrosoma.
	La forma legata al \emph{GTP} promuove la polimerizzazione, mentre quella legata a \emph{GDP} la dissociazione.

	\subsection{Polimerizzazione}
	I dimeri di $\alpha$ e $\beta$ tubilina polimerizzano formando i microtubuli.
	Un microtubulo \`e costituito da $13$ protofilamenti lineari che si uniscono a formare una struttura cava che conferisce rigidit\`a e flessibilit\`a.
	L'interazione tra i protofilamenti avviene tra le stesse specie di tubulina dando origine a una struttura elocoidale.

		\subsubsection{Protofilamenti}
		I protofilamenti sono composti da due catene testa-coda di dimeri di tubulina disposti parallelamente.
		Queste strutture sono pertanto polarizzate, con polarit\`a positiva verso la subunit\`a $\beta$ e la negativa verso la $\alpha$.
		L'accrescimento veloce avviene verso la terminazione positiva.
		Un cappuccio di estremit\`a con dimeri leganti \emph{GTP} il processo di polimerizzazione avviene comunque.
		In presenza di tubulina-\emph{GDP} massiva il microtubulo si destabilizza e si disassembla in un evento catastrofico.

	\subsection{Funzioni}
	I microtubuli:
	\begin{multicols}{2}
		\begin{itemize}
			\item Vengono utilizzati dalla proteine motrici per il trasporto di vescicole, organelli e RNA.
			\item Vengono utilizzato per la segregazione dei cromosomi durante la mitosi.
			\item Mantengono la forma e permettono alla cellula di funzionare in maniera corretta.
			\item Sono importanti per alcuni meccanismi di locomozione.
		\end{itemize}
	\end{multicols}

	\subsection{Dinamicit\`a}
	I microtubuli non sono strutture fisse ma vengono continuamente riarrangiati: si parla di instabilit\`a dinamica dei microtubuli.
	Alternano infatti cicli di accrescimento e accorgimento:
	\begin{multicols}{2}
		\begin{itemize}
			\item Quando le molecole di tubulina uno aggiunte pi\`u velocemente di quanto il \emph{GTP} viene idrolizzato, i microtubuli mantengono un cappuccio di \emph{GTP} alla terminazione e l'accrescimento continua.
			\item Quando il tasso di polimerizzazione si riduce il \emph{GTP} viene idrolizzato e la tubulina legata al \emph{GDP} si dissocia causando l'accorciamento del microtubulo.
			\item Si pu\`o raggiungere un equilibrio come nel caso dell'actina e si assiste al fenomeno mulinello.
		\end{itemize}
	\end{multicols}

		\subsubsection{Regolazione}
		L'instabili\t`a dinamica \`e importante nel caso del rimodellamento del citoscheletro durante la mitosi.

			\paragraph{Sostanze che interferiscono con la polimerizzazione}
			\begin{multicols}{2}
				\begin{itemize}
					\item Colchina e colcemide: legano la tubulina e impediscono la polimerizzazione dei microtubuli con blocco della mitosi.
					\item Vincristina e vinblastina: inibiscono in modo selettivo le cellule che si moltiplicano in modo eccessivamente rapido.
					\item Nocodazolo: lega le subunit\`a di tubulina favorendo la depolimerizzazione.
					\item Taxolo: stabilizza i microtubuli riducendone la dinamicit\`a.
				\end{itemize}
			\end{multicols}
	
	\subsection{Formazione dei microtubuli}
	Si forma un nucleo ad opera della $\gamma$-tubulina, componente importante del \emph{MTOC} o centro organizzatore dei microtubuli.
	In esso sono presenti altre proteine accessorie che costituiscono il complesso $\gamma$-TuSC>
	La polarit\`a positiva dei microtubuli rimane verso la periferia, mentre la negativa in prossimit\`a del centrosoma.
	Questa \`e importante in quanto viene riconosciuta da proteine motrici.

		\subsubsection{Fattori coinvolti}
		\begin{multicols}{2}
			\begin{itemize}
				\item \emph{MAP2}: nei dendriti e nelle cellule neuronali: si associano ai protofilamenti legando tra di loro microtubuli adiacendi.
				\item \emph{Tau}: si trova negli assoni: provoca un impacchettamento maggiore rispetto a \emph{MAP2} in quanto ha un dominio pi\`u corto.
				\item \emph{XMAP215} stabilizza il cap del microtubulo.
				\item \emph{Chinesina-13}: fattore catastrofico, destabilizza il microtubulo.
				\item \emph{$+$TIP} si lega all'estremit\`a $+$.
				\item \emph{EB1}: si lega all'estremit\`a $+$ e recluta altri fattori.
				\item Stamina: proteina ad $\alpha$ elica che si lega a due dimeri di tubulina impedendo che vengano incorporati nel cap.
				\item Catamina: rompe il legame all'interno dei microtubuli.
			\end{itemize}
		\end{multicols}

	\subsection{Proteine motrici}
	\begin{multicols}{2}
		\begin{itemize}
			\item Chinesine: dominio che interagisce con i micotubuli al $N$-terminale o alla porzione $C$-terminale nel caso delle chinesine $13$ e $14$.
				Sono simili alle miosine.
				Le catene leggere riconoscono uno specifico cargo.
				Il dominio motore \`e un dominio \emph{ATPasico} che causa cambi conformazionali idrolizzando \emph{ATP}.
				La testa avanza.
				I neck donano flessibilit\`a permettendo alle teste di muoversi lungo i microtubuli.
			\item Dineine: le dineine presentano catene pesanti, leggere e intermedie.
				Hanno un dominio motore \emph{ATPasico}.
				Possono essere citoplasmatiche o associate all'essonema.
				Ricostruiscono il reticolo endoplasmatico e l'apparato di Golgi dopo la mitosi.
				Possono effetturare grandi spostamenti.
		\end{itemize}
	\end{multicols}

	\subsection{Centrosoma}
	Il centrosoma \`e costituito da pi\`u centri di nucleazione e due centrioli.
	SI trova al centro della cellula in prossimit\`a del nucleo.
	\`E costituito da $\gamma$-tubulina e altre prtoeine.

		\subsubsection{Centro di organizzazione dei microtubuli}
		Durante la mitosi il centrosoma va incontro a duplicazione e migra ai poli opposti della cellula.
		In questa zona vengono organizzati tutti i fasci di microtubuli che si estendono dai centrosomi duplicati verso il citoplasma formando il fuso mitotico.
		Il centrosoma inizia la crescita dei microtubuli: essi infatti sono attaccati al centrosoma alle loro terminazioni negative e si allungano le terminazioni positive verso la periferia della cellula.
		La colcemide \`e in grado di disassemblare il centrosoma.

		\subsubsection{$\mathbf{\gamma}$-tubulina}
		La $\gamma$-tubulina si trova associata ad $8$ proteine a formare un complesso ad anello.
		La cellula pu\`o polimerizzare microtubuli anche in assenza di un centrosome, ma questo agisce come centro di accrescimento rapido e mantiene la forma e premette i trasporti.

		\subsubsection{Centrioli}
		Il centrosoma presenta due centrioli disposti ortonogalmente l'uno rispetto all'altro e circondati da materiale pericentriolare amorfo.
		Sono importanti per la mitosi e formano i corpi basali di ciglia e flagelli.
			
			\paragraph{Struttura}
			Sono strutture cilindriche formate da $9$ triplette di micrrotubuli.
			Uno dei tre \`e completo mentre $2$ formano una mezzaluna.
			Contengono diverse proteine come la $\beta$ e la $\delta$ tubulina.
			Sono strutture complessa con polarit\`a.

			\paragraph{Funzioni}
			Oltre a costituire il corpo basale di ciglia e flagelli hanno un ruolo nel ciclo cellulare.
			Durante la mitosi i centrioli sono collegati da fibre di centrina, una proteina che forma fibre e connette i due centrioli formando il fuso mitotico.

	\subsection{Organizzazione dei microtubuli nelle cellule}
	I microtubuli possono essere modificati da modifiche post-traduzionali come acetilazione o fosforilazione con un ruolo importante nella loro stabilit\`a.
	Queste modifiche permettono alle proteine associate ai microtubuli di interagire con essi.
	
		\subsubsection{Proteine associate ai microtubuli}
		Queste proteine agendo sulla stabilit\`a garantiscono il funzionamento dei microtubuli e un trasporto ottimale.

			\paragraph{\emph{MAP}}
			Le \emph{MAP} hanno una funzione destabilizzatne: spostano l'equilibrio verso la depolarizzazione.
			\emph{MAP2} unisce due microtubuli mantenendo una distanza maggiore rispetto a \emph{TAU}.

				\subparagraph{Localizzazione}
				\begin{multicols}{2}
					\begin{itemize}
						\item \emph{MAP1,2}: cellule nervose.
						\item \emph{MAP4}: microtubuli di tutte le cellule.
					\end{itemize}
				\end{multicols}

			\paragraph{\emph{TAU}}
			La proteina \emph{TAU} unisce i microtubuli a stretta distanza.

				\subparagraph{Localizzazione}
				\emph{TAU} si trova unicamente nelle cellule nervose.


			\paragraph{Cellule nervose}
			Nei neuroni \emph{TAU} viene espressa solo nell'assone, mentre \emph{MAP2} nei dendriti e nel corpo cellulare.
			Si nota come nell'assone i microtubuli hanno polarit\`a omogenea, mentre i microtubuli dei dendriti sono orientati in entrambe le direzioni.
			Questo \`e importante per il trasporto e le proteine motrici.

	\subsection{Movimento}
	I microtubuli sono responsabili del trasporto di vescicole e organelli e della separazione dei cromosomi nella mitosi e del movimento di ciglia e flagelli

		\subsubsection{Trasporto}
		Il trasporto avviene grazie alla presenza di proteine motrici che fungono da ponte tra il binario e quello che deve essere trasportato.
		Sono dineine che si muovono verso l'estremit\`a negativa e chinesine verso la positiva.
		Il movimento delle ciglia \`e causato da dineine.
		Alcune chinesine si muovono in direzione opposta: questo dipende rispetto a quale estremit\`a lega il cargo: C terminale verso la terminazione negativa, N-terminale verso la positiva.

		\subsubsection{Trasporto e organizzazione intracellulare}
		I microtubuli sono importanti per il trasporto di vescicole organelli e RNA attraverso il citoplasma.
		\`E importante negli assoni delle cellule nervose: i ribosomi presenti nel corpo cellulare e nei dendriti e le vescicole e organelli devono essere trasportati nell'assone.
		La chinesina $2$ trasporta RNA nel polo di un'oocita.
		La chinesina $1$ trasporta la $\beta$-actina nella porzione terminale dei fibroblasti.

			\paragraph{Trasporto ed espansione del reticolo endoplasmatico}
			Le chinesine spingono il reticolo endoplasmatico lungo i microtubuli in direzione positiva verso la periferia della cellula determinandone l'espansione.
			Intervengono nel posizionamento dei lisosomi l'ontano dal centro della cellule.

			\paragraph{Posizionamento dell'apparato di Golgi}
			Le dineine durante la mitosi ricostruiscono l'apparato di Golgi che si trova in prossimit\`a del centrosoma e viene disassemblato durante la mitosi.
			Quando i microtubuli si riorganizzano le vescicole del Golgi vengono portate dalla dineine nella parte centrale della cellula dove si assemblano e ricostruiscono l'apparato.

		\subsubsection{Flagelli e ciglia}
		Flagelli e ciglia sono estroflessioni di membrana che contengono microtubuli e sono responsabili del movimento cellulare.
		Sono costituiti da $9$ dimeri di micritubuli, uno completo e l'altro a mezzaluna.
		All'interno si trova una coppia di microtubuli uniti da proteine accessorie e circondati da una guanina.
		I microtubuli esterni sono legati da nessina.
		La dineina presenta un braccio interno e uno esterno e l'interazione tra i microtubuli adiacenti permette il movimento.

			\paragraph{Ciglia}
			Le ciglia si trovano negli epiteli e sono importanti per muovere e facilitare il passaggio di sostanza.
			Hanno un battito coordinato che permette lo spostamento di una cellula attraverso un fluido o il passaggio di fluido attraverso la cellula.
			Spostano fluidi e muco sulla superficie degli epiteli evitando accumuli di sostanze.
			L'interazione con il citoscheletro avviene attraverso un centriolo madre.

			\paragraph{Flagelli}
			I flagelli differiscono dalle ciglia per la lunghezza e per il movimento ad onda.

				\subparagraph{Struttura}
				La struttura base dei flagelli \`e detta assonema.
				\`E costituito da microtubuli e proteine associate.
				Una membrana esterna riveste un sistema di microtubuli costituito da $0$ coppie di microtubuli esterni e $1$ coppia di microtubuli centrale.
				I mitrotubuli in ogni coppia presentano un microtubulo completo a $13$ protofilamenti e $1$ incompleto con $10$-$11$ protofilamenti.
				Le coppie esterne di microtubuli sono connesse alla coppia interna da filamenti radiali e si associano grazie alla nexina.
				AI microtubuli sono associate dineine responsabili del battito.

				\subparagraph{Ancoraggio}
				Le terminazioni negative sono ancorate al corpo basale, una struttura simile al centriolo formata da $9$ triplette di microtubuli.
				Sono importanti nell'organizzazione dei microtubuli nell'assonema.

				\subparagraph{Movimento}
				Il movimento a battito \`e legato alla presenza di dineine associate ai microtubuli e al fatto che microtubuli adiacenti sono ancorati l'uno all'altro attraverso la nexina.
				Lo scorrimento dei microtubuli esterno mediato dalle dineine causa il battito.
				I microtubuli sono disposti con il lato positivo verso l'esterno e le dineine si muovono indirezione della terminazione negativa.
				Il movimento delle dienine ripiega la struttura.
				La regolazione dell'attivit\`a delle dineine permette un battito coordinato.
				Il processo utilizza \emph{ATP}.

	\subsection{Mitosi}
	I microtubuli vanno incontro a un riarrangiamento importante durante la mitosi che porta alla formazione di $4$ tipi specifici di microtubuli.
	Questo forma il fuso mitotico e separa i cromosomi nelle cellule figlie.
	Il centrosoma si duplica formando $2$ centri di organizzazione dei microtubuli ai poli opposti della cellula formando i due poli del fuso mitotico.

		\subsubsection{Microtubuli del cinetocore}
		Il cinetocore \`e la struttura legata al centromero dei cromosomi.
		\`E ricco in eterocromatina costitutiva a cui si associano microtubuli e componenti proteiche.
		L'attacco al cinetocore stabilizza i microtubuli.

			\paragraph{Microtubuli dei cromosomi}
			I microtubuli dei cromosomi partono dall'estermit\`a dei centrosomi e si legano all'estremit\`a dei cromosomi.

			\paragraph{Microtubuli polari}
			I microtubuli polari non si attaccano ai cromosomi ma si stabilizzano sovrapponendosi l'uno con l'altro al centro della cella.
			Sono importanti nella fase di allontanamento dei due poli e si estendono dai centrosomi ai due poli della cellula.
			Hanno pi\`u terminazioni libere.

			\paragraph{Movimento}
			Dopo la migrazione dei centrosomi ai poli opposti della cellula i cromosomi duplicati si legano al cintetocore e ai microtubuli dei cromosomi.
			Questo allineamento \`e reso possibile grazie a dineine.
			L'allenamento dei cromosomi durante la metafase causa la rottura dei legami tra i cromatidi e l'inizio dell'anafase.

			\paragraph{Movimento dei cromosomi}

				\subparagraph{Anafase \emph{A}}
				L'anafase $A$ consiste nel movimento dei cromosomi verso i poli del fuso, dalla placca del cinetocore verso il polo.
				Avviene grazie alla presenza dei microtubuli del cinetocore.
				La dineina che media il processo consente il movimento dei cromosomi verso la terminazione negativa e facilita la depolimerizzazione dei microtubuli del cintetocore e il loro accorciamento.
				I cromatidi fratelli vengono separati e si allontanano dalla placca metafasica.

				\subparagraph{Anafase \emph{B}}
				Durante l'anafase $B$ avviene la separazione dei poli del fuso accompagnata dall'allungamento dei microtubuli polari e all'accorciamento dei microtubuli dell'aster.
				I microtubuli polari si allungano e sovrappongono spingendo le terminazioni verso le estremit\`a e facilitando l'allontanamento elle due cellule.
				Questo \`e mediato da chinesine che legano i microtubuli del fuso e li spostano verso la zona della loro sovrapposizione verso la terminazione positiva.
				I poli del fuso possono essere spinti dai microtubuli dell'aster.
				In questo caso il movimento \`e mediato da una dineina ancorata ai microtubuli dell'aster che spinge le fibre verso la periferia ella cellula.
				Questo processo garantisce la ripartizione precisa dei cromosomi nelle due cellule figlie.
				Simoltaneamente delle chinesine alterano il turnover e dei microtubuli e promuovono la depolimerizzazione dei microtubuli dell'aster.
				La depolimerizzazione permette la separazione dei cromosomi e il loro movimento verso i poli del fosu.
				Le molecole che interferiscono in questa fase possono dare origine a eventi che portano alla lisi.

	\subsection{Movimento cellulare}
	Il movimento cellulare viene mediato da \emph{GTPasi} monomeriche.

		\subsubsection{Processo}
		\begin{multicols}{2}
			\begin{enumerate}
				\item L'attivazione di \emph{Rhp} causa la formazione di filamenti di actina che permettono un forte ancoraggio al substrato o stress fibers.
				\item Si forma una protrusione da parte della membrana come filopodi, lamellipodi o invadopodi grazie all'attivazione di \emph{Rac}.
				\item L'attivazione di \emph{Cdc42} invece porta alla formazione di filopodi.
					Questa attiva \emph{WASP} che promuove la nucleazione dei monomeri di actina.
				\item Durante l'attacco la membrana citoplasmatica e il citoscheletro sottostante interagiscono con il substrato.
				\item Durante la trazione la parte citoplasmatica viene spinta e portata in avanti.
			\end{enumerate}
		\end{multicols}




