\chapter{Citoscheletro}

\section{Panoramica}
Il citoscheletro \`e una rete di proteine filamentose che si estendono attraverso il citoplasma delle cellule eucariote.

	\subsection{Componenti principali}
	Il citoscheletro \`e composto da tre tipi principali di proteine:
	\begin{multicols}{3}
		\begin{itemize}
			\item Actina.
			\item Tubulina.
			\item Filamenti intermedi.
		\end{itemize}
	\end{multicols}
	I filamenti intermedi si pongono a met\`a tra i filamenti costituiti da actina e i microtubuli costituiti da tubulina.

	\subsection{Funzioni principali}
	Il citoscheletro determina lo scheletro della cellula.
	\begin{multicols}{2}
		\begin{itemize}
			\item Mantene la forma e la struttura della cellula.
			\item Determina la posizione degli organelli e l'organizzazione generale del citoplasma.
			\item Impedisce il collasso del nucleo.
			\item Conferisce rigidit\`a e resistenza a stress.
			\item Conferisce capacit\`a di movimento: l'actina gli permette di interagire con il substrato.
			\item Mantiene in posizione gli organelli.
		\end{itemize}
	\end{multicols}

	\subsection{Fattori coinvolti nell'assemblaggio del citoscheletro}
	Nell'assemblaggio del citoscheletro sono coinvolti diversi fattori:
	\begin{multicols}{2}
		\begin{itemize}
			\item Fattori molecolari: proteine che inibiscono o aiutano la polimerizzazione.
			\item Fattori esterni come la temperatura: ad esempio basse temperature causano la depolimerizzazione dei microtubuli.
		\end{itemize}
	\end{multicols}

	\subsection{Miosina}
	La miosina permette la contrazione muscolare e il trasporto vescicolare.
	Il citoscheletro \`e in grado di fungere da substrato per il trasporto cellulare.
	Questi trasporti avvengono grazie a proteine motrici e miosine.
	
		\subsubsection{Struttura}
		La miosina \`e una delle molecole pi\`u grandi del nostro organismo a $500kDa$.
		\`E un eterodimeri:
		\begin{multicols}{2}
			\begin{itemize}
				\item Due catene passanti che terminano con una testa globulare e una struttura ad $\alpha$-elica con cui interagiscono formando il dimero.
				\item Quattro catene leggere sede di regolazione.
			\end{itemize}
		\end{multicols}

		\subsubsection{Tipologie}
		Diversi tipi di miosine hanno ruoli diversi e interagiscono in modo diverso con il citoscheletro e i filamenti di actina.

\section{Actina}

	\subsection{Panoramica}
	L'actina \`e la proteina citoscheletrica pi\`u diffusa nelle nostre cellule.
	Viene organizzata in modo diverso e polimerizza a formare de filamenti sottili del diametro di $7nm$ detti microfilamenti.
	In alcune cellule, come nel muscolo i filamenti possono essere disposti in modo ordinato, mentre in altri casi formano una maglia tridimensionale con diversi gradi di organizzazione.
	Si trova al disotto della membrana plasmatica dove forma una rete che fornisce supporto meccanico.
	Si pensa abbia un ruolo nel trasporto delle proteine nel nucleo, mentre nel fibroblasto permette un movimento attivo.

	\subsection{Actina monomerica}
	I filamenti di actina sono caratterizzati da strutture globulari di $375aa$ e dal peso di $40kDa$.
	Questi sono detti monomeri di actina o actina globulare $G$.

		\subsubsection{Tipologie di actina globulare}
		\begin{multicols}{3}
			\begin{itemize}
				\item $\alpha$ nei muscoli.
				\item $\beta$.
				\item $\gamma$.
			\end{itemize}
		\end{multicols}

		\subsubsection{Polarit\`a}
		I monomeri di actina possiedono due estremit\`a polari:
		\begin{multicols}{2}
			\begin{itemize}
				\item Estremit\`a appuntita: polarit\`a positiva.
				\item Estremit\`a a barbigli: polarit\`a negativa, presenta il sito di legame per \emph{ATP}.
			\end{itemize}
		\end{multicols}
		La polarit\`a \`e importante in quanto alle diverse estremit\`a si possono legare diversi fattori coinvolti nella polimerizzazione o depolimerizzazione del filamento o al suo ancoraggio nella membrana.
		La polarit\`a definisce pertanto la direzione in cui si muovono le proteine motrici.

		\subsubsection{Legame con \emph{ATP}}
		Un monomero di actina lega \emph{ATP} o \emph{ADP}, rispettivamente \emph{T-actina} e \emph{D-actina}.
		I monomeri liberi in grado di essere reclutati all'interno del filamento legano \emph{ATP} e una volta che sono stati incorporati lo idrolizzano in \emph{ADP}.
		I monomeri alle estremit\`a che legano \emph{ADP} vengono rimssi dal filamento.

	\subsection{Actina filamentosa}
	L'actina filamentosa \`e il polimero formato da actina globulare.
	I monomeri interagiscono polarmente: un'estremit\`a appuntita si lega a una a barbigli dando una direzione al filamento.
	Il filamento non \`e completamente lineare ma si presenta leggermente arrotolato.


		\subsubsection{Centro di nucleazione}
		Si intende per centro di nucleazione o trimero la struttura minima che d\`a inizio al processo di polimerizzazione.
		Pu\`o formarsi in maniera spontanea o essere stimolato da fattori.
		Il processo \`e favorito dalla presenza di \emph{ATP}.
		Dal momento in cui si forma il nucleo la velocit\`a di polimerizzazione aumenta notevolmente fino ad un equilibrio.
		Questo \`e determinato dalla disponibilit\`a delle subunit\`a.
		
		\subsubsection{Polimerizzazione}
		\begin{multicols}{2}
			\begin{enumerate}
				\item Il monomero di actina $G$ legato ad \emph{ATP} viene inserito all'interno della catena.
					L'inserimento pu\`o avvenire a qualsiasi delle due estremit\`a, ma la crescita in direzione positiva \`e pi\`u veloce.
				\item Una volta inserito viene idrolizzato \emph{ATP} e il monomero rimane legato a \emph{ADP}.
				\item Il legame con \emph{ADP} favorisce il distacco del monomero dal filamento.
			\end{enumerate}
		\end{multicols}
		Si nota come a un'estremit\`a un monomero idrolizza \emph{ATP}, mentre all'altra il monomero idrolizzato viene rilasciato.
		L'assemblaggio \`e reversibile, pertanto in caso di necessit\`a i filamenti possono depolimerizzare per dissociazione dei monomeri.
		
		\subsubsection{Equilibrio}
		Si nota come alla base della formazione dei filamenti di actina si trova una continua aggiunta e rimozione di monomeri fino ad arrivare ad una condizione di equilibrio in cui i monomeri aggiunti sono uguali ai monomeri rimossi.
		L'equilibrio \`e determinato dalla concentrazione dei monomeri.
			
			\paragraph{Concentrazione critica dei monomeri}
			Si intende per concentrazione critica dei monomeri la concentrazione per cui la velocit\`a di assemblaggio equivale quella di sisassemblaggio.

			\paragraph{Fattori coinvolti}
			L'equilibrio pu\`o essere modificato da fattori che depolarizzano la membrana aumentando la costante di dissociazione del filamento aumentando il pool di monomeri liberi.
			I fattori che faclitano la polimerizzazione spostano l'equilibrio velocizzando il processo di polimerizzazione favorendo la formazione dei filamenti di actina.

		\subsubsection{Velocit\`a di allungamento}
		I monomeri possono entrare da entrambe le direzioni, ma la velocit\`a di aggiunta dei monomeri \`e diversa in base alla polarit\`a: \`e infatti maggiore nella parte dei barbigli.
		Questo \`e legato al fatto che il complesso \emph{ATP}-actina si dissocia meno velocemente del complesso \emph{ADP}-actina.
		La concentrazione critica \`e pertanto diversa alle estremit\`a, dando origine al fenomeno di treadmilling o mulinello.

			\paragraph{Fenomeno mulinello}
			Il fenomeno mulinello riguarda il comportamento dinamico dell'actina.
			\`E importante per il  turnover e per la polimerizzazione di filamenti.
			I monomeri entrano a un'estremit\`a e si spostano lungo il filamento fino ad arrivare fino all'estremit\`a opposta dove vengono rilasciati.
			La perdita dei monomeri all'estremit\`a viene controbilanciata dall'aggiunta dei monomeri all'estremit\`a opposta.

				\subparagraph{Sostanze che bloccano il fenomeno}
				Sostanze che bloccano il fenomeno mulinello possono essere usate come antitumorali.
				\begin{multicols}{2}
					\begin{itemize}
						\item Latruncolina: depolimerizza i filamenti di actina destabilizzando le subunit\`a.
						\item Citocalasine: si legano nell'estremit\`a $+$ dei filamenti e ne impediscono l'allungamento.
						\item Fallodidina: impedisce la depolimerizzazione dei filamenti e li stabilizza.
					\end{itemize}
				\end{multicols}

		\subsubsection{Regolazione della polimerizzazione}
		Diverse proteine che legano l'actina vanno a controllare l'assemblaggio e disassemblaggio dei filamenti.
		\begin{multicols}{2}
			\begin{itemize}
				\item Timosina: lega il monomero rendendolo indisponibile.
				\item Profilina: lega il monomero promuovendo la sua polimerizzazione.
					Viene inibita attraverso fosforilazione.
				\item \emph{ARP2/3}: favorisce la formazione dei nuclei.
					\emph{Arp2/3} presentano una struttura simile a quella dell'actina.
					Si legano all'estremit\`a $+$ dopo che sono state attivate permettendo un rapido allungamento all'estremit\`a $+$.
					Permettono la formazione di ramificazioni partendo da un filamento preesistente.
				\item Formine: promuovono la formazione dei centri di nucleazione ma non \`e in grado di formare ramificazioni.
					\`E un dimero che si lega ai monomeri.
					Possono presentare domini capaci di reclutare altri fattori che promuovono il reclutamento di monomeri di actina all'interno del filamento nascente come profilina.
				\item Tropomiosina: una proteina allungata che si posiziona sopra i siti di attacco dell'actina con la miosina.
					Blocca l'interazione delle teste di miosina con i siti dell'actina.
					Rende il filamento rigido.
				\item \emph{CapZ}: si localizza nella banda $Z$ del muscolo e stabilizza un'estremit\`a del filamento di actina impedendo la crescita in quella direzione.
				\item Tropomodilina: riveste i filamenti di actina e li lega tra di loro.
				\item Gelsolina: viene attivata dal rilascio di \emph{$Ca^{2+}$} e destabilizza il filamento promuovendo la depolimerizzazione.
				\item Cofilina: \`e un fattore di depolimerizzazione che forza il filamento a compattarsi rendendolo instabile.
					Facilita il distacco dei monomeri \emph{D-actina}.
					Utilizzata per il rinnovo dei filamenti.
			\end{itemize}
		\end{multicols}
		
			

			
