\chapter{Esperienza di laboratorio}
Lo striscio di sangue o striscio di sangue periferico \`e un esame di laboratorio che serve a ottenere una fotografia della popolazione cellulare presente in una goccia di sangue, il 
termine striscio deriva dal fatto ceh per eseguire l'esame si striscia una goccia di sangue sul vetrino. Si valutano i globuli rossi o eritrociti che trasportano l'ossigeno ai 
tessuti, le poastrine o trombociti, piccoli frammenti cellulari importanti per la formazione del coagulo e i globuli bianchi o leucociti che intervengono nella risposta immunitaria. 
Si esegue in quanto esistono patologie che alterano il numero delle cellule del sangue, morfologia e funzionalit\`a, \`e uno dei test migliore per valutare in maniera corretta la 
maturazione delle cellule del sangue e se queste hanno anomalie. Se osservando il sangue si nota che molte cellule hanno forma anomala o sono presenti in numeri troppo alti o molto
bassi si nota come il paziente \`e affetto da una malattia del sangue. Una delle pi\`u comuni \`e l'anemia falciforme che causa un'alterazione della forma dei globuli rossi da 
biconcavi a forma di falce. Lo striscio di sangue si prende una goccia di sangue, si appoggia sul lato del vetrino, si prende il secondo vetrino, si inclina di circa $45$ gradi si 
poggia il bordo del vetrino sulla goccia, si striscia il vetrino uno sull'altro facendo pressione. In questo modo si forma una striscia uniforme. Si agita la provetta per sospendere la
frazione corpuscolata in quella liquida, si preleva un'aliquota di campione con una pipetta, si deposita una piccola goccia in prossomit\`a della banda sabbiata e si procede allo
striscio con un secondo vetrino facendo attenzione a non far debordare il liquido, si lascia fissare all'aria lo striscio per qualche minuto. In laboratorio si pulisce il vetrino con
l'etanolo a $70\%$ per togliere la polvere e sgrassare il vetrino. Il sangue viene smaltito una volta preparati i vetrini va smaltito nei rifuiti biologici, tappe griche sotto la tappa
chimica e si mette un volume di cangeggina con una concentrazione a circa il $10\%$ per tutta la tanica e sotto cappa chimica si svuotano le boccettine di sangue nella tanica. Il sangue
rimane liquido in quanto le boccette contegono anticoagulante, di solito stabile per una settimana. Gocciolina circa $20$ ml di sangue prelevati con pipetta, si appoggia il puntale su
un bordo del vetrino, si deposita la goccia di sangue si prende il secondo vetrino. Lo struscio va lasciato asciugare sotto cappa e una volta asciutto va colorato. Per valutare la 
qualit\`a dello striscio: tutte le cellule devono essere distribuite uniformemente, le cellule su monostrato devono mantenere le caratteristiche e non devono esserci artefatti (corpi 
esterni). Una volta strisciato il sangue va colorato per riuscire a distinguere le cellule del sangue e i tipi di globuli bianchi. Per farlo si colore citoplasma e nucleo delle cellule, 
una coloraizone diffusa \`e la colorazione Diff-Quik, composto da tre soluzione la  prima di colore verde chiaro \`e un fissativo a base di etanolo che serve a fissare le cellule sul 
vetrino, una soluzione a base di eosina con colore rosso e una composta da tiazina con colore blu-violetto. L'eosina \`e una molecola acida con affinit\`a per le proteine del citoplasma, 
con reazione basica se una porzione si colora di sfumatura di rosso viene detta acitofila o eosinofila (affinit\`a per l'eosina). I coloranti tiazinici conferiscono alle molecole acide
cui si legano un colore blu e violetto, utilizzati di solito nucleo e ribosomi ricchi di acridi nucleici, colorano ER rugoso e matrice cartilagine. Per la coloraizone il vetrino ad
asciugare lo si immerge nella prima soluzione fissattiva per $30$ secondi, lo si solleva, lo si sciacqua lo si asciuga appoggiandolo i bordi esterni del vetrino, si immerge il vetrino
nella soluzione 1 per 1 secondo 5 volte, si sciacqua con acqua e si afa asiugare su carta assorbente tamponendo, si immerge nella soluzone 2 5 volte per un seocondo, si sciacqua con 
acqua, si tampona e si monta il vetrino. Per montare il vetrino si intende appoggiare un vetrino coprioggetto su quello che serve per proteggere il preparato. Il vetrino coprioggetto si 
fissa nei bordi si spalma un po' di smalto trasparente. Per osservare un vetrino lo striscio si osserva a un ingrandimento grande per valutare se le cellule sono distribuite uniformemente
e se la colorazione \`e venuta bene, si esamina lo striscio in quanto le popolazioni cellulari hanno quantit\`a molto piccle, si possono esaminare esami morfologici fino a $100X$, si 
valutano i globuli rossi o eritrociti o emazione, sono le cellule pi\`u abbondanti nel sangue e trasportano ossigeno dai polmoni rilasciando CO2, nei mammiferi sono anucleati, discoidali 
e biconcavi, nei non mammiferi possono essere nucleati, ellissoidali, biconvessi. Sono eosinofili (colore rosso) per la presenza della proteina basica emoglobina. Si tiene conto di 
forma, grandezza, e colore in quanto permette di valutare molte malattie loro colorate. Negli uccelli i globuli rossi sono nucleati. I cammelli presentano dei globuli rossi allungati in
quanto il cammello pu\`o trascorrere giorni senza bere e la forma allungata gli permette di passare meglio nei vasi sanguigni deidratati che si restringono e gli permette di espandersi
del $240\%$ e quando il cammello riesce a bere il globulo rosso riesce ad espandersi. Osservando a microscopio un globulo rosso si pu\`o capire la provenienza, si pu\`o valutare il 
numero di globuli rossi, se sono in un numero inferiore rispetto a quello che ci aspettiamo si ha un'anemia e osservando la forma dei globuli rossi si valuta malattie genetiche come
anemia falciforme dovuto a una mutazione missenso, sostituzione di acido glutammico con valina che fa si che si abbia scompenso nella formazione di emoglobina e il globulo abbia forma
di falce. Causa stanchezza e crisi dolorose croniche, una maggiore frequenza di infezioni ma nel nord Africa rimane selezionata e protegge dalle infezioni da parte della malaria. Un
altra cellula del sangue sono le piastrine, residui cellulari, frammenti che derivano da un megacariocito che si frammenta, privi di nucleo, partecipano ai fenomeni di emostasi chiudendo
lesioni formatesi nella parete dei vasi sanguigni e dalla coagulazione (coagulo o tappo piastrinico), hano forma sferica o elissoidale ehanno una zonea centrale intensamente coloroate, 
sono molto piccole, possono essere contate e la forma pu\`o essere colorata, sono cellule colorate di blu o violetto. Si pu\`o capire se si ha una carenza o trombocitopenia che si 
verifica quando il midollo osseo non produce una quantit\`a sufficiente di proteine, quando ne viene distrutto un numero eccessivo o quando si accumulano nella milza ingrossata. I
globuli bianchi sono anche chiamati leucociti e sono le cellule del sangue coinvolte nella risposta immunitaria. Sono osservati facilmente e si distinguono facilmente tra di loro
osservando forma, grandezza e aspetto generale in quanto osservando dimensioni relative la forma di nucleo e citoplasma si determina a quale dei cinque categorie appartengono: 
neutrofili, linfociti, monociti, basofili e eosinofili e inoltre si determina la percentuale relativa dei globuli bianchi tra di loro o conta differenziale, che misura quanti
globuli bianchi di ciascun tipo ci sono e con percentuale relativa si determina se le varie cellule sono presenti in proporzione normale tra di loro, se un tipo di cellule \`e aumentato
o diminuito o se sono presenti cellule immature. Osservango la morfologia si capisce se si \`e in presenza di malattie che determinano una non corretta maturazione delle cellule stesse.
La formula leucocitaria \`e la determinazione percentuale dei vari tipi cellulari di leucociti presenti nello strisci di sangue periferico. Solitamente i globuli bianchi pi\`u presenti
sono i neutrofili, presenti in una percentuale del $60-70\%$, minore basofili $0.5-1\%$. Nell'ordine decrescente: neutrofili, linfociti B e T, monociti, eosinofili e basofili, possono
variare tra specie e et\`a. \`E importante determinare le percentuali relative per fare una valutazione di malattie come infezioni. La conta differenziale \`e usata con gli esami del
sangue pu\`o essere utilizzata per diagnosticare e o monitorare patologie e condizioni che colpiscono uno o pi\`u tipi di globuli bianchi, patolocie possono colpier solo un tipo di 
globulo bianco determinando suo aumento o diminuizione che pu\`o anche fornire un inidizio su quale risposta immunitaria si ha in quel momento. In base alla popolazione si pu\`o avere
un'indicazione su infezione batterica o virale. Si dividono in due popolazioni: agranulociti: linfociti e monociti e i granulociti: neutrofili, eosinofili e basofili. 
I linfociti ($25\%$) hanno le dimensioni minori e sono simili a monociti con un citoplasma basofilo e un nucleo sferico al centro della cellula, la differenza \`e che i linfociti sono
pi\`u piccoli (hanno dimensioni minori). Non si distingue i tipi di linfociti (B, T, NK). La linfocitopenia \`e la diminuizioine patologica del numero di linfociti e pu\`o essere 
acuta (transitoria) per un periodo limitato dovuta a un digiuno, grave stress fisico, infezioni virali (influenza / epatite), corticosteroidi, chemioterapia e o radioterapia per tumore.
La linfocitopenia cronica \`e una diminuizione del numero di linfociti per un periodo prolungato: patologie autoimmuni (artirede remautoide), infezioni croniche (AIDS), conseguenza di 
tumori come leucemie e linfomi. Si pu\`o avere un aumento patologico: leucocitosi linfocitica la causa pi\`u comune \`e un infezione viirale. Si presentano in forma attivata in seguito 
all'inferzioni virali e si pu\`o osservare come abbiano morfologia alterata o sono immaturi, in questo caso ci si trova in presenza di leucemia o linfoma.
I monociti sono il $3-8\%$ dei leucociti totali hanno un citoplasma di colore blu e un nucleo molto grande detto reniforme in posizione eccentrica, sono le cellule pi\`u grandi e se
osservati a ingrandi menti maggiori nell'interno al citoplasma si trovano granuli azzurrofili di piccoli dimensioni e molto densi. Sono i precursori dei macrofagi tissutali e quando
attivati diventano macrofagi. Si pu\`o avere un'altreazione del numero di monociti: monocitosi aumento della percentuale relativa in caso di infezioni da parte di parassiti, malattie
del sangue, patologie autoimmuni, condizioni particolari come splenectomia. La percentuale dei monociti dei leucociti aumenta. I linfociti granulociti hanno la presnza di granuli 
nel citoplasma. I neutrofili sono la componente cellulare pi\`u abbondante fino al $70\%$, presentano attivit\`a fagocitaria si distingono con nucleo multilobato, nel citoplasma hanno
granuli di piccoli dimensioni con scarsa affinit\`a per i coloranti. Se si vedono delle cellule con citoplasma con granuli trasparenti o scarsamente colorati si \`e in presenza di un
neutrofilo (neutro, nessuna colorazione). I neutrofili si trovano aumentati in infezioni batteriche o infiammazioni. Si pu\`o avere una formula leucocitaria invertita in cui si riduce il
numero dei neutrofili con aumento dei linfociti, a volte \`e costituzionale e pu\`o essere causata da infezioni virali, neoplasie, disordini immunitari, assunzione di alcuni farmaci. 
La leucoticosi neutrofila \`e un aumento di neutrofili a seguito di un'infezioni in quanto sono il first responder in caso di infezioni batteriche (funghi e protozoi). Pu\`o essere 
causata da lesioni, disturbi infiammatori e leucemie, determinati farmaci. La condizione contraria quando diminuisce la percentuale di neutrofili si parla di neutropenia, numero 
patologicamente basso di neutrofili. Se si ha una neutropenia grave si aumenta il rischio di contrarre un'infezione potanzialmetnte fatale come effetto collaterale di chemioterapia e 
radioterapia. Gli altri granulociti sono i basofili ($0.5-1\%$) dei leucociti, a differenza dei neutrofili hanno un nucleo bilobato o reniforme simile a quello dei monociti, la differenza
\`e che nel citoplasma dei basofili si trovano granuli molto grandi con intensa colorazione basofili violetti, I basofili rilasciano istamina, bradichina e serotonina in caso di
infortunio e o infezione che aumentano la permeabilit\`a capillare e il flusso di sangue nella zona interessata favorendo l'arrivo delle altre molecole e cellule coinvolte. Sono coinvolti
nelle reazioni allergiche e nei fenomeni di ipersensibilit\`a, producono eparina che ha una funzione chiave nel processo finale di coagulazione del sangue. Con la formula leucocitaria 
si pu\`o notare basofilia con incremento di percentuale di numero di basofili, si manifesta nei soggetti con ipotiroidismo e malattie mieloproliferativee (mielofibrosi, malattia che
fa si che le cellule progenitrici delle cellule del sangue diventino cellule fibrose), la diminuizione basopenia, reazioni acute di persensibilit\`a o infezioni. Gli eosinofili sono
il $2-4\%$) solo $1\%$ circola nel sangue, altri nel midollo osseo e tessuti, hanno un nucleo bilobato lobi collegati da un sottile segmento di cromatina, il citoplasma \`e ricco di 
grossi  granuli acidofili evidenziati dall'eosina, aumentano in numero in presenza di infezioni perassitarie. L;aumento di numero di eosinofili di eosinofilia disurbi allergici o 
da infezioni parassitarie, o alcuni tumori (linfoma di Hodgkin). Oppure la carenza eosinopenia si pu\`o manifestare nella sindrome di Cusching, infezionoi del torrente ematico (sepsi)
e durante il trattamento con corticosteroidi, non causa problemi in quanto altre parti del sistema immunitario la compensano adeguatamente. Entrando nei laboratori, lavare mani indossare
camice cotrollare protocollo e controllare di avere sul balcone cosa pu\`o servire (pipette puntali, provette, soluzioni). I guanti dopo aver lavorato sul bancone non toccare in faccia
in quanto si pu\`o contaminare s\`e stessi. Se volere fare foto prima togliere guanti, buttarli, fare foto e rimettersi i guanti, lo stesso con elemtni esterni. Finita l'esercitazione si
ripulisce buttando i liquidi bioligici e quelli solidi nei biobox, contenitori di cartone con sacchetto giallo e simbolo di pericolo biologico, lavare bene becker, tubi e cilindri, 
acqua e sapone e ultimo risciacquo con acqua distillata, va pulito il bancone con etanolo $70\%$, togliere camice e guanti, lavare mani e uscire. 
\section{Seconda giornata}
L'estrazione di proteine e DNA sono le tecniche base della biologia molecolare. Si estraggono da cellule, batteri, tessuti, sangue, saliva e a seconda del campione di partenza ci sono
modifiche al protocollo, noi vediamo estrazione da colture cellulari. Per l'estrazione delle proteine si osservano le cellule al microscopio ottico per verificare che le cellule nelle
piastre siano cresciute. Dopo di che si fa un lavaggio delle cellule con PBS per rimuovere i residui del terreno e i residui cellulari, si aggiunge il tampone di lisi, si prelevano
le cellule dalla piastra, si trasferiscono in un tubo, si incubano per 10 minuti, si centrifugano per pellettare i residui cellulari e si lascia nel surnatante il lisato proteico, 
dopo di che le proteine sono pronte. In laboratorio si usano le cellule Hela che sono la prima linea cellulare creata, raccolte nel 51 da Henrietta Lacks affetta da cancro alla 
cervice uterina, una volta fatta l'autopsia si \`e notato che le cellule erano in grado di dividersi per un numero infinito di volte. \`E stata creata una linea cellulare immortalizzata.
La differenza tra immortalizzate e non si vede poi. L'isolamento delle cellule ha cambiato il modo di fare scienza. Le cellule Hela si vedono al microscopio ottico come low o high 
density, \`e importante osservarle per essere certi che stiano bene: ce ne si accorge in base alla loro forma: ogni cellula in adesione alle piastre ha una forma particolare, Hela
hanno una forma trapezoidale e una volta osservate si capisce se stanno bene o no: se si vedono cellule con conformazione diversa vuol dire che a loro sta succedendo qualcosa: 
contaminazione batterica iniziano ad arrotondarsi e inoltre si osservano pallini neri viaggiare per il terreno. Una cellula senza forma caratteristica \`e una cellula che sta male e 
inutilizzabile. Ogni linea cellulare ha una forma. Si deve tenere conto. Non hanno questa forma se sono morte o si stanno sedimentando. Oltre a questo si deve controllare la confluenza
delle cellule, quanto sono vicine tra di loro. Con il tempo la confluenza aumenta in quanto le cellule si dividono e riempiono la superficie della piastra, dopo poche ore dalla semina
si passa da $10\%$ a $90\%$ dopo un giorno. \`E importante in quanto si vogliono cellule confluenti ma non ammassate ($80\%$), si deve osservare tutta la piastra per essere sicuri
che siano state seminate in maniera uniforme e che abbiano una confluenza uniforme. Per lisare le cellule si usa UN tampone, quello pi\`u utilizato \`e il RIPA buffer, con composizone
di Tris-HCl tampone che mantiene il pH a 7.4 evitando la denaturazione delle proteine. Questo componente deve distruggere ogni componente cellulare tranne protiene. Poi si aggiunge
una soluzione salina NaCl che previene l'aggregazione proteica non specifica (interazioni tra protiene non volute), si aggiunge l'NP-40, detergiente non ionico e l'SDS, detergente ionico
che distruggono le membrane, l'SDS mantiene le proteine solubili, si aggiunge un cocktail di inibitori delle proteasi per mantenere le proteine integre. Dopo di che si aggiungono anche
DNAsi e RNAasi, enzimi che degradano DNA e RNA. I cocktail inibitori si utilizzano tre inibitori diversi come eupeptina, pepstatina e di solito sono venduti sottoforma di tablets gi\`a 
miscelati, alternativamente si possono aggiungere uno a uno con concentrazioni variabili. Dopo aver aggiunto il buffer di lisi mantenendo la piastra in ghiaccio si devono staccare le
cellule con lo scraper grattandole via dalla superficie della piastra. Lo scraper \`e una sorta di spazzolino con un manico e una testa con tante setoline di plastica semirigida che
grattano via le cellule dalla superficie della piastra. La piastra possiede un coating che favorisce la crescita delle cellule. Dopo aver tolto il terreno di coltura si lava la quantit\`a
di PBS variabile (1 ml si solito), si distribuisce uniformemente e poi si aspira via, di solito si fa pi\`u di un lavaggio. Le cellule si fa lo scraping e si grattano. Con una pipetta 
con puntale pulito si prelevano le cellule e le si trasferiscono in una provetta. La piastra va smaltita nei rifiuti biologici solidi, il bidoncino giallo di plastica rigida sopra i 
banconi serve per tutto ci\`o che \`e accuminato e puntito e pu\`o ferire l'operatore, per rifiuti biologici soliti \`e una scatola di cartone con sacchetto giallo. Per comodit\`a per
i rifiuti biologiic solidi si appoggia sul bangone un sacchetto di plastica trasparente. Ora si aggiunge nella provetta il buffer di lisi, lo si lascia agire per 10 minuti mescolando
ogni due tre, si centrifuga per depositare sul fondo della provetta i detriti cellulari e lasciare nel surnatante le proteine estratte. Unna votla centrifugato si preleva il surnatante e 
si trasferisce in una provetta nuova, a questo punto si devono quantificare le proteine nella quantificazione, la determinazione della quantit\`a delle proteine presenti nella provetta. 
Ci sono varie metodiche per quantificarle e si usa la metodica tramite reagente di Bradford. Si possono quantificare tramite metodi di spettrofotometria diretta o colorimetrici, non
si pu\`o usare la spettrofotometria diretta per un problema di assorbimento: i batteri assorbno nello spettro del visibile prorteine e DNA assorbono nell'ultraiolentto (proteine a 
280 nm), si devono usare cuvette al quarzo. Per evitare di usare le cuvette al quarzo e per avere una quantificazione pi\`u precisa si usa il metodo colorimetrico tra cui si trova quello
di bradford messo appunto nel 66 basato sul comassie brillianto blue, un colorante con pH acido e si lega con i residui basici di arginina istidina, fenilanalina, triptofano e tirosina.
Il colorante \`e di un color marroncino mattone e quando si lega ai residui basici assume un colore blu brillante. Il colorante libero in forma cationica presneta un massimo di 
assorbimento a $465$ nm e dopo il legame con proteine si sposta a $595$ n, e presenta un colore blu brillante, pi\`u la soluzione proteina e colorante \`e blu brillante maggiore la 
quantit\`a di proteine in quanto l'intensit\`a \`e direttamente proporzionale con la quantit\`a di proteine. I vantaggi del bradfordo sono la semplicit\`a di preparazione del reattivo:
si aggiungono in una cuvetta acqua, pochi millilitri del campione e il colorante, miscelo e aspetto 5 minuti che si sviluppi la reazoine e misuro a spettrofotometro, il colore si 
sviluppa veloclemente, il complesso \`e stabile esi possono misurare fino a 2 ore. \`E molto sensibile ed \`e compatibile con la maggior parte dei tamponi di lisi o quelle usati per 
conservarle. Come svantaggi \`e che il reagente colora le cuvette ed \`e difficile da rimuovere, le cuvette vanno buttate nei rifiuti biologici solidi, il contenuto nei rifiuti 
biologici liquidi, svuotate in un becker smaltito sotto cappa chimica. Il colorante si lega ad alcuni amminoacidi e ci sono variazioni in base alla composizione delle proteine che \`e
irrilevante in molti casi in quanto nel caso di totale poche variazioni, alcune proteine non sono solubili nella miscela di rezione acida, problema per proteine ricombinanti o singole, 
per intero estratto cellulare no. Per fare il bradford assay. Le cuvette hanno un verso, una parte zigrinata per tenerne euna trasparente dove passa il fascio di luce, le cuvette non
vanno mai toccate nella parte della luce e se si scrive qualcosa lo si deve fare al di sopra altrimenti disturba la lettura del fascio stesso. Per costruire la curva standard per 
quantificare le provette si deve settare lo spettro e misurarle con qualcosa di noto. Lo spettro fotometro ha un fascio di luce in entrata e il lettore legge quello in uscita, la
differenza tra i due fasci determina l'assorbanza del campione. Prima di leggere i campioni si deve settare il bianco in quanto tutte le soluzioni assorbono della luce, per non
falsare il dato si fa il bianco che \`e sempre la soluzione in cui \`e disciolto il campione da sola (tampone di lisi e reagente di bradford) si ha un volume finale di 1 ml ma si mettono
200 microlitri di colorante e si porta a volume con acqua distillata. Prima di leggere il campione si fa una curva standard per capire e fare una corrispondenza tra lettura del campione
stesso al valore di concentrazione. La curva standard per misurare la quantit\`a esatta di proteine in soluzione si deve  convertire l'assorbanza data dallo spettrofotometro in un 
valore di concentrazione, un milligrammi/ml, per farlo si utilizza una proteina a concentrazione nota con cui si crea la curva standard. Una delle proteine pi\`u usate \`e la BSA o 
albumina sierica bovina una proteina del siero isolato dai bovini che trasporta acidi grassi e coinvolta nel mantenimento del pH del plasma, svolge da proteina di rifermineto o 
proteina con cui si crea la curva standard, lo standard. Si presenta sottoforma di cristalli di colore marroncino chiaro, l'albumina sierica bovina va conservata a temperatura inferiore
a quella ambiente e si pu\`o conservare in frigo o a $20$ gradi. Si prepara la BSA in soluzioni con concentrazione nota crescenti, per costruire la curva standard si devono misuare 
almeno 5 punti e ogni misurazione va replicata almeno in duplicato o triplicato per eliminare errori di pippettaggio e assicurarsi che la curva standard non risulti falsata. Si fanno
gli standard in doppio e se si ottengono letture discordanti si preparano terzi campioni. Si devono fare almeno 5 punti di misurazione a concentrazioni diverse crescenti. Inoltre per
ogni punto si prepara pi\`u di una cuvetta per eliminare errori derivati da un pippettaggio non corretto e soluzioni errate. Si legge in doppio almeno due preparazioni. Per gli 
esperimenti di routine si prepara una sola soluzione della proteina di riferimento, se le letture in doppio discordano leggo una terza provetta. \`E importante settare il bianco che
deve contenere tutti gli agenti presenti tranne la sostanza da quantificare. Molti spettrifotometri permettono di sottrarre direttamente il valore del bianco dal valore di assorbanza 
dal campione. Dopo aver misurato i campioni di BSA si costruisce la curvatura standard o retta di taratura con assorbanza su Y e concentrazione su asse X. Di solito si prepara una
soluzione madre di BSA e si fanno diluizioni seriali, ottenendo cos\`i almeno 5 standard a concentrazione nota, si aggiunge la soluzione di bradford, si miscela invertendo le cuvette, 
si incuba per 5 minuti si legge l'assorbanza in cui si \`e letto precedente il bianco e si costruisce la retta di taratura, con tanti campioni e un lettore di piastre si pu\`o costruire
la retta di calibrazione su piastra con diversi pozzetti. La prima \`e il bianco. \`E importante leggere il campione pi\`u di una volta per lo stesso motivo dello standard. 
La retta di taratura la si costruisce con i valori di assorbanza su Y e i valori di concentrazione di BSA sull'asse X e in base alla retta si calcola la concentrazione dei campioni. Se
si d\`a $y$ il valore letto, $y = a + bx$ e il valore della concentrazione della proteina letta $x = \frac{x-a}{b}$. Riportando i valori di assorbanza si calcola la concentrazione finale.
Oltre a leggere i campioni in duplicato \`e importante restare nel range della curva, se un campione supera in assorbanza il valore della curva o sotto non si pu\`o dedurre con 
fedelt\`a la quantit\`a, un campione poco concentrato non sviluppa abbastanza colore per essere letto per i limiti di sensibilit\`a dello spettrofotometro. La differenza minima non
viene letta dallo spettrofotometro, anche in concentrazioni troppo elevate se troppo brillante lo spettrofotometro ha un valore troppo alto perch\`e lo spettrofotometro lo legga in 
maniera preceisa, idealmente il colore non deve svilupparsi troppo, motivo per cui si diluisce il campione 1 a 10, 1 a 50 e 1 a 100, e il campione concentrato o una delle soluzione
entra nel range dello spettrofotometro e dalla lettura si pu\`o capire anche dalla lettura di valori troppo elevati o troppo bassi: precisa da $0.1$ fino a $0.7$, dopo $0.8$ meglio
diluire 1 a 5 per avere una lettura precisa dello spettrofotometro. Quando si legge si fa attenzione che i valori non superino $0.8$ caso in cui va diluito. 
La prima parte dell'estrazione del DNA \`e identica a quelle delle proteine: eliminare il terreno di coltura, lavare due volte con PBS, aggiungere il tampone di lisi. Il tampone di lisi
del DNA \`e diverso da quello per le proteine: una differenza fondamentale \`e la presenza di proteasi e assenza di DNAasi. Tutte le soluzioni che entrano nel tampone di lisi devono
essere libere da DNAasi e RNAasi come l'acqua. Si indossano sempre i guanti che si indossano per proteggere i campioni in quanto nelle mani sono presenti DNAasi e RNAasi. Oltre a usare
soluzioni RNAasi free si deve stare attenti a non toccare niente con le mani per non contaminare il materiale con cui si lavora. Lo stesso vale per l'RNA. Si usa un kit one step
detto genes in a bottle nel tampone di lisi si trovano i detergenti per rompere la membrana fosfolipidica e rilasciare il DNA nella soluzione, un tampone per mantenere il pH costante
e preservare il DNA e sono presenti proteasi, la pi\`u usata \`e la proteasi K. Il DNA sono presenti molte proteine come gli istoni e si deve liberare il DNA da tutte le proteine 
presenti nella cellula. E la proteasi K agisce a 50 gradi, motivo per cui l'estratto cellulare viene incubato a 50 gradi. Dopo aver incubato si aggiunge isopropanolo in quanto si vuole
far precipitare o floculare il DNA, il tampone di lisi conteiene sali in modo che il DNA sia meno solubile dell'estratto cellulare in quanto contiene una carica elettrica negativa e 
quando si aggiunge un sale in soluzione sono carichi positivamente e vengono attratti dalla carica del DNA neutralizzandolo facendo s\`i che le moleocle di DNA tendano ad uniris tra di
loro, ora si aggiunge alla soluzione etanolo o isopropanolo freddi che causano la precipitazione del DNA in quanto non sono pi\`u in grado di rimanere in soluzione ad un alta 
concentrzione salina e un alta concentrazione di alcol. Nel momento in cui si aggiunge etanolo questo inizia a floculare o a formare tanti filamenti bianchi e se si inverte il tubo una
decina di volte si nota al centro del tubo un gomitolo, il DNA. Pu\`o succedere che l'estrazione non sia perfetta pi\`u \`e spocro con redidui galli / arancioni / chiari vuol dire che
non \`e stato purificato bene dalle altre componenti e invece di galleggiare pu\`o affondare nella soluzione. DNA non puro e contaminato da componenti cellulari. Il DNA sporco deve 
essere rifatto precipitare per poter essere poi utilizzato nelle successive precipitazioni. Dopo aver estratto DNA sporco lo si purifica: dopo aver ottenuto il floculo lo si pelletta, 
rimosso il surnatante sospeso in sali, aggiunto alcol freddo come etanolo o isopropanolo puro, lasciato incubare mezz'ora a $4$ gradi o due ore a $-80$ o tutta la notte a $-20$, una volta
incubato si centrifuga, si elimina il surnatante e si aggiunge l'etanolo a 70 percento per reidratare il DNA, si ricentrifuga, si elimina il surnatante si asciuga il campione e lo si 
pu\`o risospendere o in TE formato da Tris-Edta, stabile per anni o si pu\`o risospenderlo in acqua in quanto molto spesso TE pu\`o reagire con rezioni successive e solitamente il DNA
lo si risospende in acqua, conservato a $-20$ gradi \`e stabile per anni.
\section{Laboratorio 3}
Le colture cellulari sono un sistema modello usato per studiare molte patologie come cancro alzheimer e un sistema principale che permette la strada alle analisi in giro, la colorazione
di organelli citoplasmatici e l'analisi di microscopia in fluorescenza. Estrazione di DNA e proteine per la colorazione di organelli. Una coltura cellulare \`e una coltura di cellule, 
una coltura di cellule che deriva da un tessuto, per preparare una coltura cellulare si rimuove una porzione di cellule da un tessuto animale o vegetale e si mettono a crescere in
un ambiente artificiale a loro favorevole, il tessuto per generare una coltura deve essere disgregato in quanto non si pu\`o serve prelevare un tessuto da un organismo, rimuoverlo da
esso e rendere la coltura a singole cellule, per disgregare le cellule si possono usare enzimi proteolitici in grado di disgreare i legami cellula cellula e renderle singole o ina
disgregazione meccanica in cui vengono utilizzati dei dissociatori meccanici che rompono i legami tra cellule. Questa viene definita coltura primaria, stadio successivamento 
all'isolamento da cellule, \`e eterogenea in quanto i tessuti hanno diversi tipi di cellule, funzionali e di sostegno o che danno nutrimento alle cellule con la funzoine del tessuto. 
La preparaizone di una coltura primaria \`e laboriosa in quanto la disgregazione richiede tempo ma il loro mantenimento che pu\`o essere per un periodo limitato in quanto non hanno
capacit\`a di replicazione infinita ma mantengono per buona parte tutte le caratteristiche del tessuto da cui vengono isolate. Si possono acquistare delle colture cellulari, cellule
immortalizzate che possono durare all'infinito, sono costituite da un solo tipo cellulare, ci sono colture finite (propagare per numero finito e andare in senescenza al raggiungimento
del numero di Hayflick e la replicazoine del DNA nella zona telomerica tendono ad accorciarsi e durante le prime fasi esiste un enzima capace di allungare le code telomeriche e sia in
grado di andare avanti con i cicli di replicazione, quando raggiunge lo stato finale la telomerasi non viene espressa e el cellule vanno incontro a senesceenza), la coltura cellulare \`e
continua in quanto le cellule vengono rese immortali e le cellule vanno avanti all'infinito, un esempio di linea cellulare continua sono le cellule HeLa che derivano dall'autopsia
del camcio della cervice uterina di Henrietta Lacks, sistema modello per il cancro e talmente tanto utilizzate che le cellule possono essere mutate. Le colture cellulari possono essere
diverse ma sono due categorie diverse per il tipo di crescita: possono crescere in sospensione: sospese nel terreno di coltura in piccoli gruppi o singoli, normalmente derivate dal 
sangue. Sono cellule che non costiuitscono tessuti solidi e libere di circolare nel fluido. POsosnoo essere in adesione o monostrato e necessitano di superficie solide trattate con
sostanze che aiutano le cellule ad aderire alla superficie. Possono avere diverse morfologie che dipendono dal tipo cellulare, in sospensione forma sferica come quelle del sistema 
sanguigno, possono essere allungate bipolari o multipolari come i fibroblasti (tessuti di sostegno lunghi ed elastici), le cellule HeLa poligonali con dimensioni regolari. Le cellule
hanno una loro et\`a e la loro et\`a dipende dal numero di passaggi: seminando cellule in un substrato aderisconoo cominciano a ricoprire la superficie del substrato si adattano e 
cominciano a replicarsi e cominciano a ricoprire tutta la superficie, a un certo punto cominciano a ricoprirla totalmente e le cellule vanno diluite e passare in quanto altrimenti le 
cellule si bloccano nella fase G1 e non si replicano pi\`u rimanendo in stallo e possono morire se lasciate l\`i per troppo tempo, le si diluisce in un altro contenitore per dargli tempo
di ricrescere e questo si chiama passaggio e l'et\`a viene definita come numero di passaggi: tutte le volte che la coltura ha raddoppiato di volume ed \`e stata diminuita, questo dipende
dalla velocit\`a di replicazione come nell'epidermide e i tessuti neuronali le cellule si replicano a velocit\`a lente o non si replicano,\`E un parametro importante in quanto non \`e
mai bene tenere le cellule troppo in coltura e quando si scongela in linea cellulare la si tiene per un numero di passaggi in coltura dopo un po' la si butta e si riscongela un'altra
linea. Si pu\`o acquistare da delle banche cellule la lina cellulare da organizzaizoni come ATCC o da altri laboratori in quanto la linea cellulare ha costo. Utilizzare una coltura
cellulare \`e un eccellente sistema modello in vitro per studiare fisiologia normale, biochimica biologia, effetto dei farmaci e composti tossici, mutagenesi e carcinogenesi, 
sviluppo di nuovi farmaci (come screening), terapia genica, consulenza genetica. Per utilizzare una linea cellulare, per coltivarla ci vuole un alto grado di sterilit\`a, le cellule
devono essere mantenute in ambiente sterile, devono ricevere nutrienti e devono essere coltivate a pH e temperatura stabili. Per mammifero pH neutro $7$ e temperatura di $37$ gradi. 
I costituienti base del terreno di coltura che possono essere acquistati e all'interno contengono buona parte di componenti e sostanze nutritive che le cellule necessitano: sali 
inorganici per il bilanciamento osmotico, adesione cellulare e sono cofattori enzimatiic per enzimi e proteine, carboidrati come glucosio e galattosio come sorgente di energia. 
Amminoacidi come glutammina per la proliferazione cellulare, vitamine, acidi grassi e lipidi, proteine e peptide (componenti della dieta), le cellule hanno bisogno di fattori di crescita
e ormoni si aggiunge al terreno di coltura un siero: il siero fetale bofino (FBS), una componente del sangue dalle mucche e viene utilizzato per crescere le cellule, questo siero 
viene controllato per la presenza di virus e micoplasma. Per aggiungere il siero va controllato batch per batch in quanto essendo diverso a secondo della mucca possono esserci variazione
nella crescita cellulare. Una cosa fondamentale per la crescita cellulare \`e il pH che deve essere mantenuto, la maggior parte delle cellule necessita di un pH neutro tra 7 e 7.4 si
ha un bilanciamento naturale attraverso CO2 atmosferica, le cellule necessitano di crescere in ambiente umidificato e incubatore con quantit\`a di CO2 stabile al $5-10\%$, presente nei
tessuti. La CO2 si bilancia con il bicarbonato presente nel terreno di coltura e il sistema tra CO2 e bicarbonato fa in modo di mantenere stabile il pH, opppure si trova un modo chimico
con sostanze che fungono da campioni. Molti terrreni di coltura contengon il rosso fenolo indicatore di pH, ci sono fiasche con terreno fresco, colore arancione rossa di base e quando
le cellule iniziano a crescere le sostanze di scarrto possono far cambiare il pH facendo virare il colore in giallo capendo che le cellule devono essere passate o hanno prodotto tanto
scarto e si deve cambiare il terreno. Il pH pu\`o diverntare viola basico dovuto all'ossigenazione delle cellule per cui se nell'incubatore o tenuto tanto le cellule sotto cappa con un
boost di ossigeno pi\`u alto del normale pH pi\`u basico e il terreno va cambiato. Le superfici e i contenitori per la crescita sno svairati, le fiasche hanno diverse dimensioni e sono 
quelle usate per il mantenimento in quanto quando si devono preparare le cellule ne serve una grande quantit\`a, hanno dimensione grande e questi contenitori sono usati per fare in modo
che l'operatore abbia un numero elevato di cellule. Ci son anche le piastre di coltura e in questo caso presentano diverse dimenisoni e possono servire nel nostro caso per far crescere
le cellule da cui si estrae DNA e proteine, le micropiastre vanno da pi\`u piccole con 96 pozzetti con condizioni diverse a quelli pi\`u grandi a 6 pozzetti in cui si pu\`o mettere
la stessa linea cellulare ripetuta 6 volte analizzando sei condizioni diverse con quattro colorazioni diverse trattato con fluorofori o fluorocromi per colorare diversi organelli. Altri
strumenti cono le falcon da 50 e da 15 e i microtubi in quando contengono sospensione cellulare. L'area di lavoro dove vengono coltivate le cellule. Per far crescere in maniera ottimale 
una coltura cellulare si necessita di un ambiente sterile dovendo maneggiare le cellule per seminarle una cappa biologica a flusso laminare con barriera di aria tra operatore e interno
della cappa in modo che sia considerato sterile, oltre a dover mantenere le colture sterile ha funzione di protezione verso l'operatore. Manegggiando linee cellulari tumorali \`e sempre
meglio non ci sia contatto diretto per preservare la sterilit\`a del campione e sicurezza dell'operatore. Serve un incubatore per la crescita e dei frigoriferi e congelatori per
mantenere le componenti, un microscopio invertito per controllare e contare le cellule per verificare la salute per contare le cellule in modo \`e sempre bene partire dalla stessa 
quanit\`a di cellule, l'azoto liquido serve per la crioconservazione. La cappa biologica \`e un area di lavoro asettica in quanto sterile, il pi\`u pulita possibile e priva di 
contaminaizone virale batterica e mitotica. \`E in grado di mantenere qualsiasi tipo di areosol che possa essere infetto generato durante la procedura di lavoro sotto la cappa e ha una
capacit\`a protettiva da polvere e contaminanti che si trovano all'esterno. MOlte contaminaizoni provengono da uno scorretto uso della cappa: alterazioni del flusso che permettono di 
entrare dell'aria nella cappa. La cappa prima di essere utilizzata va disinfettata con etanolo $70\%$ qualsiasi cosa che si vuole portare dentro la cappa, l'ultima persona che usa la
cappa deve sterilizzarla con raggi UV in modo da renderla asettica e funzionale per far s\`i che l'operatore successivo possa usarla in buona sicurezza. \`E un area di lavoro e non
di conservazione. Le cellule vanno conservate a temperatura ottimale per cui serve un incubatore che ha la funzionalit\`a di poter mantenere una temperatura fissa e 
all'interno ci sar\`a una quantit\`a fissa di CO2 pressione parziale presente nei tessuti e la CO2 contrasta l'acidifcazione del terreno, serve un'atmosfera umidificaca in quanto un
liquido potrebbe cominciare ad evaporare in modo che il terreno non evapori e viene mantenuta la quantit\`a sufficiente a far crescere le cellule. Maneggiare le cellule al di sotto della
cappa biologica \`e fondamentale per evitare contaminaizoni, la peggiore delle quali \`e quella biologica.  La coltura cellulare si contamina con batteri, ugnchi e microplasmi. se 
contaminata non \`e pi\`u utilizzabili in quanto si possono vedere pallini (batteri) e con coltura cellulare contaminata i batteri saturano l'ambiente e si replicano in maniera 
incontrollata facendo cambiare le condizioni di crescita, unica soluzione buttare la cellula. Altra contaminazione \`e quella da lievito nelle stanze cellule dove ci sono diverse 
cappe e diverse persone possono coltivare linee cellulari, aprendo la diversa con molti  lieviti che circolano nell'aria e il passaggio di persone pu\`o creare dei moti che permettono 
il passaggio di lievito e contaminare la coltura, il microplasma cambia la capacit\`a di replicazione delle cellule, rallentandola o rendendola incontrollata. La capacit\`a di lavorare in
maniera sterile permette di evitare contaminazione. Per evitare le contaminaizoni sono gli operatori e si deve cercare di lavorare in maniera efficiente e pulita, maniera asettica, 
lavarsi le mani indossare guanti e camice, disinfettaer superficie della cappa guanti e tutto ci\`o che si vuole portare sotto cappa, dispensare liquidi utilizzando ppipeette o 
pipettatore sterili, tutto il materiale sottocappa \`e sterile e qualsiasi pipetta sierologica, scatole di puntale aperte solo quando si trovano sottocappa. Come norma di sicurezza ogni
cosa che va sotto cappa va sterilizzata: o stata sterilizzata dall'operatore o vanno sterilizzate utilizzando un autoclave, una pentola a pressione gigante on temperature elevatee e 
pressione bassa. Le norme di sicurezza (DA LEGGERE SULLE SLIDE PORCO DIO), le cellule vanno mantenute osservandole tutti i giorni osservando il colore del mezzo indicativo dello stato del
pH e va osservata al microscopio di densit\`a e morfologia di cellue. Dopo ci saremmo trovato con una fiasca con del terreno e la coltura cellulare, poi si sarebbero dovute staccare le 
cellule dal substrato, contarle e seminarle in un'altro contenitore per la colorazione degli organelli citoplasmatici. Per rimuovere le cellule dal substrato \`e possibile staccarle dal
fondo in maniera meccanica con pipetta gratando il fondo della piastra o con i cell scraper che si possono pasare sulla superficie della piastra rimuovendo le cellule dalla superficie, 
noi avremmo usato la tripsina che stacca i legami cellula cellule e cellula substrato in maniera chimica. Molto meno duro sulle cellule e la tripsina se tenuta sulle cellule per tempo
adeguato non \`e tossica. Le procedure di sicurezza per lavorare sottocappa: indossare meccanismi di protezione personale come camici, ci si lava le mani prima e dopo la procedura e si 
indossano guanti (fondamentali), le cellule crescono a $37$ gradi e i terreni usati vanno riscaldati a $37$ gradi, per cui si metono a bagno: i terreni non devono galleggiare, la cosa
fondamentale \`e utilizzare meccanismi di protezione personale, legare capelli lunghi e lavarsi le mani prima e dopo, la maggior parte delle contaminaizoni portate dall'operatore 
mantenedo la sterilit\`a e come si staca una linea cellulare. Se si deve lavorare al CIBIO si deve prenotare la cappa e prima cosa da fare \`e pulire la cappa con etanolo, interno ed
esterno, importante sterilizzare interno ed esterno quello che si porta all'intenro deve essere disinfettato con etanolo in quanto tutto venga disinfettato, stessa cosa si fa con tutte 
le varie cose utili per passare le cellule, l'unica cosa che non si spruzza sono le fiasche con le cellule in quanto etanolo \`e un fissante, la cappa viene saparata in aera pulita, 
sporca e di lavoro, le griglie esterne non vanno coperte in quanto sono quelle che mantengono il flusso. Si controlla lo stato della coltura cellulare e si osserva al microscopioe se
si trova qualcosa che si muove \`e possibile avere una contaminazione batterica (cellule contaminate), a questo punto si deve staccare le cellule dalla superficie e poterle seminare, si
prende la fiasca di cellule, si scrive sulle falcon cosa si trova, si rimuove il terreno di coltura all'interno in quanto \`e consumato e si fa un lavaggio con PBS in quanto il contenuto
del terreno potrebbe inibire l'attivit\`a della tripsina, si rimuove il PBS e rimangono solo le cellule in superficie, si aggiunge la tripsina con pochi ml per fare in modo di staccare
le cellule, funziona a 37 gradi, dopo un periodo di incubazione si fa tap sulla flasca per far staccare le cellule da superficie, si usa terreno nuovo per inibire la tripsina e si
trasferiscono le cellule in un contenitore nuovo. Una volta prese le cellule e raccolte si vuole eliminre il terreno di scarto e i risidui di trispsina, pertanto si centrifugano le 
cellule e sulla base si forma un pellet sulla falcon, si aspira il terreno e se ne aggiunge di fresco e ei trasferisce sulla nuova fiasca le cellule. Prima si aggiunge il terreno per
permettere il maggior scambio di gas. Certe accortezza vanno aumentate se si vuole lavorare con specie importanti come cellule conv ettori virali o cellule immortalizzate con un virus 
si usano diversi accorgimenti per cui sotto cappa si tiene una bottiglia di candeggina per ammazzare i virus. Quello che avremmo fatto \`e prendere le cellule, osservare la morfologia
verificare la correttezza della crescita di morfologia e quantit\`a di cellule, dopo di che si rimuove il terreno usato, si lava con soluzione fisiologica, si aspira e si aggiunge 
enzima proteolitico per staccare le cellule l'una dall'altra e dal substrato, importante per contare le cellule, importante in quanto la quantit\`a di cellule di partenza deve essere
la stessa in tutte le condizioni, per sapere quante ne vanno nei supporti, si necessita pertanto che le cellule non siano aggregate. Quello che si fa \`e dopo averle 
staccate con la tripsina, VEDERE QUANTIT\`A su slide. Per contarle si utilizza un supporto detta camper di BUerker e una colorazione vitale dettta trypan blu, la camera di Buerker 
(emocitometro), per contare cellule contenute in fiaca e si sa all'interno della fiasca quante cellule ci sono. Per contare le cellule si usa la camera di Buerker in cui si mette la
porzione della coltura, sono disegnate a laser con 9 quadrati diversi, quattro composti da 16 quadratini, quello centrale da 25 divisi in quadratini da 16, avremmo usato quelli esterni
da 16, ha delle scalanature per mettere la sospensione. \`E un vetrino portaoggetto che va coperto da un vetrino coprioggetto, i 9 quadrati sono di un millimetro quadro e all'interno
si trova un decimillesimo di millimetro. Il vetrino va pulito tutte le volte con etanolo, si aggiunge una piccola quantit\`a di acqua che aiuta l'adsione del vetrino coprioggetto e 
impedisce il suo movimento, si prende un aprte della coltura cellulare e la si mette nell'intercapedine tra i vetrini, si va sotto il microscopio e le grigle disegnate si dovrebbero 
avere pi\`u di $5$ cellule per quadrante, se ce ne sono troppe si necessita di diluire il campione. A questo punto con una quantit\`a normale la si conta e si contano le cellule 
all'interno di uno dei quadrati, si contano i quadrati esterni normalmente quelli da 16, il fatto che ci siano i quadrati aiuta a tenere traccia delle cellule contate, se ce ne sono
troppe si pu\`o contare uno dei quadratini, il quadrato centrale spesso manca in quanto sono difficili da contare, per aiutare a tenere conto si usa un contapersone, si contano i 4 
quadrati si fa la loro media, la si moltiplica di $10^4$ in quanto il volume \` e diecimila volta meno di quello messo dentro, normalmente tutti i quadrati esterni. Si utilizza un 
colorante vitale prima di metterle sulla camera di Buerker, procedura difficile e per cui si utilizza un colorante vitale trypan blue, pur essendo colorante vitali \`e capace di 
penetrare nelle cellule morte per cui le cellule vive non vengono colorate e quelle morte molto scure, conta le cellule vive diventano pi\`u visibili e si riesce a contarle in maniera
migliore, siccome il trypan blue dilusice la coltura si deve moltiplicare anche per il fattore di diluizione, nel caso fattore di diluizione 2. Utilizzando il colorante rende le cellule
pi\`u visibil aiutando a contarle. Si possono contare le cellule. \`E buona norma contare le cellule nel quadrato anteriore e di destra anche quelle sui bordi in quanto i moti convettivi
non sono sempre equivalenti nel vetrino, (quelle all'interno del bordo non si contano). Dopo aver preso le cellule con una piccola quantit\`a nelle camere di buerker e contate, si 
deve seminare un milione di cellule in una dish, in un'altra lo stesso e duecento mila nei pozzetti per le fluorescenze per poter visualizzare gli organelli citoplasmatici. Per fare
questo si inserisce un vetrino coprioggetto nel pozzetto in quanto per la colorazione si deve poter mettere le cellule si un vetrino, pertanto sopra quello inserito le cellule crescono,
dopo si pu\`o mettere su un vetrino portaoggetti possono crescere. IL vetrino server per far crescere le cellule sulla superficie del pozzetto e sul vetrino. Ogni volta che si mette 
qualcosa nella piastra la si apra e chiuda per evitare i batteri. All'interno della falcon si trova una sospensione cellulare. Quello che si fa \`e trasferire pare delle cellule. Dopo 
aver seminato i quattro pozzetti si fa un movimento per distribuire le cellule il pi\`u possibile. A questo punto si mettono nell'incubatore. 
\section{Laboratorio 4}
La colorazione di organelli citoplasmatici attraverso fluorescenza, un tipo di luminescenza che utilizza l;eccitazione: alcone sostanze possono essere eccitate da radiaizoni 
elettromaghieticeh emettendone a lungehezza d'onga maggioer per cui una lunghezza d;onda viene emessa ceh le colpisce e queste ne riemettono una a lunghezza d'onda maggioer. Possono
essere utilizzate per studiare gil organelli citoplasmatici. Si possono utilizzare i florocromi e sono molecole in grado di assorbire la radiazione elettromagnietica, e emettere una
lunghezza d'nda superiore visualizzabile. Questa propriet\`a \`e transiforia con tempo di decadimento e quando si smette di eccitarli non continuano ad emettere. Contegono nella
struttrua anelli aromatici che danno una struttura particolare alle molecole rendendole stabili e pi\`u stabile pi\`u florescenza emette. Una molecola di fluorocromo \`e il DAPI, agente
intercalante, molecola in grado di legarsi alla doppia elica di DNA (adenina e timina) e legandosi ad esso quando le cellule vengono eccitate questo ne emette u'altra visibile al 
microscopio a fluorescenza, hanno una tendenza a decadere d'intensit\`a ma sono molecole molto piccole con vasta gamma di colorazione (eccitabilit\`a e colorazione). Hanno la 
possibilit\`a di essere legati ad altre molecole per aumentare la specificit\`a. Il photobleaching porta a  una diminuizione della fluorescenza che si pu\`o dare al campione dovuta 
alla degradazione fotochimica del fluoroforo che se continuamente eccitato dai fotoni a lungo andare gli fanno perdere l'intensit\`a luminosa in quanto danneggiano i legami 
covalenti nella molecola e il fluoroforo perde struttura e stabilit\`a e non \`e pi\`u in grado di emettere fluorescenza e il campione diventa inutilizzabile. Non conviene esporre il 
campione alla luce per tanto tempo. Per visualizzare il campione si utilizza il microscopio a fluorescenza ottico che permette di osservare campioni marcati con fluorofori o anticorpi 
fluoresceenti, la luce viene emessa e visualizzata nel microscopio, che permette la magnificazione dell'immagine. Si possono con microscopi pi\`u specifici si possono visualizzare le
vescicole citoplasmatiche. Un microscopio ha una lampada a sorgente luminosa che emette luce diversa a seconda di filtri che danno la possibilit\`a di convertirla in diverse lunghezze
d'onda, si sceglie il filtro adeguato, si colpisce il campione, arriva la luce e il campione emette la lunghezza d'onda di emissione che viene individuata dall'oculare e visibile
all'operatore. UN filtro permette di esprimere diverse lunghezze d'onda. Si ha una lampada con diverse densit\`a, carrellino, obiettivi (10 X) per mettere a fuoco le cellule e dopo 
di che si sposta il filtro a seconda dell'intensit\`a luminosa, una volta utilizzato il campione emette fluorescenza e si individua all'oculare o attraverso una camera che proietta 
immagini a computer, si utilizza anche ingrandimento 20X e 40X. Lo scopo della preparazione \`e di effettuare colorazioni vitali con fluorofori specifici, si fissano che serve in quanto
le cellule vengono freezate nella condizione in cui sono e crea una fotografia perfetta della cellula e si osservano le strutture: membrana plasmatica, citoscheletro, mitocondri, reticolo
endoplasmatico e nucleo. Per colorare il reticolo endoplasmatico \`e un sistema membranoso con vecicole cisterne euno degli organelli pi\`u grandi, molto esteso e coprire anche il $90\%$ 
del citoplasma si divide in ruvido, liscio e di transizione. L'ER pu\`o svolgere diverse funzioni: trasporto di proteine e appena tradotte possono esssere modificate con glicosilazione
e dopo questa le proteine possono essere indirizzate verso la direzione finale o il Golgi con ulteriore glicosilazione o secrete, pu\`o controllare il misfolding delle proteine. Oltre
a questo \`e una riserva di ioni calcio utilizzati come messaggeri secondari (contrazione cellulare, apoptosi processi neurosinaptici, vitali per la motilit\`a degli spermatozoi). Si 
colora con ERtracker, sostanza colorazione vitale, la molecola deve essere aggiunta alle cellule con le funzionoi intatte, \`e composto da glibenclamide, un farmaco delle sulfaniloree e 
viene tuilizzato da pazienti diabetici in quanto favorisce la secrezione di insulina. Si lega al recettore SUR1 associato ai canali potassio ATP dipendenti prominenti sul reticolo 
endoplasmatico, se eccitata a 504 nm emette a 511. Il secondo organello con colorazione vitale sono i mitocondri, organelli citoplasmatici e sono la centrale energetica della cellula in
quanto sono in grado di produrre ATP, sembra si nata da simbiosi tra batterio e cellula superiore. Sono in grado di demolire carboidrati nella respirazione cellulare in ambiente aerobico 
producendo ATP, indispensabile affinch\`e possano essere identificati tattraverso il mitotracker, molecola permeabile alla membrana e le cellule vitali hanno la membrana intatta e 
molecole che possono permeare, molecole permeabili e non troppo grandi. Quella che si usanon \`e fluorescente con un'estremti\`a di clorometile reattiva ai tioli e lo pu\`o fare in
quanto arrivando al mitocondrio viene ossidata quando la respirazione cellulare \`e attiva entra nei mitrocondri e proteine la possono coniugare ai tioli che la rendono fluorescente. 
Eccitata da 579 produce a 599. Le cellule dopo aver aggiunto il tracker le si incuba per mezzora a $37$ in quando temperatura ottimale di crescita e mantenimento. Visto affinch\`e 
questi coloranti possono essere in grado di penetrare le cellule devono essere messse nell'ambiente pi\`u favorevole. Dopo l'incubazione le cellule vanno fissate in quanto le cellule
hanno svolto la reazione e il colorante \`e attivato. Le cellule si fissano in modo da ottenere che le cellule rimangano intatte riducendo al minimo il danno alle strutture del campione.
Si usano composti organici come aldeidi in grado di conservare la struttura cellulare ($4\%$ di paraformaldeide), sono composti tossici per cui ogni volta che si usa la si deve usare
sotto cappa chimica e si usa una pipetta sierologica per rimuovere il terreno, uccedno ma mantenendo intatta la struttura. Si aspira totalmente il campione cercando di rimuovere il 
terreno, si aggiunge la soluzione di paraformaldeide e lo si fa utilizzando una micropipetta e va maneggiata sotto cappa chimica, la sterilit\`a non \`e necessario mantenerla. La 
cappa biologica \`e incapace di aiutare e prevenire il passaggio delle aldeidi da dentro a fuori si deve pertanto usare la cappa chimica. Il fissaggio \`e importante in quanto permette
di mantenere le cellule fisse e stabili senza alterare la struttura. Le cellule bloccate avranno strutture interne intatte, mitocondri e ER fluorescente. A questo punto per quanto 
riguarda colorazione WGA-594 per membrana cellulare le cellule son pronte, mentre per la falloidina si deve fare un altro proecsso in quanto la membrana non \`e permeabile a tutto. Per
permeabilizzare si deve utilizzare una miscela di saponi che servono a distruggere la membrana fosfolipidica SDS o tryphon che aiutano a formare pori di membrana e renderla pi\`u 
permeabile all'ingresso di alcune molecole. Se si vogliono usare anticorpi con immunofluorescenza usando anticorpi diretti per proteina o struttura sono molto pi\`u grandi dei 
fluorofori e per fare questo bisogna rendere la membrana cellulare pi\`u prona a farsi attraversare e per cui viene utilizzata la permeabilizzaizone. Per marcare la membrana cellulare
si usa WGA (wheat germa gglutinin) \`e una lectina che protegge il grano da insetti, lieviti e batteri, le lectine sono altamente specifiche per zuccheri presenti sulla superficie 
della membrana. Le lectine riconosco polisaccaridi presenti sulla membrna cellulare, Posoono essere sfrittate dai virus per riconoscere e attaccare alle membrane cellulari e infettare
la cellula. La WGA viene eccitata a 590 e emette a 617. Torna spesso colorazione rossa e verde in quanto si hanno tre filtri in laboratorio: blu verde e rosso in modo da avere tre
colorazioni nel campione. La falloidina \`e un grado di colorare il citoscheletro con pi\`u funzioni per sostenere la membrana citoplasmatica, ha altre funzioni come la trazione
sui cromosomi, sepearare la cellule, dirigere il traffico cellulare, pu\`o anche sostenere il movimento cellulare pu\`o fare in modo di estendere protrusioni, motilit\`a cellulare, 
assoni e dendriti. Avremmo evindenziato la F actina. Per farlo si usa la falloidina deriva dall'amanita phalloides con diversi anelli aromatici, reagisce stechiometricamente con l'actina
e sembra una molecla di actina e una di falloidina, molto ben visibila, \`e specie aspecifica e il legame con actina \`e specifico. In questo caso ha eccitazoine di 495 nm per emettere
in 518 nm, le colorazioni visualizzabili sono pertanto blu verde e rosso. La parte di colorazione \`e semplice dopo aver aggiunto la parafolmaldeide si toglie e aggiunge PBS, all'interno
della soluzione fisioligica ottenuta si aggiunggono i corolanti. In questo caso le colorazioni sono fotosensibili e si mette in incubazione coperte da stagnola per evitae la degradaizone
del campione. Attesi i 20 minuti di incubazione si deve recuperare dalla superficie cellulare il vetrino e montarlo sul vetrino portaoggetti. Il vetrino portaoggetti nei pozzetti non
ci sta, pertanto abbiamo usato quello coprioggetti, diametro piccolo ma abbastanza grande per permettere la visualizzazione. Le cellule messe a crescere, colorate e fissate, alcune
permeabilizzate e colorate per marcare membrana e citoscheletro per poterle visualizzare si deve spostare su vetrino in quanto i supporti di crescita permettono una visualizzazione in 
campo chiaro ma in caso di fluorescenza la plastica non \`e ottimale, si usa il vetro che d\`a meno dispersione ottica. La plastica d\`a una diffrazione non ottimale, oppure hanno
inventato delle piastre da 96 con fondo ottico similvetro. Quello che si fa \`e utilizzare i vetrini in quanto danno risoluzione migliore, per prendere le cellule serve fissare il
coprioggetto al portaoggetto, per fare questo si usa un mounting media fatto da sostanze che fungono da colla, liquide che solidificano senza alterare la struttura ottica del campione, 
all'interno dei quali si trova un colorante in grado di colorare il nucleo: il DAPI, un organello con doppia membrana e contiene il materiale genetico: cromosomi, istoni e nucleolo, 
il DAPI \`e in grado di colorare il nucleo. Si lega in maniera specifica a adenine e timine, i buchi neri sono nucleoli composti per la maggior paret dda proteine e RNA e non riconosce
proteine e ha bassa affinit\`a per il DNA e non li colora. Si lega al solco minore della dopia elica ed \`e stato integrato nel mounting media, fondamentale in quanto si deve usare il
microscopio con un olio per proteggere l'obiettivo per fare in modo che l'obiettivo possa scorrere in maniera sufficicente. Il mezzo di montaggio permette pertanto di fissare in modo 
molto forte il vetrino coprioggetto a quello portaoggetto si rischia altrimenti scivolamento che distrugga il campione. Un altra possibilit\`a \`e che gli obiettivi si avvicinano e 
allontanano per metterlo a fuoco, se non si presta attenzione essendo molto vicino l'olio pu\`o evitare che si vada troppo strettamente a contatto con il vetrino e lo si rompa. Il DAPI
\`e coniugato con il mezzo di montaggio, basta aggiungere il mezzo di montaggio e il DAPI nel momento in cui il mezzo di montaggio comincia a legare i vetrini il DAPI penetra nelle 
cellule e colora il nucleo, ha molti anelli aromatici ed ha la colorazione pi\`u visibile e quella che si usa per capire dove sono le celule. Viene eccitato a 358 ed emette a 461. Il DAPI
permette di analizzare le cellule a diverse fasi. Si pu\`o voler fare in modo che tutte le cellule stiano nella stessa fase e per fare questo bisogna sottoporre le cellule a uno stress:
togliendo dal terreno di coltura il siero le cellule smettono di crescere e si bloccano tutte in fase G1. Non l'avremmo fatto e avremmo potuto osservare cellule in diversi stati del 
ciclo cellulare. Il DAPI si trova in tutte le colorazioni in quanto evidenzia i nuclei e permette di contare le cellule

