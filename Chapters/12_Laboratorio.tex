\chapter{Esperienza di laboratorio}
\section{Giorno 1 - esame di uno striscio di sangue}

	\subsection{Descrizione}
	Lo striscio di sangue periferico \`e un esame di laboratorio che serve a ottenere uno stato istantaneo della popolazione cellulare presente in una goccia di sangue. 
	L'esame si esegue strisciando una goccia di sangue sul vetrino. 

		\subsubsection{Dati raccolti}
		Lo striscio di sangue permette di valutare:
		\begin{itemize}
			\item Globuli rossi (eritrociti): trasportano l'ossigeno ai tessuti.
			\item Piastrine (trombociti): piccoli frammenti cellulari importanti per la formazione del coagulo.
			\item Globuli bianchi (leucociti): intervengono nella risposta immunitaria.
		\end{itemize}

		\subsubsection{Risvolti clinici}
		Esistono condizioni patologiche che influiscono sul numero, morfologia, funzionalit\`a e vita delle cellule del sangue. 
		Lo striscio di sangue periferico \`e ritenuto il miglior test per valutare e identificare in modo corretto anormalit\`a e immaturit\`a delle cellule del sangue. 
		Nel caso vengano evidenziate cellule anomale in modo significavo \`e possibile che il paziente sia affetto da una patologia e diventa necessario eseguire esami di approfondimento.

	\subsection{Allestimento di uno striscio}
	Per preparare uno striscio di sangue si agita la provetta contenente il sangue (e un anticoagulante per riuscire a mantenerlo stabile per una settimana) per sospendere la frazione corpuscolata in quella liquida e si prelevano $20\si{mL}$ di campione con una pipetta.
	Successivamente si deposita una piccola goccia in prossimit\`a della banda sabbiata di un vetrino pulito con etanolo $70\%$.
	Si appoggia un secondo vetrino (pulito allo stesso modo) sulla goccia, lo si inclina di $45\si{\degree}$ e si striscia, formando cos\`i una striscia uniforme.
	Infine si lascia fissare lo striscio all'aria sotto cappa per qualche minuto prima di essere colorato.
	Il sangue rimanente viene smaltito nei rifiuti biologici, taniche grigie sotto cappa contenenti un volume di candeggina al $10\%$.

	\subsection{Qualit\`a dello striscio}
	La qualit\`a di uno striscio viene valutata in base a tre caratteristiche:
	\begin{itemize}
		\item \`E necessaria una distribuzione cellulare uniforme.
		\item Le cellule devono mantenere le proprie caratteristiche il pi\`u possibile, anche se su monostrato.
		\item Non devono esserci artefatti (corpi esterni).
	\end{itemize}

	\subsection{Colorazione dello striscio}
	Lo striscio va colorato in modo da riuscire a distinguere le diverse cellule del sangue e i globuli bianchi tra di loro. 
	Il Diff-Quik \`e una colorazione citologica utile per raggiungere questo scopo.
	\`E composto da tre soluzioni:
	\begin{itemize}
		\item Fissativo a base di metanolo di colore verde chiaro.
		\item Soluzione $I$: eosina in tampone fosfato di colore rosso.
		\item Soluzione $II$: tiazina in tampone fosfato di colore blu.
	\end{itemize}

		\subsubsection{Eosina}
		L'eosina \`e una molecola acida con alta affinit\`a per costituenti cellulari con reazione basica.
		Ha come target in questo caso le proteine del citosol.
		Una porzione colorata di una sfumatura del rosso viene detta acidofila o eosinofila.

		\subsubsection{Coloranti tiazinici}
		I coloranti tiazinici sono molecole basiche con alta affinit\`a per costituenti cellulari con reazione acida.
		Hanno come target in questo caso nucleo, ribosomi ricchi di acidi nucleici, ER rugoso e la matrice cartilaginea.

		\subsubsection{Procedimento}
		Dopo aver fatto fissare lo striscio sotto cappa:
		\begin{enumerate}
			\item Lo si immerge nella soluzione fissativa per $30$ secondi, lo si sciacqua e lo si asciuga (appoggiando i bordi esterni del vetrino).
			\item Lo si immerge nella soluzione $1$ (eosina) per $1$ secondo $5$ volte, si sciacqua e si asciuga.
			\item Lo si immerge nella soluzione $2$ (tiazina) per $1$ secondo $5$ volte, si sciacqua, si tampona e si monta il vetrino.
		\end{enumerate}

			\paragraph{Montare il vetrino}
			Con montare il vetrino si intende appoggiare un vetrino copri-oggetto su di esso in modo da proteggere il preparato.
			Il vetrino copri-oggetto si fissa nei bordi e si fissa con smalto trasparente. 

	\subsection{Esame dei vetrini}
	Si inizia con osservare lo striscio con un piccolo ingrandimento in modo da valutare l'adeguatezza della distribuzione cellulare e della colorazione.
	Inoltre si osserva l'intero vetrino in modo da assicurarsi di non perdere popolazioni cellulari che si potrebbero concentrare ai margini.
	Si effettuano poi esami morfologici e conte delle cellule usando un ingrandimento maggiore a $100\times$.

		\subsubsection{Globuli rossi}
		Gli eritrociti o emazie sono le cellule pi\`u abbondanti nel sangue e sono specializzate nel trasporto dei gasi respiratori.
		Sono eosinofili a causa della presenza di emoglobina basica e sono valutati in base a forma, grandezza e colore (contenuto emoglobinico)
			
			\paragraph{Tipologie di globuli rossi}
			\begin{itemize}
				\item Mammiferi: anucleati, discoidali e biconcavi.
				\item Altri: nucleati, ellissoidali e biconvessi.
			\end{itemize}
			
			\paragraph{Anemia falciforme}
			Un esempio di patologia osservabile direttamente con lo striscio di sangue \`e l'anemia falciforme. 
			Questa patologia causa un'alterazione della forma dei globuli rossi, che passano da essere biconcavi ad avere una forma simile a una falce. 
			La malattia \`e dovuta ad un difetto genetico: una mutazione puntiforme da $GAG$ in $GUG$ sostituisce un acido glutammico con una valina. 
			La diffusione di questa malattia nelle popolazioni africane \`e dovuta alla resistenza che conferisce contro la malaria.
				
				\subparagraph{Sintomi}
				\begin{itemize}
					\item Anemia: emolisi pi\`u agevole, carenza cronica di globuli rossi. 
						Il paziente prova stanchezza, debolezza, mancanza di fiato, pallore, cefalea e difficolt\`a visive.
					\item Crisi dolorose: insorgono repentinamente e con durata variabile.
						Sono causate dalle occlusioni provocate dall'aggregazione dei globuli anomali che impediscono il flusso del sangue.
						I dolori sono al torace, addome o articolazione con alta frequenza.
					\item Sindrome mani-piede: estremit\`a degli arti gonfie: uno dei primi segnali dell'anemia falciforme nei bambini.
					\item Infezioni: dovute alle lesioni della milza dei globuli anomali.
					\item Ritardo dello sviluppo.
					\item Problemi della vista.
					\item Pelle fredda e gonfiore (edemi) di mani e piedi.
					\item Ittero.
				\end{itemize}

		\subsubsection{Piastrine}
		Le piastrine (trombociti nei non mammiferi) sono frammenti cellulari derivanti dai megacariociti. 
		Sono privi di nucleo ma dotati di membrana plasmatica. 
		Partecipano all'emostasi (chiusura delle lesioni formatesi nella parete dei vasi sanguigni) e alla coagulazione (formazione del coagulo o tappo piastrinico).
		Hanno una forma sferica o elissoidale e presentano una zona centrale intensamente colorata di blu o violetto.
		Delle piastrine si pu\`o valutare il numero e l'aspetto.
			
			\paragraph{Trombocitopenia}
			Si intende per trombocitopenia una carenza di piastrine nel sangue che aumenta il rischio di sanguinamento. 
			Si verifica quando il midollo osseo produce una quantit\`a insufficiente di piastrine, quando ne viene distrutto un numero eccessivo o quando si accumulano nella milza ingrossata.

		\subsubsection{Globuli bianchi}
		I leucociti sono cellule coinvolte nella risposta immunitaria.
		Sono facilmente osservabili e possono essere stimati il numero e tipo di cellule presenti. 
		Dei globuli bianchi si osserva forma, grandezza e aspetto generale. 
		Vengono classificati in cinque diversi tipi e se ne determina la percentuale relativa o conta differenziale. 

			\paragraph{Conta differenziale}
			La conta differenziale misura il numero di ogni tipo di globulo bianco determinando se le cellule sono presenti o meno in proporzione normale tra loro o se sono presenti cellule immature.

				\subparagraph{Formula leucocitaria}
				Si intende per formula leucocitaria la determinazione percentuale dei vari tipi cellulari di leucociti presenti nello striscio di sangue periferico.
				\begin{center}
					\begin{tabular}{|c|c|c|}
						\hline
						\makecell{Formula leucocitaria \\ valori normali nell'adulto} & Percentuali & Assoluti \\
						\hline
						Neutrofili & $40$-$70\%$ & \num{2000}-\num{8000}\si{\milli\metre\cubed} \\
						\hline
						Linfociti $B$ e $T$ & $25$-$55\%$ & \num{1500}-\num{5000}\si{\milli\metre\cubed} \\
						\hline
						Monociti & $2$-$10\%$ & \num{100}-\num{900}\si{\milli\metre\cubed} \\
						\hline
						Eosinofili & $0.5$-$6\%$ & \num{20}-\num{600}\si{\milli\metre\cubed} \\
						\hline
						Basofili & $0$-$2\%$ & \num{2}-\num{150}\si{\milli\metre\cubed} \\
						\hline
					\end{tabular}
				\end{center}
			
				\subparagraph{Informazioni ottenibili}
				La conta differenziale viene usata come parte della conta completa delle cellule del sangue e del check-up generale.
				Pu\`o essere utilizzata per diagnosticare e monitorare patologie e condizioni che colpiscono uno o pi\`u tipi di globuli bianchi.
				\`E un supporto nella diagnosi di patologie che colpiscono la produzione di certi tipi di globuli bianchi, informando su quale tipo sia basso o alto.
				Pu\`o fornire indizi sulla causa specifica di una risposta immunitaria, aiutando a determinare se l'infezione \`e causata da batteri o virus.
			
			\paragraph{Agranulociti - linfociti}
			I linfociti rappresentano il $20$-$30\%$ dei leucociti.
			Sono i globuli bianchi di dimensione inferiore e svolgono le principali funzioni effettrici. 
			Si dividono in linfociti $B$, $T$ e $NK$, ma le loro differenze non sono apprezzabili al microscopio.
			Il nucleo \`e sferico e ben evidente: occupa la maggior parte del volume cellulare.
			Il citoplasma circonda il nucleo in un sottile alone leggermente basofilo e con rare granulazioni azzurrofile.
				
				\subparagraph{Linfocitopenia}
				La linfocitopenia \`e la diminuzione patologica del numero di linfociti nel sangue.
				Pu\`o essere:
				\begin{itemize}
					\item Linfocitopenia acuta: il numero di linfociti pu\`o diminuire temporaneamente durante:
						\begin{itemize}
							\item Infezioni virali.
							\item Digiuno.
							\item Periodi di grave stress fisico.
							\item Uso di corticosteroidi (prednisone).
							\item Chemioterapia e radioterapia per un tumore.
						\end{itemize}
					\item Linfocitopenia cronica: il numero di linfociti pu\`o restare basso per un lungo periodo quando un soggetto \`e affetto da: 
						\begin{itemize}
							\item Patologie autoimmuni come il lupus eritematoso sistemico o l'artrite reumatoide.
							\item Infezioni cromiche come AIDS e tubercolosi miliare.
							\item Tumori come leucemie e linfomi.
						\end{itemize}
				\end{itemize}
				
				\subparagraph{Leucocitosi linfocitica}
				La leucocitosi linfocitica \`e l'aumento patologico del numero dei linfociti nel sangue. 
				La sua causa pi\`u comune \`e un'infezione virale.
				Quando si rileva un loro aumento, si esamina un campione al microscopio per determinare se i linfociti:
				\begin{itemize}
					\item Si presentano in forma attivata (infezioni virali).
					\item Appaiono immaturi o alterati (leucemie o linfomi).
				\end{itemize}
			\paragraph{Agranulociti - monociti}
			I monociti rappresentano il $3$-$8\%$ dei leucociti. 
			Quelli che circolano nel sangue periferico sono i precursori dei macrofagi tissutali o fagociti mononucleati. 
			Sono i globuli bianchi pi\`u voluminosi:
			\begin{itemize}
				\item Il nucleo in posizione eccentrica \`e voluminoso e reniforme.
				\item Nel citoplasma sono visibili granuli azzurrofili di piccole dimensioni.
			\end{itemize}

				\subparagraph{Monocitosi}
				Si intende per monocitosi un'elevata concentrazione dei monociti nel sangue. 
				Si verifica in caso di:
				\begin{itemize}
					\item Infezioni.
					\item Malattie ematologiche.
					\item Patologie autoimmuni.
					\item Condizioni particolari come splenectomia.
				\end{itemize}

			\paragraph{Granulociti - neutrofili}
			I neutrofili sono la componente cellulare pi\`u abbondante dei leucociti $50$-$70\%$. 
			Presentano attivit\`a fagocitaria essendo fagociti polimorfonuclati.
			Il nucleo \`e multilobato e ben visibile, mentre nel citoplasma si possono osservare numerosi granuli di piccole dimensioni con scarsa affinit\`a per i coloranti.
			Aumentano di numero in presenza di infezioni batteriche o disturbi infiammatori. 

				\subparagraph{Formula leucocitaria invertita}
				L'inversione della formula leucocitaria \`e la riduzione dei neutrofili associata all'aumento dei linfociti.
				A volte \`e costituzionale ma pu\`o anche essere causata da:
				\begin{itemize}
					\item Infezioni virali.
					\item Neoplasie.
					\item Disordini immunitari.
					\item Assunzione di alcuni farmaci.
				\end{itemize}

				\subparagraph{Leucocitosi neutrofila}
				La leucocitosi neutrofila \`e l'aumento del numero di neutrofili.
				La sua causa pi\`u comune \`e la normale risposta dell'organismo a un'infezione: i neutrofili sono i primi a rispondere in caso di infezioni batteriche, funghi (micosi) e protozoi.
				Pu\`o essere anche dovuta a:
				\begin{itemize}
					\item Lesioni.
					\item Disturbi infiammatori.
					\item Determinati farmaci.
					\item Alcune leucemie.
				\end{itemize}

				\subparagraph{Neutropenia}
				La neutropenia \`e caratterizzata da un numero patologicamente basso di neutrofili nel sangue.
				Se grave aumenta il rischio di contrarre un'infezione potenzialmente fatale e compare spesso come effetto collaterale di chemioterapia o radioterapia.

			\paragraph{Granulociti: basofili}
			I basofili rappresentano lo $0.5$-$1\%$ dei leucociti. 
			Hanno un nucleo bilobato o reniforme e nel citoplasma sono presenti granuli molto grandi con intensa colorazione basofila e metacromatica.
			Rilasciano mediatori come istamina, bradichina e serotonina in caso di infortunio o infezione, aumentando la permeabilit\`a capillare e il flusso di sangue nella zona interessata, favorendo l'arrivo delle altre molecole e cellule coinvolte.
			Sono inoltre coinvolti nelle reazioni allergiche e nei fenomeni di ipersensibilit\`a oltre a produrre eparina, sostanza fondamentale nel processo finale di coagulazione del sangue.


				\subparagraph{Basofilia}
				L'incremento del numero di basofili o basofilia si manifesta nei soggetti con ipotiroidismo e nelle malattie mieloproliferative. 
				Un esempio \`e la mielofibrosi, una malattia che fa s\`i che le cellule progenitrici delle cellule del sangue diventino cellule fibrose.

				\subparagraph{Basopenia}
				La diminuzione del numero di basofili o basopenia si manifesta nelle reazioni acute da ipersensibilit\`a e nelle infezioni.

			\paragraph{Granulociti: eosinofili}
			Gli eosinofili sono il $2$-$4\%$ dei leucociti, di cui meno del $1\%$ circola nel sangue, mentre la parte restante localizza nel midollo osseo rosso e nei tessuti.
			Il nucleo \`e generalmente bilobati con i lobi collegati da un sottile segmento di cromatina e il citoplasma \`e ricco di grossi granuli acidofili evidenziati dall'eosina. 

				\subparagraph{Eosinopenia}
				La carenza del numero di eosinofili o eosinopenia si manifesta nella sindrome di Cushing, nelle infezioni del torrente ematico (sepsi) e durante il trattamento con corticosteroidi. 
				Generalmente non causa problemi in quanto altre parti del sistema immunitario la compensano adeguatamente.

				\subparagraph{Eosinofilia}
				Un aumento del numero di eosinofili o eosinofilia si manifesta nei disturbi allergici o nelle infezioni parassitarie. 
				Anche alcuni tumori come il linfoma di Hodgkin e leucemia e malattie mieloproliferative possono causare eosinofilia.
		\subsubsection{Distinguere le cellule presenti nel sangue}
		\begin{center}
			\begin{tabular}{|c|c|}
				\hline
				Corpuscolo & Colore \\
				\hline
				Globuli rossi & Rosa, rosso, giallognolo \\
				\hline
				Piastrine & Viola, granuli blu \\
				\hline
				Neutrofili & Nuclei blu, citoplasma rosa, granuli violetti\\
				\hline
				Eosinofili &  Nuclei blu, citoplasma blu, granuli rossi\\
				\hline
				Basofili &  Nuclei viola o blu scuro, granuli violetti\\
				\hline
				Monociti &  Nuclei viola, citoplasma blu chiaro\\
				\hline
				Linfociti &  Nuclei viola, citoplasma blu chiaro\\
				\hline
			\end{tabular}
		\end{center}

\section{Giorno 2 - estrazione di proteine e DNA}
	
	\subsection{Cellule \emph{Hela}}
	Le cellule \emph{Hela} sono parte della prima linea cellulare creata.
	Sono cellule immortalizzate raccolte nel $1951$ dai tessuti di un cancro alla cervice uterina di Henrietta Lacks.
	Queste cellule possono osservarsi ad alta densit\`a o a bassa densit\`a in base alla risoluzione necessaria all'esperimento.
		
		\subsubsection{Cellule immortalizzate}
		Si intende per cellule immortalizzate cellule che, se vengono mantenute nell'ambiente appropriato \emph{in vitro} possono dividersi un indefinito numero di volte.
		Le cellule normali, invece hanno un numero massimo di divisioni possibili.
		
		\subsubsection{Stato di salute delle cellule}
		Prima di compiere degli esperimenti su di esse ci si deve assicurare del loro stato di salute. 
		Questo si controlla osservando la loro forma: ogni cellula in adesione alle piastre ha una forma particolare.
		Le cellule \emph{Hela} in particolare assumono forma trapezoidale e altre conformazioni indicano uno stato di stress.
		Per esempio in caso di infezione batterica le cellule si arrotondano e si osservano pallini neri che viaggiano per il terreno. 
		
		\subsubsection{Confluenza}
		Si intende per confluenza delle cellule quanto queste si trovano vicine tra loro. 
		Con il tempo questo valore aumenta in quanto le cellule si dividono e riempiono la superficie della piastra.
		Si pu\`o passare in un giorno dal $10\%$ al $90\%$ di confluenza.
		Un valore di confluenza ottimale si pone al $80\%$ in quanto le cellule sono confluenti ma non ammassate.
		\`E importante durante i controlli osservare tutta la piastra.


	\subsection{Estrazione di proteine}

		\subsubsection{Protocollo}
		\begin{enumerate}
			\item Osservazione delle cellule \emph{HeLa} al microscopio ottico (obiettivo $10\times$).
			\item Eliminazione del terreno della piastra con le cellule \emph{HeLa}, versandolo nel becker di vetro e lavando con $5\si{ml}$ di \emph{PBS} utilizzando pipetta pasteur.
			\item Aspirazione del resto di \emph{PBS} con pipetta $P1000$ mantenendo la piastra inclinata di $45\si{\degree}$.
			\item Aggiunta di $500\si{\micro\litre}$ di protein lysis buffer mantenendo la piastra sul ghiaccio.
			\item Stacco delle cellule con lo scraper e raccolta della sospensione con la $P1000$ trasferendola in una delle eppendorf.
			\item Centrifuga a \num{13000} \emph{rpm} per $10$ minuti a $4\si{\degree}$.
			\item Preparazione di una provetta eppendorf mettendola in ghiaccio.
			\item Trasferimento del surnatante (lisato proteico) nella provetta eppendorf in ghiaccio, senza toccare il pellet.
			\item Eliminazione della provetta eppendorf con il pellet.
		\end{enumerate}

		\subsubsection{Tampone di lisi - \emph{RIPA buffer}}
		Il \emph{RIPA} buffer \`e una soluzione tampone che viene utilizzata per l'estrazione di proteine da cellule di mammifero.
			
			\paragraph{Composizione}
			\begin{itemize}
				\item $50mM$ \emph{Tris-HCl}, $pH\ 7.4$: tampone per prevenire la denaturazione delle proteine.
				\item $150mM$ \emph{NaCl}: impedisce l'aggregazione proteica non specifica.
				\item $1\%$ \emph{NP-40}: detergente non ionico per estrarre le proteine, distruzione delle membrane.
				\item $0.1\%$ \emph{SDS}: detergente ionico per solubilizzare le proteine.
			\end{itemize}
			Si aggiungono inoltre cocktail inibitori contenenti:
			\begin{itemize}
				\item Inibitori delle proteasi: leupeptina o pepsatina
				\item DNAasi.
				\item RNAasi.
			\end{itemize}

		\subsubsection{Cell scraping}
		Il cell scraping \`e il processo di rimozione delle cellule dalla piastra.
		Per farlo si utilizza una sorta di spazzolino con setole di plastica semirigida che grattano via le cellule dalla superficie della piastra. 

		\subsubsection{Protocollo di quantificazione}
		\begin{enumerate}
			\item Aggiunta di $20\si{\micro\litre}$ di lisato proteico in una cuvetta.
			\item Trasferimento della cuvetta sotto cappa e aggiunta di $1\si{mL}$ di reagente di Bradford, chiudere con ``parafilm'' e mescolare la soluzione.
			\item Attesa di $2$ minuti.
			\item Impostazione dello spettrofotometro con standard method/single $\lambda$.
			\item Lettura dell'assorbanza dei campioni a $595\si{nm}$:
				\begin{enumerate}
					\item Misurazione del bianco composto da $20\si{\micro\litre}$ di buffer di estrazione e $1\si{mL}$ di reagente di Bradford.
						Elimina il segnale di fondo.
					\item Misurazione del campione.
				\end{enumerate}
			\item Eliminazione della cuvetta.
		\end{enumerate}

			\paragraph{Bradford - metodi colorimetrici}
			Essendo che proteine e DNA non assorbono nel visibile si rende necessario utilizzare metodi colorimetrici.
			Il saggio Bradford \`e basato sull'utilizzo del colorante Coomassie Brilliant Blue \emph{G-250}. 
			Il meccanismo di base \`e il legame del colorante a $pH$ acido con i residui basici di arginina, istidina, fenilanalina, triptofano e tirosina (a $pH$ basico)
				
				\subparagraph{Colorante libero}
				Il colorante libero in forma cationica presenta un massimo di assorbimento a $465\si{nm}$ e un colore rosso.
				
				\subparagraph{Legato a proteine}
				Dopo il legame con le proteine si osserva uno spostamento del massimo di assorbimento a $595\si{nm}$ a causa della stabilizzazione della forma anionica del colorante.
				Questo ora presenta un colore blu.

				\subparagraph{Vantaggi e svantaggi}\mbox{}
				\begin{multicols}{2}
					Vantaggi:
					\begin{itemize}
						\item Semplice da preparare.
						\item Immediato sviluppo del colore.
						\item Complesso stabile (osservazione possibile fino a $2$ ore dopo la creazione della soluzione).
						\item Sensibilit\`a elevata (fino a $22\frac{\si{\micro\gram}}{\si{mL}}$).
						\item Compatibilit\`a con la maggior parte di tamponi comuni, agenti denaturanti come guanidina \emph{HCl} $6M$ e urea $8M$ e preservanti come sodio azide.
					\end{itemize}
					\columnbreak
					Svantaggi:
					\begin{itemize}
						\item Il reagente colora le cuvette ed \`e difficile da rimuovere.
						\item La quantit\`a di colorante che si lega alle proteine dipende dal contenuto di amminoacidi standard, rendendo difficile la scelta di uno standard.
						\item Molte proteine non sono solubili nella miscela di reazione acida.
					\end{itemize}
				\end{multicols}

			\paragraph{Curva standard}
			La curva standard per la concentrazione proteica viene ottenuta utilizzando concentrazioni gi\`a note di albumina sierica bovina \emph{BSA}, proteina sierica che trasporta acidi grassi e importante per il mantenimento del $pH$ del plasma.
			La \emph{BSA} ha pertanto il ruolo di proteina di riferimento.
			
				\subparagraph{Creazione}
				Per creare una curva standard si misura l'assorbanza a concentrazioni crescenti sulla proteina di riferimento \emph{BSA} e si misura il bianco.
				Occorrono numerosi punti di osservazione (almeno $5$) e misure ripetute in doppio o triplo.
				Inoltre si nota come il bianco deve contenere tutti i reagenti tranne la sostanza da determinare.
				\begin{enumerate}
					\item Incubazione la soluzione per $5$ minuti.
					\item Lettura dell'assorbanza a $595\si{nm}$.
					\item Costruzione della retta di taratura.
				\end{enumerate}

			\paragraph{Retta di taratura}
			Si costruisce la retta di taratura mettendo i valori di assorbanza sull'asse delle $Y$ e i valori crescenti di concentrazione di \emph{BSA} sull'asse delle $X$. 
			Si ottiene pertanto una retta che mette in relazione concentrazione proteica e assorbanza.

			\paragraph{Quantificazione}
			Usando l'equazione fornita dalla retta di taratura ottenuta si pu\`o ricavare la concentrazione del campione incognito:
			\[y = a + bx\]
			Dove:
			\begin{multicols}{2}
				\begin{itemize}
					\item $y$ \`e il valore di assorbanza letto.
					\item $x$ \`e l'incognita, concentrazione delle proteine.
				\end{itemize}
			\end{multicols}
			La concentrazione delle proteine del campione sar\`a pertanto:
			\[x=\dfrac{y-a}{b}\]

			\paragraph{Range di lettura}
			\`E imporrante per avere letture precise rimanere in un certo range di concentrazione, ottimale da $0.1$ a $0.7$: se il blu \`e troppo brillante o troppo poco lo spettrofotometro non riesce ad essere preciso.
			Per ovviare a questo problema in caso di eccesso si fanno diluizioni del campione, prima $1:10$, poi $1:50$ e infine $1:100$.
			Queste concentrazioni andranno poi tenute in considerazione nel calcolo della concentrazione finale. 

	\subsection{Estrazione di DNA}

		\subsubsection{Protocollo}
		\begin{enumerate}
			\item Osservazione delle cellule \emph{HeLa} al microscopio ottico (obiettivo $10\times$).
			\item Eliminazione del terreno della piastra con le cellule \emph{HeLa}, versandolo nel becker di vetro e lavando con $5\si{ml}$ di \emph{PBS} utilizzando pipetta pasteur.
			\item Aspirazione del resto di \emph{PBS} con pipetta $P1000$ mantenendo la piastra inclinata di $45\si{\degree}$.
			\item Aggiunta di $1\si{\milli\litre}$ di lysis buffer.
			\item Stacco delle cellule con lo scraper e raccolta della sospensione con la $P1000$ trasferendola nel tubo con $3\si{mL}$ di acqua bidistillata.
			\item Aggiunta del restante $1\si{\milli\litre}$ di lysis buffer.
			\item Capovolgere il tubo $5$ volte, facendo attenzione e non agitare con forza.
			\item Aggiungere $300\si{\micro\litre}$ di proteasi $K$ al tubo.
			\item Capovolgere il tubo $5$ volte, facendo attenzione e non agitare con forza.
			\item Incubare il tubo a $50\si{\degree}$ per $10$ minuti nel bagnetto.
			\item Versare lentamente nel tubo $10\si{mL}$ di isopropanolo freddo mantenendo il tubo inclinato a $45\si{\degree}$.
			\item Incubare a temperatura ambiente per $5$ minuti.
			\item Capovolgere il tubo $5$ volte, facendo attenzione e non agitare con forza.
		\end{enumerate}

		\subsubsection{Tampone di lisi \emph{DNA}}
		Il tampone di lisi contiene un detergente capace di rompere la membrana cellulare fosfolipidica e la membrana nucleare rilasciando il DNA in soluzione.
		La soluzione contiene anche un agente tamponante per mantenere il $pH$ della soluzione in modo da preservare la stabilit\`a del DNA.
		Viene anche aggiunta una proteasi per rimuovere le proteine legate al DNA e distruggere enzimi cellulari che lo digerirebbero.
		L'estratto cellulare contenente la proteasi viene incubato a $50\si{\degree}$, temperatura ottimale per l'attivit\`a della proteasi.

		\subsubsection{Precipitazione o floculazione del \emph{DNA}}
			\paragraph{Metodo di precipitazione}
				\subparagraph{Sali}
				Il tampone di lisi contiene anche sali che rendono il DNA meno solubile nell'estratto cellulare.
				La carica negativa del DNA legata ai gruppi fosfato lo rende solubile.
				Quando un sale viene aggiunto al campione gli ioni sodio del sale sono attratti dalle cariche negative del DNA e le neutralizzano, permettendo alle molecole di DNA di unirsi tra di loro.
				
				\subparagraph{Alcol}
				L'aggiunta di alcol freddo precipita il DNA in quanto insolubile in presenza di un'alta concentrazione salina e di alcol.

			\paragraph{Visualizzare il DNA}
			Il DNA precipitato \`e visibile come fini filamenti bianchi al limite dello strato alcolico, mentre le altre sostanze cellulari rimangono in soluzione.
			Sono necessari migliaia di filamenti di DNA per formare una fibra grande abbastanza da essere visibile.
			Il DNA in caso di contaminazioni pu\`o assumere colorazione giallognola o rossiccia e non galleggiare nella soluzione. 
				
				\subparagraph{Contaminazioni}
				In caso di contaminazioni il DNA deve essere fatto precipitare di nuovo: si pelletta il floculo, si rimuove il surnatante, si aggiunge alcol freddo e si lascia incubare
				a basse temperature e lo si risospende in \emph{TE} (tris-edta) o acqua.
		
			\paragraph{Altre molecole}
			Le altre molecole rimangono in soluzione e non sono pertanto visibili.


		


\section{Laboratorio 3}
Le colture cellulari sono un sistema modello usato per studiare molte patologie come cancro alzheimer e un sistema principale che permette la strada alle analisi in giro, la colorazione
di organelli citoplasmatici e l'analisi di microscopia in fluorescenza. Estrazione di DNA e proteine per la colorazione di organelli. Una coltura cellulare \`e una coltura di cellule, 
una coltura di cellule che deriva da un tessuto, per preparare una coltura cellulare si rimuove una porzione di cellule da un tessuto animale o vegetale e si mettono a crescere in
un ambiente artificiale a loro favorevole, il tessuto per generare una coltura deve essere disgregato in quanto non si pu\`o serve prelevare un tessuto da un organismo, rimuoverlo da
esso e rendere la coltura a singole cellule, per disgregare le cellule si possono usare enzimi proteolitici in grado di disgreare i legami cellula cellula e renderle singole o ina
disgregazione meccanica in cui vengono utilizzati dei dissociatori meccanici che rompono i legami tra cellule. Questa viene definita coltura primaria, stadio successivamento 
all'isolamento da cellule, \`e eterogenea in quanto i tessuti hanno diversi tipi di cellule, funzionali e di sostegno o che danno nutrimento alle cellule con la funzoine del tessuto. 
La preparaizone di una coltura primaria \`e laboriosa in quanto la disgregazione richiede tempo ma il loro mantenimento che pu\`o essere per un periodo limitato in quanto non hanno
capacit\`a di replicazione infinita ma mantengono per buona parte tutte le caratteristiche del tessuto da cui vengono isolate. Si possono acquistare delle colture cellulari, cellule
immortalizzate che possono durare all'infinito, sono costituite da un solo tipo cellulare, ci sono colture finite (propagare per numero finito e andare in senescenza al raggiungimento
del numero di Hayflick e la replicazoine del DNA nella zona telomerica tendono ad accorciarsi e durante le prime fasi esiste un enzima capace di allungare le code telomeriche e sia in
grado di andare avanti con i cicli di replicazione, quando raggiunge lo stato finale la telomerasi non viene espressa e el cellule vanno incontro a senesceenza), la coltura cellulare \`e
continua in quanto le cellule vengono rese immortali e le cellule vanno avanti all'infinito, un esempio di linea cellulare continua sono le cellule HeLa che derivano dall'autopsia
del camcio della cervice uterina di Henrietta Lacks, sistema modello per il cancro e talmente tanto utilizzate che le cellule possono essere mutate. Le colture cellulari possono essere
diverse ma sono due categorie diverse per il tipo di crescita: possono crescere in sospensione: sospese nel terreno di coltura in piccoli gruppi o singoli, normalmente derivate dal 
sangue. Sono cellule che non costiuitscono tessuti solidi e libere di circolare nel fluido. POsosnoo essere in adesione o monostrato e necessitano di superficie solide trattate con
sostanze che aiutano le cellule ad aderire alla superficie. Possono avere diverse morfologie che dipendono dal tipo cellulare, in sospensione forma sferica come quelle del sistema 
sanguigno, possono essere allungate bipolari o multipolari come i fibroblasti (tessuti di sostegno lunghi ed elastici), le cellule HeLa poligonali con dimensioni regolari. Le cellule
hanno una loro et\`a e la loro et\`a dipende dal numero di passaggi: seminando cellule in un substrato aderisconoo cominciano a ricoprire la superficie del substrato si adattano e 
cominciano a replicarsi e cominciano a ricoprire tutta la superficie, a un certo punto cominciano a ricoprirla totalmente e le cellule vanno diluite e passare in quanto altrimenti le 
cellule si bloccano nella fase G1 e non si replicano pi\`u rimanendo in stallo e possono morire se lasciate l\`i per troppo tempo, le si diluisce in un altro contenitore per dargli tempo
di ricrescere e questo si chiama passaggio e l'et\`a viene definita come numero di passaggi: tutte le volte che la coltura ha raddoppiato di volume ed \`e stata diminuita, questo dipende
dalla velocit\`a di replicazione come nell'epidermide e i tessuti neuronali le cellule si replicano a velocit\`a lente o non si replicano,\`E un parametro importante in quanto non \`e
mai bene tenere le cellule troppo in coltura e quando si scongela in linea cellulare la si tiene per un numero di passaggi in coltura dopo un po' la si butta e si riscongela un'altra
linea. Si pu\`o acquistare da delle banche cellule la lina cellulare da organizzaizoni come ATCC o da altri laboratori in quanto la linea cellulare ha costo. Utilizzare una coltura
cellulare \`e un eccellente sistema modello in vitro per studiare fisiologia normale, biochimica biologia, effetto dei farmaci e composti tossici, mutagenesi e carcinogenesi, 
sviluppo di nuovi farmaci (come screening), terapia genica, consulenza genetica. Per utilizzare una linea cellulare, per coltivarla ci vuole un alto grado di sterilit\`a, le cellule
devono essere mantenute in ambiente sterile, devono ricevere nutrienti e devono essere coltivate a pH e temperatura stabili. Per mammifero pH neutro $7$ e temperatura di $37$ gradi. 
I costituienti base del terreno di coltura che possono essere acquistati e all'interno contengono buona parte di componenti e sostanze nutritive che le cellule necessitano: sali 
inorganici per il bilanciamento osmotico, adesione cellulare e sono cofattori enzimatiic per enzimi e proteine, carboidrati come glucosio e galattosio come sorgente di energia. 
Amminoacidi come glutammina per la proliferazione cellulare, vitamine, acidi grassi e lipidi, proteine e peptide (componenti della dieta), le cellule hanno bisogno di fattori di crescita
e ormoni si aggiunge al terreno di coltura un siero: il siero fetale bofino (FBS), una componente del sangue dalle mucche e viene utilizzato per crescere le cellule, questo siero 
viene controllato per la presenza di virus e micoplasma. Per aggiungere il siero va controllato batch per batch in quanto essendo diverso a secondo della mucca possono esserci variazione
nella crescita cellulare. Una cosa fondamentale per la crescita cellulare \`e il pH che deve essere mantenuto, la maggior parte delle cellule necessita di un pH neutro tra 7 e 7.4 si
ha un bilanciamento naturale attraverso CO2 atmosferica, le cellule necessitano di crescere in ambiente umidificato e incubatore con quantit\`a di CO2 stabile al $5-10\%$, presente nei
tessuti. La CO2 si bilancia con il bicarbonato presente nel terreno di coltura e il sistema tra CO2 e bicarbonato fa in modo di mantenere stabile il pH, opppure si trova un modo chimico
con sostanze che fungono da campioni. Molti terrreni di coltura contengon il rosso fenolo indicatore di pH, ci sono fiasche con terreno fresco, colore arancione rossa di base e quando
le cellule iniziano a crescere le sostanze di scarrto possono far cambiare il pH facendo virare il colore in giallo capendo che le cellule devono essere passate o hanno prodotto tanto
scarto e si deve cambiare il terreno. Il pH pu\`o diverntare viola basico dovuto all'ossigenazione delle cellule per cui se nell'incubatore o tenuto tanto le cellule sotto cappa con un
boost di ossigeno pi\`u alto del normale pH pi\`u basico e il terreno va cambiato. Le superfici e i contenitori per la crescita sno svairati, le fiasche hanno diverse dimensioni e sono 
quelle usate per il mantenimento in quanto quando si devono preparare le cellule ne serve una grande quantit\`a, hanno dimensione grande e questi contenitori sono usati per fare in modo
che l'operatore abbia un numero elevato di cellule. Ci son anche le piastre di coltura e in questo caso presentano diverse dimenisoni e possono servire nel nostro caso per far crescere
le cellule da cui si estrae DNA e proteine, le micropiastre vanno da pi\`u piccole con 96 pozzetti con condizioni diverse a quelli pi\`u grandi a 6 pozzetti in cui si pu\`o mettere
la stessa linea cellulare ripetuta 6 volte analizzando sei condizioni diverse con quattro colorazioni diverse trattato con fluorofori o fluorocromi per colorare diversi organelli. Altri
strumenti cono le falcon da 50 e da 15 e i microtubi in quando contengono sospensione cellulare. L'area di lavoro dove vengono coltivate le cellule. Per far crescere in maniera ottimale 
una coltura cellulare si necessita di un ambiente sterile dovendo maneggiare le cellule per seminarle una cappa biologica a flusso laminare con barriera di aria tra operatore e interno
della cappa in modo che sia considerato sterile, oltre a dover mantenere le colture sterile ha funzione di protezione verso l'operatore. Manegggiando linee cellulari tumorali \`e sempre
meglio non ci sia contatto diretto per preservare la sterilit\`a del campione e sicurezza dell'operatore. Serve un incubatore per la crescita e dei frigoriferi e congelatori per
mantenere le componenti, un microscopio invertito per controllare e contare le cellule per verificare la salute per contare le cellule in modo \`e sempre bene partire dalla stessa 
quanit\`a di cellule, l'azoto liquido serve per la crioconservazione. La cappa biologica \`e un area di lavoro asettica in quanto sterile, il pi\`u pulita possibile e priva di 
contaminaizone virale batterica e mitotica. \`E in grado di mantenere qualsiasi tipo di areosol che possa essere infetto generato durante la procedura di lavoro sotto la cappa e ha una
capacit\`a protettiva da polvere e contaminanti che si trovano all'esterno. MOlte contaminaizoni provengono da uno scorretto uso della cappa: alterazioni del flusso che permettono di 
entrare dell'aria nella cappa. La cappa prima di essere utilizzata va disinfettata con etanolo $70\%$ qualsiasi cosa che si vuole portare dentro la cappa, l'ultima persona che usa la
cappa deve sterilizzarla con raggi UV in modo da renderla asettica e funzionale per far s\`i che l'operatore successivo possa usarla in buona sicurezza. \`E un area di lavoro e non
di conservazione. Le cellule vanno conservate a temperatura ottimale per cui serve un incubatore che ha la funzionalit\`a di poter mantenere una temperatura fissa e 
all'interno ci sar\`a una quantit\`a fissa di CO2 pressione parziale presente nei tessuti e la CO2 contrasta l'acidifcazione del terreno, serve un'atmosfera umidificaca in quanto un
liquido potrebbe cominciare ad evaporare in modo che il terreno non evapori e viene mantenuta la quantit\`a sufficiente a far crescere le cellule. Maneggiare le cellule al di sotto della
cappa biologica \`e fondamentale per evitare contaminaizoni, la peggiore delle quali \`e quella biologica.  La coltura cellulare si contamina con batteri, ugnchi e microplasmi. se 
contaminata non \`e pi\`u utilizzabili in quanto si possono vedere pallini (batteri) e con coltura cellulare contaminata i batteri saturano l'ambiente e si replicano in maniera 
incontrollata facendo cambiare le condizioni di crescita, unica soluzione buttare la cellula. Altra contaminazione \`e quella da lievito nelle stanze cellule dove ci sono diverse 
cappe e diverse persone possono coltivare linee cellulari, aprendo la diversa con molti  lieviti che circolano nell'aria e il passaggio di persone pu\`o creare dei moti che permettono 
il passaggio di lievito e contaminare la coltura, il microplasma cambia la capacit\`a di replicazione delle cellule, rallentandola o rendendola incontrollata. La capacit\`a di lavorare in
maniera sterile permette di evitare contaminazione. Per evitare le contaminaizoni sono gli operatori e si deve cercare di lavorare in maniera efficiente e pulita, maniera asettica, 
lavarsi le mani indossare guanti e camice, disinfettaer superficie della cappa guanti e tutto ci\`o che si vuole portare sotto cappa, dispensare liquidi utilizzando ppipeette o 
pipettatore sterili, tutto il materiale sottocappa \`e sterile e qualsiasi pipetta sierologica, scatole di puntale aperte solo quando si trovano sottocappa. Come norma di sicurezza ogni
cosa che va sotto cappa va sterilizzata: o stata sterilizzata dall'operatore o vanno sterilizzate utilizzando un autoclave, una pentola a pressione gigante on temperature elevatee e 
pressione bassa. Le norme di sicurezza (DA LEGGERE SULLE SLIDE PORCO DIO), le cellule vanno mantenute osservandole tutti i giorni osservando il colore del mezzo indicativo dello stato del
pH e va osservata al microscopio di densit\`a e morfologia di cellue. Dopo ci saremmo trovato con una fiasca con del terreno e la coltura cellulare, poi si sarebbero dovute staccare le 
cellule dal substrato, contarle e seminarle in un'altro contenitore per la colorazione degli organelli citoplasmatici. Per rimuovere le cellule dal substrato \`e possibile staccarle dal
fondo in maniera meccanica con pipetta gratando il fondo della piastra o con i cell scraper che si possono pasare sulla superficie della piastra rimuovendo le cellule dalla superficie, 
noi avremmo usato la tripsina che stacca i legami cellula cellule e cellula substrato in maniera chimica. Molto meno duro sulle cellule e la tripsina se tenuta sulle cellule per tempo
adeguato non \`e tossica. Le procedure di sicurezza per lavorare sottocappa: indossare meccanismi di protezione personale come camici, ci si lava le mani prima e dopo la procedura e si 
indossano guanti (fondamentali), le cellule crescono a $37$ gradi e i terreni usati vanno riscaldati a $37$ gradi, per cui si metono a bagno: i terreni non devono galleggiare, la cosa
fondamentale \`e utilizzare meccanismi di protezione personale, legare capelli lunghi e lavarsi le mani prima e dopo, la maggior parte delle contaminaizoni portate dall'operatore 
mantenedo la sterilit\`a e come si staca una linea cellulare. Se si deve lavorare al CIBIO si deve prenotare la cappa e prima cosa da fare \`e pulire la cappa con etanolo, interno ed
esterno, importante sterilizzare interno ed esterno quello che si porta all'intenro deve essere disinfettato con etanolo in quanto tutto venga disinfettato, stessa cosa si fa con tutte 
le varie cose utili per passare le cellule, l'unica cosa che non si spruzza sono le fiasche con le cellule in quanto etanolo \`e un fissante, la cappa viene saparata in aera pulita, 
sporca e di lavoro, le griglie esterne non vanno coperte in quanto sono quelle che mantengono il flusso. Si controlla lo stato della coltura cellulare e si osserva al microscopioe se
si trova qualcosa che si muove \`e possibile avere una contaminazione batterica (cellule contaminate), a questo punto si deve staccare le cellule dalla superficie e poterle seminare, si
prende la fiasca di cellule, si scrive sulle falcon cosa si trova, si rimuove il terreno di coltura all'interno in quanto \`e consumato e si fa un lavaggio con PBS in quanto il contenuto
del terreno potrebbe inibire l'attivit\`a della tripsina, si rimuove il PBS e rimangono solo le cellule in superficie, si aggiunge la tripsina con pochi ml per fare in modo di staccare
le cellule, funziona a 37 gradi, dopo un periodo di incubazione si fa tap sulla flasca per far staccare le cellule da superficie, si usa terreno nuovo per inibire la tripsina e si
trasferiscono le cellule in un contenitore nuovo. Una volta prese le cellule e raccolte si vuole eliminre il terreno di scarto e i risidui di trispsina, pertanto si centrifugano le 
cellule e sulla base si forma un pellet sulla falcon, si aspira il terreno e se ne aggiunge di fresco e ei trasferisce sulla nuova fiasca le cellule. Prima si aggiunge il terreno per
permettere il maggior scambio di gas. Certe accortezza vanno aumentate se si vuole lavorare con specie importanti come cellule conv ettori virali o cellule immortalizzate con un virus 
si usano diversi accorgimenti per cui sotto cappa si tiene una bottiglia di candeggina per ammazzare i virus. Quello che avremmo fatto \`e prendere le cellule, osservare la morfologia
verificare la correttezza della crescita di morfologia e quantit\`a di cellule, dopo di che si rimuove il terreno usato, si lava con soluzione fisiologica, si aspira e si aggiunge 
enzima proteolitico per staccare le cellule l'una dall'altra e dal substrato, importante per contare le cellule, importante in quanto la quantit\`a di cellule di partenza deve essere
la stessa in tutte le condizioni, per sapere quante ne vanno nei supporti, si necessita pertanto che le cellule non siano aggregate. Quello che si fa \`e dopo averle 
staccate con la tripsina, VEDERE QUANTIT\`A su slide. Per contarle si utilizza un supporto detta camper di BUerker e una colorazione vitale dettta trypan blu, la camera di Buerker 
(emocitometro), per contare cellule contenute in fiaca e si sa all'interno della fiasca quante cellule ci sono. Per contare le cellule si usa la camera di Buerker in cui si mette la
porzione della coltura, sono disegnate a laser con 9 quadrati diversi, quattro composti da 16 quadratini, quello centrale da 25 divisi in quadratini da 16, avremmo usato quelli esterni
da 16, ha delle scalanature per mettere la sospensione. \`E un vetrino portaoggetto che va coperto da un vetrino coprioggetto, i 9 quadrati sono di un millimetro quadro e all'interno
si trova un decimillesimo di millimetro. Il vetrino va pulito tutte le volte con etanolo, si aggiunge una piccola quantit\`a di acqua che aiuta l'adsione del vetrino coprioggetto e 
impedisce il suo movimento, si prende un aprte della coltura cellulare e la si mette nell'intercapedine tra i vetrini, si va sotto il microscopio e le grigle disegnate si dovrebbero 
avere pi\`u di $5$ cellule per quadrante, se ce ne sono troppe si necessita di diluire il campione. A questo punto con una quantit\`a normale la si conta e si contano le cellule 
all'interno di uno dei quadrati, si contano i quadrati esterni normalmente quelli da 16, il fatto che ci siano i quadrati aiuta a tenere traccia delle cellule contate, se ce ne sono
troppe si pu\`o contare uno dei quadratini, il quadrato centrale spesso manca in quanto sono difficili da contare, per aiutare a tenere conto si usa un contapersone, si contano i 4 
quadrati si fa la loro media, la si moltiplica di $10^4$ in quanto il volume \` e diecimila volta meno di quello messo dentro, normalmente tutti i quadrati esterni. Si utilizza un 
colorante vitale prima di metterle sulla camera di Buerker, procedura difficile e per cui si utilizza un colorante vitale trypan blue, pur essendo colorante vitali \`e capace di 
penetrare nelle cellule morte per cui le cellule vive non vengono colorate e quelle morte molto scure, conta le cellule vive diventano pi\`u visibili e si riesce a contarle in maniera
migliore, siccome il trypan blue dilusice la coltura si deve moltiplicare anche per il fattore di diluizione, nel caso fattore di diluizione 2. Utilizzando il colorante rende le cellule
pi\`u visibil aiutando a contarle. Si possono contare le cellule. \`E buona norma contare le cellule nel quadrato anteriore e di destra anche quelle sui bordi in quanto i moti convettivi
non sono sempre equivalenti nel vetrino, (quelle all'interno del bordo non si contano). Dopo aver preso le cellule con una piccola quantit\`a nelle camere di buerker e contate, si 
deve seminare un milione di cellule in una dish, in un'altra lo stesso e duecento mila nei pozzetti per le fluorescenze per poter visualizzare gli organelli citoplasmatici. Per fare
questo si inserisce un vetrino coprioggetto nel pozzetto in quanto per la colorazione si deve poter mettere le cellule si un vetrino, pertanto sopra quello inserito le cellule crescono,
dopo si pu\`o mettere su un vetrino portaoggetti possono crescere. IL vetrino server per far crescere le cellule sulla superficie del pozzetto e sul vetrino. Ogni volta che si mette 
qualcosa nella piastra la si apra e chiuda per evitare i batteri. All'interno della falcon si trova una sospensione cellulare. Quello che si fa \`e trasferire pare delle cellule. Dopo 
aver seminato i quattro pozzetti si fa un movimento per distribuire le cellule il pi\`u possibile. A questo punto si mettono nell'incubatore. 
\section{Laboratorio 4}
La colorazione di organelli citoplasmatici attraverso fluorescenza, un tipo di luminescenza che utilizza l;eccitazione: alcone sostanze possono essere eccitate da radiaizoni 
elettromaghieticeh emettendone a lungehezza d'onga maggioer per cui una lunghezza d;onda viene emessa ceh le colpisce e queste ne riemettono una a lunghezza d'onda maggioer. Possono
essere utilizzate per studiare gil organelli citoplasmatici. Si possono utilizzare i florocromi e sono molecole in grado di assorbire la radiazione elettromagnietica, e emettere una
lunghezza d'nda superiore visualizzabile. Questa propriet\`a \`e transiforia con tempo di decadimento e quando si smette di eccitarli non continuano ad emettere. Contegono nella
struttrua anelli aromatici che danno una struttura particolare alle molecole rendendole stabili e pi\`u stabile pi\`u florescenza emette. Una molecola di fluorocromo \`e il DAPI, agente
intercalante, molecola in grado di legarsi alla doppia elica di DNA (adenina e timina) e legandosi ad esso quando le cellule vengono eccitate questo ne emette u'altra visibile al 
microscopio a fluorescenza, hanno una tendenza a decadere d'intensit\`a ma sono molecole molto piccole con vasta gamma di colorazione (eccitabilit\`a e colorazione). Hanno la 
possibilit\`a di essere legati ad altre molecole per aumentare la specificit\`a. Il photobleaching porta a  una diminuizione della fluorescenza che si pu\`o dare al campione dovuta 
alla degradazione fotochimica del fluoroforo che se continuamente eccitato dai fotoni a lungo andare gli fanno perdere l'intensit\`a luminosa in quanto danneggiano i legami 
covalenti nella molecola e il fluoroforo perde struttura e stabilit\`a e non \`e pi\`u in grado di emettere fluorescenza e il campione diventa inutilizzabile. Non conviene esporre il 
campione alla luce per tanto tempo. Per visualizzare il campione si utilizza il microscopio a fluorescenza ottico che permette di osservare campioni marcati con fluorofori o anticorpi 
fluoresceenti, la luce viene emessa e visualizzata nel microscopio, che permette la magnificazione dell'immagine. Si possono con microscopi pi\`u specifici si possono visualizzare le
vescicole citoplasmatiche. Un microscopio ha una lampada a sorgente luminosa che emette luce diversa a seconda di filtri che danno la possibilit\`a di convertirla in diverse lunghezze
d'onda, si sceglie il filtro adeguato, si colpisce il campione, arriva la luce e il campione emette la lunghezza d'onda di emissione che viene individuata dall'oculare e visibile
all'operatore. UN filtro permette di esprimere diverse lunghezze d'onda. Si ha una lampada con diverse densit\`a, carrellino, obiettivi (10 X) per mettere a fuoco le cellule e dopo 
di che si sposta il filtro a seconda dell'intensit\`a luminosa, una volta utilizzato il campione emette fluorescenza e si individua all'oculare o attraverso una camera che proietta 
immagini a computer, si utilizza anche ingrandimento 20X e 40X. Lo scopo della preparazione \`e di effettuare colorazioni vitali con fluorofori specifici, si fissano che serve in quanto
le cellule vengono freezate nella condizione in cui sono e crea una fotografia perfetta della cellula e si osservano le strutture: membrana plasmatica, citoscheletro, mitocondri, reticolo
endoplasmatico e nucleo. Per colorare il reticolo endoplasmatico \`e un sistema membranoso con vecicole cisterne euno degli organelli pi\`u grandi, molto esteso e coprire anche il $90\%$ 
del citoplasma si divide in ruvido, liscio e di transizione. L'ER pu\`o svolgere diverse funzioni: trasporto di proteine e appena tradotte possono esssere modificate con glicosilazione
e dopo questa le proteine possono essere indirizzate verso la direzione finale o il Golgi con ulteriore glicosilazione o secrete, pu\`o controllare il misfolding delle proteine. Oltre
a questo \`e una riserva di ioni calcio utilizzati come messaggeri secondari (contrazione cellulare, apoptosi processi neurosinaptici, vitali per la motilit\`a degli spermatozoi). Si 
colora con ERtracker, sostanza colorazione vitale, la molecola deve essere aggiunta alle cellule con le funzionoi intatte, \`e composto da glibenclamide, un farmaco delle sulfaniloree e 
viene tuilizzato da pazienti diabetici in quanto favorisce la secrezione di insulina. Si lega al recettore SUR1 associato ai canali potassio ATP dipendenti prominenti sul reticolo 
endoplasmatico, se eccitata a 504 nm emette a 511. Il secondo organello con colorazione vitale sono i mitocondri, organelli citoplasmatici e sono la centrale energetica della cellula in
quanto sono in grado di produrre ATP, sembra si nata da simbiosi tra batterio e cellula superiore. Sono in grado di demolire carboidrati nella respirazione cellulare in ambiente aerobico 
producendo ATP, indispensabile affinch\`e possano essere identificati tattraverso il mitotracker, molecola permeabile alla membrana e le cellule vitali hanno la membrana intatta e 
molecole che possono permeare, molecole permeabili e non troppo grandi. Quella che si usanon \`e fluorescente con un'estremti\`a di clorometile reattiva ai tioli e lo pu\`o fare in
quanto arrivando al mitocondrio viene ossidata quando la respirazione cellulare \`e attiva entra nei mitrocondri e proteine la possono coniugare ai tioli che la rendono fluorescente. 
Eccitata da 579 produce a 599. Le cellule dopo aver aggiunto il tracker le si incuba per mezzora a $37$ in quando temperatura ottimale di crescita e mantenimento. Visto affinch\`e 
questi coloranti possono essere in grado di penetrare le cellule devono essere messse nell'ambiente pi\`u favorevole. Dopo l'incubazione le cellule vanno fissate in quanto le cellule
hanno svolto la reazione e il colorante \`e attivato. Le cellule si fissano in modo da ottenere che le cellule rimangano intatte riducendo al minimo il danno alle strutture del campione.
Si usano composti organici come aldeidi in grado di conservare la struttura cellulare ($4\%$ di paraformaldeide), sono composti tossici per cui ogni volta che si usa la si deve usare
sotto cappa chimica e si usa una pipetta sierologica per rimuovere il terreno, uccedno ma mantenendo intatta la struttura. Si aspira totalmente il campione cercando di rimuovere il 
terreno, si aggiunge la soluzione di paraformaldeide e lo si fa utilizzando una micropipetta e va maneggiata sotto cappa chimica, la sterilit\`a non \`e necessario mantenerla. La 
cappa biologica \`e incapace di aiutare e prevenire il passaggio delle aldeidi da dentro a fuori si deve pertanto usare la cappa chimica. Il fissaggio \`e importante in quanto permette
di mantenere le cellule fisse e stabili senza alterare la struttura. Le cellule bloccate avranno strutture interne intatte, mitocondri e ER fluorescente. A questo punto per quanto 
riguarda colorazione WGA-594 per membrana cellulare le cellule son pronte, mentre per la falloidina si deve fare un altro proecsso in quanto la membrana non \`e permeabile a tutto. Per
permeabilizzare si deve utilizzare una miscela di saponi che servono a distruggere la membrana fosfolipidica SDS o tryphon che aiutano a formare pori di membrana e renderla pi\`u 
permeabile all'ingresso di alcune molecole. Se si vogliono usare anticorpi con immunofluorescenza usando anticorpi diretti per proteina o struttura sono molto pi\`u grandi dei 
fluorofori e per fare questo bisogna rendere la membrana cellulare pi\`u prona a farsi attraversare e per cui viene utilizzata la permeabilizzaizone. Per marcare la membrana cellulare
si usa WGA (wheat germa gglutinin) \`e una lectina che protegge il grano da insetti, lieviti e batteri, le lectine sono altamente specifiche per zuccheri presenti sulla superficie 
della membrana. Le lectine riconosco polisaccaridi presenti sulla membrna cellulare, Posoono essere sfrittate dai virus per riconoscere e attaccare alle membrane cellulari e infettare
la cellula. La WGA viene eccitata a 590 e emette a 617. Torna spesso colorazione rossa e verde in quanto si hanno tre filtri in laboratorio: blu verde e rosso in modo da avere tre
colorazioni nel campione. La falloidina \`e un grado di colorare il citoscheletro con pi\`u funzioni per sostenere la membrana citoplasmatica, ha altre funzioni come la trazione
sui cromosomi, sepearare la cellule, dirigere il traffico cellulare, pu\`o anche sostenere il movimento cellulare pu\`o fare in modo di estendere protrusioni, motilit\`a cellulare, 
assoni e dendriti. Avremmo evindenziato la F actina. Per farlo si usa la falloidina deriva dall'amanita phalloides con diversi anelli aromatici, reagisce stechiometricamente con l'actina
e sembra una molecla di actina e una di falloidina, molto ben visibila, \`e specie aspecifica e il legame con actina \`e specifico. In questo caso ha eccitazoine di 495 nm per emettere
in 518 nm, le colorazioni visualizzabili sono pertanto blu verde e rosso. La parte di colorazione \`e semplice dopo aver aggiunto la parafolmaldeide si toglie e aggiunge PBS, all'interno
della soluzione fisioligica ottenuta si aggiunggono i corolanti. In questo caso le colorazioni sono fotosensibili e si mette in incubazione coperte da stagnola per evitae la degradaizone
del campione. Attesi i 20 minuti di incubazione si deve recuperare dalla superficie cellulare il vetrino e montarlo sul vetrino portaoggetti. Il vetrino portaoggetti nei pozzetti non
ci sta, pertanto abbiamo usato quello coprioggetti, diametro piccolo ma abbastanza grande per permettere la visualizzazione. Le cellule messe a crescere, colorate e fissate, alcune
permeabilizzate e colorate per marcare membrana e citoscheletro per poterle visualizzare si deve spostare su vetrino in quanto i supporti di crescita permettono una visualizzazione in 
campo chiaro ma in caso di fluorescenza la plastica non \`e ottimale, si usa il vetro che d\`a meno dispersione ottica. La plastica d\`a una diffrazione non ottimale, oppure hanno
inventato delle piastre da 96 con fondo ottico similvetro. Quello che si fa \`e utilizzare i vetrini in quanto danno risoluzione migliore, per prendere le cellule serve fissare il
coprioggetto al portaoggetto, per fare questo si usa un mounting media fatto da sostanze che fungono da colla, liquide che solidificano senza alterare la struttura ottica del campione, 
all'interno dei quali si trova un colorante in grado di colorare il nucleo: il DAPI, un organello con doppia membrana e contiene il materiale genetico: cromosomi, istoni e nucleolo, 
il DAPI \`e in grado di colorare il nucleo. Si lega in maniera specifica a adenine e timine, i buchi neri sono nucleoli composti per la maggior paret dda proteine e RNA e non riconosce
proteine e ha bassa affinit\`a per il DNA e non li colora. Si lega al solco minore della dopia elica ed \`e stato integrato nel mounting media, fondamentale in quanto si deve usare il
microscopio con un olio per proteggere l'obiettivo per fare in modo che l'obiettivo possa scorrere in maniera sufficicente. Il mezzo di montaggio permette pertanto di fissare in modo 
molto forte il vetrino coprioggetto a quello portaoggetto si rischia altrimenti scivolamento che distrugga il campione. Un altra possibilit\`a \`e che gli obiettivi si avvicinano e 
allontanano per metterlo a fuoco, se non si presta attenzione essendo molto vicino l'olio pu\`o evitare che si vada troppo strettamente a contatto con il vetrino e lo si rompa. Il DAPI
\`e coniugato con il mezzo di montaggio, basta aggiungere il mezzo di montaggio e il DAPI nel momento in cui il mezzo di montaggio comincia a legare i vetrini il DAPI penetra nelle 
cellule e colora il nucleo, ha molti anelli aromatici ed ha la colorazione pi\`u visibile e quella che si usa per capire dove sono le celule. Viene eccitato a 358 ed emette a 461. Il DAPI
permette di analizzare le cellule a diverse fasi. Si pu\`o voler fare in modo che tutte le cellule stiano nella stessa fase e per fare questo bisogna sottoporre le cellule a uno stress:
togliendo dal terreno di coltura il siero le cellule smettono di crescere e si bloccano tutte in fase G1. Non l'avremmo fatto e avremmo potuto osservare cellule in diversi stati del 
ciclo cellulare. Il DAPI si trova in tutte le colorazioni in quanto evidenzia i nuclei e permette di contare le cellule

