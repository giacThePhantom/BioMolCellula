\chapter{Esperienza di laboratorio}
\section{Giorno 1 - esame di uno striscio di sangue}

	\subsection{Descrizione}
	Lo striscio di sangue periferico \`e un esame di laboratorio che serve a ottenere uno stato istantaneo della popolazione cellulare presente in una goccia di sangue. 
	L'esame si esegue strisciando una goccia di sangue sul vetrino. 

		\subsubsection{Dati raccolti}
		Lo striscio di sangue permette di valutare:
		\begin{itemize}
			\item Globuli rossi (eritrociti): trasportano l'ossigeno ai tessuti.
			\item Piastrine (trombociti): piccoli frammenti cellulari importanti per la formazione del coagulo.
			\item Globuli bianchi (leucociti): intervengono nella risposta immunitaria.
		\end{itemize}

		\subsubsection{Risvolti clinici}
		Esistono condizioni patologiche che influiscono sul numero, morfologia, funzionalit\`a e vita delle cellule del sangue. 
		Lo striscio di sangue periferico \`e ritenuto il miglior test per valutare e identificare in modo corretto anormalit\`a e immaturit\`a delle cellule del sangue. 
		Nel caso vengano evidenziate cellule anomale in modo significavo \`e possibile che il paziente sia affetto da una patologia e diventa necessario eseguire esami di approfondimento.

	\subsection{Allestimento di uno striscio}
	Per preparare uno striscio di sangue si agita la provetta contenente il sangue (e un anticoagulante per riuscire a mantenerlo stabile per una settimana) per sospendere la frazione corpuscolata in quella liquida e si prelevano $20\si{mL}$ di campione con una pipetta.
	Successivamente si deposita una piccola goccia in prossimit\`a della banda sabbiata di un vetrino pulito con etanolo $70\%$.
	Si appoggia un secondo vetrino (pulito allo stesso modo) sulla goccia, lo si inclina di $45\si{\degree}$ e si striscia, formando cos\`i una striscia uniforme.
	Infine si lascia fissare lo striscio all'aria sotto cappa per qualche minuto prima di essere colorato.
	Il sangue rimanente viene smaltito nei rifiuti biologici, taniche grigie sotto cappa contenenti un volume di candeggina al $10\%$.

	\subsection{Qualit\`a dello striscio}
	La qualit\`a di uno striscio viene valutata in base a tre caratteristiche:
	\begin{itemize}
		\item \`E necessaria una distribuzione cellulare uniforme.
		\item Le cellule devono mantenere le proprie caratteristiche il pi\`u possibile, anche se su monostrato.
		\item Non devono esserci artefatti (corpi esterni).
	\end{itemize}

	\subsection{Colorazione dello striscio}
	Lo striscio va colorato in modo da riuscire a distinguere le diverse cellule del sangue e i globuli bianchi tra di loro. 
	Il Diff-Quik \`e una colorazione citologica utile per raggiungere questo scopo.
	\`E composto da tre soluzioni:
	\begin{itemize}
		\item Fissativo a base di metanolo di colore verde chiaro.
		\item Soluzione $I$: eosina in tampone fosfato di colore rosso.
		\item Soluzione $II$: tiazina in tampone fosfato di colore blu.
	\end{itemize}

		\subsubsection{Eosina}
		L'eosina \`e una molecola acida con alta affinit\`a per costituenti cellulari con reazione basica.
		Ha come target in questo caso le proteine del citosol.
		Una porzione colorata di una sfumatura del rosso viene detta acidofila o eosinofila.

		\subsubsection{Coloranti tiazinici}
		I coloranti tiazinici sono molecole basiche con alta affinit\`a per costituenti cellulari con reazione acida.
		Hanno come target in questo caso nucleo, ribosomi ricchi di acidi nucleici, ER rugoso e la matrice cartilaginea.

		\subsubsection{Procedimento}
		Dopo aver fatto fissare lo striscio sotto cappa:
		\begin{enumerate}
			\item Lo si immerge nella soluzione fissativa per $30$ secondi, lo si sciacqua e lo si asciuga (appoggiando i bordi esterni del vetrino).
			\item Lo si immerge nella soluzione $1$ (eosina) per $1$ secondo $5$ volte, si sciacqua e si asciuga.
			\item Lo si immerge nella soluzione $2$ (tiazina) per $1$ secondo $5$ volte, si sciacqua, si tampona e si monta il vetrino.
		\end{enumerate}

			\paragraph{Montare il vetrino}
			Con montare il vetrino si intende appoggiare un vetrino copri-oggetto su di esso in modo da proteggere il preparato.
			Il vetrino copri-oggetto si fissa nei bordi e si fissa con smalto trasparente. 

	\subsection{Esame dei vetrini}
	Si inizia con osservare lo striscio con un piccolo ingrandimento in modo da valutare l'adeguatezza della distribuzione cellulare e della colorazione.
	Inoltre si osserva l'intero vetrino in modo da assicurarsi di non perdere popolazioni cellulari che si potrebbero concentrare ai margini.
	Si effettuano poi esami morfologici e conte delle cellule usando un ingrandimento maggiore a $100\times$.

		\subsubsection{Globuli rossi}
		Gli eritrociti o emazie sono le cellule pi\`u abbondanti nel sangue e sono specializzate nel trasporto dei gasi respiratori.
		Sono eosinofili a causa della presenza di emoglobina basica e sono valutati in base a forma, grandezza e colore (contenuto emoglobinico)
			
			\paragraph{Tipologie di globuli rossi}
			\begin{itemize}
				\item Mammiferi: anucleati, discoidali e biconcavi.
				\item Altri: nucleati, ellissoidali e biconvessi.
			\end{itemize}
			
			\paragraph{Anemia falciforme}
			Un esempio di patologia osservabile direttamente con lo striscio di sangue \`e l'anemia falciforme. 
			Questa patologia causa un'alterazione della forma dei globuli rossi, che passano da essere biconcavi ad avere una forma simile a una falce. 
			La malattia \`e dovuta ad un difetto genetico: una mutazione puntiforme da $GAG$ in $GUG$ sostituisce un acido glutammico con una valina. 
			La diffusione di questa malattia nelle popolazioni africane \`e dovuta alla resistenza che conferisce contro la malaria.
				
				\subparagraph{Sintomi}
				\begin{itemize}
					\item Anemia: emolisi pi\`u agevole, carenza cronica di globuli rossi. 
						Il paziente prova stanchezza, debolezza, mancanza di fiato, pallore, cefalea e difficolt\`a visive.
					\item Crisi dolorose: insorgono repentinamente e con durata variabile.
						Sono causate dalle occlusioni provocate dall'aggregazione dei globuli anomali che impediscono il flusso del sangue.
						I dolori sono al torace, addome o articolazione con alta frequenza.
					\item Sindrome mani-piede: estremit\`a degli arti gonfie: uno dei primi segnali dell'anemia falciforme nei bambini.
					\item Infezioni: dovute alle lesioni della milza dei globuli anomali.
					\item Ritardo dello sviluppo.
					\item Problemi della vista.
					\item Pelle fredda e gonfiore (edemi) di mani e piedi.
					\item Ittero.
				\end{itemize}

		\subsubsection{Piastrine}
		Le piastrine (trombociti nei non mammiferi) sono frammenti cellulari derivanti dai megacariociti. 
		Sono privi di nucleo ma dotati di membrana plasmatica. 
		Partecipano all'emostasi (chiusura delle lesioni formatesi nella parete dei vasi sanguigni) e alla coagulazione (formazione del coagulo o tappo piastrinico).
		Hanno una forma sferica o elissoidale e presentano una zona centrale intensamente colorata di blu o violetto.
		Delle piastrine si pu\`o valutare il numero e l'aspetto.
			
			\paragraph{Trombocitopenia}
			Si intende per trombocitopenia una carenza di piastrine nel sangue che aumenta il rischio di sanguinamento. 
			Si verifica quando il midollo osseo produce una quantit\`a insufficiente di piastrine, quando ne viene distrutto un numero eccessivo o quando si accumulano nella milza ingrossata.

		\subsubsection{Globuli bianchi}
		I leucociti sono cellule coinvolte nella risposta immunitaria.
		Sono facilmente osservabili e possono essere stimati il numero e tipo di cellule presenti. 
		Dei globuli bianchi si osserva forma, grandezza e aspetto generale. 
		Vengono classificati in cinque diversi tipi e se ne determina la percentuale relativa o conta differenziale. 

			\paragraph{Conta differenziale}
			La conta differenziale misura il numero di ogni tipo di globulo bianco determinando se le cellule sono presenti o meno in proporzione normale tra loro o se sono presenti cellule immature.

				\subparagraph{Formula leucocitaria}
				Si intende per formula leucocitaria la determinazione percentuale dei vari tipi cellulari di leucociti presenti nello striscio di sangue periferico.
				\begin{center}
					\begin{tabular}{|c|c|c|}
						\hline
						\makecell{Formula leucocitaria \\ valori normali nell'adulto} & Percentuali & Assoluti \\
						\hline
						Neutrofili & $40$-$70\%$ & \num{2000}-\num{8000}\si{\milli\metre\cubed} \\
						\hline
						Linfociti $B$ e $T$ & $25$-$55\%$ & \num{1500}-\num{5000}\si{\milli\metre\cubed} \\
						\hline
						Monociti & $2$-$10\%$ & \num{100}-\num{900}\si{\milli\metre\cubed} \\
						\hline
						Eosinofili & $0.5$-$6\%$ & \num{20}-\num{600}\si{\milli\metre\cubed} \\
						\hline
						Basofili & $0$-$2\%$ & \num{2}-\num{150}\si{\milli\metre\cubed} \\
						\hline
					\end{tabular}
				\end{center}
			
				\subparagraph{Informazioni ottenibili}
				La conta differenziale viene usata come parte della conta completa delle cellule del sangue e del check-up generale.
				Pu\`o essere utilizzata per diagnosticare e monitorare patologie e condizioni che colpiscono uno o pi\`u tipi di globuli bianchi.
				\`E un supporto nella diagnosi di patologie che colpiscono la produzione di certi tipi di globuli bianchi, informando su quale tipo sia basso o alto.
				Pu\`o fornire indizi sulla causa specifica di una risposta immunitaria, aiutando a determinare se l'infezione \`e causata da batteri o virus.
			
			\paragraph{Agranulociti - linfociti}
			I linfociti rappresentano il $20$-$30\%$ dei leucociti.
			Sono i globuli bianchi di dimensione inferiore e svolgono le principali funzioni effettrici. 
			Si dividono in linfociti $B$, $T$ e $NK$, ma le loro differenze non sono apprezzabili al microscopio.
			Il nucleo \`e sferico e ben evidente: occupa la maggior parte del volume cellulare.
			Il citoplasma circonda il nucleo in un sottile alone leggermente basofilo e con rare granulazioni azzurrofile.
				
				\subparagraph{Linfocitopenia}
				La linfocitopenia \`e la diminuzione patologica del numero di linfociti nel sangue.
				Pu\`o essere:
				\begin{itemize}
					\item Linfocitopenia acuta: il numero di linfociti pu\`o diminuire temporaneamente durante:
						\begin{itemize}
							\item Infezioni virali.
							\item Digiuno.
							\item Periodi di grave stress fisico.
							\item Uso di corticosteroidi (prednisone).
							\item Chemioterapia e radioterapia per un tumore.
						\end{itemize}
					\item Linfocitopenia cronica: il numero di linfociti pu\`o restare basso per un lungo periodo quando un soggetto \`e affetto da: 
						\begin{itemize}
							\item Patologie autoimmuni come il lupus eritematoso sistemico o l'artrite reumatoide.
							\item Infezioni cromiche come AIDS e tubercolosi miliare.
							\item Tumori come leucemie e linfomi.
						\end{itemize}
				\end{itemize}
				
				\subparagraph{Leucocitosi linfocitica}
				La leucocitosi linfocitica \`e l'aumento patologico del numero dei linfociti nel sangue. 
				La sua causa pi\`u comune \`e un'infezione virale.
				Quando si rileva un loro aumento, si esamina un campione al microscopio per determinare se i linfociti:
				\begin{itemize}
					\item Si presentano in forma attivata (infezioni virali).
					\item Appaiono immaturi o alterati (leucemie o linfomi).
				\end{itemize}
			\paragraph{Agranulociti - monociti}
			I monociti rappresentano il $3$-$8\%$ dei leucociti. 
			Quelli che circolano nel sangue periferico sono i precursori dei macrofagi tissutali o fagociti mononucleati. 
			Sono i globuli bianchi pi\`u voluminosi:
			\begin{itemize}
				\item Il nucleo in posizione eccentrica \`e voluminoso e reniforme.
				\item Nel citoplasma sono visibili granuli azzurrofili di piccole dimensioni.
			\end{itemize}

				\subparagraph{Monocitosi}
				Si intende per monocitosi un'elevata concentrazione dei monociti nel sangue. 
				Si verifica in caso di:
				\begin{itemize}
					\item Infezioni.
					\item Malattie ematologiche.
					\item Patologie autoimmuni.
					\item Condizioni particolari come splenectomia.
				\end{itemize}

			\paragraph{Granulociti - neutrofili}
			I neutrofili sono la componente cellulare pi\`u abbondante dei leucociti $50$-$70\%$. 
			Presentano attivit\`a fagocitaria essendo fagociti polimorfonuclati.
			Il nucleo \`e multilobato e ben visibile, mentre nel citoplasma si possono osservare numerosi granuli di piccole dimensioni con scarsa affinit\`a per i coloranti.
			Aumentano di numero in presenza di infezioni batteriche o disturbi infiammatori. 

				\subparagraph{Formula leucocitaria invertita}
				L'inversione della formula leucocitaria \`e la riduzione dei neutrofili associata all'aumento dei linfociti.
				A volte \`e costituzionale ma pu\`o anche essere causata da:
				\begin{itemize}
					\item Infezioni virali.
					\item Neoplasie.
					\item Disordini immunitari.
					\item Assunzione di alcuni farmaci.
				\end{itemize}

				\subparagraph{Leucocitosi neutrofila}
				La leucocitosi neutrofila \`e l'aumento del numero di neutrofili.
				La sua causa pi\`u comune \`e la normale risposta dell'organismo a un'infezione: i neutrofili sono i primi a rispondere in caso di infezioni batteriche, funghi (micosi) e protozoi.
				Pu\`o essere anche dovuta a:
				\begin{itemize}
					\item Lesioni.
					\item Disturbi infiammatori.
					\item Determinati farmaci.
					\item Alcune leucemie.
				\end{itemize}

				\subparagraph{Neutropenia}
				La neutropenia \`e caratterizzata da un numero patologicamente basso di neutrofili nel sangue.
				Se grave aumenta il rischio di contrarre un'infezione potenzialmente fatale e compare spesso come effetto collaterale di chemioterapia o radioterapia.

			\paragraph{Granulociti: basofili}
			I basofili rappresentano lo $0.5$-$1\%$ dei leucociti. 
			Hanno un nucleo bilobato o reniforme e nel citoplasma sono presenti granuli molto grandi con intensa colorazione basofila e metacromatica.
			Rilasciano mediatori come istamina, bradichina e serotonina in caso di infortunio o infezione, aumentando la permeabilit\`a capillare e il flusso di sangue nella zona interessata, favorendo l'arrivo delle altre molecole e cellule coinvolte.
			Sono inoltre coinvolti nelle reazioni allergiche e nei fenomeni di ipersensibilit\`a oltre a produrre eparina, sostanza fondamentale nel processo finale di coagulazione del sangue.


				\subparagraph{Basofilia}
				L'incremento del numero di basofili o basofilia si manifesta nei soggetti con ipotiroidismo e nelle malattie mieloproliferative. 
				Un esempio \`e la mielofibrosi, una malattia che fa s\`i che le cellule progenitrici delle cellule del sangue diventino cellule fibrose.

				\subparagraph{Basopenia}
				La diminuzione del numero di basofili o basopenia si manifesta nelle reazioni acute da ipersensibilit\`a e nelle infezioni.

			\paragraph{Granulociti: eosinofili}
			Gli eosinofili sono il $2$-$4\%$ dei leucociti, di cui meno del $1\%$ circola nel sangue, mentre la parte restante localizza nel midollo osseo rosso e nei tessuti.
			Il nucleo \`e generalmente bilobati con i lobi collegati da un sottile segmento di cromatina e il citoplasma \`e ricco di grossi granuli acidofili evidenziati dall'eosina. 

				\subparagraph{Eosinopenia}
				La carenza del numero di eosinofili o eosinopenia si manifesta nella sindrome di Cushing, nelle infezioni del torrente ematico (sepsi) e durante il trattamento con corticosteroidi. 
				Generalmente non causa problemi in quanto altre parti del sistema immunitario la compensano adeguatamente.

				\subparagraph{Eosinofilia}
				Un aumento del numero di eosinofili o eosinofilia si manifesta nei disturbi allergici o nelle infezioni parassitarie. 
				Anche alcuni tumori come il linfoma di Hodgkin e leucemia e malattie mieloproliferative possono causare eosinofilia.
		\subsubsection{Distinguere le cellule presenti nel sangue}
		\begin{center}
			\begin{tabular}{|c|c|}
				\hline
				Corpuscolo & Colore \\
				\hline
				Globuli rossi & Rosa, rosso, giallognolo \\
				\hline
				Piastrine & Viola, granuli blu \\
				\hline
				Neutrofili & Nuclei blu, citoplasma rosa, granuli violetti\\
				\hline
				Eosinofili &  Nuclei blu, citoplasma blu, granuli rossi\\
				\hline
				Basofili &  Nuclei viola o blu scuro, granuli violetti\\
				\hline
				Monociti &  Nuclei viola, citoplasma blu chiaro\\
				\hline
				Linfociti &  Nuclei viola, citoplasma blu chiaro\\
				\hline
			\end{tabular}
		\end{center}

\section{Giorno 2 - estrazione di proteine e DNA}
	
	\subsection{Cellule \emph{Hela}}
	Le cellule \emph{Hela} sono parte della prima linea cellulare creata.
	Sono cellule immortalizzate raccolte nel $1951$ dai tessuti di un cancro alla cervice uterina di Henrietta Lacks.
	Queste cellule possono osservarsi ad alta densit\`a o a bassa densit\`a in base alla risoluzione necessaria all'esperimento.
		
		\subsubsection{Cellule immortalizzate}
		Si intende per cellule immortalizzate cellule che, se vengono mantenute nell'ambiente appropriato \emph{in vitro} possono dividersi un indefinito numero di volte.
		Le cellule normali, invece hanno un numero massimo di divisioni possibili.
		
		\subsubsection{Stato di salute delle cellule}
		Prima di compiere degli esperimenti su di esse ci si deve assicurare del loro stato di salute. 
		Questo si controlla osservando la loro forma: ogni cellula in adesione alle piastre ha una forma particolare.
		Le cellule \emph{Hela} in particolare assumono forma trapezoidale e altre conformazioni indicano uno stato di stress.
		Per esempio in caso di infezione batterica le cellule si arrotondano e si osservano pallini neri che viaggiano per il terreno. 
		
		\subsubsection{Confluenza}
		Si intende per confluenza delle cellule quanto queste si trovano vicine tra loro. 
		Con il tempo questo valore aumenta in quanto le cellule si dividono e riempiono la superficie della piastra.
		Si pu\`o passare in un giorno dal $10\%$ al $90\%$ di confluenza.
		Un valore di confluenza ottimale si pone al $80\%$ in quanto le cellule sono confluenti ma non ammassate.
		\`E importante durante i controlli osservare tutta la piastra.


	\subsection{Estrazione di proteine}

		\subsubsection{Protocollo}
		\begin{enumerate}
			\item Osservazione delle cellule \emph{HeLa} al microscopio ottico (obiettivo $10\times$).
			\item Eliminazione del terreno della piastra con le cellule \emph{HeLa}, versandolo nel becker di vetro e lavando con $5\si{ml}$ di \emph{PBS} utilizzando pipetta pasteur.
			\item Aspirazione del resto di \emph{PBS} con pipetta $P1000$ mantenendo la piastra inclinata di $45\si{\degree}$.
			\item Aggiunta di $500\si{\micro\litre}$ di protein lysis buffer mantenendo la piastra sul ghiaccio.
			\item Stacco delle cellule con lo scraper e raccolta della sospensione con la $P1000$ trasferendola in una delle eppendorf.
			\item Centrifuga a \num{13000} \emph{rpm} per $10$ minuti a $4\si{\degree}$.
			\item Preparazione di una provetta eppendorf mettendola in ghiaccio.
			\item Trasferimento del surnatante (lisato proteico) nella provetta eppendorf in ghiaccio, senza toccare il pellet.
			\item Eliminazione della provetta eppendorf con il pellet.
		\end{enumerate}

		\subsubsection{Tampone di lisi - \emph{RIPA buffer}}
		Il \emph{RIPA} buffer \`e una soluzione tampone che viene utilizzata per l'estrazione di proteine da cellule di mammifero.
			
			\paragraph{Composizione}
			\begin{itemize}
				\item $50mM$ \emph{Tris-HCl}, $pH\ 7.4$: tampone per prevenire la denaturazione delle proteine.
				\item $150mM$ \emph{NaCl}: impedisce l'aggregazione proteica non specifica.
				\item $1\%$ \emph{NP-40}: detergente non ionico per estrarre le proteine, distruzione delle membrane.
				\item $0.1\%$ \emph{SDS}: detergente ionico per solubilizzare le proteine.
			\end{itemize}
			Si aggiungono inoltre cocktail inibitori contenenti:
			\begin{itemize}
				\item Inibitori delle proteasi: leupeptina o pepsatina
				\item DNAasi.
				\item RNAasi.
			\end{itemize}

		\subsubsection{Cell scraping}
		Il cell scraping \`e il processo di rimozione delle cellule dalla piastra.
		Per farlo si utilizza una sorta di spazzolino con setole di plastica semirigida che grattano via le cellule dalla superficie della piastra. 

		\subsubsection{Protocollo di quantificazione}
		\begin{enumerate}
			\item Aggiunta di $20\si{\micro\litre}$ di lisato proteico in una cuvetta.
			\item Trasferimento della cuvetta sotto cappa e aggiunta di $1\si{mL}$ di reagente di Bradford, chiudere con ``parafilm'' e mescolare la soluzione.
			\item Attesa di $2$ minuti.
			\item Impostazione dello spettrofotometro con standard method/single $\lambda$.
			\item Lettura dell'assorbanza dei campioni a $595\si{nm}$:
				\begin{enumerate}
					\item Misurazione del bianco composto da $20\si{\micro\litre}$ di buffer di estrazione e $1\si{mL}$ di reagente di Bradford.
						Elimina il segnale di fondo.
					\item Misurazione del campione.
				\end{enumerate}
			\item Eliminazione della cuvetta.
		\end{enumerate}

			\paragraph{Bradford - metodi colorimetrici}
			Essendo che proteine e DNA non assorbono nel visibile si rende necessario utilizzare metodi colorimetrici.
			Il saggio Bradford \`e basato sull'utilizzo del colorante Coomassie Brilliant Blue \emph{G-250}. 
			Il meccanismo di base \`e il legame del colorante a $pH$ acido con i residui basici di arginina, istidina, fenilanalina, triptofano e tirosina (a $pH$ basico)
				
				\subparagraph{Colorante libero}
				Il colorante libero in forma cationica presenta un massimo di assorbimento a $465\si{nm}$ e un colore rosso.
				
				\subparagraph{Legato a proteine}
				Dopo il legame con le proteine si osserva uno spostamento del massimo di assorbimento a $595\si{nm}$ a causa della stabilizzazione della forma anionica del colorante.
				Questo ora presenta un colore blu.

				\subparagraph{Vantaggi e svantaggi}\mbox{}
				\begin{multicols}{2}
					Vantaggi:
					\begin{itemize}
						\item Semplice da preparare.
						\item Immediato sviluppo del colore.
						\item Complesso stabile (osservazione possibile fino a $2$ ore dopo la creazione della soluzione).
						\item Sensibilit\`a elevata (fino a $22\frac{\si{\micro\gram}}{\si{mL}}$).
						\item Compatibilit\`a con la maggior parte di tamponi comuni, agenti denaturanti come guanidina \emph{HCl} $6M$ e urea $8M$ e preservanti come sodio azide.
					\end{itemize}
					\columnbreak
					Svantaggi:
					\begin{itemize}
						\item Il reagente colora le cuvette ed \`e difficile da rimuovere.
						\item La quantit\`a di colorante che si lega alle proteine dipende dal contenuto di amminoacidi standard, rendendo difficile la scelta di uno standard.
						\item Molte proteine non sono solubili nella miscela di reazione acida.
					\end{itemize}
				\end{multicols}

			\paragraph{Curva standard}
			La curva standard per la concentrazione proteica viene ottenuta utilizzando concentrazioni gi\`a note di albumina sierica bovina \emph{BSA}, proteina sierica che trasporta acidi grassi e importante per il mantenimento del $pH$ del plasma.
			La \emph{BSA} ha pertanto il ruolo di proteina di riferimento.
			
				\subparagraph{Creazione}
				Per creare una curva standard si misura l'assorbanza a concentrazioni crescenti sulla proteina di riferimento \emph{BSA} e si misura il bianco.
				Occorrono numerosi punti di osservazione (almeno $5$) e misure ripetute in doppio o triplo.
				Inoltre si nota come il bianco deve contenere tutti i reagenti tranne la sostanza da determinare.
				\begin{enumerate}
					\item Incubazione la soluzione per $5$ minuti.
					\item Lettura dell'assorbanza a $595\si{nm}$.
					\item Costruzione della retta di taratura.
				\end{enumerate}

			\paragraph{Retta di taratura}
			Si costruisce la retta di taratura mettendo i valori di assorbanza sull'asse delle $Y$ e i valori crescenti di concentrazione di \emph{BSA} sull'asse delle $X$. 
			Si ottiene pertanto una retta che mette in relazione concentrazione proteica e assorbanza.

			\paragraph{Quantificazione}
			Usando l'equazione fornita dalla retta di taratura ottenuta si pu\`o ricavare la concentrazione del campione incognito:
			\[y = a + bx\]
			Dove:
			\begin{multicols}{2}
				\begin{itemize}
					\item $y$ \`e il valore di assorbanza letto.
					\item $x$ \`e l'incognita, concentrazione delle proteine.
				\end{itemize}
			\end{multicols}
			La concentrazione delle proteine del campione sar\`a pertanto:
			\[x=\dfrac{y-a}{b}\]

			\paragraph{Range di lettura}
			\`E imporrante per avere letture precise rimanere in un certo range di concentrazione, ottimale da $0.1$ a $0.7$: se il blu \`e troppo brillante o troppo poco lo spettrofotometro non riesce ad essere preciso.
			Per ovviare a questo problema in caso di eccesso si fanno diluizioni del campione, prima $1:10$, poi $1:50$ e infine $1:100$.
			Queste concentrazioni andranno poi tenute in considerazione nel calcolo della concentrazione finale. 

	\subsection{Estrazione di DNA}

		\subsubsection{Protocollo}
		\begin{enumerate}
			\item Osservazione delle cellule \emph{HeLa} al microscopio ottico (obiettivo $10\times$).
			\item Eliminazione del terreno della piastra con le cellule \emph{HeLa}, versandolo nel becker di vetro e lavando con $5\si{ml}$ di \emph{PBS} utilizzando pipetta pasteur.
			\item Aspirazione del resto di \emph{PBS} con pipetta $P1000$ mantenendo la piastra inclinata di $45\si{\degree}$.
			\item Aggiunta di $1\si{\milli\litre}$ di lysis buffer.
			\item Stacco delle cellule con lo scraper e raccolta della sospensione con la $P1000$ trasferendola nel tubo con $3\si{mL}$ di acqua bidistillata.
			\item Aggiunta del restante $1\si{\milli\litre}$ di lysis buffer.
			\item Capovolgere il tubo $5$ volte, facendo attenzione e non agitare con forza.
			\item Aggiungere $300\si{\micro\litre}$ di proteasi $K$ al tubo.
			\item Capovolgere il tubo $5$ volte, facendo attenzione e non agitare con forza.
			\item Incubare il tubo a $50\si{\degree}$ per $10$ minuti nel bagnetto.
			\item Versare lentamente nel tubo $10\si{mL}$ di isopropanolo freddo mantenendo il tubo inclinato a $45\si{\degree}$.
			\item Incubare a temperatura ambiente per $5$ minuti.
			\item Capovolgere il tubo $5$ volte, facendo attenzione e non agitare con forza.
		\end{enumerate}

		\subsubsection{Tampone di lisi \emph{DNA}}
		Il tampone di lisi contiene un detergente capace di rompere la membrana cellulare fosfolipidica e la membrana nucleare rilasciando il DNA in soluzione.
		La soluzione contiene anche un agente tamponante per mantenere il $pH$ della soluzione in modo da preservare la stabilit\`a del DNA.
		Viene anche aggiunta una proteasi per rimuovere le proteine legate al DNA e distruggere enzimi cellulari che lo digerirebbero.
		L'estratto cellulare contenente la proteasi viene incubato a $50\si{\degree}$, temperatura ottimale per l'attivit\`a della proteasi.

		\subsubsection{Precipitazione o floculazione del \emph{DNA}}
			\paragraph{Metodo di precipitazione}
				\subparagraph{Sali}
				Il tampone di lisi contiene anche sali che rendono il DNA meno solubile nell'estratto cellulare.
				La carica negativa del DNA legata ai gruppi fosfato lo rende solubile.
				Quando un sale viene aggiunto al campione gli ioni sodio del sale sono attratti dalle cariche negative del DNA e le neutralizzano, permettendo alle molecole di DNA di unirsi tra di loro.
				
				\subparagraph{Alcol}
				L'aggiunta di alcol freddo precipita il DNA in quanto insolubile in presenza di un'alta concentrazione salina e di alcol.

			\paragraph{Visualizzare il DNA}
			Il DNA precipitato \`e visibile come fini filamenti bianchi al limite dello strato alcolico, mentre le altre sostanze cellulari rimangono in soluzione.
			Sono necessari migliaia di filamenti di DNA per formare una fibra grande abbastanza da essere visibile.
			Il DNA in caso di contaminazioni pu\`o assumere colorazione giallognola o rossiccia e non galleggiare nella soluzione. 
				
				\subparagraph{Contaminazioni}
				In caso di contaminazioni il DNA deve essere fatto precipitare di nuovo: si pelletta il floculo, si rimuove il surnatante, si aggiunge alcol freddo e si lascia incubare
				a basse temperature e lo si risospende in \emph{TE} (tris-edta) o acqua.
		
			\paragraph{Altre molecole}
			Le altre molecole rimangono in soluzione e non sono pertanto visibili.

\section{Giorno 3 - Colture cellulari}
Si intende per coltura cellulare, una coltura di cellule che deriva da un tessuto e fatta crescere in un ambiente artificiale a loro favorevole.
Dal tessuto di origine animale o vegetale si rimuove pertanto una porzione di cellule.
	
	\subsection{Disgregazione del tessuto}
	Il tessuto deve essere disgregato per generare un coltura.
	La disgregazione pu\`o essere:
	\begin{itemize}
		\item Enzimatica: enzimi proteolitici.
		\item Meccanica: disgregatori meccanici per rompere i legami cellula cellula.
	\end{itemize}

	\subsection{Coltura primaria}
	Si definisce coltura primaria lo stadio successivo all'isolamento delle cellule.
	\`E eterogenea in quanto i tessuti contengono diversi tipi di cellule.
	Il mantenimento \`e limitato in quanto le cellule non possiedono capacit\`a di replicazione infinita ma permette di osservare caratteristiche delle cellule in vivo.
	Queste proliferano in appropriate condizioni fino a che saturano il terreno e devono essere trasferite. 
	Il passaggio o subculture genera una linea cellulare. 

	\subsection{Linea cellulare}
	La linea cellulare \`e una coltura cellulare omogenea, costituit\`a ovvero da un solo tipo cellulare.
			
		\subsubsection{Finita}
		Le cellule si dividono e propagano solo per un numero limitato di volte prima di perdere l'abilit\`a di proliferare ed andare incontro a senescenza.

		\subsubsection{Continua}
		Le cellule si propagano in maniera indefinita in quanto sono diventate immortali spontaneamente (derivano da un tumore) in seguito a trasformazione con oncogeni virali o trattamenti chimici.
		Le cellule trasformate crescono molto velocemente ma perdono molte delle caratteristiche originali in vivo. 

	\subsection{Morfologia della coltura cellulare}

		\subsubsection{Colture in sospensione}
		Le colture in sospensione sono cresciute nel terreno di coltura in piccoli gruppi o solitarie.
		Crescono senza aderire.
		Sono tipicamente derivate dal sangue, non costituiscono tessuti solidi e sono libere di circolare nel fluido.

		\subsubsection{Colture in adesione}
		Le colture in adesione o su monostrato necessitando per crescere di superfici solide trattate con sostanze specifiche.
		Aderiscono alla superficie del contenitore.
		Queste cellule possiedono morfologie specifiche in base al tipo cellulare: allungate bipolari o multipolari. 
		
		\subsubsection{Forma}
		La forma delle cellule dipende dal tessuto dal quale si ricavano e pu\`o essere:
		\begin{itemize}
			\item Sferica per le cellule in sospensione.
			\item Allungata per cellula bipolari o multipolari come i fibroblasti.
			\item Poligonale con dimensioni pi\`u regolari come le cellule \emph{HeLa}.
		\end{itemize}

	\subsection{Et\`a delle colture}
	L'et\`a delle colture pu\`o essere indicata come numeri di passaggi (numero di volte in cui una linea cellulare viene sottocoltivata) o come livello di raddoppio.

		\subsubsection{Passaggi}
		Seminando le cellule in un substrato queste cominciano a replicarsi e a ricoprire tutta la superficie disponibile.
		Quando saturano l'ambiente vanno diluite altrimenti si bloccano nella fase $G_1$ e vanno incontro ad apoptosi. 
		Si rende necessario pertanto fare un passaggio: le si diluisce in un altro contenitore per dargli il tempo di crescere.
		Il tempo di passaggio dipende dalla velocit\`a di replicazione delle cellule. 

		\subsubsection{Livello di raddoppio}
		Si intende per livello di raddoppio il numero di volte totali in cui una popolazione cellulare ha raddoppiato a partire dal suo isolamento primario.
		\`E analogo al population doubling time, il tempo in cui una popolazione raddoppia.

	\subsection{Ottenere una coltura cellulare}
	Oltre alla messa a punto della coltura primaria in laboratorio si pu\`o acquistare da organizzazioni come \emph{ATCC} o richiederle da altri laboratori.

	\subsection{Applicazioni della coltura cellulare}
	Una coltura cellulare \`e uno dei maggiori strumenti della biologia cellulare e molecolare.
	\`E un eccellente sistema modello in vitro per studiare:
	\begin{multicols}{2}
		\begin{itemize}
			\item Normale fisiologia, biochimica e biologia.
			\item Effetto di farmaci e composti tossici.
			\item Mutagenesi e carcinogenesi.
			\item Sviluppo di nuovi farmaci e produzione di composti biologici.
			\item Terapia genica.
			\item Consulenza genetica.
		\end{itemize}
	\end{multicols}

	\subsection{Ambiente cellulare - terreno di coltura}
	Per coltivare una linea cellulare le cellule devono essere mantenute con un alto grado di sterilit\`a, devono ricevere i nutrienti e trovarsi a temperatura e $pH$ stabili.

		\subsubsection{Costituenti base}
		I costituenti base di un terreno di coltura sono le componenti e sostanze nutritive necessarie alla crescita delle cellule.
		\begin{multicols}{2}
			\begin{itemize}
				\item Sali inorganici: bilanciamento osmotico, adesione cellulare, mantenimento del potenziale di membrana.
				\item Cofattori enzimatici.
				\item Carboidrati come fonte di energia, glucosio e galattosio.
				\item Amminoacidi per la proliferazione cellulare.
				\item Vitamine per crescita e proliferazione cellulare
				\item Acidi grassi.
				\item Lipidi.
				\item Fattori di crescita e ormoni (siero fetale bovino \emph{FBS}).
				\item Assenza di virus o micoplasmi.
			\end{itemize}
		\end{multicols}
			\paragraph{Siero}
			Il siero \`e uno dei costituenti base del terreno di coltura, \`e un mix di albumine, fattori e inibiti di crescita che serve per:
			\begin{itemize}
				\item Crescita cellulare.
				\item Neutralizzazione di tossine.
			\end{itemize}
			Un siero \`e il siero bovino fetale \emph{FBS} che deve esser controllato in ogni batch per l'assenza di virus e micoplasmi. 
		
		\subsubsection{Mantenimento del $\mathbf{pH}$}
		Le cellule necessitano di un $pH$ neutro tra $7.2$ e $7.4$.
		La regolazione del $Ph$ avviene tramite due sistemi di tamponamento o buffering.

			\paragraph{Tamponamento naturale}
			Il bilanciamento avviene attraverso \emph{$CO_2$} atmosferica, nell'incubatore si trova a pressione parziale al $5$-$10\%$.
			Nel terreno si inserisce anche bicarbinato che si accoppa con \emph{$CO_2$} per creare un tampone.
			\`E poco costoso e non tossico.

			\paragraph{Tamponamento chimico}
			Si aggiunge \emph{HEPES} al terreno. 
			Questo garantisce una capacit\`a di tamponamento maggiore e non richiede atmosfera gassosa controllata.
			\`E costoso e tossico ad elevate concentrazioni.

			\paragraph{Rosso fenolo}
			Molti terreni ne contengono un identificatore come il rosso fenolo. 
			Questo cambia colore quando cambia il $pH$: con $pH$ neutro \`e rosso, ma diventa giallo con $pH$ acido e viola con $pH$ basico.
			In caso di cambio di colore si rende necessario cambiare il terreno. 
	
	\subsection{Contenitori e strumenti per le colture cellulari}
	Per far crescere le colture cellulari si rendono necessari diversi strumenti.
	
		\subsubsection{Contenitori per cellule}
		Si dividono in:
		\begin{itemize}
			\item Flask, con tappo ventilato per permettere lo scambio di \emph{$CO_2$}.
			\item Piastra di coltura con chiusura non ermetica.
			\item Microplate con chiusura non ermetica. 
				Le microplate permettono di osservare colture diverse in parallelo: verr\`a utilizzata quella a $6$ pozzetti per contenere le colture, di cui $4$ verrano colorate.
		\end{itemize}

		\subsubsection{Altri strumenti}
		\begin{multicols}{2}
			\begin{itemize}
				\item Tubi da $50$ e $15\si{mL}$.
				\item Microtubi da $0.5$, $1.5$ e $2\si{mL}$.
				\item Pipette sierologiche $1$, $2$, $5$, $10$, $25\si{mL}$.
				\item Pipette.
				\item Puntali.
				\item Micropipette a volume variabile.
			\end{itemize}
		\end{multicols}
	
	\subsection{Area di lavoro}

		\subsubsection{Cappa biologica}
		La cappa biologica \`e un'area di lavoro asettica che garantisce:
		\begin{itemize}
			\item Il contenimento di liquidi e aerosol infettivi che si generano durante una procedura di lavoro.
			\item Protezione da polvere e contaminanti microbici presenti nell'aria.
		\end{itemize}
		\`E presente un flusso costante unidirezionale di aria filtrata da filtri \emph{HEPA} sull'area di lavoro che \`e orizzontale e verticale per proteggere operatore e coltura cellulare.
			\paragraph{Utilizzo}
			Si disinfetta la superficie con etanolo $70\%$  e si asciuga con carta prima e dopo il lavoro.
			Si sterilizza prima e dopo con raggi \emph{UV}. 
			Si noti come \`e un'area di lavoro e non di conservazione.
			Non si deve coprire le griglie esterne in quanto sono quelle responsabili del mantenimento del flusso. 

		\subsubsection{Incubatore}
		L'incubatore \`e il luogo in cui vengono conservate le cellule in crescita.
		\`E in grado di mantenere la temperatura ottimale ($37\si{\degree}$)e una quantit\`a fissa di \emph{$CO_2$} $5\%$ a pressione parziale in grado di contrastare l'acidificazione dei terreni.
		Anche l'umidit\`a rimane stabile in modo da impedire l'evaporazione del terreno di coltura.

		\subsubsection{Contaminazioni}

			\paragraph{Contaminazioni chimiche}
			Possono essere origine di contaminazioni chimiche endotossine, ioni metallici, tracce di disinfettanti.
			La contaminazione chimica \`e difficile da rilevare.

			\paragraph{Contaminazioni biologiche}
			Se la coltura non viene lavorata correttamente pu\`o venir contaminata da batteri, funghi o microplasmi.
			Quando questo avviene non \`e pi\`u utilizzabile: questi organismi cambiano infatti le condizioni di crescita.
			Si nota la presenza di batteri grazie a caratteristici pallini nella coltura.
			Un altra contaminazione comune \`e quella da lievito, presente nell'aria.

			\paragraph{Evitare contaminazioni}
			La sorgente principale di contaminazioni \`e l'operatore.
			La cappa biologica dedicata a colture cellulari si trova in una stanza dedicata ad esse.
			L'operatore deve lavorare in maniera asettica:
			\begin{multicols}{2}
				\begin{itemize}
					\item Lavarsi le mani.
					\item Indossare guanti e camice.
					\item Disinfettare i guanti con etanolo $70\%$.
					\item Disinfettare la superficie prima e dopo il lavoro.
					\item Disinfettare tutto ci\`o che viene messo sotto cappa.
					\item Dispensare liquidi con pipette o pipettatore.
					\item Usare materiale sterile.
					\item Aprire pipette, scatole di puntali, fiasche, tubi e reagenti solo sotto cappa.
					\item Dopo aver aperto un contenitore posizionare il tappo all'in gi\`u.
					\item Richiudere ogni contenitore con cellule o reagenti appena possibile.
					\item Utilizzare antibiotici nel terreno
				\end{itemize}
			\end{multicols}

		\subsubsection{Norme di sicurezza}
		Un laboratorio di coltura cellulare presenta rischi specifici: manipolazione delle colture di cellule umane e animali e reagenti tossici, corrosivi e mutageni.
		\begin{itemize}
			\item Autoclavare tutti i rifiuti biologici solidi che vengono a contatto con cellule o reagenti.
			\item Raccogliere rifiuti liquidi dopo ciascun esperimento e trattare con candeggina. 
			\item Si assuma che le colture cellulari sono pericolose in quanto possono contenere virus o altri organismi.
			\item Leggere il safett data sheet dei reagenti.
			\item Non si mangia e non si beve.
			\item Si indossano  guanti e camice dedicato.
			\item Si usa materiale sterile.
			\item Si usano dispositivi appropriati per maneggiare liquidi contenenti cellule.
			\item Si lavora sotto cappa biologica.
			\item Si decontaminano le superfici di lavoro con disinfettante.
			\item Si lavano le dopo aver lavorato con le cellule e prima di lasciare il laboratorio.
		\end{itemize}

	\subsection{Mantenimento delle cellule}

		\subsubsection{Controllo delle cellule}
		Le cellule vanno controllate ogni giorno, osservando colore del terreno di coltura e la loro morfologia.
		In caso di terreno esausto lo si rimuove e se ne aggiunge di fresco.
		In presenza di stato di semi-confluenza delle cellule si rimuovono e si fa un passaggio o semina in altre fiasche.
			
		\subsubsection{Rimozione delle cellule - harvesting}
		In caso le cellule risultino sane si pu\`o staccarle dal substrato. 
			
			\paragraph{Rimozione meccanica}
			La rimozione meccanica delle cellule avviene attraverso pipetta o cell scraper.

			\paragraph{Rimozione con enzimi proteolitici}
			Per la rimozione con enzimi proteolitici si utilizzano enzimi che disgregano i legami cellula-cellula in quanto pone meno stress alle cellule.
			Esempi sono tripsina, collagenasi, accutasi o \emph{EDTA}.
			Questi causano il distacco delle cellule dalla superficie in maniera veloce ed efficace.
			La reazione di proteolisi pu\`o essere bloccata aggiungendo il terreno con siero.

	\subsection{Esercitazione}
	Si indossano guanti e camice e si utilizzano cellule \emph{HeLa} in fiasca $T$-$75$ per ogni gruppo.
	Nell'aula di microscopia vengono osservate le cellule al microscopio invertito in campo chiaro con obiettivo $2.5\times$ e $10\times$ e si valuta la confluenza cellulare.
	Sotto cappa si controllano i reagenti. 
		
		\subsubsection{Conta cellulare}
		\begin{enumerate}
			\item Aprire la fiasca con le cellule ed aspirare il terreno.
			\item Aggiungere $10\si{mL}$ di \emph{PBS}, chiudere la fiasca e muoverla per lavare la superficie interna.
			\item Aspirare il \emph{PBS} e aggiungere $1\si{mL}$ di tripsina.
			\item Mettere la fiasca a $37\si{\degree}$ nell'incubatore per $2$ minuti,
			\item Bloccare l'azione della tripsina aggiungendo $8\si{mL}$ di terreno di coltura.
			\item Spipettare la sospensione cellulare cercando di lavare tutta la superficie della fiasca.
			\item Raccogliere la sospensione cellulare e trasferirla nel tubo da $50\si{mL}$.
			\item Trasferire $20\si{\micro\litre}$ di sospensione con la pipetta in una vial e aggiungere $20\si{\micro\litre}$ di trypan blue. 
				Mescolare con la pipetta.
			\item Trasferire $10\si{\micro\litre}$ nella camera di Buerker.
			\item Contare le cellule di $3$ quadrati, fare la media a calcolare il numero di cellule per millilitro con:
				\[numero\ medio\ cellule \cdot fattore\ di\ diluizione \cdot \num{10000}\]
		\end{enumerate}
			
			\paragraph{Camera di Buerker}
			La camera di Buerker \`e un vetrino particolare in cui si posiziona la porzione di coltura. 
			Il vetrino portaoggetto va fissato con acqua in modo da impedirne movimenti.
			Il vetrino contiene delle scanalature che lo suddividono in $9$ quadrati.
			Quelli esterni sono a loro volta divisi in $16$ quadratini e quello centrale in $25$.
			Queste divisioni rendono pi\`u facile la conta.
			Per convenzione si contano anche le cellule presenti nel quadrato anteriore di destra e sui bordi.

			\paragraph{Trypan blue}
			Trypan blue \`e internalizzato dalle cellule morte, che pertanto diventano blu opaco. 
			Le cellule vive invece sono bianche e luminose e rifrangono la luce emessa dal microscopio.
			Essendo che diluisce la coltura introduce un fattore di diluizione di $2$.
		
		\subsubsection{Preparazione delle colture}
		\begin{enumerate}
			\item Aggiungere alle due colture dish $7\si{mL}$ di terreno e alla $6$-well plate $2\si{mL}$ per pozzetto.
			\item Seminare $1\cdot 10^6$ cellule in un culture dish per l'estrazione del DNA.
			\item Seminare $1\cdot 10^6$ cellule in un culture dish per l'estrazione delle proteine.
			\item Seminare $2\cdot 10^5$ cellule in ogni pozzetto della $6$-well plate.
			\item Indicare su ogni culture disc e sulla $6$-well plate il numero della linea cellulare, data e nome del gruppo.
			\item Mettere tutto nell'incubatore a $37\si{\degree}$.
		\end{enumerate}




\section{Giorno 4 - Colorazione di organelli citoplasmatici}
La colorazione di organelli citoplasmatici avviene attraverso coloranti a fluorescenza.

	\subsection{Fluorescenza}
	La fluorescenza \`e un caso particolare di luminescenza che si verifica per eccitazione: \`e la propriet\`a di alcune sostanze di rimettere a lunghezza d'onda maggiore le radiazioni elettromagnetiche ricevute.

		\subsubsection{Fluorocromi}
		I fluorocromi sono molecole che assorbono radiazione magnetica e sono eccitate da una luce di un colore ed emettono una luce di colore diverso con lunghezza d'onda maggiore.
		Quando si smette di eccitarli non continuano ad emettere luce.

			\paragraph{Struttura}
			Contengono tipicamente anelli aromatici che permettono la pi\`u intensa e utile emissione fluorescente molecolare.

			\begin{multicols}{2}
				\paragraph{Vantaggi}
				\begin{itemize}
					\item Molecole piccole.
					\item Vasta gamma di colori.
					\item Possibilit\`a di legarsi ad altre molecole.
				\end{itemize}
				
				\paragraph{Svantaggi}
				\begin{itemize}
					\item Tendenza a decadere d'intensit\`a nel corso dell'osservazione o photobleaching.
				\end{itemize}
			\end{multicols}

			\paragraph{Photobleaching}
			Si intende per fotobleaching la diminuzione della fluorescenza di un campione dovuta alla degradazione fotochimica del fluoroforo.
			Avviene quando un fluoroforo perde permanentemente la capacit\`a di fluorescere a causa di danneggiamenti chimici introdotti da fotoni e modifiche di legami covalenti.

				\subparagraph{Ridurre il photobleaching}
				Per ridurre il photobleaching si pu\`o:
				\begin{itemize}
					\item Diminuire l'intensit\`a dell'indicatore luminoso.
					\item Diminuire il tempo di esposizione all'indicatore luminoso.
				\end{itemize}

		\subsubsection{Microscopia a fluorescenza}
		La microscopia a fluorescenza \`e un tipo di microscopia ottica in cui il campione viene colorato con anticorpi fluorescenti o sostanze specifiche coniugate con fluorocromi.
		La luce utilizzata ha lunghezze d'onda specifiche per essere eccitate ed emetteranno nel visibile.

			\paragraph{Struttura del microscopio}
			Il microscopio possiede una lampada a sorgente luminosa che emette luce diversa a seconda di filtri.
			Questa luce a una lunghezza d'onda specifica eccita il fluoroforo che emette la luce catturata dall'oculare.

	\subsection{Scopo dell'esercitazione}
	Lo scopo dell'esercitazione \`e preparare colorazioni vitali e fissare le cellule in coltura per colorarne strutture intracellulari con fluorofori.
	Successivamente avviene l'osservazione di strutture intracellulari tipiche della cellula eucariote.
	La coltura cellulare proviene da cellule umane in coltura cresciute su vetrino.

	\subsection{Reticolo endoplasmatico}
	Il reticolo endoplasmatico \`e un sistema membranoso composto da vescicole, cisterne, sacculi e canalicoli con aspetto reticolare.
	Tra i diversi compartimenti il reticolo endoplasmatico \`e quello pi\`u esteso e arriva ad occupare fino al $90\%$ della totalit\`a delle membrane.

		\subsubsection{Composizione}
		Il reticolo endoplasmatico si divide in:
		\begin{itemize}
			\item Reticolo endoplasmatico ruvido $RER$.
			\item Reticolo endoplasmatico liscio $SER$.
			\item Reticolo endoplasmatico transizionale.
		\end{itemize}

		\subsubsection{Funzioni}
		Le funzioni del reticolo endoplasmatico sono:
		\begin{multicols}{2}
			\begin{itemize}
				\item Trasporto di proteine appena tradotte dall'insieme o pool di ribosomi della cellula.
				\item Glicosilazione di proteine e loro indirizzamento verso le sedi finali come Golgi e altri organelli o secrezione.
				\item Controllo e degradazione di proteine mal-ripiegate.
				\item Riserva di ioni calcio, utile in caso di processi come contrazione, apoptosi, iniziazione di processi cellulari ciclici, vie di segnalazione.
				\item Detossificazione di sostanze esterne come xenobiotici.
			\end{itemize}
		\end{multicols}

		\subsubsection{ERTracker}
		La glibenclamide \`e un farmaco utilizzato da pazienti diabetici per correggere l'iperglicemia.
		Fa parte della famiglia delle sulfaniloree.
		La molecola stimola e favorisce la secrezione di insulina da parte delle cellule $\beta$ del pancreas.
		Si lega al recettore \emph{SUR1} associato al canale $K^+$ \emph{ATP}-dipendente.
		Questi canali sono prominenti sul reticolo endoplasmatico.
		Questo colorante assorbe a $504\si{nm}$ nel blu ed emette a $511\si{nm}$ nel verde.

	\subsection{Mitocondri}
	I mitocondri sono organelli citoplasmatici delle cellule eucariote.

		\subsubsection{Composizione}
		Sono a forma di fagiolo con due membrane sovrapposte.
		La membrana interna forma creste che si ripiegano dentro ad una matrice.

		\subsubsection{Funzione}
		La funzione principale dei mitocondri \`e la produzione di energia sotto forma di \emph{ATP} dalla demolizione di carboidrati in ambiente aerobico.

		\subsubsection{MitoTracker}
		Il MitoTracker \`e una molecola permeabile alla membrana cellulare che contiene un'estremit\`a di clorometile reattiva ai tioli.
		Quando entra nel mitocondrio durante la respirazione cellulare viene ossidata e viene coniugata a tioli che la rendono fluorescente.
		Questo colorante assorbe a $579\si{nm}$ nel verde ed emette a $599\si{nm}$ nell'arancione.

	\subsection{Membrana citoplasmatica}

		\subsubsection{Wheat germ agglutinin \emph{WGA}}
		La membrana citoplasmatica viene marcata con \emph{WGA}, una lectina che protegge il grano da insetti, lieviti e batteri.
		\`E una proteina agglutinina e si lega al N-acetil-D-glucosammide.
		Questo colorante assorbe a $590\si{nm}$ nel verde ed emette a $617\si{nm}$ nel rosso.
			
			\paragraph{Lectine}
			Le lectine sono una famiglia di proteine altamente specifiche per determinati zuccheri.
			Svolgono un importante ruolo biologico nel processo di riconoscimento dei polisaccaridi sulle membrane cellulari.
			

	\subsection{Citoscheletro}

		\subsubsection{Composizione}
		Il citoscheletro \`e composto da polimeri di $G$-actina che si polimerizza in $F$-actina e altri polimeri.

		\subsubsection{Funzioni}
		Il citoscheletro:
		\begin{multicols}{2}
			\begin{itemize}
				\item Esercita la trazione sui cromosomi.
				\item Separa la cellula in divisione.
				\item Dirige il traffico intracellulare.
				\item Sostiene la membrana plasmatica.
				\item Crea collegamenti meccanici per la resistenza allo shock.
				\item \`E responsabile del movimento cellulare e dell'estensione delle protrusioni cellulari.
			\end{itemize}
		\end{multicols}

		\subsubsection{Falloidina}
		Il marcatore del citoscheletro \`e la falloidina che reagisce stechiometricamente con il citoscheletro e crea un legame specifico con actina.
		Il colorante assorbe a $495\si{nm}$ nel blu e trasmette a $518\si{nm}$ nel verde.
		
	\subsection{Nucleo}

		\subsubsection{Composizione}
		Il nucleo \`e un organello a doppia membrana e contiene il \emph{DNA} e nucleoli.

		\subsubsection{\emph{DAPI}}
		\emph{DAPI} o \emph{$4'$, $6'$-diammin-$2$-fenilindolo} si lega nel solco minore della doppia elica del \emph{DNA} (ad adenina e timina) attraverso un prolong antifade.
		Passa attraverso le membrane cellulari di cellule fissate e vitali.
		Il nucleolo non viene colorato in quanto ricco di proteine e RNA.
		Permette inoltre si determinare la fase del ciclo cellulare della cellula.
		Questo colorante assorbe a $359\si{nm}$ nel viola ed emette a $461\si{nm}$ nel blu.

	\subsection{Colorazione vitale}
		



			
\section{Laboratorio 4}
Lo scopo della preparazione \`e di effettuare colorazioni vitali con fluorofori specifici, si fissano che serve in quanto
le cellule vengono freezate nella condizione in cui sono e crea una fotografia perfetta della cellula e si osservano le strutture: membrana plasmatica, citoscheletro, mitocondri, reticolo
endoplasmatico e nucleo. \\

Le cellule dopo aver aggiunto il tracker le si incuba per mezzora a $37$ in quando temperatura ottimale di crescita e mantenimento. Visto affinch\`e 
questi coloranti possono essere in grado di penetrare le cellule devono essere messse nell'ambiente pi\`u favorevole. Dopo l'incubazione le cellule vanno fissate in quanto le cellule
hanno svolto la reazione e il colorante \`e attivato. Le cellule si fissano in modo da ottenere che le cellule rimangano intatte riducendo al minimo il danno alle strutture del campione.
Si usano composti organici come aldeidi in grado di conservare la struttura cellulare ($4\%$ di paraformaldeide), sono composti tossici per cui ogni volta che si usa la si deve usare
sotto cappa chimica e si usa una pipetta sierologica per rimuovere il terreno, uccedno ma mantenendo intatta la struttura. Si aspira totalmente il campione cercando di rimuovere il 
terreno, si aggiunge la soluzione di paraformaldeide e lo si fa utilizzando una micropipetta e va maneggiata sotto cappa chimica, la sterilit\`a non \`e necessario mantenerla. La 
cappa biologica \`e incapace di aiutare e prevenire il passaggio delle aldeidi da dentro a fuori si deve pertanto usare la cappa chimica. Il fissaggio \`e importante in quanto permette
di mantenere le cellule fisse e stabili senza alterare la struttura. Le cellule bloccate avranno strutture interne intatte, mitocondri e ER fluorescente. A questo punto per quanto 
riguarda colorazione WGA-594 per membrana cellulare le cellule son pronte, mentre per la falloidina si deve fare un altro proecsso in quanto la membrana non \`e permeabile a tutto. Per
permeabilizzare si deve utilizzare una miscela di saponi che servono a distruggere la membrana fosfolipidica SDS o tryphon che aiutano a formare pori di membrana e renderla pi\`u 
permeabile all'ingresso di alcune molecole. Se si vogliono usare anticorpi con immunofluorescenza usando anticorpi diretti per proteina o struttura sono molto pi\`u grandi dei 
fluorofori e per fare questo bisogna rendere la membrana cellulare pi\`u prona a farsi attraversare e per cui viene utilizzata la permeabilizzaizone.


La parte di colorazione \`e semplice dopo aver aggiunto la parafolmaldeide si toglie e aggiunge PBS, all'interno
della soluzione fisioligica ottenuta si aggiunggono i corolanti. In questo caso le colorazioni sono fotosensibili e si mette in incubazione coperte da stagnola per evitae la degradaizone
del campione. Attesi i 20 minuti di incubazione si deve recuperare dalla superficie cellulare il vetrino e montarlo sul vetrino portaoggetti. Il vetrino portaoggetti nei pozzetti non
ci sta, pertanto abbiamo usato quello coprioggetti, diametro piccolo ma abbastanza grande per permettere la visualizzazione. Le cellule messe a crescere, colorate e fissate, alcune
permeabilizzate e colorate per marcare membrana e citoscheletro per poterle visualizzare si deve spostare su vetrino in quanto i supporti di crescita permettono una visualizzazione in 
campo chiaro ma in caso di fluorescenza la plastica non \`e ottimale, si usa il vetro che d\`a meno dispersione ottica. La plastica d\`a una diffrazione non ottimale, oppure hanno
inventato delle piastre da 96 con fondo ottico similvetro. Quello che si fa \`e utilizzare i vetrini in quanto danno risoluzione migliore, per prendere le cellule serve fissare il
coprioggetto al portaoggetto, per fare questo si usa un mounting media fatto da sostanze che fungono da colla, liquide che solidificano senza alterare la struttura ottica del campione, 
all'interno dei quali si trova un colorante in grado di colorare il nucleo: il DAPI \`e stato integrato nel mounting media, fondamentale in quanto si deve usare il
microscopio con un olio per proteggere l'obiettivo per fare in modo che l'obiettivo possa scorrere in maniera sufficicente. Il mezzo di montaggio permette pertanto di fissare in modo 
molto forte il vetrino coprioggetto a quello portaoggetto si rischia altrimenti scivolamento che distrugga il campione. Un altra possibilit\`a \`e che gli obiettivi si avvicinano e 
allontanano per metterlo a fuoco, se non si presta attenzione essendo molto vicino l'olio pu\`o evitare che si vada troppo strettamente a contatto con il vetrino e lo si rompa. Il DAPI
\`e coniugato con il mezzo di montaggio, basta aggiungere il mezzo di montaggio e il DAPI nel momento in cui il mezzo di montaggio comincia a legare i vetrini il DAPI penetra nelle 
cellule e colora il nucleo, ha molti anelli aromatici ed ha la colorazione pi\`u visibile e quella che si usa per capire dove sono le celule.   Si pu\`o voler fare in modo che tutte le cellule stiano nella stessa fase e per fare questo bisogna sottoporre le cellule a uno stress:
togliendo dal terreno di coltura il siero le cellule smettono di crescere e si bloccano tutte in fase G1. Non l'avremmo fatto e avremmo potuto osservare cellule in diversi stati del 
ciclo cellulare. Il DAPI si trova in tutte le colorazioni in quanto evidenzia i nuclei e permette di contare le cellule

