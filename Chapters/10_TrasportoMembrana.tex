\chapter{Trasporto di membrana di piccole molecole e propriet\`a elettriche delle membrane}
A causa dell'interno idrofobico il bistrato della membrana cellulare limita il passaggio della maggior parte delle molecole porlari in modo da mantenere
concentrazioni di soluti nel citosol che differiscono da quelle del fluido extracellulare e in ogni compartimento intracellulare. Specifiche molecole solubili in acqua e ioni vengono
trasportati attraverso le membrane in modo da digerire nutrienti, espellere scarti e regolare concentrazioni ioniche intracellulari attraverso proteine di trasporto specializzate. 
\section{Principi del trasporto di membrana}
\subsection{Bistrati lipidici liberi da proteine sono impermeabili agli ioni}
Dato abbastanza tempo ogni molecola si diffonderebbe attraverso un bistrato lipidico senza proteine secondo il gradiente di concentrazione. Il tasso di diffusione varia enormemente in 
base alla dimensione della molecola e principalemnte in base all'idrofobicit\`a relativa. Pi\`u piccola la molecola e meno pi\`u idrofobica pi\`u si diffonde facilmente nel bistrato.
\subsection{Ci sono due classi di proteine di trasporto di membrana: trasportatori e canali}
Speciali proteine di membrana trasferiscono molecole polari attraverso la membrana. Tutte queste proteine sono a multipassaggio formando un passaggio attraverso la membrana permettono a
soluti idrofilisci specifici di attraversarla. I trasportatori e i canali sono le due classi principali. I trasportatori o permeasi legano il soluto che deve essre trasportato e fanno 
una serie di cambi conformazionali che espongono alternativamente siti di legami su un lato della membrana e trasferiscono il soluto sull'altra. I canali interagiscono con il soluto 
che deve essere trasportato pi\`u debolmente formando pori che si estendono per tutta la membrana e quando aprti permettoo il passaggio di soluti o molecole specifiche. 
\subsection{Il trasporto attivo \`e mediato da trasportatori accoppiati da una fonte di energia}
Tutti i canali e molti dei trasportatori permettono il passaggio dei soluti passivamente nel trasporto passivo guidato dal gradiente di concentrazione o dal potenziale di membrana che
insieme formano il gradiente elettrochimico. Quasi tutte le membrane hanno un potenziale negativo all'interno favorendo l'entrata di ioni positivi. Alcune proteine svolgono il trasporto
in contrasto con il gradiente elettrochimico nel trasporto attivo, mediato da trasportatori la cui attivit\`a di pompaggio \`e accoppiata da una fonte di energia metabolica. 
\section{Trasportatori e trasporto di membrana attivo}
Il processo di trasferiment odi una molecola attraverso il bistrato assomiglia a una reazione tra enzima e substrato. Il trasportatore non modifica il soluto ma lo consegna uguale 
all'altro lato della membrana. Ogni trasportatore possiede dei siti di legame specifici per il suo soluto e lo trasferisce attraverso cambi conformazionali reversibili che espongono 
tale sito alternativamente su ogni lato della membrana. Si trova uno stato intermetio in cui il soluto \`e inaccessibile da ogni parte della membrana. Quando \`e saturato il tasso di
trasporto \`e massimale che determina il tasso in cui il trasportatore pu\`o passare tra gli stati conformazionali. Ogni trasportatore possiede anche un'affinit\`a con il soluto, 
indicata come la concentrazione del soluto dove il tasso di trasporto \`e met\`a del valore massimo. Il trasporto attivo avviene attraverso trasportatori accoppiati che intrappolano
l'energia conservata in gradienti di concentrazione per guidare il trasporto sfavorevole, pompe guidate dall'ATP e pompe guidate da luce o redox. In molti casi ci sono similitudini
tra trasportatori attivi e passivi. 
\subsection{Il trasporto attivo pu\`o essere guidato da gradienti di concentrazione ionica}
Alcuni trasportatori mediano passivamente il movimento di un singolo soluto a un tasso determinato da $V_{max}$ e $K_m$ e sono detti uniporters. Altri funzionano come trasportatori
accoppiati in cui il trasferimento di un soluto dipende dal trasporto di un secondo nella stessa direzione svolto da symporters o in direzione opposta svolto da antiporters. Questo 
stretto accoppiamento permette di recuperare l'energia conservata nel gradiente elettrochimico di un soluto per trasportare l'altro. L'energia ligera rilasciata durante il movimento di
uno ione inorganico \`e usata per guidare la pompa dell'altro soluto contro il gradiente. Nella membrana plasmatica delle cellule animale lo ione co-trasportatore \`e tipicamente 
\ce{Na+} che quando entra nella cellula \`e traportato fuori da una pompa ad ATP per \ce{Na+-K+} che mantentendo il gradiente di \ce{Na+} guida il trasporto accoppiato. Tali 
trasportatori mediano il trasporto attivo secondario, mentre quelle guidate da ATP il primario. Questi trasportatori sono tipicamente formati da insiemi di $10$ o pi\`u $\alpha$-eliche
che attraversano la membrana. I siti di legame per gli ioni e per il soluto si trovano a met\`a della membrana dove alcune eliche sono rotte o distorte. Nelle conformazioni aperte 
i siti di legame sono accessibili solo da un lato della membrana. Nel passaggio tra le conformazioni la proteina adotta transientemente una conformazione chiusa che previene 
l'attraversamento non accompagnato. Possono lavorare in condizione inversa con gradienti appropriati: i trasportatori sono costruiti da ripetizioni inverse e sono detti pseudosimmetrici.
\subsection{Trasportatori nella membrana plasmatica regolano il pH citosilico}
La maggior parte delle proteine operano ottimalmente ad un pH particolare e si rende pertanto necessario regolare il pH dei compartimenti intracellulari. La maggior parte delle cellule
possiedono antiporter guidati da \ce{Na+} nella membrana plasmatica che aiutano a mantenere il pH citosilico a $7.2$. Usano l'energia del gradiente \ce{Na+} per pompare fuori \ce{H+} in
eccesso trasportandolo fuori direttamente o portando all'interno della cellula \ce{HCO3^-} per neutralizzare il primo. Uno scambiatore \ce{Cl}-\ce{HCO3^-} indipendente dal \ce{Na+} 
aggiusta il pH nel direzione opposta. Altri trasportatori nelle membrane intracellulari regolano il pH di compartimenti specifici.
\subsection{Una distribuzione asimemtrica dei trasportatori nelle cellule epiteliali sottosta al trasporto transcellulare dei soluti}
Nelle cellule epiteliali i trasportatori sono distribuiti non uniformemente nella membrana plasmatica e contribuiscono al trasporto transcellulare dei soluti muovendoli attraverso il 
fluido extracellulare in cui queste cellule si trovano fino al sangue. Symporter legati al \ce{Na+} si trovano nel dominio apicale delle cellule epitaliali e trasportano attivamente i 
nutrienti nella cellula, creando gradienti  di soluto elevati e uniporter nei domini basali e laterali permettono a questi nutrienti di lasciare la cellula passivamente. In tali cellule
l'area della membvrana plasmatica \`e aumentata da microvilli che aumentano la superficie assorbente. 
\subsection{Ci sono tre classi di pompe guidate dall'ATP}
Tali pompe sono dette ATPasi di traporto in quanto idrolizzano ATP in ADP e fosfato utilizzando l'energia liberata per pompare ioni o altri soluti. Ce ne sono di tre tipi:
\begin{itemize}
	\item Pompe P-tipo: sono strutturalmente e funzionalmente imparente con le proteine multipassaggo, devono il nome al fatto che fosforilano loro stesse, includono molte delle 
		pompe ioniche responsabili del mantenimento del gradiente di \ce{Na+}, \ce{K+}, \ce{H+} e \ce{Ca^{2+}}.
	\item Trasportatori APC (ATP-Binding Cassette transporters) differiscono dalle precedenti strutturalmente e pompano piccole molecole.
	\item Pompe V-tipo:: sono proteine simili a turbine con diverse subunit\`a, trasferiscono \ce{H+} in organelli come i lisosomi, vescicole sinaptiche e vacuoli acidificandone
		l'interno.
\end{itemize}
Molto imparentate con le pompe V-tipo sono le ATPasi F-tipo o ATP sintetasi in chanto usano il gradiente \ce{H+} per guidare la sintesi di ATP. Si trovano nella membrana plasmatica dei
batteri e in quella interna dei mitocondri. 
\subsection{Una ATPasi P-tipo pompa \ce{Ca^{2+}} nel reticolo sarcoplasmico nelle cellule muscolari}
Le cellule eucariotiche mantangono una concentrazione di \ce{Ca^{2+}} di $\sim 10^{-7}M$ nonostante una concentrazione extracellulare di $\sim 10^{-3}$, pertanto piccole variazioni 
esterne portano a una veloce variazione di gradiente. Tale flusso viene utilizzato per trasmettere rapidamente segnali extracellulari attraverso la membrana plasmatica. Si deve pertanto
mantenere un ripido gradiente di \ce{Ca^{2+}} attraverso trasportatori come il P-tipo \ce{Ca^{2+}} ATPasi e lo scambiatore \ce{Na+}-\ce{Ca^{2+}} guidato dal gradiente elettrochimico del
\ce{Na+}. La \ce{Ca^{2+}} ATPasi nel reticolo sarcoplasmatico (SR), una membrana nelle cellule di muscolo scheletrico fa parte della prima categoria. L'ST \`e un tipo di reticolo
endoplasmatico che forma una rete di sacche tubulari che servono come magazzini di \ce{Ca^{2+}}. Quano un azione potenzialmente depolarizzarebbe la membrana plasmatica della cellula
muscolare \ce{Ca^{2+}} viene rilasciato nel citosol dall'SR attraverso canali di rilascio che stimulano la contrazione del muscolo. La pompa \ce{Ca^{2+}} fa ritornare il \ce{Ca^{2+}} 
nell'SR. Le ATPasi P-tipo hanno strutture simili contenenti $10$ $\alpha$-eliche transmembrana connesse a tre dominici citosilici. Nella pompa dell'ST ci sono catene laterali 
di amminoacidi protrundenti dalle eliche che formano siti di legami per il \ce{Ca^{2+}}. Nello stato non fosforilato sono accessibili unicamente dal lato citosilico e il legame di 
\ce{Ca^{2+}} causa una serie di cambi confomrazionali che chiudono il passaggio e attivano una reazione di trasferimento del gruppo fosmfato in cui viene trasferito in un aspartato 
altamente conservato in utte le ATPasi di P-tipo. L'ADP si dissocia ed \`e ricambiato con un ATP causando un altro cambio conformazionale che apre un passaggio al lume dell'SR. I 
\ce{Ca^{2+}} che escono sono sostituiti da due ioni \ce{H+} e una molecola di acqua che stabilizzano il sito di legame e chiudono il passaggio verso il lume. L'idrolisi del legame
forforil-aspartato fa ritornare la pompa nella condizione iniziale. 
\subsection{La pompa \ce{Na+}-\ce{K+} della membrana plasmatica stabilisce gradienti \ce{Na+} e \ce{K+} attraverso la membrana plasmatica}
La pompa \ce{Na+}-\ce{K+} o ATPasi \ce{Na+}-\ce{K+} si trova nella membrana plasmatica di tutte le cellule animali e mantiene i gradienti di \ce{K+} e \ce{Na+}. Fa parte della famiglia
delle ATPasi P-tipo e opera come un antiporter guidato da ATP, pompando \ce{Na+} fuori dalla cellula contro il gradiente e \ce{K+} all'interno. Tale pompa genera il gradiente di \ce{Na+}
necessario per il trasporto della maggior parte dei nutrienti e regola il pH citosilico e pertanto un terzo dell'energia viene dedicata per questa pompa. Si dice elettrogenica in quanto
porta fuori tre ioni carichi positivamente e per ogni due che porta dentro creando una corrente elttrica netta attraverso la membrana e creando un potenziale elettrico con l'interno 
negativo. 
\subsection{I trasportatori ABC costituiscono la pi\`u grande famiglia di proteine di trasporto di membrana}
I trasportatori ABC contengono due domini ATPasi altamente conservati o cassetti leganti dell'ATP sul lato citosilico della membrana. Il legame con l'ATP unisce i due domini e l'idrolisi
causa la loro dissociazione. Questi movimenti sono trasmessi ai segmenti transmembrana guidando i cicli di cambi conformazionali che espongono siti di legame per il soluto su ogni lato
della membrana alternativamente. Utilizzano pertanto l'energia liberata con il legame dell'ATP e la sua idrolisi per guidare il trasporto attraverso il bistrato nella direzione 
determinata dal cambio conformazionale determinato dal sito di legame legato all'idrolisi. Sono utilizzati dai batteri per trasportatre piccole molecole e nutrienti. Nella maggior parte
delle cellule dei vertebrati un trasportatore ABC del reticolo endoplasmatico o trasportatore TAP pompa peptidi dal citosol nel lume ER che sono prodotti dalla degradazione delle 
proteine da parte dei proteosomi. Sono poi trasportati nella superficie cellulare dove vanno sotto lo scrutinio di linfociti T citotossici che uccidono la cellula se sono derivati da
virus o altri microorganismi. 
\section{Canali e le propriet\`a elettriche delle membrane}
I canali formano pori attraverso la membrana e una classe forma gap junctionts tra cellule adiacenti. Le porine formano pori permissivi in batteri, mitocondri e cloroplasti. Nella
membrana plasmatica di piant e animali i pori sono altamente selettivi e si chiudono rapidamente. Sono tipicamente detti canali ionici e non posono essere accoppiati a una fonte di 
energia per svolgere trasporto attivo. La loro funzione \`e permettere mil passaggio di ioni inorganici per diffonderli lungo il loro gradiente elettrochimico. 
\subsection{Le acquaporine sono permeabili all'acqua ma impermeabili agli ioni}
Essendo le cellule principalmente acqua il suo movimento attraverso le membrane \`e fondamentale alla vita. Le cellule contengono inoltre alce concentrazioni di soluti come molecole
organiche cariche negativamente confinate nella cellula (fixed anions) e i cationi che ne bilanciano la carica. Si crea un gradiente osmotico bilanciato da un gradiente opposto generato
da ioni inorganici nel fluido extracellulare. La forza osmotica rimanente tende a portare acqua nella cellula causandone il rigonfiamento fino a che le forze sono in equilibrio. A 
causa di questo tutte le membrane biolofiche sono permeabili all'acqua. Nelle cellule animali l'osmosi ha piccolo ruolo nella regolazione del volume in quanto il citoplasma \`e in uno
stato a gel e resiste grandi cambiamenti di volume. Oltre alla diffusione diretta alcuni eucarioti e procarioti possiedono canali acquosi o acquaporine nella membrana plasmatica che
permettono un rapido movimento d'acqua. Per evitare la distruzione di gradienti ionici devono permettere il passaggio di molecole d'acqua rapido e bloccare quello di ioni. Questo accade
grazie alla loro struttura tridimensionale. I canali hanno un poro stretto che permette il passaggio di una singola fila di molecole d'acqua. Sono impermeabili a \ce{H+}, presente 
principalmente nella forma \ce{H3O+} grazie a due arpargine che legano l'atomo di ossigeno della molecola di acqua imponendo una bipolari\`a all'intera catena. 
\subsection{Canali ionici sono selettivi degli ioni e fluttuano tra stato aperto e chiuso}
I canali ionici sono selettivi e suggerise che devono essere sottili abbastanza in posti per forzare ioni permeanti in contatto intipo con i canali del canale in modo che solo quelli
con la carica e dimensione appropriata possano passare. Devono liberarsi della maggior parte delle molecole d'acqua associata per passare. La parte sottile \`e detto filtro selettivo
e limita il tasso di passaggio e quando la concentrazione aumenta il flusso di ioni attraverso un canale aumenta proporzionalmente fino alla saturazione a un tasso massimo.

