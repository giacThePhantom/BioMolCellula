\chapter{Trasporto di membrana di piccole molecole e propriet\`a elettriche delle membrane}
A causa dell'interno idrofobico il bistrato della membrana cellulare limita il passaggio della maggior parte delle molecole porlari in modo da mantenere
concentrazioni di soluti nel citosol che differiscono da quelle del fluido extracellulare e in ogni compartimento intracellulare. Specifiche molecole solubili in acqua e ioni vengono
trasportati attraverso le membrane in modo da digerire nutrienti, espellere scarti e regolare concentrazioni ioniche intracellulari attraverso proteine di trasporto specializzate. 
\section{Principi del trasporto di membrana}
\subsection{Bistrati lipidici liberi da proteine sono impermeabili agli ioni}
Dato abbastanza tempo ogni molecola si diffonderebbe attraverso un bistrato lipidico senza proteine secondo il gradiente di concentrazione. Il tasso di diffusione varia enormemente in 
base alla dimensione della molecola e principalemnte in base all'idrofobicit\`a relativa. Pi\`u piccola la molecola e meno pi\`u idrofobica pi\`u si diffonde facilmente nel bistrato.
\subsection{Ci sono due classi di proteine di trasporto di membrana: trasportatori e canali}
Speciali proteine di membrana trasferiscono molecole polari attraverso la membrana. Tutte queste proteine sono a multipassaggio formando un passaggio attraverso la membrana permettono a
soluti idrofilisci specifici di attraversarla. I trasportatori e i canali sono le due classi principali. I trasportatori o permeasi legano il soluto che deve essre trasportato e fanno 
una serie di cambi conformazionali che espongono alternativamente siti di legami su un lato della membrana e trasferiscono il soluto sull'altra. I canali interagiscono con il soluto 
che deve essere trasportato pi\`u debolmente formando pori che si estendono per tutta la membrana e quando aprti permettoo il passaggio di soluti o molecole specifiche. 
\subsection{Il trasporto attivo \`e mediato da trasportatori accoppiati da una fonte di energia}
Tutti i canali e molti dei trasportatori permettono il passaggio dei soluti passivamente nel trasporto passivo guidato dal gradiente di concentrazione o dal potenziale di membrana che
insieme formano il gradiente elettrochimico. Quasi tutte le membrane hanno un potenziale negativo all'interno favorendo l'entrata di ioni positivi. Alcune proteine svolgono il trasporto
in contrasto con il gradiente elettrochimico nel trasporto attivo, mediato da trasportatori la cui attivit\`a di pompaggio \`e accoppiata da una fonte di energia metabolica. 
\section{Trasportatori e trasporto di membrana attivo}
Il processo di trasferiment odi una molecola attraverso il bistrato assomiglia a una reazione tra enzima e substrato. Il trasportatore non modifica il soluto ma lo consegna uguale 
all'altro lato della membrana. Ogni trasportatore possiede dei siti di legame specifici per il suo soluto e lo trasferisce attraverso cambi conformazionali reversibili che espongono 
tale sito alternativamente su ogni lato della membrana. Si trova uno stato intermetio in cui il soluto \`e inaccessibile da ogni parte della membrana. Quando \`e saturato il tasso di
trasporto \`e massimale che determina il tasso in cui il trasportatore pu\`o passare tra gli stati conformazionali. Ogni trasportatore possiede anche un'affinit\`a con il soluto, 
indicata come la concentrazione del soluto dove il tasso di trasporto \`e met\`a del valore massimo. Il trasporto attivo avviene attraverso trasportatori accoppiati che intrappolano
l'energia conservata in gradienti di concentrazione per guidare il trasporto sfavorevole, pompe guidate dall'ATP e pompe guidate da luce o redox. In molti casi ci sono similitudini
tra trasportatori attivi e passivi. 
\subsection{Il trasporto attivo pu\`o essere guidato da gradienti di concentrazione ionica}
Alcuni trasportatori mediano passivamente il movimento di un singolo soluto a un tasso determinato da $V_{max}$ e $K_m$ e sono detti uniporters. Altri funzionano come trasportatori
accoppiati in cui il trasferimento di un soluto dipende dal trasporto di un secondo nella stessa direzione svolto da symporters o in direzione opposta svolto da antiporters. Questo 
stretto accoppiamento permette di recuperare l'energia conservata nel gradiente elettrochimico di un soluto per trasportare l'altro. L'energia ligera rilasciata durante il movimento di
uno ione inorganico \`e usata per guidare la pompa dell'altro soluto contro il gradiente. Nella membrana plasmatica delle cellule animale lo ione co-trasportatore \`e tipicamente 
\ce{Na+} che quando entra nella cellula \`e traportato fuori da una pompa ad ATP per \ce{Na+-K+} che mantentendo il gradiente di \ce{Na+} guida il trasporto accoppiato. Tali 
trasportatori mediano il trasporto attivo secondario, mentre quelle guidate da ATP il primario. Questi trasportatori sono tipicamente formati da insiemi di $10$ o pi\`u $\alpha$-eliche
che attraversano la membrana. I siti di legame per gli ioni e per il soluto si trovano a met\`a della membrana dove alcune eliche sono rotte o distorte. Nelle conformazioni aperte 
i siti di legame sono accessibili solo da un lato della membrana. Nel passaggio tra le conformazioni la proteina adotta transientemente una conformazione chiusa che previene 
l'attraversamento non accompagnato. Possono lavorare in condizione inversa con gradienti appropriati: i trasportatori sono costruiti da ripetizioni inverse e sono detti pseudosimmetrici.
\subsection{Trasportatori nella membrana plasmatica regolano il pH citosilico}
