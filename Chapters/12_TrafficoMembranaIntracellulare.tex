\chapter{Traffico di membrana intracellulare}
Le cellule per nutrirsi, comunicare e rispondere a cambi nell'ambiente aggiustano la composizione della loro membrana plasmatica e compartimenti interni in risposta alle loro 
necessit\`a. Usano un sistema di membrane interne per aggiungere e rimuovere proteine di superficie come recettori, canali ionici e trasportatori. Attraverso esocitosi i cammini 
secretori portano nuovi proteine, carboidrati e lipidi sintetizzati alla membrana plasmatica o allo spazio extracellulare. Attraverso endocitosi rimuovono componenti della membrana 
plasmatica e li portano verso endosomi dove posano essere riportati a regioni della membrana plasmatica o verso i lisosomi per la degradazione. L'esocitosi viene utilizzata per 
catturare nutrienti. Il lume di ogni compartimento lungo i cammini secretori o endocitici \`e equivalente al lume di altri compartimenti o all'esterno cellulare. I contenitori
di trasporto sono formati dal compartimento donatore e da vescicole sferiche piccole o altre larghe e irregolari o tubuli, riferite come vescicole di trasporto. In una cellula 
eucariotica le vescicole di trasporto si separano da una membrana per unirsi a un'altra trasportando componenti di membrana e molecole lumenali dette cargo. Il traffico vescicolare
fluisce lungo cammini organizzati che permettono secrezione, nutrimento e rimodellamento della membrana plasmatica e degli organelli. Il cammino secretorio porta fuori dal reticolo 
endoplasmatico verso l'apparato di Golgi e la superficie cellulare, mentre il cammino endocitico porta all'interno dalla membrana plasmatica. In ogni caso cammini di riporto equilibrano
i flussi delle membrane nelle direzioni opposte riportando indietro certe molecole. Per far questo sono necessarie vescicole di trasporto selettive. 
\section{I meccanismi molecolari di trasporto di membrana e il mantenimento della diversit\`a compartimentale}
Il trasporto attraverso vescicole media uno scambio continuo di componenti tra i pi\`u di $10$ compartimenti chimicamente distinti che comprendono i cammini secretori ed endocitici. 
Ogni compartimento \`e capace di mantenere la propria identit\`a grazie alla composizione della membrana: marcatori sul lato citosilico servono come guida per il traffico in arrivo e 
una loro combinazione d\`a ad ogni compartimento il suo indirizzo molecolare. Le operazioni di separazione e trasferimento di parti specifiche delle membrane permettono di mantenere
alte o basse concentrazioni di marcatori su un compartimento. 
\subsection{Ci sono vari tipi di vescicole incapsulate}
La maggior parte delle vescicole si forma da regioni di membrana specializzate. Si separano come vescicole incapsulate, con una specifica gabbia proteica che copre la superficie 
citosilica. Prima che le vescicole si fondano con una membrana questa gabbia viene eliminata. La capsula ha una struttura a due strati che riflette le due funzioni: uno strato interno
concentra proteine di membrana in superfici specializzate che danno origine alla membrana vescicolare in modo da selezionare le molecole di membrana appropriate per il trasporto. Lo
strato esterno si assembla in un lattice curvo che deforma la superficie di membrana e d\`a forma alla vescicola. Ci sono tre tipi di vescicole incappucciate: incapsulata da clatrina, da
COPI e da COPII. Ogni tipo \`e utilizzata per diversi passi del trasporto. Le vescicole incapsulate da clatrina mediano il trasporto dall'apparato di Golgi e dalla membrana plasmatica,
mentre quelle incapsulate da COPI e da COPII il trasporto dall'ER e dalle cisterne di Golgi. 
\subsection{L'assemblaggio di una capsula di clatrina guida la formazione della vescicola}
Le vescicole incapsulate da clatrina trasportano materiale dalla membrana plasmatica e tra compartimenti endosomiali e di Golgi. Quelle incapsulate da COPI e COPII trasportano materiale
presto nei cammini secretori: la prima si separa dai compartimenti di Golgi, mentre la seconda dall'ER. Il maggior componente proteico delle vescicole incapsulate da clatrina \`e la 
clatrina che forma lo strato esterno. Ogni subunit\`a clatrinica consiste di tre grandi e tre piccole catene polipeptidiche che formano una struttura a tre gambe detta triskelion che
a sua volta si assembla in una struttura di esagoni e pentagoni simile a un cesto che forma pozzi incapsulati sulla superficie citosilica della membrana. Sotto appropriate condizioni
i triskelioni isolati si assemblano in gabbie poliedriche e determinano la geometria della gabbia clatrinica.
\subsection{Proteine adattatrici selezionano il cargo nelle vescicole incapsulate a clatrina}
Le proteine adattatrici sono un altro componente della capsula formano uno strato interno della capsula tra la gabbia clatrinica e la membrana. Legano la gabbia alla membrana e 
intrappolano varie proteine transmembrana come recettori che capturano le molecole di cargo all'interno della vescicola e permettono la selezione di un insieme di proteine trasmembrana
e delle proteine che interagiscono con loro e le impacchettano all'interno della nuova vescicola di trasporto. Ci sono diversi tipi di proteine adattatrici, quelle meglio caratterizzate
possiedono quattro subunimt\`a proteiche, altre proteine a singola catena. Ogni tipo \`e specifico a un insieme di recettori di cargo. L'assemblaggio di proteine adattrici \`e 
strettamente controllao dalla loro interazione con altre componenti della capsula. AP2 quando si lega a un lipide fosfatidiloinositolo fosforilato altera la sua conformazione esponendo 
siti di legame per recettori di cargo nella membrana. Il legame simultaneo ai recettori di cargo e al gruppo di testa lipidico aumento il legame di AP2 alla membrana. La proteina
agisce come un individuatore di coincidenze che si assembla solo al tempo e spazio giusto. Quando si legano inducono curvatura di membrana che rende causa un circuito a feedback 
positivo amplificato dal legame della clatrina che porta alla formazione e separazione della vescicola. 
\subsection{I fosfonositidi marcano gli organelli e i domini di membrana}
I fosfolipidi inositoli hanno un'importante funzione regolatoria in quanto possono svolgere rapidi cicli di fosforilazione e defosforilazione alle posizioni $3'$, $4'$ e $5'$ dei gruppi
di testa inositolo e produrre vari tipi di fosfoinositidi (PIP). L'interconversione di fosfatidilinositolo e PIP \`e diversa tra i compartimenti e la distribuzione, regolazione e 
equilibrio  determina lo stato di distribuzione di ogni specie di PIP che varia tra gli organelli e spesso tra regioni continue di membrana. Molte proteine coinvolte nel trasporto 
vescicolare contengono domini che si legano  ai gruppi di testa di particolari PIP. Il controllo locale di PI, PIP chinasi e PIP fosfatasi pu\`o essere usato per controllare il legame 
di proteine a una membrana. La produzione di un PIP recluta proteine contenenti domini leganti a PIP corrispondenti che aiutano a regolare la formazione di vescicole e altri passi nel
loro traffico. 
\subsection{Proteine piegatrici della membrana aiutano a deformare la membrana durante la formazione delle vescicole}
La forza generata dall'assemblaggio della clarina non \`e sufficiente per dare forza e separare la vescicola dalla membrana: altre proteine partecipano a ogni passo del processo:
proteine piegatrici della membrana che contenono  domini a mezzaluna o domini BAT legano e impongono la loro forma alla membrana sottostante attraverso interazioni elettrostatiche e 
aiutano AP2 a nucelare endocitosi dando una forma alla membrana plasmatica in modo che la vescicola si formi. Contengono eliche anfipatiche che inducono una curvatura di membrana
dopo essere state inserite nel lato citoplasmatico della membrana.
\subsection{Proteine citoplasmatiche regolano la separazione e decapsulamento delle vescicole}
Mentre la bolla incapsulata da clatrina cresce la dinamina si assembla al collo di ogni bolla. Dinamina contiene un dominio che si lega a $PI(4,5)$ $P_2$ che unisce la proteina alla
membrana e un dominio GTPasi che regola il tasso di separazione delle vescicole dalla membrana. IL processo di separazione unisce i due lati non citosilici della membrana in prossimit\`a
e li fonde isolando la vescicola che si sta formando. La dinamina recluta altre proteine al collo della bolla che aiutano a riparare la membrana distorcendo la struttura bistrato o 
cambiando la sua composizione lipidica o entrambe le cose. Una volta rilasciata la vescicola perde la capsula di clatrina: una fosfatasi PIP co-impacchettata nelle vescicole esaurisce
$PI(4, 5$ $P_2$ dalla membrana che indebolisce il legame delle proteine adattatrici. Oltre a questo una proteina hsp70 accompagnatrice che funziona come un'ATPasi decapsulatrice, usando
l'idrolisi dell'ATP. Il rilascio non deve avvenire prematuramente e si rendono necessari controlli ausiliari in modo che non venga rimossa prima che abbia formato una vescicola completa.
\subsection{GTPasi monomerica controlla l'assemblaggio della capsula}
Per equilibrare il traffico da e per un compartimento le proteine di capsula si devono assemblare solo quando necessario. Se la produzione locale di PIPs ha un ruolo centrale nella
regolazione dell'assemblaggio delle capsule di clatrina sulla membrana plasmatica e sugli apparati di Golgi la cellula superimpone modi addizionail di regolare la formazione della
capsula. GTPasi di reclutamento della capsula controllano l'assemblaggio di capsule di clatrina sugli endosomi e le capsule CPOI e COPII sulle membrane ER e di Golgi. Molti passaggi
del trasporto vescicolare dipendono da una variet\`a di proteine che si legano a GTP regolano la maggior parte dei processi nelle cellule eucariotiche. Agiscono come interruttori 
molecolari tra due stati con quello attivo con GTP legato e inattivo con GDP legato. Il cambio di stato \`e regolato da fattori di scambio del nucleotide guanina (GEF) che attivano la
proteina catalizzando lo scambio di GDP per GTP e le proteine attivatrici GTPasi disattivano la proteina causando l'idrolisi del GTP in GDP. Le GTPasi di reclutamento della capsula
sono membri di una famiglia di GTPasi monomeriche come le proteine ARF, responsabili per l'assemblaggio di COPY e della capsula clatrinica alle membrane di Golgi e la proteina Sra1, 
responsabile per l'assemblaggio delle capsule COPII alla membrana ER. Si trovano ad alte concentrazioni nel citosol in stato inattivo. Quando una vescicola incapsulata in COPII deve
separarsi dalla membrnaa ER una Sar1-GEF lega al Sar1 citosilico causando il suo cambio di GDP in GTP. In questo stato Sar1 espone un'elica anfipatica che si iniferice nel lato
citoplasmatico del bistrato della membrana ER. Sar1 ora strettamente legata recluta subunit\`a di proiteine adattatrici alla membrana ER per iniziare la separazione. Le GTPasi
di reclutamento della capsula hanno anche un ruolo nel disassemblaggio della capsula: l'idrolisi di GTP causa un cambio conformazionale che causa un'uscita della sua coda idrofobica
dalla membrana causando un disassemblaggio della capsula. COPII accellera l'idrolisi del GTP in Sar1 e una vescicola completamente formata viene prodotta solo quando la formazione
del sistema di separazione avviene pi\`u velocemente rispetto al processo di disassemblaggio temporizzato. Una volta che la vescicola si separa l'idrolisi del GTP rilascia Sar1 ma
la capsula isolata \`e abbastanza stabilizzata da interazioni cooperative in modo che sia presente fino a che la vescicola arrivi alla membrana obiettivo dove una chinasi fosforila
la capsula proteica che la disassembla e prepara la vescicola per la fusione. Le vescicole incapsulate da clatrina e da COPI perdono la capsula poco dopo che si separano. Per le seconde
la curvatura della membrana della vescicola causa l'inizio del decapsulamento. Un ARF-GAP viene reclutato alla COPI mentre si assembla e interagisce con la membrana percependo 
la densit\`a di impacchettamento dei lipidi. Si attiva quando la curvatura raggiunge quella di una vescicola di trasporto e disattiva ARF, causando il disassemblaggio della capsula.
\subsection{Non tutte le vescicole di trasporto sono sferiche}
Ogni membrana risponde diversamente alla creazione di vescicole: nella membrana plasmatica, piatta e rigida a causa del colesterolo, l'azione coordinata di capsule di clatrina e
di proteine che piegano la membrana deve produrre forza sufficiente per l'introduzione di curvatura, specialmente al collo. Per le membrane intracellulari avviene invece in regioni
gi\`a curve dove la funzione primaria della capsula \`e catturare il cargo appropriato. Le vescicole di trasporto sono create in varie forme e dimensioni. Molte vescicole COPII sono
richieste per il trasporto di grande molecole di cargo come il collagene che non entra nelle normali vescicole: la molecola di cargo si lega a proteine di impacchettamento nell'ER che
guidano l'assemblaggio di vescicole COPII molto pi\`u grandi che permettono l'entrata di un cargo molto pi\`u grande. Negli endosomi e nella rete di Golgi trans si trovano lunghi tubuli
che reclutano cargo e si separano con un grande rapporto superficie/volume e sono ricchi di proteine di membrana. 
\subsection{Le proteine Rab guidano le vescicole di trasporto alle loro membrane obiettivo}
Le vescicole di trasporto sono estremamente selettive verso la membrana obiettivo, caratteristica assicurata in quanto tutte presentano marcatori superficiali che le identificano 
secondo la loro origine e il loro cargo e le membrane obiettivo presentano recettori complementari che li riconoscono. Prima di tutto proteine Rab e effettori Rab portano la 
vescicola a posti specifici sulla membrana obiettivo e proteine e regolatori SNARE mediano la fusione del bistrato lipidico. Le proteine Rab causano la specificit\`a del trasporto
vescicolare e sono GTPasi monomeriche e ognuna di essa \`e associata con uno o pi\`u organelli racchiusi da membrana dei cammini endocitici o secretori che ne possiede almeno una sulla
sua superficie. L'alta selettiva distribuzione su queste membrane le rendi ideali come marcatori per identificare i tipi di membrana e guidare il traffico di vescicole tra di loro. 
Possono funzionare sulle vescicole di trasporto, sulle membrane obiettivo o entrambi. Ciclano tra la membrana e il citosol e regolano l'assemblaggio reversibile di complessi proteici
sulla membrana. Nello stato legato a GDP sono inattive e  legate ad altre proteine come i Rab-GDP dissociation inhibitor (GDI) che le mantiene solubili nel citosol. Nello stato legato
a TP sono attive e associate strettamente con la membrana di un organello o un vescicola di trasporto. Rab-GEF legate a membrana attivano le proteine Rab sulle vescicole e sulle
membrane obiettivo e per alcune fusioni sono richieste entrambe. Nello stato legato a GTP e legate a membrana attraverso un ancora si legano ad altre proteine o Rab effettori che
sono i mediatori a valle del trasporto vescicolare, filamento di proteine e fusione di membrane. Il tasso dell'idrolisi del GTP determina la concentrazione di Rab attive e degli 
effettori sulla membrana. Le strutture degli effettori sono varie e le stesse proteine possono legarsi a diversi effettori. Alcuni effettori sono proteine motrici, altre leganti
che possono collegare due membrane. Il complesso che fa attraccare vescicole COPII contiene una chinasi che fosforila la capsula per completare il processo di decapsulamento. 
L'accoppiamento del decapsulamento con la consegna delle vescicole aiuta ad assicurare la direzionalit\`a e la fusione ocn la propria membrana. Gli effettori Rab possono anche 
interagire con SNARE per accoppiare il filamento della membrana alla fusione. L'assemblaggio delle proteine Rab e i loro effettori \`e il risultato di aree di membrana: Rab5 concentra il
processo di proteine di legame che catturano vescicole in arrivo. Il suo assemblaggio comincia quando un complesso Rab5-GDP/GDI incontra un Rab-GEF: GDI \`e rilasciato e viene convertito
in Rab5-GTP. Rab5-GTP attivo si ancora alla membrana e recluta pi\`u Rab-GEF all'endosoma. Attiva inoltre una PI 3-chinasi che converte PI in PI(3)P che lega alcuni effettori Rab come
proteine di legame stabilizzando il loro legame con la membrana.
\subsection{Cascate di Rab possono cambiare l'identit\`a di un organello}
Un dominio Rab pu\`o essere sostituito cambiando l'identit\`a di un organdello: tale reclutamento di proteine Rab sequenziale \`e detto cascata di Rab: i domini Rab5 sono sostituiti
da domini Rab7 sulle membrane endosomiche, trasformandolo da endosoma giovane a vecchio alterandone le dinamiche di membrana e la sua posizione (maturazione endosomica), in un processo
irreversibile e unidirezionale.
\subsection{Le SNARE mediano la fusione di membrana}
Una volta che una vescicola di trasporto \`e stata legata alla membrana obiettivo scarica il cargo attraverso fusione di membrana in cui i bistrati si avvicinano fino a che possono
unirsi. Quando le membrane si trovano a $1.5nm$ i lipidi possono fluire tra un bistrato all'altro e l'acqua deve essere spostata dalla superficie idrofilica grazie a proteine di
fusione che superano la barriera energetica necessaria. Le proteine SNARE catalizzano le reazioni di fusione di membrana, sono specifiche all'organello, sono proteine transmembrana che
esistono come insiemi complementari: v-SNARE sulle vescicole e t-SNARE su quelle obiettivo. v-SNARE sono singole catene polippetidiche, mentre t-SNARE sono composte da tre proteine.
Hanno entrambe domini elicali e quando interagiscono (con alta specificit\`a) si arrotolano tra di loro formando un insieme di quattro eliche molto stabile. Il complesso trans-SNARE 
risultante blocca insieme le due membrane e catalizza la fusione utilizzando l'energia liberata dall'interazione e arrotolamento delle eliche per portare le membrane insieme facendo
contemporaneamente uscire l'acqua. Nella catalizzazione cooperano anche effettori Rab. Le proteine Rab possono regolare la disponibilit\`a di t-SNARES che sono associate con
proteine inibitorie che devono essere prima rilasciate grazie a un segnale di una proteina Rab e del suo effettore. 
\subsection{SNARE interagenti devono essere separati prima che possano funzionare ancora}
I complessi trans-SNARE devono disassemblarsi prima di mediare nuovi round di trasporto. NSF cicla tra membrane e il citosol e catalizza il processo di disassemblaggio. \`E una ATPasi
esamerica della famiglia delle AAA-ATPasi che usano l'idrolisi per svolgere le interazioni tra i domini elicali delle proteine SNARE accoppiate. La richiesta di questo catalizzatore
aiuta ad impedire una fusione indiscriminata delle membrane. 
\section{Trasporto dall'ER attraverso l'apparato di Golgi}
Le proteine sintetizzate attraversano la membrana ER dal citosol per entrare nel cammino secretorio. Durante il trasporto seguente dall'ER attraverso l'apparato di Golgi fino alla
superficie cellulare le proteine sono modificate mentre passano attraverso una serie di compartimenti: il trasferimento da uno all'altro coinvolge un equilibrio tra cammini di
trasporto in avanti e all'indietro. Alcune vescicole di trasporto selezionano il cargo e lo muovono nel prossimo compartimento, mentre altre recuperano proteine sfuggite e le
riportano a un compartimento precedente. L'apparato di Golgi \`e il sito principale della sintesi di carboidrati e una stazione di ordinamento per i prodotti dell'ER. Una grande
percentuale di carboidrati sono attaccati come oligosaccaridi a molte proteine e lipidi inviati dall'ER in modo da marcarli verso vescicole per i lisosomi, mentre la maggior parte
sono riconosciute in altri modi per trasportarle verso altre destinazioni.
\subsection{Le proteine lasciano l'ER in vescicole di trasporto incapsulate in COPII}
Per iniziare il viaggio lungo il cammino secretorio le proteine che entrano la membrana ER e sono destinate per l'apparato di Golgi o dopo di esso sono incapsulate in vescicole di 
trasporto incapsulate da COPII che si separano dal sito di uscita ER la cui membrana manca ribosomi e sono distribuite su tutta la superficie. L'entrata nelle vescicole pu\`o essere
uno processo selettivo o Pu\`o accadere di default. Molte proteine di membrana sono reclutate attivamente nella vescicola dove si concentrano e mostrano segnali di uscita sul lato 
citosilico che la proteina adattrice interna alla COPII riconosce. Alcune di questi componenti agiscono come recettrici di cargo e sono riciclate nell'ER dopo il trasporto. Proteine
di cargo solubili nel lume ER hanno segnali di uscita che le attaccano a recettori di cargo transmembrana. Nella vescicola possono anche entrare proteine senza il segnale di uscita come
le proteine ER residenti alcune delle quali sfuggono lentamente dall'ER e sono trasportate all'apparato di Golgi. Proteine di cargo diverse entrano nelle vescicole a tassi diversi
a causa di diverse piegature, efficienze di oligomerizzazione e cinetica. Il passo di uscita \`e il punto di controllo sule proteine che la cella secerne. Il segnale di uscita che 
porta fuori le proteine solubili dall'ER non sono compresi: alcuni recettori del cargo sono lectine che legano gli olisaccaridi sulla mannosio della proteina secreta.
\subsection{Solo proteine appropriatamente piegate e assemblate possono lasciare l'ER}
Per uscire dall'ER le proteine devono essere piegate e se sono proteine di complessi multi-proteici devono essere completamente assemblati. Quelle che non lo sono rimangono nell'Er dove
sono legate da proteine accompagnatrici come BiP o calnexina che possono coprire i segnali di uscita o ancorarle nell'ER. Tali proteine sono trasportate nel citosol dove sono degradate
da proteosomi in modo da prevenire ulteriore trasporto in avanti di proteine errate che possono interferire nel comportamento di quelle corrette. In molti casi la cellula crea delle
proteine in gran eccesso in modo che alcune possano piegarsi, assemblarsi e funzionare in maniera corretta. 
\subsection{Raggruppamenti vescicolari tubulari mediano il trasporto dall'ER all'apparato di Golgi}
Dopo che le vescicole di trasporto si sono separate dal sito di uscita ER e hanno perso la loro capsula cominciano a fondersi tra di loro attraverso fusione omotipica che richiede un
insieme di SNARE complementari. In questo caso l'interazione \`e simmetrica con entrambe contribuenti v-SNARE e t-SNARE. La struttura che si forma \`e detta di raggruppamenti vescicolari
tubulari a causa delle loro apparenza. Questi raggruppamenti costituiscono un compartimento separato dall'ER e a cui mancano molte delle proteine che vi si trovano. Sono continuamente
generati e funzionano come contenitori di trasporto che portano materiale dall'ER all'apparato di Golgi. I raggruppamenti si muovono velocemente lungo microtubuli verso l'apparato di 
Golgi con cui si fondono. Appena si formano cominciano a separare da loro vescicole di trasporto  incapsulate da COPI uniche in quanto i componenti che creano gli strati della
capsula sono reclutati come un unico elemento (coatomero). Funzionano come cammini di recupero trasportando proteine residenti nell'ER che sono sfuggite come recettori del 
cargo e SNARE che partecipano nella separazione e fusione delle vescicole. L'assemblaggio COPI inizia secondi dopo la separazione della capsula COPII e non si sa come venga controllato.
Il trasporto inverso continua mentre i raggruppamenti si muovono verso l'apparato di Golgi. I raggruppamenti maturano continuamente cambiando la propria composizione mentre le proteine
selezionate sono ritornate all'ER, cosa che continua dall'apparato di Golgi. 
\subsection{Il cammino di recupero all'ER usa segnali di ordinamento}
Il cammino per far ritornare proteine nell'ER dipende da segnali di recupero ER: le proteine residenti nell'ER contengono segnali che si legano direttamente alle capsule COPI e sono
impacchettate in vescicole di trasporto incapsulate da COPI per la consegna inversa all'ER. Il segnale meglio caratterizzato consiste di due lisine e qualsiasi altra coppia di 
amminoacidi alla terminazione C della proteina di membrana ER o una sequenza KKXX. Inoltre proteine solubili come BiP contengono un corto segnale di recupero alla terminazione C
con una sequenza Lys-Asp-Glu-Leu o simile. Questa sequenza KDEL causa il ritorno efficiente della proteina nell'ER dove \`e accumulata. Le proteine residenti nell'ER solubili devono 
legarsi a recettori come il recettore KDEL, una proteina transmembrana a multipassaggio che si lega alla sequenza e la impacchetta nelle vescicole per il trasporto inverso. Tale 
recettore cicla tra l'ER e l'apparato di Golgi con diverse affinit\`a per la sequenza: pi\`u alta nei raggruppamenti di vescicole tubulari e nell'apparato di Golgi e pi\`u bassa
nell'ER. Il cambio di affinit\`a \`e dovuto al minor pH nell'apparato di Golgi regolato da pompe \ce{H+}. Nel cammino inverso entrano anche le proteine che funzionano da interfaccia
tra l'ER e l'apparato di Golgi.
\subsection{Molte proteine sono selettivamente trattenute nei compartimenti dove funzionano}
Sembra esista un meccanismo indipendente dal segnale KDEL che trattiene le proteine residenti nell'ER. Un meccanismo ipotizzato \`e che le proteine residenti si leghino tra di loro 
formando complessi troppo grandi per entrare nelle vescicole. 
\subsection{L'apparato di Golfi consiste di una serie ordinata di compartimenti}
L'apparato di Golgi consiste di cisterne racchiuse da membrana appiattite organizzate in stack da $4$ a $6$ elementi. Nelle cellule animali connessioni tubulari tra cisterne 
corrispondenti collegano gli stack formando un complesso singolo vicino a nucleo e al centrosoma, localizzazione che dipende dai microtubuli. Durante il passaggio attraverso l'apparato 
di Golgi le molecole subiscono una serie di modifiche covalenti: ogni stack possiede una faccia di entrata (cis) e una di uscita (trans), entrambe sono associate con compartimenti
composti di una rete di strutture microtubulari e cisternali: la rete cis Golgi (CGN) e la rete trans Golgi (TGN). La prima \`e una collezione di raggruppamenti tubulari vescicolari 
fusi che arrivano dall'ER. Proteine e lipidi entrano dalla rete cis Golgi ed escono dalla rete trans Golgi verso un altro compartimento o la superficie cellulare. Entrambe le reti
ordinano le proteine: quelle che entrano nel CGN possono andare avanti o essere ritornate nell'ER e quelle uscenti dal TGN si muovono in avanti e sono ordinate rispetto alla loro 
direzione: endosomi, vescicole secretorie, la superficie cellulare o un compartimento precedente. Alcune proteine di membrana sono trattenute nella parte dell'apparato dove funzionano.
Si nota come un oligosaccaride legato a \ce{N} \`e ltato a molte proteine nell'ER e rifilato mentre vi risiede. Gli oligosaccaridi intermedi aiutano il piegamento delle proteine e
il trasporto di quelle mal piegate al citosol per la degradazione. Una volta che le proteine escono l'oligosaccaride viene utilizzato dall'apparato di Golgi per generare 
strutture oligosaccaride eterogenee viste nelle proteine mature. Dopo l'arrivo nel CGN le proteine entrano nella cisterna cis Golgi per un primo processamento. Sono poi mosse alla
cisterna mediale e infine alla cisterna trans dove viene completata la glicosilazione. Il lume della cisterna trans \`e continuo con il TGN. Il processamento degli oligosaccaridi
avviene per passi in una sequenza ordinata nello stack, con ogni cisterna contenente un insieme caratteristico di enzimi. Le proteine sono modificate in stati successivi mentre
passano attraverso lo stack in modo che formi un'unit\`a di processamento a multi stato. 
\subsection{Le catene oligosaccaridi sono processate nell'apparato di Golgi}
Le proteine residenti nell'apparato di Golgi sono tutte legate dalla membrana: tutte le glicosidasi e glicosil-trasferasi sono proteine transmembrana a singolo passaggio, molte 
organizzate in complessi multienzima. Due vaste classi di oligosaccaridi legati a \ce{N}, gli oligosaccaridi complessi e gli oligosaccaridi a mannosio alto sono attaccati alle
glicoproteine mammifere. I primi sono generati quando nell'ER l'oligosaccaride originale \`e rifilato e altri zuccheri sono aggiunti e contiene una carica negativa, mentre i secondi sono
rifilati ma senza addizioni di zuccheri. Il passaggio da oligosaccaride ad alto mannosio a complesso dipende dalla sua posizione sulla proteina: se l'oligosaccaride \`e accessibile
agli enzimi nell'apparato di Golgi viene convertito nella forma complessa, altrimenti no. 
\subsection{I proteoglicani sono assemblati nell'apparato di Golgi}
Molte proteine sono ulteriormente modificate mentre passano attraverso l'apparato di Golgi: ad alcune vengono aggiunti zuccheri ai gruppi idrossili di serine o treonine, o come nel 
collagene a proline o lisine idrossilate. Questa glicosilazione legata a \ce{O} \`e catalizzata da una serie di glicosil-trasverasi che usano zuccheri nucleotidi nel lume dell'apparato
per aggiungerli uno alla vulta. Tipicamente \`e aggiunta prima \ce{N-}acetilgalactosamina, seguita da un numero di zuccheri addizionali. La glicosilazione legata a \ce{O} pi\`u pesante
avviene sulle mucine e sulle proteine del nucleo proteoglicano che modifica per produrre proteoglicani attraverso la polimerizzazione di catene di glicosamminoglicano su serine di una
proteina nucleo. Molti di questi sono secreti e diventano parte della matrice extracellulare, mentre altri rimangono ancorati alla superficie extracellulare della membrana plasmatica. 
Gli zuccheri aggiunti sono sulfati dopo la creazione dei polimeri dando loro una forte carica negativa. 
\subsection{Il ruolo della glicosilazione}
I carboidrati complessi richiedono pi\`u enizimi diversi ado gni passo, ognuno riconosciuto unicamente come il substrato esclusivo per il passo successivo. La glicosilazione legata a 
\ce{N} \`e prevalente in tutti gli eucarioti e promuovono il piegamento delle proteine rendendo gli intermedi pi\`u solubili prevenendo ulteriore aggregazioni e le modificazioni 
sequenziali creano un codice che marca la progressione del pieegamento e media il legame di proteine ad accompagnatori e lectine. In quanto le catene di zuccheri hanno una flessibilit\`a
limitata possono limitare l'avvicinarsi di altre macromolecole alla superficie proteica rendendole pi\`u resistenti alla digestione attraverso enzimi proteolitici. Le catene di 
zuccheri sevno anche come protezione di patogeni nella forma di coperture di muco nelle cellule di polmoni e intestinali. Il riconoscimento di catene cellulari da parte della lectina
nello spazio extracellulare serve nel processo di sviluppo e nel riconoscimento tra cellule. La presenza di oligosaccaridi pu\`o modificare la funzione e propriet\`a antigenica di 
una proteina. Pu\`o avere anche ruoli regolatori. 
\subsection{Il trasporto attraverso l'apparato di Golgi potrebbe accadere attraverso maturazione cisternale}
\`E ancora incerto come l'apparato di Golgi mantenga la sua struttura polarizzata e come le molecole si muovano tra una cisterna e l'altra. Un ipotesi della del modello di maturazione
cisternale vede le cisterne come strutture dinamiche che maturano ottenendo e perdendo specifiche proteine residenti del Golgi. Le nuove cis cisterne si formano continuamente come
raggruppamenti vescicolari tubulari che arrivano dall'ER e maturano per diventare ciserne mediali e poi trans. Un cisterna si muove attraverso lo stack con il cargo nel suo lume. Il
trasporto inverso attraverso separazione di vescicole inpasulate da COPI spiega la distribuzione caratteristica. Secondo il modello di trasporto di vescicole il flusso di vescicole
retrograde recupera le proteine e le ritorna nel compartimento a monte. Il flusso direzionale \`e ottenuto attraverso l'impacchettamento di molecole di cargo che si muovono in avanti
in vescicole specifiche. Le due vescicole diverse sono riconosciute attraverso proteine adattrici diverse. Probabilmente aspetti di entrambi i modelli sono veri. 
\subsection{La matrice di proteine del Golgi aiuta ad organizzare lo stack}
L'architettura dell'apparato di Golgi dipende dal citoscheletro di microtubili e dalle proteine citoplasmatiche della matrice di Golgi che formano un'impalcatura tra cisterne adiacenti
e danno l'integrit\`a strutturale allo stack. Alcune di queste dette golgine formano lunghi filamenti composti di domini rigidi a bobina tra regioni cardine intersperse, insieme formano
un insieme di tentacoli che aiutano a mantenere vicino al Golgi vescicole di trasporto attraverso interazioni con proteine Rab. Quando la cellula si prepara a dividersi 
proteine chinasi mitotiche fosforilano tali proteine causando la frammentazione dell'apparato e la sua dispersione nel citosol. I frammenti sono poi distribuiti nelle cellule figlio
in ugual misura dove sono defosforilate permettendo il riassemblaggio dello stack. 
\section{Il trasporto dalla rete trans Golgi ai lisosomi}
La rete trans Golgi ordina tutte le proteine che passano attraverso l'apparato secondo la destinazione attraverso un processo selettivo come avviene per quelle destinate al lume dei 
lisosomi.
\subsection{I lisosomi sono il sito principale della digestione intracellulare}
I lisosomi sono organelli chiusi da membrana riempiti da enzimi idrolitici che digeriscono macromolecole, contengono proteasi, nuclasi, glicosidasi, lipasi, fosfolipasi, fosfopatasi e 
sulfatasi. Sono tutte idrolasi acide (lavorano a pH acidi). Per attivit\`a ottima devono essere attivati attraverso rottura proteolitica che richiede anche un ambiente acido. Il lisosoma
mantiene un pH di $4.5-5.0$. I contenuti del citosol sono protetti attraverso attacchi dal sistema digestivo: la membrana del lisosoma mantiene gli enzimi digestivi fuori dal citosol
ma, anche in caso di uscita possono fare pochi danni al pH citosilico ($7.2$ pH). Anche la membrana \`e caratteristica> la maggior parte delle proteine sono altamente glicolisate in 
modo da proteggerle dalle proteasi nel lume. Le proteine di trasporto portano fuori dal lisosoma i prodotti finali della digestione delle macromolecola (amminoacidi, zuccheri e 
nucleotidi) al citosol dove vengono utilizzati dalla cellula o secreti. Un \ce{H+} ATPasi vacuolare nella membrana del lisosoma usa l'energia dell'idrolisi dell'ATP per pompare \ce{H+}
nel lisosoma mantenendo il lume al pH acidico. Tali pompe appartengono alle ATPasi v-tipo e lavorano solo pompando \ce{H+} nell'organello. Pompe simili acidificano tutti gli organelli
endocitici o esocitici. Oltre a creare l'ambiente acido il gradiente di \ce{H+} fornisce una fonte di energia per guidare il trasporto dei piccoli metaboliti attraverso la membrana. 
\subsection{I lisosomi sono eterogenei}
I lisosomi hanno una morfologia eterogenea che riflette le varie funzioni digestive che le idrolasi acide mediano e il modo in cui si formano. Gli endosomi maturi contenti materiale
ricevuto dalla membrana plasmatica e nuove idrolasi lisosomiali si fondono con lisosomi preesistenti per formare endolisosomi che si fondono poi tra di loro. Quando il materiale 
endocitato \`e stato digerito dall'endolisosoma questo ritorna un lisosoma classico, organelli piccoli, densi e sferici che possono ricominciare il ciclo di fusione con gli endosomi.
Per questa ragione sono visti come un insieme eterogeneo di organelli distinti con la caratteristica comune di avere un alto contenuto di enzimi idrolitici.
\subsection{I vacuoli delle piante e dei funghi sono lisosomi versatili}
La maggior parte delle cellule delle piante e dei funghi contengono grandi vescicole riempite da liquido dette vacuoli, imparentati con i lisosomi animali e contenenti un gran numero di
enzimi idrolitici, pur avendo funzione diversa. Il vacuolo della pianta pu\`o agire come un organello di conservazione per nutrienti e prodotti di scarto, un compartimento degradatorio,
un modo per aumentare la dimensione della cellula e come controllore della pressione di turgore. \`E un importante dispositivo omeostatico che rende la pianta capace di sopportare 
cambi dell'ambiente.
\subsection{Multipli cammini portano materiali ai lisosomi}
I lisosomi sono punti in cui vari flussi di traffico intracellulare converge. La maggior parte degli enzimi digestivi arrivano dall'apparato di Golgi, mentre almeno quattro cammini 
portano sostanze per la digestione. Il pi\`u studiato \`e il cammino di degradazione delle macromolecole catturate dal fluido extracellulare attraverso endocitosi. Si trova un
cammino simile nelle cellule fagocitiche \`e dedicato alla cattura  o fagocitosi di grandi particelle e microorganismi per formare fagosomi. Un terzo cammino detto macropinocitosi si
specializza nel recupero non specifico di fluidi, membrane e particelle attaccate alla membrana plasmatica. Un quarto cammino detto autofagia viene utilizzato per digerire il citosol
e gli organelli consumati. 
\subsection{L'autofagia degrada proteine ed organelli non voluti}
Tutti i tipi cellulari si liberano di parti obsolete grazie ad un processo dipendente dai lisosomi: l'autofagia. \`E importante per la ristrutturazione delle cellule differenzianti e in
risposte adattive ad eventi esterni. Pu\`o rimuovere grandi oggetti come organelli. Nel passo iniziale dell'autofagia il cargo citoplasmatico viene circondato da una doppia membrana
che si assembla per fusione di piccole vescicole detta autofasogioma. Il processo avviene in $5$ passi:
\begin{enumerate}
	\item Induzione da attivazione di molecole di segnale: chinasi proteiche (complesso mTOR $1$) che inoltrano informazioni riguardo lo stato metabolico della cellula, si attivano
		e segnalano al macchinario autofago.
	\item Nucleazione ed estensione di una membrana delimitante in una forma a coppa, le vescicole di membrana, caratterizzate da ATG9 sono reclutate a un sito di assemblaggio dove
		nucleano la formazione dell'autofasogioma. ATG9 viene rimossa da un cammino di recupero.
	\item Chiusura della coppa di membrana intorno all'obiettivo per formare un autofasogioma isolato da una doppia membrana.
	\item Fusione con lisosomi catalizzata da SNARE.
	\item Digestione della membrana interna e dei contenuti del lume dell'autofasogioma.
\end{enumerate}
L'autofagia pu\`o essere selettiva o no. In quella non selettiva una porzione del citoplasma viene sequestrata da autofasogiomi. Pu\`o accadere in condizioni di mancanza di nutrienti. 
In quella selettiva viene isolato un cargo specifico che tende a contenere poco citosol, con una forma che riflette quella del cargo, media la degradazione di mitocondri, perossisomi, 
ribosomi e ER non voluti o consumati (anche per microbi invadenti). 
\subsection{Un recettore mannosio $\mathbf{6}$-fosfato ordina le idrolasi lisosomiali nella rete trans Golgi}
Si considera ora il cammino che porta le idrolasi lisosomiali dal TGN ai lisosomi. Gli enzimi sono prima portati a endosomi in vescicole di trasporto selettive che si separano dal 
TGN. Il riconoscimento di tali idrolasi avviene grazie a un marcatore nella fomra di gruppi di mannosio $6$-fosfato, aggiunti esclusivamente a oligosaccaridi legati a \ce{N} di questi
enzimi mentre passano attraverso il lume della rete cis Golgi. Proteine recettrici M6P riconoscono tali gruppi e legano al lato lumenale della membrana e a proteine adattrici in 
capsule di clatrina che si stanno costruendo sul lato citosolico. I recettori impacchettano le idrolasi in vescicole incapsulate da clatrina che si separano dal TGN e portano i loro
contenuti a giovani endosomi. La proteina recettrice di M6P si lega a esso a pH di $6.5-6.7$ nel TGN e si rilascia a pH $6$, presente nel lume degli endosomi. Dopo che il recettore
arriva in esso le idrolasi lisosomiali si dissociano da esso e il recettore viene ritrasportato indietro da vescicole che si separano dagli endosomi incapsualte da un retromero. 
Il trasporto in entrambe le direzioni richiede segnali nella coda citosilica del recettore M6P che direziona la proteina verso l'endosoma o il TGN: sono riconosciuti da un complesso 
retromero che recluta i recettori in vescicole di trasporto che si separano dall'endosoma. Il riciclo assomiglia a quello del recettore KDEL. Alcune molecole di idrolasi marcate 
raggiungono il lissooma: alcune sfuggono al processo di impacchettamento nella rete trans Golgi e vengono portate nel fluido extracellulare. Alcuni recettori M6P possono arrivare alla
membrana plasmatica dove ricatturano le idrolasi lisosomiali e le ritornano attraverso endocitosi mediata da recettore ai lisosomi attraverso endosomi giovani o maturi. Affinch\`e
funzionino i gruppi M6P devono essere aggiunti solo alle glicoproteine appropriate nell'apparato di Golgi, che richiede un riconoscimento specifico dell'idrolasi dagli enzimi che 
aggiungono M6P. Tale segnale deve essere presente nella catena polipeptidica di ogni idrolasi come raggruppamento di amminoacidi vicini sulla superficie della proteina. 
\subsection{Alcuni lisosomi e corpi multivescicolari subiscono esocitosi}
La secrezione lisosomiale di contenuto non digerito permette alla cellula di eliminare detriti non digeribili. Alcune cellule contengono lisosomi specializzati con il macchinario 
necessario per la fusione con la membrana plasmatica come i melanociti nella pelle. In alcune circostanze questo pu\`o accadere anche a corpi multivescicolari che rilasciano le
vescicole dalle cellule. Si osservano questi esosomi nel sangue e si pensa che trasportino componenti tra le cellule. 
\section{Il trasporto nella cellula dalla membrana plasmatica: endocitosi}
Il percorso che porta all'interno dalla superficie cellulare inizia attraverso endocitosi, in cui la cellula recupera componenti della membrana plasmatica, fluidi, soluti, macromolecole
e sostanze particolari. Il cargo degli endociti comprende complessi di recettori, nutrienti e loro trasportatori, componenti della matrice extracellulare, detriti cellulari, batteri, 
virus e in alcuni casi altre cellule. Attraverso l'endocitosi si regola la composizione della membrana plasmatica in risposta al cambiamento delle condizioni extracellulari. Il 
materiale che deve essere ingerito viene progressivamente chiuso da una piccola porzione della membrana plasmatica che si invagina e si separa formando una vescicola endocitica 
attraverso pinocitosi oltre a cammini dedicati per catturare larghe particelle su richiesta attraverso fagocitosi. Una volta generate le vescicole endocitiche si fondono in un
endosoma giovane dove il cargo internalizzato \`e ordinatoL ritornato alla membrana plasmatica direttamente o attraverso un endosoma di riciclo o designato per la degradazione 
includendolo in un endosoma maturo. Questi ultimi formano una porzione vacuolare attraverso maturazione endosomica. Il processo di conversione cambia la composizione delle proteine
della membrana endosomica, aree della membrana si invaginano formando vescicole intralumenali mentre l'endosoma si sposta verso il nucleo. Mentre matura l'endosoma smette di riciclare
materiale e lo destina tutto alla degradazione. Ogni passaggio della maturazione \`e connesso attraverso cammini di trasporto vescicolare al TGN, cammini che permettono l'inserimento
di materiale appena sintetizzato e il recupero di componenti.
\subsection{Vescicole pinocitiche si formano da pozzi incapsulati nella membrana plasmatica}
Tutte le cellule eucariotiche ingeriscono continuamente porzioni della loro membrana plasmatica per formare piccole vescicole pinocitiche. Il tasso di pinoticitosi varia in base al tipo
cellulare ma \`e sempre alto. La stessa quantit\`a di membrana rimossa viene aggiunta attraverso esocitosi, in legame con l'endocitosi nel ciclo di endocitosi-esocitosi. Tale 
accoppiamento \`e particolarmente rigido in strutture caratterizzate da un alto turnover di membrana. La parte endocitica del ciclo inizia a pozzi incapsulati da clatrina con vita corta:
si invaginano velocemente formando una vescicola incapsulata da clatrina. Tali vescicole sono ancora pi\`u transienti rispetto ai pozzi: si fondono con gli endosomi giovani.
\subsection{Non tutte le vescicole pinocitiche sono incapsulate da clatrina}
Oltre a tali vescicole si formano altre vescicole pinocitiche come le caveole sono presenti nella membrana plasmatica della maggior parte dei tipi cellulari e si formano da fessure
lipidiche nella membrana plasmatica ricche di colesterolo, glicosfingolipidi e proteine GPI ancorate alla membrana. Le maggiori proteine strutturali nelle caveole sono le caveoline, 
proteine integrali di membrana che inseriscono un anello ifdofobico nella membrana dal lato citosilico senza estendersi attraverso essa. Le caveoline sono legate al lato citosilico
a grossi complessi di proteine che si pensa stabilizzano la curvatura di membrana. Le caveole sono tipicamente strutture statiche, ma possono essere indotte a separarsi e servire come
vescicole endocitiche per trasportare cargo agli endosomi giovani o alla membrana plasmatica sul lato opposto di una cellula polarizzata (transcitosi). La macropinocitosi \`e un
altro meccanismo indipendente dalla clatrina che pu\`o essere attivato in tutte le cellule animali. Viene indotto per un tempo limitato in risposta ad attivazione di recettori sulla
superficie cellulare in risposta a cargo specifici come fattori di crescita che attivano una serie di segnali che cambiano le dinamiche dell'actina e formano protrusioni sulla
superficie cellulari che quando collassano nella cellula formano grandi vescicole piene di liquido dette macropinosomi che aumentano il recupero di fluido della cellula. Sono dedicate a
un cammino degradativo: si acidificano e uniscono con endosomi maturi o endolisosomi senza riciclare il loro cargo alla membrana plasmatica.
\subsection{Le cellule usano endocitosi mediata da recettori per importare macromolecole selezionate extracellulari}
Nella maggior parte delle cellule animali i pozzi incapsulati da clatrina e le vescicole forniscono un cammino efficiente per il recupero di macromolecole: l'endocitosi mediata da
macromolecole, in cui queste ultime si legano a proteine recettrici transmembrana complementari che si accumulano nei pozzi di clatrina e dopo entrano nella cellula come complessi
recettore-macromolecola in vescicole incapsulate da clatrina. Essendo i ligandi catturati selettivamente mette a dispoizione un meccanismo di concentrazione che aumenta l'efficienza di 
internalizzazione di ligandi particolari. Molte cellule animali recuperano colesterolo in questo modo. Se il recupero \`e bloccato il colesterolo si accumula nel sangue e pu\`o formare
pareti di placche arteriosclerotiche. La maggior parte del colesterolo \`e trasportato nel sangue come colesteril esterre nella forma di particelle di lipide-proteina dette lipoproteine
a bassa densit\`a. Quando una cellula necessita di colesterolo per la sintesi crea recettori transmembrana per LDL e le inserisce nella membrana plasmatica dove si diffondono fino a che
si associano con pozzi a clatrina che si stanno formando dove un senglae di encocitosi nella coda citoplsamatica dei recettori LDL si lega con la proteina adattatrice AP2 dopo che la
sua conformazione \`e stata sbloccata localmente da legame con PI(4, 5)$P_2$ sulla membrana plasmatica. Le vescicole formatesi da questi pozzi, dopo aver perso la propria capsula 
portano il loro contenuto agli endosomi giovani. LDL \`e rilasciato dal recettore e portato ai lisosomi dove il colesteril estere \`e idrolizzato in colesterolo libero e reso disponibile
per la sintesi di nuove membrane. In eccesso di colesterolo la cellula blocca la sintesi interna di colesterolo e di recettori LDL. Tutti i recettori conosciuti usano strade di 
internalizzazione dipendenti da clatrine e sono guidati da segnali nella coda citoplasmatica che si lega alla proteina adattatrice nella capsula clatrinica. Molti di questi recettori 
come LDL entrano nel pozzo anche se non hanno legato un ligando. I pozzi servono come filtri molecolari, collezionando certe proteine recettrici rispetto ad altre. 
\subsection{Proteine specifiche sono recuperate dagli endosomi giovani e ritornate alla membrana plasmatica}
Gli endosomi giovani sono i punti principali per l'ordinamento nei cammini endocitici: nel loro ambiente leggermente acido molte proteine recettrici cambiano la loro conformazione e 
rilasciano il ligando marcandolo per distruzione nel lisosoma con altri contenuti solubili dell'endosoma. Altri ligandi rimangono legati ai loro recettori e condividono il loro destino.
Nell'endosoma giovane il recettore LDL si dissocia con il ligando ed \`e riciclato alla membrana plasmatica. Il recettore trasferina segue un cammino simile al recettore LDL ma ricicla
anche il proprio ligando. La trasferina \`e una proteina solubile che trasporta il ferro nel sangue. I recettori supeficiali di trasferina la portano con il ferro legato a un endosoma 
giovane attraverso endocitosi mediata dal recettore. Il  pH basso nell'endosoma causa il rilascio del ferro, ma la trasferina stessa rimane legata al recettore e il complesso entra
nell'estensione tubulare dell'endosoma giovane dove \`e riciclata alla membrana plasmatica, dove si dissocia dal recettore e ritorna libera di recuperare altro ferro e iniziare il ciclo.
\subsection{I recettori segnalanti nella membrana plasmatica sono regolati da degradazione nei lisosomi}
Un secondo cammino che i recettori endocitati possono seguire \`e seguito da molti recettori come quelli che legano i fattori di crescita epidermici (EGF), una piccola proteina di 
segnale che stimola la divisione di molti tipi di cellule. Tali recettori si legano in pozzi solo dopo che si sono legati al ligando le la maggior parte sono degradati nel lisosoma 
insieme all'EGF che pertanto attiva il cammino di segnale e poi porta a una diminuzione nella concentrazione di recettori EGF sulla superficie cellulare o sottoregolazione di recettore 
che riduve la seguente sensibilit\`a a EGF. Tale meccanismo \`e altamente regolato: i recettori attivati sono prima modificati covalentemente sulla faccia citosilica con ubiquitina che
aggiunge una o poche singole molecole di ubiquitina alla proteina nella monouibiquitilazione o multiubiquitazione. Tal iproteine riconoscono l'ubiquitina e aiutano a direzionarla
nei pozzi. Dopo la consegna altre proteine che si legano a ubiquitina parte dei complessi ESCRT (complesso di ordinamento di endosomi richiesto per il trasrporto) che riconoscono e
ordinano i recettori ubiquitilati in vescicole intralumenali che sono ritenute negli endosomi maturi. In questo modo l'addizione di ubiquitina blocca il riciclo di recettori.
\subsection{Gli endosomi giovani maturano in endosomi maturi}
La nascita degli endosomi giovani \`e causata da vescicole endocitiche che si fondono tra di loro, sono relativamente piccoli e si trovano nel citoplasma sotto la membrana plasmatica
dove si muovono lungo microtubuli catturando vescicole in arrivo. Gli endosomi giovani hanno domini tubulari e vacuolari: la maggior parte della superficie \`e nei tubuli e la maggior 
parte del volume \`e nei vacuoli. Durante la maturazione endosomica le porzioni vacuolari sono trasformate negli endosomi maturi mentre la porzione tubulare si restringe. Gli endosomi in
maturazione o corpi multivescicolari migrano lungo i microtubuli verso l'interno della cellula perdendo tubuli e vescicole che riciclano materiale verso la membrana plasmatica e TGN
e ricevono nuove proteine sintetizzate. Mentre si concentrano alla regione perinucleare si fondono tra di loro e con endolisosomi e lisosomi. Durante il processo di maturazione avvengono
molte variazioni: l'endosoma si sposta e cambia forma, le proteine Rab, lipidi fosfoinositidi, macchinari di fusione e microtubuli cambiano la superficie citosilica dell'endosoma 
cambiandone la funzione. Una ATPasi v-tipo pompa \ce{H+} dal citosol nel lume e acidifica l'organello che rende le idrolasi lisosomiche pi\`u attive e influenza le interazioni 
tra recettore e ligandi, vescicole intralumenali sequestrano recettori di segnale endocitati nell'endosoma bloccando l'attivit\`a di segnalazione, le proteine del lisosoma sono
portate dal TGN all'endosoma in maturazione. 
\subsection{Complessi proteici ESCRT mediano la formazione di vescicole intralumenali nei corpi multivescicolari}
Mentre un endosoma matura aree della sua membrana si invaginano nel lume e si separano formando vescicole intralumenali, per questo motivo gli endosomi in maturazione vengono anche 
chiamati corpi multivescicolari. Questi ultimi trasportano proteine di membrana endocitate che devono essere degradate che vengono selettivamente partizionate in membrane che si stanno
creando in modo da sequestrare recettori e proteine di segnale fortemente legate al citosol dove potrebbero continuare la segnalazione. Sono totalmente accessibili agli enzimi digestivi
che le degradano. I corpi multivescicolari includono contenuto solubile dell'endosoma giovane destinato alla digestione. L'ordinamento nelle vescicole intralumenali richiede marcature
di ubiquitina, aggiunte ai domini citosilici delle proteine di membrana che aiutano inizialmente a guidare le proteine in vescicole incapsulate da clatrina nella membrana plasmatica.
Una volta portate alle membrana endosomiale sono ancora riconosciute da una serie di complessi di  proteine ESCRT citosiliche (ESCRT-0, -II e -III) che legano sequenzialmente e mediano
il processo di ordinamento nelle vescicole intralumenale. Le invaginazioni dipendono anche su una chinasi lipidica che fosforila fofatidilinositolo per produrre PI(3)P che serve come un
sito di attracco per i complessi ESCRT. ESCRT-III forma grandi assemblaggi multimerici sulla membrana che la piegano.
\subsection{Endosomi riciclanti regolano la composizione della membrana plasmatica}
La maggior parte dei recettori sono riciclati e ritornati allo stesso dominio della membrana plasmatica da cui sono arrivati, alcuni procedono a un dominio diverso attraverso 
transcitosi, mentre altri ancora sono degradati nei lisosomi. I recettori sulla superficie delle cellula epiteliali polarizzate possono trasferire macromolecole specifiche da uno 
spazio extracellulare all'altro attraverso transcitosi. Il cammino transcitico non \`e diretto: i recettori si muovono prima dall'endosoma giovane all'endosoma riciclante. Da questo si
evince che oltre ai siti di legame per i ligandi e per i pozzi molti recettori possiedono segnali di ordinamento che li guidano nel cammino di trasporto appropriato. Le cellule possono
regolare il rilascio di proteine di membrana dagli endosomi riciclanti modificando il flusso di proteine attraverso i percorsi transcitotici in modo da controllare la concentrazione di
specifiche membrane plasmatiche in risposta a segnali. 
\subsection{Cellule fagocitiche specializzate possono ingerire grandi particelle}
La fagocitosi \`e una forma speciale di endocitosi in cui una ceellula usa grandi vescicole endocitiche dette fagosomi per ingerire grandi particelle come microorganismi e cellule
morte. Nei protozoi \`e una forma di nutrimento: le particelle finiscono in lisosomi e i prodotti della digestione passano nel citosol. Poche cellule sono capaci di ingerire grandi 
particelle efficientemente: nello stomaco degli animali processi extracellulari rompono le particelle di cibo e le cellule importano i piccoli prodotti dell'idrolisi. La fagocitosi \`e
importante per la maggior parte degli animali per altri motivi e avviene grazie a i fagociti specializzati come due classi di cellule del sangue bianco : i macrofagi e i neutrofili che
si sviluppano da cellule staminali emopoieitche che ingeriscono microorganismi invadenti per proteggere contro infezioni. I macrofagi inoltre cercano cellule morte per apoptosi. Il 
diametro di un fasogioma \`e determinato dalla dimensione della particella ingerita, che possono anche essere grandi quasi come le cellule ingerenti. Si fondono poi con i lisosomi dove
il materiale ingerito \`e degradato. Le sostanze indigeste rimangono nei lisosomi formando corpi residui che vengono secreti attraverso esocitosi. La fagocitosi \`e un processo
causato dal cargo: richiede l'attivazione di recettori che trasmettono segnali all'interno della cellula e le particelle per essere fagocitate devono prima legarsi alla superficie
del fagocita che possiedono vari recettori uniti funzionalmente al macchinario fagocitico. Gli attuatori meglio caratterizzati sono gli anticorpi che iniziano una serie di eventi che
culminano con la fagocitazione dell'organismo invadente. Inizialmente il patogeno viene incapsulato con molecole di anticorpi che legano a recettori Fc sulla superficie di macrofagi e 
neutrofili attivando i recettori per indurre l'esteionse pseudopoda della cellula che racchiude la particella e fonde alle estremit\`a per formare un fagosoma. La polimerizzazione di
actina locale iniziata dalla famiglia di GTPasi Rho e le Rho-GEF attivanti da forma allo pseudopoda. La Rho GTPasi attivata attiva l'attivit\`a chinasi di PI chinasi locali per produrre
PI(4, 5)$P_2$ nella membrana che stimolano la polimerizzazione di actina. Per isolare il fasogioma e completare l'ingestione l'actina \`e depolimerizzata da una chinasi PI 3 che
converte PI(4, 5)$P_2$ in PI(3, 4, 5)$P_3$ richiesto per la chiusura del fasogioma e per la riformazione della rete di actina in modo da guidare l'invaginazione del fasogioma formante. 
Altre classi di recettori riconoscono componenti complementari altre riconoscono oligosaccaridi sulla superficie di certi patogeni altri riconoscono cellule morte per apoptosi che 
perdono la distribuzione asimmetrica di fosfolipidi nella membrana plasmatica causando un esposizione di cariche negative. I macrofagi non fagocitano le cellule viventi in quanto 
sono presenti segnali affinch\`e questo non avvenga nella forma di proteine sulla superficie della cellula dei macrofagi. I recettori inibitori reclutano tirosina fosfatasi che 
antagonizza gli eventi di segnalazione intracellulare richiesti per l'inizio della fagocitosi inibendo localmente tale processo. La fagocitosi dipende pertanto da un equilibrio tra
segnali positivi che attivano il processo e segnali negativi che lo inibiscono.
\section{Il trasporto dalla rete trans Golfi all'esterno cellulare: esocitosi}
Le vescicole di trasporto normalmente destinate per la membrana plasmatica lasciano il TGN in un flusso costante come tubuli irregolari. Le proteine di membrana e i lipidi forniscono
nuove componenti per la membrana plasmatica, mentre le proteine solubili all'interno sono secrete nello spazio extracellulare. La fusione delle vescicole con la membrana plasmatica
\`e detto esocitosi. Tutte le cellule richiedono questo cammino secretorio costitutivo che opera continuamente, mentre cellule specializzate ne possiedono un altro in cui
proteine solubili e altre sostanze sono prima conservate in vescicole secretorie per rilascio successivo nel cammino secretorio regolato. 
\subsection{Molte proteine e lipidi sono trasportate automaticamente alla rete trans Golgi alla superficie extracellulare}
Una cellula capace di secrezione deve separare almeno tre classi di proteine prima che lascino il TGN: quelle destinate per i lisosomi, quelle per le vescicole secretorie e quelle che
vanno portate immediatamente alla superficie cellulare. Le proteine secretorie sono trasportate in vescicole secretorie attraverso segnali. Il cammino di secrezione costitutivo non \`e
selettivo e trasporta la maggior parte delle altre molecole direttamente alla superficie cellulare e in quanto non richiede segnale \`e detto il cammino di default. In una cellula non
polarizzata ogni proteina nel lume dell'apparato di Golgi viene automaticamente trasportata verso la superficie se non \`e specificato altrimenti, mentre nelle cellule polarizzate ci
sono opzioni pi\`u complesse.
\subsection{Le vescicole secretorie si separano dalla rete trans Golgi}
Le cellule specializzate per la secrezione di alcuni prodotti rapidamente su richiesta li concentrano in vescicole secretorie o granuli secretori a nucleo denso che si formano dal TGN
e rilasciano il loro contenuto all'esterno in risposta a segnali specifici. Il secreto pu\`o essere una piccola molecola o una proteina. Le proteine secretrici sono impacchettate in
un meccanismo che coinvolge aggregazione specifica di esse. I segnali che le direzionano non sono ben definiti. Le vescicole secretorie hanno proteine uniche nelle loro membrane, alcune
delle quali potrebbero servire come recettori per aggregazioni nel TGN. Il recupero degli aggregati nelle vescicole secretorie assomiglia al processo di fagocitosi a causa della loro 
dimensione. Inizialmente le vescicole che lasciano il TGN circondano debolmente l'aggregato e queste vescicole secretorie immature assomigliano morfologicamente a cisterne trans Golgi
dilatate che hanno lasciato lo stack. Mentre le vescicole maturano si fondono tra di loro e concentrano i loro contenuti come risultato del continuo riciclo di membrana al TGN e della
continua acidificazione del lume della vescicola (aumento di ATPasi v tipo). Le proteine di membrana e secretorie si concentrano mentre si muovono dall'ER attraverso l'apparato di
Golgi a causa di un esteso processo di recupero mediato da vescicole di trasporto incapsulate da COPI che trasportano proteine residenti nell'ER indietro escludendo le proteine
secretorie e di membrana. Il riciclo di membrana \`e importante per ritornare al Golgi i suoi componenti e per concentrare i contenuti delle vescicole secretorie. A causa della grande 
densit\`a di contenuti le vescicole secretorie possono espellere grandi quantit\`a di materiale prontamente.
\subsection{I precursori di proteine secretorie sono processate proteoliticamente durante la formazione di vescicole secretorie}
Molte proteine secrete sono sintetizzate come precursori inattivi e la proteolisi \`e necessaria per liberare le molecole attive da questi precursori. La rottura avviene nelle vescicole
secretorie e a volte dopo la secrezione. Oltre a quello molti precursori possiedono un propeptide \ce{N} terminale rotto per creare la proteina matura e sono sintetizzate come 
pre-pro-proteine con il prepeptide consistente di un sengale ER che viene rotto nell'ER rugoso. In altri casi le molecole di segnale sono create come poliproteine con molte coppie
di una sequenza di amminoacidi, inoltre una variet\`a di peptidi di segnale sono sintetizzati come parte di una singola poliproteina che agisce come un precursore di diversi prodotti
finali che sono rotti singolarmente dalla catena iniziale. Questi metodi oltre a permettere il trasporto della proteina garantiscono un ritardo nella sua attivazione fino a che questa 
\`e richiesta. 
\subsection{Vescicole secretorie aspettano vicino la membrana plasmatica fino a che vengono segnalate di rilasciare i loro contenuti}
Una volta cariche le vescicole secretorie devono raggiungere il sito della secrezione e per farlo proteine motrici le muovono lungo microtubili il cui orientamento guida la vescicola
nella direzione corretta. Le vescicole qui aspettano un segnale per secernere e successivamente si fondono.
\subsection{Per esocitosi rapida le vescicole sinaptiche sono preparate alla membrana plasmatica presinaptica}
Le cellule nervose possiedono due tipi di vescicole secretorie che impacchettano proteine e neuropeptidi in vescicole secretorie nel modo standard e una classe di piccole vescicole 
secretorie dette vescicole sinaptiche che conservano neurotrasmettitori che mediano la segnalazione tra cellule nervose e i loro obiettivi: l'arrivo di un potenziale d'azione causa
un influsso di \ce{Ca^{2+}} che causa la fusione della vescicola sinaptica con la membrana plasmatica e rilasciare il suo contenuto nello spazio extracellulare. La velocit\`a del 
rilascio indica che la fusione \`e un processo veloce e quando le vescicole attraccano alla membrana plasmatica subiscono un passo di preparazione che le prepara per una fusione rapida:
le SNARE sono accoppiate parzialmente attraverso complessine che bloccano i complessi SNARE nello stato metastabile. La rottura della complessina \`e rilascaita dalla sinaptotagmina
con domini leganti a \ce{Ca^{2+}}. A una sinapsi tipica solo un piccolo numero di vescicole attraccate sono preparate per l'esocitosi in modo da permettere pi\`u round di segnali.
\subsection{Vescicole sinaptiche si possono formare direttamente da vescicole endocitiche}
Affinch\`e il terminale nervoso risponda rapidamente e ripetutamente le vescicole sinaptiche devono essere ripristinate velocemente dopo che si scaricano. Per questo la maggior parte 
delle vescicole sinaptiche sono generate da riciclo locale dalla membrana plasmatica nel terminale nervoso e similmente nuove componenti della vescicola sinaptica sono inizialmente 
portate alla membrana plsamatica da cammini di secrezione ; che si scaricano. Per questo la maggior parte 
delle vescicole sinaptiche sono generate da riciclo locale dalla membrana plasmatica nel terminale nervoso e similmente nuove componenti della vescicola sinaptica sono inizialmente 
portate alla membrana plasmatica da cammini di secrezione  costitutiva e recuperati attraverso endocitosi. I componenti della membrana di una vescicola sinaptica includono trasportatori
specializzati per il recupero di neurotrasmettitori dal citosol dove sono sintetizzati. Una volta riempita la vescicola pu\`o essere riutilizzata. 
\subsection{I componenti della membrana di una vescicola secretoria sono velocemente rimossi dalla membrana plasmatica}
Quando una vescicola secretoria si fonde con la membrana plasmatica i suoi contenuti sono scaricati attraverso esocitosi e la membrana diventa parte della membrana plasmatica, 
aumentandone la superficie transientemente in quanto le componenti sono successivamente rimosse dalla superficie attraverso endocitosi. Dopo la loro rimozione sono riciclate o 
trasportate in lisosomi per la degradazione. Il controllo del traffico di membrana ha un ruolo centrale nel mantenere la composizione delle varie membrane della cellula. 
\subsection{Alcuni eventi di esocitosi regolata servono per ingrandire la membrana plasmatica}
Un compito dell'esocitosi regolata \`e di portare pi\`u membrana per aumentare la superficie della membrana plasmatica di una cellula quando necessario. Un esempio \`e il processo di 
cellularizzazione in cui una singola cellula contenenti diversi nuclei si separa in cellule distinti con un nucleo ognuna attraverso aggiunta di membrana ottenuta dalla fusione di 
vescicole citoplasmatiche. Molte cellule animali soggette a stress meccanico subiscono piccole rotture nella loro membrana plasmatica in un processo che coinvolge fusione tra vescicole
omotipiche ed esoticosi viene aggiunta rapidamente una pezza che oltre a isolare la cellula riduce la tensione di membrana sull'area ferita. Questo processo \`e causato da un improvviso
aumento di \ce{Ca^{2+}}.
\subsection{Cellule polarizzate direzionano le proteine dalla rete trans Golgi al dominio appropriato della membrana plasmatica}
La maggior parte delle cellule nei tessuti sono polarizzate con domini nella membrana plasmatica distinti. Deve pertanto esistere un meccanismo per la consegna di membrana dal Golgi
in modo da mantenere le differenze tra i domini. Una tipica cellula epiteliale contiene un dominio apicale verso il mondo esterno e un dominio basolaterale che copre il resto della
cellula separati da giunzioni che impediscono la diffusione dei lipidi nei due domini. In principio le componenti potrebbero essere consegnate indistintamente e rimosse se necessario, 
ma in molti casi la consegna avviene specificatamente al dominio appropriato. Sono presenti pertanto segnali nelle proteine e nei lipidi di membrana che direzionano le vescicole nel
dominio appropriato. 
