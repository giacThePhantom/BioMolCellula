\chapter{Traffico di membrana intracellulare}
Le cellule per nutrirsi, comunicare e rispondere a cambi nell'ambiente aggiustano la composizione della loro membrana plasmatica e compartimenti interni in risposta alle loro 
necessit\`a. Usano un sistema di membrane interne per aggiungere e rimuovere proteine di superficie come recettori, canali ionici e trasportatori. Attraverso esocitosi i cammini 
secretori portano nuovi proteine, carboidrati e lipidi sintetizzati alla membrana plasmatica o allo spazio extracellulare. Attraverso endocitosi rimuovono componenti della membrana 
plasmatica e li portano verso endosomi dove posano essere riportati a regioni della membrana plasmatica o verso i lisosomi per la degradazione. L'esocitosi viene utilizzata per 
catturare nutrienti. Il lume di ogni compartimento lungo i cammini secretori o endocitici \`e equivalente al lume di altri compartimenti o all'esterno cellulare. I contenitori
di trasporto sono formati dal compartimento donatore e da vescicole sferiche piccole o altre larghe e irregolari o tubuli, riferite come vescicole di trasporto. In una cellula 
eucariotica le vescicole di trasporto si separano da una membrana per unirsi a un'altra trasportando componenti di membrana e molecole lumenali dette cargo. Il traffico vescicolare
fluisce lungo cammini organizzati che permettono secrezione, nutrimento e rimodellamento della membrana plasmatica e degli organelli. Il cammino secretorio porta fuori dal reticolo 
endoplasmatico verso l'apparato di Golgi e la superficie cellulare, mentre il cammino endocitico porta all'interno dalla membrana plasmatica. In ogni caso cammini di riporto equilibrano
i flussi delle membrane nelle direzioni opposte riportando indietro certe molecole. Per far questo sono necessarie vescicole di trasporto selettive. 
\section{I meccanismi molecolari di trasporto di membrana e il mantenimento della diversit\`a compartimentale}
