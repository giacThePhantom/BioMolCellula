\chapter{Traffico di membrana intracellulare}
Le cellule per nutrirsi, comunicare e rispondere a cambi nell'ambiente aggiustano la composizione della loro membrana plasmatica e compartimenti interni in risposta alle loro 
necessit\`a. Usano un sistema di membrane interne per aggiungere e rimuovere proteine di superficie come recettori, canali ionici e trasportatori. Attraverso esocitosi i cammini 
secretori portano nuovi proteine, carboidrati e lipidi sintetizzati alla membrana plasmatica o allo spazio extracellulare. Attraverso endocitosi rimuovono componenti della membrana 
plasmatica e li portano verso endosomi dove posano essere riportati a regioni della membrana plasmatica o verso i lisosomi per la degradazione. L'esocitosi viene utilizzata per 
catturare nutrienti. Il lume di ogni compartimento lungo i cammini secretori o endocitici \`e equivalente al lume di altri compartimenti o all'esterno cellulare. I contenitori
di trasporto sono formati dal compartimento donatore e da vescicole sferiche piccole o altre larghe e irregolari o tubuli, riferite come vescicole di trasporto. In una cellula 
eucariotica le vescicole di trasporto si separano da una membrana per unirsi a un'altra trasportando componenti di membrana e molecole lumenali dette cargo. Il traffico vescicolare
fluisce lungo cammini organizzati che permettono secrezione, nutrimento e rimodellamento della membrana plasmatica e degli organelli. Il cammino secretorio porta fuori dal reticolo 
endoplasmatico verso l'apparato di Golgi e la superficie cellulare, mentre il cammino endocitico porta all'interno dalla membrana plasmatica. In ogni caso cammini di riporto equilibrano
i flussi delle membrane nelle direzioni opposte riportando indietro certe molecole. Per far questo sono necessarie vescicole di trasporto selettive. 
\section{I meccanismi molecolari di trasporto di membrana e il mantenimento della diversit\`a compartimentale}
Il trasporto attraverso vescicole media uno scambio continuo di componenti tra i pi\`u di $10$ compartimenti chimicamente distinti che comprendono i cammini secretori ed endocitici. 
Ogni compartimento \`e capace di mantenere la propria identit\`a grazie alla composizione della membrana: marcatori sul lato citosilico servono come guida per il traffico in arrivo e 
una loro combinazione d\`a ad ogni compartimento il suo indirizzo molecolare. Le operazioni di separazione e trasferimento di parti specifiche delle membrane permettono di mantenere
alte o basse concentrazioni di marcatori su un compartimento. 
\subsection{Ci sono vari tipi di vescicole incapsulate}
La maggior parte delle vescicole si forma da regioni di membrana specializzate. Si separano come vescicole incapsulate, con una specifica gabbia proteica che copre la superficie 
citosilica. Prima che le vescicole si fondano con una membrana questa gabbia viene eliminata. La capsula ha una struttura a due strati che riflette le due funzioni: uno strato interno
concentra proteine di membrana in superfici specializzate che danno origine alla membrana vescicolare in modo da selezionare le molecole di membrana appropriate per il trasporto. Lo
strato esterno si assembla in un lattice curvo che deforma la superficie di membrana e d\`a forma alla vescicola. Ci sono tre tipi di vescicole incappucciate: incapsulata da clatrina, da
COPI e da COPII. Ogni tipo \`e utilizzata per diversi passi del trasporto. Le vescicole incapsulate da clatrina mediano il trasporto dall'apparato di Golgi e dalla membrana plasmatica,
mentre quelle incapsulate da COPI e da COPII il trasporto dall'ER e dalle cisterne di Golgi. 
\subsection{L'assemblaggio di una capsula di clatrina guida la formazione della vescicola}
Le vescicole incapsulate da clatrina trasportano materiale dalla membrana plasmatica e tra compartimenti endosomiali e di Golgi. Quelle incapsulate da COPI e COPII trasportano materiale
presto nei cammini secretori: la prima si separa dai compartimenti di Golgi, mentre la seconda dall'ER. Il maggior componente proteico delle vescicole incapsulate da clatrina \`e la 
clatrina che forma lo strato esterno. Ogni subunit\`a clatrinica consiste di tre grandi e tre piccole catene polipeptidiche che formano una struttura a tre gambe detta triskelion che
a sua volta si assembla in una struttura di esagoni e pentagoni simile a un cesto che forma pozzi incapsulati sulla superficie citosilica della membrana. Sotto appropriate condizioni
i triskelioni isolati si assemblano in gabbie poliedriche e determinano la geometria della gabbia clatrinica.
\subsection{Proteine adattatrici selezionano il cargo nelle vescicole incapsulate a clatrina}
Le proteine adattatrici sono un altro componente della capsula formano uno strato interno della capsula tra la gabbia clatrinica e la membrana. Legano la gabbia alla membrana e 
intrappolano varie proteine transmembrana come recettori che capturano le molecole di cargo all'interno della vescicola e permettono la selezione di un insieme di proteine trasmembrana
e delle proteine che interagiscono con loro e le impacchettano all'interno della nuova vescicola di trasporto. Ci sono diversi tipi di proteine adattatrici, quelle meglio caratterizzate
possiedono quattro subunimt\`a proteiche, altre proteine a singola catena. Ogni tipo \`e specifico a un insieme di recettori di cargo. L'assemblaggio di proteine adattrici \`e 
strettamente controllao dalla loro interazione con altre componenti della capsula. AP2 quando si lega a un lipide fosfatidiloinositolo fosforilato altera la sua conformazione esponendo 
siti di legame per recettori di cargo nella membrana. Il legame simultaneo ai recettori di cargo e al gruppo di testa lipidico aumento il legame di AP2 alla membrana. La proteina
agisce come un individuatore di coincidenze che si assembla solo al tempo e spazio giusto. Quando si legano inducono curvatura di membrana che rende causa un circuito a feedback 
positivo amplificato dal legame della clatrina che porta alla formazione e separazione della vescicola. 
\subsection{I fosfonositidi marcano gli organelli e i domini di membrana}
I fosfolipidi inositoli hanno un'importante funzione regolatoria in quanto possono svolgere rapidi cicli di fosforilazione e defosforilazione alle posizioni $3'$, $4'$ e $5'$ dei gruppi
di testa inositolo e produrre vari tipi di fosfoinositidi (PIP). L'interconversione di fosfatidilinositolo e PIP \`e diversa tra i compartimenti e la distribuzione, regolazione e 
equilibrio  determina lo stato di distribuzione di ogni specie di PIP che varia tra gli organelli e spesso tra regioni continue di membrana. Molte proteine coinvolte nel trasporto 
vescicolare contengono domini che si legano  ai gruppi di testa di particolari PIP. Il controllo locale di PI, PIP chinasi e PIP fosfatasi pu\`o essere usato per controllare il legame 
di proteine a una membrana. La produzione di un PIP recluta proteine contenenti domini leganti a PIP corrispondenti che aiutano a regolare la formazione di vescicole e altri passi nel
loro traffico. 
\subsection{Proteine piegatrici della membrana aiutano a deformare la membrana durante la formazione delle vescicole}
La forza generata dall'assemblaggio della clarina non \`e sufficiente per dare forza e separare la vescicola dalla membrana: altre proteine partecipano a ogni passo del processo:
proteine piegatrici della membrana che contenono  domini a mezzaluna o domini BAT legano e impongono la loro forma alla membrana sottostante attraverso interazioni elettrostatiche e 
aiutano AP2 a nucelare endocitosi dando una forma alla membrana plasmatica in modo che la vescicola si formi. Contengono eliche anfipatiche che inducono una curvatura di membrana
dopo essere state inserite nel lato citoplasmatico della membrana.
\subsection{Proteine citoplasmatiche regolano la separazione e decapsulamento delle vescicole}
Mentre la bolla incapsulata da clatrina cresce la dinamina si assembla al collo di ogni bolla. Dinamina contiene un dominio che si lega a $PI(4,5)$ $P_2$ che unisce la proteina alla
membrana e un dominio GTPasi che regola il tasso di separazione delle vescicole dalla membrana. IL processo di separazione unisce i due lati non citosilici della membrana in prossimit\`a
e li fonde isolando la vescicola che si sta formando. La dinamina recluta altre proteine al collo della bolla che aiutano a riparare la membrana distorcendo la struttura bistrato o 
cambiando la sua composizione lipidica o entrambe le cose. Una volta rilasciata la vescicola perde la capsula di clatrina: una fosfatasi PIP co-impacchettata nelle vescicole esaurisce
$PI(4, 5$ $P_2$ dalla membrana che indebolisce il legame delle proteine adattatrici. Oltre a questo una proteina hsp70 accompagnatrice che funziona come un'ATPasi decapsulatrice, usando
l'idrolisi dell'ATP. Il rilascio non deve avvenire prematuramente e si rendono necessari controlli ausiliari in modo che non venga rimossa prima che abbia formato una vescicola completa.
\subsection{GTPasi monomerica controlla l'assemblaggio della capsula}
Per equilibrare il traffico da e per un compartimento le proteine di capsula si devono assemblare solo quando necessario. Se la produzione locale di PIPs ha un ruolo centrale nella
regolazione dell'assemblaggio delle capsule di clatrina sulla membrana plasmatica e sugli apparati di Golgi la cellula superimpone modi addizionail di regolare la formazione della
capsula. GTPasi di reclutamento della capsula controllano l'assemblaggio di capsule di clatrina sugli endosomi e le capsule CPOI e COPII sulle membrane ER e di Golgi. Molti passaggi
del trasporto vescicolare dipendono da una variet\`a di proteine che si legano a GTP regolano la maggior parte dei processi nelle cellule eucariotiche. Agiscono come interruttori 
molecolari tra due stati con quello attivo con GTP legato e inattivo con GDP legato. Il cambio di stato \`e regolato da fattori di scambio del nucleotide guanina (GEF) che attivano la
proteina catalizzando lo scambio di GDP per GTP e le proteine attivatrici GTPasi disattivano la proteina causando l'idrolisi del GTP in GDP. Le GTPasi di reclutamento della capsula
sono membri di una famiglia di GTPasi monomeriche come le proteine ARF, responsabili per l'assemblaggio di COPY e della capsula clatrinica alle membrane di Golgi e la proteina Sra1, 
responsabile per l'assemblaggio delle capsule COPII alla membrana ER. Si trovano ad alte concentrazioni nel citosol in stato inattivo. Quando una vescicola incapsulata in COPII deve
separarsi dalla membrnaa ER una Sar1-GEF lega al Sar1 citosilico causando il suo cambio di GDP in GTP. In questo stato Sar1 espone un'elica anfipatica che si iniferice nel lato
citoplasmatico del bistrato della membrana ER. Sar1 ora strettamente legata recluta subunit\`a di proiteine adattatrici alla membrana ER per iniziare la separazione. Le GTPasi
di reclutamento della capsula hanno anche un ruolo nel disassemblaggio della capsula: l'idrolisi di GTP causa un cambio conformazionale che causa un'uscita della sua coda idrofobica
dalla membrana causando un disassemblaggio della capsula. COPII accellera l'idrolisi del GTP in Sar1 e una vescicola completamente formata viene prodotta solo quando la formazione
del sistema di separazione avviene pi\`u velocemente rispetto al processo di disassemblaggio temporizzato. Una volta che la vescicola si separa l'idrolisi del GTP rilascia Sar1 ma
la capsula isolata \`e abbastanza stabilizzata da interazioni cooperative in modo che sia presente fino a che la vescicola arrivi alla membrana obiettivo dove una chinasi fosforila
la capsula proteica che la disassembla e prepara la vescicola per la fusione. Le vescicole incapsulate da clatrina e da COPI perdono la capsula poco dopo che si separano. Per le seconde
la curvatura della membrana della vescicola causa l'inizio del decapsulamento. Un ARF-GAP viene reclutato alla COPI mentre si assembla e interagisce con la membrana percependo 
la densit\`a di impacchettamento dei lipidi. Si attiva quando la curvatura raggiunge quella di una vescicola di trasporto e disattiva ARF, causando il disassemblaggio della capsula.
\subsection{Non tutte le vescicole di trasporto sono sferiche}
Ogni membrana risponde diversamente alla creazione di vescicole: nella membrana plasmatica, piatta e rigida a causa del colesterolo, l'azione coordinata di capsule di clatrina e
di proteine che piegano la membrana deve produrre forza sufficiente per l'introduzione di curvatura, specialmente al collo. Per le membrane intracellulari avviene invece in regioni
gi\`a curve dove la funzione primaria della capsula \`e catturare il cargo appropriato. Le vescicole di trasporto sono create in varie forme e dimensioni. Molte vescicole COPII sono
richieste per il trasporto di grande molecole di cargo come il collagene che non entra nelle normali vescicole: la molecola di cargo si lega a proteine di impacchettamento nell'ER che
guidano l'assemblaggio di vescicole COPII molto pi\`u grandi che permettono l'entrata di un cargo molto pi\`u grande. Negli endosomi e nella rete di Golgi trans si trovano lunghi tubuli
che reclutano cargo e si separano con un grande rapporto superficie/volume e sono ricchi di proteine di membrana. 
\subsection{Le proteine Rab guidano le vescicole di trasporto alle loro membrane obiettivo}
Le vescicole di trasporto sono estremamente selettive verso la membrana obiettivo, caratteristica assicurata in quanto tutte presentano marcatori superficiali che le identificano 
secondo la loro origine e il loro cargo e le membrane obiettivo presentano recettori complementari che li riconoscono. Prima di tutto proteine Rab e effettori Rab portano la 
vescicola a posti specifici sulla membrana obiettivo e proteine e regolatori SNARE mediano la fusione del bistrato lipidico. Le proteine Rab causano la specificit\`a del trasporto
vescicolare e sono GTPasi monomeriche e ognuna di essa \`e associata con uno o pi\`u organelli racchiusi da membrana dei cammini endocitici o secretori che ne possiede almeno una sulla
sua superficie. L'alta selettiva distribuzione su queste membrane le rendi ideali come marcatori per identificare i tipi di membrana e guidare il traffico di vescicole tra di loro. 
Possono funzionare sulle vescicole di trasporto, sulle membrane obiettivo o entrambi. Ciclano tra la membrana e il citosol e regolano l'assemblaggio reversibile di complessi proteici
sulla membrana. Nello stato legato a GDP sono inattive e  legate ad altre proteine come i Rab-GDP dissociation inhibitor (GDI) che le mantiene solubili nel citosol. Nello stato legato
a TP sono attive e associate strettamente con la membrana di un organello o un vescicola di trasporto. Rab-GEF legate a membrana attivano le proteine Rab sulle vescicole e sulle
membrane obiettivo e per alcune fusioni sono richieste entrambe. Nello stato legato a GTP e legate a membrana attraverso un ancora si legano ad altre proteine o Rab effettori che
sono i mediatori a valle del trasporto vescicolare, filamento di proteine e fusione di membrane. Il tasso dell'idrolisi del GTP determina la concentrazione di Rab attive e degli 
effettori sulla membrana. Le strutture degli effettori sono varie e le stesse proteine possono legarsi a diversi effettori. Alcuni effettori sono proteine motrici, altre leganti
che possono collegare due membrane. Il complesso che fa attraccare vescicole COPII contiene una chinasi che fosforila la capsula per completare il processo di decapsulamento. 
L'accoppiamento del decapsulamento con la consegna delle vescicole aiuta ad assicurare la direzionalit\`a e la fusione ocn la propria membrana. Gli effettori Rab possono anche 
interagire con SNARE per accoppiare il filamento della membrana alla fusione. L'assemblaggio delle proteine Rab e i loro effettori \`e il risultato di aree di membrana: Rab5 concentra il
processo di proteine di legame che catturano vescicole in arrivo. Il suo assemblaggio comincia quando un complesso Rab5-GDP/GDI incontra un Rab-GEF: GDI \`e rilasciato e viene convertito
in Rab5-GTP. Rab5-GTP attivo si ancora alla membrana e recluta pi\`u Rab-GEF all'endosoma. Attiva inoltre una PI 3-chinasi che converte PI in PI(3)P che lega alcuni effettori Rab come
proteine di legame stabilizzando il loro legame con la membrana.
\subsection{Cascate di Rab possono cambiare l'identit\`a di un organello}
Un dominio Rab pu\`o essere sostituito cambiando l'identit\`a di un organdello: tale reclutamento di proteine Rab sequenziale \`e detto cascata di Rab: i domini Rab5 sono sostituiti
da domini Rab7 sulle membrane endosomiche, trasformandolo da endosoma giovane a vecchio alterandone le dinamiche di membrana e la sua posizione (maturazione endosomica), in un processo
irreversibile e unidirezionale.
\subsection{Le SNARE mediano la fusione di membrana}
Una volta che una vescicola di trasporto \`e stata legata alla membrana obiettivo scarica il cargo attraverso fusione di membrana in cui i bistrati si avvicinano fino a che possono
unirsi. Quando le membrane si trovano a $1.5nm$ i lipidi possono fluire tra un bistrato all'altro e l'acqua deve essere spostata dalla superficie idrofilica grazie a proteine di
fusione che superano la barriera energetica necessaria. Le proteine SNARE catalizzano le reazioni di fusione di membrana, sono specifiche all'organello, sono proteine transmembrana che
esistono come insiemi complementari: v-SNARE sulle vescicole e t-SNARE su quelle obiettivo. v-SNARE sono singole catene polippetidiche, mentre t-SNARE sono composte da tre proteine.
Hanno entrambe domini elicali e quando interagiscono (con alta specificit\`a) si arrotolano tra di loro formando un insieme di quattro eliche molto stabile. Il complesso trans-SNARE 
risultante blocca insieme le due membrane e catalizza la fusione utilizzando l'energia liberata dall'interazione e arrotolamento delle eliche per portare le membrane insieme facendo
contemporaneamente uscire l'acqua. Nella catalizzazione cooperano anche effettori Rab. Le proteine Rab possono regolare la disponibilit\`a di t-SNARES che sono associate con
proteine inibitorie che devono essere prima rilasciate grazie a un segnale di una proteina Rab e del suo effettore. 
\subsection{SNARE interagenti devono essere separati prima che possano funzionare ancora}
I complessi trans-SNARE devono disassemblarsi prima di mediare nuovi round di trasporto. NSF cicla tra membrane e il citosol e catalizza il processo di disassemblaggio. \`E una ATPasi
esamerica della famiglia delle AAA-ATPasi che usano l'idrolisi per svolgere le interazioni tra i domini elicali delle proteine SNARE accoppiate. La richiesta di questo catalizzatore
aiuta ad impedire una fusione indiscriminata delle membrane. 
\section{Trasporto dall'ER attraverso l'apparato di Golgi}
Le proteine sintetizzate attraversano la membrana ER dal citosol per entrare nel cammino secretorio. Durante il trasporto seguente dall'ER attraverso l'apparato di Golgi fino alla
superficie cellulare le proteine sono modificate mentre passano attraverso una serie di compartimenti: il trasferimento da uno all'altro coinvolge un equilibrio tra cammini di
trasporto in avanti e all'indietro. Alcune vescicole di trasporto selezionano il cargo e lo muovono nel prossimo compartimento, mentre altre recuperano proteine sfuggite e le
riportano a un compartimento precedente. L'apparato di Golgi \`e il sito principale della sintesi di carboidrati e una stazione di ordinamento per i prodotti dell'ER. Una grande
percentuale di carboidrati sono attaccati come oligosaccaridi a molte proteine e lipidi inviati dall'ER in modo da marcarli verso vescicole per i lisosomi, mentre la maggior parte
sono riconosciute in altri modi per trasportarle verso altre destinazioni.
\subsection{Le proteine lasciano l'ER in vescicole di trasporto incapsulate in COPII}
Per iniziare il viaggio lungo il cammino secretorio le proteine che entrano la membrana ER e sono destinate per l'apparato di Golgi o dopo di esso sono incapsulate in vescicole di 
trasporto incapsulate da COPII che si separano dal sito di uscita ER la cui membrana manca ribosomi e sono distribuite su tutta la superficie. L'entrata nelle vescicole pu\`o essere
uno processo selettivo o Pu\`o accadere di default. Molte proteine di membrana sono reclutate attivamente nella vescicola dove si concentrano e mostrano segnali di uscita sul lato 
citosilico che la proteina adattrice interna alla COPII riconosce. Alcune di questi componenti agiscono come recettrici di cargo e sono riciclate nell'ER dopo il trasporto. Proteine
di cargo solubili nel lume ER hanno segnali di uscita che le attaccano a recettori di cargo transmembrana. Nella vescicola possono anche entrare proteine senza il segnale di uscita come
le proteine ER residenti alcune delle quali sfuggono lentamente dall'ER e sono trasportate all'apparato di Golgi. Proteine di cargo diverse entrano nelle vescicole a tassi diversi
a causa di diverse piegature, efficienze di oligomerizzazione e cinetica. Il passo di uscita \`e il punto di controllo sule proteine che la cella secerne. Il segnale di uscita che 
porta fuori le proteine solubili dall'ER non sono compresi: alcuni recettori del cargo sono lectine che legano gli olisaccaridi sulla mannosio della proteina secreta.
\subsection{Solo proteine appropriatamente piegate e assemblate possono lasciare l'ER}
Per uscire dall'ER le proteine devono essere piegate e se sono proteine di complessi multi-proteici devono essere completamente assemblati. Quelle che non lo sono rimangono nell'Er dove
sono legate da proteine accompagnatrici come BiP o calnexina che possono coprire i segnali di uscita o ancorarle nell'ER. Tali proteine sono trasportate nel citosol dove sono degradate
da proteosomi in modo da prevenire ulteriore trasporto in avanti di proteine errate che possono interferire nel comportamento di quelle corrette. In molti casi la cellula crea delle
proteine in gran eccesso in modo che alcune possano piegarsi, assemblarsi e funzionare in maniera corretta. 
\subsection{Raggruppamenti vescicolari tubulari mediano il trasporto dall'ER all'apparato di Golgi}
Dopo che le vescicole di trasporto si sono separate dal sito di uscita ER e hanno perso la loro capsula cominciano a fondersi tra di loro attraverso fusione omotipica che richiede un
insieme di SNARE complementari. In questo caso l'interazione \`e simmetrica con entrambe contribuenti v-SNARE e t-SNARE. La struttura che si forma \`e detta di raggruppamenti vescicolari
tubulari a causa delle loro apparenza. Questi raggruppamenti costituiscono un compartimento separato dall'ER e a cui mancano molte delle proteine che vi si trovano. Sono continuamente
generati e funzionano come contenitori di trasporto che portano materiale dall'ER all'apparato di Golgi. I raggruppamenti si muovono velocemente lungo microtubuli verso l'apparato di 
Golgi con cui si fondono. Appena si formano cominciano a separare da loro vescicole di trasporto  incapsulate da COPI uniche in quanto i componenti che creano gli strati della
capsula sono reclutati come un unico elemento (coatomero). Funzionano come cammini di recupero trasportando proteine residenti nell'ER che sono sfuggite come recettori del 
cargo e SNARE che partecipano nella separazione e fusione delle vescicole. L'assemblaggio COPI inizia secondi dopo la separazione della capsula COPII e non si sa come venga controllato.
Il trasporto inverso continua mentre i raggruppamenti si muovono verso l'apparato di Golgi. I raggruppamenti maturano continuamente cambiando la propria composizione mentre le proteine
selezionate sono ritornate all'ER, cosa che continua dall'apparato di Golgi. 
\subsection{Il cammino di recupero all'ER usa segnali di ordinamento}
Il cammino per far ritornare proteine nell'ER dipende da segnali di recupero ER: le proteine residenti nell'ER contengono segnali che si legano direttamente alle capsule COPI e sono
impacchettate in vescicole di trasporto incapsulate da COPI per la consegna inversa all'ER. Il segnale meglio caratterizzato consiste di due lisine e qualsiasi altra coppia di 
amminoacidi alla terminazione C della proteina di membrana ER o una sequenza KKXX. Inoltre proteine solubili come BiP contengono un corto segnale di recupero alla terminazione C
con una sequenza Lys-Asp-Glu-Leu o simile. Questa sequenza KDEL causa il ritorno efficiente della proteina nell'ER dove \`e accumulata. Le proteine residenti nell'ER solubili devono 
legarsi a recettori come il recettore KDEL, una proteina transmembrana a multipassaggio che si lega alla sequenza e la impacchetta nelle vescicole per il trasporto inverso. Tale 
recettore cicla tra l'ER e l'apparato di Golgi con diverse affinit\`a per la sequenza: pi\`u alta nei raggruppamenti di vescicole tubulari e nell'apparato di Golgi e pi\`u bassa
nell'ER. Il cambio di affinit\`a \`e dovuto al minor pH nell'apparato di Golgi regolato da pompe \ce{H+}. Nel cammino inverso entrano anche le proteine che funzionano da interfaccia
tra l'ER e l'apparato di Golgi.
\subsection{Molte proteine sono selettivamente trattenute nei compartimenti dove funzionano}
Sembra esista un meccanismo indipendente dal segnale KDEL che trattiene le proteine residenti nell'ER. Un meccanismo ipotizzato \`e che le proteine residenti si leghino tra di loro 
formando complessi troppo grandi per entrare nelle vescicole. 
\subsection{L'apparato di Golfi consiste di una serie ordinata di compartimenti}
L'apparato di Golgi consiste di cisterne racchiuse da membrana appiattite organizzate in stack da $4$ a $6$ elementi. Nelle cellule animali connessioni tubulari tra cisterne 
corrispondenti collegano gli stack formando un complesso singolo vicino a nucleo e al centrosoma, localizzazione che dipende dai microtubuli. Durante il passaggio attraverso l'apparato 
di Golgi le molecole subiscono una serie di modifiche covalenti: ogni stack possiede una faccia di entrata (cis) e una di uscita (trans), entrambe sono associate con compartimenti
composti di una rete di strutture microtubulari e cisternali: la rete cis Golgi (CGN) e la rete trans Golgi (TGN). La prima \`e una collezione di raggruppamenti tubulari vescicolari 
fusi che arrivano dall'ER. Proteine e lipidi entrano dalla rete cis Golgi ed escono dalla rete trans Golgi verso un altro compartimento o la superficie cellulare. Entrambe le reti
ordinano le proteine: quelle che entrano nel CGN possono andare avanti o essere ritornate nell'ER e quelle uscenti dal TGN si muovono in avanti e sono ordinate rispetto alla loro 
direzione: endosomi, vescicole secretorie, la superficie cellulare o un compartimento precedente. Alcune proteine di membrana sono trattenute nella parte dell'apparato dove funzionano.
Si nota come un oligosaccaride legato a \ce{N} \`e ltato a molte proteine nell'ER e rifilato mentre vi risiede. Gli oligosaccaridi intermedi aiutano il piegamento delle proteine e
il trasporto di quelle mal piegate al citosol per la degradazione. Una volta che le proteine escono l'oligosaccaride viene utilizzato dall'apparato di Golgi per generare 
strutture oligosaccaride eterogenee viste nelle proteine mature. Dopo l'arrivo nel CGN le proteine entrano nella cisterna cis Golgi per un primo processamento. Sono poi mosse alla
cisterna mediale e infine alla cisterna trans dove viene completata la glicosilazione. Il lume della cisterna trans \`e continuo con il TGN. Il processamento degli oligosaccaridi
avviene per passi in una sequenza ordinata nello stack, con ogni cisterna contenente un insieme caratteristico di enzimi. Le proteine sono modificate in stati successivi mentre
passano attraverso lo stack in modo che formi un'unit\`a di processamento a multi stato. 
\subsection{Le catene oligosaccaridi sono processate nell'apparato di Golgi}
Le proteine residenti nell'apparato di Golgi sono tutte legate dalla membrana: tutte le glicosidasi e glicosil-trasferasi sono proteine transmembrana a singolo passaggio, molte 
organizzate in complessi multienzima. Due vaste classi di oligosaccaridi legati a \ce{N}, gli oligosaccaridi complessi e gli oligosaccaridi a mannosio alto sono attaccati alle
glicoproteine mammifere. I primi sono generati quando nell'ER l'oligosaccaride originale \`e rifilato e altri zuccheri sono aggiunti e contiene una carica negativa, mentre i secondi sono
rifilati ma senza addizioni di zuccheri. Il passaggio da oligosaccaride ad alto mannosio a complesso dipende dalla sua posizione sulla proteina: se l'oligosaccaride \`e accessibile
agli enzimi nell'apparato di Golgi viene convertito nella forma complessa, altrimenti no. 
\subsection{I proteoglicani sono assemblati nell'apparato di Golgi}
Molte proteine sono ulteriormente modificate mentre passano attraverso l'apparato di Golgi: ad alcune vengono aggiunti zuccheri ai gruppi idrossili di serine o treonine, o come nel 
collagene a proline o lisine idrossilate. Questa glicosilazione legata a \ce{O} \`e catalizzata da una serie di glicosil-trasverasi che usano zuccheri nucleotidi nel lume dell'apparato
per aggiungerli uno alla vulta. Tipicamente \`e aggiunta prima \ce{N-}acetilgalactosamina, seguita da un numero di zuccheri addizionali. La glicosilazione legata a \ce{O} pi\`u pesante
avviene sulle mucine e sulle proteine del nucleo proteoglicano che modifica per produrre proteoglicani attraverso la polimerizzazione di catene di glicosamminoglicano su serine di una
proteina nucleo. Molti di questi sono secreti e diventano parte della matrice extracellulare, mentre altri rimangono ancorati alla superficie extracellulare della membrana plasmatica. 
Gli zuccheri aggiunti sono sulfati dopo la creazione dei polimeri dando loro una forte carica negativa. 
\subsection{Il ruolo della glicosilazione}
I carboidrati complessi richiedono pi\`u enizimi diversi ado gni passo, ognuno riconosciuto unicamente come il substrato esclusivo per il passo successivo. La glicosilazione legata a 
\ce{N} \`e prevalente in tutti gli eucarioti e promuovono il piegamento delle proteine rendendo gli intermedi pi\`u solubili prevenendo ulteriore aggregazioni e le modificazioni 
sequenziali creano un codice che marca la progressione del pieegamento e media il legame di proteine ad accompagnatori e lectine. In quanto le catene di zuccheri hanno una flessibilit\`a
limitata possono limitare l'avvicinarsi di altre macromolecole alla superficie proteica rendendole pi\`u resistenti alla digestione attraverso enzimi proteolitici. Le catene di 
zuccheri sevno anche come protezione di patogeni nella forma di coperture di muco nelle cellule di polmoni e intestinali. Il riconoscimento di catene cellulari da parte della lectina
nello spazio extracellulare serve nel processo di sviluppo e nel riconoscimento tra cellule. La presenza di oligosaccaridi pu\`o modificare la funzione e propriet\`a antigenica di 
una proteina. Pu\`o avere anche ruoli regolatori. 
\subsection{Il trasporto attraverso l'apparato di Golgi potrebbe accadere attraverso maturazione cisternale}
\`E ancora incerto come l'apparato di Golgi mantenga la sua struttura polarizzata e come le molecole si muovano tra una cisterna e l'altra. Un ipotesi della del modello di maturazione
cisternale vede le cisterne come strutture dinamiche che maturano ottenendo e perdendo specifiche proteine residenti del Golgi. Le nuove cis cisterne si formano continuamente come
raggruppamenti vescicolari tubulari che arrivano dall'ER e maturano per diventare ciserne mediali e poi trans. Un cisterna si muove attraverso lo stack con il cargo nel suo lume. Il
trasporto inverso attraverso separazione di vescicole inpasulate da COPI spiega la distribuzione caratteristica. Secondo il modello di trasporto di vescicole il flusso di vescicole
retrograde recupera le proteine e le ritorna nel compartimento a monte. Il flusso direzionale \`e ottenuto attraverso l'impacchettamento di molecole di cargo che si muovono in avanti
in vescicole specifiche. Le due vescicole diverse sono riconosciute attraverso proteine adattrici diverse. Probabilmente aspetti di entrambi i modelli sono veri. 
\subsection{La matrice di proteine del Golgi aiuta ad organizzare lo stack}
L'architettura dell'apparato di Golgi dipende dal citoscheletro di microtubili e dalle proteine citoplasmatiche della matrice di Golgi che formano un'impalcatura tra cisterne adiacenti
e danno l'integrit\`a strutturale allo stack. Alcune di queste dette golgine formano lunghi filamenti composti di domini rigidi a bobina tra regioni cardine intersperse, insieme formano
un insieme di tentacoli che aiutano a mantenere vicino al Golgi vescicole di trasporto attraverso interazioni con proteine Rab. Quando la cellula si prepara a dividersi 
proteine chinasi mitotiche fosforilano tali proteine causando la frammentazione dell'apparato e la sua dispersione nel citosol. I frammenti sono poi distribuiti nelle cellule figlio
in ugual misura dove sono defosforilate permettendo il riassemblaggio dello stack. 
\section{Il trasporto dalla rete trans Golgi ai lisosomi}
La rete trans Golgi ordina tutte le proteine che passano attraverso l'apparato secondo la destinazione attraverso un processo selettivo come avviene per quelle destinate al lume dei 
lisosomi.
\subsection{I lisosomi sono il sito principale della digestione intracellulare}
I lisosomi sono organelli chiusi da membrana riempiti da enzimi idrolitici che digeriscono macromolecole, contengono proteasi, nuclasi, glicosidasi, lipasi, fosfolipasi, fosfopatasi e 
sulfatasi. Sono tutte idrolasi acide (lavorano a pH acidi). Per attivit\`a ottima devono essere attivati attraverso rottura proteolitica che richiede anche un ambiente acido. Il lisosoma
mantiene un pH di $4.5-5.0$. I contenuti del citosol sono protetti attraverso attacchi dal sistema digestivo: la membrana del lisosoma mantiene gli enzimi digestivi fuori dal citosol
ma, anche in caso di uscita possono fare pochi danni al pH citosilico ($7.2$ pH). Anche la membrana \`e caratteristica> la maggior parte delle proteine sono altamente glicolisate in 
modo da proteggerle dalle proteasi nel lume. Le proteine di trasporto portano fuori dal lisosoma i prodotti finali della digestione delle macromolecola (amminoacidi, zuccheri e 
nucleotidi) al citosol dove vengono utilizzati dalla cellula o secreti. Un \ce{H+} ATPasi vacuolare nella membrana del lisosoma usa l'energia dell'idrolisi dell'ATP per pompare \ce{H+}
nel lisosoma mantenendo il lume al pH acidico. Tali pompe appartengono alle ATPasi v-tipo e lavorano solo pompando \ce{H+} nell'organello. Pompe simili acidificano tutti gli organelli
endocitici o esocitici. Oltre a creare l'ambiente acido il gradiente di \ce{H+} fornisce una fonte di energia per guidare il trasporto dei piccoli metaboliti attraverso la membrana. 
\subsection{I lisosomi sono eterogenei}
I lisosomi hanno una morfologia eterogenea che riflette le varie funzioni digestive che le idrolasi acide mediano e il modo in cui si formano. Gli endosomi maturi contenti materiale
ricevuto dalla membrana plasmatica e nuove idrolasi lisosomiali si fondono con lisosomi preesistenti per formare endolisosomi che si fondono poi tra di loro. Quando il materiale 
endocitato \`e stato digerito dall'endolisosoma questo ritorna un lisosoma classico, organelli piccoli, densi e sferici che possono ricominciare il ciclo di fusione con gli endosomi.
Per questa ragione sono visti come un insieme eterogeneo di organelli distinti con la caratteristica comune di avere un alto contenuto di enzimi idrolitici.
\subsection{I vacuoli delle piante e dei funghi sono lisosomi versatili}
La maggior parte delle cellule delle piante e dei funghi contengono grandi vescicole riempite da liquido dette vacuoli, imparentati con i lisosomi animali e contenenti un gran numero di
enzimi idrolitici, pur avendo funzione diversa. Il vacuolo della pianta pu\`o agire come un organello di conservazione per nutrienti e prodotti di scarto, un compartimento degradatorio,
un modo per aumentare la dimensione della cellula e come controllore della pressione di turgore. \`E un importante dispositivo omeostatico che rende la pianta capace di sopportare 
cambi dell'ambiente.
\subsection{Multipli cammini portano materiali ai lisosomi}
I lisosomi sono punti in cui vari flussi di traffico intracellulare converge. La maggior parte degli enzimi digestivi arrivano dall'apparato di Golgi, mentre almeno quattro cammini 
portano sostanze per la digestione. Il pi\`u studiato \`e il cammino di degradazione delle macromolecole catturate dal fluido extracellulare attraverso endocitosi. Si trova un
cammino simile nelle cellule fagocitiche \`e dedicato alla cattura  o fagocitosi di grandi particelle e microorganismi per formare fagosomi. Un terzo cammino detto macropinocitosi si
specializza nel recupero non specifico di fluidi, membrane e particelle attaccate alla membrana plasmatica. Un quarto cammino detto autofagia viene utilizzato per digerire il citosol
e gli organelli consumati. 
\subsection{L'autofagia degrada proteine ed organelli non voluti}
Tutti i tipi cellulari si liberano di parti obsolete grazie ad un processo dipendente dai lisosomi: l'autofagia. \`E importante per la ristrutturazione delle cellule differenzianti e in
risposte adattive ad eventi esterni. Pu\`o rimuovere grandi oggetti come organelli. Nel passo iniziale dell'autofagia il cargo citoplasmatico viene circondato da una doppia membrana
che si assembla per fusione di piccole vescicole detta autofasogioma. Il processo avviene in $5$ passi:
\begin{enumerate}
	\item Induzione da attivazione di molecole di segnale: chinasi proteiche (complesso mTOR $1$) che inoltrano informazioni riguardo lo stato metabolico della cellula, si attivano
		e segnalano al macchinario autofago.
	\item Nucleazione ed estensione di una membrana delimitante in una forma a coppa, le vescicole di membrana, caratterizzate da ATG9 sono reclutate a un sito di assemblaggio dove
		nucleano la formazione dell'autofasogioma. ATG9 viene rimossa da un cammino di recupero.
	\item Chiusura della coppa di membrana intorno all'obiettivo per formare un autofasogioma isolato da una doppia membrana.
	\item Fusione con lisosomi catalizzata da SNARE.
	\item Digestione della membrana interna e dei contenuti del lume dell'autofasogioma.
\end{enumerate}
L'autofagia pu\`o essere selettiva o no. In quella non selettiva una porzione del citoplasma viene sequestrata da autofasogiomi. Pu\`o accadere in condizioni di mancanza di nutrienti. 
In quella selettiva viene isolato un cargo specifico che tende a contenere poco citosol, con una forma che riflette quella del cargo, media la degradazione di mitocondri, perossisomi, 
ribosomi e ER non voluti o consumati (anche per microbi invadenti). 
\subsection{Un recettore mannosio $\mathbf{6}$-fosfato ordina le idrolasi lisosomiali nella rete trans Golgi}
Si considera ora il cammino che porta le idrolasi lisosomiali dal TGN ai lisosomi. Gli enzimi sono prima portati a endosomi in vescicole di trasporto selettive che si separano dal 
TGN. Il riconoscimento di tali idrolasi avviene grazie a un marcatore nella fomra di gruppi di mannosio $6$-fosfato, aggiunti esclusivamente a oligosaccaridi legati a \ce{N} di questi
enzimi mentre passano attraverso il lume della rete cis Golgi. Proteine recettrici M6P riconoscono tali gruppi e legano al lato lumenale della membrana e a proteine adattrici in 
capsule di clatrina che si stanno costruendo sul lato citosolico. I recettori impacchettano le idrolasi in vescicole incapsulate da clatrina che si separano dal TGN e portano i loro
contenuti a giovani endosomi. La proteina recettrice di M6P si lega a esso a pH di $6.5-6.7$ nel TGN e si rilascia a pH $6$, presente nel lume degli endosomi. Dopo che il recettore
arriva in esso le idrolasi lisosomiali si dissociano da esso e il recettore viene ritrasportato indietro da vescicole che si separano dagli endosomi incapsualte da un retromero. 
Il trasporto in entrambe le direzioni richiede segnali nella coda citosilica del recettore M6P che direziona la proteina verso l'endosoma o il TGN: sono riconosciuti da un complesso 
retromero che recluta i recettori in vescicole di trasporto che si separano dall'endosoma. Il riciclo assomiglia a quello del recettore KDEL. Alcune molecole di idrolasi marcate 
raggiungono il lissooma: alcune sfuggono al processo di impacchettamento nella rete trans Golgi e vengono portate nel fluido extracellulare. Alcuni recettori M6P possono arrivare alla
membrana plasmatica dove ricatturano le idrolasi lisosomiali e le ritornano attraverso endocitosi mediata da recettore ai lisosomi attraverso endosomi giovani o maturi. Affinch\`e
funzionino i gruppi M6P devono essere aggiunti solo alle glicoproteine appropriate nell'apparato di Golgi, che richiede un riconoscimento specifico dell'idrolasi dagli enzimi che 
aggiungono M6P. Tale segnale deve essere presente nella catena polipeptidica di ogni idrolasi come raggruppamento di amminoacidi vicini sulla superficie della proteina. 
\subsection{Alcuni lisosomi e corpi multivescicolari subiscono esocitosi}
La secrezione lisosomiale di contenuto non digerito permette alla cellula di eliminare detriti non digeribili. Alcune cellule contengono lisosomi specializzati con il macchinario 
necessario per la fusione con la membrana plasmatica come i melanociti nella pelle. In alcune circostanze questo pu\`o accadere anche a corpi multivescicolari che rilasciano le
vescicole dalle cellule. Si osservano questi esosomi nel sangue e si pensa che trasportino componenti tra le cellule. 
\section{Il trasporto nella cellula dalla membrana plasmatica: endocitosi}
