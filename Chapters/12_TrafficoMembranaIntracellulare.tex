\chapter{Traffico di membrana intracellulare}
Le cellule per nutrirsi, comunicare e rispondere a cambi nell'ambiente aggiustano la composizione della loro membrana plasmatica e compartimenti interni in risposta alle loro 
necessit\`a. Usano un sistema di membrane interne per aggiungere e rimuovere proteine di superficie come recettori, canali ionici e trasportatori. Attraverso esocitosi i cammini 
secretori portano nuovi proteine, carboidrati e lipidi sintetizzati alla membrana plasmatica o allo spazio extracellulare. Attraverso endocitosi rimuovono componenti della membrana 
plasmatica e li portano verso endosomi dove posano essere riportati a regioni della membrana plasmatica o verso i lisosomi per la degradazione. L'esocitosi viene utilizzata per 
catturare nutrienti. Il lume di ogni compartimento lungo i cammini secretori o endocitici \`e equivalente al lume di altri compartimenti o all'esterno cellulare. I contenitori
di trasporto sono formati dal compartimento donatore e da vescicole sferiche piccole o altre larghe e irregolari o tubuli, riferite come vescicole di trasporto. In una cellula 
eucariotica le vescicole di trasporto si separano da una membrana per unirsi a un'altra trasportando componenti di membrana e molecole lumenali dette cargo. Il traffico vescicolare
fluisce lungo cammini organizzati che permettono secrezione, nutrimento e rimodellamento della membrana plasmatica e degli organelli. Il cammino secretorio porta fuori dal reticolo 
endoplasmatico verso l'apparato di Golgi e la superficie cellulare, mentre il cammino endocitico porta all'interno dalla membrana plasmatica. In ogni caso cammini di riporto equilibrano
i flussi delle membrane nelle direzioni opposte riportando indietro certe molecole. Per far questo sono necessarie vescicole di trasporto selettive. 
\section{I meccanismi molecolari di trasporto di membrana e il mantenimento della diversit\`a compartimentale}
Il trasporto attraverso vescicole media uno scambio continuo di componenti tra i pi\`u di $10$ compartimenti chimicamente distinti che comprendono i cammini secretori ed endocitici. 
Ogni compartimento \`e capace di mantenere la propria identit\`a grazie alla composizione della membrana: marcatori sul lato citosilico servono come guida per il traffico in arrivo e 
una loro combinazione d\`a ad ogni compartimento il suo indirizzo molecolare. Le operazioni di separazione e trasferimento di parti specifiche delle membrane permettono di mantenere
alte o basse concentrazioni di marcatori su un compartimento. 
\subsection{Ci sono vari tipi di vescicole incapsulate}
La maggior parte delle vescicole si forma da regioni di membrana specializzate. Si separano come vescicole incapsulate, con una specifica gabbia proteica che copre la superficie 
citosilica. Prima che le vescicole si fondano con una membrana questa gabbia viene eliminata. La capsula ha una struttura a due strati che riflette le due funzioni: uno strato interno
concentra proteine di membrana in superfici specializzate che danno origine alla membrana vescicolare in modo da selezionare le molecole di membrana appropriate per il trasporto. Lo
strato esterno si assembla in un lattice curvo che deforma la superficie di membrana e d\`a forma alla vescicola. Ci sono tre tipi di vescicole incappucciate: incapsulata da clatrina, da
COPI e da COPII. Ogni tipo \`e utilizzata per diversi passi del trasporto. Le vescicole incapsulate da clatrina mediano il trasporto dall'apparato di Golgi e dalla membrana plasmatica,
mentre quelle incapsulate da COPI e da COPII il trasporto dall'ER e dalle cisterne di Golgi. 
\subsection{L'assemblaggio di una capsula di clatrina guida la formazione della vescicola}
Le vescicole incapsulate da clatrina trasportano materiale dalla membrana plasmatica e tra compartimenti endosomiali e di Golgi. Quelle incapsulate da COPI e COPII trasportano materiale
presto nei cammini secretori: la prima si separa dai compartimenti di Golgi, mentre la seconda dall'ER. Il maggior componente proteico delle vescicole incapsulate da clatrina \`e la 
clatrina che forma lo strato esterno. Ogni subunit\`a clatrinica consiste di tre grandi e tre piccole catene polipeptidiche che formano una struttura a tre gambe detta triskelion che
a sua volta si assembla in una struttura di esagoni e pentagoni simile a un cesto che forma pozzi incapsulati sulla superficie citosilica della membrana. Sotto appropriate condizioni
i triskelioni isolati si assemblano in gabbie poliedriche e determinano la geometria della gabbia clatrinica.
\subsection{Proteine adattatrici selezionano il cargo nelle vescicole incapsulate a clatrina}
Le proteine adattatrici sono un altro componente della capsula formano uno strato interno della capsula tra la gabbia clatrinica e la membrana. Legano la gabbia alla membrana e 
intrappolano varie proteine transmembrana come recettori che capturano le molecole di cargo all'interno della vescicola e permettono la selezione di un insieme di proteine trasmembrana
e delle proteine che interagiscono con loro e le impacchettano all'interno della nuova vescicola di trasporto. Ci sono diversi tipi di proteine adattatrici, quelle meglio caratterizzate
possiedono quattro subunimt\`a proteiche, altre proteine a singola catena. Ogni tipo \`e specifico a un insieme di recettori di cargo. L'assemblaggio di proteine adattrici \`e 
strettamente controllao dalla loro interazione con altre componenti della capsula. AP2 quando si lega a un lipide fosfatidiloinositolo fosforilato altera la sua conformazione esponendo 
siti di legame per recettori di cargo nella membrana. Il legame simultaneo ai recettori di cargo e al gruppo di testa lipidico aumento il legame di AP2 alla membrana. La proteina
agisce come un individuatore di coincidenze che si assembla solo al tempo e spazio giusto. Quando si legano inducono curvatura di membrana che rende causa un circuito a feedback 
positivo amplificato dal legame della clatrina che porta alla formazione e separazione della vescicola. 
\subsection{I fosfonositidi marcano gli organelli e i domini di membrana}
I fosfolipidi inositoli hanno un'importante funzione regolatoria in quanto possono svolgere rapidi cicli di fosforilazione e defosforilazione alle posizioni $3'$, $4'$ e $5'$ dei gruppi
di testa inositolo e produrre vari tipi di fosfoinositidi (PIP). L'interconversione di fosfatidilinositolo e PIP \`e diversa tra i compartimenti e la distribuzione, regolazione e 
equilibrio  determina lo stato di distribuzione di ogni specie di PIP che varia tra gli organelli e spesso tra regioni continue di membrana. Molte proteine coinvolte nel trasporto 
vescicolare contengono domini che si legano  ai gruppi di testa di particolari PIP. Il controllo locale di PI, PIP chinasi e PIP fosfatasi pu\`o essere usato per controllare il legame 
di proteine a una membrana. La produzione di un PIP recluta proteine contenenti domini leganti a PIP corrispondenti che aiutano a regolare la formazione di vescicole e altri passi nel
loro traffico. 
\subsection{Proteine piegatrici della membrana aiutano a deformare la membrana durante la formazione delle vescicole}
La forza generata dall'assemblaggio della clarina non \`e sufficiente per dare forza e separare la vescicola dalla membrana: altre proteine partecipano a ogni passo del processo:
proteine piegatrici della membrana che contenono  domini a mezzaluna o domini BAT legano e impongono la loro forma alla membrana sottostante attraverso interazioni elettrostatiche e 
aiutano AP2 a nucelare endocitosi dando una forma alla membrana plasmatica in modo che la vescicola si formi. Contengono eliche anfipatiche che inducono una curvatura di membrana
dopo essere state inserite nel lato citoplasmatico della membrana.
\subsection{Proteine citoplasmatiche regolano la separazione e decapsulamento delle vescicole}
\begin{Huge}
	TO DO ? 
\end{Huge}

