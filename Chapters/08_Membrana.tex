\chapter{Membrana}

\section{Funzioni}
La membrana plasmatica funge da involucro della cellula ed \`e coinvolta negli scambi di sostanze.
Contiene infatti una grande variet\`a di trasportatori, proteine che attraversano la membrana trasportando sostanze nutrienti all'interno e prodotti di scarto all'esterno.

\section{Struttura}
La membrana si pu\`o considerare come un foglietto continuo di molecole e strutture fosfolipidiche spesso tra i $4$ e i $5nm$ che incorpora diverse proteine.
\`E estremamente fluida e i lipidi sono in grado di diffondersi nelle membrana.

	\subsection{Fase semisolida}
	A una certa temperatura la membrana assume uno stato di gel o fase semisolida.
	La temperatura varia in base alla composizione.
	Il colesterolo rende le membrane meno fluide.

	\subsection{Zattere lipidiche}
	Le zattere lipidiche hanno un ruolo funzionale in quanto permettono l'aggregazione di proteine e la creazione di domini con funzioni specifiche.
	Per esempio nell'epitelio la composizione proteica cambia tra la porzione apicale o baso-laterale.
	Giunzioni strette tra le cellule impediscono la diffusione delle proteine.

	\subsection{Asimmetria}
	Il doppio strato non \`e omogeneo: si trova asimmetria rispetto ai lipidi del monostrato interno ed esterno.
	Questo pu\`o variare in diverse condizioni grazie a flippasi.
	
		\subsubsection{Apoptosi}
		Durante l'apoptosi la fosfatidilserina viene spostata verso l'esterno causando il richiamo di macrofagi.

\section{Composizione}
	
	\subsection{Colesterolo}
	Il colesterolo \`e formato da un gruppo polare, una struttura steroidea rigida ad anello e una catena carburica non polare.
	Si trova nella membrana plasmatica e del reticolo endoplasmatico.
	Sono strutture che stabilizzano la membrana plasmatica.

	\subsection{Glicolipidi}
	I glicolipidi si trovano nel monostrato non citosolico e costituiscono il $5\%$ della membrana.
		
		\subsubsection{Gangliosidi}
		I gangliosidi sono oligosaccaridi con residui di acido sialico.
		Si trovano nel monostrato citosilico.
		Il ganglioside \emph{Gm1} permette l'ingresso della tossine del colera.

	\subsection{Proteine di membrana}
	Le proteine di membrana presentano domini ad $\alpha$ elica che attraversano la membrana.
	Possono inoltre non attraversarla ma possedere domini ancorati ad essa mediante legami specifici con il glicosifosfatidilinositolo o ancora \emph{GPI}.
	I domini transmembrana presentano amminoacidi idrofobici.
	Le acquaprine sono canali che permettono il passaggio di acqua.
	La batteriorodopsina permette la trasformazione di luce in \emph{ATP} attraverso un cromoforo.
	Possiedono tipicamente siti specifici che legano molecole di segnale extracellulare agendo come amplificatori per esso.

	\subsection{Movimenti delle sostanze}
	Le sostanze sono libere di muoversi nella membrana.

		\subsubsection{Movimento laterale}
		Nel movimento laterale le sostanze che compongono la membrana si muovono all'interno di un monostrato.
		Tendono infatti a diffondere in modo da eliminare il gradiente di concentrazione nel piano.

		\subsubsection{Flip flop}
		Il movimento di flip flop avviene raramente ed \`e mediato da flippasi.
		In questo movimento una molecola passa da un monostrato ad un altro.


\section{Passaggio delle sostanze}

	\subsection{Passaggio attivo}
	Il passaggio attivo \`e un tipo di trasporto selettivo che consuma \emph{ATP}, un esempio sono le pompe che lavorano contro gradiente di concentrazione.

	\subsection{Endocitosi}
	Nell'endocitosi una regione della membrana plasmatica si invagina, racchiudendo al suo interno un piccolo volume di liquido esterno.
	L'invaginazione si chiude su s\`e stessa formando una vescicola nella cellula.

		\subsubsection{Fagocitosi}
		La fagocitosi \`e un caso particolare dell'endocitosi in cui il materiale trasportato all'interno della cellula viene particolato.

	\subsection{Esoticosi}
	L'esocitosi \`e il processo inverso ell'endocitosi: le vescicole si muovono dal citoplasma verso la faccia interna dell membrana plasmatica a cui si fondono.
	Il materiale contenuto nella vescicola viene rilasciato all'esterno.

	\subsection{Passaggio passivo}
	Il passaggio passivo si verifica per diffusione secondo gradiente di concentrazione.
	\`E tipicamente pi\`u lento di quello attivo.
