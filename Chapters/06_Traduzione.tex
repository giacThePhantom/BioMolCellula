\chapter{Traduzione}
La traduzione \`e il processo attraverso cui gli mRNA vengono tradotti in proteine funzionali alla cellula.
Viene svolta dai ribosomi.

\section{Ribosomi}
I ribosomi sono complessi ribonucleoproteici formati da due subunit\`a principali: la subunit\`a maggiore e la subunit\`a minore.
Si trovano liberi nel citoplasma, attorno ai mitocondri o attaccati al reticolo endoplasmatico rugoso.

	\subsection{Struttura}
	Ogni subunit\`a del ribosoma \`e divisa in proteine ribosomiali \emph{LRP/LRS} large ribosomal protein e \emph{SRP}, small ribonucleoprotein.
	Queste sono associate a RNA in base al peso molecolare per formare interazioni proteina-proteina e proteina-RNA.
	La subunit\`a maggiore e minore banno incontro a modifiche conformazionali per permettere il movimento dei ribosomi lungo il mRNA e permettere al tRNA di accedere ad esso.

		\subsubsection{Subunit\`a minore}
		La subunit\`a minore forma la base che si posiziona sul mRNA.
		Contiene una cresta.
		
			\paragraph{Procarioti}
			Nei procarioti la subunit\`a minore \`e $30S$, riconosce le sequenze del mRNA e contiene RNA $16S$ e $21$ proteine.

			\paragraph{Eucarioti}
			Negli eucarioti la subunit\`a minore \`e $40S$ e contiene RNA $18S$ e $33$ proteine.

		\subsubsection{Subunit\`a maggiore}
		La subunit\`a maggiore \`e formata da una cresta, una protuberanza centrale e un peduncolo.
			
			\paragraph{Procarioti}
			Nei procarioti la subunit\`a maggiore \`e $50S$ e contiene RNA $5S$, $23S$ e $34$ proteine.

			\paragraph{Eucarioti}
			Negli eucarioti la subunit\`a maggiore \`e $60S$, contiene RNA $5S$, $28S$, $5.8S$ e $34$ proteine.

		\subsubsection{Subunit\`a assemblate}
		Il ribosoma, una volta assemblato \`e:
		\begin{multicols}{2}
			\begin{itemize}
				\item $80S$ negli eucarioti.
				\item $70S$ nei procarioti.
			\end{itemize}
		\end{multicols}
		L'unione delle due subunit\`a crea inoltre tre siti:
		\begin{multicols}{3}
			\begin{itemize}
				\item $E$: sito di uscita per il rilascio di tRNA scarico.
				\item $P$: sito peptidico di sintesi del legame peptidico.
				\item $A$: sito accettore per l'entrata del tRNA.
			\end{itemize}
		\end{multicols}


	\subsection{Assemblaggio}
	L'assemblaggio dei ribosomi avviene nel nucleolo, una zona granulare del nucleo che contiene oltre $500$ proteine.
	Da un unico precursore a RNA vengono prodotti gli rRNA $18S$, $5.8S$ e $28S$.
	Il processo di assemblaggio \`e altamente coordinato.
	\begin{multicols}{2}
		\begin{itemize}
			\item Sintesi e modifica pre-rRNA.
			\item Assemblaggio subunit\`a.
			\item Interazione transitoria con proteine e snRNA.
		\end{itemize}
	\end{multicols}

		\subsubsection{Processo}
		\begin{multicols}{2}
			\begin{itemize}
				\item Si forma il precursore $90S$: le due subunit\`a sono insieme con fattori ribosomiali e non ribosomiali.
					Questi ultimi sono proteine che aiutano la formazione del ribosoma come \emph{snRNP U3}.
				\item Il $90S$ si scinde formando le subunit\`a pre-$40S$ e pre-$60S$.
					La successiva rimozione dei fattori non ribosomiali porta alla formazione di subunit\`a mature.
				\item Le due subunit\`a lasciano il nucleo nel citoplasma dove si dissociano.
			\end{itemize}
		\end{multicols}

	\subsection{rRNA}
	Gli rRNA sono la componente a RNA presente in maggior quantit\`a nelle cellule.
	Nella subunit\`a maggiore dei procarioti si trovano i $5S$ e $23S$, mentre negli eucarioti se ne trovano $3$.
	Il $16S$ \`e presente invece nella minore dei ribosomi procarioti.
	Vengono trascritti dalla RNA polimerasi $I$ tranne il $5S$ che viene trascritto dalla RNA polimerasi $III$.

		\subsubsection{Funzioni}
		\begin{multicols}{2}
			\begin{itemize}
				\item Possono interagire con proteine.
				\item Riconoscono il sito di inizio per la sintesi proteica.
				\item Catalizzano la formazione del legame peptidico.
				\item Funzione strutturale.
			\end{itemize}
		\end{multicols}

		\subsubsection{Struttura}
		Gli rRNA hanno strutture conservate e permettono di identificare diverse specie.
		Le regioni con ruolo funzionale e biologico sono analoghe.
		Si trovano sia basi canoniche che modificate attraverso modifiche post-trascrizionli.

		\subsubsection{Sintesi}
		La sintesi degli rRNA avviene nel nucleolo, permettendo cos\`i l'assemblaggio con le proteine ribosomiali.
		I geni sono contenuti nella zona fibrillare del nucleolo.


	\subsection{tRNA}
	i tRNA sono le molecole di RNA che si occupano di trasportare un amminoacido permettendo cos\`i l'allungamento della catena peptidica in formazione.

		\subsubsection{Struttura}
		I tRNA presentano una struttura a trifoglio con una lunghezza tra i $70$ e i $95nt$.
		\begin{multicols}{2}
			\begin{itemize}
				\item Braccio accettore: trasporta l'amminoacido, ha un $AAC$ terminale dove questo si attacca.
				\item Braccio $D$ contiene la diidrouridina.
				\item Braccio dell'anticodone: permette il riconoscimento del codone.
				\item Braccio \emph{TYC} con pseudouridina, forma un ansa e possiede molte modifiche post-trascrizionali.
			\end{itemize}
		\end{multicols}

		\subsubsection{Notazione}
		
			\paragraph{tRNA scarico}
			\[tRNA^{aa}\]
			
			\paragraph{tRNA carico}
			\[aatRNA^{aa}\]

		\subsubsection{Caricamento}
		Il caricamento avviene ad opera dell'enzima amminoacil-tRNA sintetasi.
		Questo \`e specifico per ogni amminoacido e possiede tre siti per il legame con:
		\begin{multicols}{3}
			\begin{itemize}
				\item \emph{ATP}.
				\item Amminoacido.
				\item tRNA.
			\end{itemize}
		\end{multicols}
		La forma dei siti di legame stabilisce la specificit\`a dell'interazione.
		L'idrolisi di \emph{ATP} con liberazione di pirofosfato causa un trasferimento del gruppo adenilato dell'amminoacido.
		Si forma cos\`i l'amminoacido adenilato.
		Successivamente viene catalizzato il legame tra l'amminoacido adenilato e il gruppo $3'$ $AAC$ del tRNA.
		Si forma pertanto un amminoacil tRNA con liberazione di \emph{AMP}.

			\paragraph{Controllo del caricamento}
			La specificit\`a del processo \`e assicurata da fatto che l'enzima contiene due regioni selezionatrici.
			La selezione avviene per dimensioni, con tolleranze ridotte nella seconda.
			Gli amminoacidi sbagliati vengono rimossi.

\section{Fasi della traduzione}

	\subsection{Inizio}
	L'inizio della traduzione \`e la fase pi\`u lunga: avvengono i meccanismi di regolazione.
	Si riconosce inoltre \emph{AUG} da parte della subunit\`a inore grazie alle regioni di Shine Dalgarno o di Kozak.
	Dopo che il ribosoma ha scansionato il mRNA e trovato le sequenze inizia l'assemblamento per dare inizio al processo.
	La subunit\`a $16S$ riconosce la sequenza complementare, assembla lasubunit\`a maggiore e permette l'inizio.

		\subsubsection{Proacarioti}
		Nei procarioti la sintesi proteica avviene in concomitanza con la trascrizione.

			\paragraph{Fattori coinvolti}
			I fattori coinvolti sono detti initiator factors come:
			\begin{multicols}{3}
				\begin{itemize}
					\item \emph{IF1}.
					\item \emph{IF2}.
					\item \emph{IF3}.
				\end{itemize}
			\end{multicols}

			\paragraph{Processo}
			\begin{multicols}{2}
				\begin{enumerate}
					\item \emph{IF1} e \emph{IF3} si posizionano nella cavit\`a $P$ ed $A$ impedendo un legame precoce delle subunit\`a.
					\item Sul sito $E$ viene reclutato un complesso con \emph{IF2} e \emph{GTP}.
					\item \emph{IF2} recluta \emph{met$tRNA^{met}$} e mRNA.
					\item Si forma il complesso di inizio $30S$ formato da mRNA, subunit\`a minore, tRNA con metionina formilata, \emph{GTP} e \emph{IF2}.
					\item La rimozione dei fattori \emph{IF1} e \emph{IF3} permette il reclutamento della subunit\`a maggiore e la formazione del ribosoma.
					\item \emph{IF2} viene rimosso attraverso idrolisi di \emph{GTP} rilasciando il tRNA carico pronto per iniziare la traduzione.
					\item Questo \`e il complesso $70S$.
				\end{enumerate}
			\end{multicols}
			Si nota come la metionina caricata \`e formilata: viene aggiunto un gruppo aldeidico attraverso la metionil-tRNA trasformilasi.

		\subsubsection{Eucarioti}
		Si nota come negli eucarioti il mRNA possiede un cap che rende la fase di inizio pi\`u complessa.

			\paragraph{Fattori coinvolti}
			I fattori coinvolti o eucariotic initiator factor sono:
			\begin{multicols}{3}
				\begin{itemize}
					\item \emph{eIF4E}: riconoscimento del cap.
					\item \emph{eIF4A}.
					\item \emph{eIF4F}: composto da $4E$ e $4F$.
				\end{itemize}
			\end{multicols}

			\paragraph{Processo}
			\begin{multicols}{2}
				\begin{enumerate}
					\item Si forma il complesso $43S$ con \emph{met$tRNA^{met}$}, \emph{eIF2} e \emph{GTP}, subunit\`a minore, mRNA e \emph{eIF4G/E}.
					\item \emph{eIF4B} possiede un dominio elicasico che risolve strutture secondarie nel mRNA.
					\item Il complesso \emph{RNAeIF4F$+$eIF4B} si lega al $43S$ formando il complesso di inizio $48S$.
					\item Il complesso $48S$ scansiona il mRNA fino alla sequenza di Kozak.
					\item Inizia la traduzione con il riconoscimento di \emph{ATG}.
					\item Si liberano i fattori \emph{eIF3}, avviene idrolisi di \emph{GTP} con liberazione di \emph{eIF2}.
					\item Viene recitata la subunit\`a maggiore, si forma il complesso di inizio $80S$.
				\end{enumerate}
			\end{multicols}
			Il mRNA forma un loop e la coda poli-A recltuta \emph{PABP} (poli-A binding protein) che interagisce con l'estremit\`a $5'$ rendendola pi\`u stabile.

			\paragraph{Processamento del mRNA}
			Il mRNA processato forma punti di legame per i fattori di inizio.

				\subparagraph{Cap}
				Al cap si legano i fattori \emph{eIF4A-E-G}.
				Il legame diretto avviene con \emph{eIF4E} che sostituisce la cap binding protein $8020$.
				\emph{eIF4G} ha ruolo strutturale, mentre \emph{eIF4A} \`e un elicasi con un dominio \emph{DEAD box} con attivit\`a elicasica.
				\emph{eIF4B} attiva l'elicasi.





		\subsubsection{\emph{IRES}}
		La sequenza \emph{IRES} \`e una sequenza di inizio alternativo.
		Compete con le sequenze di Shine Dalgarno o Kozak per il legame del ribosoma.
		Pu\`o avere un efficienza minore rispetto alla traduzione cap-dipendente e viene utilizzata da RNA virali o per risposta a condizioni di stress cellulare.

	\subsection{Allungamento}
	L'allungamento \`e la fase successiva all'inizio.
	In questa fase viene prodotta la catena pepdtidica.
	
		\subsubsection{Condizione di partenza}
		Dopo l'inizio sul ribosoma si trova \emph{met$tRNA^{met}$} nel sito $P$, mentre il sito $A$ ed $E$ sono vuoti.

		\subsubsection{Arrivo di un tRNA}
		Quando arriva il tRNA successivo si forma il legame peptidico con il trasferimento del polipeptide nascente dal sito $P$ al sito $A$ dopo la formazione del legame mediato dal rRNA.
		La subunit\`a maggiore si sposta di tre nucleotidi a causa della formazione del legame.
		La subunit\`a minore invece si sposta grazie a fattori di traslocazione che idrolizzano \emph{GTP}.
		Il tRNA che si trovava in $P$ si trova ora in $E$ e una nuova tripletta libera si trova in $A$.

			\paragraph{Riconoscimento del tRNA}
			L'amminoacido corretto viene riconosciuto dai fattori \emph{EF1,2,3} (elongation factors) che si associano ai tRNA carico.

				\subparagraph{Appaiamento corretto}
				Un appaiamento corretto induce idrolisi di \emph{GTP} con il rilascio dei fattori.
				Il tRNA pu\`o pertanto rimanere all'interno del ribosoma.
				La proteina coinvolta \`e \emph{EF-Tu}.
					
				\subparagraph{Appaiamento scorretto}
				Se non avviene l'idrolisi si riconosce un legame scorretto e il fattore \emph{EF}, ancora attaccato al tRNA lo fa uscire dal ribosoma.

			\paragraph{Formazione del legame peptidico}
			La formazione del legame peptidico comporta l'attacco del \emph{$NH_2$} al \emph{$COOH$} nel sito $P$ con rilascio di \emph{$H_2O$}.
			Durante questa reazione si rompe il legame tra amminoacido e tRNA.

		\subsubsection{Dopo la formazione del legame peptidico}
		Dopo la formazione del legame peptidico che determina lo shift della subunit\`a maggiore il sito peptidico \`e vuoto.
		L'idrolisi di \emph{GTP} determina lo shift della subunit\`a minore.
		Il processo avviene grazie a \emph{EF-G}.

		\subsubsection{Proteina nascente}
		La proteine esce dal ribosoma con una struttura lineare: non assume strutture secondarie o terziarie in quanto potrebbe bloccare tutto il sistema per le sue dimensioni.
		Si trova un canale nella subunit\`a maggiore del ribosoma che ne permette l'uscita.

	\subsection{Terminazione}
	La terminazione avviene quanto il ribosoma trova sul mRNA una sequenza di stop:
	\begin{multicols}{3}
		\begin{itemize}
			\item $UAA$.
			\item $UAG$.
			\item $UGA$.
		\end{itemize}
	\end{multicols}
	Proteine dette fattori di rilascio che attraverso mimetismo molecolare mimano un tRNA catalizzando la rottura del legame peptidico e il rilascio del peptide neosintetizzato.

\section{Inibitori della sintesi proteica}
\begin{multicols}{2}
	\begin{itemize}
		\item \emph{EGTA}: agente chelante ioni calcio, tende a dissociare le due subunit\`a.
		\item Cicloesimide: antibiotico che stabilizza il legame tra ribosomi e RNA.
		\item Puromicina: antibiotico.
		\item Sostanze che bloccano il movimento del ribosoma durante la sintesi.
	\end{itemize}
\end{multicols}
