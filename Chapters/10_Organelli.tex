\chapter{Organelli}
I compartimenti intrascellulari e il trasporto delle proteiine all'interno della cellula. Si \`e partiti da meccanismi molecolari piccoli si vede come i lipidi e le membrane formano delle
strutture e dei compartimenti all'interno della cellula e si trovano dalle cellule eucariotiche, nei batteri tutto avviene all'interno del citoplasma, non esistono membrane, solo 
la membrana cellulare e la parete batterica. Nelle cellule eucariotiche esiste un sistema di membrane alla base della loro funzionalit\`a e successo evolutivo. Questa compartimentaizone
delimita degli ambienti con caratteristiche e funzioni specifiche. Il Golgi per esempio aggiunge zuccheri alle proteine della membrana, i mitocondri producono energia e i lisosomi 
degradano molti dei processi biochimici che avvengono sulle superfici delle membrane avvengono al loro interno. Questo comporta delle problematiche legate al traporto delle proteine 
necessarie per il funzionamento dell'organello. Le protiene prodotte nel citoplasma per raggiungere il compartimento ed entrare in esso devono compiere delle cose. Questo campo va a 
studiare i meccanismi di trasfermineo di proteine specifiche all'interno dei compartimenti. Nelle cellule eucariotiche ci sono una struttura compartimentazione omogenea. Si trova un 
nucleo che contiene il materiale genetico circondato dalla emmbrana nucleare con una membrana interna e una esterna. La membrna nucleare \`e ricca di pori detti pori nucleari 
caratterizzati e formati da proteine dette nucleopporine, Il nucleo \`e uno scolapasta pieno di buchi il cui numero dipende dal tipo cellulare e dallo stato fisiologico della cellula.
Le membrane nucleari sno in continuit\`a con il reticolo endoplasmatico che pu\`o essere liscio importante in quanto al suo interno si accumula lo ione calcio e avviene la sintesi dei
lipidi el'ER rugoso che prende il nome dalla morfologia da come appare in microscopia, presenta sulla superficie i ribosomi. Le proteine che entrano nell'ER possono essere inserite in
concomintanza con la loro traduzione. Ci sono proteine che vanno in compartimenti in cui la proteine deve eessere fatta completamente per essere poi indirizzata come lisosomi e 
perossisomi. In prossimit\`a dell'ER rugoso si trova l'apparato di Golgi che prende il nome da Camillo Golgi che prese il premio nobel inisieme a Cajal per la messa a punto di un sistema
di colorazione che ha permesso l'identificazione di molti compartimenti cellulari e di formulare a Cajal la teoria neuronale in cui ilsistema nervoso \`e caratterizzato da cellule singole
dei neuroni. L'apparato del Golgi attuta una serie di mofiiche post trascrizionali a carico di proteine a cio vengonon aggiunti zuccheri e vengono prodotte le glicoproteine e i 
glicolipidi e le glicoproteine vengono portate sulla membrna plasmatica o secrete. Altri compartimenti sono i mitocondri, tutte le strutture che sono delimitate da membrana sono 
dette organelli. I mitocondri la cui funzione \`e di produrre energia, ATP, gli endosomi sono delle strutture che si originano dalle invaginazioni della memrbana plasmatica e portano 
all'interno materiale presente dall'ambiente extracellulare. Altri organelli sono i lisosomi, organelli di membrana che contengono diversi enzimi in grado di degradare i polimeri 
biologici, l'apparato digestivo della cellula. Oltre i lisosomi si trovano i perossisomi, organelli racchiusi da una membrana, contengono enzimi coinvolti nelle reazioni metaboliche 
in cui viene coinvolta l'ossidazione. Non contegono a differenza dei mitocondri del DNA proprio e le proteine che vanno entrano nei lisosomi e dei perossisomi vengono prodotte nel 
citomplasma e inserite in maniera specifica. Un altro organelli sono gli autofasogomi o fasogomi. Si trova anche una diverso posizionamento all'interno della cellula in base al tipo 
cellulare. Si pu\`o pertanto distinguere una certa polarit\`a in termini di posizionamento di questi elementi. Alcune di queste strutture sono visibili ad un microscopio ottico, \`e
difficile vedere altre strutture e pertanto si utilizza la microscopia elettronica o degli anticorpi che riconoscono componenti specifici dei vari compartimenti. Come proteine 
esclusive dell'apparato del Golgi attraverso immunofluorescenza. Un anticorpo della proteina del mitocondrio si identifica il mitocondrio. \`E possibile andare a studiare nel dettaglio
morfologia struttura e caratteristiche dei compartimenti. I volumi relativi in una cellula epatica, la maggior parte del compartimento \`e costituito dal citosol, sostituito dai 
mitocondri, l'ER rugoso, ER liscio e cisterne del Golgi, nucleo ($6\%$), perossisomi, lisosomi ed endosomi. Queste percentuali variano in base al tipo cellulare. Si vede come le 
cellule del sistema immunitario T che media la risposta immunitaria cellulo mediata e il linfocita B produge anticorpi, proteine glicosilate che vengono secrete dal linfocita e vanno 
ad attaccare una determinata sostanza o un antigene che non appartiene all'organismo. Ci sono diverse immunoglobuline M durante il primo incontro con l'antigene dopo di che le G che 
riconoscono in maniera pi\`u precisa l'antigene, motivo del richiamo delle vaccinazioni. Sia le immonuglobuline M che G sono proteine glicosilate e presentano dei gruppo e degli 
zuccheri dopo modifiche post traduzionali che avvengono nell'apparato del Golgi e prodotte dal reticolo endoplasmatico. La percentuale e il volume relativo dei compartimenti all'interno
della cellula dipende dal tipo cellulare anche se la maggior parte del volume \`e occupata da mitocondri e reticolo endoplasmatico. Si trova un movimento di vescicole che si producono
dal GOlgi e che ritornano nell'ER e c'\`e pertanto un movimento di vescicole dall'ER al GOlgi e dalla membrana ma anche dal Golgi indietro. Si nota come \`e interessante capire come
la cellula regoli questi trasporti e il trasporto delle proteine all'interno di questi compartimenti. E come il trasporto sia specifico. Da un punto di vista evolutivo per capire come
le cellule hanno acquisito questi organelli. Indicazioni della prima cellula eucariotica \`e datata 1,4 miliardi di anni fa. I primi organismi pluricellulari di cui si ha evidenza 
fossile si hanno nel cambriano nel 225000 anni fa. Per un certo periodo di tempo c'\`e stato uno stallo che ha creato gli organismi pluricellulari. UN passaggio importante \`e stato 
il passaggo dalla cellula procariotica a quella eucariotica con la formaizone di una membrena nucleare per proteggere reazioni regolare che avvengono che coinvolgono il DNA. Un altto 
step importante \`e stato l'acquisizione di organismi che erano in grado di utilizzare ossigeno per produrre energia tramite un meccanismo di simbiosi questi organismi batteri in grado 
di usare l'ossigeno sono stati inglobati all'interno dando orgine a una cellula eucariotica in grado di produrre energia usando ossigeno. Questo ha protato a un passaggio di competenze
dal mitocondrio alla cellula: la maggior parte dei geni importanti per l'organismo sono stati presi in carico dal nucleo dalla cellula ospitante ocmportantdo che le proteine prodotte nel
citoplasma dovessero andare a finire nel micocondrio. Sono stati creati segnali particolari per svolgere la funzione. Le cellule vegetali hanno acquisito la capacit\`a di fissare CO2 
atmosferica attraverso cloroplasti che appartengono alla familgia di plastidi. UNa proteina \`e in grado di muoversi ed essere indirizzata in un compartimetno. Questo meccanismo pu\`o 
avvenire secondo un trasporto regolato come avviene per esempio nel trasporto id protine dal nucleo al citoplasma e viceversa. Le roteine si muovono in questi compartimenti passando 
attraverso i pori nucleari e si tratta di un trasporto che coinvolge delle importine e delle esportine che riconoscono particolari segnali presenti all'untero delle proteine che 
subiscono il traporto e meidano il passaggio attraverso i pori nucleari. La seconda modalit\`a \`e la translocazione proteica o trasporto di transmembrana che avviene nelle proteine che
vanno nei plastidi negl iorganelli deputati alla fotosintesi, nei mitocondri e nei perossisomi, un trasporto in cui le protene prodotte nel citosol arrivano in prossimit\`a del 
mitocondrio e riconosciuti da recettori che mediano utilizzando ATP o un gradiente per trasportare le protine all'interno delgi organelli. INfine si ha un trasporto vescicolare che
pertmette il trafficking di proteine da ER a Golgi con direzione in entrambi i sensi, lo stesso trasporto vescicolare porta le proteine modificate nel GOlgi sulla membrana e secrete, lo 
stesso avviene per i trasporti che vedono proteine andare in lisoosmi, endosomi. Queste sono le tre modalit\`a con cui la cellula indirizza proteine componenti nei vari organelli. 
Il trasporto specifico di una proteina vviene grazie a segnali scritti nella sequenza amminoacidica che fungono da zip codes, la sequenza di una sequenza con lis lis lis arg lis \`e un
segnale di importo nel nucleo (NLS) e vis ono segnali specifici per l'esporto, sengali per l'importo nei mitocondri che \`e localizzato nella porzione N terminale con lunghezza maggiore
all'NES e con amminoacidi specifici e sequenza comune predicibile e delle sequenze che permettono di indirizzare la proteine all'interno dei plastidi e sequenze di importo nei 
perossisomi, che si trovano nella coda C terminale. Sequenze di importo per prorteine all'interno dell'ER localizzata nella porzione N terminale peptide o leader segnale ed infine c'\`e
una sequenze che permette a proteine nel citoplasma di rietnrare nell'ER e localizzata nella coda C terminale. Queste s equenze sono importanti in quanto vengono riconosciute dai
fattori che mediano importo o esporto di prtoeine nei compartimenti. Se modificate possono indirizzare priteine in altri compartimenti. MOlti dei marcatori fluorescenti prodotte e 
sintetizate  dalla cellula sono state modificate in modo tale da entrare in compartimenti in modo da renderli visibili. Mutazioni a carico di questi amminoacidi interferiscno e 
compromettono il trafficking e l'indirizzamento di proteine verso i compartimenti. Il citoscheletro nel trafficking ha un ruolo fondamentale: perch\`e fornisce la base per il 
trasporto stesso. I microtubuli sono binari su cui avviene il trasporto di vescicole e proteine e forniscono supporto al trafficking e stabilizzano gli organelli stabilendo il loro
assemblamento e l'ER collassa in assenza. Sono importanti dineine e chinesine che trasportano le proteine o le vescicole all'interno della cellula. 
\section{Trafficking tra nucleo e citoplasma}
Si vede come avviene il trasporto di proteine dal nucleo al citoplasma e viceversa. Il nucleo \`e delimitato dall'involucro nucleare, una struttura formata da due membrane, una interna
e una esterna la quale \`e incontinuit\`a con il reticolo endomplasmatico. Queste due membrane sono differenti con proteine diverse e la membrana interna \`e in contatto con una rete
di filamenti proteici intermedi detta lamina nucleare, un intreccio costituito che riveste la superficie nucleare,  un reticolo fibroso che mette in contatto componenti del DNA che 
si associano ad esso e mette in contatto proteine presenti sulla membrna nucleare interna. Questi filamenti proteici intermedi sono lamine e i mammiferi possiedono tre geni A B e C che
codificano per sette protene. Queste lamine hanno uno stelo, un manico e una testa come le mazze da golf. Mettendo due mazze da golf si forma un dimero di lamina e pi\`u dimeri di lamina
interagiscono tra di loro formando una rete di lamine che sostiene la membrna interna nucleare. Loro mutaiozni danno patologie diverse come quella di emery dreufus o la progeria 
di hutchinsin gilford con Sammy basso caratterizzata da un precoce invecchiamento dovuto a queste mutazioni per la lamina A che compromette funzionalit\`a del nucleo. Questa membrana
\`e importante in quanto isola l'ambiente del DNA da quello esterno ed \`e sede di molti trasporti di acidi nucleici come mRNA, microRNA e lnRNA e di proteine che lasciano il nucleo
da sole o assieme agli acidi nucleici o che dal citoplasma entrano nel nucleo per svolgere le funzioni. Le proteine sono sitentizzate nel citoplasma e poi entrano nel nucleo. La membrana
deve pertanto permettere il pasaggio delle sostanze. La membrna presenta dei pori che fomranio il complesso dei pori nucleari NPC, ciascun complesso \`e composto da 30 proteine 
nucleoporine una struttura notevole dal punto di vista del peso che strutturale. Il numero di questi pori nucleari dipende pu\`o variare all'interno dello stesso tessuto come tra glia
o del cervelletto il numero varia dal tipo cellulare anche all'interno dello stesso tessuto. IL trasporto che avviene attraverso questi pori \`e di circa 1000 macromolecole al secondo. 
UN poro nucleare ha un aofmra asimmetrica: la porta rivolta verso il nucleo a forma a cono e quella citoplasmatica \`e caratterizzata da proteina filiforme. Ci sono proteine che 
delimitano il canale, altre che lo ancorano alla membrana e altre proteine strutturali. \`E una composizione notevole. Le proteine nella parte iterna fungono da filtro, creano un gel
e sono proteine ricche in amminoacidi di glicina e fenilanalina che rendono le proteine non strutturate e senza struttura ben determinata e formano una sorta di gel filtro che funge 
da barriera per l'entrata di proteine che lascia passare proteine a basso peso molecolare metnre protiene pi\`u grandi ne vengono escluse. Le proteine grandi come le proteine ribosomali
che vengono assemblate nel nucleo prima di uscire permettono la fuoriuscita grazie a dei trasportatori. Le componenti del poro nucleare possono svolgere funzioni al di fuori della
struttura come ruolo di mantenere rendendo stabile la zona dove si genera il potenziale d'azione nell'assone e colocalizza con marcatori dell'ER (NUp358). Per passare il filtro le
proteine devono avere la NLS nuclear localisation / import signal. Quando presente la proteina viene importata nel nucleo, mutando un singolo amminoacido la proteina rimane 
citoplasmatica. La presenza di un NLS \`e una condizione necessaria per un trasporto specifico attivo. L'NLS \`e indipendente dal peso molecolare. Si pu\`o marcare la proteina in esame
con nanoparticelle d'oro che appaiono opace in microscopia elettronica. La proteina interagisce con la componente fibrillare del nucleoporo, si avvicina e viene inserita nella rete 
al centro del poro nucleare e dopo di che passa alla parte citoplasmatica. L'NLS media il trasporto dal citoplasma grazie ad un recettore, una molecola che la riconosce che prende il 
nome di importine che mediano l'entrata nel nucleod i preotine che contengono l'NLS la mediazione pu\`o essere diretta o pu\`o essere indiretta tramite la presenza di un adattatore, la
proteina viene riconosciuta da un adatttatore e quando la porteine cargo interagisce on l'adattatore che cambia conformaizone esponendo l'NLS che viene riconosciuta dall'importina. 
L'adattatore introduce una problematica in quanto l'adattatore rende non necessario l'NLS sulla proteina cargo. Accanto al segnale che consente il traffcking da citoplasma a nucleo 
si trova una sequenza a una proteina presente nel nucleo di lasciarlo e di andare nel citoplasma. Si trovano entrambi in proteine che vanno avanti e indietro come le proteine adattatrici.
La sequenza NES nuclear export sequence questa molto spesso \`e meno chiara in quanto i software predicono con una precisione inferiore rispetto all'NLS. L'NES \`e ricca in leucina a
una certa distanza e separati da altri amminoacidi, il consensus \`e meno evidente rispetto ad un NLS. L'NES permette a proteine nel nucleo di essere riinviate nel citoplasma. L'importo
di proteine attraverso il nucleoporo avviene grazie alla presenza di importina e alla presenza dei molecolar switches (trasduzione del segnale) intrerrutori molecolari, le RAN GTPasi, 
interruttori molecolari che possono trovarsi in due conformazioni a seconda che leghino GDP o GTP sono importanti in quanto consentono e mediano il trasporto dal nucleo al citoplasma e
viceversa. Nel citosol si ha una percentuale di RAN legate al GDP, mentre nel nucleo RAN \`e legato a GTP. Il cambio conformaizonale \`e mediata da una Ran-GAP, una proteina che attiva 
la GTPasi, quando RAN lega il GTP passa in RAN-GDP quando idrolizza il GTP, cosa promossa da Ran-GAP che attiva la GTPasi (GTPasi activating protein) il passaggio inverso viene mediato da
Ran-GEF (Ran fattore di scambio della guanina GTP excange factor) questi due fattori promuovono il legame di GTP o GDP. I due fattori si trovano nel citoplasma per Ran-GAP e Ran-GEF 
nel nucleo con un gradiente di Ran-GTP nel nucleo e di Ran-GDP nel citoplasma. L'idrolisi da GTP a GDP e lo scambio inverso pu\`o avvenire anche senza attivatori, ma in loro presenza il
loro turnover \`e pi\`u rapido ed efficiente. La proteina cargo \`e legata da importina che una volta che lo lega interagisce con la componente citoplasmatica del poro e lo avvicina al 
poro, l'importina ora tende ad interagire con domini con le proteine ricche di glicina e fenilanalina che forma il filtro e le rimuove spostandole e permette creando un buco di passaggio
per l'entrata del complesso. ALl'intreno del nucleo l'importina si lega al Ran-GTP e legandosi a Ran GTP cambia conformaizone e si distacca dal cargo che viene rilasciato nel nucleo e 
il complesso RAn-GTP importina fuoriesce dal nucleo, ritorna nel citoplasma e Ran passa a Ran GDP in quanto trova Ran-Gap che determina un cambio conformazionale dell'importina che porta
al rilascio di Ran-GDP. Il meccanismo di trasporto \`e analogo ma che avviene nel senso opposto, l'esportina lega il Ran-GTP e il cargo con l'NES, il complesso esce e nel citoplasma si
trova il fattore che promuove l'attivit\`a GTPasica di RAN che diventa Ran-GDP che porta al cambio conformazionale dell'esportina che causa il rilascio della proteina cargo all'esterno 
del citoplasma. Lo stesso meccanismo viene utilizzato per importare ed esportare proteine. Ran-GAP interagisce con la parte fibrillare esterna del poro nucleare e di fatto la si 
trova tra le fibrille, quando il complesso esce trova Ran-GAP in loco che favorisce l'interazione con Ran-GTP e ne stimola l'attivit\`a GTPasica con la produzione di Ran-GDP e 
il fosfato. Una stessa proteina che potenzialmente contiene un NLS non sia presente nel nucleo: questo avviene con significato biologico importante in quanto il trafficking nucleo
citoplasma \`e altamente regolato. Si pu\`o mantenerlo nel citoplasma attraverso modifiche come fosforilazioni che avvengono a carico dell'NLS che presente nella sequenza e quando
si vede nella proteina si vede nel citoplasma. L'embrione della drosophila \`e un sincizio e il fattore di trascrizione \`e presente in alcuni nuclei in uci il fattore ha subito 
delle modifiche per cui entra solo in alcuni nuclei, pertanto le cellule hanno un pattern trascrizionale (mRNA) diversi tra di loro. Il meccanismo pu\`o essere regolato come nella 
proteina delle proteine nuclear factor activated T cells, fattore nucleare delle cellule T attivate e iun condizioni di non attivazione si trova nel citoplasma e contiene un NLS. NON
va nel nucleo in quando quando non \`e attivato \`e fosforilato che va a mascherare l'NLS che in qualche modo viene reso invisibile. Quando il fattore viene attivato a seguito della
presenza di un antigene in cui occorre trascrivere geni che aiutano la risposta immunitaria, quadno questo \`e presente crea una cascata intracellulare che porta un aumento dei livelli
di calcio che va ad attivare la calcineurina, una fosfatasi che attivata dal calcio rimuove i gruppi fosfato del fattore nucleare NF-AT, rimuovendoli rende accessibile il segnale di 
localizzazione nucleare e la proteina viene portata all'interno del nucleo dove trascrive geni. Questo \`e un esempio di come una modifica post traduzionale \`e in grado di mascherare
in segnale di localizzazione. UN altro esempio \`e quello della proteina SREBP coinvolta nella sintesi del colesterolo (sterol response element binding protein) che \`e coinvolta nella
sintesi del colesterolo e quando la cellula la necessita la proteina la attiva e va nel nucleo e trascrive una serie di enzimi importanti per la sintesi del colesterolo. In presenza di 
colesterolo la proteina non serve e il colesterolo interagisce con SCAP (SREBP cleavage ectivation protein). Il fattore SREBP \`e il fattore trascrizionale e SCAP \`e un fattore che 
taglia SREBP: in presenza di colesterolo SCAP lo lega e si attiva nel tagliare SREBP, il tutto rimane quindi bloccato a livello dell'ER in presenza di colesterolo. In sua assenza i 
due componenti lasciano l'ER vanno nell'apparato del Golgi e trovano delle proteasi che vanno a tagliare il fattore SREBP enzimi che lo modificano post traduzionalmente e modificandolo
liberano un frammento che va nel nucleo e attiva la trascrizione per fattori coinvolti nella sintesi del colesterolo. Il passaggio da ER a Golgi \`e deterimato dal cambio confroamzionale
ad opera di SCAP che non \`e pi\`u legato al colesterolo che attiva il processo di trasporto, il taglio del fattore trascrizionale, rilascio e che contenente NLS va nel nucleo e trascrive
per elementi coinvolti nella biosintesi del colesterolo. Durante la mitosi l'apparato nucleare e  la memmbrana e l'ER, apparato del Golgi vengono disassemblate che inizia a coinvolgere 
la struttrua a lamina e il tutto viene regolato da meccanismi di modifica post-traduzilnael come fosforilazione delle lamine e delle proteine del poro nucleare. Quando si ha la 
separazione dei cromosomi all'interno delle cellule figlie si riforma la membrana nucleare e la lamina con le due cellule figlie perfettamente funzionanti.
\subsection{Trasporto in mitocondri e plastidi}
I plaastidi sono organismi presenti nelle cellule vegetali e svolgono funzioni importanti nell'ambito della fotosintesi clorofilliana e metabolismo delle cellule vegetali. Il rasporto
nei mitocondri prevede il riconoscimento da parte di recettori di segnali peptidici di proteine che devono essere trasportati. Sia il mitocondrio che il cloroplasto sno delimitata da
una membrana esterna, ne presentano una interna che nel mitocondrio presenta diverse creste dove sono presenti l'apparato deputato alla produzione di ATP. All'interno si distingue una 
regione detta matrice, tra membrana esterna e interna si trova uno spazio intermembrana, nei cloroplasti si trova anche un sistema di membrane dette membrane tilacoidi che delimitano
lo spazio tilacoide dove si trova il pigmento coinvolto nel processo di fotosintesi clorofilliana. A differenza di quello che accade per prorteine che entrano nell'ER che possono entrare
in esso in contemporanea alla loro traduzione, per proteine introdotte nei mitocondri o in altri organelli queste devono essere completametne tradotte da ribosomi che possono essere 
liberi nel citoplasma o che si trovano sulla superficie esterna dei mitocondri. UNa proteina deve essre completamente tradotta in quando le sequenze che peremetton il targeting possono
trovarsi anche nella porzione C terminale. Il peptide segnale che interagisce con i recettori presenti sui recettori \`e caratterizzato da una sequenza di 18 amminoacidi alcuni dei quali
polari e altri non polari e questo peptide forma una struttura ad alpha elica cone la distribuzione asimmetrica, con da una parte amminoacidi polari e dall'altra non polari. Questo viene
riconosciuto dal complesso che costituisce il traslocatore per le proteine mitocondriali che prende il nome di TOP per il trasporto dal citoplasma attraverso la membrana esterne e 
pernde il nome di TIM responsabile del trasporto attraverso la membrna interna nella matrice del mitocondrio. Il primo riconosimento della sequenza sengale avviene ad opera del complesso 
TOP con una componente recettoriale e una che fomra un canale come botte nella membrana che permette l'entrata della proteina attraverso la membrana. La proteina che entra all'interno del
mitocondrio viene tradotta nel citoplasma e questa inizia ad assumere una struttura secondaria caratteristica e per potersi infilare in realt\`a esiste una proteina detta HSP70 (heat 
shock protein 70), un chaperon, una proteina che aiuta le proteine ad assumere una struttura secondaria peculiare e in questo caso impedisce alla proteina di assumere la struttura che
ne impedirebbe l'entrata all'intenro del canale, la mantiene il pi\`u lineare possibile permettendole di entrare attraverso la membrana mitocondriale esterna. La proteina possiede il
peptide segnale riconosciuto dal recettore del complesso TOM e una votla che interagisce la proteina viene forzata ad aentrare nel canale, passa la membrana esterna ed entra nello spazio
intermembrana ed entra in contatto con il comlpesso TIM con una componente a canale e passa attraverso di esso e trova altre porteine chaperon che le impediscono di assumere un astruutura
terziaria prima che abbia completato il passaggio e quando finisce assume la struttura specifica e si ha il taglio del peptide segnale. Il complesso TOM e TIM lavorano insieme anche se
possono lavorare in maniera indipendente: ci sono proteine residenti dello spazio intermembrana. A cavallo della emmbrana interna si trova un gradiente elettrochimico di ioni H+ creato 
da pompe che lo pompano dalla matrice allo spazio intermembrana, uno dei due fattori che fornisce energia per la traslocazione delle proteine deal citoplasma alla matrice cellulare, 
l'altro \`e l'idrolisi dell'ATP a carico del complesso TOM e la seconda fonte di energia \`e il gradiente elettrochimico creato dalla pompa protonica. Dissipando il gradiente si blocca il
trasporto. Da un lato si ha idrolisi dall'altro il gradiente elettrochimico che spinge il meccanismo. L'inserimento di una proteina che deve rimanere associata alla membrana interna 
questo tipo di proteine contengono la sequenza segnale e una sequenza di stop-trasnfer, quando la proteina grazie alla sequenza segnale passa a TOM, passa a TIM e quando viene inserita la
sequenza di stop-transfer di fatto si blocca l'ulteriore passaggio dalla membrana interna verso la matrice. La proteina continua ad entrare dal citoplasma e si blocca nello spazio 
intermembranan, viene rilasciata da TIM e rimane ancorata la membrana. Un altro meccanismo prevede il ruolo del complesso OXA che \`e associato con il complesso TIM e un meccanismo 
che prevede l'inserzione di proteine nella membrana interna del mitocondrio che proviene dal citoplasma, la proteina entra attraverso il complesso TIM e si trova un secondo peptide
segnale riconosiuto dal complesso OXA che ancora la proteina all'interno della membrana interna. Si trova una parte bloccata dal complesso OXA. La presenza di una peptidasi che taglia
il dominio di ancoraggio della emmbrana permette al mitocondrio di ottenere proteine sollubili presenti nello spaizo intermembrana. UN altro meccanismo coinvolge la Mia40, una disolfuro
isomerasi che catalizza formaizone e rotture di legami disolfuro che avvengono in presenza di determinati amminoacidi all'interno di na protreina che quando viene attraversata dal 
mitocondrio Mia40 forma un ponte disolfuro temporaneo con la proteina che sta arrivando e il ciclo di formaizone e rottura di ponti disolfuro permette di dare l'energia per introdurre
la proteina nello spazio intermembrana, successivamente si formano ponti disolfuro nella proteina stessa che formano la struttura secondaria. La funzione dei mitocondri \`e quella di 
produrre ATP usando l'ossigeno e loro altri ruoli come il processo apoptotico. 
\section{Perossisomi}
I perossisomi del trafficking avviene con modalit\`a simile aquella dei mitocondri e i perossisomi non hanno DNA e tutte le componenti importanti per l'attivit\`a dei perossisomi
vengono prodotti dalla cellula e vengono indirizzati in questi organelli. Si penssa siano vestigia di antichi organelli con ruolo nel metabolismo dell'ossigeno. L'ossigeno \`e importante
in quanto rende efficiente la prodizione di ATP ed \`e una molecola reattiva e alcuni dei suoi metaboliti sono molecole reattive come H2O2, molecola potenzialmente dannosa, si pensa 
che i perossisomi fossero organelli deputati a detossificare la cellula dall'ossigeno e da molecole reattive che si formano a seguito della presenza di ossigeno. Abbassano 
la concentraizone intracellulare dell'ossiegno per sbolgere reazioni utili. Sono delimitati da membrana e contengono enzimi con diverse funzioni: una delle quali \`e di rimuovere degli
atomi di idrogeno da substrati organici specifici come RH2 + O2 -> R + H2O2, la formazione di acqua ossigenata \`e una specie altamente reattiva e grazie alla catalasi questa molecola
viene poi inattivata ad acqua. Il meccanismo per cui le cellule inattivano sostanze dannose come acido formico, formaldeide e alcol andando a produrre nell'ultimo caso acetaldeide. 
Andando a vedere le cellule epatiche hanno un elevato numero di perossisomi. L'altra funzione \`e la demolizione di acidi grassi o beta ossidazione che vengono convertiti in sostanze a 
due atomi di carbonio come acetil CoA che viene coinvolta in altre reazioni biosintetiche. Una funizone importante \`e quella di catalizzare le prime reazioni che portano alla formazione
di plasmaalogeni, fosfolipidi presenti nella mielina, una guaina che riveste gli assoni delle cellule nervose e peremtte una traspoto efficiente dell'impulso nervoso. Come il rivestimento
isolante dei fili di rame che permette il flusso. La guaina mielinica introdotta dalle cellule di Schwann permette una trasmissione efficiente e a lunga distanza. Mutazioni di enzimi
coinvolti nella formazione dei plasmalogeni porta gravi conseguenze. I perossisomi hanno una struttura rotondeggiante e con la presenza di inclusioni dette inclusioni paracristalline 
costituite dall'enzima dell'urato ossidasi coinvotli nei meccansimi di ossidazione all'interno dei perossisomi. Presenti nelle cellule vegetali dove sono coinvolti in vari meccanismi come
il ciclo del gliossilato, la cnversione di acidi grassi in zuccheri importante per la germinaizone del seme, non presente nelle cellule animali. Il targeting di proteine ed enzimi versi
i perossisomi avviene grazie a segnali, una sequenza di serina, lisina, leucuina nella porzione N o C terminale nelle proteina peroxisomal localization sequence (PLS), proteine che la
presentano vengono riconosciute dai trasportatori perossine (Pex) ce ne sono diverse come uno \`e la Pex5 che riconosce la sequenza PLS C terminale. Normalemnte il trasporto e 
riconoscimento nei perossisomi avviene con diverse Pex1, e 6 che usano ATP e permettono l'importo di queste proteina. La sindrome di Zellweger che porta anomalie cerebrali, fegato e rete
con mutazioni di Pex con pazienti con perossisomi che non svolgono la loro funzione e cosa che comporta danni notevoli all'interno di questi tessuti. Un altra prtoeina che riconosce la 
sequenza PLS \`e Pex7 che la riconosce a livello N terminale. I perossisomi originano dal reticolo endoplasmatico che gemma e forma una struttura vescicolare grazie a Pex3 transmembranan
e Pex19 asosciata alla membrana tramite un legame con lipidi e Pex3 e Pex19 sono importanti per la formazione della vescicola a partrire dal reticolo endoplasmatico per gemmazione che
dopo funge da protoperossisoma arricchito di altre componenti e poi i perossisomi possono andare incontro a divisione e formare altri perossisomi. La funzione dei perosssisomi si \`e 
relegata ai meccanismi di ossidazione di lipidi o alla degradazione e al metabolismo di alcol e altre sostanze e meno alla produzione di energia in quanto assunta dai mitocondri. 
\section{Reticolo endoplasmatico}
Nel reticolo endoplasmatico oltre ad esserci meccanismi di trasporto mediante traslocazione proteica si trova un trasporto vescicolare. UN trasporto che prevede la formazione di vescicole
che possono andare in membrana nel citoplasma, verso il Golgi o tornare verso l'ER. Si tratta di un organello che occupa il maggior volume all'interno della cellula, si tratta di un
labirinto di tubuli e piatti appiattiti che si estende nel citosol. \`E in continuit\`a con la membrana nucleare e delimita il lume del reticolo endoplasmatico o spazio delle cisterne del
reticolo endoplasmatico, la sua funzione sono diverse e all'interno dell'ER si possono individuare zone diverse: la pi\`u evidente \`e la divisione in uno liscio e uno rugoso, dove
appare evidente in quello rugoso la presenza di ribosomi presenti sulla membrana. Ha diverse funzioni. Quello liscio \`e importante come riserva di calcio con una strtutura che avvolge 
l'intera fibra muscolare nei muscoli scheletrici e contiene caclcio e rilasciarlo permettendo la contrazione del muscolo. \`E coinvolto nella formazione di lipidi e molecole che vengono
indirizzate in diversi distretti. L'ER \`e dinamico, \`e una struttura dinamica che va incontro a un notevole riarrangamento legato ai microtubili e di proteine motrici che permettono 
il movimento di ciscterne e reticoli di questo organello che portano alla fusione di una rete meno ramificato o pi\`u complessi. Si tratta di una strtutura dinamica. Oltre a ER liscio
e rugoso si pu\`o identificare un ER di transizione che \`e quella parte dell'ER liscio in cui si formano vesciocle che trasportano componenti al Golgi. Il trasporto di proteine 
verso il reticolo pu\`o avvenire in modo cotraduzionale o un importo post traduzionale: le proteine possono essere importate in fase di traduzione o dopo che sono state complestamente
tradotte dai ribosomi, una differenza tra l'importo nei mitocondri in cui la proteina deve essere completamente tradotta. In questo caso esiste un importo cotraduzionale in cui una 
proteina inizia a essere tradota, si trova una pausa e inizia il trasporto nell'ER e la traduzione riprende. Il ribsoma e la catena nascente vengono portati sull'ER dove la traduzuone
continua, nell'altro caso la portiena viene indirizzata dopo la traduzione, traslocazione co e post traduzionale. I ribosomi sono gli stessi e non cambiano strutturalmente. La differenza
sta nella localizzazione. I ribosomi del primo da citoplasmatici diventano adesi all'ER gli altri rimangono liberi nel citoplasma. La differenza sta nella localizzazione dei ribosomi. 
Il reticolo endoplasmatico si sviluppa come cellule diverse hanno nel pancreas esocrino presenta un ER rugoso maggiore rispetto alle cellule che producono testosterone, composti a base
lipidica. L'estensione dell'ER rgoso e liscio dipende dall'attivit\`a della cellula, in molti casi \`e difficile distinguere tra i due in quanto ci sono zone nell'ER rugoso prive di 
ribosomi. \`E possibile isolare il reticolo endoplasmatico attraverso un'approccio biochimico in cui la cellula viene omogeneizzata e il processo comporta la rottura delle membrana 
dell'ER che tendono poi a riformare delle vescicole pi\`u piccole dette microsomi. L'ER omogenato d\`a origine a vescicole pi\`u piccole che possono essere lisce o rugose in quanto 
contengono i ribosomi. Quelle rugose sono parte dell'Er rugoso e quelle lisce potranno essere parte dell'ER liscio o oderivare da altri organelli che sono andati incontro a rottura e che 
si sono riformate come i mitocondri e perossisomi. I microsomi lisci contengon membrane da altri organelli. Per dividere i microsomi si usa una centrifugazione su gradiente di saccarosio
La popolazione pi\`u leggera si dispone nella zona a concentrazione minore, menre la popolazione pi\`u pesante tende ad andare verso il fondo con concentrazione maggiore, la popolazione
pi\`u leggera sono le membrane non associate a ribosomi. In questo modo si isola la frazione microsomale liscia da quella rugosa. Questo \`e stato utile per capire la funzione 
dell'ER rugoso rispetto al liscio. Alla base del meccanismo della traslocazione di entrambi i tipi c'\`e la presenza di un peptide segnale la cui sequenza pu\`o avere delle variazione
con certo consensu sriconosciuto da un sistema di riconoscimento, il meccanismo \`e stato studiato gruazie alla possibilit\`a dei microsomi rugosi, identificato da Gunter Blobel. Il
tutto partiva dall'osservazione che le proteine prodotte nell'ER e poi secrete avevano una lunghezza inferiore rispetto alla stessa proteina nel citosolo. Il passaggio da citosol a ER
comportavala riduzione del peso molecolare. Si ipotizza la presenza di un peptide segnale prodotto da un ribosoma, tagliato e rimosso e la rimozione importante per la traslocazione 
poteva spiegare perch\`e le proteine prodotte erano pi\`u corte rispetto a quella prodotta al di fuori. L'esperimento si \`e preso un RNA per una proteina prodotta nell'ER e lo si \`e
tradott usando da un lato i microsomi rugosi e dall'altro ribosomi liberi nel citoplasma e si nota che l'RNA tradotto nei ribosoomi liberi ha una lunghezza superiore rispetto alla 
proteina prodotta dai microsomi rugosi ed entra nell'ER dove viene tragliata. Il passaggio dalla protieina citoplasmatica a quella del reticolo endoplasmatico comporta il taglio che 
avviene a carico della sequenza segnale. Il meccaismo di trasporto il peptide segnale permette alla proteina di esser portata sull"Er, esservi inserita, dopo una peptidasi rimuove il 
peptide segnale. Facendo correre su gel le proteine quella sul citoplasma corre pi\`u lentamente. Il peptide segnale a sua volta viene riconosciuto da un complesso detto SRN (complesso
che riconosce il peptide segnale) e riconosciuto dal sito di legame nel polo del traslocatore, una struttura attraverso cui passa la proteina. La SRP signal recognition particle \`e 
quella che dirige le proteine che contengono la proteina segnale ad un recettore presente sulla membrana nell'ER la struttura contiene, un complesso ribonucleoproteico con una catena 
ad RNA che ha associato di verse proteine che legano l'RNA. Il complesso costituisce il Signal recognition particle. All'interno di questi complessi ci sono 7 domini con diverse funzioni:
una componente interagisce con i ribosomi, una componente che accoglie e riconosce la sequenza segnale presente sulla proteina in fase di sintesi o sintetizzata. Questa tasca riconosce 
diverse sequenze segnale perch\`e ha una struttura costituita da amminoacidi metionina senza catene laterarli con una sacca idrofobica in grado di riconoscere sequenze ssegnali con 
variazioni e poi una porzione interagisce ocn i ribomsomi. UN'altra regione della proteina che interagsice con un recettore presente sulla membrana dell'ER: ci sono tre domini importanti
per il riconoscimetno della sequenza segnale, uno che interagisce con il ribosoma e uno che interagisce con il recettore specifico sulla emmbrana dell'ER. Il ruolo di questi domini per 
quanti riguarda il riconoscimetno della sequneza segnale \`e obbio, quello dell'interazione del ribosoma \`e di bloccare temporaneamente la traduzione, quando SRN si lega alla sequenza
segnale, l'interazione del ribosoma blocca temportanemente blocca temportaneamente la sintesi protecia permettendo al complesso di essre reclutato sulla emmbrana dell'ER grazie 
all'interazione con il recettore. La fase di stallo se non avvenisse la proteina se avesse con attivit\`a degradativa o dovrebbe essere attivata potrebbe creare danni nel citoplasma. 
Bloccando la sintesi temporaneamente e permettendo l'avviciniamento del complesso all'ER e si ha il distacco di SRP, il ribosoma interagisce con un altro componente dell'apparato 
deputato all'importo, la proteina traslocatrice e pu\`o continuare la sintesi. La proteina entra nell'ER e la si trova al suo interno. La proteina traslocatrice \`e il complesso Sec61. 
Sulla membrnaa dell"ER si trovano dei poliribosomi in quanto la sintesi di un mRNA avviene a carico di pi\`u ribosomi e si possono trovare poliribosomi presenti sull'ER. Il complesso 
Sec61 consiste di tre subunit\`a conservate $\alpha$, $\beta$ e $\gamma$, questo complesso \`e un poro e ogni subunit\`a ha funizoni diversi. I domini che attravesano la membrana sono
costituiti da $\alpha$ elica come quello che funge da tappo quando il traslocatore \`e chiuso, dominio $\alpha$ che quando deve aprirsi cambia conformaizone e si apre creando un poro. 
Un altro dominio, l'hinge, il dominio che permette apertura e chiusura del poro stesso. Il plug permette l'entrata della catena in fase di sintesi e due domini che permettono al poro
di aprirsi e traslocare proteine sintetizzate che rimangono in membrana. L'hinge \`e la cerniera che permette i cambi conformazionali. Una volta che la proteina \`e passata nell'ER viene
legata da un aproteina BiP (binding protein), un chaperon che impedisce alla proteina di assumere una conformazione e la mantiene il pi\`u non struttutata possibile per facilitare il 
passaggio attraverso il traslocatore come nella membrana mitocondriale. Associata al complesso del traslocatore si trova una peptidasi che va a rimuovere il peptide segnale. Oltre al
traslocatore associato ad esso si trva la peptidasi che riconosce e taglia il peptide segnale. La sintesi di proteine transmembrana avviene in maniera analoga e contiene un peptide 
segnale che la porta a livlelo del traslocatore e proteine con dominio transmembranico contengono una sequenza di stop transfer: la proteina viene portata nell"ER e la presenza del 
peptide segnale consente l'inserimetno della proteina che contiene fino a quando viene sintetizzata la sequenza dello stop transfer che quando entra determina l'apertura del traslocatore
e la proteina viene rilasciata sulla membrana dell'ER e si ancora ad essa. La peptidasi staglia il peptide sengale. In questo modo vengono sintetizzate proteine transmembrana con la 
terminazione N nell'ER e la terminazione C nel reticolo citoplasmatico. Tramite un altro meccansimo \`e possibile avere la porzione N terminale verso il citosol, questo avviene 
grazie al fatto che la sequenza che inizia il trasferimento sia interna e crea porzioni diverse. Per le proteine che attraversano le membrane pi\`u volte: pi\`u domini transmembrana, 
questo avviene, queste proteine presentano una sequenza di inserimetno all'interno del traslocone seguito da una sequenza di stop del trasferimento e quando entra e il traslocone 
interagisce con la sequenza di stop e viene lasciata una sequenza che presenta due domini transmembrana, quando sono di pi\`u si ha un alternazna di inizio stop e crea proteine che
passano la membrana diverse volte. Le proteine che sono legate alla membrana tramite un dominio corto nella porzione C terminale come le proteine SNARE, che mediano la fusione delle
vescicole tra i vari compartimenti. Queste proteine vengono ancorate tramite un corto dominio idrofobico C terminale, il targeting all'ER vengono coinvolti attraverso altre proteine 
dette Cet con meccanismo simile a SNP, la proteina riconosce la sequenza la porta sull'ER usando Get tre dove viene riconosiuta da Get1-Get2, l'ER controlla la sintesi di diverse proteine
con diverse caratteristiche. UN altra funzione dell'ER \`e quella di modificare le proteine che vengono prodotte, in particolare \`e coinvolto nella modifica post traduzinooale che porta
alle glicoproteine che verranno processate nell'apparato del Golgi, molte proteine sono glicoproteine e presentano questa modifica. Nell'ER viene aggiunto un oligosaccaride precursore
costituito da 14 zuccheri che sono due moleocle di N acetilglucosammina, nove molecole di mannosio e tre molecole di glucosio. Molte delle proteine sintetizzate nell'ER rugoso sono
glicosilate mediante l'aggiunta del precursore comune costituito da 14 zuccheri. IL legame avviene attraverso l'amminoacido aspargina. Tre molecole di zucchero e una di mannosio vengono
rimossi in modo che l'ER possa capire che la proteina ha assunto una conformazione corretta ed \`e pronta per andare o in membrana o nell'apparato del Golgi. L'aggiunta di questi 
zuccheri avviene grazie al precursore di 14 zuccheri \`e ancorato alla membrana tramite il dolicolo una moleocla lipidica, la proteina \`e presente enl lime del reticolo endoplasmatico
e sulla membrana si trova l'enzima oligosaccaril tranferasi che trasferisce l'oligosaccaride alla proteina. L'oligosaccaril transferasi presenta il dominio catalitico nel lume dell'ER. 
Si libera pirofosfato, si forma il legame tra oligosaccaride e proteina. La produzione del precursore oligosaccaride avviene attraverso diversi step con diversi enzimi che portano a
un aggiunta progressiva di molecole di zucchero. I primi passi avvengono nel citosol con dolicolo e si aggiungono le prime componenti. A un certo punto quando sono stati aggiunti l'n 
acetilglucosammina e il mannosio avviene il flipping: la componente citosilica viene rivolta nel lume del reticolo endoplasmatico dove continua l'aggiunta degli zuccheri fino alla
formazione del precursore definitivo. In questo caso il precursore legato al dolicolo presente nella membrana interna dell'ER in quanto la modifica avviene all'interno. Le proteine
prodotte e modificate nell'ER prima di lasciarlo vanno incontro a un controllo qualit\`a che verifica che il folding sia corretto, altrimenti vanno incontro a degradazione. Per alcune
proteine l'$80\%$ dei tentativi fallisce. Alla base del riconoscimento questa modifca svolge un ruolo importante. Sono coinvolti dei chaperon come calnessina e calreticulina la cui
attivit\`a \`e legata al calcio. Per la traslocazione post traduzionale il poro \`e costituito da Sec61 insieme a Sec62, 63, 71 e 72 che sono associati a Sec61 e sono quelli che mediano
la traslocazione di proteine formate nel citoplasma. In questo modello la proteina Bip ha la funzione della HSP70 nei mitocondri, una chaperon che legandosi alla porzione di proteina
presente nell'ER permette il trasferimento della proteina all'interno del lume attaccandosi impendendo la formazione di strutture secondaire e idrolizzando ATP e attaccandosi e 
staccandosi crea la forza necessaria epr il passaggio della proteina nel traslocone. In questo caso pertanto il trasoprto prevede l'idrolisi di ATP. Il reticolo endoplasmatico ha il 
ruolo di glicosilare le proteine, un processo importante che svolge un ruolo importante nel riconoscimento di proteine che non hanno assunto una conformazione corretta. Questo 
meccanismo avviene a carico dell'aspargina e comporta l'aggiunta dell'oligosaccaride precursore di cui gli ultimi 4 sono rimossi. Il tutto \`e mediato dall'oligosaccaril trasferasi
con la prozione catalitica nel lumer dell'ER e media la glicosazione portanto il complesos oligosaccaridico ancorato alla membrana da dolicolo e media la formazione del legame. I 
precursori del complesso oligosaccaridico viene assemblato all'inizio della parte citosolica della membrana dove si trova il dolicolo attivato tramite fosforilazione tramite CTP, dopo
di che i vari monomeri di zuccheri sono aggiunti e sono attivati e legati a dei nucletoidi UDP e GDP fino a quando si forma un complesso intermedio con 7 molecole di glucosio attaccato 
al dolicolo, interviene un enzima che flippa il complesso dove enzimi all'interno dell"Er finiscono la formazione del complesso di 14 molecole di zucchero. IL meccanismo e aggiunta di 
zuccheri \`e importante per riconoscere se le proteine inserite nell'ER assumono una struttura tridimensionale corretta i chaperon, proteine che si legano a prtoeine fatte nell'ER e sono
state glicosilate e la calnessina e calreticulina e differiscono in cui la calnessina \`e ancorata in membrana mentre la calreticulina \`e presente nel lume, aiutano le proteina ad 
assumere la struttura funzionale corretta e impediscono a proteine senza la struttura corretta che tendono a interagire tra di loro formando complessi dannosi per la cellula. Aiutano
la proteina ad assumere la struttura corretta e impediscono che le proteine in fase di acquisizione della struttura interagiscono formando aggregati pericolosi. La calnessina riconosce
la proteina a cui \`e stato attaccato il complesso dei 14 zuccheri a cui ne sono stati rimossi 3, vengono rimossi tre molecole di glucosio lasciando il precursore con proteina e la
catena di zuccheri con un unica catena di glucosio. Riconosciuta dalla calnessina e il polipeptide che sta assumendo la struttura tridimensionale. La calnessina si lega e dopo di che
interviene la glucosidasi che rimuove l'ultima molecola di glucosio. A questo punto se la proteina ha la struttura corretta lascia l'ER e va in un altro compartimento, se invece non
assume la struttura corretta interviene la glucosil trasferasi che riattacca un'altra molecola di glucosio e la proteina viene riconoscuita dalla calnessina che la aiuta a riprendere 
la forma corretta. Non \`e chiaro come si faccia a riconoscere la struttura corretta. Il mannosio viene perso in un passaggio successivo. Nel momento in cui la proteina non assume la
struttura in maniera corretta \`e potenzialmente pericolosa in quanto possono essere insolubili e formare aggregati che alla fine comporta un grave danno per la cellula. Di conseguenza
questo meccanismo aiuta le proteine ad assumere la proteina corretta e quando questo non succede devno essere eliminate, esiste un meccanismo che prevede la rimozione di proteine
misfolded dal reticolo endoplasmatico verso l'esterno. Una votla che raggiungono il citosol vengono degradate grazie all'aggiunta di ubiquitina che porta le proteine nell'apparato del 
proteasoma per essere degradate. Alcune proteine dell'ER l'$80\%$ non riescono a raggiungere lo stato corretto. Il folding delle proteine dipende da molti fattori, in molti casi 
uno stato di stress porta a proteine ad assumere conformazioni errate come l'aumento della temperatura. Il trasferimento di una prorteina misfolded avviene grazie ad un traslocatore, 
un complesso la cui proteina viene riconosciuta e legata da una disolfur isomerasi che aiuta la formazione e rottura di ponti disolfuro, va incontro a questa modifica e vengono creati
dei punti disolfuro, una lectina riconosce la compoennte glicidica e si forma il complesso proteina chaperon isomerasi e lectina, il complesso lascia l"ER dove il processo di 
traslocazione \`e accoppiato da un'ubiquitina ligasi che attacca le ubiquitine alle proteine e viene subito modificata metre lascia l'ER. La proteina associata al complesso \`e una ATPasi
che idrolizza l'ATP e permette alla proteina di uscire e una N glicanasi rimuove il complesso glicidico, la proteina ubiquitinata viene indirizzata al proteasoma per essere degradato. 
Quando si trova un accumulo di proteine non corrette vegnon attivati altri meccanismi che permettono alla cellula di ridurre il pio\`u possibile la quantit\`a di proteine con struttura
scorretta. Da un lato producono una maggior quantit\`a di chaperon che aiutano il folding delle proteine nell'ER e quasto pu\`o avvenire in deverse vie: in una si attiva il complesso
IRE1 presente sulla membrana dell"ER con una componente citosolica attivata mediante fosforilazioni di chinasi specifiche e una porizone nel lume dell'ER che funge da sensore per il 
misfolding proteico, una volta che il complesso IRE1, un dimero viene attivato causa una cascata di attivaizone intracellulare che porta alla regolazione di un RNA detto XBP1 che va
incontro a splicing nel citoplasma dell'RNA che codifica per XBP1, un fattore trascrizionale e la modifica post trascrizionale di XBP1 che viene tradotto produce la proteina che 
va nel nucleo, funge da regolatore trascrizionale e porta alla trascrizione di geni che aumenta la capacit\`a della cellula di portare avanti il folding delle proteine. Vi sono 
altri due meccanismi di regolazione, uno attiva ATF6 che attivato viene importato nell'apparato del Golgi dove un enzima lo taglia e causa la formazione di rilascio di subunit\`a che
entra nel nucleo e agisce da fattore trascrizionale. UN altro meccanismo attiva Perc in cui si ha un blocco traduzionale di alcuni RNA e si va ad aumetnare la traduzione in maniera 
specifica di fattori regolatori della trascrizione e si ha un aumento di traduzione di questi fattori che aumentano la traduzione dei fattori. Il reticolo endoplasmatico produce i 
lipidi di membrana, una delle modifcihe che porta a froamzione di proteine legate al glicosilfosfatidilinositolo, inserita nel;'ER con componente C terminale che lo ancora alla membrana
all'interno dell'ER avviene la rottura a livello della membrana della proteina e un attacco della proteina al lipide di membrana fomrano la proetina legata alla membrana mediante
ancoraggio GPI, questo importante in quanto non \`e pi\`u attaccata alla emmbrana con dominio ma tramite il legame rotto e rimosso da enzimi come la fosfolipasi C presente nell'ambiente
extracellulare e una proteina che se tagliata diventa secreta. Queste molecole le si trova eni meccanismi con cui le cellule nervose riconoscono un particolare target. Questo tipo di 
modifica permette a una proteina di membrana di divetnare ancorata ad essa e rilasciata quando l'ancoraggio viene rotto. Questo avviene all'interno dell"ER. Queste proteine si 
riconoscono hanno una coda C terminale altamente idrofobica. IL motoneurone riconosce nell'ambiente circostante molecole che lo attirano verso una determinata zona e grazie a questa
guidance raggiunge la cellula con cui forma la connesione. Questi tipi di molecole che guidano la risposta dei neuroni sno presenti sulla membrana del neurone e altri sulle cellule che
si trovano sul percorso. Le proteine che trova utilizzano questo ancoraggio e viene utilizzato in quanto permette all'enzima di tagliare e il cartello di indirizzamento pu\`o diffondere
a una distanza maggiore ed essere attraente a distanza maggiore. l'ER \`e coinvolto nella produzione della maggior parte dei lipidi di membrana come la fosfatidilcolina e viene prodotto
in tre step: un primo l'acido grasso che \`e associato ad una proteina viene portato nella porzione citosolica della emmbrana dell'ER e viene ancorato e introdotto nel monolayer 
citosolico dove viene attivato grazie all'aggiunta di due coenzimi A e interviene un enzima che attacca il glicerolo trifosfato formando l'acido fosfatidico e infine l'attacco della 
colina formando la fosfatidil colina.  con il diacilglicerolo come intermediario. Si nota come il monolayer citosolico \`e asimmetrico rispetto alla parte verso il lume e il meccanimso
comporta aggiunta di molecole aggiuntive. In realt\`a si osserva che i due monolayer sono omogenei, questo avviene in quanto intervengono enzimi detti scramblasi che prendono i lipidi
su un monostrato e li spostano nell'altro monostrato in modo da avere una crescita simmetrica di entrambi i monolayer. Questi enzimi sono diversi rispetto a quelli visti da membrana
che determinano il flip flop in quanto non si altera il numero di molecole presenti nei due monostrati nell'ultimo caso. La membrana plasamtica contiene un altro di traslcoatore detti
flippasi che riconoscono in particolare i fosfolipidi con cariche fosatidiletanolammina e fosfatidilserina e le spostano verso la parte esterna e nella membrana plasmatica grazie alle
flippasi specifiche ha composizione omogenea di fosfolipidi della parte citoplasmatica rispetto a quella esterna, permettedno il ricnoscimetno di cellule in apoptosi inn quanto i 
fosfolipidi vengono esposti verso l'esterno, segnale da parte del sistema immunitario. L'ER produce colesterolo e ceramide l'ultima prodotta dall'amminoacido serina attaccato a un
acido grasso formando la sfingosina costituita da un acido grasso e dall'amminoacido serina, successivamente una seconda catena di acido grasso viene aggiunta alla sfingosina cosa che
avviene nell'apparato del Golgi e la formazione della Ceramide \`e importante in quanto componente di molte membrane come quelle del sistema nervoso importanti per la trasmissione 
corretta del sistema nervoso. 
\section{Traffico intracellulare di vescicole}
Ci sono due meccanismi principali con cui la cellula \`e in grado di introdurre sostanze con l'ambiente esterno e l'altro di rimuovere e trasportare sostanze dall'interno, rispettivamente
endocitosi ed esocitosi. L'endocitosi comporta un endoflessione della membrana plasmatica, la formazione di una vescicola che viene internalizzata, mediante questo meccanismo la
molecola \`e in grado di internalizzare molecole esterne che possono interagire con recettori presenti sulla membrana e l'endocitosi pu\`o essere attivata tramite interazione con 
le sostanze e recettori, come per la transferina legata allio ione ferro che induce la formazione di una vescicola endocitotica internalizzata e subisce delle modifiche. L'esocitosi
consiste nel trasferimento del materiale dall'interno della cellula verso l'esterno. Questi due meccanismi comportano la formazione di vescicole di trasporto che possono essere pi\`u 
o meno sferiche. Queste vescicole possono formarsi anche all'interno di organelli come nel reticolo endoplasmatico e tali vescicole stanno alla base del tasferimento dall'ER all'apparato 
del golgi. La formazione di vescicole comporta la presenza di un compartimento donatore in cui si origina la vescicola. Al suo interno si trovano le proteine cargo, proteine per il 
riconoscimetno del cargo che interagiscno con altre molecole importanti per la formaizone della vescicola. Quando raggiunge il target la vescicola si fonde causando il rilascio della 
sostanza il target pu\`o essere extracellulare o un organello intracellulare. Si tratta di un viavai intenso di vescicole e questo comporta la nascita di domande come si formano le 
vescicole e cosa le guida verso il giusto compartimento. Si trova un traffico bidirezionale tra ER e Golgi e tra il Golgi e la membrana plasmatica con la formazione di vescicole 
contenenti sostanze che andranno nello spazio extracellulare. Si ha poi la formazione di un trafficking tra il Golgi e formazione di vescicole bidirezionale tra Golgi e late endosome o 
endosomi tardivi e da questi verso i lisosomi, il compartimento in cui avviene la degradazione di sostanze interne o esterne della cellula. Attraverso l'endocitosi la cellula \`e in 
grado di introdurre sostanze dall'esterno attraverso la formazione di endosomi con la presenza di vescicole che formano gli endosomi tardivi che si fondono con i lisoosome. Le molecoel
coinbolti nel traffficking sono proteine ben caratterizzate che formano caratterizzano le vescicole di ciascun compartimento, sono ben caratterizzate e sono le clatrine, COPI e COPII, 
(coat protein), i tre tipi di proteine che caratterizzano tre tipi specifici di vescicole. Si trovano le clatrine caratterizzano le vescicole che si originano dall'apparato del Golgi nel
trans GOlgi network, l'apparato del Golgi consiste in una serie di cisterne e si possono distinguere tre aree: una zona che guarda verso la parte citosolica e verso la membrana (trans
golgi) e una parte che guarda verso l'ER (cis golgi) e una prozioe intermedia. Il cis golgi accoglie vescicole che si fomrano dall'ER che si fonddono con le membrane del cis golgi e 
migrano nei diversi compartimenti del Golgi fino a quando giungono nel trans golgi dove vengono modificate e smistate nei vari organelli o andare in membrana. L'appareato del Golgi 
contiene degli enzimi e marcatori diversi a seconda della regione. La clatrina caratterizza le vescicole che si originano dal trans golgi. COP1 si trova nelle vescicole che si orgiinano
a livello dell'apparato del Golgi intermedio e cis, le proteine COPII sono specifiche per le vescicole che si originano nel reticolo endoplasmatico e vanno nella parte del Golgi cis. 
Questo permette di utilizzare le vescicole come marcatori delle posizioni del GOlgi o per identificare l'ER. Si pu\`o fare un immunoistochimica individuando la prozioen dell'ER che forma
vescicole che poi vanno all'apparato del Golgi, un anticorpo che riconosce le clatrine per la parte trans dell'apparato del GOlgi. Possono essere usate come marcatori per identificare 
le posizioni degli organelli. Queste proteine e in particolare le vescicole rivestite da clatrina, le prime scoperte e studiate, sono vescicole che trapsotrano materiale dalla emmbrana
plasmatica e tra i componenti endosomiali e del Golgi. Sono carattterizzata dalla presenz adi clatrina, una proteina che forma una struttura detta trischelio formata dalle proteine 
clatrina questa struttura \`e caratterizzata da tre catene pesanti e da tre catene leggere importatni per le interazioni tra i trischeli che ne permettono l'unione, la prozione N 
terminale interagisce con le proteine adattatricic che mediano l'interazione tra il complesso e la membrana plasamatica. Pi\`u trischeli si uniscono tra di loro formando la struttura che
ingloba la membrana plasmsatica e il contenuto interno, possono essere strutture esameriche e pentameriche le cui dimensionio variano in base al tipo cellulare e dell'organismo. I 
vari trischeli si uniscono tra di loro formando il cesto che si trova all'esterno e all'interno si trova la membrana. Queste strutture globulari presente nella porzione N terminale 
della catena terminale interagisconocon le proteine adattatrici che sono proteine che fungono da potne tra le molecole di clatrina e i recettori che sono presenti sulla membrana che 
va incontro alla formaiozne della vescicola che riconoscono la proteina cargo, riconoscono i recettori con la molecola cargo e la clatrine. Queste protine normalmente sono AP2 costituite
da quattro subunit\`a $\alpha$, $\beta$, $\gamma$ e $\delta$ riconoscono non solo i recettori sulla membrana dell'organello ma anche componenti della membrana: i fosfoinositoli presenti
sulla membrana. Nel momento in cui riconoscono entrambi contemporaneamente queste due componenti si ha un cambio conformazionale che permette l'apertura con la proteina adattatrice e la
interazione con la  membrana da cui si former\`a la vescicola. Le due componenti quella recettoriale che si legano a proteine all'interno del lume che entrano nella vescicola di trasporto
e la funzione \`e legata alla componente dei lipidi di membrana in particolare l'inositolo ($10\%$ dei fosfolipidi della membrana) con ruolo di regolazione importante. Inositolo va 
incontro a modifiche fosforilazioni ad opera di chinasi: il fosfatidil inositolo 3 - fosfato. Il meccanismo di fosforilazione \`e legato all'attivaizone di chinasi che va a
fosforilare nella posizione 3 o 4 nello zucchero del fosfoinositide formando il fosfoinositolo 3, 4, bifosfato, si tratta di processi reversibili in quanto si ha la rimozione di gruppi
fosfato ad opera di fosfatasi che tolgono i gruppi fosfati. I lipidi cos\`i processati si trovano distribuiti in maniera diversa il fosfatidilinositolo 4, 5 bifosfato nelle cisterne del
cis GOlgi e la distribuzione dei lipidi \`e specifica e marca in maniera particolare le membrane che vengono riconsciute da proteine adattatrici specifiche. CI sono proteien che 
aiutano a deformare la membrana con domini Bar la cui funzione \`e quella di interagire con la membrana e facilitarne la deformazione permettendo l'interazione dei trischeli tra di loro
e la costituzione del cesto che avvolge la vescicola e la foramzione della veescicola matura per il trasporto. Sono importanti quando si modifica a formare vescicole di membrana in 
quanto a differenza di quelle intracellulari quella plasmatica \`e molto pi\`u rigida in quanto contiene una rete di filamenti di citoscheletro ed \`e pi\`u rigida ed \`e pi\`u difficile
deformare la membrana stessa. Si tratta di domini ad alfa elica con residui idrofobici che aiutano la membrana a formarsi. Quando si \`e iniziata a formare la vescicoal di trasprto
entra in gioco la dinamina che ha la funzione di rompere il legame tra la membrana permeettendo la formazione della vescicola matura, rompendo il collo della struttura a gemma. 
Questo viene effettuato grazie alla dinamina al collo delle vescicole che si stannoo formando. \`E una proteina con diversi domini ad  alpha elica e GTPasi e a seguito della sua idrolisi
cambia conformazione e restringe il colletto formato fino alla rottura e alla formazione della vescicola di trasporto. Una volta che si \`e formata la vescicola di trasporto questa 
viene rilasciata dalla membrana da cui si \`e originata e va a perdere il rivestimento di clatrina, importante in quanto la vescicola deve andare a fondersi con la membrana di un altro 
organello o quella plasmatica, ci sono delle fosfatasi che eliminano le mofiiche dei lipidi della membrana che determina il distacco della proteina adattatrice e il rivestimento esterno
si stacca e permette la fusione con tra la vescicola di trasporto e l'organello bersaglio. Vi sono altre proteine GTPasi monomeriche che soono importanti per la formazione delle
vescicole che coinvolgono COPI e COPII e possono essere coinvolti nella formazione di vescicle di clatrina dell'apparato del golgi. Possono essere GTPasi in una conformaizone inattiva
quando legano GTP e inattiva quando legano GDP, sono la proteina ARF, porteina per la formazione di vescicole che contengono COPI e clatrina e Sar1 per la formazione di vescicole COPII
sull'ER. Lattivit\`a di queste GTPasi monomeriche \`e controllata dai fattori GEF che le attivano in quanto permettono lo scambio GDP - GTP e GAP che favoriscono lo scambio GTP - GDP
disattivandole. Il meccanismo si svolge con la proteina Sar1 per le vescicoel da COPII che \`e inattiva in quanto nel citosol si lega a GDP, la presenz di Sar1 GEF che ne promuove lo 
scambio GDP in GTP permette a Sar1 attivata di legarsi alla membrana e quando si trova in uno stato inattivo Sar1 presenta un'elica con una certa affinit\`a per la membrana che \`e 
mascherata, quando passa a legare il GTP attivandosi con Sar1-GEF sulla membrana dell'ER Sar1 vienea ttivato esponendo il dominio e va a legarlo alla membrana dell"ER, importante in 
quanto la presneza di Sar1 recluta il fattore Sec23 e Sec24, l'ultimo \`e un aproteina adattatrice che lega il recettore che ha riconosciuto il cargo e recluta altre porteine Sec13 
Sec31 i quali formano il cesto costituito da COPII. Si tratta di un meccanismo complesso in quanto comporta la presenza di prorteine G monomeriche la cui attivit\`a \`e regolata da
GEF e GAP che determinano il reclutamento di molecole adattatrici che portano alla formazione di vescicole COPII, quando si vole disassemblare la capsula si attva la GAP che determina 
lo scambio Sar1 GTP in GDP che viene scisso e determina il cambio conformazoinale di Sar1 che ripassa in forma solubile e determina il disassemblamento della struttura. Non tutte le 
vescicole di trasporto sono sferiche , vescicole con il protocollagene che non sono sferiche ma hanno una forma diversa in quando trasportano proteine con un ingombro molto elevato. 
Il complesso COPII si parla di un complesso costiutito da Sar1, Sec23 e Sec24, 13 e 31. Uno dei fattori importanti in quanto contribuisce a rendere il traposrto specifico sono i 
fattori RAB vhe danno un contributo rendendo specifico il trasporto vescicolare verso un organello. \`E una famiglia di proetine GPTasi attivate nel momento in cui passano da uno 
stato GDP a GTP con localizzazione precisa, Rab1 ER e Golgi, Rab2 cis Golgi e Rab3 sinapsi secrete. Le proteine Rab  garantiscono la specificita, che una vescicola che deve andare a
un determinato compartimento lo raggiunga, Rab lavora grazie alla presenza di effettori Rab che possono essere proteine motrici che interagendo con Rab spostano la vescicola 
utilizzando i microtubuli o possono essere porteine di attracco e favoriscono il riconoscimento della vescicola con la membrana dell'organello con cui deve fondersi. La cascata di Rab
pu\`o attivare alrte Rab creando formando dei domini ricchi di Rab ed effettori di Rab e creare domini importanti per il riconoscimento specifico delle vescicole. L'attivazione di Rab
ad opera di GEF che passa da una situazione citosolica al seguito dell'attivazione da parte di GEF Rab5 va in membrana e l'arrivo in membrana determina l'attivazione di una chinasi
che forforila in fosfatidil inositolo 3 fosfato che recluta altri Rab gef che attiva altre Rab e le clusterizza in membrana, si forma una zattera sulla emmbrana con reclutamento di Rab5
e fattori ad esso associati, importante in quanto garantisce il riconoscimento specifico di una vescicola nei confronti di un determinato organello. Nel momento in cui la vescicola grazie
a Rab raggiunge il target deve avvenire la fusione tra le membrane e questo avviene grazie a due proteine dette SNARE v e t dove v sta per vesicle sulla vescicola e t si trova sul 
target. Le v Snare sno monomeri e le t snare sono trimeri. Le snare sono proteine che si inseriscono sulal membrana tramite un corto dominio idrofobico e mediano la fusione tra la 
vesicola di trasporto e il target, le due proteine si avvicinano entrano in contatto, si forma una struttura tetramerica stabile, si rimuovono le molecole d'acqua e si ha emifusione e 
fusione le due membrane. E il rilascio del contenuto della vescicola con l'altro compartimento. Quando si sono fuse tra di loro \`e importante che il complesso formato delle quattro snare
venga rimosso altrimenti potrebbe indurre la fusione delle vescicole, effettuato graize al fattore NSF con porteine accessorie che \`e un'ATPasi che svolge il complesso tetramerico 
riformando v snare e t snare grazie all'utilizzo di ATP. Molti fattori del trasporto sono stati identificati e se ne sottolinea l'importanza dove mancano. Alcune domande rimangono aperte.
\section{Apparato del Golgi}
L'apparato del Golgi pu\`o essere distinto in cis Golgi, porzione intermedia e trans Golgi. Le vescicole che si formano nell'ER normalmente prive di ribosomi. Le vescicole che contengono
proteine che devono essere smistate si formano nell'ER liscio e si fondono con la porzione cis dell'apparato del Golgi, le protene dall'ER vanno nel Golgi (alcune devono rimanere nell'ER
come enzimi e fattori coinvolti nella modifica di proteine). Il trafficking viene regolato, le proteine rimangono nell'ER in quanto proteine che risiedono nell'ER contengono delle 
sequenze segnale che sono localizzate nella porzione C terminale e consistono di due lisine seguite da altri due amminoacidi e questo avviene per le proteine ancorate alla memmbrana, 
proteine stazionarie nell'ER presentano le lisine C terminali. Le proteine solubili la sequenza segnale \`e costituita da quattro amminoacidi: lisina aspartato glutammato e leucina o 
sequenza KDEL. Le proteine secretorie dal GOlgi e solubili nell'ER pu\`o capitare che proteine residenti vadano nel Golgi e quando vi arrivano l'apparato presenta dei recettori che
riconoscono la sequenza segnale nelle proteine che devono rimanere nell'ER e si lega in quanto l'interazione recettore sequenza segnale quando si abbassa il pH che avviene nel Golgi, 
il recettore nel Golgi riconosce la proteina con sequenza KDEL interagisce, forma una vescicola che dal Golgi ritorna nell'ER, questo meccanismo permette il recupero di proteine andate
nel Golgi ma che devono essere presenti nell'ER per svolgere la loro funzione. La sequenza \`e sufficiente per far rimanere la proteina nell'ER si tratta di un meccanismo importante che
permette la permanenza nell'ER di proteine solubili che svolgono un'attivit\`a importante nell'organello. I due meccanismi non sono le uniche che garantiscono la permanenza di proteine
residenti nell'ER e ce ne sono altre con meccanismi non noti. L'apparato del GOlgi ha un dominio cis in cui si ha un movimento di vescicole da ER al Golgi e viceversa, si tratta di un
movimento bidirezionale, dal trans si formano vescicole vescicole di membrana o destinate ad altri compartimenti come endosomi o lisosomi. Queste cisterne al loro interno avvengono 
delle reazioni, lapparato del Golgi ha la funzione di processare proteine con modifiche post traduzionali con la formazione di glicoproteine e l'aggiunta di zuccheri. Le classi di 
oligosaccaridi prodotti nel Golgi sono gli oligosaccaridi complessi e quelli ad alto contenuto di mannosio. Nell'ER si aggiungono zuccheri del complesso di 14 zuccheri che comprende due
molecole di n acetilglucosammina, due di mannosio e il glucosio che viene rimosso con un mannosio prima dall'uscita. Questi zuccheri vengono modificati nel golgi e possono dare origine
ai due gruppi precedenti. Nel caso dell'oligosaccaride complesso si ha la rimozione di gruppi di mannosio e l'aggiunta di altri gruppi di acetilglucosammina, galattosio e acido sialico, 
il precursore del'ER viene modificato, si taglia le catene di mannosio e a quelle restanti si aggiungono altri zuccheri. Nel caso dell'oligosaccaride ad alto mannosio consiste 
nell'aggiunta di gruppi di mannosio. La glicosilazione \`e importante per modificare post traduzionalmente proteine. La formazione e questa modifica \`e complessa da aver fatto nascere
un ramo che va a studiare i meccanismi di regolazione di aggiunta di zuccheri e la loro funzione. L'acido sialico contribuisce a creare cariche negative e esposte in membrana creano un 
ambiente che svolge fenomeno per l'interazione delle cellule modifica delle propriet\`a biofisiche della membrana. Partendo dall'ER si forma l'oligosaccaride a 14 zuccheri e poi si 
formano le prtoeine ad alto contenuto di mannosio o oligosaccaridi complessi. UN altra modifica \`e la glicosilazione non sul gruppo N dell'aspargina ma a carico del gruppo OH nelle 
catene laterali di amminoacidi come treonina, serina, prolina, lisina e in questo caso si dice O - glicosilazione. Questi due tipi di glicosilazione danno origine a una modifica struttura
le diversa e si inserisce un altro grado di complessit\`a alla struttura. La maturazione e il mantenimento dei compartimento del GOlgi ha due modelli, uno della matruazione delle 
ceusterne dal ER si formano vescicole che clusterizzano e maturano dando origine alla zona cis dell'apparato del GOlgi e poi maturano acquisendo altri fattori diventando parte mediana e 
poi trans e si trova un continuo riciclo tra le componenti avanti e indietro permettendo la formazione di nuove cisterne. Il modello di vescicola non si trova una maturazione in toto
di una cisterna verso la successiva ma un trasporto di vescicole che provvede a mantenere la funzionalit\`a e specificit\`a dei compartimenti. NOn si sa quale modello sia corretto. 
Il trasporto dall'apparato del Golgi ai lisosomi. 
\section{Lisosomi}
I lisosomi sono organelli con enzimi idrolitici come proteasi, glicosilasi, lipasi, fosfolipasi, fosfatasi, tutto quello che \`e importante per degradare le componenti. Sono enzimi che
lavorano ad un pH basso (idrolasi acide), l'acidit\`a \`e mantenuta grazie alla presenza di una pompa che usa ATP e pompa ioni H+ dall'esterno all'interno creando un gradiente di ioni
H+ e uno di tipo elettrico a cavallo della membrana. Questo gradiente pu\`o essere sfruttato per l'importo ed esporto di altre proteine. I lisosomi sono strutture sferiche e dense il cui
numero dipende dal tipo gellulare e lo stato fisiologico della cellula, sono elementi molto eterogenei in quanto possono fondersi con gli endosoma formando l'endolisosoma. Si fonde in 
quanto il lisosoma conteiene le idrolasi acide e l'endosoma \`e quello che si forma sulla membrana plasmatica che permette alla cellula di catturare materiale all'esterno che viene
digerito e il lisosoma si fonde con l'endosoma in modo da digerirne il contenuto. Gli enzimi \`e importante che trovino il pH basso solo nei lisoosmi in quanto altrimenti potrebbero 
produrre danni in altri compartimenti. Vi sono altre vie utilizzate dalla cellula per degradare materiale che convergono nei lisosomi, questi sono endocitosi, fagocitosi che avviene
ad opera di cellule particolari in cui si ha un batterio o una cellula viene inglobata e si forma un complesso detto fagosoma che si fonde con il lisosoma. L'altro meccanismo di 
pinocitosi che si tratta dell'assorbimento di materiale non specifico come fluidi membrane o particelle attaccate a membrana come nel caso della macropinocitosi e l'autofagia \`e il 
quarto meccanismoche consiste nella degradazione dei materiali interni al citoplasma, la cellula usa componenti per ottener materiale, si formano compomnenti detti autofasogomi che
convergono nei lisosomi le vie permettono la degradazione del materiale per un suo utilizzo. Le proteine che vengono progessate nel reticolo endoplasmatico e nel Golgi per andare nei
lisosomi subiscono un'aggiunta di zuccheri e contengono il mannosio 6 fosfato, modifica importante in quanto forma vescicole il cui trasporto \`e specifico nei lisosomi. Proteine 
lisosomali contengono questa modifica che se non avviene le proteine vengono secrete. Le malattie lsiosomiali sono mutazioni di enzimi nei lisosomi cheporta la formazione di lisosomi
deficitari con accumulo di materiale per la cellula con conseguenze abbastanza gravi. L'autofagia si tratta di un meccanismo \`e il porcesso con cui la cellula usa sostanze in caso di 
starvation, quando si trova in una condizione di stress dell'apporto di sostanze, ci pu\`o essere un'autofagia non speficica con l'accolta di materiale citoplasmatico o specifica quando 
pu\`o essere attivata nel caso in cui i mitocondri sono danneggiati e viene attivata l'autofagia specifica con una pink1 che invece di essere importata rimane all'esterno che viene
riconosciuta da parkin che va a indurre l'autofagia che comporta la fomrazione di autofagosoma con mitocondrio che va incontro a degradazione, parkin sue mutazioni comportano
l'insorgenza del morbo di parkinson. Nelle cellule vegetali la degradazione avviene ad opera del vacuolo. Nelle modifiche post traduzionali con la formazione di proteine all'interno
dell'apparato del Golgi nel caso della funzione dell'ER la maggior parte delle proteine va incontro a modifica di un gruppo di 14 moleocle di zucchero modificato nell'ER la cui \`e 
impotrante nel folding, permette ai chaperons presenti nel reticolo endoplasmatico di vedere se la prtoeina \`e stata foldata in maniera corretta, se no viene degradata altrimenti 
passata nel Golgi dove viene modificata in prtoeine con catena di oligosaccaride complessa o in una proteina conpolisaccaridi ad alto conrenuto di mannosi. Il utto avviene ad opera di 
mannosidasi nell'ER e la manossidasi I rimuove tre gruppi di mannosio e un'altra da origine a un precursore con 3 molecole di mannosio a quelli si aggiungono mannosi o altri zuccheri 
come acetil glucossammina, acido sialico e galattosio. Si sa poco di tutte le funzioni connesse con l'aggiunta di questi zuccheri, alcuni importanti per il riconoscimento delle 
cellule per il substrato per modificare la carica dell'ambiente caratteristico (studiato dalla glicobiologia). Le proteine destinate ai lisosomi vanno incontro a una modifica post
traduzionale che consiste nell'aggiunta di un mannosio 6 fosfato, umportante in quanto ci sono recettori che la riconoscono specificatamente e le proteine che devono essere portate 
presso l'apparato lisosomiale, alcuni di questi enzimi vengono portati fuori dalla cellula, entrano in vescicole secretorie, trovano dei recettori, si legano, interagiscono con recettori 
e mediante meccanismi di endocitosi vengono reindirizzati verso i lisosomi. L'aggiunta del gruppo mannosio 6 fosfato solo per le proteine che devono andare nei lisosomi usando tecnihce
di ingegneria genetica esiste un segnale in queste proteine detto zona segnale o regione segnale la quale viene riconosciuta dall'acetilglucosammina fosfotransferasi che \`e quella
deputata a fare la modifica specifica per le proteine lisosomiali, la fosfotranferasi si lega alla proteine, ne provoca la modifica e in altri processi la proteine viene modificata che
permette il trasporto. Vi sono patologie in cui le proteine lisosomiali non vengono modificate, patologie che coinbolgono mutazioni a carico di questo enzima non pu\`i in grado di 
riconoscere la sequenza segnale o apportare la modifica, con conseguenza che le proteine invece di andare nei lisosomi vengono secrete con gravi danni ai tessuti con accumulo di proteine
dannose. 
\section{Endosomi}
L'endocitosi \`e il meccanismo che porta la cellula ad introdurre materiale presente nell'ambiente extracellulare al suo interno, meccanismo portato avanti grazie alla formazione di 
vescicole endocitiche che si formano sulla membrana plasmatica e ingoblano materiale che si trova all'esterno, ci pu\`o essere endocitosi non mediata da recettori o mediata dal legame
con un ligando. Il caso pi\`u estremo \`e la fagocitosi dove si ha l'endocitosi di cellule danneggiate o batteri. La pinocitosi in cui si ha introduzione di sostanze liquide portato 
avanti dalla formazione di piccole vescicole introdotte all'interno della cellula e la macropinocitosi. La formazione degli endosomi \`e complessa e prevede una serie di fasi e 
maturazione. La formazione da origine a una prima struttura detto endosoma precoce e il trafficking \`e bidirezionale, una volta che si \`e formato questo pu\`o formare una vescicola
che pu\`o ritornare in membrana. Una votla che si \`e formato l'endosoma pi\`u endosomi precoci possono fondersi tra di loro e modificare la composizione interna ed esterna e si 
formano gli endosomi trardivi, una parte di questi si fonde con i lisoosmi per la degradazione degli elementi endocitati. Pi\`u vescicole endocitotiche possono fondersi tra di loro
formando un endosoma precoce, e in qualche modo si muovano grazie alla presenza di microtubuli raccogliendo altre vescicole endocitotiche. Successivamente questo va incontro a 
maturazione, una parte dell'endosoma gemma e ritorna in membrana o forma gli endosomi di riciclo o pu\`o andare a maturare formando gli endosomi tardivi, da precoci e tardivi comporta
la formazione di vescicole all'interno dell'endosoma stesso che viene detto corpo multivescicolare. Il late endosome si fonde con il lisosomfa formando l'endolisosoma in cui le 
sostanze dell'endosoma vengono digerite. La formazione delle vecscicole endocitotiche avviene grazie alla presenza di proteine clatrine che tendono a formare delle zone accumulandosi
determinando le fosse rivestite da clatrina dove avviene l'introflessione della membrana insieme ad altri fattori che porta alla formazione della vescicola. CI sono altre strutture detti
caveole che sono delle fiasche a forma di fiasca, invaginazioni di membrana rivestite da caveoline strutturali, un modo che la cellula ha per produrre introflessioni per introdurre 
sostanze. Le caveoline a differenza delle clatrine tendono a rimanere ad associate con la membrana mentre le clatrine vengono rimosse dalla membrana. Le caveoline sono associate alla
membrana e formono dei domini all'intrno della membrana (lipid raft) che contengono zone ricche di colesterolo e sfingolipidi e proteine di membrana legate tramite il legame GPI
glicosilfosfatidilinositolo ancorare proteine extramembrana anche senza dominio transmembrana. Le caveoline si formano nelle vescicole pinocitotiche e un meccanismo che avviene e 
la cellula ha per introdurre sostanze dall'esterno \`e la macropinocitosi che non comporta il reclutamento di clatrine o caveoline ma un riarrangiamento del citoscheletro dell'actina, 
un componente importante del citoscheletro. Normalemente viene attivato dalla presenza di un ligando che interagisce con un recettore specifico, solitamente fattori di crescita che 
possono indurre questo tipo di cascata di attivazione, virus, fattori di crescita, ligandi per le integrine, questo meccanismo attiva il recettore che da una serie di cascate 
che convergono sull'actina riordinando il citoscheletro formando protrsioni di membrana (ruffle), evaginazioni che si richiudono su s\`e stesse formando vescicoeld dette macropinosoma che
si fonde con i lisosomi. L'endocitosi non sempre \`e specifica, pu\`o essere attivata anche senza legame specifico, questo avviene e l'endocitosi mediata da recettori, un esempio 
coinvolge l'endocitosi delle particelle lipoproteiche a bassa densit\`a, il colesterolo circola nel sangue complessato con queste particelle costituite da un monostrato di fosfolipidi e 
all'interno si accumulano le molecole del colesterolo attorno ad esso si trova una cimponente proteica riconosciuto da un recettore esposto quando queste lo necessitano. Porta in membrana
i recettori che riconoscono la componente dell'LDL e quando il recettore interagisce induce endocitosi e porta al suo interno recettore con attaccata LDL, il recettore viene riciclato
in membrana mentre il colesterolo viene rilasciato nella cellula, vi sono mutaizoni che portano una scarsa capacit\`a di internalizzare il colesterolo e il sangue presenta un elevato
numero di LDL che nell'ipercolesterolemia che forma delle placche che vanno a ostruire i vasi che da origine a infarto pu\`o insorgere con l'et\`a o avere una causa genetica legata a 
mutazioni a carico del ligando o del recettore e di conseguenza non si trova endocitosi. Una delle proteine coinvolte nel processo di endocitosi della lipoproteina a bassa densit\`a \`e 
la proteina AP2 che \`e quella che fa da ponte da clatrine e i recettori presenti sulle membrane. Coinvolta nel meccansimo di endocitosi e hanno i fosfolipidi di membrana come
fosfatidil inositolo 4, 5 difosfato presente sulla membrana. Devono riconoscere contemporaneamente particolari lipidi di membrana oltre al recettore legato. Le vescicole endocitiche
convergono nell'endosoma precoce. NOn tutte le molecole endocitata vanno incontro a degratazione e il recettore p\`o essere rimandato in superficie dove pu\`o svolgere la sua funzione.
Di conseguenza l'endosoma precoce \`e una stazione di smistamento iniziale per i recettori che possono essere riportati in membrana. Ruolo importante dall'endosoma precoce. Il
colesterolo con il ligando al recettore viene degradato, in alcuni casi viene portato insuperficioe con il suo ligando come accade del metabolismo del ferro. Il recettore per la 
tansferina e la tranferina vengono riportati insieme sulla membrana. In altri casi il recettore e il ligando vengono entrambi degradati. Il recettore da solo viene riportato da solo
in superficie in altri casi solo il ligando in latri entrambi venfono degradati. Il complesso deve essere marcato per la degradazione. Dall'endosoma precoce posi si passa all'endosoma 
maturo il quale si fonde con i lisosomi formando l'endolisosoma e degradando il contenuto. Un recettore che si lega ad un ligando, entrambi vengono incorporati nell'early endosome il
recettore va incontro a degradazione e questo avviene in quanto quando il ligando si \`e legato al recettore questo viene ubiquitinato, che non comporta la degradazione del complesso 
con il proteasoma ma si tratta di una marcatura, un tag che indirizza il recettore ligando all'interno dell'endosoma e lo porta alla degradazione. Questo meccanismo \`e mediato cal
complesso ESCRT (complessi proteici citosolici) che riconoscono il recettore ubiquitinato (mono), riconoscono un fosfolipide di memmbrana il PI(3)P, il primo legame di uno dei quattro 
compoennti determina il reclutamento di altri tre fattori e comporta la formazinoe di una vescicola all'interno dell'endosoma, vescicole intraluminali l'ubiquitina viene rimossa prima che
si formi la vescicola, i recettori con il ligando si trova nell'endosoma e si fonde con il lisosoma e si ha la degradazione del ligando con il recettore. I recettori che subiscono questo
destino sono quelli delle EGF (epidermal grow factor), tutto questo meccanismo avviene dopo che il legame del ligando con il recettore ha dato origine alla cascata di segnali 
intracellulari. Per la fagocitosi si formano pseudopodi estroflessioni del citoscheletro che poi si fondono e incorporano il batterio. La formazione delle vescicole dall'apparato del
Golgi che vanno incontro a secrezione pu\`o essere di due tipi, una costitutiva in cui le vescicole si formano e quando si formano vengono portate in membrana ddove si fondono o pu\`o
essere regolata, la regoalzione dipende da segnali esterni che vanno ad attivare una cascata che porta la fusione delle vescicole con al membrana e il rilascio del loro contenuto. Le
vescicole di secrezione il cui destino \`e regolato sono caratterizzato da un'elevata concetrazione del materiale al loro interno, durante la formazione \`e accompagnata da un accumulo
notevole come le vescicole di acetilcolina con un elevata quantit\`a di neurotrasmettitore, vengono prodotte, maturano e rimangono in prossimit\`a della membrana e rimangono in attesa
nel caso dell'acetilcolina di un potenziale d'azione che comporta il rilascio di calcio, il segnale che inizia le fusione. Come esempio di secrezione regolata le vescicole di 
neurotrasmettitore. Le vescicole contengono sostanze che sono dei precursori come enzimi, sostanze che vengono attivate nel mmomento in cui trovano ambiente favorevole in pH o 
vengono processate all'interno di vescicole esvologno funzione al momento della funzione, si possono trovare precursori che possono essere modfiicati all'itenro o all'esterno. Il 
meccanismo che permette alle vescicole contenenti neurotrasmettitore di fondersi alla membrana all'arrivo del segnale, l'ingresso di calcio alla termiazione nervosa. La vescicola e la
fusione tra vescicole \`e mediata dalle SNARE, la v SNARE e la t SNARE in questo caso sulla terminazione nervosa. Nel caso delle cellule nervose v SNARE \`e sinaptobrevina e 
t SNARE \`e sintaxina. Le SNARE lavorano quelle che si trovano sul target come trimero e dall'altra parte monomero (sinaptobrevina), il trimero ha sintaxina e snap25, un'unica proteina
con due dominiche costituiscono e formano il trimero della t SNARE e da un dominio poco strutturato e molto flessibile con ruolo importante. La fusione della vescicola con la membrana,
esocitosi regolata richiede un segnale, che \`e il calcio, quando la cellula nervosa manda il segnale e arriva allat ermiazione nervosa va ad aprire dei canali che fanno etnrare calcio
che quando entra causa la fusione della vescicola con la membrana e il rilascio del neurotrasmettitore. La fusione \`e mediata dal calcio in quanto un ruolo \`e svolto dallo SNARE e 
l'altro dalla sinaptotagmina che \`e una calcium binding protein con quattro siti di legame presente nella vescicola. Un altro componente \`e la complexina che interasgisce sia con v 
SNARE che con t SNARE e forma un complesso e impedisce la formazione di un tetramero utile per la fusione della vescicola, la presenz adella complessina impedisce che v SNARE e t SNAREE
vadano a promuovere la fusione della membrana. Il complesso che si forma \`e uno che non permette la fusione se non in presenza di calcio, se il calcio non \`e presente la vescicola, 
t e v SNARE che interagiscno ma non svolgono la funzione in quanto \`e presente la complessina che blocca l'attivit\`a. In presenza di calcio questo si lega alla sinaptotagmina che cambia
conformazine e rimuove il blococ legato alla presenz adella complessina e le t e v SNARE sono libere e promuovono la fusione della vescicola con la membrana sinaptica che determina il
rilascio del neurotrasmettitore e determina l'eccitamento della cellula che riceve il messaggio. \`E un mecacnismo mediato dalla v e t SNARE e in qualhe modo viene bloccato dalla 
complessina che viene rimosso dalla presenz della sinaptotagmina che lega calcio e rimuove complexina e permettendo la fusione delle due vescicole. I meccanismi di endocitosi ed esocitosi
sono strettamente correlati e si riequilibrano tra di loro mantenendo costante la grandezza della membrana, si tratta di meccanismi che \`e importatnte siano regoalti. \`E importante
che l'esocitosi sia promossa per portare materiali sulla membrana plasmatica durante divisione cellulare, fagocitosi, riparazione della membrana e cellularizzazione di un sincizio. Un
meccanismo poco noto legato alla modalit\`a con cui la cellula e l'apparato del Golgi mantiene la polarit\`a cellulare. Considerando un epitelio costituito da cellule che delimitano 
uno spazio esterno e uno interno, con porzione apicale e una basolaterale. Le cellule sono polarizzate con compoente apicale che dal punto di vista morfologico (tanti villi che aumentano
la possibilit\`a di assorbire sostanze) diversa dalla membrana basolaterale e con propriet\`a enzimatiche diversa rispetto a quella basolaterale. Si trovano delle specializzazioni di 
membrana nono presenti nell'altra zona, differiscono anche per la composizione dei lipidi. Questa polarit\`a viene mantenuta in quanto occorre che ci sia un trasporto specifico e una
via specifica che permetta la polarizzazione. CI sono due modelli, con entreambi veri, il primo detto sorting diretto dal trasn Golgi che nel trans golgi si distinguono dei domini che
producono vescicole trasportate in maniera specifica verso una delle due polarit\`a. L'altro modello di tipo indiretto: il trans golgi prodice vescicole proteine indirizzate in entrambe
le parti che si fondono con la bmembrana laterale e dopo di che si ha un processo di riciclo in cui proteine destinate alla parte apicale sono endocitate e si formano endosomi precoci che
portano le proteine specifiche alla parte apicale. Un altra componente importante per il mantenimento della polarit\`a sono le tight junction, barriere intorno alla parte apicale dalla
cellula e saldato dalla cellula successiva che impedisce la migrazione dei lipidi del monolayer esterno vesrso la zona basolaterale e alle proteine di diffondersi tra le polarit\`a. Sono
importanti in quanto impediscono alle sostanze di passare all'interno e passare i trasportatori specifici e forniscono una resistenza a stress e trazione. 

