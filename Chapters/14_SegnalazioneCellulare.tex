\chapter{Segnalazione cellulare}
Ogni cellula monitora l'ambiente extra e intracellulare, processa le informazioni che raccoglie e risponde appropriatamente. Al cuore di questo sistema ci sono proteine regolatorie che
producono segnali chimici inviati da un posto all'altro nei corpi o nella cellula, processati lungo la via e integrati con altri. 
\section{Principi di segnalazione cellulare}
Molti organismi unicellulari primordiali avevano sviluppato meccanismi per rispondere alla presenza di altre cellule. Molti batteri rispondono a segnali chimici secreti dai loro vicini
e si accumulano ad alte densit\`a di popolazione e attraverso questo quorum sensing permette ai batteri di coordinare i propri comportamenti. Durante l'evoluzione degli organismi 
multicellulari la comunicazione intercellulare \`e diventata estremamente complessa: essi sono societ\`a di cellule i cui il benessere del singolo \`e messo da parte per il beneficio
dell'organismo. La comunicazione tra cellule in organismi multicellulari \`e mediata da molecole di segnale extracellulare, alcune delle quali operano a grandi distanze, altre solo tra
vicini immediati. La maggior parte delle cellule sono sia mittenti che destinatari. La recezione dipende da proteine recettrici che si legano alla molecola di segnale. Il legame
attiva il recettore che attiva sistemi o cammini di segnalazione intracellulare che dipendono da proteine di segnalazione intracellulare che processano il segnale all'interno
della cellula e lo distribuiscono agli obiettivi. Gli obiettivi alla fine di questi sistemi sono proteine effettrici che sono alterate dal segnale e implementano il cambio appropriato
nel comportamento cellulare. Le caratteristiche fondamentali della segnalazione sono conservate nell'evoluzione degli eucarioti nonostante per alcuni organismi siano diventate molto
pi\`u elaborate. Il genoma umano contiene pi\`u di $1500$ geni per codificare proteine recettrici. 
\subsection{Segnali extracellulari possono agire lungo distanze corte o lunghe}
Molte molecole di segnalazione extracellulare rimangono legate alla superficie della cellula segnalatrice e influenzano solo la cellula a contatto, importante durante lo sviluppo e 
nella risposta immunitaria. Nella maggior parte dei casi le cellule di segnale secernono mediatori locali che agiscono solo sull'ambiente locale nella segnalazione paracrina in cui
tipicamente le cellule mittente e destinatarie sono di tipo diverso, ma pu\`o accadere che le cellule rispondano a un segnale che loro stesse hanno mandato (segnalazione autocrina).
Negli organismi multicellulari grandi si rendono necessari meccanismi di segnalazione a lungo raggio per coordinare il comportamento delle cellule in parti remote del corpo e si sono
evoluti tipi di cellule specializzate per comunicazione intercellulare lungo grandi distanze come i neuroni che estendono lunghi assoni ramificati che permettono loro di contattare
cellule obiettivo distanti attraverso sinapsi chimica. Quando un neurone \`e attivato da uno stimolo invia un potenziale d'azione lungo l'assone che quando raggiunge la sinapsi 
causa la secrezione di un segnale chimico detto neurotrasmettitore. Un altra strategia sono le cellule endocrine che secernono gli ormoni nel flusso sanguigno che le trasporta in
lontananza. 
\subsection{Molecole di segnale extracellulare si legano a recettori specifici}
Le cellule negli animali multicellulari comunicano attraverso centinaia di tipi di molecole di segnale come proteine, piccoli peptidi, amminoacidi, nucleotidi, steroidi, retinoidi, 
derivati degli acidi grassi e gas dissolti. Queste molecole di segnale sono rilasciate nello spazio extracellulare attraverso esocitosi dalla cellula segnalatrice, mentre alcune sono
emesse per diffusione attraverso la membrana cellulare, mentre altre rimangono attaccate alla superficie esterna della cellula. Le proteine di segnale transmembrana possono operare in
questo modo o possono possedere domini che possono rilasciare attraverso rottura proteolitica e agire a distanza. La cellula obiettivo risponde in ogni caso attraverso un recettore che 
lega la molecola di segnale e inizia una risposta. Il sito di legame ha una struttura complessa che gli dona alta specificit\`a e alta affinit\`a. Nella maggior parte dei casi i 
recettori sono proteine transmembrana sulla superficie. Quando legano il segnale si attivano e generano segnali intracellulari che alterno il comportamento della cellula. In 
altri casi si trovano all'interno e la molecola di segnale deve entrare nella cellula, cosa che richiede che il segnale sia piccolo e abbastanza idrofobico da diffondersi attravero 
la membrana plasmatica. 
\subsection{Ogni cellula \`e programmata per rispondere a specifiche combinazioni di segnali extracellulari}
Una cellula \`e esposta a centinaia di segnali diversi e risponde selettivamente esprimendo solo i recettori che rispondono ai segnali richiesti per la regolazione di s\`e. La maggior
parte delle cellule rispondono a molti segnali diversi e alcuni segnali influenzano la risposta ad altri segnali. In principio le molecole di segnale possono essere usate in
combinazioni illimitate per controllare i diversi comportamenti della cellula con alta specificit\`a. Una molecola di segnale ha diversi effetti in base al tipo della cellula
obiettivo. 
\subsection{Ci sono tre classi principali di proteine recettrici sulla superficie cellulare}
La maggior parte delle molecole di segnale extracellulari si legano a proteine recettrici specifiche sulla superficie e non entrano nel citosol. Questi recettori agiscono come
trasduttori di segnale convertendo il legame in un segnale intracellulare che altera il comportamento della cellula. Tali proteine appartengono a tre categorie definite dal loro
meccanismo di trasduzione. I recettori accoppiati a canali ionici o canali ionici transmitter-gated o recettori ionotropici sono coinvolti nelle rapide segnalazioni sinaptiche, mediato
da neurotrasmettitori che aprono o chiudono transientemente un canale ionico formato da una proteina con cui si legano, cambiando la permeabilit\`a della membrana plasmatica e
l'eccitabilit\`a della molecola postsinaptica. La maggior parte di questi appartengono a una famiglia di proteine transmembrana a multipassaggio omologhe. I recettori accoppiati a G 
agiscono regolando indirettamente l'attivit\`a di una proteina obiettivo legata alla membrana, un enzima o un canale ionico. Una proteina legata a GTP trimerica (proteina G) media
l'interazione tra il recettore attivato e la proteina obiettivo. L'attivazione della proteina obiettivo pu\`o cambiare la concentrazione di pi\`u molecole di segnalazione intracellulare
o la permeabilit\`a ionica della membrana plasmatica, le prime agiscono in turni per cambiare il comportamento di altre proteine segnalatrici nella cellula. I recettori accoppiati da
enzimi funzionano come enzimi o si associano direttamente con gli enzimi che attivano. Sono proteine transmembrana a singolo passaggio con il sito di legame all'esterno e il sito 
catalitico all'interno. Sono eterogenei in struttura, ma la maggior parte sono proteine chimasi che fosforilano insiemi di proteine specifiche quando attivati. 
\subsection{Recettori di superficie cellulare inoltrano segnali attraverso molecole di segnale intracellulare}
Molte molecole intracellulari inoltrano il segnale ricevuto dai recettori nell'interno della cellula generando una catena di eventi di segnalazione che altera proteine effettrici 
responsabili del cambio di comportamento della cellula. Alcune sono piccole e dette messaggeri secondari, generati in grandi quantit\`a in risposta all'attivazione del recettore e 
si diffondono diffondendo il segnale ad altre parti della cellula, alcune come AMP ciclico e \ce{Ca^{2+}} sono solubili in acqua e si diffondono nel citosol, mentre altre come i 
diacigliceroli si diffondono nel piano della membrana plasmatica. La maggior parte di queste molecole sono proteine che inoltrano il segnale generando messaggeri secondari o attivando
il successivo segnale o effettore. Si comportano come interruttori molecolari: quando ricevono il segnale passano in una forma attiva fino a che un altro processo le spegne. La classe 
pi\`u grande di questi consiste di proteine attivate o inattivate da fosforilazione. Lo stato \`e cambiato da una proteina chinasi che aggiunge convalentemente dei gruppi fosfato ad
amminoacidi specifici e da una proteina fosfatasi che lo rimuove. Esistono due gruppi di chinasi: il principale sono le chinasi a serina o treonina, le altre chinasi a tirosina. Molte
proteine di segnale intracellulare controllate da fosforilazione sono esse stesse chinasi e sono spesso organizzate in cascate di chinasi in cui una proteina chinasi forforilata 
fosforila la prossima chinasi nella sequenza e cos\`i via. Un altra classe di interruttori molecolari consiste di proteine leganti a GTP che si accendono quando GTP \`e legato e 
spengono quando \`e legato GDP. Nello stato attivo hanno un'attivit\`a di GTPasi intrinseca e si disattivano da sole idrolizzando il GTP legato in GDP. Di queste ne esistono due tipi:
pi\`u grando trimerice o proteine G che inoltrano il segnale da recettori accoppiati a esse che le attivano e piccole GTPasi monomeriche che inoltrano segnali da molte classi di 
recettori di superficie cellulare. Specifiche regolatorie specifiche controllano entrambi i tipi delle proteine leganti a GTP: le proteine GTPasi attivanti (GAP) disattivano il segnale
aumentando il tasso di idrolisi di GTP legato e i fattori di scambio del nucleotide guanina (GEF) le attivano promuovendo il rilascio di GDP che permette il legame di un nuovo GTP. Nel
caso delle proteine G il recettore attivato serve come GEF. Altri interruttori molecolari sono attivati e disattivati dal legame con un messaggero secondario come AMP ciclico o 
\ce{Ca^{2}} o altre modifiche covalenti. La maggior parte dei cammini di segnale contengono passi inibitori e possono accadere attivazioni da due segnali inibitori diversi. 
\subsection{I segnali intracellulari devono essere specifici e precisi nel citoplasma rumoroso}
Le molecole di segnale intracellulare condividono il citoplasma con molte molecole di segnale molto simili che controllano altri insiemi di processi cellulari. Si devono pertanto 
limitare le interazioni errate e difendersi da eventuali errori. Un modo per cui questo accade viene dall'alta affinit\`a delle interazioni tra le molecole di segnale e i loro partner.
Un altro sistema dipende dall'abilit\`a di molte proteine a valle del segnale di ignorare interazioni non volute: queste rispondono a un segnale solo quando il segnale a monte raggiunge
alte concentrazioni o livello di attivit\`a. Molecole individuali in una grande popolazione variano la loro attivit\`a o interazione con altre molecole. Questa variaiblit\`a di segnale
introduce altro rumore che il sistema di segnale riesce a gestire. Tale robustezza dipende dalla presenza di un meccanismo di cammini ridondanti.
\subsection{Complessi di segnalazione intracellulare si formano sui recettori attivati}
Un modo per aumentare la specificit\`a delle interazioni \`e di localizzarle nella stessa parte della cellula o in grandi complessi proteici, assicurandosi che interagiscono solo le
parti corrette. Tali meccanismi coinvolgono proteine di impalcatura che uniscono gruppi di proteine segnalanti in complessi segnalanti che a causa della prossimit\`a delle proteine
interagenti permettono risposte a segnali extracellulari sequenziali rapide, efficienti e selettive, evitando comunicazioni non volute tra cammini diversi. In altri casi i complessi di 
segnale si formano transientemente in risposta a un segnale extracellulare e si disassemblano quando il segnale termina, spesso intorno un recettore dopo che la molecola extracellulare
lo ha attivato. In molti casi la coda del recettore viene fosforilata e l'amminoacido funge da sito di attracco per l'assemblaggio delle proteine di segnale. In altri casi l'attivazione
del recettore produce di molecole di fosfolipidi modificate (fosfoinositidi) nella membrana plasmatica adiacente che recluta poi proteine specifiche e le attiva.
\subsection{Domini di interazione modulari mediano le interazioni tra proteine di segnale intracellulari}
La prossimit\`a indotta viene utilizzata per inoltrare i segnali da proteina a proteina lungo il cammino. L'assemblaggio di tali complessi dipende da piccoli domini di interazione che
si trovano in molte proteine di segnale intracellulare e si lega a un particolare motivo strutturale (catena peptidica, modifica covalente o dominio proteico) su un'altra proteina o 
lipide. Ci sono molti tipi di domini interagenti come domini Src omologia 2 (SH2) e leganti fosfotirosina (PTB) che si legano a tirosine fosforilate in una particlare sequenza specifica.
I domini Src omologia 3 (SH3) si legano a sequenze ricche di prolina e plecstrina omologia (PH) si legano alle teste cariche di fosfoinositidi specifici. Alcune proteine di segnale 
consistono di domini di interazione e funzionano solo come adattori. Un altro modo per unire recettori e proteine di segnale \`e di concentrarli in una specifica regione della cellula
come nel ciglio primario. 
\subsection{La relazione tra segnale e risposta varia in cammini di segnale diversi}
\begin{itemize}
	\item Il tempo di risposta varia in diversi sistemi di segnale in base alla velocit\`a richiesta.
	\item La sensibilit\`a al segnale extracellulare \`e controllata cambiando il numero o l'affinit\`a dei recettori. Nell'amplificazione un piccolo numero di recettori causa
		una grande risposta intracellulare creando grandi quantit\`a di messaggeri secondari o attivando molte copie di una proteina di segnale a valle.
	\item L'intervallo dinamico di attivazione \`e simile alla sensibilit\`a: in alcuni sistemi sono responsivi solo a uno stretto intervallo di concentrazioni di segnale. Questo
		si ottiene attraverso meccanismi di adattamento che modificano la responsivit\`a del sistema in base al segnale.
	\item La persistenza della risposta varia grazie a vari meccanismi come circuiti a feedback positivo.
	\item Il processo dei segnali pu\`o convertire un segnale semplice in una risposta complessa attraverso feedback.
	\item L'integrazione permette a una risposta di essere governata da diversi segnali e dipende da rilevatori di coincidenze che agiscono come gate AND.
	\item La coordinazione di risposte multiple pu\`o essere ottenuta da un singolo segnale e dipende dal meccanismo di distribuzione di esso a diversi effettori che permettono la
		modulazione della forza della risposta.
\end{itemize}
\subsection{La velocit\`a di una risposta dipende dal turnover delle molecole di segnale}
La velocit\`a di ogni risposta dipende dalla natura delle molecole di segnali intracellulari che la svolgono e pu\`o cambiare da pochi millisecondi a molte ore. Si deve
