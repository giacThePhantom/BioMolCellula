\chapter{Segnalazione cellulare}
Ogni cellula monitora l'ambiente extra e intracellulare, processa le informazioni che raccoglie e risponde appropriatamente. Al cuore di questo sistema ci sono proteine regolatorie che
producono segnali chimici inviati da un posto all'altro nei corpi o nella cellula, processati lungo la via e integrati con altri. 
\section{Principi di segnalazione cellulare}
Molti organismi unicellulari primordiali avevano sviluppato meccanismi per rispondere alla presenza di altre cellule. Molti batteri rispondono a segnali chimici secreti dai loro vicini
e si accumulano ad alte densit\`a di popolazione e attraverso questo quorum sensing permette ai batteri di coordinare i propri comportamenti. Durante l'evoluzione degli organismi 
multicellulari la comunicazione intercellulare \`e diventata estremamente complessa: essi sono societ\`a di cellule i cui il benessere del singolo \`e messo da parte per il beneficio
dell'organismo. La comunicazione tra cellule in organismi multicellulari \`e mediata da molecole di segnale extracellulare, alcune delle quali operano a grandi distanze, altre solo tra
vicini immediati. La maggior parte delle cellule sono sia mittenti che destinatari. La recezione dipende da proteine recettrici che si legano alla molecola di segnale. Il legame
attiva il recettore che attiva sistemi o cammini di segnalazione intracellulare che dipendono da proteine di segnalazione intracellulare che processano il segnale all'interno
della cellula e lo distribuiscono agli obiettivi. Gli obiettivi alla fine di questi sistemi sono proteine effettrici che sono alterate dal segnale e implementano il cambio appropriato
nel comportamento cellulare. Le caratteristiche fondamentali della segnalazione sono conservate nell'evoluzione degli eucarioti nonostante per alcuni organismi siano diventate molto
pi\`u elaborate. Il genoma umano contiene pi\`u di $1500$ geni per codificare proteine recettrici. 
\subsection{Segnali extracellulari possono agire lungo distanze corte o lunghe}
Molte molecole di segnalazione extracellulare rimangono legate alla superficie della cellula segnalatrice e influenzano solo la cellula a contatto, importante durante lo sviluppo e 
nella risposta immunitaria. Nella maggior parte dei casi le cellule di segnale secernono mediatori locali che agiscono solo sull'ambiente locale nella segnalazione paracrina in cui
tipicamente le cellule mittente e destinatarie sono di tipo diverso, ma pu\`o accadere che le cellule rispondano a un segnale che loro stesse hanno mandato (segnalazione autocrina).
Negli organismi multicellulari grandi si rendono necessari meccanismi di segnalazione a lungo raggio per coordinare il comportamento delle cellule in parti remote del corpo e si sono
evoluti tipi di cellule specializzate per comunicazione intercellulare lungo grandi distanze come i neuroni che estendono lunghi assoni ramificati che permettono loro di contattare
cellule obiettivo distanti attraverso sinapsi chimica. Quando un neurone \`e attivato da uno stimolo invia un potenziale d'azione lungo l'assone che quando raggiunge la sinapsi 
causa la secrezione di un segnale chimico detto neurotrasmettitore. Un altra strategia sono le cellule endocrine che secernono gli ormoni nel flusso sanguigno che le trasporta in
lontananza. 
\subsection{Molecole di segnale extracellulare si legano a recettori specifici}
Le cellule negli animali multicellulari comunicano attraverso centinaia di tipi di molecole di segnale come proteine, piccoli peptidi, amminoacidi, nucleotidi, steroidi, retinoidi, 
derivati degli acidi grassi e gas dissolti. Queste molecole di segnale sono rilasciate nello spazio extracellulare attraverso esocitosi dalla cellula segnalatrice, mentre alcune sono
emesse per diffusione attraverso la membrana cellulare, mentre altre rimangono attaccate alla superficie esterna della cellula. Le proteine di segnale transmembrana possono operare in
questo modo o possono possedere domini che possono rilasciare attraverso rottura proteolitica e agire a distanza. La cellula obiettivo risponde in ogni caso attraverso un recettore che 
lega la molecola di segnale e inizia una risposta. Il sito di legame ha una struttura complessa che gli dona alta specificit\`a e alta affinit\`a. Nella maggior parte dei casi i 
recettori sono proteine transmembrana sulla superficie. Quando legano il segnale si attivano e generano segnali intracellulari che alterano il comportamento della cellula. In 
altri casi si trovano all'interno e la molecola di segnale deve entrare nella cellula, cosa che richiede che il segnale sia piccolo e abbastanza idrofobico da diffondersi attravero 
la membrana plasmatica. 
\subsection{Ogni cellula \`e programmata per rispondere a specifiche combinazioni di segnali extracellulari}
Una cellula \`e esposta a centinaia di segnali diversi e risponde selettivamente esprimendo solo i recettori che rispondono ai segnali richiesti per la regolazione di s\`e. La maggior
parte delle cellule rispondono a molti segnali diversi e alcuni segnali influenzano la risposta ad altri segnali. In principio le molecole di segnale possono essere usate in
combinazioni illimitate per controllare i diversi comportamenti della cellula con alta specificit\`a. Una molecola di segnale ha diversi effetti in base al tipo della cellula
obiettivo. 
\subsection{Ci sono tre classi principali di proteine recettrici sulla superficie cellulare}
La maggior parte delle molecole di segnale extracellulari si legano a proteine recettrici specifiche sulla superficie e non entrano nel citosol. Questi recettori agiscono come
trasduttori di segnale convertendo il legame in un segnale intracellulare che altera il comportamento della cellula. Tali proteine appartengono a tre categorie definite dal loro
meccanismo di trasduzione. I recettori accoppiati a canali ionici o canali ionici transmitter-gated o recettori ionotropici sono coinvolti nelle rapide segnalazioni sinaptiche, mediato
da neurotrasmettitori che aprono o chiudono transientemente un canale ionico formato da una proteina con cui si legano, cambiando la permeabilit\`a della membrana plasmatica e
l'eccitabilit\`a della molecola postsinaptica. La maggior parte di questi appartengono a una famiglia di proteine transmembrana a multipassaggio omologhe. I recettori accoppiati a G 
agiscono regolando indirettamente l'attivit\`a di una proteina obiettivo legata alla membrana, un enzima o un canale ionico. Una proteina legata a GTP trimerica (proteina G) media
l'interazione tra il recettore attivato e la proteina obiettivo. L'attivazione della proteina obiettivo pu\`o cambiare la concentrazione di pi\`u molecole di segnalazione intracellulare
o la permeabilit\`a ionica della membrana plasmatica, le prime agiscono in turni per cambiare il comportamento di altre proteine segnalatrici nella cellula. I recettori accoppiati da
enzimi funzionano come enzimi o si associano direttamente con gli enzimi che attivano. Sono proteine transmembrana a singolo passaggio con il sito di legame all'esterno e il sito 
catalitico all'interno. Sono eterogenei in struttura, ma la maggior parte sono proteine chinasi che fosforilano insiemi di proteine specifiche quando attivati. 
\subsection{Recettori di superficie cellulare inoltrano segnali attraverso molecole di segnale intracellulare}
Molte molecole intracellulari inoltrano il segnale ricevuto dai recettori nell'interno della cellula generando una catena di eventi di segnalazione che altera proteine effettrici 
responsabili del cambio di comportamento della cellula. Alcune sono piccole e dette messaggeri secondari, generati in grandi quantit\`a in risposta all'attivazione del recettore e 
si diffondono diffondendo il segnale ad altre parti della cellula, alcune come AMP ciclico e \ce{Ca^{2+}} sono solubili in acqua e si diffondono nel citosol, mentre altre come i 
diagliceroli si diffondono nel piano della membrana plasmatica. La maggior parte di queste molecole sono proteine che inoltrano il segnale generando messaggeri secondari o attivando
il successivo segnale o effettore. Si comportano come interruttori molecolari: quando ricevono il segnale passano in una forma attiva fino a che un altro processo le spegne. La classe 
pi\`u grande di questi consiste di proteine attivate o inattivate da fosforilazione. Lo stato \`e cambiato da una proteina chinasi che aggiunge convalentemente dei gruppi fosfato ad
amminoacidi specifici e da una proteina fosfatasi che lo rimuove. Esistono due gruppi di chinasi: il principale sono le chinasi a serina o treonina, le altre chinasi a tirosina. Molte
proteine di segnale intracellulare controllate da fosforilazione sono esse stesse chinasi e sono spesso organizzate in cascate di chinasi in cui una proteina chinasi forforilata 
fosforila la prossima chinasi nella sequenza e cos\`i via. Un altra classe di interruttori molecolari consiste di proteine leganti a GTP che si accendono quando GTP \`e legato e 
spengono quando \`e legato GDP. Nello stato attivo hanno un'attivit\`a di GTPasi intrinseca e si disattivano da sole idrolizzando il GTP legato in GDP. Di queste ne esistono due tipi:
pi\`u grande trimerico o proteine G che inoltrano il segnale da recettori accoppiati a esse che le attivano e piccole GTPasi monomeriche che inoltrano segnali da molte classi di 
recettori di superficie cellulare. Specifiche regolatorie specifiche controllano entrambi i tipi delle proteine leganti a GTP: le proteine GTPasi attivanti (GAP) disattivano il segnale
aumentando il tasso di idrolisi di GTP legato e i fattori di scambio del nucleotide guanina (GEF) le attivano promuovendo il rilascio di GDP che permette il legame di un nuovo GTP. Nel
caso delle proteine G il recettore attivato serve come GEF. Altri interruttori molecolari sono attivati e disattivati dal legame con un messaggero secondario come AMP ciclico o 
\ce{Ca^{2+}} o altre modifiche covalenti. La maggior parte dei cammini di segnale contengono passi inibitori e possono accadere attivazioni da due segnali inibitori diversi. 
\subsection{I segnali intracellulari devono essere specifici e precisi nel citoplasma rumoroso}
Le molecole di segnale intracellulare condividono il citoplasma con molte molecole di segnale molto simili che controllano altri insiemi di processi cellulari. Si devono pertanto 
limitare le interazioni errate e difendersi da eventuali errori. Un modo per cui questo accade viene dall'alta affinit\`a delle interazioni tra le molecole di segnale e i loro partner.
Un altro sistema dipende dall'abilit\`a di molte proteine a valle del segnale di ignorare interazioni non volute: queste rispondono a un segnale solo quando il segnale a monte raggiunge
alte concentrazioni o livello di attivit\`a. Molecole individuali in una grande popolazione variano la loro attivit\`a o interazione con altre molecole. Questa variaiblit\`a di segnale
introduce altro rumore che il sistema di segnale riesce a gestire. Tale robustezza dipende dalla presenza di un meccanismo di cammini ridondanti.
\subsection{Complessi di segnalazione intracellulare si formano sui recettori attivati}
Un modo per aumentare la specificit\`a delle interazioni \`e di localizzarle nella stessa parte della cellula o in grandi complessi proteici, assicurandosi che interagiscono solo le
parti corrette. Tali meccanismi coinvolgono proteine di impalcatura che uniscono gruppi di proteine segnalanti in complessi segnalanti che a causa della prossimit\`a delle proteine
interagenti permettono risposte a segnali extracellulari sequenziali rapide, efficienti e selettive, evitando comunicazioni non volute tra cammini diversi. In altri casi i complessi di 
segnale si formano transientemente in risposta a un segnale extracellulare e si disassemblano quando il segnale termina, spesso intorno un recettore dopo che la molecola extracellulare
lo ha attivato. In molti casi la coda del recettore viene fosforilata e l'amminoacido funge da sito di attracco per l'assemblaggio delle proteine di segnale. In altri casi l'attivazione
del recettore produce di molecole di fosfolipidi modificate (fosfoinositidi) nella membrana plasmatica adiacente che recluta poi proteine specifiche e le attiva.
\subsection{Domini di interazione modulari mediano le interazioni tra proteine di segnale intracellulari}
La prossimit\`a indotta viene utilizzata per inoltrare i segnali da proteina a proteina lungo il cammino. L'assemblaggio di tali complessi dipende da piccoli domini di interazione che
si trovano in molte proteine di segnale intracellulare e si lega a un particolare motivo strutturale (catena peptidica, modifica covalente o dominio proteico) su un'altra proteina o 
lipide. Ci sono molti tipi di domini interagenti come domini Src omologia 2 (SH2) e leganti fosfotirosina (PTB) che si legano a tirosine fosforilate in una particlare sequenza specifica.
I domini Src omologia 3 (SH3) si legano a sequenze ricche di prolina e plecstrina omologia (PH) si legano alle teste cariche di fosfoinositidi specifici. Alcune proteine di segnale 
consistono di domini di interazione e funzionano solo come adattori. Un altro modo per unire recettori e proteine di segnale \`e di concentrarli in una specifica regione della cellula
come nel ciglio primario. 
\subsection{La relazione tra segnale e risposta varia in cammini di segnale diversi}
\begin{itemize}
	\item Il tempo di risposta varia in diversi sistemi di segnale in base alla velocit\`a richiesta.
	\item La sensibilit\`a al segnale extracellulare \`e controllata cambiando il numero o l'affinit\`a dei recettori. Nell'amplificazione un piccolo numero di recettori causa
		una grande risposta intracellulare creando grandi quantit\`a di messaggeri secondari o attivando molte copie di una proteina di segnale a valle.
	\item L'intervallo dinamico di attivazione \`e simile alla sensibilit\`a: in alcuni sistemi sono responsivi solo a uno stretto intervallo di concentrazioni di segnale. Questo
		si ottiene attraverso meccanismi di adattamento che modificano la responsivit\`a del sistema in base al segnale.
	\item La persistenza della risposta varia grazie a vari meccanismi come circuiti a feedback positivo.
	\item Il processo dei segnali pu\`o convertire un segnale semplice in una risposta complessa attraverso feedback.
	\item L'integrazione permette a una risposta di essere governata da diversi segnali e dipende da rilevatori di coincidenze che agiscono come gate AND.
	\item La coordinazione di risposte multiple pu\`o essere ottenuta da un singolo segnale e dipende dal meccanismo di distribuzione di esso a diversi effettori che permettono la
		modulazione della forza della risposta.
\end{itemize}
\subsection{La velocit\`a di una risposta dipende dal turnover delle molecole di segnale}
La velocit\`a di ogni risposta dipende dalla natura delle molecole di segnali intracellulari che la svolgono e pu\`o cambiare da pochi millisecondi a molte ore. Si deve considerare 
pertanto anche considerare cosa succede quando il segnale viene a mancare. Nell a maggior parte dei tessuti adulti la risposta diminuisce quando un segnale finisce con un effetto 
transiente in quanto il segnale altera la concentrazione di molecole instabili che vengono continuamente cambiate e quando il segnale cessa la degradazione elimina le tracce della sua
azione. La velocit\`a di risposta alla rimozione di un segnale dipende dal tasso di distruzione o turnover della molecola che il segnale modifica. Il turnover pu\`o anche determinare
la prontezza della risposta quando arriva il segnale extracellulare. Nella maggior parte dei casi esistono proteine con vite corte con ruolo regolatorio la cui concentrazione \`e 
rapidamente concentrata dalla cellula cambiando il tasso di sintesi. Molte risposte ai segnali dipendono dalla conversione di proteine da una forma inattiva in attiva, la cui attivazione
deve essere rapidamente e continuamente annullata per permettere segnalazioni rapide. Il processo di attivazione ha un ruolo centrale in determinare la magnitudine, rapidit\`a e 
durata della risposta. 
\subsection{Le cellule possono rispondere improvvisamente a un segnale in graduale aumento}
Alcuni sistemi di segnalazione sono in grado di generare risposte graduate su un largo intervallo di concentrazioni di segnale extracellulare, mentre altri generano risposte 
significative solo quando tale concentrazione supera un valore di soglia con una risposta sigmoidale in cui basse concentrazioni di stimolo non hanno grande effetto ma poi la risposta
aumenta ripidamente e continuamente a livelli di stimolo intermedi. Tali sistemi filtrano per ridurre risposte inappropriate a rumore ma rispondono con alta sensibilit\`a quando il 
segnale arriva a un piccolo intervallo di concentrazione. Un secondo tipo \`e detto discontinuo o binario in cui la risposta \`e completamente accesa (a volte irreversibilmente) quando
il segnale raggiunge la concentrazione di soglia e sono utili per controllare la scelta rispetto a due stati cellulari e coinvolgono feedback positivi. La risposta sigmoidale viene
raggiunta grazie a vari meccanismi. In uno pi\`u di una molecola di segnale intracellulare deve legarsi alla sua proteina obiettivo a valle per indurre una risposta, quando una proteina
deve venir fosforilata a pi\`u siti e quando molecole che attivano un enzima inibiscono un altro che catalizza la reazione inversa. 
\subsection{Il feedback positivo pu\`o generare una risposta binaria}
La maggior parte dei sistemi di segnalazione incorporano circuiti a feedback in cui il risultato di un processo regola lo stesso. Quelli che regolano la segnalazione cellulare possono
operare esclusivamente nella cellula obiettivo o coinvolgono la secrezione di segnali extracellulari. Il feedback positivo in un cammino di segnalazione pu\`o trasformare il 
comportamento della cellula rispondente: se ha forza moderata rende pi\`u ripida la forza del segnale, ma se \`e forte abbastanza pu\`o produrre una risposta binaria e spesso la
cellula pu\`o autosostenere questa condizione di risposta che pu\`o persistere anche dopo che il segnale \`e terminato. In quest'ultimo caso il sistema \`e detto bistabile: pu\`o 
esistere solo in uno stato attivo o inattivo e uno stimolo temporaneo pu\`o invertire tale stato. Grazie a questo circuito si possono indurre cambi nella cellula a lungo termine che
possono persistere anche nella progenie. 
\subsection{Il feedback negativo \`e un motivo comune nei sistemi di segnalazione}
Il feedback negativo agisce contro l'effetto di uno stimolo abbreviando e limitando il livello della risposta rendendo il sistema meno sensibile a perturbazioni. Si possono ottenere
diverse risposte quando il feedback opera con pi\`u forza. Un feedback negativo abbastanza ritardato pu\`o produrre risposte oscillatorie che possono persistere anche fino a quando lo
stimolo \`e presente o possono generarsi spontaneamente senza segnali esterni. La presenza di feedback positivi all'interno di tali sistemi generano oscillazioni pi\`u ripide. Se il 
feedback negativo opera con un breve ritardo il sistema si comporta come un rivelatore di cambiamento: da una forte risposta allo stimolo che crolla rapidamente anche se lo stimolo
persiste e varia solo con le sue variazioni nel fenomeno di adattamento.
\subsection{Le cellule possono variare la loro sensibilit\`a a un segnale}
Rispondendo a diversi stimoli le cellule e gli organismi sono in grado di individuare la stessa percentuale di cambi in un grande intervallo di forze di stimolo attraverso adattamento o 
desensibilizzazione dove esposizione prolungata a uno stimolo diminuisce la risposta della cellula a tale livello di stimolo. L'adattamento permette alla cellula di rispondere a cambi
nella concentrazione di segnali extracellulari in un grande intervallo di concentrazioni. Il meccanismo alla base \`e un feedback negativo che opera con breve ritardo: una risposta 
forte modifica il macchinario di segnale coinvolto in modo che questo si resetti in modo da diventare meno responsivo allo stesso livello di segnale. L'adattamento pu\`o derivare 
dall'inattivazione dei recettori attraverso endocitosi e distrutti in receptor down-regulation o la loro inattivazione fosforilandosi dopo essere stati attivati. Pu\`o accadere anche
a valle dei recettori cambiando proteine coinvolte nella trasudzione o attraverso la produzione di un inibitore. 
\section{Segnalazione attraverso recettori accoppiai a proteine G}
I recettori accoppiati alle proteine G (GPCR) formano la pi\`u grande famiglia di recettori sulla superficie cellulare e mediano la maggior parte delle risposte dal mondo esterno e da
altre molecole come ormoni, neurotrasmettitori e mediatori locali. Le molecole di segnale che agiscono sui GPCR variano in struttura e funzione e includono proteine, piccoli peptidi, 
derivati di amminoacidi e acidi grassi, fotoni e tutte le molecole che generano gusto o odore. Esistono diversi recettori per lo stesso segnale che generano risposte diverse e sono
espressi in diversi tipi di cellule. Tutti i GPCR hanno una struttura simile: consistono di una singola catena polipeptidica che attraversa il bistrato sette volte formando una 
struttura cilindrica con un sito profondo di legame al centro e tutti usano le proteine G per inoltrare i segnali all'interno della cellula. La famiglia include la rodopsina, la proteina
attivata dalla luce negli occhi dei vertebrati e un grande numero di recettori olfattori nel naso. Altre si trovano anche in organismi unicellulari. 
\subsection{Proteine G trimeriche inoltrano i segnali da GPCR}
Quando un segnale extracellulare si lega a un GPCR il recettore subisce un cambio conformazionale che gli permette di attivare una proteina trimerica legante GTP o proteina G che 
accoppia il recettore all'enzima o al canale ionico nella membrana. In alcuni casi \`e associata fisicamente, mentre in altri si lega al recettore solo dopo la sua attivazione. Le
proteine G sono composte da tre subunit\`a $\alpha$ $\beta$ e $\gamma$. Nello stato inattivo $\alpha$ ha legato GDP e quando GPCR viene attivato agisce come un GEF che induce la
liberazione del GDP permettendo il legame con GTP che poi causa un cambio conformazionale in $\alpha$ rilasciando la proteina dal recettore e causando la dissociazione di $\alpha$
da $\beta\gamma$ che poi interagiscono con vari obiettivi inoltrando il segnale. L'attivit\`a di GTPasi e pertanto la velocit\`a dell'idrolisi viene aumentata  da un regolatore di 
segnalazione di proteine G segnalanti (RGS) che agiscono come GAP specifiche alla subunit\`a $\alpha$ e aiutano a chiudere la risposta in tutti gli eucarioti. 
\subsection{Alcune proteine G regolano la produzione di AMP ciclico}
L'AMP ciclico (cAMP) agisce come un messaggero secondario in alcuni cammini e un segnale extracellulare pu\`o produrre grandi variazioni di cAMP. Questa risposta rapida richiede 
l'equilibrio di sintesi e degradazione della molecola. AMP ciclico \`e sintetizzato dall'ATP da un adenilil ciclasi e degradato da AMP ciclico fosfodiesterasi. La prima \`e una grande 
proteina transmembrana a multi passo con il dominio catalitico sul lato citosolico. La maggior parte dei suoi isoformi sono regolati da proteine G e \ce{Ca^{2+}}. Molti segnali 
extracellulari aumentano la concentrazione di cAMP nella cellula attivando GPCR accoppiate a proteine G stimolatorie ($G_s$). La subunit\`a $\alpha$ di $G_s$ attivate lega e attiva 
adenilil ciclasi. Altri segnali riducono tale concentrazione attivando proteine G inibitorie ($G_i$) che inibiscono l'adenilil ciclasi. Tipi cellulari diversi rispondono diversamente a
un aumento nella concentrazione di cAMP: le cellule grasse attivano adenili ciclasi in risposta a multipli ormoni che stimolano la degradazione di trigliceridi in acidi grassi. 
\subsection{La proteina chinas dipendente da AMP ciclico (PKA) media la maggior parte degli effetti dell'AMP ciclico}
cAMP effettua la sue funzione attivando PKA che forsforila serine o treonine specifiche su proteine obiettivo come proteine di segnalazione extracellulare e proteine effettrici regolando
la loro attivit\`a. In uno stato inattivo PKA consiste di un complesso di due subunit\`a catalitiche e due regolatorie. Il legame di cAMP alle subunit\`a regolatorie altera la loro
conformazione facendole dissociare. Le subunit\`a catalitiche ora sono attivate per fosforilare specifiche proteine obiettivo. Le subunit\`a regolatorie o A-chinasi sono importanti 
per la sua localizzazione: una proteina di ancoraggio di A-chinasi lega le subunit\`a regolatorie e componenti del citoscheletro o una membrana bloccando il complesso enzimatico in
un particolare compartimento cellulare. Se alcune risposte mediate da cAMP possono accadere in secondi, mentre altre dipendono da cambi della trascrizione di geni specifici e possono
arrivare ore. Su molti geni controllati da cAMP si trova una sequenza cis-regolatoria detta elemento di risposta a AMP ciclico (CRE) e il regolatore di trascrizione proteina legante a 
CRE (CREB) la riconosce. Quando PKA si attiva fosforila CREB su una singola serina che poi rebluta un coattivatore trascrizionale detto proteina legante a CREB (CBP) che stimola la
trascrizione dei geni obiettivo. 
\subsection{Alcune proteine G segnalano attraverso fosfolipidi}
Molti GPCR svolgono i loro effetti attraverso proteine G che attivano l'enzima legato a membrana plasmatica fosfolipasi C-$\beta$ (PLC$\beta$) che agisce un fosfolipide inositolo 
fosforilato (fosfoinositide) detto fosfatidilinositolo $4,5$-bifosfato (PI(4, 5)$P_2$). Tali recettori lo fanno attraverso $G_q$ che attiva la fosfolipasi che rompe il PI(4, 5)$P_2$
per generare inositolo 1, 4, 5-trifosfato ($IP_3$) e diacilglicerolo e il cammino si segnale si divide in due rami. I$P_3$ \`e una molecola solubile in acqua che lascia la membrana
plasmatica e si diffonde nel citosol e quando raggiunge l'ER si lega e apre canali di rilascio di \ce{Ca^{2+}} I$P_3$-gated o recettori I$P_3$ nella membrana ER. Il \ce{Ca^{2+}} nell'ER
viene rilasciato aumentando la sua concentrazione nel citosol che propaga il segnale influenzando proteine intracellulari a lui sensibili. Allo stesso tempo l'altro prodotto il 
diacilgicerolo agisce come un messaggero secondario ma rimane inglobato nella membrana plasmatica con diversi ruoli di segnalazione, uno dei quali \`e di attivare una proteina chinasi
C (PKC) dipendente da \ce{Ca^{2+}} l'aumento iniziale in \ce{Ca^{2+}} causato da I$P_3$ altera la PKC in modo che si traslochi dal citosol alla membrana plasmatica dove \`e attivata da
\ce{Ca^{2+}}, diaglicerolo e la fosfatidilserina negativamente carico. PKC fosforila proteine in base al loro tipo cellulare. Il diaglicerolo pu\`o essere ulteriormente rotto 
per rilasciare acido arachidonico che pu\`o agire come segnale o essere usato nella sintesi di eicosanoidi che partecipano in risposte infiammatorie e del dolore. 
\subsection{\ce{Ca^{2+}} agisce come un mediatore intracellulare onnipresente}
Molti segnali extracellulari causano un aumento nella concentrazione citosilica di \ce{Ca^{2+}} che ha molte funzioni in diversi tipi cellulari \`e effettivo in quanto la sua 
concentrazione nel citosol \`e bassa mentre nel fluido extracellulare e nel lume dell'ER \`e elevata con un grande gradiente che tende a spostarlo verso il citosol. Quando un segnale
apre transientemnte canali di \ce{Ca^{2+}} questo si sposta velocemente nel citosol attivando proteine a lui sensibili. Alcuni stimoli causano un influsso dal fluido extracellulare 
mentre altri agiscono attraverso recettori I$P_3$ indiretti che lo rilasciano dall'ER. La membrana dell'ER contiene anche un secondo tipo di canale \ce{Ca^{2+}} regolato detto
recettore rianodina che si apre in risposta a un aumento dei livelli di \ce{Ca^P{2+}} amplificando il segnale. Molti meccanismi terminano velocemente il segnale \ce{Ca^{2+}} e sono
responsabili per mantenere la sua concentrazione bassa nella condizione di riposo. Si trovano pompe per esso nella membrana plasmatica e dell'ER che idrolizzano l'ATP per pomparlo
fuori dal citosol. 
\subsection{Il feedback genera onde e oscillazioni di \ce{Ca^{2+}}}
I recettori I$P_3$ e rianodina nella membrana ER sono stimolati da concentrazioni moderate di \ce{Ca^{2+}} e il rilascio di calcio indotto da \ce{Ca^{2+}} (CICR) causa feedback positivo.
Quando una molecola stimola la produzione di I$P_3$ si notano piccole uscite di \ce{Ca^{2+}} in regioni discrete della cellula che riflettono aperture dei canali locali. Il segnale pu\`o
rimanere localizzato a causa di buffer che restringono la sua diffusione, ma se il segnale \`e abbastanza forte e persistente le concentrazioni locali raggiungono un livello sufficiente
per attivare recettori I$P_3$ causando onde rigenerative di rilascio di \ce{Ca^{2+}} che si muove attraverso il citosol. I recettori I$P_3$ e riadonina sono inibiti dopo un ritardo da
alte concentrazioni di \ce{Ca^{2+}} e il suo declino rallenta questo feedback negativo che risulta in un'oscillazione di concentrazione dello ione. La frequenza e  ampiezza 
dell'oscillazione possono essere modulate da altri meccanismi di segnalazione. 
\subsection{Proteine chinasi dipendenti da \ce{Ca^{2+}} e da calmodulina mediano molte risposte a segnali di \ce{Ca^{2+}}}
Varie proteine leganti a \ce{Ca^{2+}} aiutano a inoltrare il segnale di \ce{Ca^{2+}} citosilico e la pi\`u importante \`e la calmodulina che funziona come un recettore multiscopo di
\ce{Ca^{2+}}, governando molti processi da esso regolati. Consiste di una singola catena polipeptidica con quattro siti di legame per \ce{Ca^{2+}}. Quando viene attivata subisce un 
cambio conformazionale e in base al fatto che due ioni si devono legare prima che acquisisca la conformazione attiva la proteine mostra una risposta sigmoidale all'aumento di 
concentrazione di \ce{Ca^{2+}}. In alcuni casi serve come subunit\`a permanente regolatoria di un complesso enzimatico, ma di solito il legame le permette di legare diverse proteine
obiettivo nella cellula alterando la loro attivit\`a. Quando la calmodulina si lega alla proteina obiettivo cambia la sua conformazione. Tra i molti obiettivi si trovano enzimi e 
proteine di trasporto. Molti effetti del \ce{Ca^{2+}} sono indiretti e mediati dalla fosforilazione catalizzata da proteine chinasi dette \ce{Ca^{2+}}/calmodulina dipendenti 
(CaM-chinasi) che fosforilano regolatori di trascrizione come CREG e attivano o inibisco la trascrizione di geni specifici. 
\subsection{Alcune proteine G regolano direttamente canali ionici}
In alcuni casi le proteine G attivano o disattivano canali ionici nella membrana plasmatica alterandone la sua permeabilit\`a agli ioni. Lo possono fare con un legame a essi attraverso
le subunit\`a $\beta\gamma$ o stimolando la fosforilazione del canale o distruggendo i nucleotidi ciclici che attivano o disattivano i canali direttametne. 
\subsection{Olfatto e vista dipendono su GPCR che regolano canali ionici}
Gli odori vengono riconosciuti attraverso neuroni olfattori recettori che utilizzano GPCR dette recettori olfattori che agiscono attraverso cAMP. Quando stimolati dal legame con la 
sostanza attivano una proteina $G_{olf}$ che attiva adenilil ciclasi. L'aumento di cAMP apre canali cationici permettendo un influsso di \ce{Na+} che depolarizza il neurone recettore e
inizia un impulso nervoso che viaggia dal suo assone fino al cervello. Ogni diverso neurone produce solo un recettore specifico che aiuta a direzionare l'assone a specifici neuroni
obiettivo con cui si connette nel cervello. Un diverso insieme di GPCR agisce similmente per mediare la risposta a feromoni. La vista usa un processo simile: sono coinvolti canali
ionici gated da nucleotidi ciclici come GMP ciclico di cui una continua rapida sintesi e degradazione controllano la concentrazione del citosol. Nelle risposte di trasduzione della vista
l'attivazione del recettore stimolata dalla luce caysa una diminuzione nel livello del nucleotide ciclico. I fotorecettori bastoncini nella retina sono responsabili per la visione
senza colore con poca luce, mentre i fotorecettori a cono sono responsabili per la visione a colori cono molta luce. I primi sono cellule altamente specializzare con un segmento interno,
uno esterno, un corpo cellulare e una regione sinaptica dove vengono inviati segnali chimici a una cellula nervosa retinale che inoltra il segnale a un altra cellula nervosa nella 
cellula e poi al cervello. L'apparato di fototrasduzione si trova nel segmento esterno del bastoncello che contiene uno stack di dischi formato da un sacco chiuso da membrana pieno di
rodopsine sensibili alla luce. La membrana plasmatica di questo segmento contiene canali cationici gated da GMP ciclico che legato a questi canali li lascia aperti nel buio. La 
luce causa un'iperpolarizzazione nella membrana plasmatica in quanto l'attivazione delle rodopsine diminuisce la concentrazione di GMP ciclico e chiude i canali cationici. Ogni 
rodopsina contiene un cromoforo, 11-cis retinale che isomerizza a trans retinale quando assorbe un singolo fotone. L'isomerizzazione cambia la forma della retina causando un cambio 
conformazionale nell'opsina e della proteina G transducina $G_t$ la cui subunit\`a $\alpha$ attiva GMP ciclica fosfodiesterasi che idrolizza GMP ciclico in modo che i suoi livelli
diminuiscono. Questo abbassamento permette ai canali sensitivi al GMP ciclico di chiudersi in modo da far passare velocemente il segnale dalla membrana del disco a quella della
membrana plasmatica e il segnale elettrico viene convertito in uno elettrico. I bastoncelli usano diversi circuiti a feedback negativo per convertirsi velocemente in uno 
stato di riposo: la rodopsina chinasi RK fosforila la coda citosilica della rodopsina attivata inibendo parzialmente la sua capacit\`a di attivare la trasducina. L'arrestina si lega poi
alla rodopsina inibendo la sua attivit\`a ulteriormente. Contemporaneamente all'arrestina una proteina RGS si attiva alla trasducina stimolando l'idrolizzazione del suo GTP in GDP
disattivandola. Oltre a questo i canali cationici chiusi sono permeabili a \ce{Ca^{2+}} e a \ce{Na+} e quando chiudono si inibisce l'influsso di questi ioni diminuiendo la 
loro concentrazione stimolando guanilil ciclasi in modo che faccia ritornare il GMP ciclico ai livelli di riposo. I meccanismi di feedback aiutano anche l'adattamento del bastoncello
diminuendo la sua risposta a un esposizione continua con la luce. 
\subsection{L'ossido nitrico \`e un mediatore di segnalazione gassoso che passa tra le cellule}
Alcune molecole sono abbastanza piccole e/o idrofobiche per passare attraverso la membrana plasmatica e trasportare il segnale a cellule vicine come l'ossido nitrico \ce{NO}. Nei 
mammiferi una delle sue funzioni \`e di rilassare i muscoli lisci nelle pareti dei capillari. L'acetilcolina stimola la sintesi di \ce{NO} attivando una GPCR sulla membrana di cellule
endoteliali che si trovano all'interno del capillare. Il recettore causa la sintesi di I$P_3$ e rilascio di \ce{Ca^{2+}} stimolando l'enzima che sintetizza \ce{NO} che si diffonde 
fuori dalla cellula verso le cellule di muscolo liscio che rilassa e la dilatazione del capillare. Agisce localmente in quanto ha vita breve. \ce{NO} viene sintetizzato dalla 
damminazione dell'arginina catalizzato dall'\ce{NO} sintetasi(\ce{NOS}) che nelle cellule endoteliali \`e detto eNOS, in quelle nervose nNOS, entrambi stimolati da \ce{Ca^{2+}}. I
macrofagi producono un NOS inducibile (iNOS), continuamente attivo ma sintetizzato solo quando la cellula \`e attivata. Il \ce{NO} si lega reversibilmente al ferro nel sito attivo
della guanilil ciclasi stimolando la sintesi di GMP ciclico. 
\subsection{Messaggeri secondarie e cascate enzimatiche amplificano i segnali}
I diversi cammini di segnalazione intracellulare causati da GPCR dipendono tutti da catene di inoltrazione di proteine di segnalazione intracellulare e messaggeri secondarie che 
forniscono molte opportunit\`a per l'amplificazione della risposta al segnale. Una singola molecola attivata Pu\`o catalizzare l'attivazione di centinaia di molecole a valle della
catena amplificando la risposta di diversi ordini di grandezza in quanto questi messaggeri hanno la funzione di effettori allosterici che attivano enzimi specifici o canali ionici
un singolo segnale extracellulare pu\`o alterare migliaia di molecole obiettivo nella cellula obiettivo. Tali meccanismi di amplificazione richiedono meccanismi di ripristino per
ricondurre il sistema allo stato di riposo quando la stimolazione cessa. 
\subsection{La desensibilazione di GPCR dipende dalla forsforilazione dei recettori}
Un importante classe di adattamento dipende dall'alterazione della quantit\`a o della condizione dei recettori. Per GPCR ci sono tre modi di adattamento: sequestramento dei recettori, 
dove sono temporaneamente mossi all'interno della cellula, nella down regulation sono distrutti dai lisosomi e nell'attivazione si alterano in modo che non possano pi\`u interagire 
con le proteine G. In ogni caso la desnsibilazione dipende dalla loro fosforilazione da parte di PKA, PKC o una famiglia delle GPCR chinasi GRK che fosforilano multiple serine e 
treonine su una GPCR solo dopo che il legame con il ligando ha attivato il recettore che allostericamente attiva GRK. In alcuni casi il recettore \`e defosforilato e riciclato 
alla membrana plasmatica, in altri \`e ubiquitilato, endocitato e degradato nei lisosomi. L'endocitosi non causa una terminazione della segnalazion in alcuni casi. 
\section{Segnalazione attraverso recettori accoppiati da enzimi}
I recettori accoppiati a enzimi sono proteine transmembrana con il sito di legame sulla superficie esterna della membrana plasmatica e il loro dominio citosilico ha un'attivit\`a 
enzimatica intrinseca o si associa direttamente con un enzima. Ogni subunit\`a di tali recettori ha un segmento transmembrana. 
\subsection{I recettori tirosina chinasi (RTK) attivati si autofosforilano}
Molti segnali extracellulari agiscono attraverso il recettore tirosina chinasi RTK che includono molte proteine della superficie e secrete che controllano il comportamento negli animali
adulti e in via di sviluppo. Ci sono circa $60$ RTK umane classificate in $20$ sottofamiglie strutturali dedicate a una famiglia di ligandi complementari. In tutti i casi il legame
della proteine di segnale attiva il dominio ctosilico chinasico che porta a una fosforilazione della tirosina sulla parte citosilica del recettore creando siti di attracco di
fosfotirosina per varie proteine intracellulari che inoltrano il segnale. Per la maggior parte delle RTK il legame causa la dimerizzazione del recettore che unisce i domini chinasici in
modo che si fosforilino tra di loro su tirosine specifiche e promuovendo cambi conformazionali che attivano entrambi i domini chinasici. In altri casi come per EGF la chinasi
\`e attivata da cambi conformazionali portati dalle interazioni tra i due domini chinasici esterni al sito attivo. 
\subsection{Le tirosine fosforilate sugli RTK servono come siti di attracco per proteine di segnalazione intracellulari}
I domini chinasici attivati fosforilano siti addizionali nella parte citosilica in regioni disordinate fuori da essi che crear siti di attracco ad alta affinit\`a per proteine di segnale
intracellulare che si lega a un particolare sito fosforilato sul recettore attivato in quanto riconosce, oltre alla fosfotirosina anche le caratteristiche della catena intorno. Una
volta legata a un'RTK una proteina di segnale si pu\`o fosforilare sulle tirosine ed essere attivata, ma in molti casi il legame \`e sufficiente per l'attivazione inducendo un
cambio conformazionale o portandola vicino alla proteina successiva nel cammino di segnalazione. La fosforilazione del recettore serve come interruttore per causare l'assemblaggio di 
un complesso di segnalazione intracellulare che inoltra il segnale. 
\subsection{Le proteine con domini SH2 si legano a tirosine fosforilate}
Una grande variet\`a di proteine si possono legare alle fosfotirosine e aiutano a inoltrare il segnale attraverso catene di interazioni proteina-proteina mediate da domini di interazione
modulari, alcune di esse sono enzime come le fosfolipasi C-$\gamma$ (PLC$\gamma$) che attivano i cammini a inositolo. Attraverso questo cammino si aumenta il livello citosilico di 
\ce{Ca^{2+}} e attivano PKC. Un altro enzima \`e la Src tirosina chinasi che fosforila altre proteine di segnale sulle tirosine. Un altro \`e la fosfoinositide 3-chinasi (PI 3-chinasi)
che fosforila i lipidi che servono come siti di legame per attrarre varie proteine di segnalazione alla membrana plasmatica. Le proteine di segnale intracellulare presentano domini
di legame per le fosfotirosine che possono essere domini SH2 (per le regioni di omologia Src) o domini PTB (legame con fosfotirosina). Riconoscendo tirosine fosforilate specifiche 
le deboli interazioni di questi domini permettono alle proteine di legarsi a RTK attivati e altre proteine di segnale transientemente fosforilate sulle tirosine. CI sono anche altri 
domini di interazione he le permettono di interagire con altre proteine come SH3 che lega a motivi ricchi di prolina. Alcune proteine che si legano a RTK attraverso SH2 diminuiscono il
processo di segnalazione fornendo feedback negativo come la proteina c-Cbl che catalizza l'ubiquitilazione dei recettori per la loro endocitosi e degradazione. Se le RTK sono
endocitate con il segnale ancora legato continuano a segnalare. Alcune proteine di segnale sono composte quasi interamente di domini SH2 e SH3 e funzionano come adattatori per accoppiare
proteine fosforilate alla tirosina a altre proteine senza il proprio dominio SH2. 
\subsection{La GTPasi Ras media la segnalazione per la maggior parte delle RTK}
La superfamiglia Ras consiste di varie famiglie di GTPasi monomeriche ma solo le famiglie Ras e Rho inoltrano il segnale dai recettori della superficie cellulare interagendo con 
diverse proteine di segnale intracellulare possono diffondere coordinatamente il segnale lungo diversi cammini a valle agendo come hub di segnalazione. Ci sono tre proteine Ras 
principali negli umani: H-, K- e N-Ras con funzioni diverse ma lo stesso funzionamento. Ras contiene dei gruppi lipidici che la ancorano alla faccia citoplasmatica della membrana dove
inoltra segnali al nucleo per stimolare la proliferazione o differenziazione della cellula. La Ras funziona come interruttore molecolare in base al fatto che sia legata a GTP o GDP, due
classi di proteine di segnale possono regolare la sua attivit\`a influenzando la transizione tra lo stato attivo (Ras-GEF) e quello inattivo (Ras-GAP). Le RTK attivano Ras attivando 
Ras-GEF o inibendo Ras-GAP, se le seconde si legano direttamente a RTK attivate attraverso il loro dominio SH2, le Ras vengono attivate dal legame indiretto del recettore a un Ras-GEF. 
Il GEF particolare determina la membrana in cui \`e attivata la GTPasi e quale proteina a valle attiva. 
\subsection{Ras attiva una molecola di segnale MAP chinasi}
Fosfatasi specifiche alla tirosina invertono velocemente la fosforilazione delle RTK e le Ras-GAP inducono la disattivazione delle Ras. Per stimolare eventi duraturi questi eventi 
devono pertanto essere convertiti in altri pi\`u duraturi attraverso meccaismi come il modulo di chinasi attivata dal mitogeno (modulo chinasi MAP) con tre componenti che formano una
molecola di segnale utilizzata in diversi contesti. Le tre componenti sono chinasi: sequenzialmente sono MAP chinasi chinasi chinasi (MAPKKK) che viene attivata dal Ras, MAP chinasi
chinasi (MAPKK) e MAP chinasi (MAPK). Nel cammino di segnalazione dei mammiferi Ras-MAP-chinasi sono dette Raf, Mek e Erk. Una volta attivata la MAP chinasi inoltra il segnale 
fosforilando altre proteine nella cellula come regolatori di trascrizione e altre chinasi. Erk entra nel nucleo e fosforila componenti di un complesso di regolazione della trascrizione
che attiva la trascrizione di un insieme di geni immediati che codificano regolatori di trascrizioni che attivano altri geni. Tra questi geni ce ne sono alcuni che stimolano la 
proliferazione della cellula. I segnali extracellulari attivano le MAP chinasi transientemente e il tempo in cui rimane attiva influenza la risposta. Molti fattori influenzano la durata
e altre caratteristiche della risposta di segnalazione come circuiti di feedback positivo e negavo che si possono combinare. 
\subsection{Proteine di impalcatura aiutano a prevenire la comunicazione tra moduli MAP chinasi paralleli}
Alcuni moduli di MAP chinasi possono usare le stesse chinasi e pertanto si deve evitare la loro comunicazione. QUesto si ottiene attraverso proteine di impalcatura che legano le 
chinasi in complessi in modo da garantire la specificit\`a. Nei mammiferi possono operare almeno $5$ moduli paralleli e se l'impalcatura fornisce precisione riduce la capacit\`a 
di amplificazione e la diffusione del segnale ad altre parti della cellula. 
\subsection{La famiglia di GTPasi Rho accoppia funzionalmente recettori della superficie cellulare al citoscheletro}
La famiglia Rho \`e formata da GTPasi monomeriche che regolano l'actina e i microtubuli del citoscheletro controllando l'adesione, polarit\`a, forma e motilit\`a della cellula, la 
progressione del suo ciclo, la trascrizione dei geni e il trasporto di membrana. Tre membri sono Rho, Rac e Cdc42 che hanno effetto su molte proteine a valle. Sono attivate da GEF e 
disattivate da GAP che possono avere diversi gradi di specificit\`a. Sono spesso legate a inibitori di dissociazione del nucleotide guanina nel citosol che impedisce la loro interazione
con i GEF alla membrana plasmatica. La segnalazione attraverso epifrine fornisce un esempio di come RTK possono attivare una GTPasi Rho: questa si lega e attiva membri della famiglia
di Eph RTK che lo attiva. La risposta dipende dall'efexina, associata alla coda citosilica del recettore Eph. Quando viene attivato il recettore attiva una tirosina chinasi che fosforila
l'efexina su una tirosina aumentando l'abilit\`a di essa di attivare la proteina RhoA che attivata regola varie proteine obiettivo come quelle che controllano l'actina. 
\subsection{La PI 3-chinasi produce siti di attracco per lipidi nella membrana plasmatica}
Una delle proteine che si lega alla coda degli RTK \`e il fosfoinositide 3-chinasi (PI 3-chinasi) che fosforila inositoli e gioca un ruolo centrale nella promozione della sopravvivenza
e crescita della cellula. Il fosfatidilinositolo pu\`o subire fosforilazione reversibile in vari siti sul gruppo di testa generando vari fosfoinositidi. Quando attivata la PI 3-chinasi
catalizza la fosforilazione alla posizione 3 dell'anello inositolico generando diversi fosfoinositidi come PI(3, 4, 5)$P_3$ centrale come sito di legame per varie molecole di segnale che
si assemblano in complessi che inoltrano segnali nella cellula. PI(3, 4, 5)$P_3$ rimane nella membrana plasmatica fino a che specifici fosfoinositide fosfatasi lo defosforilano come 
la PTEN fosfatasi. Le PI 3-chinasi attivate da RTK e GPCR appartengono alla classe I e sono eterodimeri composti da una subunit\`a comune catalitica e diverse subunit\`a regolatorie. 
RTK attivano la classe Ia, in qui le subunit\`a regolatorie sono proteine adattatrici che legano due fosfotirosine in RTK attivati attraverso i suoi due domini SH2, mentre GPCR attivano
la classe di Ib PI 3-chinasi con subunit\`a regolatorie che legando al complesso $\beta\gamma$ sulle proteine trimeriche G attivate. Le proteine di segnale intracellulare si legano a 
PI(3, 4, 5)$P_3$ attraverso il dominio di omologia a pleckstrina (PH) che funziona come dominio di interazione proteina-proteina.
\subsection{Il cammino di segnalazione PI-3-chinas-Akt stimola la sopravvivenza e crescita delle cellule animali}
Segnali extracellulari sono richiesti alle cellule animali per crescere, dividersi e sopravvivere. Membri della famiglia dei fattori di crescita simili a insulina (IGF) stimolano 
molti di pi di cellule legandosi a RTK specifici che attivano PI 3-chinasi per produrre PI(3, 4, 5)$P_3$ che recluta due proteine chinasi alla membrana plasmatica attraverso i 
loro domini PH: Akt (proteina chinasi B o PKB) e chinasi proteica 1 fosfoinositide dipendente (PKK1) che porta all'attivazione di Akt. Una volta attivato fosforila varie proteine nella
membrana plasmatica, nel citosol e nel nucleo. Ha principalmente un ruolo inibitorio. Il controllo della crescita cellulare da parte di questo cammino dipende da una grande proteina
chinasi detta TOR, mTOR nei mammiferi, dove esiste in due complessi funzionalmente distinti: complesso mTOR 1 che contiene la proteina raptor e sensitivo  alla rapamicina, stimola
la crescita cellulare promuovendo la produzione dei ribosomi e la sintesi proteica, inibendo la degradazione delle stesse, oltre a stimolare il recupero di nutrienti. Il complesso mTOR 2
contiene  la proteina rictor e aiuta ad attivare Akt, regola il citoscheletro actinico. 
\subsection{RTK e GPCR attivano cammini di segnalazione sovrapposti}
RTK e GPCR possono attivare gli stessi cammini di segnalazione o comunque cammini che a un certo punto possono convergere sule stesse proteine. Le interazioni tra i cammini 
permettono diverse molecole di segnale di modulare e coordinare gli effetti reciproci. 
\subsection{Alcuni recettori accoppiati ad enzimi si associano con la chinasi tirosina citoplasmatica}
Molti recettori sulla superficie cellulare dipendono dalla fosforilazione della tirosina per la loro attivit\`a ma non possiedono un dominio di tirosina chinasi e pertanto agiscono
attraverso tirosine chinasi citosiliche, associate con il recettore e che fosforilano diverse proteine obiettivo come i recettori quando si legano al ligando. Questi recettori funzionano
pertanto in maniera analoga alle RTK tranne per il fatto che il dominio chinasi \`e codificato da un gene diverso. Appartengono a questa classe i recettori per gli antigeni e per le
interleuchine sui limfociti, integrine e recettori per citochine e ormoni. Sono dimeri o portati in una forma dimerica quando si legano al ligando. Alcuni di questi recettori dipendono
da membri della famiglia Src che contengono domini SH2 e SH3 sul lato citoplasmatico della membrana, tenuti l\`i da interazioni con le proteine recettrici e catene lipidiche attaccate
covalentemente. Diversi membri sono associati con diversi recettori e fosforilano insiemi sovrapposti ma distinti di proteine. Un tipo di tirosina chinasi citoplasmatica si associa con 
le integrine, il recettore che la cellula usa per legarsi alla matrice extracellulare il cui legame causa l'attivazione di cammini di segnalazione che influenzano il comportamento della
cellula. Quando si raggruppa nei siti del contatto con la matrice aiutano a causare l'assemblaggio di giunzioni dette adesioni focali in cui \`e presente la chinasi di adesione focale
(FAK) che si lega alla coda citosilica di una delle subunit\`a dell'integrina dove si fosforilano tra di loro creando siti di legame dove la chinasi Src si pu\`o legare. In questo
modo i due segnali di tirosina chinasi arrivano alla cellula che ha aderito al substrato. Tali segnali sono inoltrati nella cellula da recettori citochina.
\subsection{I recettori citochina attivano il cammino di segnalazione JAK-STAT}
La famiglia di recettori citochina include recettori per molti mediatori locali e per ormoni. Tali recettori sono associati stabilmente con tirosine chinasi citoplasmatiche dette 
giano chinasi (JAK) che fosforilano e attivano il regolatore di trascrizione STAT (trasduttori di segnale e attivatori di trascrizione) che si trovano nel citosol e sono anche detti
regolatori di trascrizione latente in quanto migrano nel nucleo e regolano la trascrizione genica solo dopo la loro attivazione. Il cammino JAK-STAT \`e uno dei pi\`u diretti per la
regolazione genica. I recettori citochina sono dimeri o trimeri associati con fino a due tra JAK1-2-3 e Tyk2. Il legame della citochina altera la conformazione avvicinando i due 
JAK in modo che si fosforilino tra di loro e aumentando l'attivit\`a dei loro domini tirosina chinasi. I JAK poi fosforila le tirosine sulle code citoplasmatiche deli recettori citochina
creando siti di legame fosfotirosina per STAT e proteine adattatrici e accoppiare i recettori ai cammini di segnalazione Ras-MAP-chinasi. Ogni STAT ha un dominio SH2 che media il 
legame di STAT a una fosfotirosina sul recettore attivato (una volta legato il JAK fosforila le sue tirosine causandone la dissociazione) e sullo STAT rilasciato media il suo legame
a una fosfotirosina a un'altra STAT formando un omo o eterodimero. Il dimero si trasloca poi nel nucleo dove si lega a una sequenza cis-regolatoria con specificit\`a in combinazione
con altre proteine regolatorie della trascrizione. La risposta mediata da questo cammino \`e regolata da feedback negativoL i dimeri STAT possono anche attivare geni che codificano 
proteine inibitorie che aiutano a terminare la risposta, ma la terminazione completa richiede la defosforilaizone delle loro fosfotirosine.
\subsection{La proteina tirosina fosfatasi inverte la fosforilazione delle tirosine}
In tutti i cammini di segnalazione che usano la fosforilazione della tirosina la fosforilazione \`e invertita da proteine tirosina fosfatasi che sono presenti in grande quantit\`a. 
Le tirosine fosfatasi si trovano in forme transmembrana e citosiliche e hanno alta specificit\`a per il substrato in e fanno in modo che le fosforilazioni della tirosina abbiano vita
brevee che il livello generale di fosforilazione della tirosina nella cellula rimanga basso. Sono in ogni caso regolate ed agiscono solo quando necessario. 
\subsection{Le proteine di segnale della superfamiglia TGF$\beta$ agiscono attraverso recettori serina/treonina chinasi e Smads}
La superfamiglia di fattori di crescita trasformanti-$\beta$ (TGF$\beta$) consiste di proteine dimeriche secrete strutturalmente simili. Agiscono come ormoni o come mediatori locali per
regolare un gran numero di funzioni biologiche. Consiste della famiglia di TGF$\beta$/activine e di proteine morfogenetiche delle ossa (BMP). Tutte queste proteine agiscono attraverso
recettori accoppiati da enzimi che sono proteine transmembrana a singolo passaggio con un dominio di serina/treonina chinasi nel lato citosilico. Ce ne sono due classi: tipo I e tipo II,
strutturalmente simili omodimeri. Ogni membro della superfamiglia si lega a una combinazione caratteristica di dimeri recettori di tipo I e tipo II e uniscono i domini chinasi insieme
in modo che il recettore di tipo II pu\`o fosforilare e attivare il tipo I, formando un complesso tetramerico attivo. Il recettore tipo-I attivato si lega e fosforila un regolatore
di trascrizione latente della famiglia Smad. I recettori TGF$\beta$/activina fosforilano Smad2 o Smad3, mentre i BMP fosforilano Smad1, Smad5 o Smad8. Una volta che queste R-Smad sono
fosforilate si dissociano dal recettore e si legano a Smad4 (co-Smad) e il complesso risultante si trasloca nel nucleo dove si associa con altri regolatori di trascrizione. I recettori e
il loro legame sono endocitati da due diverse strade, una che aumenta l'attivazione e un'altra che la inibisce. La strada di attivazione dipende da vescicole incapsulate da clatrina e
porta agli endosomi giovani dove le Smad vengono attivate. Una proteina ancora SATA (ancora Smad per l'attivazione del recettore) \`e concentrata negli endosomi giovani e si lega
ai recettori TGF$\beta$ attivati e Smad aumentando l'efficienza della fosforilazione mediata da Smad. La strada di disattivazione dipende da calveole e porta all'ubiquitilazione del
recettore e alla sua degradazione nei cromosomi. Durante la risposta al segnale Smad si muove tra il citoplasma e il nucleo: nel secondo sono defosforilate ed esportate nel 
citoplasma dove possono essere rifosforilate. In questo modo l'effetto sui geni riflette la concentrazione del segnale extracellulare e il tempo che il segnale agisce sui recettori. 
Come in altri sistemi il cammino \`e regolato da feedback negativo: tra i geni attivati dai complessi Smad si trovano quelli che codificano Smad inibitorie: Smad6 o Smad7 che si 
legano alle code citosiliche dei recettori attivati inibendo la sua capacit\`a di segnalazione competendo con le R-Smad per i siti di legame sul recettore diminuendone la sua 
fosforilazione, recluta una ligasi ubiquitina Smurf che ubiquitila il recettore portando alla sua internalizzazione e  degradazione e recluta una fosfatasi che defosforila e inattiva
il recettore. Le Smad inibitorie si legano anche al co-Smad inibendolo impedendo il suo legame con R-Smad o promuovendo la sua ubiquitilazione. 
\section{Strade di segnalazione alternative nella regolazione genica}
\begin{Huge}
	NON SO SE \`E DA FARE
\end{Huge}
\section{Segnalazione nelle piante}
\begin{Huge}
	NON SO SE \`E DA FARE
\end{Huge}
