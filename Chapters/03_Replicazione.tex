\chapter{Replicazione e riparazione del DNA}

\section{Modello di replicazione}
La duplicazione delle cellule richiede la duplicazione del DNA, in modo che entrambe le cellule figlie possano ricevere una copia uguale del materiale genetico.
La natura a doppia elica delle molecole di DNA permette la creazione di tre ipotesi riguardo il modello di replicazione.
\begin{multicols}{2}
	\begin{itemize}
		\item Replicazione dispersiva: la molecola di DNA dopo la replicazione \`e un mosaico di parti di nuova sintesi e parti di origine parentale.
		\item Replicazione semiconservativa: la molecola di DNA dopo la replicazione \`e formata da un filamento di nuova sintesi e uno di origine parentale.
		\item Replicazione conservativa: la catena parentale si duplica in due catene in modo che in una rimangano entrambi i filamenti parentali mentre nell'altra ci siano solo filamenti di nuova sintesi.
	\end{itemize}
\end{multicols}

	\subsection{Esperimento Meselson-Stahl}
	L'esperimento Meselson-Stahl si propone di determinare il modello di replicazione.
	Viene compiuto su batteri con l'utilizzo di $2$ isotopi radioattivi dell'azoto\emph{$\ ^{15}N\ ^{14}N$}.
	L'isotopo dell'azoto viene incorporato dai batteri nelle molecole di DNA durante la loro crescita.
	Il peso diverso dei due isotopi permette di distinguere le molecole di DNA che ne contengono diversi.

		\subsubsection{Processo}
		Si fa crescere una generazione di batteri in coltura in presenza di \emph{$\ ^{15}N$}.
		Questi presentano pertanto una doppia elica pesante.
		La prima generazione viene trasferita in condizioni normali: \emph{$\ ^{14}N$}.
		Si compie un ulteriore ciclo si replicazione in \emph{$\ ^{14}N$}.
		Da questa si estraggono le molecole di DNA e le si separa secondo gradiente.
		
		\subsubsection{Risultati}
		Si nota come nella seconda generazione si trovano molecole di DNA ibride, mentre in quella successiva da $2$ molecole ibride se ne ottengono $2$ leggere e $2$ ibride.
		Questo esperimento determina che il modello di replicazione \`e semiconservativo.


\section{Panoramica}

	\subsection{Reazione di polimerizzazione}
	La replicazione del DNA richiede la polimerizzaizone di nuove catene.
	La polimerizzazione avviene attraverso l'aggiunta di un deossiribonuclotide $3P$ all'estremit\`a $3'$ di una catena polinucleotidica detta filamento primer.
	Si nota come il filamento pu\`o pertanto polimerizzare unicamente in direzione $5'$-$3'$.
	Questo processo viene guidato dall'accoppiamento tra le basi e un deossiribonucleotide trifosfato.
	Il legame causa il rilascio di pirofosfato.

		\subsubsection{Substrati}

			\paragraph{Deossiribonucleotidi trifosfato}
			Quattro deossiribonucleotidi trifosfato \emph{dNTP} possiedono tre gruppi fosforici $\alpha$, $\beta$ e $\gamma$ attaccati al $2'$-deossiribosio tramite un legame $5'$-ossidrilico.
			Vengono attaccati al filamento in sintesi con rilascio della molecola di pirofosfato.
			\begin{multicols}{4}
				\begin{itemize}
					\item \emph{dGTP}.
					\item \emph{dCTP}.
					\item \emph{dATP}.
					\item \emph{dTTP}.
				\end{itemize}
			\end{multicols}

			\paragraph{Giunzione innesco}
			La giunzione innesco \`e formata da un filamento singolo di DNA che dirige l'aggiunta dei \emph{dNTP} complementari.
			L'innesco \`e un primer a DNA o RNA molto pi\`u corto del filamento che serve come sito di inizio dell'aggiunta dei nuovi \emph{dNTP} liberi.
			Necessario in quanto presenta un \emph{$3'$-OH} libero per l'inizio della reazione.

		\subsubsection{Crescita del filamento}
		Il filamento di nuova sintesi cresce allungando l'innesco al $3'$.
		Avviene la formazione di un legame fosfodiesterico tra innesco e \emph{dNTP}.
		\emph{-OH} a $3'$ dell'innesco esegue un attacco nucleofilo al fosfato $\alpha$ del \emph{dNTP} creando il legame fosfodiesterico e causando l'uscita del pirofosfato con i fosfati $\alpha$ e $\beta$.

		\subsubsection{Energia}
		Pur essendo la reazione termodinamicamente sfavorevole, l'idrolisi del pirofosfato effettuata da una pirofosfatasi fornisce l'energia necessaria alla sua attivazione.
		Il pirofosfato rilasciato dalla reazione viene pertanto idrolizzato in due molecole di fosfato inorganico.
		In questo modo $\Delta G_{tot} < 0$.

	\subsection{Mutazioni}
	Il processo di replicazione non \`e perfetto: durante esso possono essere infatti introdotti errori.
	
		\subsubsection{Tasso di mutazione}
		Si nota come per un singolo gene per una proteina media lungo $1000nt$ accumula una mutazione ogni $10^6$ generazioni.
		Avvengono pertanto $3$ cambi nucleotidici per $10^{10}$ nucleotidi per generaizone di cellule.
		Nell'essere umano avviene $1$ mutazione ogni $10^{10}$ divisioni celluari.

		\subsubsection{Tipologie di mutazioni}

			\paragraph{Geniche}
			Le mutazioni geniche avvengono per singole basi:
			\begin{multicols}{2}
				\begin{itemize}
					\item Sinonime o di sostituzione: viene introdotto un codone diverso che codifica per lo stesso amminoacido.
					\item Di senso errato: il codone viene sostituito con un altro che codifica per un amminoacido diverso.
					\item Non senso: viene formato un codone di stop.
					\item Frameshift: avviene uno spostamento dell'ordine di lettura per inserimento o delezione.
					\item Per sequenze ripetute.
				\end{itemize}
			\end{multicols}
			
			\paragraph{Cromosomiche}
			Le mutazioni cromosomiche consistono di traslocazioni di geni tra cromosomi.

			\paragraph{Genomiche}
			Le mutazioni genomiche consistono di perdita o aggiunta di numerosi geni.

	\subsection{DNA polimerasi}
	La DNA polimerasi \`e l'enzima responsabile della sintesi della catena di DNA.
	Lavora in modo processavo in direzione $3'$-$5'$.
	\`E un oloenzima, possiede ovvero numerosi domini con funzioni diverse.
	Oltre all'attivit\`a polimerasica possiede un'attivit\`a esonucleasica $3'$-$5'$ che rimuove i nucleotidi erroneamente appaiati e una esonuclasica $5'$-$3'$.



	\subsection{Meccanismi di correzione}
	L'alta fedelt\`a del meccanismo della replicazione richiede meccanismi di correzione.

		\subsubsection{DNA polimerasi}
		La DNA polimerasi compie essa stessa una correzione prima dell'aggiunta di un nuovo nucleotide.
		Il nucleotide corretto ha infatti un'affinit\`a maggiore per la polimerasi in movimento del nucleotide non corretto.

		\subsubsection{Attivit\`a esonucleotidica}
		La DNA polimerasi possiede un'attivit\`a esonucleotidica.
		Questa viene attivata quando viene aggiunto un nucleotide scorretto: le DNA polimerasi sono selettive riguardo la catena in allungamento.
		Viene pertanto eliminato qualsiasi residuo non appaiato al terminale del primer.
		La rimozione continua all'indietro fino a che non sono stati rimossi abbastanza nucleotidi da rigenerare un \emph{-OH} terminale appaiato.

	\subsection{Sintesi sul filamento discontinuo}
	Se al filamento principale \`e necessario un solo primer all'inizio della replicazione, l'attivit\`a unidirezionale della DNA polimerasi richiede un processo diverso per il filamento in ritardo.
	Questo filamento viene diviso in frammenti di Okazaki, corte regioni di DNA che la DNA polimerasi \`e in grado di sintetizzare.
	Ognuno di questi frammenti richiede un primer, e dopo che la DNA polimerasi completa la sintesi di uno si sposta al successivo.

		\subsubsection{DNA primasi}
		Il processo di produzione del primer dipende dalla DNA primasi, che usa trifosfati ribonucleici per sintetizzare corti primer a RNA sul filamento in ritardo.
		Il primer a RNA crea legami con il filamento in ritardo generando una doppia elica ibrida.
		Il primer contiene una terminazione \emph{$3'$-OH} e pu\`o essere allungato dalla DNA polimerasi.
		
		\subsubsection{Terminazione del frammento}
		La DNA polimerasi che allunga il primer si ferma quando incontra il primer attaccato alla terminazione $5'$ del frammento precedente.
		Il frammento \`e reso continuo da un sistema di riparazione del DNA, che elimina i primer e li sostituisce con DNA.
		Infine la DNA ligasi unisce le terminazioni dei frammenti.
		La rimozione del primer avviene a carico di \emph{RNAasi H}.

	\subsection{Apertura della doppia elica}
	Essendo la doppia elica stabile in condizioni fisiologiche sono necessarie DNA elicasi per aprire la doppia elica.
	Queste proteine idrolizzano \emph{ATP} e sono legate a un singolo filamento del DNA.
	Quando incontrano una regione a doppia elica continuano a muoversi lungo il proprio filamento separandolo a $1000\frac{nt}{sec}$.
	Possono lavorare in entrambe le direzioni della polarit\`a.
	Proteine che legano filamenti singoli di DNA o che destabilizzano l'elica si legano cooperativamente per esporre singoli filamenti senza coprire le basi.
	In questo modo stabilizzano la conformazione a singolo filamento e impediscono la formazione di corte eliche a forcina nel filamento in ritardo.

	\subsection{Mantenimento di una DNA polimerasi in movimento sul DNA}
	Le DNA polimerasi tendono a sintetizzare una piccola stringa di nucleotidi prima di separarsi dallo stampo.
	Questa bassa affinit\`a con il DNA le permette di separarsi ed essere riciclata velocemente dopo la sintesi di un frammento di Okazaki.
	Rende per\`o difficile la sintesi di lunghe sequenze.

		\subsubsection{Morsetto scorrevole}
		La proteina \emph{PCNA} funziona come morsetto scorrevole: mantene la polimerasi fermamente sul DNA mentre si muove e la rilascia appena incontra una regione a doppia elica.
		Forma un grande anelli intorno al filamento.
		Una faccia si lega al retro della DNA polimerasi, mentre l'anello scorre lungo il DNA.
		L'assemblaggio richiede idrolisi di \emph{ATP} da parte di un complesso detto caricatore del morsetto che la pone su una giunzione primer.
		Sul filamento principale la DNA polimerasi rimane legata al morsetto per molto tempo, mentre su quello in ritardo si dissociano ogni volta che la polimerasi arriva sulla terminazione $5'$ di un frammento di Okazaki.

	\subsection{Modello a trombone}
	Il modello a trombone indica il processo con il quale viene replicato il DNA.
	Si nota come entrambi i filamenti sono sintetizzati simultaneamente.
	Nel modello l'elicasi si muove in direzione $5'$-$3'$.
	L'oloenzima di replicazione interagisce con l'elicasi attraverso un fattore $\tau$ che lega entrambe le polimerasi.
	Un core della DNA polimerasi $III$ sintetizza il filamento guida e l'altro quello in ritardo.
	Il ssDNA viene costantemente coperto da \emph{SSB} che ne stabilizzano la struttura.
	Periodicamente la DNA primasi si associa all'elicasi per formare il primer sul filamento discontinuo.
	Quando la polimerasi di quest ultimo completa un frammento di Okazaki viene rilasciata dalla pinza e dal DNA.
	Il filamento con un innesco diventa il bersaglio dell'attivit\`a del caricatore della pinza che riconoscie la giunzione innesco stampo.
	Questa legata dalla pinza richiama la DNA polimerasi che riparte con la sintesi del frammento di Okazaki successivo.

	\subsection{DNA topoisomerasi}
	Il movimento della forcella di replicazione lungo il DNA causa superavvolgimenti.
	Questi aumentano lo stress sulla doppia elica contrastando le elicasi.
	Devono pertanto intervenire topoisomerasi, nucleasi reversibili che rompono i legami fosfodiesterici nel backbone in modo da ridurre i superavvolgimenti e permettere la continuazione della polimerasi.

\section{Complesso di pre-replicazione}

	\subsection{Formazione}
	Il complesso di pre-replicazione si forma alle origini di replicazione.

		\subsubsection{Origine di replicazione}
		Un'origine di replicazione \emph{Ori} \`e una sequenza specifica che contiene:
		\begin{multicols}{2}
			\begin{itemize}
				\item Sito di legame per la proteina iniziatrice \emph{ORC} (origin recognition complex).
				\item Tratto ricco di $A$-$T$ per facilit\`a di separazione.
				\item Sito di legame per proteine che facilitano il legame con \emph{ORC}.
			\end{itemize}
		\end{multicols}

		\subsubsection{Processo di formazione}
		\begin{multicols}{2}
			\begin{itemize}
				\item \emph{ORC} riconosce e si lega a \emph{Ori}.
				\item \emph{ORC} recluta \emph{Cdt1} e \emph{Cdc6}.
				\item \emph{Cdt1} e \emph{Cdc6} reclutano le elicasi.
				\item Le elicasi \emph{MCM2-7} sono caricate sul DNA vicino a \emph{ORC} durante la fase $G_1$.
				\item Il complesso \`e ora assemblato sul DNA e rimane inattivo.
			\end{itemize}
		\end{multicols}
		
	\subsection{Attivazione}
	Il complesso di pre-replicazione viene attivato quando la cellula entra nella fase $S$.

		\subsubsection{Chinasi ciclina dipendente}
		Le chinasi ciclina dipendente \emph{CDK} sono regolate dal ciclo cellulare: innescano la replicazione e impediscono l'assemblaggio simultaneo di nuovi complessi replicativi fino alla fase $M$.
		Fosforilano anche \emph{ORC} impedendo che recluti altre elicasi.

		\subsubsection{Processo di attivazione}
		\begin{multicols}{2}
			\begin{itemize}
				\item \emph{Cdk} e \emph{Ddk} fosforilano \emph{Cdt1} e \emph{Cdc6} promuovendo distacco e proteolisi.
				\item La fosforilazione di \emph{MCM2-7} le attiva e causa il reclutamento di altre proteine.
				\item Vengono reclutate le DNA polimerasi $\delta$, $\epsilon$, $\alpha$ e la DNA primasi.
				\item Viene reclutata la pinza e il suo caricatore \emph{PCNA} e \emph{RF-C}.
				\item Avviene il distacco della primasi.
				\item La DNA polimerasi $\delta$ inizia la replicazione sul filamento discontinuo.
				\item La DNA polimerasi $\epsilon$ inizia la replicazione sul filamento guida.
				\item Due forcelle di replicazione vengono attivate contemporaneamente in direzioni opposte.
			\end{itemize}
		\end{multicols}

\section{Replisoma}
Il replisoma \`e il complesso multi-enzimatico responsabile della replicazione del DNA.

	\subsection{Composizione}

		\subsubsection{\emph{MCM-7}}
		\emph{MCM-7} o minicromosome maintenance \`e un complesso proteico ad attivit\`a elicasica dipendente da \emph{ATP}.
		Il suo caricamento \`e efficiente solo a seguito del legame sequenziale con \emph{Cdt1} e \emph{Cdc6}.
		Apre la doppia elica formando la bolla di trascrizione.
		Viene successivamente sostituito da un'elicasi pi\`u processiva.

		\subsubsection{DNA ligasi}
		La DNA ligasi unisce le estremit\`a $3'$ del nuovo filamento a quella $5'$ del precedente.

		\subsubsection{Elicasi}
		Le elicasi sono enzimi che aprono la doppia elica utilizzando \emph{ATP}.
		Oscillano cambiando conformaizone in modo ciclico.
		Lavorano in entrambe le direzioni.

		\subsubsection{Single strand binding protein}
		Le proteine \emph{SSB} si legano al ssDNA stabilizzandolo e impedendo la formazione di hairpins e mantenendo la catena lineare.
		Si staccano all'avvicinamento della DNA polimerasi.

		\subsubsection{DNA primasi}
		La DNA primasi sintetizza brevi primers sul filamento ritardato fornendo l'innesco alla DNA polimerasi.

		\subsubsection{DNA polimerasi}
		La DNA polimerasi \`e un enzima che sintetizza il DNA in direzione $5'$-$3'$.
		Possono essere coinvolte anche nei processi di riparazione del DNA e si distinguono in:
		\begin{multicols}{2}
			\begin{itemize}
				\item $\alpha$: fa partire la sintesi.
				\item $\delta$: pi\`u attiva nel filamento principale.
				\item $\epsilon$: pi\`u attiva nel filamento discontinuo.
				\item $\gamma$: DNA polimerasi mitocondriale.
			\end{itemize}
		\end{multicols}

			\paragraph{Tipologie}

				\subsection{Tipo $\mathbf{I}$}
				Le DNA polimerasi di tipo $I$ hanno una struttura a mano capace di riconoscere e correggere errori.
				Le correzioni sono effettuate attraverso:
				\begin{multicols}{2}
					\begin{itemize}
						\item Attivit\`a esonucleasica: degrada dall'esterno con \emph{-OH} libero.
						\item Attivit\`a endonucleasica: taglia il nucleotide dall'interno.
					\end{itemize}
				\end{multicols}
				L'attivit\`a \`e processiva in direzione $5'$-$3'$ nel filamento leader, mentre non processiva nel filamento discontinuo.
				Nel secondo caso necessita di primer a RNA allungati in frammenti di Okazaki e riattaccati grazie a DNA ligasi.
				L'attivit\`a cataliticha le permette di formare il legame fosfodiesterico.

		\subsubsection{Esonucleasi}
		Le esonucleasi sono gli enzimi deputati alla rimozione dei primer.
		
		\subsubsection{Complesso $\mathbf{\beta}$ o pinza scorrevole}
		La pinza scorrevole \`e formata da $2$ subunit\`a a ciambella che si posizionano sul DNA.
		Il complesso $y$ la carica sul DNA.
		Permette alla DNA polimerasi di rimanere attaccata al singolo filamento di DNA.
		Si stacca nelle regioni a doppia elica.

		\subsubsection{\emph{CDK}}
		Le \emph{CDK} hanno un'attivit\`a di fosforilazione che segue le cicline.

		\subsubsection{\emph{TBP/TUS}}
		\emph{TBP/TUS} o ter binding proteins rimuovono gli istoni da \emph{OriC} e sono riconosciute da sequenze \emph{TER}.

		\subsubsection{Complesso di pre-replicazione}
		\emph{ORC} permette il recltuamento del repliosoma.

		\subsubsection{DNA topoisomerasi}
		Le DNA topoisomerasi permettono la risoluzione dei superavvolgimenti.




\section{Processo di replicazione}

	\subsection{Inizio della replicazione}
	La replicazione inizia da regioni specifiche ricche di $A$ e $T$ in quanto questi nucleotidi formano meno legami a idrogeno e sono pi\`u facilmente separabili.
	Attraggono le proteine iniziatrici.

		\subsubsection{Procarioti}
		Il genoma dei procarioti \`e tipicamente contenuto in una molecola di DNA circolare.
		La replicazione inizia ad un singolo sito e le due forcelle procedono in direzioni opposte fino a che si incontrano.
		
			\paragraph{Regolazione}
			L'inizio \`e l'unico punto in cui la replicazione viene regolata.
			Il processo inizia quando proteine legate a \emph{ATP} si legano in copie multiple in siti specifici del DNA avvolgendolo.
			Si forma un complesso che destabilizza la doppia elica vicina.
			Questo attrae due DNA elicasi legate a un caricatore che le mantiene inattive fino al caricamento giusto.
			L'elicasi una volta caricata svolge il DNA esportando i filamenti in modo che la DNA primasi sintetizzi il primer che porta all'assemblaggio delle proteine  necessarie alle creazione delle due forcelle di replicazione.
			La proteina iniziatrice viene disattivata e l'origine di replicazione subisce un periodo refrattario causato da un ritardo nella metilazione dei nuovi nucleotidi.

			\paragraph{\emph{Ori}}
			\emph{Ori} \`e formata da $4$ ripetizioni di $9$ nucleotidi e $3$ ripetizioni di $13b[$ a monte di esse.
			Viene riconosciuta dalle \emph{DNA A} che si lega alle ripetizioni facilitando la denaturazione e recluta le elicasi \emph{DNA C-B}.

		\subsubsection{Eucarioti}
		Il genoma degli eucarioti si trova tipicamente in cromosomi lineari.
		La replicazione inizia a diversi \emph{Ori} e la bolla di replicazione si espande fino alla terminazione del cromosoma o fino a che si incontra un'altra forcella.
		Nei lieviti le \emph{Ori} prendono il nome di \emph{ARS} (autonomous replicating sequence).

		\subsubsection{Identificazione della regione contente un \emph{Ori}}
		Per identificare la regione che contiene l'origine di replicazione si sequenzia il DNA del lievito mediante enzimi di restrizione, lo si introduce nel DNA plasmidico contenente un gene per la sintesi di istidina.
		Il plasmide che contiene un \emph{ARS} sar\`a in grado di diffondersi e creer\`a coltura in terreno privo di istidina.
		Si restringe la regione aggiunta al plasmide in modo da identificare precisamente la sequenza.

	\subsection{Velocit\`a di replicazione}
	Le forcelle di replicazione eucariotiche si muovono di circa $50$ nucleotidi al secondo a causa dello stato cromatinico.

	\subsection{Organizzazione temporale}
	Se i batteri replicano il DNA in maniera continua negli eucarioti avviene durante la fase $S$, che in una cellula mammifera dura $8$ ore.
	Al termine della fase $S$ ogni cromosoma \`e stato completamente replicato in due coppie che rimangono unite al centromero fino alla fase $M$.

		\subsubsection{Temporizzazione delle \emph{Ori}}
		Negli eucarioti le origini di replicazione sono attivate in cluster di $50$ adiacenti.
		L'ordine di attivazione dipende dalla struttura cromatinica.

	\subsection{Assemblaggio dei nucleosomi}
	A causa della grande quantit\`a di proteine istoniche necessarie per produrre i nuovi nucleosomi la maggior parte degli eucarioti possiede copie multiple del gene di ciascun istone.
	La duplicazione dei cromosomi richiede la sintesi e l'assemblaggio di nuove proteine cromosomiche sul DNA in coda alla forcella.
	Sono sintetizzati durante la fase $S$.
	Un meccanismo di feedback regola il livello degli istoni liberi in modo che la loro quantit\`a rimanga costante e attiva i loro geni quando vengono inclusi in massa nel DNA durante la fase $S$.
	Quando la forcella di replicazione passa attraverso nucleosomi un complesso di rimodellamento della cromatina destabilizza il nucleosoma permettendo la replicazione.
	Mentre la forcella passa gli istoni sono spostati: l'ottamero viene rotto in un tetramero \emph{2(H3-H4)} che rimane associato con il DNA e viene distribuito tra uno dei due figli e in due dimeri \emph{H2A-H2B} vengono completamente rilasciati.
	\emph{2(H3-H4)} di nuova sintesi sono usati per riempire i buchi e i dimeri sono aggiunti a caso per completare i nucoleosmi.
	La lunghezza dei frammenti di Okazaki \`e determinata dal punto in cui la DNA polimerasi \`e bloccata dal nucleosoma.

	\subsection{Terminazione della replicazione}

		\subsubsection{Procarioti}
		Il DNA circolare dei procarioti risolve il problema della replicazione delle terminazioni: l'incontro tra le due forcelle di replicazione causa la terminazione della replicazione.

		\subsubsection{Eucarioti}
		Alle terminazioni il meccanismo di replicazione del filamento in ritardo incontra problemi: il primer a RNA finale non pu\`o essere sostituito da DNA in quanto non trova un \emph{$3'$-OH} disponibile.
		Il problema viene risolto attraverso i telomeri.

			\paragraph{Telomeri}
			I telomeri sono sequenze specializzate contenenti ripetizioni di sequenze corte $GGGTTA$ riconosciute dalla telomerasi che le rifornisce ogni volta che la cellula si divide.

				\subparagraph{Telomerasi}
				La telomerasi riconosce la fine di un telomero e la allunga nella ripetizione usando uno stampo a RNA presente nell'enzima.
				Lo stampo viene utilizzato per sintetizzare nuove copie della ripetizione.
				Dopo l'estensione del filo genitore la replicazione del filamento discontinuo pu\`o essere completata dalla DNA poliemrasi standard che usa le estensioni per sintetizzare il filamento complementare.

				\subparagraph{Funzionamento della telomerasi}
				Il terminale $3'$ del telomero viene legato da \emph{RNA TERC} (teloerase RNA component) creando una giunzione innesco-stampo.
				La giunzione viene usata dalla telomerasi \emph{TERT} per allungare l'estremit\`a di $6$ nucleotidi.
				\emph{RNA TERC} dopo essere usato come stampo si stabilizza sulla nuova estremit\`a saltando di $6$ nucleotidi e ripetendo il processo pi\`u volte.
				Terminata l'azione della telomerasi il filamento funziona da stampo per la sintesi del secondo filamento.
				La rimozione dell'ultimo innesco crea un terminale $3'$ sporgente rispetto al $5'$.

				\subparagraph{T-loop}
				Le proteine legate al telomero oltre a regolare la telomerasi hanno un rulo protettivo dell'estremit\`a $3'$: impediscono che venga riconosciuta come rottura del DNA.
				I telomeri formano un'ansa data dall'estremit\`a a filamento singolo che invade il doppio filamento del telomero stesso.

				\subparagraph{Lunghezza dei telomeri}
				Essendo i processi di regolazione della sequenza telomerica bilanciati approssimativamente una fine cromosomica contiene un numero variabile di ripetizioni telomeriche.
				Molte cellule hanno un numero di meccanismo che mantengono il numero delle ripetizioni in un intervallo, come \emph{POT, TRF1,2} che contano le sequenze aggiunte e bloccano l'enzima.
				Si nota come nella maggior parte delle divisioni cellulari le ripetizioni telomeriche sono erose con il tempo.
				Dopo molte generazioni le cellule discendenti avranno cromosomi senza funzione telomerica e le cellule vanno incontro a senescenza.
				Il numero di divisioni massime viene detto limite di Hayflick.
				Non permettere la degradazione dell'attivit\`a della telomerasi causa tumorigenesi.

				\subparagraph{Discheratosi congenita}
				La discheratosi congenita \`e una malattia genetica causata da mutazioni del gene discherina, una chinasi coinvolta nei processi di modifica di RNA ribosomiali che interagisce con la parte ad RNA della telomerasi influenzandone la stabilit\`a.
				Questa causa porta a una perdita della funzione della telomerasi e causa:
				\begin{multicols}{2}
					\begin{itemize}
						\item Pigemntazione anomala della pelle.
						\item Distrofia unguela (corrugamento, distruzione e perdita delle unghie).
						\item Ingrigimento precoce.
						\item Cirrosi epatica.
						\item Disordini intestinali.
					\end{itemize}
				\end{multicols}

\section{Riparazione del DNA}

	\subsection{Motivi biologici della riparazione}
	Nonostante il DNA sia stabile \`e suscettibile a cambi spontanei che porterebbero a mutazioni se lasciate non riparate.
	Le basi del DNA possono essere danneggiate da una collisione con metaboliti reattivi o da radiazioni ultraviolette.
	Queste modifiche se accumulate portano alla cancellazione di basi o a sostituzioni durante la replicazione portando a conseguenze letali.
	Si nota come la struttura a doppia elica \`e adatta alla riparazione in quanto porta due copie dell'informazione genetica: danni ad un filamento possono essere riparati usando l'altro come stampo.

	\subsection{Processi di riparazione}
	Le cellule possiedono diversi modi di riparare il DNA.
	Differiscono per come il danno viene eliminato.
	Questo processo pu\`o essere accoppiato alla trascrizione: la RNA polimerasi infatti stalla ad errori, causando un reclutamento degli enzimi necessari alla riparazione.

		\subsubsection{Riparazione tramite asportazione della base}
		Il cammino di riparazione tramite asportazione della base le DNA glicolasi riconoscono un tipo di base nel DNA e catalizzano una rimozione idrolitica.
		Una base alterata viene riconosciuta attraverso un flip-out del nucleotide.
		Il buco creato dalla DNA glicolasi viene riconosciuto da \emph{AP endonucleasi} che taglia il backbone riparando lo zucchero.

		\subsubsection{Asportazione del nucleotide}
		Il cammino di riparazione tramite asportazione del nucleotide pu\`o riparare danni causati da grandi cambi nella struttura: questa viene scansionata da un complesso multienzima che cerca distorsioni nella doppia elica.
		La DNA elicasi rimuove il filamento che contiene la lesione.
		Il gap prodotto \`e riparto da DNA polimerasi e ligasi.

	\subsection{DNA polimerasi specializzate}
	In caso di danni pesanti la replicazione del DNA viene fermata e si utilizzano DNA polimerasi diverse, versatili ma meno precise dette rtranslesion polimerasi per replicare attraverso il danno.
	Alcune possono riconoscere un danno specifico e aggiungere i nucleotidi necessari, mentre altre fanno congetture.
	Non possiedono proofreading e sono meno discriminanti nella scelta del nucleotide.

	\subsection{Rotture a doppio filamento}
	Le rotture a doppio filamento possono essere riparate attraverso unione delle terminazioni non omologa.
	In questo meccanismo le terminazioni rotte sono riunite attraverso DNA ligation con perdita dei nucleotidi nel sito dell'unione.
	Porta a mutazioni.

	\subsection{Effetti del danno al DNA}
	L'attivit\`a dei processi di riparazione blocca il ciclo cellulare, permettendogli pertanto di riparare a tutti i danni prima che la cellula replichi il DNA.

	\subsection{Riparazione omologa}
	La riparazione omologa avviene quando la forcella stala o viene rotta indipendentemente.
	Durante la meiosi catalizza lo scambio di informazioni genetiche tra cromosomi omologhi materni e paterni.
	Nel meccanismo avviene uno scambio di filamenti di DNA tra un paio di duplex omologhi della sequenza di DNA molto simili nella sequenza nucleotidica.
	Una rottura a doppio filamento programmata seguita da ricombinazione omologa porta al crossing-over durante la meiosi aumentando la variabilit\`a genetica durante la riproduzione sessuata.

		\subsubsection{Meccanismo di guida}
		Il meccanismo avviene tra duplex di DNA con una sequenza omologa estensiva.
		Testano la sequenza a vicenda.
		L'interazione pu\`o essere limitata permettendo a una doppia elica di riformarsi dai singoli filamenti.

		\subsubsection{Capacit\`a della riparazione omologa}
		La ricombinazione omologa pu\`o riparte doppi filamenti accuratamente senza perdita di informazione.
		Il procesos avviene dopo la replicazione.

		\subsubsection{Cambio di filamenti}
		\emph{RecA} in E. coli e \emph{Rad51} negli eucarioti catalizzano lo scambio tra i filamenti legandosi cooperativamente al filamento invasore.
		Questo si lega al duplex allungandolo, destabilizzando e facilitando la separazione dei filamenti.

\section{Mitocondrio}
I mitocondri sono organelli citoplasmatici che si trovano tipicamente in prossimit\`a del ER in quanto sfruttano il gradiente di \emph{$Ca^{2+}$} in esso contenuto sequestrando lo ione in modo da agire come sua riserva.
La loro funzione principale \`e di produrre \emph{ATP} durante la catena di trasporto finale degli elettroni.
Sono associati al citoscheletro, che ne determina distribuzione e orientamento.

	\subsection{Teoria endosimbiontica}
	Secondo la teoria endosimbiontica gli eucarioti si sarebbero originati dalla fusione di procarioti.
	Il mitocondrio e i cloroplasti corrisponderebbero ad organismi fagocitati da altre cellule in modo che ne traessero un vantaggio evoluzionistico.

		\subsubsection{Evidenze}
		\begin{multicols}{2}
			\begin{itemize}
				\item I mitocondri sono dotati di un doppio sistema di  membrane.
				\item Il mitocondrio codifica per ribosomi $70S$.
				\item Il mitocondrio contiene DNA extra-nucleare.
				\item Sono capaci di dividersi autonomamente per scissione binaria.
			\end{itemize}
		\end{multicols}

	\subsection{Numero e morfologia}
	Le cellule contengono tipicamente tra i \num{1000} e i \num{2000} mitocondri per cellula.
	Il numero varia in base al tipo cellulare e un elevato numero di mitocondri \`e indice di un'elevata richiesta energetica da parte della cellula.
	Nell'oocita se ne trovano \num{30000}.
	I mitocondri delle cellule epatiche contengono il $30$-$35\%$ delle proteine.

		\subsubsection{Membrana esterna}
		La membrana esterna contiene una porina che forma un canale ed \`e pertanto permeabile ad un elevato numero di molecole.
		Il trasportatore \`e detto complesso \emph{TOM}.
		
		\subsubsection{Spazio intermembrana}
		Lo spazio intermembrana contiene numerosi enzimi che direzionano e digeriscono le sostanze importante attraverso la membrana esterna.

		\subsubsection{Membrana interna}
		La membrana interna \`e ripiegata in numerose creste e invaginazioni in modo da aumentarne la superficie.
		Contiene proteine per l'importo \emph{TIM}.

		\subsubsection{Matrice}
		La matrice contiene una miscela altamente concentrata di enzimi.
		Contiene parecchie copie identiche del DNA del genoma mitocondriale, ribosomi mitocondriali e tRNA.

		\subsubsection{Variazioni morfologiche}
		Le creste differiscono per lunghezza, forma e numero a seconda delle richieste energetiche della cellula.

			\paragraph{Cellule normali}
			Tipicamente nelle cellule le creste si allungano per met\`a della matrice e sono corte in corrispondenza di bassa richiesta energetica.

			\paragraph{Cellule muscolari}
			Nelle cellule muscolari le creste attraversano tutta la matrice.
			Sono impacchettate molto strette e si trovano in numero elevato in corrispondenza di elevata richiesta energetica.

	\subsection{DNA mitocondriale}
	Il genoma del mitocondrio contiene geni per la produzione di un sistema di traduzione con un codice di riconoscimento proprio.
	\`E comunque estremamente limitato: la maggior parte delle proteine viene introdotta dalla cellula dopo la traduzione.
	Queste proteine contengono una sequenza \emph{MLS} (mitochondria localization element).
	Essendo che il DNA mitocondriale \`e di origine materna \`e possibile utilizzarlo per tracciare la matrilienearit\`a.
	Ogni mitocondrio porta dieci copie del genoma mitocondriale associate in regioni nucleoidi multipli.
	Il DNA \`e circolare e a doppio filamento.
