\chapter{Replicazione e riparazione del DNA}

\section{Modello di replicazione}
La duplicazione delle cellule richiede la duplicazione del DNA, in modo che entrambe le cellule figlie possano ricevere una copia uguale del materiale genetico.
La natura a doppia elica delle molecole di DNA permette la creazione di tre ipotesi riguardo il modello di replicazione.
\begin{multicols}{2}
	\begin{itemize}
		\item Replicazione dispersiva: la molecola di DNA dopo la replicazione \`e un mosaico di parti di nuova sintesi e parti di origine parentale.
		\item Replicazione semiconservativa: la molecola di DNA dopo la replicazione \`e formata da un filamento di nuova sintesi e uno di origine parentale.
		\item Replicazione conservativa: la catena parentale si duplica in due catene in modo che in una rimangano entrambi i filamenti parentali mentre nell'altra ci siano solo filamenti di nuova sintesi.
	\end{itemize}
\end{multicols}

	\subsection{Esperimento Meselson-Stahl}
	L'esperimento Meselson-Stahl si propone di determinare il modello di replicazione.
	Viene compiuto su batteri con l'utilizzo di $2$ isotopi radioattivi dell'azoto\emph{$\ ^{15}N\ ^{14}N$}.
	L'isotopo dell'azoto viene incorporato dai batteri nelle molecole di DNA durante la loro crescita.
	Il peso diverso dei due isotopi permette di distinguere le molecole di DNA che ne contengono diversi.

		\subsubsection{Processo}
		Si fa crescere una generazione di batteri in coltura in presenza di \emph{$\ ^{15}N$}.
		Questi presentano pertanto una doppia elica pesante.
		La prima generazione viene trasferita in condizioni normali: \emph{$\ ^{14}N$}.
		Si compie un ulteriore ciclo si replicazione in \emph{$\ ^{14}N$}.
		Da questa si estraggono le molecole di DNA e le si separa secondo gradiente.
		
		\subsubsection{Risultati}
		Si nota come nella seconda generazione si trovano molecole di DNA ibride, mentre in quella successiva da $2$ molecole ibride se ne ottengono $2$ leggere e $2$ ibride.
		Questo esperimento determina che il modello di replicazione \`e semiconservativo.


\section{Panoramica}

	\subsection{Reazione di polimerizzazione}
	La replicazione del DNA richiede la polimerizzaizone di nuove catene.
	La polimerizzazione avviene attraverso l'aggiunta di un deossiribonuclotide $3P$ all'estremit\`a $3'$ di una catena polinucleotidica detta filamento primer.
	Si nota come il filamento pu\`o pertanto polimerizzare unicamente in direzione $5'$-$3'$.
	Questo processo viene guidato dall'accoppiamento tra le basi e un deossiribonucleotide trifosfato.
	Il legame causa il rilascio di pirofosfato.

		\subsubsection{Substrati}

			\paragraph{Deossiribonucleotidi trifosfato}
			Quattro deossiribonucleotidi trifosfato \emph{dNTP} possiedono tre gruppi fosforici $\alpha$, $\beta$ e $\gamma$ attaccati al $2'$-deossiribosio tramite un legame $5'$-ossidrilico.
			Vengono attaccati al filamento in sintesi con rilascio della molecola di pirofosfato.
			\begin{multicols}{4}
				\begin{itemize}
					\item \emph{dGTP}.
					\item \emph{dCTP}.
					\item \emph{dATP}.
					\item \emph{dTTP}.
				\end{itemize}
			\end{multicols}

			\paragraph{Giunzione innesco}
			La giunzione innesco \`e formata da un filamento singolo di DNA che dirige l'aggiunta dei \emph{dNTP} complementari.
			L'innesco \`e un primer a DNA o RNA molto pi\`u corto del filamento che serve come sito di inizio dell'aggiunta dei nuovi \emph{dNTP} liberi.
			Necessario in quanto presenta un \emph{$3'$-OH} libero per l'inizio della reazione.

		\subsubsection{Crescita del filamento}
		Il filamento di nuova sintesi cresce allungando l'innesco al $3'$.
		Avviene la formazione di un legame fosfodiesterico tra innesco e \emph{dNTP}.
		\emph{-OH} a $3'$ dell'innesco esegue un attacco nucleofilo al fosfato $\alpha$ del \emph{dNTP} creando il legame fosfodiesterico e causando l'uscita del pirofosfato con i fosfati $\alpha$ e $\beta$.

		\subsubsection{Energia}
		Pur essendo la reazione termodinamicamente sfavorevole, l'idrolisi del pirofosfato effettuata da una pirofosfatasi fornisce l'energia necessaria alla sua attivazione.
		Il pirofosfato rilasciato dalla reazione viene pertanto idrolizzato in due molecole di fosfato inorganico.
		In questo modo $\Delta G_{tot} < 0$.

	\subsection{Mutazioni}
	Il processo di replicazione non \`e perfetto: durante esso possono essere infatti introdotti errori.
	
		\subsubsection{Tasso di mutazione}
		Si nota come per un singolo gene per una proteina media lungo $1000nt$ accumula una mutazione ogni $10^6$ generazioni.
		Avvengono pertanto $3$ cambi nucleotidici per $10^{10}$ nucleotidi per generaizone di cellule.
		Nell'essere umano avviene $1$ mutazione ogni $10^{10}$ divisioni celluari.

		\subsubsection{Tipologie di mutazioni}

			\paragraph{Geniche}
			Le mutazioni geniche avvengono per singole basi:
			\begin{multicols}{2}
				\begin{itemize}
					\item Sinonime o di sostituzione: viene introdotto un codone diverso che codifica per lo stesso amminoacido.
					\item Di senso errato: il codone viene sostituito con un altro che codifica per un amminoacido diverso.
					\item Non senso: viene formato un codone di stop.
					\item Frameshift: avviene uno spostamento dell'ordine di lettura per inserimento o delezione.
					\item Per sequenze ripetute.
				\end{itemize}
			\end{multicols}
			
			\paragraph{Cromosomiche}
			Le mutazioni cromosomiche consistono di traslocazioni di geni tra cromosomi.

			\paragraph{Genomiche}
			Le mutazioni genomiche consistono di perdita o aggiunta di numerosi geni.

	\subsection{DNA polimerasi}
	La DNA polimerasi \`e l'enzima responsabile della sintesi della catena di DNA.
	Lavora in modo processavo in direzione $3'$-$5'$.
	\`E un oloenzima, possiede ovvero numerosi domini con funzioni diverse.
	Oltre all'attivit\`a polimerasica possiede un'attivit\`a esonucleasica $3'$-$5'$ che rimuove i nucleotidi erroneamente appaiati e una esonuclasica $5'$-$3'$.



	\subsection{Meccanismi di correzione}
	L'alta fedelt\`a del meccanismo della replicazione richiede meccanismi di correzione.

		\subsubsection{DNA polimerasi}
		La DNA polimerasi compie essa stessa una correzione prima dell'aggiunta di un nuovo nucleotide.
		Il nucleotide corretto ha infatti un'affinit\`a maggiore per la polimerasi in movimento del nucleotide non corretto.

		\subsubsection{Attivit\`a esonucleotidica}
		La DNA polimerasi possiede un'attivit\`a esonucleotidica.
		Questa viene attivata quando viene aggiunto un nucleotide scorretto: le DNA polimerasi sono selettive riguardo la catena in allungamento.
		Viene pertanto eliminato qualsiasi residuo non appaiato al terminale del primer.
		La rimozione continua all'indietro fino a che non sono stati rimossi abbastanza nucleotidi da rigenerare un \emph{-OH} terminale appaiato.

	\subsection{Sintesi sul filamento discontinuo}
	Se al filamento principale \`e necessario un solo primer all'inizio della replicazione, l'attivit\`a unidirezionale della DNA polimerasi richiede un processo diverso per il filamento in ritardo.
	Questo filamento viene diviso in frammenti di Okazaki, corte regioni di DNA che la DNA polimerasi \`e in grado di sintetizzare.
	Ognuno di questi frammenti richiede un primer, e dopo che la DNA polimerasi completa la sintesi di uno si sposta al successivo.

		\subsubsection{DNA primasi}
		Il processo di produzione del primer dipende dalla DNA primasi, che usa trifosfati ribonucleici per sintetizzare corti primer a RNA sul filamento in ritardo.
		Il primer a RNA crea legami con il filamento in ritardo generando una doppia elica ibrida.
		Il primer contiene una terminazione \emph{$3'$-OH} e pu\`o essere allungato dalla DNA polimerasi.
		
		\subsubsection{Terminazione del frammento}
		La DNA polimerasi che allunga il primer si ferma quando incontra il primer attaccato alla terminazione $5'$ del frammento precedente.
		Il frammento \`e reso continuo da un sistema di riparazione del DNA, che elimina i primer e li sostituisce con DNA.
		Infine la DNA ligasi unisce le terminazioni dei frammenti.
		La rimozione del primer avviene a carico di \emph{RNAasi H}.

	\subsection{Apertura della doppia elica}
	Essendo la doppia elica stabile in condizioni fisiologiche sono necessarie DNA elicasi per aprire la doppia elica.
	Queste proteine idrolizzano \emph{ATP} e sono legate a un singolo filamento del DNA.
	Quando incontrano una regione a doppia elica continuano a muoversi lungo il proprio filamento separandolo a $1000\frac{nt}{sec}$.
	Possono lavorare in entrambe le direzioni della polarit\`a.
	Proteine che legano filamenti singoli di DNA o che destabilizzano l'elica si legano cooperativamente per esporre singoli filamenti senza coprire le basi.
	In questo modo stabilizzano la conformazione a singolo filamento e impediscono la formazione di corte eliche a forcina nel filamento in ritardo.

	\subsection{Mantenimento di una DNA polimerasi in movimento sul DNA}
	Le DNA polimerasi tendono a sintetizzare una piccola stringa di nucleotidi prima di separarsi dallo stampo.
	Questa bassa affinit\`a con il DNA le permette di separarsi ed essere riciclata velocemente dopo la sintesi di un frammento di Okazaki.
	Rende per\`o difficile la sintesi di lunghe sequenze.

		\subsubsection{Morsetto scorrevole}
		La proteina \emph{PCNA} funziona come morsetto scorrevole: mantene la polimerasi fermamente sul DNA mentre si muove e la rilascia appena incontra una regione a doppia elica.
		Forma un grande anelli intorno al filamento.
		Una faccia si lega al retro della DNA polimerasi, mentre l'anello scorre lungo il DNA.
		L'assemblaggio richiede idrolisi di \emph{ATP} da parte di un complesso detto caricatore del morsetto che la pone su una giunzione primer.
		Sul filamento principale la DNA polimerasi rimane legata al morsetto per molto tempo, mentre su quello in ritardo si dissociano ogni volta che la polimerasi arriva sulla terminazione $5'$ di un frammento di Okazaki.

	\subsection{Modello a trombone}
	Il modello a trombone indica il processo con il quale viene replicato il DNA.
	Si nota come entrambi i filamenti sono sintetizzati simultaneamente.
	Nel modello l'elicasi si muove in direzione $5'$-$3'$.
	L'oloenzima di replicazione interagisce con l'elicasi attraverso un fattore $\tau$ che lega entrambe le polimerasi.
	Un core della DNA polimerasi $III$ sintetizza il filamento guida e l'altro quello in ritardo.
	Il ssDNA viene costantemente coperto da \emph{SSB} che ne stabilizzano la struttura.
	Periodicamente la DNA primasi si associa all'elicasi per formare il primer sul filamento discontinuo.
	Quando la polimerasi di quest ultimo completa un frammento di Okazaki viene rilasciata dalla pinza e dal DNA.
	Il filamento con un innesco diventa il bersaglio dell'attivit\`a del caricatore della pinza che riconoscie la giunzione innesco stampo.
	Questa legata dalla pinza richiama la DNA polimerasi che riparte con la sintesi del frammento di Okazaki successivo.

	\subsection{DNA topoisomerasi}
	Il movimento della forcella di replicazione lungo il DNA causa superavvolgimenti.
	Questi aumentano lo stress sulla doppia elica contrastando le elicasi.
	Devono pertanto intervenire topoisomerasi, nucleasi reversibili che rompono i legami fosfodiesterici nel backbone in modo da ridurre i superavvolgimenti e permettere la continuazione della polimerasi.

\section{Complesso di pre-replicazione}

	\subsection{Formazione}
	Il complesso di pre-replicazione si forma alle origini di replicazione.

		\subsubsection{Origine di replicazione}
		Un'origine di replicazione \emph{Ori} \`e una sequenza specifica che contiene:
		\begin{multicols}{2}
			\begin{itemize}
				\item Sito di legame per la proteina iniziatrice \emph{ORC} (origin recognition complex).
				\item Tratto ricco di $A$-$T$ per facilit\`a di separazione.
				\item Sito di legame per proteine che facilitano il legame con \emph{ORC}.
			\end{itemize}
		\end{multicols}

		\subsubsection{Processo di formazione}
		\begin{multicols}{2}
			\begin{itemize}
				\item \emph{ORC} riconosce e si lega a \emph{Ori}.
				\item \emph{ORC} recluta \emph{Cdt1} e \emph{Cdc6}.
				\item \emph{Cdt1} e \emph{Cdc6} reclutano le elicasi.
				\item Le elicasi \emph{MCM2-7} sono caricate sul DNA vicino a \emph{ORC} durante la fase $G_1$.
				\item Il complesso \`e ora assemblato sul DNA e rimane inattivo.
			\end{itemize}
		\end{multicols}
		
	\subsection{Attivazione}
	Il complesso di pre-replicazione viene attivato quando la cellula entra nella fase $S$.

		\subsubsection{Chinasi ciclina dipendente}
		Le chinasi ciclina dipendente \emph{CDK} sono regolate dal ciclo cellulare: innescano la replicazione e impediscono l'assemblaggio simultaneo di nuovi complessi replicativi fino alla fase $M$.
		Fosforilano anche \emph{ORC} impedendo che recluti altre elicasi.

		\subsubsection{Processo di attivazione}
		\begin{multicols}{2}
			\begin{itemize}
				\item \emph{Cdk} e \emph{Ddk} fosforilano \emph{Cdt1} e \emph{Cdc6} promuovendo distacco e proteolisi.
				\item La fosforilazione di \emph{MCM2-7} le attiva e causa il reclutamento di altre proteine.
				\item Vengono reclutate le DNA polimerasi $\delta$, $\epsilon$, $\alpha$ e la DNA primasi.
				\item Viene reclutata la pinza e il suo caricatore \emph{PCNA} e \emph{RF-C}.
				\item Avviene il distacco della primasi.
				\item La DNA polimerasi $\delta$ inizia la replicazione sul filamento discontinuo.
				\item La DNA polimerasi $\epsilon$ inizia la replicazione sul filamento guida.
				\item Due forcelle di replicazione vengono attivate contemporaneamente in direzioni opposte.
			\end{itemize}
		\end{multicols}

\section{Replisoma}
Il replisoma \`e il complesso multi-enzimatico responsabile della replicazione del DNA.

	\subsection{Composizione}

		\subsubsection{\emph{MCM-7}}
		\emph{MCM-7} o minicromosome maintenance \`e un complesso proteico ad attivit\`a elicasica dipendente da \emph{ATP}.
		Il suo caricamento \`e efficiente solo a seguito del legame sequenziale con \emph{Cdt1} e \emph{Cdc6}.
		Apre la doppia elica formando la bolla di trascrizione.
		Viene successivamente sostituito da un'elicasi pi\`u processiva.

		\subsubsection{DNA ligasi}
		La DNA ligasi unisce le estremit\`a $3'$ del nuovo filamento a quella $5'$ del precedente.

		\subsubsection{Elicasi}
		Le elicasi sono enzimi che aprono la doppia elica utilizzando \emph{ATP}.
		Oscillano cambiando conformaizone in modo ciclico.
		Lavorano in entrambe le direzioni.

		\subsubsection{Single strand binding protein}
		Le proteine \emph{SSB} si legano al ssDNA stabilizzandolo e impedendo la formazione di hairpins e mantenendo la catena lineare.
		Si staccano all'avvicinamento della DNA polimerasi.

		\subsubsection{DNA primasi}
		La DNA primasi sintetizza brevi primers sul filamento ritardato fornendo l'innesco alla DNA polimerasi.

		\subsubsection{DNA polimerasi}
		La DNA polimerasi \`e un enzima che sintetizza il DNA in direzione $5'$-$3'$.
		Possono essere coinvolte anche nei processi di riparazione del DNA e si distinguono in:
		\begin{multicols}{2}
			\begin{itemize}
				\item $\alpha$: fa partire la sintesi.
				\item $\delta$: pi\`u attiva nel filamento principale.
				\item $\epsilon$: pi\`u attiva nel filamento discontinuo.
				\item $\gamma$: DNA polimerasi mitocondriale.
			\end{itemize}
		\end{multicols}

			\paragraph{Tipologie}

				\subsection{Tipo $\mathbf{I}$}
				Le DNA polimerasi di tipo $I$ hanno una struttura a mano capace di riconoscere e correggere errori.
				Le correzioni sono effettuate attraverso:
				\begin{multicols}{2}
					\begin{itemize}
						\item Attivit\`a esonucleasica: degrada dall'esterno con \emph{-OH} libero.
						\item Attivit\`a endonucleasica: taglia il nucleotide dall'interno.
					\end{itemize}
				\end{multicols}
				L'attivit\`a \`e processiva in direzione $5'$-$3'$ nel filamento leader, mentre non processiva nel filamento discontinuo.
				Nel secondo caso necessita di primer a RNA allungati in frammenti di Okazaki e riattaccati grazie a DNA ligasi.
				L'attivit\`a cataliticha le permette di formare il legame fosfodiesterico.

		\subsubsection{Esonucleasi}
		Le esonucleasi sono gli enzimi deputati alla rimozione dei primer.
		
		\subsubsection{Complesso $\mathbf{\beta}$ o pinza scorrevole}
		La pinza scorrevole \`e formata da $2$ subunit\`a a ciambella che si posizionano sul DNA.
		Il complesso $y$ la carica sul DNA.
		Permette alla DNA polimerasi di rimanere attaccata al singolo filamento di DNA.
		Si stacca nelle regioni a doppia elica.

		\subsubsection{\emph{CDK}}
		Le \emph{CDK} hanno un'attivit\`a di fosforilazione che segue le cicline.

		\subsubsection{\emph{TBP/TUS}}
		\emph{TBP/TUS} o ter binding proteins rimuovono gli istoni da \emph{OriC} e sono riconosciute da sequenze \emph{TER}.

		\subsubsection{Complesso di pre-replicazione}
		\emph{ORC} permette il recltuamento del repliosoma.

		\subsubsection{DNA topoisomerasi}
		Le DNA topoisomerasi permettono la risoluzione dei superavvolgimenti.




\section{Processo di replicazione}

	\subsection{Inizio della replicazione}
	La replicazione inizia da regioni specifiche ricche di $A$ e $T$ in quanto questi nucleotidi formano meno legami a idrogeno e sono pi\`u facilmente separabili.
	Attraggono le proteine iniziatrici.

		\subsubsection{Procarioti}
		Il genoma dei procarioti \`e tipicamente contenuto in una molecola di DNA circolare.
		La replicazione inizia ad un singolo sito e le due forcelle procedono in direzioni opposte fino a che si incontrano.
		
			\paragraph{Regolazione}
			L'inizio \`e l'unico punto in cui la replicazione viene regolata.
			Il processo inizia quando proteine legate a \emph{ATP} si legano in copie multiple in siti specifici del DNA avvolgendolo.
			Si forma un complesso che destabilizza la doppia elica vicina.
			Questo attrae due DNA elicasi legate a un caricatore che le mantiene inattive fino al caricamento giusto.
			L'elicasi una volta caricata svolge il DNA esportando i filamenti in modo che la DNA primasi sintetizzi il primer che porta all'assemblaggio delle proteine  necessarie alle creazione delle due forcelle di replicazione.
			La proteina iniziatrice viene disattivata e l'origine di replicazione subisce un periodo refrattario causato da un ritardo nella metilazione dei nuovi nucleotidi.

			\paragraph{\emph{Ori}}
			\emph{Ori} \`e formata da $4$ ripetizioni di $9$ nucleotidi e $3$ ripetizioni di $13b[$ a monte di esse.
			Viene riconosciuta dalle \emph{DNA A} che si lega alle ripetizioni facilitando la denaturazione e recluta le elicasi \emph{DNA C-B}.

		\subsubsection{Eucarioti}
		Il genoma degli eucarioti si trova tipicamente in cromosomi lineari.
		La replicazione inizia a diversi \emph{Ori} e la bolla di replicazione si espande fino alla terminazione del cromosoma o fino a che si incontra un'altra forcella.
		Nei lieviti le \emph{Ori} prendono il nome di \emph{ARS} (autonomous replicating sequence).

		\subsubsection{Identificazione della regione contente un \emph{Ori}}
		Per identificare la regione che contiene l'origine di replicazione si sequenzia il DNA del lievito mediante enzimi di restrizione, lo si introduce nel DNA plasmidico contenente un gene per la sintesi di istidina.
		Il plasmide che contiene un \emph{ARS} sar\`a in grado di diffondersi e creer\`a coltura in terreno privo di istidina.
		Si restringe la regione aggiunta al plasmide in modo da identificare precisamente la sequenza.

	\subsection{Velocit\`a di replicazione}
	Le forcelle di replicazione eucariotiche si muovono di circa $50$ nucleotidi al secondo a causa dello stato cromatinico.

	\subsection{Organizzazione temporale}
	Se i batteri replicano il DNA in maniera continua negli eucarioti avviene durante la fase $S$, che in una cellula mammifera dura $8$ ore.
	Al termine della fase $S$ ogni cromosoma \`e stato completamente replicato in due coppie che rimangono unite al centromero fino alla fase $M$.

		\subsubsection{Temporizzazione delle \emph{Ori}}
		Negli eucarioti le origini di replicazione sono attivate in cluster di $50$ adiacenti.
		L'ordine di attivazione dipende dalla struttura cromatinica.

	\subsection{Assemblaggio dei nucleosomi}
	A causa della grande quantit\`a di proteine istoniche necessarie per produrre i nuovi nucleosomi la maggior parte degli eucarioti possiede copie multiple del gene di ciascun istone.
	La duplicazione dei cromosomi richiede la sintesi e l'assemblaggio di nuove proteine cromosomiche sul DNA in coda alla forcella.
	Sono sintetizzati durante la fase $S$.
	Un meccanismo di feedback regola il livello degli istoni liberi in modo che la loro quantit\`a rimanga costante e attiva i loro geni quando vengono inclusi in massa nel DNA durante la fase $S$.
	Quando la forcella di replicazione passa attraverso nucleosomi un complesso di rimodellamento della cromatina destabilizza il nucleosoma permettendo la replicazione.
	Mentre la forcella passa gli istoni sono spostati: l'ottamero viene rotto in un tetramero \emph{2(H3-H4)} che rimane associato con il DNA e viene distribuito tra uno dei due figli e in due dimeri \emph{H2A-H2B} vengono completamente rilasciati.
	\emph{2(H3-H4)} di nuova sintesi sono usati per riempire i buchi e i dimeri sono aggiunti a caso per completare i nucoleosmi.
	La lunghezza dei frammenti di Okazaki \`e determinata dal punto in cui la DNA polimerasi \`e bloccata dal nucleosoma.

	\subsection{Terminazione della replicazione}

		\subsubsection{Procarioti}
		Il DNA circolare dei procarioti risolve il problema della replicazione delle terminazioni: l'incontro tra le due forcelle di replicazione causa la terminazione della replicazione.

		\subsubsection{Eucarioti}
		Alle terminazioni il meccanismo di replicazione del filamento in ritardo incontra problemi: il primer a RNA finale non pu\`o essere sostituito da DNA in quanto non trova un \emph{$3'$-OH} disponibile.
		Il problema viene risolto attraverso i telomeri.

			\paragraph{Telomeri}
			I telomeri sono sequenze specializzate contenenti ripetizioni di sequenze corte $GGGTTA$ riconosciute dalla telomerasi che le rifornisce ogni volta che la cellula si divide.

				\subparagraph{Telomerasi}
				La telomerasi riconosce la fine di un telomero e la allunga nella ripetizione usando uno stampo a RNA presente nell'enzima.
				Lo stampo viene utilizzato per sintetizzare nuove copie della ripetizione.
				Dopo l'estensione del filo genitore la replicazione del filamento discontinuo pu\`o essere completata dalla DNA poliemrasi standard che usa le estensioni per sintetizzare il filamento complementare.

				\subparagraph{Funzionamento della telomerasi}
				Il terminale $3'$ del telomero viene legato da \emph{RNA TERC} (teloerase RNA component) creando una giunzione innesco-stampo.
				La giunzione viene usata dalla telomerasi \emph{TERT} per allungare l'estremit\`a di $6$ nucleotidi.
				\emph{RNA TERC} dopo essere usato come stampo si stabilizza sulla nuova estremit\`a saltando di $6$ nucleotidi e ripetendo il processo pi\`u volte.
				Terminata l'azione della telomerasi il filamento funziona da stampo per la sintesi del secondo filamento.
				La rimozione dell'ultimo innesco crea un terminale $3'$ sporgente rispetto al $5'$.

				\subparagraph{T-loop}
				Le proteine legate al telomero oltre a regolare la telomerasi hanno un rulo protettivo dell'estremit\`a $3'$: impediscono che venga riconosciuta come rottura del DNA.
				I telomeri formano un'ansa data dall'estremit\`a a filamento singolo che invade il doppio filamento del telomero stesso.

				\subparagraph{Lunghezza dei telomeri}
				Essendo i processi di regolazione della sequenza telomerica bilanciati approssimativamente una fine cromosomica contiene un numero variabile di ripetizioni telomeriche.
				Molte cellule hanno un numero di meccanismo che mantengono il numero delle ripetizioni in un intervallo, come \emph{POT, TRF1,2} che contano le sequenze aggiunte e bloccano l'enzima.
				Si nota come nella maggior parte delle divisioni cellulari le ripetizioni telomeriche sono erose con il tempo.
				Dopo molte generazioni le cellule discendenti avranno cromosomi senza funzione telomerica e le cellule vanno incontro a senescenza.
				Il numero di divisioni massime viene detto limite di Hayflick.
				Non permettere la degradazione dell'attivit\`a della telomerasi causa tumorigenesi.

				\subparagraph{Discheratosi congenita}
				La discheratosi congenita \`e una malattia genetica causata da mutazioni del gene discherina, una chinasi coinvolta nei processi di modifica di RNA ribosomiali che interagisce con la parte ad RNA della telomerasi influenzandone la stabilit\`a.
				Questa causa porta a una perdita della funzione della telomerasi e causa:
				\begin{multicols}{2}
					\begin{itemize}
						\item Pigemntazione anomala della pelle.
						\item Distrofia unguela (corrugamento, distruzione e perdita delle unghie).
						\item Ingrigimento precoce.
						\item Cirrosi epatica.
						\item Disordini intestinali.
					\end{itemize}
				\end{multicols}

\section{Riparazione del DNA}

	\subsection{Motivi biologici della riparazione}
	Nonostante il DNA sia stabile \`e suscettibile a cambi spontanei che porterebbero a mutazioni se lasciate non riparate.
	Le basi del DNA possono essere danneggiate da una collisione con metaboliti reattivi o da radiazioni ultraviolette.
	Queste modifiche se accumulate portano alla cancellazione di basi o a sostituzioni durante la replicazione portando a conseguenze letali.
	Si nota come la struttura a doppia elica \`e adatta alla riparazione in quanto porta due copie dell'informazione genetica: danni ad un filamento possono essere riparati usando l'altro come stampo.

	\subsection{Processi di riparazione}
	Le cellule possiedono diversi modi di riparare il DNA.
	Differiscono per come il danno viene eliminato.
	Questo processo pu\`o essere accoppiato alla trascrizione: la RNA polimerasi infatti stalla ad errori, causando un reclutamento degli enzimi necessari alla riparazione.

		\subsubsection{Riparazione tramite asportazione della base}
		Il cammino di riparazione tramite asportazione della base le DNA glicolasi riconoscono un tipo di base nel DNA e catalizzano una rimozione idrolitica.
		Una base alterata viene riconosciuta attraverso un flip-out del nucleotide.
		Il buco creato dalla DNA glicolasi viene riconosciuto da \emph{AP endonucleasi} che taglia il backbone riparando lo zucchero.

		\subsubsection{Asportazione del nucleotide}
		Il cammino di riparazione tramite asportazione del nucleotide pu\`o riparare danni causati da grandi cambi nella struttura: questa viene scansionata da un complesso multienzima che cerca distorsioni nella doppia elica.
		La DNA elicasi rimuove il filamento che contiene la lesione.
		Il gap prodotto \`e riparto da DNA polimerasi e ligasi.

	\subsection{DNA polimerasi specializzate}
	In caso di danni pesanti la replicazione del DNA viene fermata e si utilizzano DNA polimerasi diverse, versatili ma meno precise dette rtranslesion polimerasi per replicare attraverso il danno.
	Alcune possono riconoscere un danno specifico e aggiungere i nucleotidi necessari, mentre altre fanno congetture.
	Non possiedono proofreading e sono meno discriminanti nella scelta del nucleotide.

	\subsection{Rotture a doppio filamento}
	Le rotture a doppio filamento possono essere riparate attraverso unione delle terminazioni non omologa.
	In questo meccanismo le terminazioni rotte sono riunite attraverso DNA ligation con perdita dei nucleotidi nel sito dell'unione.
	Porta a mutazioni.

	\subsection{Effetti del danno al DNA}
	L'attivit\`a dei processi di riparazione blocca il ciclo cellulare, permettendogli pertanto di riparare a tutti i danni prima che la cellula replichi il DNA.

	\subsection{Riparazione omologa}
	La riparazione omologa avviene quando la forcella stala o viene rotta indipendentemente.
	Durante la meiosi catalizza lo scambio di informazioni genetiche tra cromosomi omologhi materni e paterni.
	Nel meccanismo avviene uno scambio di filamenti di DNA tra un paio di duplex omologhi della sequenza di DNA molto simili nella sequenza nucleotidica.
	Una rottura a doppio filamento programmata seguita da ricombinazione omologa porta al crossing-over durante la meiosi aumentando la variabilit\`a genetica durante la riproduzione sessuata.

		\subsubsection{Meccanismo di guida}
		Il meccanismo avviene tra duplex di DNA con una sequenza omologa estensiva.
		Testano la sequenza a vicenda.
		L'interazione pu\`o essere limitata permettendo a una doppia elica di riformarsi dai singoli filamenti.

		\subsubsection{Capacit\`a della riparazione omologa}
		La ricombinazione omologa pu\`o riparte doppi filamenti accuratamente senza perdita di informazione.
		Il procesos avviene dopo la replicazione.

		\subsubsection{Cambio di filamenti}
		\emph{RecA} in E. coli e \emph{Rad51} negli eucarioti catalizzano lo scambio tra i filamenti legandosi cooperativamente al filamento invasore.
		Questo si lega al duplex allungandolo, destabilizzando e facilitando la separazione dei filamenti.

\section{Mitocondrio}

	\subsection{Teoria endosimbiontica}

	\subsection{Variazioni morfologiche}

	\subsection{DNA mitocondriale}












Nella duplicazione del DNA si identifica un filamento stampo e uno di nuova sintesi. I due filamenti si identificano anche per la loro polarit\`a $3'-5'$ e viceversa. I filamenti stampo
hanno polarit\`a diverse, importante per il funzionamento della DNA polimerasi. Utilizzando il filamento stampo e l'appaiamento delle basi. Utilizzando un filamento primer la DNA
polimerasi introduce le basi complementari creando un nuovo filamento utilizzando quello parentale come stampo e va a promuovere la formazione del legame fosfodiesterico tra il gruppo
fosfato del nucleotide libero e quello OH presente al $3'$ viene scisso ATP con la liberazione di pirofosfato. 
\section{Mutazioni}
Il processo di replicazione \`e accurato ma non perfetto. Senza mutazioni non ci potrebbe essere evoluzione. Il processo pu\`o funzionare in maniera non del tutto precisa. Nei batteri
si possono introdurre errori ogni $10^{10}$ nucleotidici di $3-4$ cambi. Le mutazioni possono essere geniche (singoli basi), cromosomico e genomiche. Mutazione genica che normalmente 
avvengono a carico dei singoli nucleotidi, possono essere sinonima o di sostituzione: cambiamento di una base che non comporta un cambio amminoacidico, mutazioni di senso errato: il
codone viene sostituito con uno che codifica per un altro amminoacido, mutazioni non senso: si forma un codone di stop prematuro, mutazioni frameshift: si sposta l'ordine di lettura, 
mutazioni per frequenze ripetute che presenti nelle zone promotrici e la duplicazione della DNA polimerasi e aumenta il numero delle sequenze ripetute e pu\`o portare problemi in quanto
pu\`o rompere i meccanismi di replicazione (causare Corea di Hintington e la sindrome dell'X fragile). La mutazione cromosomica si riarrangiano parti di cromosoma con traslocazioni 
Inter cromosomiche, tipico di tumori e leucemie (poco tollerata), nelle mutazioni genomiche varia il numero di cromosomi (sindrome di Down, trisomia 21, 18). 
\section{Modello di duplicazione}
\subsection{Modello semi-conservativo}
La cellula figlia riceve una catena neosintetizzata e una parentale.
\subsection{Modello conservativo}
La cellula madre duplica l'intera catena del DNA e da origine a due cellule figlie una con il filamento parentale e una con il filamento neosintetizzato. 
\subsection{Modello dispersivo}
Ciascun filamento \`e un mosaico tra il DNA parentale e quello neosintetizzato. 
Per la conferma sperimentale si usano colonie batteriche contenenti azoto pesante \ce{^14N} i batteri venivano cresciuti in presenza di \ce{^15N} per una generazione in modo che tutte le
cellule lo introdussero, dopo di che vengono messe in un terreno con azoto leggero. Alla  duplicazione si incorpora azoto leggero. Utilizzando tecniche di centrifugazione si possono 
discriminare quelle contenenti azoto leggero e pesante. Si osserva che nella prima generazione si ha una miscela costituita da azoto leggero e pesante e nella generazione successiva 
scompare quella pesante e compare quella leggera e la miscela, risultato che conferma il modello semi-conservativo. 

La duplicazione del DNA avviene in un origine nei batteri nell'origine di replicazione dove vengono reclutata la macchina per la duplicazione, si forma la bolla di replicazione: 
le due catene si separano, viene reclutata la macchina di duplicazione e la duplicazione inizia e si forma la struttura theta che si allarga sempre di pi\`u fino a creare due catene
di DNA costituite da una catena parentale e una neosintetizzata. A livello di microscopia elettronica si pu\`o dare un analogo radioattivo (timina triziata) incorporata nel DNA di nuova
sintesi che rende visibile dove avviene la sintesi e incorporazione. L'origine di replicazione \`e caratterizzata attraverso esperimenti che comportano la digestione da un enzima di 
restrizione che taglia e i frammenti vengono legati e si va a vedere se il frammento del plasmide va incontro a duplicazione oppure no. Isolando il frammento e sequenziandolo \`e 
possibile capire la struttura del DNA responsabile dell'origine di replicazione. Negli eucarioti ci sono diverse origine di replicazione, alcune precoci nel ciclo cellulare, altre
attivate pi\`u tardivamente in  base alla struttura in cui si trova il DNA nel tempo di duplicazione: livello di compattazione. Ci sono diverse origine in quanto il DNA \`e molto lungo.
Attivando pi\`u origini di replicazione si riduce il tempo che la polimerasi impiega a completare la replicazione. Le bolle di replicazione si fonderanno successivamente tra di loro.
Le origini di replicazione o fabbriche di replicazione. L'apparato deputato alla replicazione \`e complesso ed \`e detto di cui l'attore principale \`e la DNA polimerasi a forma di mano
in cui il DNA si lega al palmo con strutture che permettono l'entrata dei nucleotidi incorporati e permette alle due catene di rimanere separate e domini di controllo della qualit\`a 
della duplicazione permettendo di notare incorporazioni errate. La DNA polimerasi lavora in modo processivo; ina volta che si attacca al DNA va a replicare lo stampo in maniera 
continuativa. UN enzima non processivo si attacca, incorpora, si stacca e si riattacca. La DNA polimerasi per la duplicazione del DNA lavora in maniera processiva. La DNA polimerasi 
\`e un enzima complesso e si dice oloenzima, costituito da subunit\`a con ruoli diversi. Aggiunge nucleotidi attivit\`a polimerasica corrispondenti a quelli presenti sulla catena stampo, 
ha attivit\`a esonucleasica $3'- 5'$ ed esonucleasica $5'=3'$ che permettono alla polimerasi di riconoscere errori. Esonucleasica: quando viene introdotto un nucleotide non complementare
viene rimosso e questa avviene in entrambe le direzioni. Il ruolo maggiore di duplicazione \`e data dalla DNA polimearsi I mentre le altre hanno ruolo in altri processi, in particolare 
nella riparazione del DNA rotture a singolo o doppio strand, cosa vera anche negli euocarioti. Considerando la bolla di replicazione la duplicazione del DNA non pu\`o avvenire in entrambe
le direzioni in quanto pu\`o incorporare nucleotidi solo in direzione $5'-3'$ solo un filamento viene duplicato. Mentre un filamento viene duplicato in maniera duplicativa sull'altro 
si formano dei piccoli frammenti di piccola dimensione detti frammenti di Okazaki, frammenti di DNA duplicato in direzione $5'-3'$. Si identificano uno strand leading o duplicato in 
maniera continuativa e uno in maniera discontinuativa con i frammenti o lagging. Questi frammenti sono formati preceduti dalla fomrazione di primer a innesco cosituiti da RNA importanti
in quanto la DNA polimerasi \`e in gradi di introdurre i nucleotidi solo se trova un gruppo Oh libero a livello del primo nucleotide. NOn \`e in grado di iniziare la sintesi da sola 
ma necessita di un filamento primer che permette l'inizio del processo di replicazione (a differenza dell'RNA polimerasi). La DNA polimerasi necessita di un innesco fornito da primer 
prodotti da una DNA primasi che si lega =, forma un innesco, brevi frammenti di RNA che forniscono il gruppo OH alla DNA polimerasi che si lega e inizia a polimerizzare in direzione 
$3'-5'$. La DNA polimerasi produce un filamento di Okazaki, poi viene rimosso da un enzima dall'una primer e dopo al gruppo OH libero la polimerasi pu\`o riempire il buco. Il complesso
\`e unico, uno per la duplicazione del filamento continuo e uno per il filamento discontinuo. In quello discontinuo forma una ansa che forma il modello a trombone. Ci sono nel complesso
pertanto due DNA polimerasi, una primasi per il primer e il filamento discontinuo, le proteine che si legano al DNA a filamento discontinuo lo proteggono e impediscono la formazione di
strutture secondarie che bloccano la polimerasi. Ci sono le elicasi che svolgono il DNA e le topoisomerasi che risolvono i superavvolgimenti e vengono reclutate tutte le proteine 
coinvolte nella formazione dei nucleosomi e dell'ottamero istonico. 

\section{Lezione 7}
La maggior parte delle DNA polimerasi sono coinvolte nelle riparazione dei danni del DNA che possono essere di diversi tipi. Esistono dei danni legati alla rottura delle catene dei
DNA e la cellula fa intervenire dei meccanismi specifiche per questi tipi di danno, meccanismo fondamentale in quanto pu\`o portare a mutazioni che se non riparate possono in alcuni 
casi portare all'apoptosi della cellula. Questo sta alla base dell'azione di molti farmaci tumorali che creano multiple rotture che possono portare alla morte cellulare e alla morte della
cellula tumorale. Il modello utilizzato che corrisponde a quello semi conservativo in cui la catena parentale si divide e funzionano da stampo, le cellule figlie ricevono un filamento
parentale e uno stampo. L'inizio della duplicazione del DNA avviene durante la fase S dove il DNA viene duplicato, si trova la preparazione che avviene nella fase G1 in cui si preparano
i siti di replicazione, i punti in cui le due catene iniziano a separarsi e viene reclutata la macchina di replicazione. Nel caso dei procarioti inizia nell'origine di replicazione, 
mentre negli eucarioti ci sono pi\`u origini di replicazione. Le forcelle di replicazione non sono distribuite omogeneamente ma sembrano raggrupparsi in foci di replicazione che
contengono ad alte concentrazione i fattori coinvolti nella replicazione del DNA. Il bromodiossiuridina \`e un analogo della timina che viene accoppiato con una molecola fluorescente 
attraverso un anticorpo. Ci sono pi\`u forcelle di replicazione distanziate da circa 50-70 kilobasi e gli essere umani possono avere dalle 100000 ai 100000 origini di replicazione. Non
sono determinate da una sequenza specifica ma sono ricche di A e T in quanto possiedono meno legami a idrogeno e sono pertanto pi\`u facili da separare. L'enzima principale coinvolto
nella duplicazione \`e la DNA polimerasi un enzima composto da una componente enzimatica (attivit\`a polimerasica) accompagnato da altri cofattori che lo aiutano a duplicare il DNA
e conferiscno altre caratteristiche: correggere gli errori un nucleotide incorporato in maniera errata l'enzima lo riconosce, rompe il legame fosfodiesterico e rimuovere la base 
scorretta. \`E molto efficiente: $10^9 - 10^{10}$ basi per un errore. La DNA polimerasi comune anche all'RNA polimerasi che produce RNA \`e che sono processive: una volta che si attacca
al DNA copia la catena stampo in maniera continuativa. La DNA polimerasi non ha una grande affinit\`a: tenderebbe a staccarsi dopo poche basi. Questa viene aumentata dalla pinza 
scorrevole che scorre sul DNA durante la duplicazione consentendo alla DNA polimerasi di rimanere attaccata al DNA. La funzione della DNA polimerasi \`e legata alla sua struttura a mano
con l'alloggiamento del DNA, un punto in cui gli ATP entrano e un punto in cui le catene escono. Il compito \`e di polimerizzare e aggiungere nucleotidi in $5'-3'$ usando un gruppo OH
libero in $3'$ di un filamento primer di innesco usa un DNTP con un trifosfato con un legame fosfodiesterico e il pirofosfato. La formazione di dlegame \`e spinta dalla scissione del
pirososfato con la formazione di due fosfati. La DNA polimerasi \`e in grado di sintetizzare in una sola direzione: pertanto per sintetizzare il filamento in direzione opposta si 
formano i frammenti di Okazaki con dimesioni variabili: da 1000-2000 per eucarioti e 100-200 per procarioti, creati attraverso la primasi che aggiunge degli inneschi formati da RNA, 
primer di dimensioni ridotte 10-20-30 che forniscono il gruppo $3'$ OH su cui la polimerasi si inserisce. Si formano frammenti con il DNA sintetizzato e una sequenza formata da RNA che
viene rimossa grazie all'RNAasi H che degrada l'RNA e riconosce l'ibrido e degrada rimuovendo i primer. A questo punto si trova un buco dove c'era il primer colmato dalla DNA polimerasi
e i frammenti vengono legati tra di loro da una DNA ligasi. In questo modo il filamento lagging produce un filamento duplicato continuo. Ci sono altri componenti reclutati durante la 
formazione dell'origine della forcella: nel caso di escherichia coli la DNA polimerasi III a cui si aggiungono altre DNA polimerasi per ruoli accessori. Il riconoscimento dell'escherichia
coli dell'origine di replicazione che \`e caratterizzata e riconosciuta da parte della DNAa, una proteina che contiene quattro ripetizioni di nove nucleotidi e tre ripetizioni di tredici
paia di basi a monte delle ripetizioni. La DNAa si attacca alle ripetizioni di nove e si lega a quelle di 13 facilitando la denaturazione del DNA e permettendo l'assemblaggio delle
proteine replicative: la DNAb e la DNAc che formano il complesso di preinnesco. La DNAb \`e un'elicasi che legando l'ATP ha il ruolo di svolgere e aprire le due catene di DNA che permette
il reclutamento della DNA primasi che crea i primer a RNA sui filamenti. Infine viene reclutata la DNA polimerasi. 

Negli eucarioti le cose sono pi\`u complicate: si ha una DNA polimerasi, fattori elicasi, primasi eccetera. Il processo di duplicazione \`e differente dalla tempistica in quanto il DNA
presenta un compattamento della cromatina accentuato e le zone molto compatte devono essere duplicate alla fine della fase S, molto tardive in quanto si devono decompattare le zone e 
intervengono fattori che decompattano la cromatina, rimuovono le modifiche post traduzionali e gli istoni stessi. I complessi che vengono nella forcella di replicazione: la DNA 
polimerasi, con subunit\`a $\alpha$ che polimerizza, due componenti con attivit\`a di proofreading a cui si aggiunge il complesso $\gamma$ o pinza di caricamento e il complesso $\beta$
che \`e la pinza scorrevole. L'apparato che duplica il DNA \`e un dimero, con due DNA polimerasi, due ................................... *audio corrotto* che vengono duplicati in maniera
asincrona. Il legame delle due DNA assicura che la velocit\`a sia singola. Queste subunit\`a $\tau$ rende dimerica la macchina deputata alla replicazione. 

La pinza scorrevole \`e la componente che permette alla DNA polimerasi di aumentare l'affinit\`a con il DNA. \`E fondamentalmente un anello che in condizioni di non interazione con DNA
si trova aperto: due subunit\`a. Quando si lega al DNA si chiude e forma un anello attorno al DNA che permette l'interazione e il mantenimento della DNA polimerasi al DNA e la segue
durante la replicazione. La pinza si carica sul DNA grazie al caricatore della pinza, che interagisce con le due DNA polimerasi. un ATPasi utilizza ATP, prende la pinza scorrevole, la
carica sul DNA, cambia conformazione, si stacca e lascia la pinza scorrevole favorendo il mantenimento  della DNA polimerasi sul DNA. L'attivit\`a della pinza scorrevole mantiene la 
DNA polimerasi e promuove l'attivit\`a della DNA polimerasi stessa che ha una sintesi molto rapida e viene mantenuta dalla presenza del fattore. Due filamento che si duplicano in 
maniera diversa ha prodotto il modello a trombone in cui si formano anse in vir\`u del fatto in cui uno dei due filamenti viene replicato diversamente. L'elicasi \`e un esamero di 
sei subunit\`a che legano l'ATP in maniera diversa: due opposte legano ATP, altre ADP e altre sono vuote: le elicasi separano i filamenti idrolizzando ATP si trovano sei subunit\`a 
identiche ma schiacciato l'anello: delle sei due opposte legano ATP, due ADP e due sono vuote. Questi tre stati si interscambiano cambiando la schiacciatura dell'anello e gli anelli
che legano il DNA oscillano e possono portare DNA nell'anello centrale svolgendolo. L'elicasi \`e uno dei primi fattori in quanto la sua funzione \`e quella di separare i due filamenti
permettendo il reclutamendo della macchina di replicazione e deve svolgere le due catene a monte. \`E precoce ma rimane associata. L'altra proteina \`e una proteina SSB che lega il 
DNA a singolo strand e impedisce che forme delle strutture secondarie stabili in modo che nel filamento discontinuo con la fomrazione dell'innesco a primer e quello a valle reso 
a singolo grazie all'elicasi che potrebbe assumere delle strutture secondiarie stabili che renderebbe difficoltosa l'attivit\`a della polimerasi. Questo viene pertanto lasciato lineare
e dopo le proteine si staccano e le strutture secondarie del DNA sono molto stabili e possono bloccare la macchina di duplicazione. 

Nel caso degli eucarioti arrivati in fondo al filamento la duplicazione termina, nel caso dei procarioti la situzione \`e semplice: la terminazione  dell'escherichia coli \` e terminata
da sequenze Ter molto brevi che sono poste a 180 gradi rispetto all'origine di replicazione. QUando le forcelle raggiungono questa zona la DNA polimerasi viene separata e vengono  
separate le due catene di DNA che si sono formate e sono legate a proteine che favoriscono la separazione. Nel caso degli eucarioti la situazione \`e pi\`u complessa in quanto \`e 
presente cromatina compatta: occorre rimuovere gli istoni dalle origini di replicazione, formare un complesso di pre replicazione presente alle origini di replicazione e tutto questo 
viene orchestrato da una macchina pi\`u complessa rispetto ai procarioti e viene coordinata da fattori dette cicline, che attivano la duplicazione del DNA durante la fase del ciclo 
cellulare specifico. Queste proteine controllano che l'attivazione avvenga nel momento corretto. Il reclutamento della macchina pu\`o avvenire prima della fase S: le cicline impediscono
che questo reclutamento dia origine immediata alla duplicazione del DNA. Il reclutamento non pu\`o avvenire prima della fase S e agisce sulle proteine che fosforilano elementi 
appartenenti alla macchina di replicazione. 
\subsection{Formazione del complesso di prereplicazione alle origini di replicazione}
Alcune proteine hanno nomi diversi con funzioni uguali: PCNA pinza scorrevole, polimerasi epsilon - polimerasi III, MCM 2-7 elicasi, polimerasi $\alpha$, primasi e SSB RPA. Il complesso
\`e comunque conservato. Esistono un numero maggiori di polimerasi con un ruolo importante nella riparazione del DNA. La duplicazione inizia negli eucarioti: un ruolo importante nel
processo di controllo di inizio \`e svolto dalle CDK che fosforilano chinasi - dipendenti dalle cicline, che vengono attivate dalle cicline che quando vengono prodotte attivano chinasi
e danno il via al processo di duplicazione. All'origine di replicazione vengono reclutati i primi attori come Orc 1 - 6 formato da sei subunit\`a e recluta altri due fattori Cdt1 r 
Cdc6 e questo complesso permette il reclutamento delle due elicasi MCM 2 - 7. Quest'ultimo complesso viene detto complesso di pre replicazione che non \`e in grado di duplicare il DNA
e non \`e ancora attivo. Viene reso attivo grazie al reclutamento di chinasi CDK che fosforila Cdc6 e Cdt1 che vengono rilasciate favorendo il reclutamento del caricatore della pinza, 
della pinza scorrevole, della DNA polimerasi, DNA primasi e cos\`i via. Senza l'attivit\`a della chinasi il complesso di pre replicazione non inizierebbe a reclutare fattori e non 
inizia la replicazione del DNA. L'attivit\`a delle cicline \`e importante per attivare la chinasi. Senza la sintesi delle cicline la duplicazione non ha inizio.  *audio corrotto* nei 
complessi di prereplicazione non funzionanti in quanto attivit\`a CDK \`e bassa. Andando a fosforilare i complessi dei siti di prereplicazione li vanno ad attivare. L'attivit\`a delle
cicline sono quattro famiglie di cicline che interagiscono con quattro CDK e sono attivate in punti strategici durante il ciclo cellulare e contribuiscono a controllare le varie fasi del
ciclo cellulare. Altri fattori associati alla macchina di replicazione sono le topoisomerasi, la replicazione del DNA con differenze legate a tempi, attivazione e componenti \`e 
simile tra eucarioti e batteri. 
\subsection{Istoni}
La maggior parte delle proteine istoniche viene sintetizzata nella fase G1 e la produzione degli istoni \`e massiccia, facilitata dal fatto che gli RNA sono molto stabili e che la 
maggior parte degli eucariote possiede geni multipli per ciascun istone. Gli istoni vengono anche sintetizzati durante la fase S. Alcuni degli istoni rimangono associati e vengono 
distribuiti in maniera randomica tra i due filamenti. 
\subsection{Telomeri}
I telomeri sono le estremit\`a dei cromosomi, caratterizzati da una sequenza ripetuta di GGGTTA. La problematica non \`e presente nei batteri in quanto hanno DNA circolare. Mentre
la loro problematica \`e presente dove non \`e presente DNA circolare ma lineare. Fu scoperta da Barbara McClinton nel mais e i cromosomi rotti tendono ad unirsi tra di loro formando
strutture non canonoiche. I telomeri \`e il ristultato della duplicazione non perfetta alle estremit\`a dei cromosomi. I telomeri sono originati e mantenuti, permettono di mantenere 
la stabilit\`a del cromosoma e li stabilizzano rendendoli incapaci di interagire con altre estremit\`a. Sono originati e mantenuto dalla telomerasi che scoprirono questa attivit\`a detta
transferasi terminale telomerasica. La telomerasi \`e un complesso con una componente proteica detto TerT e una componente a RNA detta TerC. Di fatto questo complesso si posizione 
nelle RNP. Ha una componente proteica a mano semi aperta e una a RNA all'interno del palmo. \`E necessaria in quanto l'ultima parte del cromosoma nel filamento lagging non possiede 
primer e non pu\`o pertanto essere sintetizzata in quanto non si trova l'OH necessario. La telomerasi usa a componente a RNA per allungare il filamento parentale e grazie alla sua 
attivit\`a di retrotrascrizione lo allunga con le sequenze ripetute $n$ volte aumentando lo spazio affinch\`e la primasi possa creare un primer e la polimerasi possa copiare la sequenza. 
Se non si aggiungessero si perderebbero informazioni. Questo processo perde efficienza con il numero di divisioni in quanto la lunghezza dei telomeri \`e inferiori. Correlazione tra 
invecchiamento cellulare e numero di telomeri. L'attivit\`a delle telomerasi \`e alta in cellule caratterizzata da un'elevata duplicazione come cellule germinali, cellule epiteliali, 
linfociti. Meno attiva nelle cellule staminali adulte e fibroblasti. La telomerasi aggiunge sequenze telomeriche e la polimerasi copia il segmento non copiato, si riforma la struttura
a doppio filamento riconosciuta da TRF1 e proteina POT (protection of telomers) che si associano alle porzioni terminali neosintetizzate. Il meccanismo non \`e ancora chiaro ma 
l'attivit\`a della telomerasi \`e bloccata quando questi segmenti diventano abbastanza grandi inibendo l'attivit\`a della telomerasi. Un attivit\`a della telomerasi alta permette
alle cellule di sopravvivere in maniera continua. Prendendo le cellule e mettendole in coltura mano a mano che si dividono vanno incontro a una fase esponenziale di crescita fino 
alla raggiunta di un plateau detto fase di senescenza, alterata potenziando l'attivit\`a delle telomerasi, questo limite prende il nome di limite di Hayflick, in cui le colture cellulari
vanno in un processo di senescenza. \`E possibile modificando l'attivit\`a delle telomerasi modificare questo limite. Nel caso dei fibroblasti con il gene della telomerasi non si ha
perdita dei telomeri e la proliferazione continua in maniera indefinita e le cellule non vanno incontro a senescenza. Si \`e scoperto che non \`e possibile andare incontro a immortalit\`a
in quanto overesprimendo la telomerasi le cellule durante la proliferazione accumulano errori che continuando a proliferare si aggiungono e aumenta la probabilit\`a del destino tumorale 
e che ci sono cellule senza attivit\`a telomerasica come le cellule del sistema nervoso che non vanno incontro a mitosi. Il ripristino dell'attivit\`a telomerasica \`e importante 
per l'aumento delle capacit\`a rigenerative. La discheratosi congenita per perdita delle funzione della telomerasi causa pigmentazione anomala, distrofia unguela, ingrigimento precoce, 
cirrosi epatica e disordini intestinali, il gene \`e localizzato sul cromosoma X e il gene mutato \`e la discherina, una chinasi che si lega a piccoli RNA nucleolari coinvolta nei 
processi di modifica dell'RNA ribosomiale. 

\section{Lezione 8}

\subsection{Mitocondri}

* Non ha registrato *

La struttura del sistema di creste del mitoconrio e la prossimit\`a con il sistema vescicolare del reticolo endoplasmatico. Non c'\`e continuit\`a di membrana e contenuti ma sono in
stretta prossimit\`a. Questo avviene in quanto i mitocondri sono un buffer per lo ione calcio contenuto all'interno dell'ER liscio e rilasciato nell'ambiente intracellulare e svolge il
ruolo di messaggero secondario. Quando non \`e pi\`u necessario rientra nel ER o nei mitocondri. I mitocondri possono dividersi (duplicazione del DNA) per fissione e possono fondersi 
tra di loro, regolato da una serie di meccanismi che risponde alle esigenze della cellula. I mitocondri sono strutture molto mobili, vengono trasportati utilizzando i microtubili come
binari. Le molecole del trasporto sono chinesine e dineine, molecole motrici che si attaccano ai mitocondri e microtubili permettendo il trasporto dalla periferia alla parte centrale e
viceversa. Il DNA mitocondriale \`e costituito da una moleocla a doppio filamento circolare. Non porta molti geni in quanto la maggior parte sono stati trasferiti all'interno  del genoma
della cellula ospita. Contiene 37 geni all'interno degli introni per l'rRNA, per il tRNA e 13 per gli mRNA che vengono tradotti in proteine. Si distingue una catena H heavy e 
complementare L low con asimmetria dei geni localizzati sulle due catene. Ci sono 5 - 10 molecole di DNA per mitocondrio con un codice genetico ridondante interpretato in maniera diversa
rispetto alla traduzione citoplasmatica. Vi sono patologie associate alla funzione dei mitocondri, un'eredit\`a materno che non segue le leggi di Mendel. L'insorgenza della malattia
dipende dalla percentuale di mitocondri che contengono quella mutazione. Recntemenente \`e stato visto che aluni RNA prodotti dal nucleo della cellula per i mitocondri vengono trasportati
in prossimit\`a dei mitocodnri, circondati da ribosomi della cellula e si localizzano in prossomit\`a dei mitocondri, esistono delle sequenze negli mRNA che trasportano l'RNA dal 
citoplasma in prossimit\`a dei mitocondri e quando lo raggiungono vengono tradotti e producono proteine e direttamente incorporate nei mitocondri e usano il sistema di trasporto TOM e 
TIM. La sequenza di segnale \`e poi rimossa dalle proteine da una peptidasi.  Senza il peptide segnale la proteina non interagisce e non entra nei mitocondri. I chaperons che si 
legano alla proteina entrante e impediscono che assuma una struttura secondaria durante il trasporto in modo da facilitare il passaggio durante il trasporto. Quando questo \`e finito 
si separano. Conoscendo la sequenza del peptide segnale si possono inserire proteine nel mitocondrio artificialmente. 
