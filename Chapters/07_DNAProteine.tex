\chapter{Come una cellula legge il genoma, dal DNA alle proteine}
Il DNA nel genoma usa l'RNA come intermediario nella sintesi delle proteine. Quando una cellula necessita una proteina utilizza la sequenza appropriata della catena nucleotidica 
copiandola in RNA durante la trascrizione che direziona direttamente la sintesi della proteina durante la traduzione. Esistono varianti di questo processo in cui i trascritti a RNA 
vengono processati nel nucleo con processi come RNA splicing prima che posano uscire da esso. Questi cambi possono cambiare il significato di una molecola di DNA. Per molti geni inoltre
il prodotto finale \`e RNA. I genomi di organismi multicellulari sono disordinati con corti esoni e lunghi introni. Sezioni che codficano il DNA sono separate da lunghe sequenze senza 
apparente significato. 
\section{Dal DNA all'RNA}
Essendo che molte copie identiche dello stesso RNA possono essere completate dallo stesso gene ogni molecola di RNA pu\`o guidare la sintesi di molte proteine identiche, ma i geni sono
trascritti e tradotti a tassi diversi, permettendo la cellula di avere vaste quantit\`a di alcune proteine e piccole di altre. Inoltre la cellula regola l'espressione di ognuno dei suoi
geni secondo i suoi bisogni, controllando la produzione del suo RNA>
\subsection{Le molecole di RNA hanno un unico filamento}
Il primo passo nella lettura delle istruzioni geniche \`e la copia di una particolare sequenza della sequenza di nucleotidi in una a RNA. L'infromazione nell'RNA \`e scritta nello stesso
linguaggio del DNA e questo processo \`e pertanto detto trascrizione. L'RNA \`e un polimero lineare composto da quattro tipi di subunit\`a nucleotidiche legate da legami a fosfodiestere.
Differisce dal DNA in quanto i nucleotidi nell'RNA sono ribonucleici, ovvero contengono ribosio e contiene la base uracile invece della timina che si pu\`o legare all'adenina. La
struttura complessiva \`e molto diversa: l'RNA \`e a filamento singolo e una catena pu\`o piegarsi in una forma simile a una proteina permettendogli di avere precise funzioni strutturali
e catalitiche.
\subsection{La trascrizione produce RNA complementare a un filamento di DNA}
L'RNA \`e sintetizzato attraverso la trascrizione del DNA, che comincia con l'apertura e lo svolgimento di una piccola porzione della doppia elica che espone le basi sui filamenti, uno
dei quali agisce come stampo per la sintesi della molecola di RNA. La sequenza di nucleotidi \`e determinata dall'accoppiamento di basi complementari tra i nucleotidi che arrivano e lo
stampo. Quando avviene una corrispondenza il ribonucleide che arriva \`e legato covalentemente con la catena crescente in una reazione catalizzata da enzimi. La catena \`e allungata un
nucleotide alla volta e possiede una sequenza complementare allo stampo. Il filamento di RNA non rimane legato con lo stampo ma \`e separato dietro al regione dove i nucleotidi sono
aggiunti causando il rilasciamento come singolo filamento. Le molecole di RNA sono inoltre molto pi\`u corte rispetto le molecole di DNA. 
\subsection{L'RNA polimerasi causa la trascrizione}
Gli enzimi che svolgono la trascrizione sono detti RNA polimerasi e catalizzano la formazione del legame fosfodiestere che lega i nucleotidi muovendosi lungo il DNA, svolgendo l'elica
sopra il sito attivo per la polimerizzazione. La catena di RNA \`e estesa nella direzione $5'$-$3'$. I substrati sono ribonicleoside trifosfato la cui idrolizzazione fornisce l'energia
necessaria alla reazione. Il rilascio immediato dell'RNA significa che le copie possono essere create in poco tempo, con la sintesi di molecole addizionali che inizia prima che quelle
prime sian completate. L'RNA polimerasi catalizza l'unione di ribonucledi e pu\`o cominciare una catena di RNA senza un primer. L'RNA polimerasi fa un errore una volta ogni $10^4$ 
nucleotidi e le conseguenze di tale errore sono meno significative. Inoltre la stessa RNA polimerasi che comincia una molecola di RNA eve finirla senza dissociarsi dallo stampo. L'RNA
polimerasi contiene un meccanismo di proofreading: se un ribonucleotide \`e aggiungo la polimerasi pu\`o indietreggiare e il sito attivo svolge una reazione di asportaizone dove una
molecola d'acqua sostituisce il pirofostato ed \`e rilasciata una molecola di monofosfato. 
\subsection{Le cellule producono diverse categorie di molecole di RNA}
La maggior parte dei geni trasportati in un DNA della cellula specificano la sequenza di amminoacidi della proteina e le molecole di RNA che sono copiate da questi geni sono detti RNA
messaggeri o mRNA. Il prodotto finale di altri geni \`e la molecola di RNA, detti RNA non codificanti che servono come componenti strutturali, enzimatiche e regolatore per molti 
processi. Molecole di RNA piccolo nucleare o snRNA direzionano lo splicing di pre-mRNA per formare mRNA, l'RNA ribosomiale o rRNA forma il nucleo del ribosoma e i tranfer RNA o tRNA
forma gli adattatori che selezionano gli amminoacidi e li mantengono in posizione. I microRNA o miRNA e RNA piccolo interferente siRNA servono come regolatori per l'espressione genica e 
RNA piwi-interagente o piRNA protegge le linee germinali dai trasposoni. I long noncoding RNA o lncRNA con funzione di impalcature e regolano diversi procesi cellulari come
l'inattivazione del cromosoma X. Ogni segmento di DNA trascritto \`e detto unit\`a di trascrizione che tipicamente possiede le informazioni di un gene. La maggior parte dell'RNA nella
cellula \`e rRNA/
\subsection{Segnali codificati nel DNA indicano l'RNA polimerasi dove iniziare e dove finire}
Per trascrivere un gene accuratamente la RNA polimerasi deve riconoscere dove iniziare e finire sul genoma. QUesto avviene in maniera diversa rispetto a batteri ed eucarioti. 
L'iniziazione di una trascrizione \`e il punto in cui la cellula regola quali proteine devono essere prodotto e a quale velocit\`a. L'RNA polimerasi batterica \`e un complesso a 
multisubunit\`a che sintetizza l'RNA. Una subunit\`a addizionale detta fattore sigma $\sigma$ associa con l'enzima nucleo e lo assiste nella lettura dei segnali nel DNA che dicono dove
iniziare la trascrizione. Il fattore $\sigma$ e l'enzima di nucleo formano un oloenzima RNA polimerasi che aderisce debolemente al DNA batterico quando collidono e scivola rapidamente 
lungo il DNA fino a dissociarsi. QUando l'oloenzima arriva a una sequenza speciale che indica il punto di inizio per la sintesi di RNA detto protomero si lega fortemente in quanto il 
fattore $\sigma$  crea contatto specifico con i limiti della base esposti all'esterno nella doppia elica. L'oloenzima polimerasi al protromero apre la doppia elica esponendo una piccola
lunghezza di nucleotidi su ogni filamento chiamata bolla di trascrizione (di circa $10$ nucleotidi), stabilizzata dal legame con il fattore $\sigma$ con le basi non accoppiate. L'altro
filamento agisce come stampo per l'accoppiamento di basi con i ribonucleotidi che arrivano, uniti dalla polimerasi per iniziare la catena di RNA. I primi $10$ nucleotidi sono 
sintetizzati attraverso un meccanismo di ``scrunching" dove la RNA polimerasi rimane legata al protomero e tira il DNA nel suo sito attivo espandendo la bolla di trascrizione. Questo
processo genere stress e ke catene di RNA sono rilasciate e forzando la polimerasi a riiniziare la sintesi. Questo processo di iniziazione abortiva \`e superato e lo stress generato 
aiuta l'enzima a rompere l'interazione con il protomero e con il fattore $\sigma$. La polimerasi inizia a muoversi lungo il DNA sintetizzando l'RNA muovendosi di base in base espandendo
la bolla e contraendola al retro. Si continua l'allungamento della catena fino a che l'enzima incontra un terminatore dove la polimerasi si ferma e rilascia la molecola di RNA e lo 
stampo a RNA. La polimearsi si riassocia con il fattore $\sigma$ ed \`e libera di riiniziare un processo di trascrizione. La maggior parte dei segnali di terminazione nei batteri \`e
formata da una stringa di coppie A-T precedute da una sequenza due volte simmetrica  di DNA che quando trascritta forma una forcina attraverso l'accoppiamento di basi che aiuta il 
disengaggio dell'RNA trascritto dal sito attivo.
\subsection{I segnali di inizio e terminazione della trascrizione sono eterogenei nella sequenza nucleotidica}
Le sequenze di inizio e fine sono codificate da sequenze in relazione, che riflettono aspetti del DNA che sono riconosciuti direttamente dal fattore $\sigma$. Queste caratteristiche 
formano una sequenza di nucleotidi consenzianti, una media di un gran numero di sequenze. Si possono anche riconoscere attraverso la frequenza relativa di basi in ogni posizione. Tale
sequenza nei batteri varia in modo da determinare la forza dei geni (il numero di eventi di iniziazione per gene). Per i terminatori la struttura di base \`e quella che forma la 
forcina nell'RNA. Il filamento scelto per la sintesi dell'RNA dipende dall'orientamento del promotore. 
\subsection{L'iniziazione della trascrizione negli eucarioti richiede molte proteine}
Gli eucarioti possiedono la RNA polimerasi I, II e III, simili strutturalmente e con subunit\`a in comune, ma trascrivono diverse categorie di geni. La I e la III trascrivono geni che 
codificano tRNA, rRNA e piccoli RNA, la II trascrive la maggior parte dei geni, inclusi quelli che codificano le proteine. La RNA polimerasi II  richiede molti fattori detti fattori di 
trascrizione generali e l'iniziazione avviene su DNA condensato in nucleosomi e forme superiori di struttura cromatinica. 
\subsection{La RNA polimearsi II richiede un insieme di fattori di trascrizione generali}
I fattori di trascrizione generali aiutano a posizionare la polimerasi correttamente al promotore, a separare i due filamenti di DNA permettendo l'inizio della trascrizione e rilasciarla
dal promotore per iniziare la modalit\`a di allumgamento. Sono generali in quanto richieste da tutti i promotori utilizzati dalla polimerasi II. Sono un insieme di proteine che 
interagiscono dette TFIIA, TFIIB, TFIIC, TFIID e hanno funzione equivalente al fattore $\sigma$. Il processo di assemblaggio inizia quando TFIID si lega a una sequenza di DNA a doppia 
elica detta TATA box attraverso la subunit\`a TBP. Il legame causa una distorsione nel DNA della TATA box che crea una marcatura per la locazione di un promotore attivo. Altri fattori
insieme alla RNA polimerasi II formano un complesso di iniziazione della trascrizione. Dopo che si \`e formato sul DNA promotore la RNA polimerasi II ottiene l'accesso al filamento 
stampo e TFIIH che contiene una DNA elicasi idrolizza l'ATP svolgendo il DNA. La RNA polimearsi II rimane al promotore sintetizzando corte lunghezze di RNA fino a subire una serie di
cambi conformazionali che le permettono di spostarsi ed entrare nella fase di allungamento. In questa transizione viene aggiunto un gruppo fosfato alla coda della RNA polimerasi detto 
CTD (C-terminal domain). Durante l'iniziazione la serina nella quinta posizione della sequenza ripetuta \`e fosforilata da TFIIH che contiene una chinasi in una delle subunit\`a. La
polimerasi pu\`o poi disengaggiarsi dal cluster e subisce una serie di cambi conformazionali che le permettono di trascrivere per lunghe distanze senza dissociarsi dal DNA. Dopo che si
\`e entrati nella fase di allungamento i fattori di trascrizione generali si separano per iniziare un altro processo. 
\subsection{La polimerasi II richiede attivatori, mediatori e proteine per la modifica della cromatina}
L'inibizione della trascrizione negli eucarioti \`e complessa e richiede molte proteine. Innanzitutto delle proteine dette attivatori trascrizionali devono legarsi a speciali sequenze
del DNA dette enhancers e auitare ad attrarre l'RNA polimerasi II al punto d'inizio. SUccessitamente \`e necessario un complesso proteico detto mediatore che permette alle proteine 
attivatrici di comunicare con la Polimearsi II e con i fattori di trascrizione generali. Alla fine l'iniziazione della trascrizione richiede il reclutamento di enzimi modificatori della
cromatina, complessi di rimodellizazzione e enzimi modificatori degli istoni che aumentano l'accesso al DNA nella cromatina. L'ordine di assemblaggio di queste proteine non segue un 
cammino preciso e differisce per gene. Per cominciare la trascrizione la RNA polimerasi II deve essere rilasciata da questo complesso attraverso proteolisii insito delle proteine 
attivatrici.
\subsection{L'allungamento della trascrizione negli eucarioti richiede proteine accessorie}
Una volta che l'RNA polimerasi ha iniziato la trascrizione si muove a scatti. RNA polimerasi allunganti sono associate con una serie di fattori di allungamento, proteine che diminuiscono
la probabilit\`a che l'RNA polimearsi si dissoci prima di aver finito la trascrizione. SI associano con la polimerasi dopo l'iniziazione. Quando l'RNA polimerasi si muove lungo un gene
alcuni enzimi legati ad essa modificano gli istoni, lasciando una traccia, che potrebbe aiutare nelle trascrizioni successive e nella coordinazione dell'allungamento.
\subsection{La trascrizione crea tensione superelicale}
Il superavvolgimento del DNA \`e una conformazione che il DNA assume quando \`e presente tensione superelicale. Un grande superavvolgimento di DNA si forma per ogni $10$ paia di 
nucleotidi svolti, processo energeticamente favorevole in quanto ripristina una torsione normale nella regione accoppiata rimanente. La RNA polimerasi crea tensione superelicale mentre
si muove lungo una lunghezza di DNA ancorata alla terminazione, con tensione positiva davanti a lei e negativa dietro. Negli eucarioti questa tensione \`e rimossa dalla DNA 
topoisomerasi, mentre nei batteri una DNA girasi usa l'energia dell'idrolisi dell'ATP per imporre superavvolgimenti al DNA mantenendolo in costante tensione, ma in senso opposto da 
quello fornito dalla polimerasi. 
\subsection{L'allungamento della trascrizione \`e strettamente accoppiato con il processamento dell'RNA}
Negli eucarioti la trascrizione \`e il primo passo per la produzione di una molecola di mRNA matura. Un passo successivo \`e la modificazione covalente delle terminazioni dell'RNA e 
la rimozione delle sequenze di introni nel processo di RNA splicing. La terminazione $5'$ viene incappucciata e la $3'$ attraverso polidenilazione che permettono alla cellula di capire
se le terminazioni sono entrambe presenti prima che sia esportata dal nucleo e tradotta. L'RNA splicing permette di sintetizzare proteine diverse dallo stesso gene. La fosforilazione
della coda CTD della polimearsi permette la disassociazione delle altre proteine e l'associazione di nuove. Alcune di queste si legano all'RNA che si sta sintetizzando processandolo. 
\subsection{L'incappucciamento dell'RNA \`e la prima modifica dei pre-mRNA eucariotici}
Appena l'RNA polimerasi II ha prodotto circa $25$ nucleotidi di RNA alla terminazione $5'$ della molecola viene aggiunto un cappuccio formato di una guanina modificata. Tre enzimi in 
successione svolgono la reazione necessaria: una fosfatasi rimuove il fsofato dalla terminazione, una guanil trasferasi aggiunge un GMP in un legame inverso e un metil trasferasi
aggiunge un gruppo metile alla guanosina. Questi ensimi si trovano alla catena della RNA polimerasi fosforilata alla posizione Ser5. Questo cappuccio metile indica la terminazione
$5'$ dell'mRNA eucariotico e lo differenzia da altri tipi di RNA. 
\subsection{L'RNA splicing rimuove le sequenze di introni dal pre-mRNA}
Essendo i geni eucariotici dispersi in sequenze di introni che vengono trascritte insieme agli esoni si devono rimuovere i primi attraverso l'RNA splicing per produrre la proteina.
Questo processo avviene alla produzione dell'mRNA. Ogni splicing rimuove un introne attraverso due rezioni di trasferimento di fosforile o transesterificazioni sequenziali che legano 
insieme gli esoni rimuovendo gli introni. Il macchinario che lo svolge \`e un complesso consistente di $5$ molecole di RNA e centinaia di proteine. Idrolizza molte molecole di ATP per
evento. La complessit\`a assicura uno splicing accurato e flessibile. Oltre agli espetti evoluzionistici della divisione in domini delle proteine per la ricombinazione lo splicing
permette di produrre un insieme di proteine diverse dallo stesso gene. 
\subsection{Sequenze di nucleotidi segnalano dove avviene lo slicing}
