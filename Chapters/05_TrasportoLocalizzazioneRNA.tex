\chapter{Trasporto e localizzazione di RNA}
Molti mRNA vengono diretti verso regioni intracellulari specifiche prima che inizi la traduzione.
Questo avviene in modo da permettere alla cellula di posizionarli vicino ai siti in cui \`e necessaria la proteina codificata.

\section{Motivi del trasporto di RNA}

	\subsection{Efficienza}
	Il trasporto di un mRNA permette la traduzione locale da un'unica copia di molteplici proteine risparmiando energia.

	\subsection{Sicurezza}
	Il trasporto di un mRNA \`e pi\`u semplice del trasporto in quanto impedisce che le proteine interagiscano tra di loro o con l'ambiente erroneamente.

	\subsubsection{Velocit\`a}
	Il trasporto di un mRNA consente cambiamenti rapidi nella concentrazione delle proteine locali.

	\subsubsection{Controllo della degradazione}
	Il trasporto del mRNA \`e un sistema utile per sottrarre il RNA stesso alla degradazione trasportandolo in un compartimento diverso.

\section{RNA trasportati}

	\subsection{RNA localizzati nel compartimento dendridico}
	\begin{multicols}{2}
		\begin{itemize}
			\item \emph{CaMKII$\alpha$}: \emph{$Ca^{2+}$}/Calmodulin-dependent Kinase $II$, una chinasi che si occupa del trasporto del calcio.
			\item \emph{MAP2}: microtubule-associated protein $2$, una proteina associata ai microtubuli.
			\item \emph{BDNF}: brain-derived neutotrophic factor.
			\item \emph{IP3R}: inositol triphosphate receptor, un recettore per l'inositolo.
			\item \emph{$\beta$-actin}: localizzata in una zona ricca di recettori e sensori, nella parte anteriore del leading edge, permette l'attacco al substrato della zona pi\`u mobile.
			\item \emph{Are}: activity regelated cytoskeleton-associated protein, proteine associate al citoscheletro.
			\item \emph{FMRP}: fragile $X$ mental retardation.
			\item \emph{BC1}: brain cytoplasmatic RNA $1$.
			\item \emph{microRNA}.
			\item \emph{Pum2}.
			\item \emph{MBP}, myelin basic protein, mRNA negli ologodendrociti.
		\end{itemize}
	\end{multicols}

	\subsection{RNA localizzati nell'oocita di Drosophila}
	\begin{multicols}{2}
		\begin{itemize}
			\item Gurken nella parte apicale, il mutante assume una forma a cetriolo.
			\item Prospero.
			\item Oskar nella parte posteriore dell'oocita.
			\item Bicoid nella parte anteriore dell'embrione, il mutante non presenta testa.
		\end{itemize}
	\end{multicols}

	\subsection{Lievito}
	Nel lievito la localizzaione di \emph{ASH1} gli permette di essere passato dalla cellula madre alla figlia.

\section{Meccanismo di trasporto}

	\subsection{Fattori coinvolti}

		\subsubsection{Cis-acting elements}
		I cis-acting elements o zipcodes sono fattori che riconoscono sequenze specifiche nella \emph{$5'$-UTR}.
		
			\paragraph{Esempi}
			\begin{multicols}{2}
				\begin{itemize}
					\item \emph{DTE} dendrites targeting elements.
					\item \emph{ALS} axon localization sequence.
					\item \emph{MLS} mitochondria localization sequence.
				\end{itemize}
			\end{multicols}
	
		\subsubsection{Trans-acting factors}
		I trans-acting factors sono proteine che riconoscono li cis-acting elements.
		
			\paragraph{Esempi}
			\begin{multicols}{2}
				\begin{itemize}
					\item Proteine motrici.
					\item \emph{hnRNP}: heterogeneus nuclear ribonucleic protein.
					\item \emph{ZBP}: zipcode binding protein.
					\item \emph{RMRP}: fragile mental retardation protein.
				\end{itemize}
			\end{multicols}

			\paragraph{Staufen}
			Staufen \`e una RNA binding protein coinvolta nella localizzazione di Oskar nella parte porteriore dell'oocita di Drosopgila e di bicoid nella parte anteriore della larva.
			Interagisce con RNA formando \emph{RNP}.

				\subparagraph{\emph{mStaufen1}}
				\emph{mStaufen1} \`e una proteina a $63kDa$ nei genitali e fegato nei mamiferi.
				Possiede $4$ domini che legano RNA a doppio flamento ad affinit\`a crescente e un dominio per legarsi ai microtubuli \emph{TBD}.
				Viene espressa nei neuroni: il trasporto con i microtubuli la localizza nei dendriti.
				Contiene RNA.

				\subparagraph{\emph{mStaufen2}}
				\emph{mStaufen2} localizza su un gene diverso, ha $5$ domini e dal suo splicing si generano tre isoforme a $62$, $59$ e $52kDa$.
				Si trova in prossimit\`a della sinapsi dove forma filopodi.
				Interagisce con l'actina.

			\paragraph{Pumulio}
			La famiglia pumilio opera un controllo traduzionale degli RNA.
			Possiede un dominio \emph{PHD} pumilio homology domain che lega RNA.
			\`E generalmente espressa nel cervello e i suoi domini riconoscono sequenze specifiche di RNA.
			Nei mammiferi \`e presente Pumilio $1$ e $2$.

				\subparagraph{\emph{Pum2}}
				\emph{Pum2} si trova associato a microtubuli polarizzati.
				Quando nelle cellule viene aggiunto un inibitore della traduzione, cambia la localizzazione e forma granuli di stress causando la degradazione di RNA nei processing bodies per mezzo di enzimi \emph{DPC1}, \emph{DPC2}.

				\subparagraph{Granuli di stress}
				I granuli di stress sono strutture che si formano quando la cellula subisce stress.
				La cellula imprigiona in essi RNA e impedisce la loro traduzione in modo da risparmiare energie.
				Se lo stress dura poco i granuli vengono degradati, se permane vengono trasportati nei processing bodies per la degradazione.

				\subparagraph{Processing bodies}
				I processing bodies sono regioni della cellula contenti enzimi che rimuovono il cap \emph{DCP1} e \emph{DCP2}.
				Sono inoltre presenti ribosomi, fattori che legano il cap e fattori coinvolti nella stabilit\`a del RNA.
