\chapter{Trasporto e localizzazione di RNA}
Molti mRNA vengono diretti verso regioni intracellulari specifiche prima che inizi la traduzione.
Questo avviene in modo da permettere alla cellula di posizionarli vicino ai siti in cui \`e necessaria la proteina codificata.

\section{Motivi del trasporto di RNA}

	\subsection{Efficienza}
	Il trasporto di un mRNA permette la traduzione locale da un'unica copia di molteplici proteine risparmiando energia.

	\subsection{Sicurezza}
	Il trasporto di un mRNA \`e pi\`u semplice del trasporto in quanto impedisce che le proteine interagiscano tra di loro o con l'ambiente erroneamente.

	\subsubsection{Velocit\`a}
	Il trasporto di un mRNA consente cambiamenti rapidi nella concentrazione delle proteine locali.

	\subsubsection{Controllo della degradazione}
	Il trasporto del mRNA \`e un sistema utile per sottrarre il RNA stesso alla degradazione trasportandolo in un compartimento diverso.

\section{RNA trasportati}

\section{Meccanismo di trasporto}

	\subsection{Fattori coinvolti}

		\subsubsection{Cis-acting elements}
		I cis-acting elements o zipcodes sono fattori che riconoscono sequenze specifiche nella \emph{$5'$-UTR}.
		
			\paragraph{Esempi}
			\begin{multicols}{2}
				\begin{itemize}
					\item \emph{DTE} dendrites targeting elements.
					\item \emph{ALS} axon localization sequence.
					\item \emph{MLS} mitochondria localization sequence.
				\end{itemize}
			\end{multicols}
	
		\subsubsection{Trans-acting factors}
		I trans-acting factors sono proteine che riconoscono li cis-acting elements.
		
			\paragraph{Esempi}
			\begin{multicols}{2}
				\begin{itemize}
					\item Proteine motrici.
					\item \emph{hnRNP}: heterogeneus nuclear ribonucleic protein.
					\item \emph{ZBP}: zipcode binding protein.
					\item \emph{RMRP}: fragile mental retardation protein.
				\end{itemize}
			\end{multicols}

			\paragraph{Staufen}
			Staufen \`e una RNA binding protein coinvolta nella localizzazione di Oskar nella parte porteriore dell'oocita di Drosopgila e di bitcoin nella parte anteriore della larva.
			Interagisce con RNA formando \emph{RNP}.

				\subparagraph{\emph{mStaufen1}}
				\emph{mStaufen1} \`e una proteina a $63kDa$ nei genitali e fegato nei mamiferi.
				Possiede $4$ domini che legano RNA a doppio flamento ad affinit\`a crescente e un dominio per legarsi ai microtubuli \emph{TBD}.
				Viene espressa nei neuroni: il trasporto con i microtubuli la localizza nei dendriti.
				Contiene RNA.

				\subparagraph{\emph{mStaufen2}}
				\emph{mStaufen2} localizza su un gene diverso, ha $5$ domini e dal suo splicing si generano tre isoforme a $62$, $59$ e $52kDa$.
				Si trova in prossimit\`a della sinapsi dove forma filopoli.
				Interagisce con l'actina.

			\paragraph{Pumulio}
			La famiglia pumilio oper un controllo traduzionale degli RNA.
			Possiede un dominio \emph{PHD} pumilio homology domain che lega RNA.
			\`E generalmente espressa nel cervello e i suoi domini riconoscono sequenze specifiche di RNA.
			Nei mammiferi \`e presente Pumilio $1$ e $2$.

			
	
	\subsection{Processo}

\section{Trans acting factors}

	\subsection{Staufen}

	\subsection{Pumilio}




\`E un meccanismo di modifica post trascrizionale, l'RNA pu\`o essere complessato da proteine e fattori che ne impediscono la traduzione e possono trasportarlo in altri compartimenti 
all'interno della cellula. Il cambio delle sinapsi in risposta all'apprendimento \`e determinato da restringere proteine a sinapsi specifiche locaizzando gli mRNA. Le proteine devono
essere trasportate e trasportare mRNA e poi tradurre solo localmente l'RNA producendo proteine in loco che modificano localmente determinati compartimenti. Scoperta da ibridaizone in 
situ. Un esempio di due RNA uno che produce la subunit\`a beta di una chinasi la cui attivit\` adipende dal calcio costiutita da una subunit\`a alpha e una beta che si uniscono tra
di loro e formano l'enzima funzionante che in presenza di calcio fosforila altre molecole. Nell'ippocampo, trasformazione da memoria a breve termine a quella a lungo termine. Questa 
regione dell'ippocampo \`e ricca di cellule nervose. Si nota come l'RNA \`e presente nel corpo cellulare dei neuroni, quello per la subunit\`a alpha \`e presente nei corpi cellulari e 
nei dendriti. Sono stati identificati altri RNA come per la MAP2 (associata ai microtubili per il riarrangiamento) recettori, l'RNA per l'actina (citoscheletro), RNA non codificanti ma
con ruolo strututrale di regolazione (microRNA). Il trasporto e localizzaizone di RNA nella cellula \`e stato studiato in particlare nella drosophila. Come dall'uovo fertilizzato si 
forma l'embrione e infine la drosophila adulta. L'oocita di drosophila possiede cellule nutrici nurse che provedono materiali all'oocita di drosophila che verr\`a fecondato e former\`a
embrione e la drosophila. Il nucleo dell'oocita si trova nella parte superiore. Il fatto che alcuni RNA localizzano in particolari compartimenti \`e stato studiato nella drosophila e 
si nota come oskar localizza nella parte posteriore dell'oocita di drosophila. Oskar viene preso e trasportato nella parte posteriore. Un altro RNA bicoid localizza nella parte anteriore
in uno stato di sviluppo pi\`u tardivo. I nomi vengono dati dal nome di mutanti: bicoid prende il nome da un mutante che se non lo possiede non sviluppa la testa, oskar deriva da un
altro significato e gurken (cetriolo) e assume una forma a cetriolo e cos\`i via. Sono nomi originali che alla cui base si trova una mutazione legata al fenotipo. Bicoid localizza nella
parte anteriore dell'embrione di drosophila. Gurken localizza nella parte apicale. Esempi di localizzazione e trasporto di RNA si trovano anche nel lievito come RNA ASH1 per un fattore 
trascrizionale che passa dalla cellula madre alla figlia quando il lievito si divide per gemmazione. L'RNA per l'actina si trova nella parte anteriore del leading edge dei fibroblasti, 
zona che indica la direzione del movimento, le cellule si muovono in quella direzione e serve alla cellula per sensire il substrato e serve alla cellula per muoversi, ricca in actina
in quanto conferisce al citoscheletro estrema motilit\`a. L'RNA per l'actina tende ad accumularsi in questa zona e lo si trova anche nella parte terminale del cono di crescita e nei 
dendriti. Esempi di localizzazione dell'RNA in compartimenti o in maniera specifica di oocita o embrione sono numerosi. Uno dei modelli utilizzati per lo studio del trasporto di RNA e 
studiare le ripercussioni e le propriet\`a biologiche sono le cellule nervose. Le cellule nervose possono essere isolate o dalla corteccia (colture corticali) o da una zona a forma 
di banana detta ippocampo, una regione molto interna del sistema nervoso centrale e sono quelle pi\`u vecchie e antiche e la si trova nei pesci e rettili ed \`e connessa riceve 
informazioni dal bublo olfattivo e sottolinea l'importanza dell'olfatto nei meccanismi di memoria. Mantenuta la funzione di essere di relay e trasformare memorie a breve termine a lungo
termine. Il paziente HM si tratta di un signore che cadendo dalla bicicletta sub\`i un trauma con attacchi epilettici (inizio 1920) e a seguito svilupp\`o crisi epilettiche ricorrenti
trattate con la rimozione dell'ippocampo. Il paziente a seguito perse la capacit\`a di trasformare memoria a breve termine a lungo termine. Non poteva formare nuove memorie ma riusciva 
a mantenere quelle antecedenti all'operazione. Gli studi successivi confermarono il ruolo della struttura nella trasformazione a breve a lungo termine. \`E possibile farle crescere con
il modello murino (ratto o topo), con bulbi olfattori, due emisferi e il cervelletto. Occorre dividere in due il cervello, aprirlo e rimuovere l'ippocampo, si rimuove il substrato per 
separare le cellule e vengono messe in coltura. Si usa questo modello in quanto ina volta messe in coltura subiscono fasi di sviluppo precise: appena messe in colture la cellula \`e 
simmetrica e dopo un giorno inizia a produrre dei prolungamenti e uno diventa p\`u sottile e cresce di pi\`u diventando l'assone e infine si ha lo sviluppo dei dendriti. Buon modello
per lo studio dei fattori coinvolti nella formazione di assoni, dendriti e sinapsi. Ciascuna modifica pu\`o essere facilmente evidenziata e lo sviluppo facilmente monitorato. Il sistema
\`e in vitro. La strtutura ordinata dell'ippocampo viene persa in vitro e di conseguenza la strettura organizzativa viene persa e inoltre il sistema in vitro \`e bidimensionale e non
tridimensionale. Il sistema in vitro non rispecchia la composizione cellulare in vivo in quanto mancano molte cellule gliali che aiutano i neuroni nella loro attivit\`a. Una coltura
arricchita di neuroni facilita lo studio del loro sviluppo. SI pu\`o trasfettare facilmente usando virus o sistemi di trasferimento delle cellule, la microscopia \`e pi\`u semplcie
e si possono fare misure fisiologiche di singole cellule. Al di l\`a delle limitazioni il sistema in vitro delle colture cellulari presenta vantaggi. \`E un sistema molto utilizzato 
per il trasporto e utile anche per capire come avviene il trasporto e se l'RNA che viene trasportato viene tradotto. Trovare RNA in un determinato compartimento \`e di dimostrare che 
l'RNA va incontro a traduzione. SI \`e cercato di vedere se ci fossero ribosomi nei dendriti e si nota come ci sono in particolare all'interno delle spine dendridiche. In zone molto
distanti dal corpo cellulare ci sono RNA e la macchina dedicata alla loro traduzione. La localizzazione avviene per aumentare l'efficienza in quanto trasportando RNA e traducendola
$n$ vengono prodotte pi\`u proteine localmente. In oltre \`e un sistema sicuro in quanto previene le interazioni tra proteine ectopiche non volute.  Come la mielina basic protein (MBP)
che forma la mielina delle cellule nervose, la guaina di isolamento che permette la trasmissione corretta ed efficiente dell'impulso nervoso. Tende ad interagire molto con le membrane
e trasportarla potrebbe creare problemi. Traducendola localmente si previene questo problema. Un altro motivo per il trasporto \`e la rapidit\`a: volendo modificare la regione della
cellula in maniera rapida si ha gi\`a l'RNA presente e si pu\`o tradurre l'RNA in proteine che modificano direttamente la regione altrimenti si dovrebbe inviare segnale da periferia a 
nucleo, che trascrive RNA che viene tradotto e la proteina trasportata alla periferia. La cellula traspotra l'RNA in periferia per proteggerlo dalla degradazione. La cellula vuole 
mantenere l'RNA e tradurlo in un secondo momento la cellula lo pu\`o degradare o impedire questo spedendolo in un altro compartimento. Il trasporto avviene dal nucleo dal processo di
trascrizione e dopo le modifiche post-trascrizionali viene legato a proteine che legano l'RNA e si formano le RNP (ribonucleoparticles) alcune delle quali lo seguono nel trasporto altre
si separano appena entrano nel citoplasma. L'RNA una volta assemblato va nel citoplasma e si lega a proteine motrici come dineine e chinesine che usando ATP trasportano usando i 
microtubuli. L'RNA e le proteine vengono ancorate e si fermano e quando serve la produzione locale di proteine l'RNA viene tradotto. Non tutti gli RNA vengono trasportati in determinati
compartimenti e per il trasporto specifico servono delle sequenze cis-acting elements normalmente nella $3'$ UTR a valle della zona codificante detti zipcodes e hanno una localizzazione
specifica peculiare tra questi si trova l'MLS (mitochondria localization sequence), DTE (dendritic targeting element), ALS (axonal localization sequence), si noti come i cis acting 
elements sono diversi tra di loro. Oltre a questi elemento servono i trans acting factors, proteine che si legano a RNA sui cis-acting elements, si legano ad essi, reclutano altri 
fattori e li trasportano con le proteine motrici. Guarda microiniezione. La microiniezione pu\`o essere utilizzata anche per seguire il trasporto del trafficking di proteine.
Un altra tecnica per seguire il trasporto di RNA che non prevede la microiniezione \`e un sistema basato sull'MS2, un sistema indiretto per evidenziare il trafficking dell'RNA. Consiste
di due costrutti trasfettati all'interno delle cellule. UNo esprime MS2-GFP caratterizzata da una GFP fusa con MS2 un dominio che lega le strutture a stem loop MS2-binding site che si 
lega a queste strutture a forcina che normalmente vengono clonate a monte dell'RNA di cui si vole tracciare il trafficking. Coesprimendo MS2-GFP e il costrutto dell'RNA la cellula li 
produce entrambi e epoduce la proteina Ms2-GFP si lega al sito di legame a monte dell'RNA. In questo caso come nella microiniezione \`e possibile seguire il tutto in vivo. I trans 
acting factors sono proteine che si legano ai cis acting elements nelle $3'$ UTR con le RNA binding proteins: alcune sono importanti per il trasporto altre importanti per altri aspetti
come la traduzione: \`e importante che gli RNA durante il trasporto non vengano tradotti e alcune di queste sono coinvolte nel mantenere silenti gli RNA come pumilio che legano il RNA
a singolo filamento, ELAV e hnRNPs altre invece sono connesse con il trasporto spesso e se togliendole l'RNA non viene pi\`u trasportato altre sono coinvolte nella stabilit\`a dell'RNA
IL numero \`e elevato. Le proteine Staufen e Pumilio sono state identificate da studi effettuati su drisophila. Staufen \`e responsabile di Oskar nella parte posteriore dell'oocita
e per l'ancoraggio anteriore del bicoide. Staufen \`e una proteina che lega l'RNA a doppio filamento, contiene cinque dsRBD (double stranded binding domains) e dopo di che si identifica
l'omologo nei mammiferi mStau1 pi\`u piccolo rispetto alla drosophila ma con quattro dsRBD e ha un dominio TBD (dominio che lega la tubulina) e permette alla proteina di interagire 
direttamente con i microtubuli, dominio importante per ancorare la proteina e l'RNA al microtubuli. Non  tutti i idomini hanno la stessa affinit\`a per l'RNA 3 maggiore poi 4 e gli altri
lo legano con affinit\`a ridotta. Il resto della proteina \`e diversa in quanto l'omologo nei mammiferi \`e stata utilizzata come regione per identificare il dominio che lega l'RNA con
maggiore conservazione. Si trova diversit\`a tra i domini dsRBD, si trova conservazione tra i domini tra le diverse specie. Per vedere se il trasporto \`e attivo si pu\`o utilizzare il
nocodazolo che disassembla i microtubili utilizzato come antitumorale in quanto responsabile della divisione dei cromosomi durante la divisione cellulare. Si tratta le cellule con 
nocodazolo e vedere se si trovano ancora le RNA nei dendriti, se il trasporto dipende dai microtubuli non si deve pi\`u vedere e questo infatti non succede. Questo esperimento dimostra
che si trova trasoprto attivo basato sui microtubuli e staufen lega gli RNA anche su mammiferi. Non c'\`e un omologo di bicoid o oskar nei mammiferi, ha mantenuto la funzione ma ha
cambiato cargo. Nei mammiferi inoltre ci sono due geni per staufen e il secondo codifica mStau2 pi\`u complessa rispetto a 1 con tre isoforme di 62, 59, 52 kilodalton con cinque 
domini che legano RNA si trova il dominio che lega la tubulina TBD. Questa proteina \`e espressa esclusivamente nel sistema nervoso. La funzione di mStau2 \`e presente lungo l'asse 
dendritico e si trova vicino alle sinapsi in cui il dendrita riceve informazioni da altri assoni. Le due proteine sono in prossimit\`a l'una all'altra e una domanda \`e cosa succede
quando si toglie l'espressione di stau2, per fare questo si usa l'approccio di utilizzare la tecnica di short interference RNA di inserire all'interno della cellula degli RNA a forcina
che vengono modificati dalla cellula e producono degli RNA molto piccoli che legandosi all'mRNA ne bloccano la traduzione. Si pu\`o prendere utilizzare questa tecnica per inibire in 
maniera specifica mStau1. Una tecnica precisa e potente per inibire l'espressione di una proteina molto simile con un'altra. Si nota come i neuroni le spene dendritiche senza staufen 1 e 
2 non hanno strutture a fungo ma scompaiono e rimangono delle strutture filopoidiali e non rassomigliano alle spine dendridiche mature. La rimozione di Stauf2 va a eliminare le sinapsi. 
Questo succede in quanto una volta tolta la proteina si osserva una modifica a livello strutturale e funzionale delle sinapsi, cosa legata ad un riarrangiamento del citoscheletro: quando
si vuole modificare una forma si modifica il citoscheletro sottostante. Si osserva l'actina e due neuroni, uno che non espime stau2 e l'altro che la esprime si vede come rispetto al
controllo il neurone senza ha meno actina e nei dendriti ci sono zone senza actina, soprattutto delle sinapsi e delle spine dendridiche, la sua rimozione abolisce il pattern dell'actina,
uno degli RNA che localizza nei leading edge e nei dendriti. Ci si \`e chiesti in che modo l'assenza di staufen 2 va a interferire con l'espressione dell'actina e con l'RNA che codifica
per la $\beta$ actina andando a vedere l'RNA per l'actina in neuroni senza staufen 2 e si \`e notato come stau2 sembra coinvolta nel traspotro di RNA, in realt\`a \`e coinvolta nella
stabilit\`a in quanto dowunregolando stau2 e monitorando i livelli di actina e si nota come i livelli dell'RNA di actina presente nella cellula in condizioni normali rimane costante nel
tempo per almeno sei ore, invece la stabilit\`a e i livelli di actina crollano del $60\%$ dopo circa $6$ ore. In drosophila staufen \`e coinvolta invece nel trasporto e nel controllo
traduzionale. Durante il trasporto \`e importante che l'RNA non venga tradotto, ci sono fattori che interagiscono con l"RAN durante il trasporto con il ruolo di inibitori traduzionali, 
sono state caratterizzate proteine con questa funzione grazie a studi sul lievto come ASH1, ci sono mutanti di pumilio in cui ASH1 non viene trasportato e viene tradotto prematuramente 
all'interno della cellula madre con mutaizoni per il gene mutiliio e l'RNA ASH1 non viene trasportato nel figlio e rimane nella cellula madre e viene tradotto. Pu\`o essere pertanto 
un buon candidato. Ci sono omologhi nei mammiferi e sono Pum1 e Pum2, il lievito ne ha 6, nelle varie specie si tratta di una famiglia conservata con un dominio PHD (pumilio homology 
domain) con otto domini strutturati ripetizioni che legano l'RNA a singolo strand in base a una sequenza specifica e conoscendola \`e possibile fare un'analisi bioinformatica andare a 
vedere quali RNA contengono la sequenza specifica e vedere quali possono essere controllati da questa proteina. Pumulio nei mammiferi \`e associato a RNA nei mammiferi che loclaizzano
e svolge un ruolo di controllo traduzionale. Gli RNA che non la legano vengono tradotti di pi\`u. Tra l'altro pumulio \`e espresso in drosophila ed \`e upregolato durante l'apprendimento
e la memoria durante questi organismi. L'apprendimento per le drosophile lo si fa attraverso uno strumento con due camere con due odori che sono riconosciuti allo stesso modo dalla 
drosophila (preferiti allo stesso modo). A questo punto si rifa l'esperimento dando shock elettrico nella carica con l'odore 1 e a quel punto le drosophile imparano che l'odore 1 \`e 
associato ad uno shock e tendono ad andare nella camera con l'odore 2. Questo sistema permette di insegnare alle drosophile e quando hanno appreso si possono isolare i neuroni e vedere 
quali geni sono stati attivati durante questo meccanismo e tra i geni upregolati che aumentano a seguti dell'esperimento si trovano pumilio e staufen. La loro importanza \`e legata alla
funzione di trasporto e controllo di traduzione dell'RNA. 
