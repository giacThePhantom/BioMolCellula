\chapter{Chimica e bioenergetica della cellula}
\section{Le componenti chimiche della cellula}
Gli organismi viventi sono composti da un piccolo sottoinsieme di elementi: carbonio (C), idrogeno (H), azoto (N) e ossigeno (O) formano il $96.5\%$ del peso della cellula. Gli atomi di
questi elementi sono legati da legami covalenti in modo da formare molecole in quanto sono pi\`u resistenti delle energie termiche all'interno della cellula e sono rotti solo durante
specifiche reazioni con altri atomi e molecole. Due molecole diverse possono essere tenute insieme da legami non covalenti molto pi\`u deboli.
\subsection{L'acqua \`e mantenuta da legami a idrogeno}
Le reazioni all'interno della cellula avvengono in ambiente acquoso, pertanto la vita si basa sulle propriet\`a chimiche dell'acqua. In ogni molecola d'acqua (\ce{H20}) i due atomi di 
H sono legati all'atomo O da due legami covalenti altamente polari, pertanto si trova una distribuzione ineguale di elettroni che causa una regione carica positivamente verso gli atomi H
e negativamente verso l'\ce{O}. Quando una parte carica positivamente si avvicina a una negativa si formano legami a idrogeno, molto meno forti di quelli covalenti e facilmente rotti 
dall'energia termica delle molecole. Questi legami durano pertanto un periodo breve. Questi legami sono responsabili dello stato liquido dell'acqua, dell'alta tensione superficiale e 
punto di ebollizione. Alcune molecole come gli alcoli che possiedono legami polari possono formare legami a idrogeno con l'acqua si dissolvono facilmente in acqua e sono chiamate 
idrofile (zuccheri, DNA, RNA e la maggior parte delle proteine). Le molecole idrofobiche invece sono apolari e non formano legami a idrogeno e pertanto non si dissolvono nell'acqua, un
importante esempio sono gli idrocarburi, in cui gli H sono legati con gli atomi di C attraverso legami non polari, questa propriet\`a \`e sfruttata dalle cellule le cui membrane sono 
costruite da molecole con lunghe catene idrocarburiche.
\subsection{Quattro tipi di attrazione non covalente aiuta a unire le molecole nelle cellule}
Molto della biologia dipende dagli specifici legami causati da legami non covalenti: attrazione elettrostatica (legami ionici), legami a idrogeno e attrazioni di van der Waals e un 
quarto fattore che \`e la forza idrofobica. Nonostante ognuna di queste forze da sola sarebbe troppo debole per essere efficace si sommano tra loro in modo da creare una forte attrazione
tra due molecole separate. Si noti anche come formando un'interazione competitiva con queste molecole l'acqua riduca fortemente la forza delle attrazioni elettrostatiche e dei legami a 
idrogeno.
\subsection{Alcune molecole polari formano acidi e basi in acqua}
Una delle reazioni chimiche pi\`u significative nella cellula occorre quando una molecola con un legame covalente altamente polare tra un idrogeno e un altro atomo si dissolve in acqua.
Tale idrogeno ha quasi completamente perso il proprio elettrone e pertanto esiste quasi come nucleo di idrogeno caricato positivamente (\ce{H+}). Quando la molecola polare viene 
circondata da molecole d'acqua il protone viene attratto dalla loro carica parzialmente negativa e si pu\`o dissociare dalla molecola originale formando uno ione idronio (\ce{H3O+}). La 
reazione inversa accade molto velocemente, pertanto in soluzione acquosa i protoni continuano a spostarsi tra una molecola e l'altra. Le sostanza che compiono questa reazione sono dette 
acidi e maggiore la concentrazione di \ce{H3O+}, pi\`u acida la soluzione. Questo ione risulta presente anche in acqua pura a causa del continuo movimento di protoni, in una 
concentrazione $10^{-7} M$. Per convenzione la concentrazione di \ce{H3O+} \`e riferita come la concentrazione di \ce{H+} e espressa utilizzando la scala del pH, logaritmica. L'acqua 
pura ha un valore di $7$ ed \`e detta neutra. Per valori di pH maggiori di $7$ \`e detta basica, per valori minori detta acida. Gli acidi si caratterizzano in forti o deboli in base a 
quanto facilmente perdono i protoni in acqua. Molti degli acidi importanti per la cellula sono deboli. A causa dell'effetto sulla natura delle molecole dei protoni liberi l'acidit\`a 
all'interno della cellula deve essere regolata. L'opposto di un acido \`e una base, molecole che accettano un protone in soluzione acquosa, ancora una volta nelle cellula sono presenti 
per la maggior parte basi deboli. Acidi e basi hanno azioni contrastanti e tendono ad annullare reciprocamente il loro effetto, pertanto l'interno della cellula \`e mantenuto vicino 
alla neutralit\`a da buffer, acidi e basi deboli che tendono a compiere scambi di protoni ad un pH vicino a $7$, mantenendo l'ambiente della cellula costante. 
\subsection{La cellula \`e formata da composti di carbonio}
Senza considerare l'acqua e gli ioni inorganici come il potassio, la maggior parte delle molecole nella cellula sono basate sul carbonio, un atomo con la capacit\`a di formare grosse
molecole. Siccome il carbonio \`e piccolo e ha possiede quattro elettroni liberi nel livello esterno pu\`o formare legami covalenti con altri atomi e con s\`e stesso, in modo da formare
catene ed anelli in modo da generare molecole grandi e complesse. I composti del carbonio sono detti molecole organiche. Alcune combinazioni di atomi compaiono ricorrentemente nella
cellula e ognuno di questi gruppi possiede propriet\`a chimiche e fisiche proprie che influiscono sul loro comportamento. 
\subsection{Le cellule contengono quattro principali famiglie di piccole molecole organiche}
Le molecole della cellula sono basate sul carbonio e hanno pesi molecolari tra $100$ e $1000$ e contengono circa $30$ atomi di carbonio. Sono solitamente trovate libere in soluzione.
Alcune sono usate come monomeri per costruire macromolecole polimeriche, altre agiscono come fonti di energia e sono divise e trasformate in altre piccole molecole con pi\`u ruoli nella
cellula. Sono molto meno presenti rispetto alle macromolecole. Tutte le molecole organiche sono sintetizzate e divise dallo stesso insieme di componenti, pertanto i composti nella 
cellula sono chimicamente simili e possono essere classificati per la maggior parte in zuccheri, acidi grassi, nucleotidi e amminoacidi. 
\subsection{La chimica della cellula \`e dominata da macromolecole con propriet\`a notevoli}
Le macromolecole sono le molecole organiche pi\`u abbondanti per peso nella cellula. Sono le strutture principali per la costruzione e la cellula e definiscono le propriet\`a degli 
organismi viventi. Le macromolecole nella cellula sono polimeri che sono costruiti legando covalentemente piccole molecole organiche (monomeri) in lunghe catene. Le proteine sono 
abbondanti e versatili, alcune servono da enzimi, che catalizzano tutte le reazioni interne alla cellula, altre hanno funzione strutturale, o per compattare il DNA nei cromosomi, altre
ancora agiscono come produttrici di forza motile. Nonostante le reazioni chimiche per la formazione di polimeri varino tra proteine, acidi nucleici e polisaccaridi in tutte la 
molecola cresce grazie all'addizione di un monomero in una fine di una catena in una reazione di condensazione, in cui una molecola di acqua \`e persa con ogni subunit\`a aggiunta. 
Questa operazione richiede gli stessi enzimi per tutta la molecola ed \`e pertanto facilmente serializzabile. Tranne i polisaccaridi i monomeri che formano le macromolecole esistono
in diverse varianti e richiedono pertanto una sequenza precisa di addizione per formare la macromolecola corretta. 
\subsection{I legami non covalenti specificano la forma di una macromolecola e i suoi legami con le altre molecole}
La maggior parte dei legami covalenti nella macromolecola permettono una rotazione degli atomi che legano, permettendo grande flessibilit\`a garantendo ad essa un gran numero di 
conformazioni possibili quando l'energia termica causa rotazioni. Nonostante questo la maggior parte delle macromolecole biologiche sono altamente costrette a causa di un gran numero di
legami non covalenti che si formano tra diverse parti della stessa molecola e causano la macromolecola in una conformazione particolare, determinata dalla sequenza lineare dei monomeri.
Questo accade nella maggior parte delle proteine e in molte delle piccole molecole di RNA. Questi legami non covalenti possono anche creare forte attrazione tra molecole diversi in modo
da creare un'interazione molecolare con alta specificit\`a e con vari gradi di affinit\`a, permettendo rapida dissociazione dove necessario. Questo processo \`e fondamentale per tutte le
catalisi biologiche, permettendo il loro funzionamento come enzimi e per la creazione di strutture cellulari complesse. 
\section{Catalisi e utilizzo dell'energia da parte delle cellule}
Una delle principali differenze tra organismi viventi e non viventi \`e che i primi creano e mantengono ordine. Per farlo necessitano di operare un insieme di reazioni chimiche in cui 
alcune molecole sono separate per mettere a disposizione altre molecole per costruirne altre.
\subsection{Il metabolismo della cellula \`e organizzato dagli enzimi}
Le reazioni chimiche che una cellula performa accadrebbero normalmente unicamente ad alte temperature, pertanto ogni reazione richiede un'accelerazione specifica nella sua reattivit\`a.
Questo fatto permette alla cellula di controllare la sua chimica. Il controllo \`e operato da catalizzatori biologici specializzati, proteine detti enzimi o RNA detto ribosomi. Ogni
enzima catalizza una delle possibili reazioni che sono connesse in serie in modo che il prodotto di una sia il substrato di un'altra. Questi cammini lineari sono legati uno con l'altro,
formando un insieme di reazioni interconnesse che permettono alla cellula di sopravvivere, crescere e riprodursi. Esistono due principali flussi di reazioni chimiche: quelle cataboliche
che separano i nutrienti in piccole molecole, generando energia e alcune  piccole molecole fondamentali e anaboliche o biosintetiche in cui le piccole molecole e l'energia vengono
utilizzate per guidare la sintesi delle molecole che formano la cellula. Insieme costituiscono il metabolismo della cellula. 
\subsection{L'ordine biologico \`e reso possibile dal rilascio di calore dalla cellula}
Le cellule devono ridurre il proprio livello di entropia e pertanto deve recuperare energia dall'ambiente sotto forma di cibo o fotoni, che viene poi utilizzata per generare l'ordine
necessario. Nel corso di queste reazioni una parte dell'energia utilizzata viene trasformata in calore. L'energia, nel caso delle cellule animali viene ottenuta rompendo i legami dei 
nutrienti e viene trasformata in energia termica. La cellula non pu\`o beneficiare del calore rilasciato a meno che queste reazioni che generano energia siano accoppiati direttamente
con i processi che generano l'ordine molecolare. 
\subsection{Le cellule ottengono energia ossidando molecole organiche}
Tutte le cellule animali e vegetali utilizzano l'energia conservata in legami chimici di molecole organiche, sia che siano zuccheri sintetizzati dalla fotosintesi sia che siano ottenuti
mangiando. L'energia \`e stratta da un processo di ossidazione graduale. L'atmosfera contiene molto ossigeno e in presenza di ossigeno la forma di carbonio pi\`u stabile \`e la $CO_2$ e
quella dell'idrogeno \ce{H2O}. Una cella \`e pertanto capace di ottenere energia permettendo a carbonio e idrogeno delle molecole di combinarsi con l'ossigeno per produrre \ce{CO_2} e 
\ce{H_2O}. Questo processo \`e chiamato respirazione aerobica. La fotosintesi e la respirazione sono processi complementari. Si nota pertanto come l'utilizzo di carbonio formi un grande 
ciclo che coinvolge l'intera biosfera. 
\subsection{Ossidazione e riduzione coinvolgono il trasferimento di elettroni}
L'ossidazione nella cellula avviene attraverso l'uso di enzimi in cui il metabolismo prende le molecole attraverso un numero di reazioni che raramente coinvolgono la diretta addizione 
di ossigeno. L'ossidazione si riferisce a quel processo in cui elettroni sono trasferiti da un atomo all'altro (il processo inverso \`e la riduzione). Essendo che il numero di elettroni
deve essere conservato durante una reazione ossidazione e riduzione accadono contemporaneamente: una molecola guadagna un elettrone e un'altra lo perde. Questi termini si riferiscono
anche a un parziale spostamento di elettroni in un legame covalente: quando se ne crea uno polare l'atomo dalla parte del delta positivo acquisici una parziale carica positiva ed \`e
detto ossidato. Quando una molecola recupera un elettrone recupera anche un protone allo stesso momento e l'effetto netto \`e l'addizione di un atomo di idrogeno alla molecola:
$$\ce{A + e- + H+ -> AH}$$
Queste reazioni, dette di idrogenazione sono riduzione, mentre quelle inverse, di deidrogenazione sono dette ossidazioni. In una molecola organica avviene un'ossidazione quando il numero
di legami C-H diminuisce, una riduzione quando aumenta. Le cellule utilizzano gli enzimi per catalizzare le ossidazioni attraverso una sequenza di reazioni che permettono il raccolto
dell'energia prodotta. 
\subsection{Gli enzimi abbassano la barriera di energia di attivazione che blocca la reazione chimica}
Si noti come le reazioni chimiche procedono spontaneamente unicamente nella direzione che porta alla perdita di energia libera (energicamente favorevoli). Essendo che le molecole negli
esseri viventi si trovano in uno stato energetico relativamente stabile \`e necessario, affinch\`e una reazione inizi di un'energia di attivazione, creato da collisioni causali
insolitamente energetiche, che diventano violente maggiore \`e l'energia. La chimica di una cellula \`e altamente controllata e il superamento del livello di energia \`e svolto da 
enzimi che si legano con un'altra molecola (substrato) in modo da ridurre l'energia di attivazione necessaria per la reazione. Questi enzimi sono detti catalizzatori e aumentano il 
tasso delle reazioni chimica in quanto permettono maggiori collisioni casuali con le molecole circostanti. 
\subsection{Gli enzimi possono guidare i substrati lungo specifici cammini di reazioni}
Un enzima non pu\`o cambiare il punto di equilibrio per una reazione in quanto aumenta anche il tasso della reazione inversa. Nonostante questo sono capaci di guidare le reazioni verso
un cammino specifico in quanto sono altamente selettivi e molto precisi, catalizzando un'unica reazione, pertanto ogni enzima selettivamente abbassa l'energia di attivazione di una delle
reazioni chimiche possibili  che il substrato pu\`o svolgere. In questo modo insiemi di enzimi possono direzionare ognuna delle molecole lungo cammini specifici. Ogni enzima possiede un
sito attivo, uno spazio in cui solo particolari substrati possono legarsi e dopo la reazione rimangono invariati e possono pertanto funzionare pi\`u volte. 
\subsection{Come gli enzimi trovano i loro substrati, la rapidit\`a dei movimenti molecolari}
Un enzima pu\`o catalizzare la reazione di migliaia di substrati al secondo. L'attacco veloce \`e possibile perch\`e il movimento causato dal calore \`e estremamente veloce. Questi 
movimenti molecolari sono classificati in:
\begin{itemize}
	\item Movimento traslatorio: il movimento della molecola da un posto all'altro.
	\item Vibrazioni: il movimento di atomi legati da legami covalenti tra di loro.
	\item Rotazioni.
\end{itemize}
Questi movimenti aiutano ad unire le superfici di molecole che interagiscono. Il movimento delle molecole causa il processo di diffusione e in questo modo ogni molecola collide con un
grano numero di altre molecole al secondo. La distanza netta alla fine di una passeggiata casuale \`e proporzionale alla radice del tempo impiegato. Essendo che gli enzimi si muovono 
pi\`u lentamente dei substrati si possono considerare fermi. Il tasso di incontro dell'enzima con il suo substrato dipende alla concentrazione dell'ultimo. Una collisione del substrato 
con il sito attivo causa immediatamente la creazione del sistema enzima-substrato. L'alta specificit\`a \`e data dal fatto che una forma errata causa legami covalenti pi\`u deboli 
dell'agitazione termica. 
\subsection{Il cambio di energia libera di una reazione determina se pu\`o avvenire spontaneamente}
Nonostante gli enzimi velocizzino le reazioni non possono forzare reazioni energeticamente sfavorevoli. Questo tipo \`e per\`o necessario per alcune operazioni della cellula, pertanto
gli enzimi accoppiano reazioni energeticamente favorevoli in modo da produrre energia che viene utilizzata per le reazioni sfavorevoli e produrre ordine biologico. Il livello di energia
libera (G) esprime l'energia disponibile per fare lavoro e viene presa in considerazione quando il sistema subisce un cambiamento ($\Delta G$), critico in quanto \`e una diretta misura
della quantit\`a di disordine creata nell'universo da una reazione. Reazioni energeticamente favorevoli hanno un $\Delta G$ negativo, quelle favorevoli positivo e possono avvenire 
unicamente se accoppiate con reazioni favorevoli tale che il $\Delta G$ totale rimanga negativo. La concentrazione dei reagenti influenza il cambio di energia libera e la direzione di 
una reazione. A causa di questo per comparare le reazioni si deve utilizzare il cambio di energia libera standard o $\Delta G^{\circ}$, definito alla concentrazione per reagenti di
$1\frac{M}{L}$. Per una reazione \ce{Y -> X} a $37^\circ C$, $\Delta G^\circ$ \`e in relazione a $\Delta G$ come:
$$\Delta G = \Delta G^\circ + RT\ln\dfrac{[X]}{[Y]}$$
Dove $R$ \`e la costante dei gas reali e $R$ la temperatura assoluta. Si noti come al procedere della reazione il rapporto tra i reagenti cambia e avvicina $\Delta G$ a zero, dove si 
raggiunge l'equilibrio chimico e non esiste un cambio di energia libera per guidare la reazione in nessuna direzione e pertanto il rapporto di prodotto e substrato raggiunge un valore
costante $K$ detta costante di equilibrio. I $\Delta G$ delle reazioni accoppiate sono additivi, pertanto una reazione sfavorevole pu\`o essere guidata da una favorevole nel caso la 
seconda segua la prima. 
\subsection{Molecole vettore attivate sono essenziali per la biosintesi}
L'energia rilasciata dall'ossidazione delle molecole deve essere temporaneamente conservata prima che possa essere canalizzata nella sintesi. Nella maggior parte dei casi questo avviene
come energia chimica di legame in un sottoinsieme di molecole dette molecole vettore con dei legami covalenti ricchi di energia. Queste molecole si diffondono rapidamente all'interno 
della cellula trasportando l'energia. Questi vettori attivati conservano energia in una forma facilmente scambiabile sotto-forma di un gruppo chimico o elettrone mantenuto ad un livello
energetico alto e pu\`o servire sia come fonte energetica che di materiali. Sono detti anche coenzimi. Esempi di queste molecole sono ATP, NADH e NADPH. La formazione di un vettore 
attivo \`e accoppiata con una reazione energeticamente favorevole come ossidazione: la quantit\`a di calore rilasciata viene ridotta di una quantit\`a pari a quella conservata nei 
legami, sufficiente a iniziare un'altra reazione.
\subsection{L'ATP \`e il vettore attivo pi\`u utilizzato}
L'ATP \`e la molecola utilizzata per conservare e utilizzare l'energia. \`E sintetizzato in una reazione di fosforilazione in cui un gruppo fosfato \`e aggiunto all'ADP (adenina 
disfosfato). Quando richiesto l'ATP dona il suo pacchetto di energia attraverso l'idrolisi in ADP e fosfato inorganico. L'ADP generato ritorna poi disponibile per la fosforilazione. 
Questa reazione \`e accoppiata con molte altre di sintesi. Una tipica reazione biosintetica \`e una in cui due molecole $A$ e $B$ sono unite per produrre \ce{A-B} durante la 
condensazione: 
$$\ce{A-H + B-OH -> A-B + H2O}$$
Esiste un cammino indiretto in cui un accoppiamento con l'idrolisi dell'ATP causa la reazione di avvenire: l'idrolisi viene utilizzata per convertire \ce{B-OH} in un composto 
intermedio pi\`u energetico che poi interagisce direttamente con \ce{A-H} per formare \ce{A-B}. Il meccanismo pi\`u semplice coinvolge il trasferimento di un fosfato dall'ATP a \ce{B-OH}
per creare \ce{B-O-PO3}:
\begin{align*}
	1. & \ce{B-OH + ATP  ->  B-O-PO_3 + ADP}\\
	2. & \ce{A-H + B-O-PO_3 -> A-B + P_i}\\
	Net\ result: & \ce{B-OH + ATP + A-H  ->   A-B + ADP + P_i}\\
\end{align*}
La reazione di condensazione \`e pertanto forzata dall'idrolisi in un cammino catalizzato da un enzima. 
\subsection{NADH e NADPH sono vettori di elettroni}
Altri vettori che partecipano nelle reazioni di ossidazione-riduzione sono parte delle reazioni accoppiate. Questi vettori sono specializzati nel trasportare elettroni mantenuti ad
alta energia e atomi di idrogeno, I principali \`e \ce{NAD+} (dinucleotide adenina nicotinammide) e \ce{NADP+} (dinucleotide adenina nicotinammide fosfato) che raccolgono un un pacchetto
di energia corrispondente a due elettroni e un protone \ce{H+} e sono convertiti in \ce{NADH} (dinucleotide adenina nicotinammide ridotto) e \ce{NADPH} (dinucleotide adenina 
nicotinammide fosfato ridotto) che possono pertanto essere considerate vettori di ioni idruri. Il NADPH \`e prodotto durante un insieme di reazioni cataboliche che producono energia 
due atomi di idrogeno sono rimossi da una molecola di substrato. Entrambi gli elettroni ma non un \ce{H-} sono aggiunti all'anello di \ce{NADP+} per formare \ce{NADPH}, mentre il protone
\ce{H+} \`e rilasciato in soluzione. Il \ce{NADPH} rilascia facilmente lo ione idruro in una reazione di ossidazione-riduzione in modo da raggiungere uno stato pi\`u stabile ottenendo
un rilascio di energia libera negativa. Il gruppo fosfato del \ce{NADPH} d\`a alla molecola una forma completamente diversa rispetto al \ce{NADH}, rendendole riconoscibili a enzimi 
completamente diversi in quanto si rende necessario regolare due reazioni di trasferimento di elettroni indipendentemente. Il \ce{NADPH} opera con gli enzimi che catalizzano le reazioni 
anaboliche di sintesi per le molecole ricche di energia, mentre il \ce{NADH} opera come intermedio nel sistema di reazioni cataboliche che generano \ce{ATP}. La genesi delle due molecole
avviene essa stessa indipendentemente in modo da avere un controllo fine sul rapporto tra \ce{NAD+} e \ce{NADH} (mantenuto alto) e tra \ce{NADP+} e \ce{NADPH}, mantenuto basso. 
\subsection{Esistono molte altre molecole vettori nella cellula}
Altri vettori attivati possono trasportare un gruppo in un legame ad alta energia e instabile. Il resto della molecola, solitamente la parte pi\`u grande permette il riconoscimento da
parte di enzimi. Molte di queste parti contengono un nucleotide (solitamente adenosina). Si noti come \ce{ATP} trasferisce fosfato, \ce{NADPH} elettroni e idrogeno. Altri vettori 
attivati trasferiscono gruppi utilizzati per la biosintesi e sono generati in reazioni accoppiati con l'idrolisi dell'\ce{ATP}.
\subsection{La sintesi di polimeri biologici \`e guidata dall'idrolisi dell'\ce{ATP}}
Le molecole della cellula sono composte da subunit\`a unite in una reazione di condensazione, in cui i costituenti di una molecola d'acqua (\ce{OH} e \ce{O}) sono rimossi dai reagenti.
Conseguentemente l'azione inversa, la rottura di tutti e tre i tipi di polimeri avviene attraverso idrolisi catalizzata, energeticamente favorevole. Acidi nucleici, proteine e 
polisaccaridi sono polimeri prodotti dall'addizione ripetuta di monomeri a un capo di una catena che cresce. La condensazione in ogni passaggio richiede energia data dall'idrolisi di un
nucleoside trifosfato. Per ogni tipo di macromolecola esiste un cammino catalizzato: il gruppo \ce{-OH} che viene rimosso nella reazione di condensazione \`e prima attivato legandosi
con una molecola che lo porta ad uno stato energetico elevato, attraverso una serie di intermedi ad alta energia. Ogni vettore attivato ha dei limiti nella sua capacit\`a di guidare una
reazione sintetica: il $\Delta G$ dell'idrolisi dell'\ce{ATP} dipende dalla concentrazione dei reagenti, ma si trova tra $-46$ e $-54\frac{kJ}{M}$ e potrebbe pertanto guidare una 
reazione con $\Delta G = +40\frac{kJ}{M}$ che non \`e abbastanza per alcune reazioni. In questi casi l'idrolisi viene alterata in modo che produca \ce{AMP} e pirofosfato (\ce{PP_i}) che
si idrolizza in un passo successivo. L'intero processo crea un $\Delta G = -100\frac{kJ}{M}$. Una reazione che avviene in questo modo \`e la sintesi degli acidi nucleici dal nucleoside
trifosfato. La condensazione ripetitiva pu\`o essere orientata verso la testa o verso la coda. Nella head polymerization, il legame reattivo \`e trasportato alla fine del polimero 
crescente e deve pertanto essere rigenerato ogni volta che un monomero viene aggiunto e il monomero trasporta con s\`e il legame che viene utilizzato per il prossimo monomero. Nella 
tail polymerization il legame \`e trasportato da ogni monomero. 
\section{Come le cellule ottengono energia dai nutrienti}
La riserva costante di energia necessaria alla generazione e mantenimento dell'ordine delle cellule \`e ottenuta dai legami chimici energetici nelle molecole dei nutrienti. Proteine, 
lipidi e polisaccaridi che li compongono sono rotti in piccole molecole prima che la cellula possa usarle attraverso digestione enzimatica e successivamente entrano il citosol della 
cellula, dove avviene una graduale ossidazione. Gli zuccheri sono un'importante fonte di energia e vengono ossidati in passi controllati in anidride carbonica (\ce{CO2}) e acqua. 
\subsection{La glicolisi \`e un cammino centrale per produrre \ce{ATP}}
Il processo principale per l'ossidazione degli zuccheri \`e la glicolisi, una serie di reazioni che produce \ce{ATP} senza ossigeno molecolare. Avviene nel cytosol e include molti 
microorganismi anaerobici. Durante la glicolisi una molecola di glucosio \`e convertita in due molecole di piruvato a tre atomi di carbonio. Per ogni molecola di glucosio sono 
idrolizzate due molecole di \ce{ATP} per fornire energia nei primi passaggi, ma alla fine sono prodotte quattro molecole di \ce{ATP}, con un guadagno netto di due \ce{ATP} e di 
\ce{NADH}. La glicolisi coinvolge $10$ reazioni separate in ognuna delle quali viene prodotto uno zucchero intermedio e catalizzata da un enzima diverso. L'ossidazione avviene 
rimuovendo elettroni grazie a \ce{NAD+} producendo \ce{NADH} dal carbonio derivato dalla molecola di glucosio. La natura a passaggi rilascia l'energia in piccoli pacchetti che possono 
essere salvati in un vettore attivo. Alcuna dell'energia rilasciata guida la sintesi di \ce{ATP} da \ce{ADP} e \ce{P_i} e altra rimane negli elettroni nel vettore \ce{NADH}. Negli 
organismi aerobici le molecole di \ce{NADH} donano gli elettroni a una catena di trasporto e il \ce{NAD+} viene utilizzato ancora per la glicolisi. 
\subsection{La fermentazione produce \ce{ATP} in assenza di ossigeno}
Per la maggior parte delle cellule la glicolisi \`e solo un preludio al passaggio finale della rottura dei nutrienti: il piruvato \`e trasportato ai mitocondri, dove \`e convertito in 
\ce{CO2} e atecile \ce{COA}, il cui gruppo acetile \`e successivamente ossidato in \ce{CO2} e \ce{H2O}. In contrasto, per molti organismi anaerobici la glcolisi rimane la fonte 
principale di \ce{ATP}. In queste condizioni anaerobiche il piruvato e gli elettroni del \ce{NADH} rimangono nel cytosol. Il piruvato \`e convertito in prodotti secreti dalla cellula 
come etanolo e \ce{CO2} nei lieviti o in acido lattico nei muscoli. In questo processo il \ce{NADH} libera i suoi elettroni ed \`e convertito in \ce{NAD+} per mantenere la reazione di
glicolisi. Questi processi sono chiamati fermentazioni. 
\subsection{La glicolisi mostra come gli enzimi accoppiano l'ossidazione alla conservazione di energia}
Due reazioni centrali della glicolisi (passi 6 e 7) convertono lo zucchero intermedio gliceraldeide 3-fosfato in 3-fosfogliceraldeide, ossidndo un gruppo aldeide in un acido 
carbossilico. La reazione completa rilascia energia libera per convertire una molecola di \ce{ADP} in una di \ce{ATP} e per traferire due elettroni e protoni dall'aldeide al \ce{NAD+}, 
contemporaneamente liberando abbastanza calore da rendere la reazione energeticamente favorevole con $\Delta G = -12.5\frac{kJ}{M}$. La reazione \`e guidata da due enzimi che si legano
agli zuccheri intermedi. Il primo enzima (gliceraldeide 3-fosfato deidrogenasi) formano un legame covalente temporaneo con l'aldeide attraverso un gruppo \ce{-SH} e catalizza la sua
ossidazione attravero \ce{NAD+} nel suo sito attivo. Il legame substrato-legame \`e successivamente rotto da uno ione fosfato per produrre un intermedio ad alta energia che viene 
rilasciato dall'enzima. Questo secondo intermedio si lega al secondo enzima fosfoglicerato chinasi che catalizza il trasferimento del fosfato all'\ce{ADP}, permando l'\ce{ATP} e 
completando il processo di ossidazione dell'aldeide in acido carbossilico. La rottura del legame fosfato crea l'energia necessaria per la sintese dell'\ce{ATP}. 
\subsection{Gli organismi conservano le molecole nutrienti in serbatoi speciali}
Tutti gli organismi devono mantenere un rapporto \ce{ATP}-\ce{ADP} alto per rimanere in vita. Per compensare a lunghi periodi senza accesso a nutrienti gli animali conservano gli acidi
grassi come goccioline di grasso composte di trigliceridi insolubili principalmente nel citoplasma di cellule di grasso specializzate (adipociti). Per conservazione a breve termine lo 
zucchero \`e conservato come glicogeno, la cui sintesi e degradazione sono rapidamente regolate al bisogno. Quando le cellule necessitano di pi\`u \ce{ATP} di quanto riescano a produrre
dai nutrienti nel sangue degradano il glicogeno in glucosio 1-fosfato che viene convertito in glucosio 6-fosfato per la glicolisi. Quantitativamente il grasso \`e pi\`u presente del
glicogeno in quanto \`e pi\`u efficiente in quanto libera pi\`u energia ed \`e insolubile in acqua. Lo zucchero e l'\ce{ATP} necessario alle piante \`e prodotto in organelli separati
come i cloroplasti per la fotosintesi e i mitocondri per l'\ce{ATP}. Essendo questi organelli isolati da una membrana e mancanti in alcune cellule, gli zuccheri sono esportati dai 
cloroplasti ai mitocondri di tutte le ccellule. Durante i periodi di eccesso di capacit\`a fotosintetica i cloroplasti convertono dello zucchero in grassi e amido, un polimero del
glucosio analogo al glicogeno. I grassi nelle piante sono trigliceridi e differiscono unicamente nel tipo di acidi grassi che predominano. Entrambi sono conservati nei cloroplasti. 
Quando il glucosio raggiunge un livello soglia negli animali i trigliceridi sono idrolizzati per produrre acidi grassi e glicerolo e sono trasferiti nel flusso sanguigno. Gli acidi 
grassi sono ossidati direttamente.
\subsection{Zuccheri e acidi grassi sono degradati in acetile \ce{CoA} nei mitocondri}
Nei metabolismi aerobici il piruvato prodotto dalla glicolisi nel citosol \`e trasportato nei mitocondri, dove \`e decarbossilato da un complesso di tre enzimi detto il complesso 
piruvato deidrogenasi. Il prodotto sono una molecola di \ce{CO2}, una di \ce{NADH} e acetil \ce{CoA}. Gli acidi grassi dal flusso sanguigno sono trasportati nei mitocondri dove sono
ossidati. Ogni molecola di acido grasso (attivata, acile grasso \ce{CoA}) \`e rotta completamente da un ciclo di reazione che rompe due atomi di carbonio dalla coda carbossile generando
una molecola di acetile \ce{CoA} per ogni ciclo. Una molecola di \ce{NADH} e una di \ce{FADH2} sono prodotte. La maggior parte di questa energia rimane conservata nelle molecole di 
acetile \ce{CoA} e pertanto entra in gioco il ciclo di reazione di acido citrico, in cui il gruppo acetile (\ce{-COCH3}) viene ossidato in \ce{CO2} e \ce{H2O}. 
\subsection{Il ciclo di acido citrico genera \ce{NADH} ossidando i gruppi acetili in \ce{CO2}}
Il ciclo di acido citrico o ciclo di acido tricarbossilico o ciclo di Krebs crea i due terzi dell'ossidazione totale dei composti di carbonio nelle cellule e i suoi prodotti principali
sono \ce{CO2} e elettroni ad alta energia nella forma di \ce{NADH}. La prima \`e secreta come scarto, mentre i secondi sono passati a catene di trasporto di elettroni legate alla 
membrana, eventualmente combinandosi con l'ossigeno per formare acqua. Il ciclo in s\`e non richiede ossigeno, ma viene utilizzato in reazioni seguenti e avviene nei mitocondri. Il 
gruppo acetile non \`e ossidato direttamente: il gruppo \`e trasferito da \ce{CoA} a una molecola di ossalacetato per formare l'acido tricarbossilico o acido citrico, che viene
gradualmente ossidato permettendo la produzione di molecole vettore ad alta energia. Alla fine delle otto reazioni l'ossalacetato viene rigenerato e rientra un altro ciclo. La molecola
di \ce{FADH2} (flavina adenina dinucleotide ridotto) viene prodotta da \ce{FAD} e il trifosfato ribonucleico \ce{GTP} dal \ce{GDP}, molto simile dell'\ce{ATP}, il cui trasferimento 
del gruppo fosfato all'\ce{ADP} produce \ce{ATP}. L'energia conservata in \ce{NADH} e \ce{FADH2} viene utilizzata per la produzione di \ce{ATP} attraverso fosforilazione ossidativa, che
richiede ossigeno gassoso. L'acqua mette a disposizione gli atomi di ossigeno necessari alla produzione di \ce{CO2}. Alcuni amminoacidi passano dal citosol ai mitocondri, dove sono 
convertiti in acetile \ce{CoA} o altri intermedi del ciclo di Krebs che insieme alla glicolisi forma un punto per le reazioi biosintetiche producendo intermedi come ossalocetato e 
$\alpha$-chetoglutarato. Alcune di queste sostanze prodotte dal catabolismo sono trasferite dal mitocondrio al citosol, dove vengono utilizzate in reazioni anaboliche come precursori
per la sintesi di molte molecole essenziali.
\subsection{Il trasporto di elettroni guida la sintesi della maggiori parte dell'\ce{ATP} nelle cellule}
La maggior parte dell'energia chimica rilasciata nell'ultima parte della degradazione di un nutriente permette il trasferimento da parte di \ce{NADH} e \ce{FADH2} degli elettroni che
hanno preso durante l'ossidazione del nutriente a catene di trasporto di elettroni, incorporate nella membrana interna del mitocondrio. Come gli elettroni passano lungo la catena di
molecole elettron-accettori e molecole elettron-donatori passano sequenzialmente a stati di energia minore. Questa energia pompa protoni \ce{H+} lungo la membrana dalla matrice 
mitocondriale allo spazio intermembrane e al cito sol, generando un gradiente di ioni \ce{H+} che serve come fonte di energia per la cellula per reazioni come la generazione di \ce{ATP}
attraverso fosforilazione di \ce{ADP}. Alla fine della serie di trasferimenti gli elettroni sono passati a molecole di ossigeno gassoso diffuso nel mitocondrio che producono acqua. 
Questo processo detto fosforilazione ossidativa succede anche nella membrana plasmatica dei batteri. L'ossidazione di una molecola di glucosio in \ce{H2O} e \ce{CO2} \`e utilizzata per
produrre $30$ molecole di \ce{ATP}.
\subsection{Amminoacidi e nucleotidi sono parte del ciclo azotato}
Gli atomi di azoto e zolfo passano da composto a composto e all'ambiente in una serie di cicli reversibili. Essendo l'azoto poco reattivo solo alcuni organismi sono in grado di 
incorporarlo attraverso fissazione. La maggior parte dell'azoto organico deriva pertanto dal passaggio tra organismi. L'azoto viene ricevuto da proteine e acidi nucleici che vengono
rotti in amminoacidi e nucleotidi e viene utilizzato per produrre proteine e acidi nucleici. Circa met\`a degli amminoacidi sono essenziali per l'uomo: non possono essere sintetizzati. 
I nucleotidi possono essere sintetizzati attraverso cammini biosintetici specializzati. Tutto l'azoto nelle basi purine e pirimidine sono derivati dall'amminoacido glutammina, dall'acido
aspartico e dalla glicina, mentre gli zuccheri del ribosio e del desossiribosio dal glucosio. Gli amminoacidi non utilizzati in biosintesi possono essere ossidati per creare energia
metabolica. Carbonio e idrogeno formano \ce{CO2} e \ce{H2O}, mentre gli atomi di azoto sono spostati in varie forme e appaiono come urea, che viene secreta. Ogni amminoacido viene
processato diversamente. Lo zolfo, per essere usato per la vita, deve essere ridotto a solfuro (\ce{S2-}). Lo stato di ossidazione dello zolfo richiesto per la biosintesi di 
metionina, cisteina, del coenzima A e i centri a ferro-zolfo per il tramorto di elettroni. La riduzione dello zolfo comincia in batteri, funghi e piante, dove enzimi e \ce{TAP} sono 
utilizzati per creare un cammino per l'assimilazione. Gli esseri umani non possono compiere questo processo.
