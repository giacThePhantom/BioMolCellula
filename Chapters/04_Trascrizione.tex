\chapter{Trascrizione}

\section{Panoramica}
La trascrizione \`e il meccanismo generale che porta alla produzione di RNA a partire dal DNA.

	\subsection{Produzione di RNA}

		\subsubsection{Lettura di DNA}
		Il RNA viene sintetizzato trascrivendo DNA.
		Il processo inizia con lo svolgimento di una piccola porzione della doppia elica.
		Le basi cos\`i esposte su un filamento agiscono come stampo per la sintesi di RNA.
		
		\subsubsection{Polimerizzazione}
		L'accoppiamento di basi complementari determina la sequenza di nucleotidi del RNA.
		Quando avviene una corrispondenza il ribonucleide viene legato con la creazione di un legame fosfodiesterico.

		\subsubsection{Separazione del RNA}
		Il filamento di RNA viene separato nella regione a monte dell'aggiunta e viene rilasciato come filamento singolo.
		Le molecole di RNA sono tipicamente pi\`u corte rispetto a quelle di DNA.

	\subsection{RNA polimerasi}
	Le RNA polimerasi sono gli enzimi responsabili della catalizzazione dell legame fosfodiestere tra i nucleotidi del RNA nascente.
	Estende la catena di RNA in direzione $5'$-$3'$ e non necessita di primer.
	Ha attivit\`a processiva.

		\subsubsection{Substrati}
		I substrati sono ribonucleoside trifosfato.
		La sua idrolisi fornisce l'energia necessaria alla reazione.

		\subsubsection{Temporizzazione}
		Il rilascio immediato del RNA vuol dire che le copie possono essere create in poco tempo, con la sinitesi di molecole addizionali prima che le altre siano completate.

		\subsubsection{Tasso di errore}
		La RNA polimerasi compie un errore ogni $104$ nucleotidi.
		Questo tasso di errore alto \`e tollerato in quanto gli RNA hanno vita breve.

		\subsubsection{Proof-reading}
		La RNA polimerasi possiede un'attivit\`a di proofreading: se un ribonucleotide errato viene aggiunto pu\`o indietreggiare e il sito attivo asporta attraverso la sostituzione di un pirofosfato con una molecola d'acqua.
		Viene rilasciata infine una molecola di monofosfato.

		\subsubsection{Struttura}
		La RNA polimerasi possiede domini per:
		\begin{multicols}{2}
			\begin{itemize}
				\item Accogliere il DNA.
				\item Accogliere i nucleotidi.
				\item Riconoscere la sequenza di inizio.
				\item Far uscire il RNA.
			\end{itemize}
		\end{multicols}

			\paragraph{Core}
			Il core della RNA polimerasi \`e conservato tra eucarioti e procarioti.
			\`E pertanto un oloenzima composto da $5$ subunit\`a e un gruppo prostetico.

				\subparagraph{Subunit\`a $\mathbf{\beta\ \beta'}$}
				La subunit\`a $\beta'$ \`e la pi\`u grande seguita da $\beta$.
				Sono codificate dai geni \emph{rpoC} e \emph{rpoB}.
				Formano il sito attivo per la sintesi di mRNA.
				Componenti creano interazioni non specifiche con DNA e RNA.
	
				\subparagraph{Subunit\`a $\mathbf{\alpha'\ \alpha''}$}
				Le subunit\`a $\alpha'$ e $\alpha''$ sono uguali tra di loro e contengono un dominio $\alpha NTD$ (dominio N terminale) e uno $\alpha CTD$ (dominio C terminale).
				Il primo permette l'assemblaggio della polimerasi, mentre il secondo reagisce con il promotore sul DNA e coi fattori di regolazione.

				\subparagraph{Subunit\`a $\mathbf{\omega}$}
				La subunit\`a $\omega$ \`e la pi\`u piccola e stabilizza la RNA polimerasi assemblata.

			\paragraph{Polimerasi eucariote}
			Le RNA polimerasi eucariote contengono $9$ domini in pi\`u per le interazione con altre proteine.
			Ne esistono tre: RNA polimerasi $I$, $II$ e $III$.

			\paragraph{Solco centrale attivo}
			Il solco centrale attivo \`e simile a quello della DNA polimerasi.
			Contiene un solo ione \emph{$Mg^{2+}$}.
			Un altro viene aggiunto con un nuovo nucleotide ogni ciclo di sintesi e rilasciato con il pirofosfato.

	\subsection{Inibitori di RNA polimerasi}
	Gli inibitori della polimerasi sono molecole come $\alpha$ amanitina e actiomicina D che impediscono la trascrizione.
	Sono in grado di bloccare specificatamente una delle tre polimerasi.
	Vengono usati per determinare l'emivita di un RNA, bloccando la sua trascrizione e osservando il graduale abbassamento di concentrazione attraverso PCR ed elettroforesi.

	\subsection{Categorie di RNA prodotto}
	Il prodotto finale dei geni pu\`o essere una proteina o una molecola di RNA.
	Nel secondo caso queste hanno funzione strutturale o catalitica.
	Possono regolare l'espressione genica.
	Si notano pertanto diverse categorie di RNA:
	\begin{multicols}{2}
		\begin{itemize}
			\item mRNA: messaggero, per la produzione di componenti enzimatici strutturali e regolatori.
			\item tRNA: transfer, per il trasporto degli amminoacidi per la traduzione.
			\item rRNA: ribosomiale, per la formazione dei nuclei catalitici dei ribosomi.
			\item snRNA: small nuclear RNA per lo splicing (\emph{U1,2,4,5,6}) in complesso con proteine.
			\item microRNA e piccoli RNA interferenti: regolano l'espressione genica.
			\item piRNA: proteggono le linee germinali dai trasposoni.
			\item Lunghi RNA non codificanti: sono impalcature o possono avere un ruolo nella regolazione.
		\end{itemize}
	\end{multicols}
\section{Trascrittosoma}
Si intende per trascrittosoma il complesso multienzimatico responsabile della trascrizione.

	\subsection{Composizione}

		\subsubsection{Elicasi}
		Le elicasi aprono la doppia elica.

		\subsubsection{Fattori di trascrizione}
		I fattori di trascrizione o \emph{TF} sono fattori che aiutano la trascrizione.
		Per esempio \emph{TFIID} e \emph{TFIIH} sono fattori di proofreading e di reattivazione della RNA polimerasi instabile.

		\subsubsection{RNA polimerasi}

			\paragraph{RNA polimerasi $\mathbf{I}$}
			La RNA polimerasi $I$ si occupa della produzione degli rRNA tranne il $5S$.

				\subparagraph{Reclutamento di fattori}
				La RNA polimerasi $I$ recluta:
				\begin{multicols}{2}
					\begin{itemize}
						\item \emph{UBF}: upstream binding factor, si lega a monte del promotore piegando il DNA e avvicinando i fattori tra di loro.
							Agisce come tetramero.
						\item \emph{SL1} o fattori \emph{TBP} ($TATA$ box binding protein), riconoscono la sequenza $TATA$ box.
					\end{itemize}
				\end{multicols}
				La formazione del tetramero tra \emph{UBF} e i \emph{SL1} come \emph{TBP} e \emph{TAF} crea un tetramero imponendo al DNA una struttura a sella che permette il corretto reclutamento della RNA polimerasi sul DNA.

			\paragraph{RNA polimerasi $\mathbf{II}$}
			La RNA polimerasi $II$ produce mRNA e microRNA.
			Si occupa pertanto della trascrizione dei geni produttori di proteine.
			Possiede una coda terminale strutturata con tirosina, serina e treonina.
			Pu\`o subire modifiche post-traduzionali per reclutare fattori coinvolti nello splicing.
			Durante l'allugnamento la coda impedisce che la RNA polimerasi si stacchi prima della fine del gene.

				\subparagraph{Reclutamento di fattori}
				La RNA polimerasi $II$ recluta:
				\begin{multicols}{2}
					\begin{itemize}
						\item \emph{TF2D}: riconosce la $TATA$ box binding protein e i complesso \emph{TAF} (\emph{TBP} associated factor), formando un primo complesso.
						\item \emph{TF2A}: stabilizza il complesso appena formato.
						\item \emph{TF2B}: recluta e posiziona RNA polimerasi in maniera corretta sul promotore.
						\item \emph{TF2F}: stabilizza l'interazione della RNA polimerasi con \emph{TBP} e \emph{TF2B}.
						\item \emph{TF2E}: attrae e regola \emph{TF2H}.
						\item \emph{TF2H}: ha attivit\`a elicasica e chinasica: fosforila infatti la coda della RNA polimerasi $II$.
						\item La fosforilazione della coda che rimane libera permette l'allungamento del filamento.
					\end{itemize}
				\end{multicols}

			\paragraph{RNA polimerasi $\mathbf{III}$}
			La RNA polimerasi $III$ produce tRNA e rRNA $5S$.

				\subparagraph{Reclutamento di fattori}
				La RNA polimerasi $III$ recluta:
				\begin{multicols}{2}
					\begin{itemize}
						\item \emph{TFIIIC} e \emph{TFIIIB} che riconoscono una \emph{BoxA} a monte e una \emph{BoxB} a valle del gene.
						\item \emph{TFIIIB} e \emph{TFIIIC} reclutano la RNA polimerasi $III$.
						\item La trascrizione del rRNA $5S$ richiede un ulteriore riconoscimento di un \emph{BoxC} all'interno del gene.
					\end{itemize}
				\end{multicols}



\section{Processo della trascrizione}

	\subsection{inizio}
	La trascrizione accurata di un gene richiede il riconoscimento del suo inizio e fine.
	L'iniziazione \`e pertanto il processo di regolazione di quali proteine devono essere prodotte e a quale velocit\`a.

		\subsubsection{Fattori sigma}
		I fattori $\sigma$ sono subunit\`a addizionali che si associano alla RNA polimerasi e la aiutano alla lettura del DNA batterico.
		Il fattore associato all'oloenzima aderisce debolmente al DNA fino a quando trova una sequenza consenso detta protomero.
		Riconosce sequenze a $-35$ e a $-10$ nelle zone promotrici.
		A quel punto si lega strettamente e d\`a inizio alla trascrizione.
		
			\paragraph{Fattori basali}
			I fattori basali sono sequenze minime nella zona promotrice che consentono l'attacco della RNA polimerasi.
			Vengono riconosciute dai fattori $\sigma$.
			\begin{multicols}{2}
				\begin{itemize}
					\item \emph{CAT box}: sono triplette \emph{CAT} in posizione $-35$.
					\item \emph{TATA box}: sequenze ricche in $A$ e $T$ in posizione $-25$, riconosciute dalla RNA polimerasi $II$.
				\end{itemize}
			\end{multicols}



		\subsubsection{Bolla di trascrizione}
		Dopo il legame con il protomero si apre una bolla di trascrizione di circa $10$ nucleotidi.
		Il filamento non legato dall'oloenzima agisce come stampo.

		\subsubsection{Modelli di inizio}
		Dopo il riconoscimento del promotore da parte dei fattori $\sigma$ la RNA polimerasi si lega al DNA.
		Si forma il complesso chiuso sul promotore e si svolge il DNA.
		Il complesso diventa aperto sul DNA svolto.

			\paragraph{Passaggio transiente}
			Nell'iniziazione a passaggio transiente la RNA polimerasi si attacca al DNA, avanza sintetizzando un filamento abortivo e torna indietro.

			\paragraph{A bruco}
			Nell'iniziazione a bruco la RNA polimerasi si attacca, cambia conformazione favorendo la formazione della bolla e trascrive un frammento.

			\paragraph{Scrunching}
			Il meccanismo di scrunching \`e responsabile della sintesi dei primi $10$ nucleotidi.
			La RNA polimerasi rimane attaccata al protomero e tira il DNA nel sito attivo espandendo la bolla di trascrizione.
			Lo stress generato causa un rilascio delle catene e un reinizio della sintesi.
			Questa iniziazione abortiva viene superata e lo stress separa l'enzima dal protomero e dal fattore $\sigma$.

		\subsubsection{Iniziazione per la RNA polimerasi $\mathbf{II}$}
		La RNA polimerasi $II$ richiede un insieme di fattori di trascrizione generali affinch\`e possa trascrivere.
		Questi la aiutano a posizionarsi al promotore, a separare i filamenti e ad essere rilasciata dal promotore in modo da iniziare con l'allungamento.
		Sono analoghi ai fattori $\sigma$ procarioti.
		\begin{multicols}{4}
			\begin{itemize}
				\item \emph{TFIIA}.
				\item \emph{TFIIB}.
				\item \emph{TFIIC}.
				\item \emph{TFIID}.
			\end{itemize}
		\end{multicols}

			\paragraph{Processo di assemblaggio}
			\begin{multicols}{2}
				\begin{enumerate}
					\item \emph{TFIID} si lega a una \emph{TATA box} attraverso la subunit\`a \emph{TBP}.
					\item La distorsione del DNA nella \emph{TATA box} marca il promotore come attivo.
					\item La RNA polimerasi $II$ insieme ad altri fattori formano un complesso di iniziazione.
					\item La RNA polimerasi $II$ ottiene l'accesso al filamento stampo.
					\item \emph{TFIIH} idrolizza \emph{ATP} e svolge il DNA.
					\item La RNA polimerasi $II$ rimane al promotore sintetizzando corte lunghezze fino a subire una serie di cambi conformazionali che le permettono di spostarsi.
					\item Viene fosforilata la RNA polimerasi $II$  da \emph{TFIIH}.
					\item I fattori di trascrizione generali si dissociano dal promotore.
				\end{enumerate}
			\end{multicols}

			\paragraph{Regolazione}

				\subparagraph{Enhancers}
				Gli enhancers sono sequenze di DNA a cui si legano gli attivatori trascrizionali.
				Questi reclutano la RNA polimerasi $II$ al punto di inizio.

				\subparagraph{Mediatore}
				Il mediatore \`e un complesso proteico che permette alle proteine attivatrici di comunicare con la RNA polimerasi $II$ e con i fattori di trascrizione generali.

				\subparagraph{Modifica della cromatina}
				Affinch\`e la RNA polimerasi $II$ possa trascrivere devono intervenire complessi rimodellatori della cromatina e modificatori degli istoni che decondensano la cromatina.

		\subsubsection{Fattori di trascrizione}

		\subsubsection{Attivatori, mediatori e proteine per la modifica della cromatina}

	\subsection{Allungamento}
	L'allungamento inizia quando la polimerasi ha sintetizzato un corto frammento di RNA di $10$ basi.
	L'enzima separa i due filamenti e li riavvolge dopo il suo passaggio mentre sintetizza il RNA.
	Mentre la reazione prosegue stacca la catena dallo stampo.
	Ha una funzione di proofreading.
	La catena viene quindi aperta ulteriormente in modo da iniziare un processo continuativo.
	Nei batteri $\sigma^4$  libera il canale per la fuoriuscita di RNA.
	La formazione del legame fosfodiesterico causa la liberazione di pirofosfato tamponato con \emph{$Mg^{2+}$}.

		\subsubsection{Meccanismi di correzione}

			\paragraph{Editing pirofosforico}

			\paragraph{Editing idrolitico}
			Nell'editing idrolitico la RNA polimerasi taglia a $2$ o $3$ nucleotidi a monte dell'errore commesso e torna indietro.

		\subsubsection{Andamento dell'allungamento}
		Una volta che la RNA polimerasi ha iniziato la trascrizione si muove a scatti.
		Viene legata da una serie di fattori di allungamento.

			\paragraph{Fattori di allungamento}
			Si intende per fattori di allungamento quelle proteine che diminuiscono la probabilit\`a che la RNA polimerasi si dissoci dal DNA prima di aver finito la trascrizione.
			Si associano ad essa dopo l'iniziazione.
			Possono modificare gli istoni lasciando una traccia che aiuta le trascrizioni successive e coordina l'allungamento.
			
			\paragraph{Tensione superelicale}
			La tensione superelicale formata dalla RNA polimerasi viene rilassata da DNA topoisomerasi.

		\subsubsection{Processamento del RNA}
		Negli eucarioti la trascrizione \`e accoppiata con il processamento del RNA.

			\paragraph{Terminazioni}
			La terminazione $3'$ viene modificata con poliadenliazione, mentre la $5'$ viene metilata.
			Le modifiche creano un meccanismo di verifica prima che il RNA lasci il nucleo

			\paragraph{RNA splicing}
			Nel processo di splicing vengono rimossi gli introni degli RNA.
			Splicing alternativi permettono la creazione di proteine diverse da uno stesso gene.

	\subsection{Terminazione}
	Durante la terminazione l'enzima rilascia il RNA prodotto e si dissocia dal DNA.

		\subsubsection{Procarioti}

			\paragraph{Rho indipendente}
			La terminazione Rho indipendente consiste di regioni a forcina sul RNA che destabilizzano l'interazione tra RNA polimerasi e DNA.
			Non necessita di fattori proteici.
			Sono sequenze palindromiche ricche di $C$ e $G$ seguite da sequenze ricche di $A$ e $T$ palindromiche.

			\paragraph{Rho dipendente}
			Nella terminazione Rho dipendente la proteina Rho lega il RNA e riconosce sequenze \emph{RUT}.
			Si sposta fino a incontrare la RNA polimerasi destabilizzandola e causando la sua dissociazione dal DNA.
			Questa sequenza si trova a valle del sito dove avviene la traduzione.

		\subsubsection{Eucarioti}

			\paragraph{Modello a torpedo}
			Nel modello a torpedo una RNAasi interagisce con la RNA polimerasi ma non degrada il RNA protetto dal cap metilico.
			La RNA polimerasi continua la traduzione fino all'aggiunta di una sequenza poliadenilata dove la RNAasi taglia il RNA prodotto causando la destabilizzazione del complesso e la terminazione.

			\paragraph{Modello allosterico}
			Nel modello allsoterico dopo il taglio del sito di poliA $AAUAAA$ un cambio conformazionale porta al distacco della RNA polimerasi.


\section{Trasporto tra nucleo e citoplasma}
Le proteine vengono prodotte nel citoplasma, ma alcune devono essere poi trasportate nel nucleo per poter svolgere la loro funzione.
Queste pertanto possiedono una particolare sequenza segnale \emph{NLS}: nuclear localization sequence.
Le proteine che invece dal nucleo devono esserne esportate presentano una \emph{NES}: nuclear export sequence.

	\subsection{Complessi coinvolti}

		\subsubsection{Complesso di importazione}
		Il complesso di importazione \emph{Importing karyopherin} riconosce la \emph{NLS}, una regione ricca di lisina o arginina per l'importazione nel nucleo.
		Trasporta la proteina dal citoplasma al nucleo.

		\subsubsection{Complesso di esportazione}
		Il complesso di esportazione \emph{Exporting karyopherin} riconosce \emph{NES}, una regione contenente leucine o isoleucine.
		Trasporta la proteina dal nucleo al citoplasma.

	\subsection{Verificare la localizzazione}

		\subsubsection{Eterocaryon essay}
		L'eterocaryon essay \`e un processo che permette di distinguere se una proteina viene importata nel nucleo.
		Per farlo si fondono due cellule di specie diverse in un'unica.
		Questo si volge attraverso \emph{PEG} che permette la fusione delle membrane.
		Questo sistema permette di studiare se una proteina nel nucleo di una cellula si muove dal nucleo al citoplasma e viceversa.
		Si trasfettano le cellule umane con la proteina fusa con \emph{GFP}.
		Dopo che \`e avvenuta la fusione se la proteina nucleare si sposta anche nel citoplasma entra nel nucleo della cellula murina.

	\subsection{Processo di importo ed esporto}
	Un importante fattore che d\`a direzionalit\`a al trasporto \`e la proteina \emph{RAN}.
	Questa pu\`o legare \emph{GTP} o \emph{GDP} in base alla regione della cellula in cui si trova.

		\subsubsection{Importazione}
		L'importazione nel nucleo \`e guidata da importine: queste si legano al cargo e passano nel nucleo.
		In questo momento \emph{RanGTP} lega l'importina causandone il distacco dal cargo.
		L'importina scarica legata a \emph{RanGTP} viene esportata e nel citoplasma avviene l'idrolisi in \emph{RanGDP}, rendendo l'importina nuovamente disponibile.

		\subsubsection{Esportazione}
		L'esportazione dal nucleo \`e guidata da esportine: queste legano \emph{RanGTP}.
		Il legame causa un cambio conformazionale che causa il legame con il cargo.
		A questo punto abbandona il nucleo e l'idrolisi in \emph{RanGDP} permette il distacco del cargo dall'esportina.

		\subsubsection{Leptomicina $\mathbf{B}$}
		La leptomicina $B$ \`e una tossina di origine fungina che compete con il cargo per il legame con l'esportina.
		Impedisce la sua esportazione.

\section{Meccanismi di regolazione della trascrizione e dell'espressione genica}

	\subsection{Operoni}
	Si intendono per operoni gruppi di geni adiacenti trascritti come un'unica molecola di mRNA policistronica tradotta nei procarioti in proteine diverse.

	\subsection{Induzione}
	Si intende per induzione un meccanismo di controllo che viene attivato in presenza di una molecola induttore.

		\subsubsection{Induzione a controllo negativo}
		Nell'induzione a controllo negativo un repressore viene represso in presenza di  un induttore.
		Questo causa una trascrizione attiva.

		\subsubsection{Induzione a controllo positivo}
		Nell'induzione a controllo positivo l'attivatore inattivo viene attivato dall'induttore causando una trascrizione attiva.

	\subsection{Repressione}
	Si intende per repressione un controllo che viene disattivato in presenza di un repressore, prodotto da una sequenza che per ingombro sterico impedisce il reclutamento della RNA polimerasi.

		\subsubsection{Repressione a controllo negativo}
		Nella repressione a controllo negativo l'attivazione del repressore reprime la trascrizione.

		\subsubsection{Repressione a controllo positivo}
		Nella repressione a controllo positivo l'attivatore viene disattivato da un repressore disattivando la trascrizione.

	\subsection{Esempi}

		\subsubsection{Operone \emph{lac}}
		L'operone \emph{lac} \`e formato da $3$ geni $Z$, $Y$ e $A$ tradotti in $\beta$-galattosidasi, permeasi (assimilazione del lattosio) e transacetilasi.
		\`E un esempio di repressione da catabolita.

			\paragraph{Repressione da catabolita}
			L'operone \emph{lac} presenta un esempio di repressione da catabolita: la sua trascrizione \`e infatti regolata dalla presenza o assenza di lattosio e \emph{cAMP}.

				\subparagraph{Assenza di glucosio e lattosio}
				In assenza di lattosio e glucosio i livelli di \emph{cAMP} sono alti: \emph{CAP} (\emph{cAMP} receptor protein) si lega sul sito a monte del promotore.
				L'assenza del lattosio causa la presenza del repressore e la trascrizione dell'operone \`e disattiva.

				\subparagraph{Assenza di lattosio e presenza di glucosio}
				In assenza di lattosio e presenza di glucosio il repressore \`e presente, \emph{CAP} non si lega al sito in quanto \emph{cAMP} \`e a bassi livelli.
				La trascrizione \`e repressa.

				\subparagraph{Presenza di glucosio e lattosio}
				In presenza di glucosio e lattosio viene rimosso il repressore dall'operatore, ma i bassi livelli di \emph{cAMP} causano una trascrizione basale.

				\subparagraph{Presenza di lattosio e assenza di glucosio}
				In presenza di lattosio e assenza di glucosio il repressore viene rimosso e \emph{CAP} si lega al sito di attacco a causa degli alti livelli di \emph{cAMP}.
				La trascrizione si attiva in modo efficiente.

		\subsubsection{Operone triptofano}
		L'operone triptofano codifica per enzimi che portano a sintesi del triptofano.
		Viene attivato in sua assenza.

			\paragraph{Struttura}
			L'operone triptofano \`e formato in sequenza da:
			\begin{multicols}{2}
				\begin{itemize}
					\item Operatore.
					\item Fattore di regolazione.
					\item Sequenza leader.
					\item $4$ regioni trascritte.
					\item Coda poli-U.
					\item Terminazione leader.
				\end{itemize}
			\end{multicols}
			La sequenza leader \`e un sensore per il triptofano: possiede molti codoni per la sintesi di tale amminoacido nella sua sequenza.
			Le $4$ regioni trascritte inoltre possono interagire tra di loro.

			\paragraph{Meccanismo di regolazione}
			Questo meccanismo di regolazione sfrutta la simultaneit\`a tra trascrizione e traduzione nei batteri.
			La disponibilit\`a del triptofano infatti determina la velocit\`a di traduzione della sequenza leader e quali delle regioni trascritte interagiscono tra di loro.

				\subparagraph{Assenza di triptofano}
				In assenza di triptofano lo stallo del ribosoma sul peptide leader permette il reclutamento di enzimi per tradurre i geni a valle e si forma una forcella tra le regioni $2$ e $3$ che non \`e un terminatore.
				Gli enzimi vengono pertanto trascritti.

				\subparagraph{Presenza di triptofano}
				In presenza di triptofano il peptide leader viene completamente tradotto e non permette l'associazione tra le regioni $2$ e $3$ a causa dell'ingombro sterico del ribosoma.
				Si forma una struttura a forcella tra $3$ e $4$ che forma un sito di terminazione Rho indipendente che bocca la trascrizione.
				Gli enzimi non vengono pertanto trascritti.

\section{Modifiche post-trascrizionali}
Vedi anche pagine 54 58

	\subsection{Capping}
	Il capping \`e una modifica subita da tutti gli mRNA.
	Avviene prima che finisca la trascrizione, dopo che la coda della RNA polimerasi $II$ viene fosforilata.
	Viene aggiunta la $7$-metilguanosina \emph{m7GppM} al $5'$ del trascritto primario.

		\subsubsection{Funzione}
		\begin{multicols}{2}
			\begin{itemize}
				\item Protegge da degradazione.
				\item Trasporto.
				\item Stabilit\`a.
				\item Traduzione.
				\item Splicing.
			\end{itemize}
		\end{multicols}

		\subsubsection{Processo}
		\begin{multicols}{2}
			\begin{enumerate}
				\item La RNA tri-fosfatasi libera il gruppo $P$ al $5'$.
				\item La Guanilil-trasferasi: attacca \emph{GMP} al primo nucleotide del RNA.
				\item Metiltrasferasi: aggiunge un gruppo metilico alla guanosina $7$.
			\end{enumerate}
		\end{multicols}

		\subsubsection{Riconoscimento}
		Il riconoscimento della base metilata avviene grazie al cap-binding complex formato da \emph{CBP20+CBP80} che si legano ad essa nel nucleo e si dissociano nel citoplasma.
		Nel citoplasma vengono sostituiti da \emph{eIF4E} (eucariotic initiator factor $4E$).
		
		\subsubsection{Assenza del cappuccio}
		Se dopo la terminazione della trascrizione un mRNA non possiede un cappuccio viene degradato da un'esonucleasi trasportata lungo la coda della polimerasi.

	\subsection{Poliadenilazione}
	La poliadenilazione al $3'$ \emph{UTR} avviene ad opera dell'enzima \emph{poliA polimerasi}.
	\`E formata da ripetizioni di \emph{AAUAAA}.

		\subsubsection{Funzione}
		La lunghezza della coda influisce sulla stabilit\`a e sull'esportazione del RNA nel citoplasma.
		L'alterazione delle code causa mattie alterate.

		\subsubsection{Riconoscimento}
		La coda poli-A viene riconosciuta da fattori che permettono il legame con \emph{PABP1} nel nucleo e \emph{PABC} nel citoplasma (poli-A binding protein).

		\subsubsection{Processamento del RNA}
		I fattori coinvolti sono:
		\begin{multicols}{2}
			\begin{itemize}
				\item \emph{CPSF}: cleavage and polyadenilation specificity factor.
				\item \emph{CstF}: cleavage stimulation factor.
				\item \emph{CFIm}, \emph{CFIIm}: cleavage factor.
			\end{itemize}
		\end{multicols}
		\emph{CstF} e \emph{CPSG} viaggiano con la coda della RNA polimerasi e sono trasferite alla terminazione $3'$ quando emerge.
		Una volta che sono trasferite si assemblano sulle sequenze riconoscimento.
		Il RNA viene rotto dalla polimerasi e un enzima poli-A polimerasi \emph{PAP} aggiunge $200$ nucleotidi $A$ alla terminazione.
		Il precursore delle addizioni \`e \emph{ATP}/
		Le proteine che si legano ad essa si assemblano durante la sintesi della catena.


		\subsubsection{Lunghezza della coda}
		La lunghezza della coda \`e fondamentale in molti organismi.
		In \emph{Xwenopus} prima della fecondazione gli RNA sono in uno stato dormiente con una coda di poli-A corta.
		Dopo la fecondazione si assiste a un mutevole allungamento della coda e all'attivazione della traduzione.

			\paragraph{PCR poliA test \emph{PAT}}
			Il \emph{PAT} \`e un test usato per verificare le variazioni di lunghezza di poli-A.
			Per farlo un primer a molte $T$ viene legato alla coda.
			Dopo migrazione su lastra d'agarosio pi\`u lunga la coda pi\`u lenta la migrazione.
			La lunghezza della coda va ad influire sulla stabilit\`a della base traduzionale.

		\subsubsection{Distrofia muscolare oculo-faringe}
	
			\paragraph{Analisi clinica}
			La distrofia muscolare oculo-faringe colpisce tra i $40$ e i $60$ anni e va in lenta progressione.
			I sintomi sono:
			\begin{multicols}{2}
				\begin{itemize}
					\item Abbassamento delle palpebre.
					\item Difficolt\`a a deglutire.
					\item Debolezza muscolare.
					\item Progressiva paralisi.
				\end{itemize}
			\end{multicols}

			\paragraph{Analisi molecolare}
			Attraverso microscopia sono presenti accumuli di RNA in foci del nucleo.
			Questo \`e dovuto a mutazioni nella poli-A binding protein $N$ che non permette un esporto corretto del mRNA.

		
	\subsection{Splicing}

		\subsubsection{Caratterizzazione RNA}

		\subsubsection{Determinazione del sesso in Drosophila melanogaster}










La trascrizione \`e il meccanismo attraverso cui l'infomrazione trascritta nel DNA viene trasportata nell'RNA. Si consideri che la trascrizione \`e un meccanismo generale che porta
alla traduzione di RNA: mRNA (codifica proteine), tRNA (codifica amminoacidi), rRNA (nei ribosmi), microRNA (ruolo regolatorio, bloccano la traduzione di mRNA), snRNA (piccoli RNA 
nucleari processi nucleari come lo splicing), snoRNA (piccoli RNA nucleolari, processare e modificare rRNA, maturazione degli rRNA), siRNA (piccoli RNA interferenti che spengono 
l'espressione genica promuovendo la degradazione di mRNA specifici), piRNA (piwi interacting RNA, piccoli RNA che interagiscono con proteine piwi che proteggono la linea germinale dagli
elementi trasponibili), lncRNA (long non coding RNA impalcatura per la regolazione di processi cellulari come inattivazione del cromosoma X). Il processo di trascrizione \`e ampio ma a 
tutta una serie di RNA con un ruolo strutturale, funzionale e di regolazione. Il processo di trascrizione si attua a una serie di fasi e l'enzima che svolge il ruolo principale \`e l'RNA
polimerasi, una complessa macchina molecolare caratterizzata da diverse subunit\`a come nella DNA polimerasi \`e un oloenzima ma a differenza della DNA polimerasi \`e in grado di iniziare
la trascrizione senza bisogno di inneschi. Utilizzando un filamento di DNA come stampo. A differenza dei procarioti gli eucarioti presentano 5 RNA polimerasi: RNA polimerasi I , II e III
rispettivamente per rRNA tranne 5s, mRNA microRNA e snRNA e snoRNA, tRNA e rRNA 5s. Una sola nei procarioti. Il processo mediante cui la cellula trascrivendo un mRNA vuole produrre 
quantit\`a di proteine: ci sono geni trascritti con un alto grado di trascrizione con una certa quantit\`a di proteine e geni trascritti cone fficienze pu\`u bassa e minore quantit\`a 
di proteine. \`E semplificativo in quanto un'unico mRNA pu\`o produrre pi\`u proteine in quanto da un'unica molecola di mRNA ci sono diverse molecole proteiche legato alla stabilit\`a 
dell'RNA e meccanismi che possono inibire la traduzione del pool di RNA pu\`o ridurre il numero di proteine prodotte. Una cellula pu\`o produrre una quantit\`a di RNA bassa ma tradotto
con efficienza elevata. Quando si beole bloccare la traduzione \`e comodo avere poche copie di RNA in giro. IN quanto lo spegnimento potrebbe richiedere pi\`u tempo. Fattori: efficienza
di traduzione e stabilit\`a di RNA, numero di RNA. GLi mRNA possiedono una sequenza $5'$ UTR untraslated region, una CR, coding region che inizia con ATG per metionina, triplette per
amminoacidi e la tripletta di stop: TAA, TGA o TAG, poi la regione $3'$ UTR, analoga alla $5'$ con dimensioni differenti di solito pi\`u lunga in quanto ci sono sequenze regolatorie 
con sequenze di stabilizzazione o di trasporto a cui si legano RNA binding protein. I due filamenti di DNA uno \`e quello stampo che funge da stampo e uno \`e quello edificante alla fine
un trascritto ha una sequenza complementare al filamento stampo ed equivalente a quello codificante.
\section{Struttura delle RNA polimerasi}
Le RNA polimerasi eucaritiche e procariotiche sono simile anche se le prime contengono altri cofattori. La struttura ricorda una mano aperta che ha la funzione di creare un solco per
l'alloggiamento del DNA e poi ha domini per accogliere i nucleotidi, per riconoscere la sequenza sul DNA e domini che servono alla molecola di RNA in fase di sintesi di lasciare e 
uscire dall'enzima. Nei batteri ci sono di due subunit\`a alfa, due beta e una omega, lo stesso per la polimerasi I, II e III a cui si aggiungono altre componenti. In particolare la
II contiene una coda, un dominio C terminale soggetta a modifiche importanti per le fasi di maturazione per l'mRNA. Le RNA polimerasi possono essere inibite da delle sostanze che 
vengono utilizzate durante gli esperimenti e sono l'actinomicina B e l'alpha aminitina prodotta dal fungo amanita muscaria. Queste sostanze bloccano l'attivit\`a dell'RNA polimearasi e 
bloccano la trascrizione impedento all'RNA polimerasi di camminare lungo il DNA e di copiare il filamento stampo. Queste sostanze inibiscono sia specifica I o II. A basse concentraizone
l'alpha aminitina solo l'attivit\`a della polimerasi I. Mentre a concentrazioni elevate sia rRNA che mRNA o tRNA. E permettono di fare degli esperimenti per analizzare la stabilit\`a 
degli RNA. Gli inibitori della sintesi di RNA come della sintesi proteica possono essere utilizzati per capire la stabilit\`a di RNA o della proteina. Le fasi importanti del processo di
trascrizione sono tre: l'inizio in cui l'RNA polimerasi interagisce con il DNA, la fase di allungamento e la terminazione. La fase pi\`u lunga \`e quella di inizio in cui l'RNA polimerasi
deve riconsocere dove posizionari sull'RNA riconoscendo i promotori, forma una bolla di trascrizione piccola simile alla bolla di replicazione, inizia a produrre i primi nucleotidi 
creando una piccola catena e se tutto \`e aposto inizia la fase di allungamento. La fase di inizio \`e quella pi\`u critica. L'allungamento \`e una polimerizzazione ocn enzima processivo
e il processo di terminazione che si conclude con il distacco dell'RNA polimerasi e della catena di RNA. Dopo di che la polimerasi pu\`o iniziare il ciclo. Queste tre fasi avvengono sia
per procarioti che eucarioti. La differenza sta che nei procarioti il processo di trascrizione precede di poco il processo di traduzione: appena sintetizzato l'mRNA viene legato dai 
ribosomi in quanto l'RNA non va incontro a modifiche come lo splicing per gli eucarioti. L'RNA contiene delle sequenze dette introni che devono essere rimosse. La rimozione di queste
sequenze avviene nel nucleo, va nel citoplasma dove viene tradotto. L'RNA polimerasi deve posizionarsi in un punto preciso in quanto deve attaccarsi in maniera specifica e iniziare a 
copiare in una determinata posizione in mood che il trascritto abbia senso biologico. Queste sequenze sono sequenze promotrici e l'RNA polimerasi una sequenza a -35, una a -10 prima 
dell'inizio della trascrizione. Si posiziona a -35 o -40 prima dell'inizio della trascrizione. La sintesi dell'RNA non fa partire subito la polimerizzaizone e allungamento ma avviene
lentamente: la polimerasi inizia a fare un filamento di fino a 9 nucleotidi che spesso abortisce, si riforma un altro piccolo frammento e inizia la fase di allungamento. L'inizio della
trascrizione \`e lento. Il riconoscimento di queste sequenze che permettono l'attacco specifico a monte dei geni. Ci sono due domini $\beta$ e $\beta'$, RPB1 e RPB2 per gli eucarioti 
hanno alta affinit\`a per il DNA l'RNA polimerasi ha una affinit\`a ed \`e in grado di legarsi al DNA in maniera forte, si attacca in maniera specifica grazie ad un altra subunit\`a 
le subunit\`a alpha mantengono insieme le altre subunit\`a e il dominio $\omega$ promuove e mantiene stabile il complesso. Un altro dominio, il fattore o dominio $\sigma$ costituito da
ina serie di domini che formano anse e strutture. Il dominio pi\`u famoso \`e il sigma 70 (dal peso molecolare) che conferisce specificit\`a e riconosce le sequenze a monte del sito di 
trascrizione, importante in quanto permette all'RNA polimerasi di riconoscere le zone promotrici. L'RNA polimerasi possiede un solco centrale di 55 armstrong con una larghezza di 25 che 
pu\`o contenere la doppia elica e all'interno del solco sono presenti 15 - 20 nucleotidi. Il fattore sigma 70 \`e il pi\`u espresso e permette la trascrizione dei geni house keeping
che mantengono la struttura cellulare. \`E un requisito per la normale crescita cellulare. Altri sigma 5, 32 che vengono utilizzati in particolari condizioni. L'interazione dell'RNA 
polimerasi con il DNA avviene grazie a un dominio a scanalatura he alloggia il DNA e incontra il timone che mantiene aperta la bolla di trascrizione * audio corrotto * se si vanno a 
digerire il DNA con degli enzimi vanno a digerire la DNA polimerasi, conoscendo la sequenza rimanente estraendola e sequenziandolo si capiscono quanti nucleotidi vengono riconosciti 
dall'RNA polimerasi il footprinting permette di capire quanti nucleotidi vengono protetti da una determinata proteina. 
\section{Inizio}
L'inizio della trascrizione prevede una fase iniziale in cui la RNA polimerasi sintetizza e poi rilascia dei corti frammenti di RNA fino al raggiungimento di 10 - 12 nucleotidi in modo 
di spiazzare il loop del dominio sigma che occupa il canale di uscita. Il meccanismo mediante cui avviene il mecanismo ha tre modelli proposti:
\begin{itemize}
	\item Passaggio transiente: l'RNA polimerasi riconosce le sequenze promotrici, si attacca, avanza e sintetizza un frammento abortivo e torna indietro.
	\item A bruco: l'RNA plimerasi si attacca, cambia conformazione, si allunga e favorisce la fomraizone della bolla di trascrizione, trascrive il frammento.
	\item Ad accorciamento in cui l'RNA polimerasi si attacca ingloba una parte del DNA e trascrive dei primi frammenti.
\end{itemize}
Non si escludano a vicenda e il modello che rispetti le osservazioni sperimentali sia il terzo. Le prime fasi non sono ben cromprese. L'inizio no n\`e contino ma c'\`e un tenativo che 
abortisce e ritenta.
\subsection{Meccanismo di trascrizione}
L'enzima si lega al promotore riconscendo le posizioni a -35 e a -10 i domini sono il dominio sigma 2 e 4. Questi domini riconoscono le sequenze a monte dall'inizio di trascrizione. 
L'interazione dei domini sul DNA forma il complesso chiuso, chiamato cos\`i in quanto il DNA non \`e ancora aperto. La formazione di questo complesso \`e reversibile. A questo punto
si passa al complesso in cui le due catene sono separate, si ha la formazione della bolla di trascrizione e uno dei due domini sigma 1, 1 che prima bloccava l'entrata del DNA si apre
e facilita l'inserimento della doppia elica nel canale, si stabilizza la bolla di trascrizione ed iniziano ad entrare i primi dntp e formarsi i primi nucleotidi di RNA. QUesto complesso
si dicce complesso aperto. A questo punto si ha la clearence del promotore, cambi conformazionali: il dominio sigma 4 ruota e favorsice la formazione e apetruare del canale per l'uscita
dell'RNA che inizia a essere sintetizzato in maniera continuativa. 
\section{Allungamento}
L'allungmaneto avviene grazie alla formazione di un legame fosfodiesterico tra i ribonucleotidi, reazione facilitata dalla lisi del pirofosfato, uno degli ioni importanti \`e il magnesio
importante in quanto tampona le cariche negative che si originano dal pirofosfato e dalla scissione in fosfato. L'RNA polimerasi \`e meno precisa rispetto alla DNA polimerasi. Esistono
meccanismi di controllo della qualit\`a ma c'\`e un certo grado di errore. I fattori che promuovono l'attivit\`a polimerizzatrie sono fattori diversi e alcuni sono coinvolti nei 
meccanismi di riparazione (GreA, GreB), TFSII o I o III, vuol dire transcription factors associati alla RNA polimerasi II, I o III rispettivamente. Il meccanismo  \`e simile in eucarioti
e procarioti.
\section{Terminazione}
La terminazione differisce tra eucarioti e procarioti. Nei procarioti si trovano due meccanismi: uno intrinseco che non dipende da Rho: Rho-indipentente e uno Rho-dipendente. Un 
terminatore intrinseco \`e un DNA ricco di sequenze palindromiche ricco in C e G seguito da un tratto di 8 10 nucleotidi ricco in A e T, una struttura che trascritta forma una
struttura a forcina (stem loop) che prodotta dall'RNA polimerasi durante la trascrizione destabilizza l'interazione tra RNA polimerasi, DNA e RNA, pertanto la polimerasi si stacca e 
la terminazione ha termine. La regione ricca in A a seguire sembra abbia un ruolo a stabilizzare la struttura e destabilizzare la bolla di trascrizione. Il meccanismo Rho-indipendente
\`e legato alla trascrizione di una sequenza che forma una strttura secondaria che destabilizza l'oloenzima. Quella Rho-dipendente \`e legata alla presenza di un fattore detto Rho e 
legata alla sua capacit\`a di accedere all'RNA. RIconosce una sequenza specifica all'interno del DNA a forcina riconosciuta dalla RNA binding protein Rho che nel momento in cui viene
trovata destabilizza l'RNA polimerasi promuovendone il rilascio e la fine della trascrizione. La proteina Rho si lega all'RNA lo scannerizza fino a quando trova la sequenza dove si blocca
causando il rilascio dell'RNA polimerasi. La proteina si lega a valle dei ribosomi che stanno traducendo i fattori in quanto se si legasse all'inizio verrebbe bloccata dai ribosomi.

Negli eucarioti invece ci sono tre RNA polimerasi sono presente altri fattori addizionali e l'RNA polimerasi II ha una coda C terminale caratterizzato dall'avere una sequenza ricca in
tirosina serina treonina e prolina le quali vanno incontro a meccanismi di modifica post traduzionale come fosforilazione con ruolo nell'attivit\`a dell'RNA polimerasi e di richiamo dei
fattori che vanno a modificare l'RNA in fase di trascrizione in quanto va incontro a processi di maturazione e modifica. Il meccanismo di trascrizione ad opera dell'RNA polimerasi I, II
e III. La formazione del complesso per l'RNA polimerasi I, di geni che trascrivono per rRNA si trova il complesso UBF1 (upstream binding factor) che, questi due si legano a monte e 
formano dei dimeri che ripiegano il DNA. Il tetramero forma il complesso SL1, multiproteico che comprende la TATA box binding protein, altri fattori detti TAF in quanto TBP associated 
factor che si associano alla TATA box binding protein i fattori UBF riconoscono una sequenza a monte della zona promotrice, si legano, formano un tetramero che facilita il reclutamento 
della TBP e fattori associati alla proteine il tutto forma SL1 e tutto facilita il reclutamento dell'RNA polimerasi I in maniera corretta sul promotore. Se questo complesso si forma in
una posizione errata determina un reclutamento errato dell'RNA polimerasi sull'RNA. Per l'RNA polimerasi III per la trascrizione per i geni dei tRNA in questo caso si trova un complesso
simile alla TBP e un fattore Transcription factor III C che riconosce sequenze a valle del proomotore detto Box A e Box B a cui si lega il TFIIIC che permette il reclutamento di TFIIIB
che promuove il reclutamento dell'RNA polimerasi III e la trascrizione dei geni per il tRNA, per i geni degli rRNA si trova anche il riconoscimento di un altro BoxB tra BoxA e BoxC. Nel
caso dell'RNA polimerasi III per l'rRNA 5s si trova un altro fattore TFIIIA che riconosce una sequenza detta BoxC a valle della zona promotrice e viene reclutato che facilita il 
reclutamento di TFIIIC che facilita il reclutamento di TFIIIB che recluta l'RNA polimerasi III, basta un fattore in pi\`u per rendere la trascrizione specifica per il gene per rRNA 5s. 
Utilizzando un fattore uin pu\`u che riconosce un boxC presente solo nei geni 5. Per gli RNA messaggeri la situazione \`e complessa riguardo le sequenze nel promotore: si trova la
TATA box, delle sequenze comuni come le sequenze BRE e altre riconosciute dai TFIID che si trovano a valle della zona promotrice e ci sono delle sequenze enhancher che si trovano 
molto distanti anche su cromosomi diversi che facilitano la trascrizione di geni specifici, i meccanismi di regolazione \`e complesso. l'RNA polimerasi II coinvolge 20 proteine. IL 
primo complesso che si lega alla TATA box \`e la TFIID che ha la TBP e fattori ad esso associati, grande 700 kilodalton e \`e costitituito da 11 fattori che riconoscono e si legano al 
DNA la struttura della TBP presenta dei domini ricchi di foglietti beta come una sella che quando riconosce la sequenza TATA si lega al DNA e gli fa fare una curva, importante in quanto
avvicina delle sequenze presenti nelle regioni al DNA facilitando il reclutamento di fattori importanti per la trascrizione: TFIIA che stabilizza l'interazione tra TFIID e il DNA e il 
fattore TFIIB la cui funzione \`e quella di marcare in maniera corretta il sito dove l'RNA polimerasi II va a legarsi al DNA. Permette all'RNA polimerasi di trascriveere partendo dal 
giusto nucleotide. Una volta formato il complesso TIID-A-B si ha il reclutamento dell'RNA polimerasi II con il fattore TFIIF che ha un ruolo simile al fattore sigma dei procarioti e si 
lega all'RNA polimerasi II quando l'eznima \`e liberi e gli impedisce di legare altre regioni del DNA e stabilizza le interaizoni tra RNA polimerasi e i tafari TFIIB-D. La coda rimane 
libera in quanto pu\`o andare incontro a modifiche ad opera di fattori reclutati successivamente come il fattore TFIIH, una chinasi con la funzinoe di fosforilare alcuni amminoacidi 
presenti sulla coda che rende favorisce la clearence del promotore, il meccanismo con cui l'RNA Polimerasi abbandona il promotore e inizia l'attivit\`a di allungamento. 
\subsection{Terminazione in eucarioti}
Sono stati proposti dei modelli in quanto non \`e ancora chiaro come possa avvenire:
\begin{itemize}
	\item Modello allsoterico: o antiterminatore \`e un modello che ricorda il modello nel meccanismo di blocco trascrizionale nei procarioti Rho indipendente: si forma una 
		struttura a stem loop che favorisce la dissociazione del complesso di trascrizione.
	\item Modello a torpedo: l'RNA trascritto una votla hce \`e stat trascritta la regione di poliadenilazione viene tagliato un'enzima che la riconosce e lo taglia. L'RNA tagliato 
		p\`o andare incontro a modifiche e l'RNA polimerasi una volta tagliato l'RNA interviene una esonucleasi che inizia a degradare l'RNA rimanente e arriva a un certo punto
		in cui comporta il distacco e dissociazione dell'RNA polimerasi e il blocco della trascrizione.
\end{itemize}
Vi sono sequenze comuni a livello della $3'$ UTR detto sito di poliadenilazione AAUAA nella $3'$ UTR, sequenza presente in tutti gli mRNA che hanno aggunta di poli A importante in 
quanto legata alla proteine PABP (poly A binding protein che si lega alla coda e stabilizza l'RNA), a livello della $5'$ UTR va incontro ad una modifica detta capping con l'aggiunta
di una modifica nel primo nucleotide normalmente un A gli si aggiunge una 7 metil guanosina con il compito di stabilizzare il trasporto e la traduzione. In $5'$ e $3'$ ci sono segnali
comuni ma altre specifiche al gene. RNA messaggeri con sequenze riconosciute da proteine che stabilizzano l'RNA. 

* NON C'E registrazione, suppongo inizio di NES e NLS *

\section{Trasporto nucleare}
La sequenza NES (nuclear export sequenze) permette alle proteine localizzate nel nucleo di uscire da esso. Una proteina con un NLS (nuclear localization sequence) andr\`a dentro e 
fuori dal nucleo. Porteine con solo NLS vanno nel nucleo e vi rimangono. Pu\`o essere vero in proteine che lgano RNA hanno NLS ma non NES in quanto entrano nel nucleo grazie all'NLS
e escono in quanto esportate insieme all'RNA. Molto spesso la predizione bioinformatica pu\`o dire se la proteina contiene un NLS e un NES (meno accurata) tuttavia si deve sempre 
verificare se la predizione \`e corretta oppure no. Una proteina contiene un NLS e un NES. Per verificare se queste due sequenze sono finzionaenti si prova a vedere dove sitrova la 
proteine attraverso l'immunostaining con anticorpo. Si pu\`o trovare nel nucleo, nel citoplasma o in entrambi. Se la si trova esclusivamente nel nucleo si trova un NLS funzionente, se la 
sitrova solo nel citoplasma NES \`e pi\`u efficiente o l'NLS non funziona in quanto pu\`o necessitare di modifiche. se si trova in entrambi i compartimenti presenta entrambe le sequenze.
\`E importante per chiarire il trafficking all'intreno della cellula. Il movimento nucleo citplasma avviene grazie alla presenza di fattori detti importine ed esportine. L'importina \`e
responsabile del trasporto dal citoplasma al nucleo di proteine che contengono l'NLS. Si lega al cargo il complesso importina cargo passa il poro nucleare, nel nucleo l'importina lega 
la RanGTP, una GTPasi che determina il distacco dell'importina dall cargo che rimane nel nucleo, l'importina e la Ran viene esportata, si idrolizza GTP la Ran GTP si separa. L'esportina
lega il RanGTP che permette al complesso esportina RanGTP di legarsi alla proteina cargo che contiene il NES che lascia il nucleo e va il citoplasma. RANGTP viene idrolizzato, cambio
conformazionale, dissociaizone del complesso trimerico e e distacco dell'esportina dal cargo e l'esportina pu\`o ritornare nel nucleo. Per legarsi al cargo l'esportina deve legarsi a 
RanGTP. La leptomicina B \`e utilizzata per lo studio di questo trafficking in quanto va a competere con l'esportina, con il cargo per il legame all'esportina legandosi ad essa che non
lega pi\`u il cargo e avendo proteine esportate dal nucleo al citoplasma aggiungendo la leptomicina B la proteina rimane nel nucleo. Tutte le proteine vengono esportate anche in 
altri modi e questo avviene in quanto le proteine nucleari vanno incontro a degradazione per il turnover. Il trafficking nucleo-citoplasma \`e regolato da importine ed esportine ma pu\`o
avvenire in altri modi anche attraverso diffusione. Inibendo l'importo proteine nel nucleo non si trovano nel citoplasma. La sequenza NLS \`e ricca in arginina e lisina. Per essere sicuri
che le sequenze funzionino come NLS si provano con dei mutanti andando a modificare delle sequenze. La sequenza di NES tipica \`e caratterizzata da una leucina seguita da tre amminoacidi,
una leucina, due amminoacidi, leucina, un amminoacido e isoleucina o leucina. Il motivo di questa complicazione \`e che questi amminoacidi in mezzo possono variare in numero. Con molta 
pi\`u variabilit\`a. 

* NON C'\`E LA REGISTRAZIONE DIO BONINO * 
Regolazione della trascrizione dell'operone lac e come la cellula batterica regola la trascrizione con i livelli di lattosio e glucosio utilizzando la repressione e il lattosio come
induttore e il repressore come lac e la proteina K che lega cAMP e i cui livelli sono legat alla quantit\`a di glucosio presenti all'interno della cellula. La zona promotrice del
gene policistronico per gli enzimi del metabolismo del lattosio presenta un sito di legame per la proteina CAP a monte della sequenze riconosciuta dall'RNA polimerasi e si trova la 
regione di DNA che viene riconosciuta dal repressore LAC che legandosi nella regione impedisce il legame dell'RNA polimerasi bloccando la trascrizione. Questo sistema \`e stato 
utilizzando cellule parzialmente diploidi per l'operone lac: mutazioni a carico dell'operatore e del repressore cercando di complementare il fenotipo utilizzando un DNA circolare plasmide
inserendolo nelle cellule mutanti e andando a vedere se il plasmide \`e in grado di ripristinare il sistema di controllo. Si \`e visto che mutazionoi a carico dell'operatore e della
regione riconosciuta dal repressore non possono essere salvate con questo meccanismo in quanto queste mutazioni avvengono sul DNA e agiscono in cis a valle sullo stesso filamento e 
mutazioni dell'operatore il plasmide inserito non \`e in grado di rendere il batterio normale e rispondente al lattosio. A differenza di mutaizoni che avvengono a carico del gene che
codifica il repressore in quanto il plasmide codifica per un represore il quale poi si lega alla sequenza dell'operatore. Il gene che trascrive per il repressore stesso agisce in trans. 
\section{Lezione 10}
Sono state isolate altre mutazioni che accadono sulla proteina CAP a cui si lega il cAMP che possono essere artificiali o naturali possono andare a colpire diverse regioni della proteina
come N terminale che caratterizzata le sequenze ad alpha elica che interagiscono con il solco maggiore o minore del DNA, mutazioni a questo libello impediscono alla proteina di legare
al DNA, nella zona C terminale la rendono incapace di legare il cAMP, un mutante dominante represssivo e non in grado di legare il cAMP e di legare il DNA e regolare e potenziare la 
trascrizione in assenza di glucosio. 
\subsection{Operone triptofano}
Per il metabolismo del triptofano, un amminoacido importante per il funzionamento della cellula e il batterio va ad attivare un'operone detto operone del triptofano che trascrive un
RNA messaggero con almeno 5 geni la cui funzione \`e di sintetizzare il triptofano in cui c'\`e carenza dell'amminoacido anche nel caso del lattosio che viene scisso e la cellula diventa
in grado di produrre una serie di enzimi che producono l'amminoacido. In questo caso l'operone triptofano possiede un operatore legato da un fattore di regolazione e a monte del operone
si trova una sequenza leader che si tratta di una sequenza di 140 nucleotidi caratterizzata da una piccola regione codificante per un peptide leader, quattro regioni trascritte e 
caratterizzate da una certa complementarit\`a e in grado di creare strutture a doppie filamento (1-2, 3-4), una coda di poli U e una terminazione del peptide leader e l'inizio del mRNA
che codifica per gli enzimi coinvolti nel metabolismo del triptovano. Il peptide leader ha funzione regolatoria e contiene almeno due amminoacidi triptofano di conseguenza viene prodotto
quando il triptofano \`e presente con libelli di triptifano sufficieni. IN presenza di scarsa quantit\`a di triptofano il peptide ha problemi ad essere tradotto. In assenza di sintesi 
proteica le sequenza 1 e 2 e 3 e 4 interagiscono tra di loso. 3 e 4 funziona come terminazione Rho indipendente. Questa struttura di 3 e 4 si comporta come terminazione Rho indipendente 
destabilizzandol'RNA polimrerasi. L'RNA viene trascritto e in presenza di triptofano il ribosoma inizia a tradurre il peptide leader e lo traduce il riboosma non va in stallo e impedisce
la formazione dell'ansa 1 e 2 ma non di 3 e 4, si trova un ribosoma in fase traduzionale la cui traduzione avviene contemporaneamente con la trascrizione e l'RNA polimerasi incontra il 
segmento 3 e 4 e si stacca e non trascrive per i geni coinvolti nel metabolismo del triptofano non trascrivendo per l'operone. Per basso triptofano il riboosma si blocca in quando il
peptide leader non avendo triptofano non continua e favorisce la formaizone di una struttura a stem loop tra 2 e 3 e la terminazione 3 4 non si forma e l'RNA polimrasi non viene 
stabilizzata e trascrive l'RNA per lintero ooperone e vengono prodotti gli enzimi per la sintesi del triptofano. Un meccanismo diverso che sottolinea l'importanza delle strutture 
secondarie che l'RNA pu\`o assumere in quanto la struttura pu\`o regolare un meccanismo. 
