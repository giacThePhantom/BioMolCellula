\chapter{Trascrizione}

\section{Panoramica}
La trascrizione \`e il meccanismo generale che porta alla produzione di RNA a partire dal DNA.

	\subsection{Produzione di RNA}

		\subsubsection{Lettura di DNA}
		Il RNA viene sintetizzato trascrivendo DNA.
		Il processo inizia con lo svolgimento di una piccola porzione della doppia elica.
		Le basi cos\`i esposte su un filamento agiscono come stampo per la sintesi di RNA.
		
		\subsubsection{Polimerizzazione}
		L'accoppiamento di basi complementari determina la sequenza di nucleotidi del RNA.
		Quando avviene una corrispondenza il ribonucleide viene legato con la creazione di un legame fosfodiesterico.

		\subsubsection{Separazione del RNA}
		Il filamento di RNA viene separato nella regione a monte dell'aggiunta e viene rilasciato come filamento singolo.
		Le molecole di RNA sono tipicamente pi\`u corte rispetto a quelle di DNA.

	\subsection{RNA polimerasi}
	Le RNA polimerasi sono gli enzimi responsabili della catalizzazione dell legame fosfodiestere tra i nucleotidi del RNA nascente.
	Estende la catena di RNA in direzione $5'$-$3'$ e non necessita di primer.
	Ha attivit\`a processiva.

		\subsubsection{Substrati}
		I substrati sono ribonucleoside trifosfato.
		La sua idrolisi fornisce l'energia necessaria alla reazione.

		\subsubsection{Temporizzazione}
		Il rilascio immediato del RNA vuol dire che le copie possono essere create in poco tempo, con la sinitesi di molecole addizionali prima che le altre siano completate.

		\subsubsection{Tasso di errore}
		La RNA polimerasi compie un errore ogni $104$ nucleotidi.
		Questo tasso di errore alto \`e tollerato in quanto gli RNA hanno vita breve.

		\subsubsection{Proof-reading}
		La RNA polimerasi possiede un'attivit\`a di proofreading: se un ribonucleotide errato viene aggiunto pu\`o indietreggiare e il sito attivo asporta attraverso la sostituzione di un pirofosfato con una molecola d'acqua.
		Viene rilasciata infine una molecola di monofosfato.

		\subsubsection{Struttura}
		La RNA polimerasi possiede domini per:
		\begin{multicols}{2}
			\begin{itemize}
				\item Accogliere il DNA.
				\item Accogliere i nucleotidi.
				\item Riconoscere la sequenza di inizio.
				\item Far uscire il RNA.
			\end{itemize}
		\end{multicols}

			\paragraph{Core}
			Il core della RNA polimerasi \`e conservato tra eucarioti e procarioti.
			\`E pertanto un oloenzima composto da $5$ subunit\`a e un gruppo prostetico.

				\subparagraph{Subunit\`a $\mathbf{\beta\ \beta'}$}
				La subunit\`a $\beta'$ \`e la pi\`u grande seguita da $\beta$.
				Sono codificate dai geni \emph{rpoC} e \emph{rpoB}.
				Formano il sito attivo per la sintesi di mRNA.
				Componenti creano interazioni non specifiche con DNA e RNA.
	
				\subparagraph{Subunit\`a $\mathbf{\alpha'\ \alpha''}$}
				Le subunit\`a $\alpha'$ e $\alpha''$ sono uguali tra di loro e contengono un dominio $\alpha NTD$ (dominio N terminale) e uno $\alpha CTD$ (dominio C terminale).
				Il primo permette l'assemblaggio della polimerasi, mentre il secondo reagisce con il promotore sul DNA e coi fattori di regolazione.

				\subparagraph{Subunit\`a $\mathbf{\omega}$}
				La subunit\`a $\omega$ \`e la pi\`u piccola e stabilizza la RNA polimerasi assemblata.

			\paragraph{Polimerasi eucariote}
			Le RNA polimerasi eucariote contengono $9$ domini in pi\`u per le interazione con altre proteine.
			Ne esistono tre: RNA polimerasi $I$, $II$ e $III$.

			\paragraph{Solco centrale attivo}
			Il solco centrale attivo \`e simile a quello della DNA polimerasi.
			Contiene un solo ione \emph{$Mg^{2+}$}.
			Un altro viene aggiunto con un nuovo nucleotide ogni ciclo di sintesi e rilasciato con il pirofosfato.

	\subsection{Inibitori di RNA polimerasi}
	Gli inibitori della polimerasi sono molecole come $\alpha$ amanitina e actiomicina D che impediscono la trascrizione.
	Sono in grado di bloccare specificatamente una delle tre polimerasi.
	Vengono usati per determinare l'emivita di un RNA, bloccando la sua trascrizione e osservando il graduale abbassamento di concentrazione attraverso PCR ed elettroforesi.

	\subsection{Categorie di RNA prodotto}
	Il prodotto finale dei geni pu\`o essere una proteina o una molecola di RNA.
	Nel secondo caso queste hanno funzione strutturale o catalitica.
	Possono regolare l'espressione genica.
	Si notano pertanto diverse categorie di RNA:
	\begin{multicols}{2}
		\begin{itemize}
			\item mRNA: messaggero, per la produzione di componenti enzimatici strutturali e regolatori.
			\item tRNA: transfer, per il trasporto degli amminoacidi per la traduzione.
			\item rRNA: ribosomiale, per la formazione dei nuclei catalitici dei ribosomi.
			\item snRNA: small nuclear RNA per lo splicing (\emph{U1,2,4,5,6}) in complesso con proteine.
			\item microRNA e piccoli RNA interferenti: regolano l'espressione genica.
			\item piRNA: proteggono le linee germinali dai trasposoni.
			\item Lunghi RNA non codificanti: sono impalcature o possono avere un ruolo nella regolazione.
		\end{itemize}
	\end{multicols}

\section{Trascrittosoma}
Si intende per trascrittosoma il complesso multienzimatico responsabile della trascrizione.

	\subsection{Composizione}

		\subsubsection{Elicasi}
		Le elicasi aprono la doppia elica.

		\subsubsection{Fattori di trascrizione}
		I fattori di trascrizione o \emph{TF} sono fattori che aiutano la trascrizione.
		Per esempio \emph{TFIID} e \emph{TFIIH} sono fattori di proofreading e di reattivazione della RNA polimerasi instabile.

		\subsubsection{RNA polimerasi}

			\paragraph{RNA polimerasi $\mathbf{I}$}
			La RNA polimerasi $I$ si occupa della produzione degli rRNA tranne il $5S$.

				\subparagraph{Reclutamento di fattori}
				La RNA polimerasi $I$ recluta:
				\begin{multicols}{2}
					\begin{itemize}
						\item \emph{UBF}: upstream binding factor, si lega a monte del promotore piegando il DNA e avvicinando i fattori tra di loro.
							Agisce come tetramero.
						\item \emph{SL1} o fattori \emph{TBP} ($TATA$ box binding protein), riconoscono la sequenza $TATA$ box.
					\end{itemize}
				\end{multicols}
				La formazione del tetramero tra \emph{UBF} e i \emph{SL1} come \emph{TBP} e \emph{TAF} crea un tetramero imponendo al DNA una struttura a sella che permette il corretto reclutamento della RNA polimerasi sul DNA.

			\paragraph{RNA polimerasi $\mathbf{II}$}
			La RNA polimerasi $II$ produce mRNA e microRNA.
			Si occupa pertanto della trascrizione dei geni produttori di proteine.
			Possiede una coda terminale strutturata con tirosina, serina e treonina.
			Pu\`o subire modifiche post-traduzionali per reclutare fattori coinvolti nello splicing.
			Durante l'allugnamento la coda impedisce che la RNA polimerasi si stacchi prima della fine del gene.

				\subparagraph{Reclutamento di fattori}
				La RNA polimerasi $II$ recluta:
				\begin{multicols}{2}
					\begin{itemize}
						\item \emph{TF2D}: riconosce la $TATA$ box binding protein e i complesso \emph{TAF} (\emph{TBP} associated factor), formando un primo complesso.
						\item \emph{TF2A}: stabilizza il complesso appena formato.
						\item \emph{TF2B}: recluta e posiziona RNA polimerasi in maniera corretta sul promotore.
						\item \emph{TF2F}: stabilizza l'interazione della RNA polimerasi con \emph{TBP} e \emph{TF2B}.
						\item \emph{TF2E}: attrae e regola \emph{TF2H}.
						\item \emph{TF2H}: ha attivit\`a elicasica e chinasica: fosforila infatti la coda della RNA polimerasi $II$.
						\item La fosforilazione della coda che rimane libera permette l'allungamento del filamento.
					\end{itemize}
				\end{multicols}

			\paragraph{RNA polimerasi $\mathbf{III}$}
			La RNA polimerasi $III$ produce tRNA e rRNA $5S$.

				\subparagraph{Reclutamento di fattori}
				La RNA polimerasi $III$ recluta:
				\begin{multicols}{2}
					\begin{itemize}
						\item \emph{TFIIIC} e \emph{TFIIIB} che riconoscono una \emph{BoxA} a monte e una \emph{BoxB} a valle del gene.
						\item \emph{TFIIIB} e \emph{TFIIIC} reclutano la RNA polimerasi $III$.
						\item La trascrizione del rRNA $5S$ richiede un ulteriore riconoscimento di un \emph{BoxC} all'interno del gene.
					\end{itemize}
				\end{multicols}



\section{Processo della trascrizione}

	\subsection{inizio}
	La trascrizione accurata di un gene richiede il riconoscimento del suo inizio e fine.
	L'iniziazione \`e pertanto il processo di regolazione di quali proteine devono essere prodotte e a quale velocit\`a.

		\subsubsection{Fattori sigma}
		I fattori $\sigma$ sono subunit\`a addizionali che si associano alla RNA polimerasi e la aiutano alla lettura del DNA batterico.
		Il fattore associato all'oloenzima aderisce debolmente al DNA fino a quando trova una sequenza consenso detta protomero.
		Riconosce sequenze a $-35$ e a $-10$ nelle zone promotrici.
		A quel punto si lega strettamente e d\`a inizio alla trascrizione.
		
			\paragraph{Fattori basali}
			I fattori basali sono sequenze minime nella zona promotrice che consentono l'attacco della RNA polimerasi.
			Vengono riconosciute dai fattori $\sigma$.
			\begin{multicols}{2}
				\begin{itemize}
					\item \emph{CAT box}: sono triplette \emph{CAT} in posizione $-35$.
					\item \emph{TATA box}: sequenze ricche in $A$ e $T$ in posizione $-25$, riconosciute dalla RNA polimerasi $II$.
				\end{itemize}
			\end{multicols}



		\subsubsection{Bolla di trascrizione}
		Dopo il legame con il protomero si apre una bolla di trascrizione di circa $10$ nucleotidi.
		Il filamento non legato dall'oloenzima agisce come stampo.

		\subsubsection{Modelli di inizio}
		Dopo il riconoscimento del promotore da parte dei fattori $\sigma$ la RNA polimerasi si lega al DNA.
		Si forma il complesso chiuso sul promotore e si svolge il DNA.
		Il complesso diventa aperto sul DNA svolto.

			\paragraph{Passaggio transiente}
			Nell'iniziazione a passaggio transiente la RNA polimerasi si attacca al DNA, avanza sintetizzando un filamento abortivo e torna indietro.

			\paragraph{A bruco}
			Nell'iniziazione a bruco la RNA polimerasi si attacca, cambia conformazione favorendo la formazione della bolla e trascrive un frammento.

			\paragraph{Scrunching}
			Il meccanismo di scrunching \`e responsabile della sintesi dei primi $10$ nucleotidi.
			La RNA polimerasi rimane attaccata al protomero e tira il DNA nel sito attivo espandendo la bolla di trascrizione.
			Lo stress generato causa un rilascio delle catene e un reinizio della sintesi.
			Questa iniziazione abortiva viene superata e lo stress separa l'enzima dal protomero e dal fattore $\sigma$.

		\subsubsection{Iniziazione per la RNA polimerasi $\mathbf{II}$}
		La RNA polimerasi $II$ richiede un insieme di fattori di trascrizione generali affinch\`e possa trascrivere.
		Questi la aiutano a posizionarsi al promotore, a separare i filamenti e ad essere rilasciata dal promotore in modo da iniziare con l'allungamento.
		Sono analoghi ai fattori $\sigma$ procarioti.
		\begin{multicols}{4}
			\begin{itemize}
				\item \emph{TFIIA}.
				\item \emph{TFIIB}.
				\item \emph{TFIIC}.
				\item \emph{TFIID}.
			\end{itemize}
		\end{multicols}

			\paragraph{Processo di assemblaggio}
			\begin{multicols}{2}
				\begin{enumerate}
					\item \emph{TFIID} si lega a una \emph{TATA box} attraverso la subunit\`a \emph{TBP}.
					\item La distorsione del DNA nella \emph{TATA box} marca il promotore come attivo.
					\item La RNA polimerasi $II$ insieme ad altri fattori formano un complesso di iniziazione.
					\item La RNA polimerasi $II$ ottiene l'accesso al filamento stampo.
					\item \emph{TFIIH} idrolizza \emph{ATP} e svolge il DNA.
					\item La RNA polimerasi $II$ rimane al promotore sintetizzando corte lunghezze fino a subire una serie di cambi conformazionali che le permettono di spostarsi.
					\item Viene fosforilata la RNA polimerasi $II$  da \emph{TFIIH}.
					\item I fattori di trascrizione generali si dissociano dal promotore.
				\end{enumerate}
			\end{multicols}

			\paragraph{Regolazione}

				\subparagraph{Enhancers}
				Gli enhancers sono sequenze di DNA a cui si legano gli attivatori trascrizionali.
				Questi reclutano la RNA polimerasi $II$ al punto di inizio.

				\subparagraph{Mediatore}
				Il mediatore \`e un complesso proteico che permette alle proteine attivatrici di comunicare con la RNA polimerasi $II$ e con i fattori di trascrizione generali.

				\subparagraph{Modifica della cromatina}
				Affinch\`e la RNA polimerasi $II$ possa trascrivere devono intervenire complessi rimodellatori della cromatina e modificatori degli istoni che decondensano la cromatina.

		\subsubsection{Fattori di trascrizione}

		\subsubsection{Attivatori, mediatori e proteine per la modifica della cromatina}

	\subsection{Allungamento}
	L'allungamento inizia quando la polimerasi ha sintetizzato un corto frammento di RNA di $10$ basi.
	L'enzima separa i due filamenti e li riavvolge dopo il suo passaggio mentre sintetizza il RNA.
	Mentre la reazione prosegue stacca la catena dallo stampo.
	Ha una funzione di proofreading.
	La catena viene quindi aperta ulteriormente in modo da iniziare un processo continuativo.
	Nei batteri $\sigma^4$  libera il canale per la fuoriuscita di RNA.
	La formazione del legame fosfodiesterico causa la liberazione di pirofosfato tamponato con \emph{$Mg^{2+}$}.

		\subsubsection{Meccanismi di correzione}

			\paragraph{Editing pirofosforico}

			\paragraph{Editing idrolitico}
			Nell'editing idrolitico la RNA polimerasi taglia a $2$ o $3$ nucleotidi a monte dell'errore commesso e torna indietro.

		\subsubsection{Andamento dell'allungamento}
		Una volta che la RNA polimerasi ha iniziato la trascrizione si muove a scatti.
		Viene legata da una serie di fattori di allungamento.

			\paragraph{Fattori di allungamento}
			Si intende per fattori di allungamento quelle proteine che diminuiscono la probabilit\`a che la RNA polimerasi si dissoci dal DNA prima di aver finito la trascrizione.
			Si associano ad essa dopo l'iniziazione.
			Possono modificare gli istoni lasciando una traccia che aiuta le trascrizioni successive e coordina l'allungamento.
			
			\paragraph{Tensione superelicale}
			La tensione superelicale formata dalla RNA polimerasi viene rilassata da DNA topoisomerasi.

		\subsubsection{Processamento del RNA}
		Negli eucarioti la trascrizione \`e accoppiata con il processamento del RNA.

			\paragraph{Terminazioni}
			La terminazione $3'$ viene modificata con poliadenliazione, mentre la $5'$ viene metilata.
			Le modifiche creano un meccanismo di verifica prima che il RNA lasci il nucleo

			\paragraph{RNA splicing}
			Nel processo di splicing vengono rimossi gli introni degli RNA.
			Splicing alternativi permettono la creazione di proteine diverse da uno stesso gene.

	\subsection{Terminazione}
	Durante la terminazione l'enzima rilascia il RNA prodotto e si dissocia dal DNA.

		\subsubsection{Procarioti}

			\paragraph{Rho indipendente}
			La terminazione Rho indipendente consiste di regioni a forcina sul RNA che destabilizzano l'interazione tra RNA polimerasi e DNA.
			Non necessita di fattori proteici.
			Sono sequenze palindromiche ricche di $C$ e $G$ seguite da sequenze ricche di $A$ e $T$ palindromiche.

			\paragraph{Rho dipendente}
			Nella terminazione Rho dipendente la proteina Rho lega il RNA e riconosce sequenze \emph{RUT}.
			Si sposta fino a incontrare la RNA polimerasi destabilizzandola e causando la sua dissociazione dal DNA.
			Questa sequenza si trova a valle del sito dove avviene la traduzione.

		\subsubsection{Eucarioti}

			\paragraph{Modello a torpedo}
			Nel modello a torpedo una RNAasi interagisce con la RNA polimerasi ma non degrada il RNA protetto dal cap metilico.
			La RNA polimerasi continua la traduzione fino all'aggiunta di una sequenza poliadenilata dove la RNAasi taglia il RNA prodotto causando la destabilizzazione del complesso e la terminazione.

			\paragraph{Modello allosterico}
			Nel modello allsoterico dopo il taglio del sito di poliA $AAUAAA$ un cambio conformazionale porta al distacco della RNA polimerasi.


\section{Trasporto tra nucleo e citoplasma}
Le proteine vengono prodotte nel citoplasma, ma alcune devono essere poi trasportate nel nucleo per poter svolgere la loro funzione.
Queste pertanto possiedono una particolare sequenza segnale \emph{NLS}: nuclear localization sequence.
Le proteine che invece dal nucleo devono esserne esportate presentano una \emph{NES}: nuclear export sequence.

	\subsection{Complessi coinvolti}

		\subsubsection{Complesso di importazione}
		Il complesso di importazione \emph{Importing karyopherin} riconosce la \emph{NLS}, una regione ricca di lisina o arginina per l'importazione nel nucleo.
		Trasporta la proteina dal citoplasma al nucleo.

		\subsubsection{Complesso di esportazione}
		Il complesso di esportazione \emph{Exporting karyopherin} riconosce \emph{NES}, una regione contenente leucine o isoleucine.
		Trasporta la proteina dal nucleo al citoplasma.

	\subsection{Verificare la localizzazione}

		\subsubsection{Eterocaryon essay}
		L'eterocaryon essay \`e un processo che permette di distinguere se una proteina viene importata nel nucleo.
		Per farlo si fondono due cellule di specie diverse in un'unica.
		Questo si volge attraverso \emph{PEG} che permette la fusione delle membrane.
		Questo sistema permette di studiare se una proteina nel nucleo di una cellula si muove dal nucleo al citoplasma e viceversa.
		Si trasfettano le cellule umane con la proteina fusa con \emph{GFP}.
		Dopo che \`e avvenuta la fusione se la proteina nucleare si sposta anche nel citoplasma entra nel nucleo della cellula murina.

	\subsection{Processo di importo ed esporto}
	Un importante fattore che d\`a direzionalit\`a al trasporto \`e la proteina \emph{RAN}.
	Questa pu\`o legare \emph{GTP} o \emph{GDP} in base alla regione della cellula in cui si trova.

		\subsubsection{Importazione}
		L'importazione nel nucleo \`e guidata da importine: queste si legano al cargo e passano nel nucleo.
		In questo momento \emph{RanGTP} lega l'importina causandone il distacco dal cargo.
		L'importina scarica legata a \emph{RanGTP} viene esportata e nel citoplasma avviene l'idrolisi in \emph{RanGDP}, rendendo l'importina nuovamente disponibile.

		\subsubsection{Esportazione}
		L'esportazione dal nucleo \`e guidata da esportine: queste legano \emph{RanGTP}.
		Il legame causa un cambio conformazionale che causa il legame con il cargo.
		A questo punto abbandona il nucleo e l'idrolisi in \emph{RanGDP} permette il distacco del cargo dall'esportina.

		\subsubsection{Leptomicina $\mathbf{B}$}
		La leptomicina $B$ \`e una tossina di origine fungina che compete con il cargo per il legame con l'esportina.
		Impedisce la sua esportazione.

\section{Meccanismi di regolazione della trascrizione e dell'espressione genica}

	\subsection{Operoni}
	Si intendono per operoni gruppi di geni adiacenti trascritti come un'unica molecola di mRNA policistronica tradotta nei procarioti in proteine diverse.

	\subsection{Induzione}
	Si intende per induzione un meccanismo di controllo che viene attivato in presenza di una molecola induttore.

		\subsubsection{Induzione a controllo negativo}
		Nell'induzione a controllo negativo un repressore viene represso in presenza di  un induttore.
		Questo causa una trascrizione attiva.

		\subsubsection{Induzione a controllo positivo}
		Nell'induzione a controllo positivo l'attivatore inattivo viene attivato dall'induttore causando una trascrizione attiva.

	\subsection{Repressione}
	Si intende per repressione un controllo che viene disattivato in presenza di un repressore, prodotto da una sequenza che per ingombro sterico impedisce il reclutamento della RNA polimerasi.

		\subsubsection{Repressione a controllo negativo}
		Nella repressione a controllo negativo l'attivazione del repressore reprime la trascrizione.

		\subsubsection{Repressione a controllo positivo}
		Nella repressione a controllo positivo l'attivatore viene disattivato da un repressore disattivando la trascrizione.

	\subsection{Enhancer}
	Negli eucarioti oltre agli elementi basali necessari per una trascrizione basale si trovano sequenze che aumentano l'efficienza del processo.

		\subsubsection{Caratteristiche}
		\begin{multicols}{2}
			\begin{itemize}
				\item Distanza di migliaia di basi o cromosomi diversi rispetto al gene che influenzano.
				\item Orientamento non specifico.
				\item Posizione a monte o a valle del gene.
			\end{itemize}
		\end{multicols}
		Si legano a fattori trascrizionali con un sito di riconoscimento per il DNA e un dominio di trans-atticazione che pu\`o essere aspecifico che pu\`o reclutare altri fattori.

		\subsubsection{Domini caratteristici}

			\paragraph{Helix-Turn-Helix}
			Il dominio Helix-turn-helix \`e composto da due $\alpha$-eliche legate da un dominio flessibile.
			Interagiscono con il solco maggiore del DNA.

			\paragraph{Domini zinc-finger}
			I domini zinc-finger contengono un $\alpha$-elica e uno ione zinco che interagisce con istidine o ciseine.
			Sono specifici per la sequenza.

			\paragraph{Leucine zipper}
			Le leucine zipper si trovano nei fattori trascrizionali e vengono mantenute da leucine che formano un nucleo idrofobico che permette l'interazione di due catene tra di loro.

	\subsection{Esempi}

		\subsubsection{Operone \emph{lac}}
		L'operone \emph{lac} \`e formato da $3$ geni $Z$, $Y$ e $A$ tradotti in $\beta$-galattosidasi, permeasi (assimilazione del lattosio) e transacetilasi.
		\`E un esempio di repressione da catabolita.

			\paragraph{Repressione da catabolita}
			L'operone \emph{lac} presenta un esempio di repressione da catabolita: la sua trascrizione \`e infatti regolata dalla presenza o assenza di lattosio e \emph{cAMP}.

				\subparagraph{Assenza di glucosio e lattosio}
				In assenza di lattosio e glucosio i livelli di \emph{cAMP} sono alti: \emph{CAP} (\emph{cAMP} receptor protein) si lega sul sito a monte del promotore.
				L'assenza del lattosio causa la presenza del repressore e la trascrizione dell'operone \`e disattiva.

				\subparagraph{Assenza di lattosio e presenza di glucosio}
				In assenza di lattosio e presenza di glucosio il repressore \`e presente, \emph{CAP} non si lega al sito in quanto \emph{cAMP} \`e a bassi livelli.
				La trascrizione \`e repressa.

				\subparagraph{Presenza di glucosio e lattosio}
				In presenza di glucosio e lattosio viene rimosso il repressore dall'operatore, ma i bassi livelli di \emph{cAMP} causano una trascrizione basale.

				\subparagraph{Presenza di lattosio e assenza di glucosio}
				In presenza di lattosio e assenza di glucosio il repressore viene rimosso e \emph{CAP} si lega al sito di attacco a causa degli alti livelli di \emph{cAMP}.
				La trascrizione si attiva in modo efficiente.

		\subsubsection{Operone triptofano}
		L'operone triptofano codifica per enzimi che portano a sintesi del triptofano.
		Viene attivato in sua assenza.

			\paragraph{Struttura}
			L'operone triptofano \`e formato in sequenza da:
			\begin{multicols}{2}
				\begin{itemize}
					\item Operatore.
					\item Fattore di regolazione.
					\item Sequenza leader.
					\item $4$ regioni trascritte.
					\item Coda poli-U.
					\item Terminazione leader.
				\end{itemize}
			\end{multicols}
			La sequenza leader \`e un sensore per il triptofano: possiede molti codoni per la sintesi di tale amminoacido nella sua sequenza.
			Le $4$ regioni trascritte inoltre possono interagire tra di loro.

			\paragraph{Meccanismo di regolazione}
			Questo meccanismo di regolazione sfrutta la simultaneit\`a tra trascrizione e traduzione nei batteri.
			La disponibilit\`a del triptofano infatti determina la velocit\`a di traduzione della sequenza leader e quali delle regioni trascritte interagiscono tra di loro.

				\subparagraph{Assenza di triptofano}
				In assenza di triptofano lo stallo del ribosoma sul peptide leader permette il reclutamento di enzimi per tradurre i geni a valle e si forma una forcella tra le regioni $2$ e $3$ che non \`e un terminatore.
				Gli enzimi vengono pertanto trascritti.

				\subparagraph{Presenza di triptofano}
				In presenza di triptofano il peptide leader viene completamente tradotto e non permette l'associazione tra le regioni $2$ e $3$ a causa dell'ingombro sterico del ribosoma.
				Si forma una struttura a forcella tra $3$ e $4$ che forma un sito di terminazione Rho indipendente che bocca la trascrizione.
				Gli enzimi non vengono pertanto trascritti.

\section{Modifiche post-trascrizionali}

	\subsection{Capping}
	Il capping \`e una modifica subita da tutti gli mRNA.
	Avviene prima che finisca la trascrizione, dopo che la coda della RNA polimerasi $II$ viene fosforilata.
	Viene aggiunta la $7$-metilguanosina \emph{m7GppM} al $5'$ del trascritto primario.

		\subsubsection{Funzione}
		\begin{multicols}{2}
			\begin{itemize}
				\item Protegge da degradazione.
				\item Trasporto.
				\item Stabilit\`a.
				\item Traduzione.
				\item Splicing.
			\end{itemize}
		\end{multicols}

		\subsubsection{Processo}
		\begin{multicols}{2}
			\begin{enumerate}
				\item La RNA tri-fosfatasi libera il gruppo $P$ al $5'$.
				\item La Guanilil-trasferasi: attacca \emph{GMP} al primo nucleotide del RNA.
				\item Metiltrasferasi: aggiunge un gruppo metilico alla guanosina $7$.
			\end{enumerate}
		\end{multicols}

		\subsubsection{Riconoscimento}
		Il riconoscimento della base metilata avviene grazie al cap-binding complex formato da \emph{CBP20+CBP80} che si legano ad essa nel nucleo e si dissociano nel citoplasma.
		Nel citoplasma vengono sostituiti da \emph{eIF4E} (eucariotic initiator factor $4E$).
		
		\subsubsection{Assenza del cappuccio}
		Se dopo la terminazione della trascrizione un mRNA non possiede un cappuccio viene degradato da un'esonucleasi trasportata lungo la coda della polimerasi.

	\subsection{Poliadenilazione}
	La poliadenilazione al $3'$ \emph{UTR} avviene ad opera dell'enzima \emph{poliA polimerasi}.
	\`E formata da ripetizioni di \emph{AAUAAA}.

		\subsubsection{Funzione}
		La lunghezza della coda influisce sulla stabilit\`a e sull'esportazione del RNA nel citoplasma.
		L'alterazione delle code causa mattie alterate.

		\subsubsection{Riconoscimento}
		La coda poli-A viene riconosciuta da fattori che permettono il legame con \emph{PABP1} nel nucleo e \emph{PABC} nel citoplasma (poli-A binding protein).

		\subsubsection{Processamento del RNA}
		I fattori coinvolti sono:
		\begin{multicols}{2}
			\begin{itemize}
				\item \emph{CPSF}: cleavage and polyadenilation specificity factor.
				\item \emph{CstF}: cleavage stimulation factor.
				\item \emph{CFIm}, \emph{CFIIm}: cleavage factor.
			\end{itemize}
		\end{multicols}
		\emph{CstF} e \emph{CPSG} viaggiano con la coda della RNA polimerasi e sono trasferite alla terminazione $3'$ quando emerge.
		Una volta che sono trasferite si assemblano sulle sequenze riconoscimento.
		Il RNA viene rotto dalla polimerasi e un enzima poli-A polimerasi \emph{PAP} aggiunge $200$ nucleotidi $A$ alla terminazione.
		Il precursore delle addizioni \`e \emph{ATP}/
		Le proteine che si legano ad essa si assemblano durante la sintesi della catena.


		\subsubsection{Lunghezza della coda}
		La lunghezza della coda \`e fondamentale in molti organismi.
		In \emph{Xwenopus} prima della fecondazione gli RNA sono in uno stato dormiente con una coda di poli-A corta.
		Dopo la fecondazione si assiste a un mutevole allungamento della coda e all'attivazione della traduzione.

			\paragraph{PCR poliA test \emph{PAT}}
			Il \emph{PAT} \`e un test usato per verificare le variazioni di lunghezza di poli-A.
			Per farlo un primer a molte $T$ viene legato alla coda.
			Dopo migrazione su lastra d'agarosio pi\`u lunga la coda pi\`u lenta la migrazione.
			La lunghezza della coda va ad influire sulla stabilit\`a della base traduzionale.

		\subsubsection{Distrofia muscolare oculo-faringe}
	
			\paragraph{Analisi clinica}
			La distrofia muscolare oculo-faringe colpisce tra i $40$ e i $60$ anni e va in lenta progressione.
			I sintomi sono:
			\begin{multicols}{2}
				\begin{itemize}
					\item Abbassamento delle palpebre.
					\item Difficolt\`a a deglutire.
					\item Debolezza muscolare.
					\item Progressiva paralisi.
				\end{itemize}
			\end{multicols}

			\paragraph{Analisi molecolare}
			Attraverso microscopia sono presenti accumuli di RNA in foci del nucleo.
			Questo \`e dovuto a mutazioni nella poli-A binding protein $N$ che non permette un esporto corretto del mRNA.

		
	\subsection{Splicing}
	Lo splicing \`e un evento di processamento del RNA che d\`a origine alla maggiore variabilit\`a genetica.
	Un RNA trascritto negli eucarioti presenta sezioni trascritte ma non tradotte.

		\subsubsection{Caratterizzazione RNA}
		Il RNA viene pertanto diviso in sequenze introniche ed esoniche alternate.
			
			\paragraph{Esoni}
			Si intende per esoni le sequenze espresse che portano a RNA codificante.

			\paragraph{Introni}
			Si intende per introni le sequenze intercalate che producono long non coding RNA.
			Non codificano la proteina ma producono RNA con ruolo funzionale con attivit\`a catalitica.

			\paragraph{Confini}
			Le zone di confine tra esoni ed introni sono conservate: al $5'$ dell'introne si trova il sito donatore $GU$, mentre al $3'$ il sito $AG$ accettore.
			In queste zone avviene la rottura del legame fosfodiesterico e possono intervenire \emph{hnRNP}, proteine ribonucleari eterogenee che svolgono le eliche a hairpin del RNA per far leggere i segnali di splicing e identificare mRNA maturo.
		
		\subsubsection{Nuclear speckles}
		Le nuclear speckles sono i luoghi del nucleo dove avviene lo splicing.

		\subsubsection{Tipologie di introni}

			\paragraph{Introni auto-catalitici di gruppo $\mathbf{I}$}
			Gli introni auto-catalitici di gruppo $I$ sono in grado di compiere auto-splicing.
			Si trovano nel genoma di mitocondri, cloroplasti e nucleari.
			L'attacco nucleofilo avviene da parte di una guanosina esterne in trans.
			Dopo il taglio avviene il secondo attacco nucleofilo e il rilascio dell'introne lineare.

				\subparagraph{Homing}
				Si intende per homing il passaggio di un introne da un allele a un altro che non lo contiene.
				\`E un evento di trasposizione: l'introne produce una endonucleasi che riconosce i siti donatori e riceventi dell'allele e crea un taglio.
				Questo attiva i sistemi di riparazione del DNA che creano un appaiamento dei due alleli e l'introne viene inserito per complementariet\`a di basi.

			\paragraph{Introni del gruppo $mathbf{II}$}
			Gli introni del gruppo $I$ sono presenti nel genoma di mitocondri e plastidi delle piante.
			Compiono due attacchi nucleofili in trans: l'adenina interna all'introne con il gruppo \emph{OH} che attacca e l'introne viene rimosso attraverso la formazione di una struttura a cappio.

				\subparagraph{Retro-homing}
				Il retro-homing avviene per i trasposoni, elementi mobili all'interno del genoma: l'allele passa da un donatore a un ricevente.
				Contiene un \emph{ORF} (open readiing frame) che codifica per un'endonucleasi e una trascrittasi inversa.
				Questo taglia un filamento dell'allele ricevente, inserisce RNA e con la trascrittasi inversa copia il DNA complementare.

		\subsubsection{Trans-splicing}
		Il trans-splicing avviene tra RNA diversi: un RNA contiene un cap, un sito donatore e una parte intronica.
		Il secondo non contiene il cap, ma adenina, una sequenza ricca di pirimidine e un esone.
		Il gruppo \emph{OH} di $A$ va ad attaccare il sito donatore creando un taglio e rimuovendo l'introne.
		Il secondo attacco nucleofilo rompe i legame con l'introne e crea una struttura a $Y$.
		Avviene una ligazione tra i due esoni.
		In questo modo possono essere tradotti in maniera indipendente due RNA in $2$ proteine.
		I due cistroni hanno la coda poli-A ma non il cap.
		L'aggiunta del cap o sequenze diverse con questo meccanismo permette la traduzione e introduce variabilit\`a genetica.

		\subsubsection{Splicing del tRNA}
		Il tRNA contiene degli introni che vengono rimossi per permettere la formazione della struttura secondaria a trifoglio.

		\subsubsection{Spliceosomi}
		Lo splicing avviene normalmente all'interno delle nostre cellule attraverso spliceosomi.
	
			\paragraph{Struttura}
			Lo spliceosoma contiene oltre a proteina degli snRNA:
			\begin{multicols}{5}
				\begin{itemize}
					\item \emph{U1}.
					\item \emph{U2}.
					\item \emph{U4}.
					\item \emph{U5}.
					\item \emph{U6}.
				\end{itemize}
			\end{multicols}
			Lo spliceosoma \`e pertanto una snRNP: small nuclear ribonucleoprotein.

			\paragraph{Funzioni}
			Lo spliceosoma:
			\begin{multicols}{2}
				\begin{itemize}
					\item Permette il riconoscimento della coppia di siti di splicing tra una moltitudine di siti simili.
					\item Permette l'avvicinamento e il corretto posizionamento nel sito catalitico per far avvenire le due transesterificazioni correttamente.
					\item Il complesso \emph{snRNA} permette il riconoscimento di siti conservati.
				\end{itemize}
			\end{multicols}

			\paragraph{Siti di splicing}
			I siti di splicing sono i siti in cui avviene l'attacco nucleofilo, oltre al sito accettore e donatore sono riconosciuti dallo spliceosoma:
			\begin{multicols}{3}
				\begin{itemize}
					\item \emph{ESE}: exonic splicing enhancer.
					\item \emph{ESS}: exonic splicing silencer.
					\item \emph{DEXH/D box}: RNA elicasi.
					\item \emph{SC35}, \emph{ASF/SF2}.
					\item Domini \emph{RRM} e \emph{RS} modulari.
				\end{itemize}
			\end{multicols}
			Il macchinario riconosce in particolare $3$ porzioni principali: sito donatore, accettore e il punto di ramificazione che forma la base con il lazo.
			Quest informazioni sono sequenze di nucleotidi simili.

			\paragraph{Fattori coinvolti}
			\begin{multicols}{2}
				\begin{itemize}
					\item \emph{Snurp snRNP}.
					\item \emph{RNA binding protein}: regolano lo splicing alternativo \emph{EXE}.
					\item Fattori di regolazione per la rimozione di esoni.
					\item \emph{DSCAM}: esportazione dal nucleolo.
				\end{itemize}
			\end{multicols}
			
			\paragraph{Meccanismo di splicing}
			Il meccanismo di splicing coinvolge la formazione di diversi complessi che operano il processo.
			Idrolizza \emph{ATP} e la rimozione di un introne coinvolge due reazioni di trasferimento di fosforile o transesterificazioni sequenziali che legano gli esoni.
			La complessit\`a del processo assicura uno splicing accurato e flessibile.
			\begin{multicols}{2}
				\begin{enumerate}
					\item Complesso precoce early $E$: contiene:
						\begin{itemize}
							\item \emph{BBP}: branching binding protein e si lega ad $A$.
							\item \emph{U2AF65}: riconosce il sito accettore poli-pirimidinico.
							\item \emph{U2AF35}: riconosce la zona di passaggio tra sito accettore ed esone.
						\end{itemize}
					\item Complesso $A$: \emph{U2} rimuove \emph{BBP} attraverso appaiamenti specifici e la sostituisce.
					\item Complesso $B1$: avvicinamento alla $A$ reattiva, attacco nucleofilo al sito donatore mediato da \emph{U4-U5-U6} reclutati.
						Rimozione di \emph{U2A35-65}.
						Piegano gli introni avvicinando il sito per l'attacco nucleofilo.
					\item Complesso $B2$: rimozione di \emph{U1} e \emph{U6} interagisce con il sito donatore.
					\item Complesso $B^*$: rimozione di \emph{U4} (inibitore), \emph{U2} interagisce con \emph{U6} e avviene l'attacco nucleofilo tra \emph{BPP} e \emph{G}.
					\item Complesso $C1$: esone a valle e introne a cappio, si rompe il legame con introne a struttura a cappio.
					\item Complesso $C2$: avviene il secondo attacco nucleofilo, si legano i due esoni e si libera l'introne.
					\item RNA maturo: vengono rimossi tutti gli $U$ e si aggiungono \emph{EJC}.
				\end{enumerate}
			\end{multicols}

			\paragraph{Exon junction complex \emph{EJC}}
			Gli \emph{EJC} sono complessi proteici che marcano i punti di giunzione tra i vari esoni.
			La traccia lasciata permette un controllo per verificare uno splicing corretto.

			\paragraph{Struttura cromatinica}
			La struttura cromatinica influenza lo splicing: i nucleosomi tendono ad essere posizionati sugli esoni.
			La proteina responsabile per la loro definizione ad assemblarsi al RNA quando emerge dalla polimerasi.
			Cambi della struttura cromatinica possono cambiare i pattern di splicing.

				\subparagraph{Tasso di movimento}
				IL tasso di movimento della RNA polimerasi ha effetto sul tasso di splicing: minore la velocit\`a minore il salto di esoni.

				\subparagraph{Modifiche agli istoni}
				Specifiche modifiche agli istoni attraggono componenti dello spliceosoma che possono essere trasferiti al RNA emergente.

		\subsubsection{Splicing alternativo}
		La scelta di quali introni ed esoni mantenerne nel RNA finale non \`e fissata ma pu\`o variare in base alle necessit\`a della cellula.
		In particolare le proteine \emph{SR} funzionano da promatori di splicing, ma possono essere bypassate in presenza di altri siti di attrazione.

			\paragraph{Esempi}

				\subparagraph{\emph{caMK2}}
				\emph{caMK2} \`e una chinasi attivata da concentrazioni di \emph{$Ca^{2+}$} e pu\`o dare origine a splicing alternativi.
				In particolare pu\`o causare la rimozione di una \emph{NLS} modificando la localizzazione di una proteina e la sua funzione.

				\subparagraph{Gene fruitless}
				Differenze nello splicing possono creare in Drosophila dei mutanti con un comportamento aberrante: non sono in grado infatti di riconoscere il sesso.

		\subsubsection{Minigeni}
		Si intende per minigene un frammento minimo di gene che include un esone e le regioni di controllo necessarie alla sua espressione.
		Forniscono un modello prezioni per valutare modelli di splicing.
		Sono usati come giunzione o vettori esone-trapping e agiscono come sonda per determinare i fattori importanti nei processi di splicing.

		\subsubsection{Determinazione del sesso in Drosophila melanogaster}
		In Drosophila la determinazione del sesso dipende da un evento di splicing alternativo.
		In particolare \emph{Sxl} (sex lehtal) funge da regolatore dello splicing del suo stesso gene.
		Si lega a RNA per determinare uno splicing alternativo del gene dopo la fase precoce producendo una proteina funzionante in grado di interagire con il fattore \emph{TRA}.
		In complesso con tra regola lo splicing di \emph{Dsx} e \emph{TRA2}.
		In questo modo vengono disattivati i geni del differenziamento maschile.
		Nel maschio invece la proteina \emph{Sxl} precoce non viene prodotta e il trascritto rimane tronco e non funizonle.

	\subsection{RNA editing}
	Si intende per RNA editing l'alteraione delle sequenze nucleotidiche dei trascritti dopo la loro sintesi.
	In questo modo viene cambiata l'informazione che codificano tramite conversione di una base in un'altra, delezione o inserzione di nucleotidi.
	Viene svolto dall'editosoma.
	
		\subsubsection{Effetti}
		Il RNA editing pu\`o:
		\begin{multicols}{2}
			\begin{itemize}
				\item Modificare la stabilit\`a del RNA.
				\item Influenzare lo splicing.
				\item Modificare la sequenza degli amminoacidi o creare una proteina tronca.
				\item Trasporto di mRNA.
				\item Efficienza di traduzione.
				\item Appaiamento con microRNA e mRNA.
				\item Aumenta la varaibilit\`a dell'espressione genica.
			\end{itemize}
		\end{multicols}

		\subsubsection{Tipologie di editing}
		L'editing tipicamente consiste di inserzione o delezione di uracili, mRNA messaggiero prodotto dalla trascrizione e inserzioni rispetto al trascritto primario.

		\subsubsection{Processo}
		\begin{multicols}{2}
			\begin{enumerate}
				\item Ancoraggio.
				\item Taglio.
				\item Inserzione o delezione.
				\item Legatura.
			\end{enumerate}
		\end{multicols}
		La complementariet\`a al $5'$ permette l'instaurazione di legami tra le basi con omologia errata.
		Interviene un endonucleasi che riconosce il mismatch inducendo un taglio nel mRNA.
		Dopo il taglio si aggiunge una $U$ per ripristinare la complementariet\`a o tagliando qualche nucleotide.

		\subsubsection{Editing da citosina a uracile}
		Per produrre uracile da citosina questa viene deamminata.

			\paragraph{Esempi}

				\subparagraph{\emph{apoB}}
				La proteina \emph{apoB} si trova in due isoforme: $100$ e $48$.
				L prima regola il trasporto \emph{LDL} e viene riconosciuta da recettori nel fegato.
				La seconda viene espressa nell'intestino e non riconosciuta dai recettori del fegato.
				L'editing avviene attraverso \emph{APOBEC1}, una citidina deaminasi, che si lega al RNA grazie a \emph{ACF1}, il fattore di complementazione.

				\paragraph{Neurofibromatosi}
				La neurofibromatosi \`e una mutazione da citidina a uridina che porta a un codone di stop e il gene \emph{Nf1} (neurofibrina1) oncorepressore non inibisce \emph{ras} causando una proliferazione cellulare.
				\emph{NF1} normalmente lo disattiva aumentando l'idrolisi di \emph{GTPb}, ma un suo editing inserisce un codone di stop impedendo la sua funzione.


		\subsubsection{Editing da adenosina a iosina}
		La iosina viene prodotta deamminando l'adenosina.
		Avviene a carico di \emph{RNA binding protein}.

			\paragraph{Esempi}

				\subparagraph{ADAR}
				\emph{ADAR 1,2} (adenosina deaminasi editing on RNA) riconosce RNA a doppio filamento e localizzano nel citoplasma nucleo e nucleoli.
				Si trovano in due isoforme a $100$ e $150kDa$.
				Nell'uomo sono $3$ e si distinguono nella regione $N$-terminale
				Sono formate da:
				\begin{multicols}{2}
					\begin{itemize}
						\item Dominio catalitico per la rimozione di \emph{NH2}.
						\item Dominio $Z$ per legare DNA.
						\item Domini per legame del doppio filamento.
					\end{itemize}
				\end{multicols}

		\subsubsection{MicroRNA editing}
		Editando la sequenza del microRNA si modifica il mRNA che regola.


	\subsection{Funzioni delle modifiche}
	La sintesi e il processamento si svolgono in maniera ordinata, ma solo una piccola percentuale del pre-mRNA prodotto viene ulteriormente utilizzato: il resto, sintetizzato o processato erroneamente viene degradato in quanto inutile e dannoso.
	La distinzione tra mRNA maturo corretto ed aberrante la molecola di RNA mentre viene processata acquisisce e perde proteine.
	Quando le proteine presenti sulla molecola di mRNA segnalano che il processamento \`e completato questo viene esportato nel citosol dove viene tradotto.
	I restanti sono degradati dall'esosoma nucleare.


