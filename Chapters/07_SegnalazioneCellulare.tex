\chapter{Segnalazione cellulare: trasduzione del segnale}
Le cellule sono delle strutture con individualit\`a e isolate dall'ambiente circostante ma con cui devono rimanere in contatto per quanto riguarda la capacit\`a di ricevere segnali.
Le cellule hanno una membrana plasmatica sede di una serie di processi e importanti per permettere alle cellule di comunicare con l'ambiente esterno. Le cellule comunicano tra di loro
in 4 vie: una via contatto-dipendente, segnalazione paracrina, sinaptica e endocrina che le permettono di ricevere informazioni dall'esterno o da altre cellule. La comunicazione 
contatto-dipendente la cellula riceve segnale grazie all'interazione diretta con un'altra cellula mediata da recettore e ligando: la cellula che segnala esprime sulla propria membrana
recepito dalla cellula che riceve grazie a un recettore che riceve il segnale. SI neceessita di contatto. Nella paracrina il messaggio viene rilasciato nell'ambiente circostante che 
viene riconosciuto grazie alla presenza di recettori. Nella segnalazione sinaptica e la si trova nel sistema nervoso in cui si trova una cellula che riceve il segnale e il neurone che
lo rilascia sottoforma di vescicole che rilasciate in un ambiene circostante molto ristretto e il neurotrasmettitore si lega ai recettori sulla cellula target simile a quella contatto 
dipendente ma non c'\`e contatto in quanto si trova uno spazio intersinaptico. La cellula target pu\`o essere nervosa, muscolare o una ghiandola. Il rilascio di neurotrasmettitori si 
lega a recettori su ghiandole per rilasciare sostanze come ormoni. La segnalazione endocrina \`e legata al rilascio di ormoni che vengono rilasciati nell'apparato circolatorio e 
raggionugono la cellula target che pu\`o trovarsi a notevole distanza. La comunicaizone endocrina \`e molto specifica: bastano piccole quantit\`a di ormoni per avere una risposta dal
recettore con alta affinit\`a nella cellula recettrice. La comunicaizone sinaptico il rilascio del neurotrasmettitore \`e molto elevato e le cellule presentano delle invaginazioni in 
quanto la quantit\`a di segnale \`e molto elevato in modo da avere pi\`u ricettori per esso. L'altra differenza sta nella velocit\`a: la segnalazione sinaptica \`e molto rapida: il 
rilascio del neurotrasmettitore in stretta prossimit\`a con il bersaglio determina una risposta molto rapida, mentre quella endocrina \`e molto lenta in quanto la cellula che risponde
d\`a origine a una serie di eventi che richiede un certo tempo. Lo svantaggio della comunicazione sinaptica nella giunzione neuromuscolare: al muscolo dopo un certo periodo che va 
incontro a stimolazione va incontro ad affaticamento come avviene ai neuroni il rilascio massivo porta ad un affaticamento delle cellule coinvolte nella comunicazione, cosa che non
avviene nella modalit\`a endocrina. La distanza nelle sinapsi \`e molto piccola (quasi a contatto), quando la cellula riceve il segnale pu\`o dare origine a una serie di attivazioni a
cascata che possono comportare diverse cose: ci possono essere delle risposte rapide da secondi a minuti che comportano un'alterazione funzionale di alcune proteine della cellula che
riceve o altre risposte pi\`u lente, minuti o ore meccanismi legati al processo di trascrizione, sintesi di mRNA, traduzione di proteine e che possono alterare il citoplasma che si
traduce in una risposta cellulare al messaggio. Le modifiche veloci che agiscono la funzione sono modifiche post traduzionali delle proteine come fosforilaizone o defosforilazione che
altera la funzione delle proteine. Lente vuol dire che si trova una trascrizione, traduzione e modifiche post traduzionali. Un altro modo in cui le cellule comunicano \`e quella delle
gap junctions, in cui le cellule sono in contatto quasi citoplasmatico tra di loro, giunzoni tra cellula e cellula in cui mettono in comunicazione il citoplasma delle due cellule. Non
\`e una condivisione di tutti i componenti in quanto le gap junction hanno un certo diametro ma passano piccoli ioni, piccole molecole e grosse proteine. Le gap junctions sono 
costituite da un'esamero di connessine che formano una struttura funzionale detta connessone ce ne sono diversi tipi: omotipici o eterotipici se sono formati dallo stesso tipo o da tipi
diversi di connessina. Questo tipo di giunzione si trova nel muscolo cardiaco in quanto le fibre muscolari devono contrarsi contemporaneamente e le informaizoni delle gap junctions 
possono passare molto velocemente. Il diametro delle gap junctions \`e regolata dal pH e dallo ione calcio. Si devono regolare queste connessioni per salvare la cellula dall'apoptosi 
dell'altra non propagando la causa. I meccanismi sono due: abbassamento del pH che avviene durante l'apoptosi e l'altro \`e l'aumento dello ione calcio sempre legato a fenomeni di morte
cellulare. Molte sequenze IRES endogene vengono attivate in condizioni di stress cellulare. Il costrutto \`e caratterizzato da un promomotore che trascrive un mRNA con la seqeunza
di Kozak, viene inserito il gene che si vuole esprimere, IRES e poi il marcatore dopo IRES, si pu\`o far esprimere ad un unico mRNA due geni. Nel plasmide si pu\`o trovare un origine
di replicazione che gli permette di andare incontro a un processo di replicazione in modo da averne una grande quantit\`a e un sito di poliadenilazione per la modifica post 
trascrizionale importante nel contesto di esporto e stabilit\`a. Un IRES si tratta di una sequenza di centinaia di nucleotidi. La trasduzione del segnale pu\`o essere legata a cinque
modalit\`a di segnalazione in tutti i casi visti si trova il ruolo importante del recettore ceh riceve l'informazione che d\`a origine a una serie di cascate che coinvolgno una serie di
fattori, porteine che vanno ad attivare proteine e fattori che portano alla produzione di messaggeri secondarii come calcio cAMP e cGMP che amplificnao il segnale e vanno ad attivare
altre molecole. In molti casi possono dare origine a una risposta rapida nel momento in cui si altera l'attivit\`a di proteine esistenti oppure pu\`o dare origine a una risposta lenta
quando coinvolge la trascrizione e traduzione di RNA specifici. La risposta cellulare pu\`o essere diverse a seconda del tipo di messaggio: quando viene a contatto con molecole che 
inducono la divisione cellulare la cellula entra in mitosi e va ad attivare tutti i cammini intracellulari come la trascrizione di geni che portano la cellula a dividersi. In presenza
di sostanze chemotossiche una cellula pu\`o allontanarsi dalla zona in cui sono presenti. In questo caso la cascata di segnalazione va a convergere sugli elementi del citoscheletro 
importanti per la motilit\`a. La risposta a messaggi pu\`o essere diversa. UN tipo di trasmissione avvieneg razie alla presenza delle gap junctions. La trasmissione di tipo sinaptico:
La trasmissione di un segnale sull'utilizzo della sinaptica si ha una cellula nervosa che forma la fibra pre sinaptica che invia il segnale contenuto all'interno di vescicole che 
contengono un neurotrasmettitore di diversi tipi. All'interno di queste vescicle si trova il neurotrasmettitore e si fondono alla terminazione sinaptica e la fusione mediata da 
calcio e clatrine con la terminazone sinaptica determina il rilascio del neurotrasmettitore nello spazio intersinaptico, tra la cellula pre sinaptica e quella che riceve il messaggio 
postsinaptica su cui sono presenti recettori specifici. IL neurotrasmettitore si lega a recettori specifici che portano alla risposta della cellula postsinaptica. L'arrivo di un 
potenziale d'azione. il meccansimo con cui la cellula nervosa trasmette il segnale \`e una variazione del potenziale di membrana e si tratta di un segnale elettrico che una volta che
giunge alla terminazione sinaptica determina un entrata dello ione calcio che fa fondere le vescicole con la membrana che fa fuoriuscire il neurotrasmettitore che si lega con recettori
e il legame determina l'apertura dei canali e l'ingresso di ioni e portano a una modifica del potenziale di membrana. Il potenziale d'azione sta alla base del funzionamento delle 
cellule nervose, dove non funziona bene pu\`o nascere alla sclerosi multipla, risposta immunitaira alle cellule che avvolgono gli assoni e una pertatno sbagliata diffusione del potenziale
di membrana. Il potenziale d'azione \`e amplificato da canali dio ioni sodio che si aprono in risposta a depolarizzazione di membrana aumentandone l'effetto e poco dopo si richiudono 
e refrattari all'apertura ripristinando il potenziale di membrana. Le cellule nervose trasducono il segnale da natura elettrica con potenziale d'azione che avviene sempre con la massima
intensit\`a e l'informazione che viene portata \`e legata alla frequenza: pi\`u eccitata maggiore la frequenza della propagazione del segnale e maggior rilascio del neurotrasmettitore, 
maggior eccitazione per la cellula post sinaptica. I neurotrasmettitori rilasciati sono diversi come glutammato, il maggior rilasciato \`e l'acetilcolina: viene prodotta a partire dalla
colina (vitamina J) e costituisce insieme all'inositolo della lecitina agente emulsionante che mantiene grassi in soluzione nel sangue e nei fluidi organici. La lecitina prende il nome
dal tuorlo duovo e componenete importante delle membrane plasmatiche. La colina \`e un componente importante introdotto con la dieta e utilizzata per la formazione dell'acetulcolina che
entra all'interno della cellula grazie ad un cotrasporot che sfrutta un gradiende di natura elettrochimica dello ione sodio in maggiore concentrazione all'esterno e sfruttandolo si 
possono imporarte altre moleocle come la colina che viene trasformata in acetilcolina grazie all'acetilcolin sintetasi, una volta che viene prodotta viene inserita in vescicole che poi
si fondono nle momneto in cui il potenziale d'azione apre i canali dello ione calcio, le vescicole si fondono con la membrana e l'acetilcolina viene rilasciata e va a legarsi con il suo
recettore nella membrana post sinaptica, l'acetilcolina nell'ambiene intrasinaptico deve essere poi rimossae questa viene rimossa grazie alla presenza dell'enzima acetilcolinesterasi che 
scinda l'acetilcolina in colina e acetato. La colina pu\`o essere reimportata e il ciclo pu\`o reiniziare. \`E uno dei neurotrasmettitori pu\`u diffusi responsabile dellac ontrazione dei
muscoli e del rilascio di ormoni da parte di ghiandole innervate. La miastenia gravis si tratta di una risposta autoimmune nei confronti del recettore per l'acetilcolina i sintomi \`e
una debolezza muscolare che comporta una difficolt\`a nei movimenti e peggiorano con il tempo. Uno dei metodi per lenire il decorso della malattia \`e di inibire l'enzima che 
degrada l'acetilcolina nello spazio intersinaptico aumentando il segnale e si rende pi\`u efficiente la contrazione muscolare. I recettori per l'acetilcolina sono di due tipi e 
costituiscono i recettori muscarinici e nicotinici, il nome deriva dal fatto che hanno un'affinit\`a i secondi molto elevata per la nicotina che si lega a questa famiglia di recettori e 
i muscarinici affinit\`a per la muscarina, alcaloide presente in funghi velenosi, oltrea affinit\`a i muscarinici sono recettori con 7 domini transmembrana e regolano messaggeri secondari
di tipo metabotropico, recettori che si legano all'acetilcolina e trasducono un segnale all'interno della cellula che comporta l'attivazione di molecole che producono messaggeri 
secondari come cAMP e IP3 (inositolo tre fosfato) i recettori nicotinici sono recettori canali, una volta che l'acetilcolina si lega al recettore questi si aprono e fanno entrare ioni
come lo ione sodio, due tipi di recettori hannoa ffiinit\`a diversa per molecole e trasducono il segnale in maniera diversa. I recettori muscarinici trasducono il segnale attivando altri
fattori. Nel mmuscarinico l'acetilcolina si lega il recettore cambia conformazione e a ad attivare una serie di proteine all'interno della cellula, le porteine G che vanno poi ad 
aumentare i lielli di messaggeri secondari che possono interagire con altri canali o attivare altre vie di segnalazione itnracellulari. L'acetilcolina \`e uno dei neurotrasmettitori
pi\`u diffusi ed ha attivit\`a diverse in base al recettore e al tipo cellulare. Nel muscolo scheletrico  si lega a un recettore nicotinico e determina una contrazione. Nel caso del
muscolo cardiaco il legame con recettore muscarinico determina un rilassamento muscolare. A livello della ghiandola salivare induce la secrezione e la fusione di vescicole contenenti 
enzimi con un recettore muscarinico. Un altra molecola utilizzata come messaggio \`e l'ossido di azoto. L'ossido di azoto \`e interessante come messaggio a livello dei capillari, 
delimitato da cellule endoteliali che poggiano su una lamina basale ecircodnati da cellule innervata della muscolatura lisca che determina la vasodilatazione e la vasocostrizione. La
terminazione \`e importante per la regolazione del diametro e l'apporto sanguigno. L'ossido di azoto in questo viene prodotto dopo il rilascio dell'acetilcolina su recettori presenti su
cellule endoteliale, NO \`e un gas e passa la membrana e interagisce con le cellule muscolari e al loro interno l'ossido di azoto determina l'aumento di livelli di cAMP, messaggero 
secondario che determina un rilassamento muscolare e una vasodilatazione. Questo meccanismo \`e stato studiato e importante in quanto ha comportato lo sviluppo di farmaci vasodilatatori
che causano un aumento di produzione di cAMP all'interno di cellule muscolari come il viagra che agisce in questo cammino e ha un rilassamento delle fibre muscolare e un aumento 
dell'afflusso sangugno all'intrno dei corpi cavernosi. Un altra sostanza importante sono gli ormoni quelli steroidei la sintesi parte dal colesterolo che oltre a essere un problema se 
la quantit\`a non viene regolata a livello circolatorio, \`e una componente importante delle membrane plasamtiche che \`e una molecolca compatta con grande porzine idrofobica e una 
porzione OH \`e il luogo di partenza per la sintesi di molti ormoni come gli ormoni tiroidei, il testosterone. Il colesterolo conferisce rigidit\`a alle membrane, limita e riduce la 
fluidit\`a delle membrana importante per l'interazione per conferire propriet\`a biofisiche alla membrana. Il recettore per gli ormoni sono estremamente specifici e bastano piccoli 
quantit\`a di ormoni per interagire e la risposta porta generalmente una trascrizione di geni specifici per quel particolare tipo di ormoni e la sintesi di fattori specifici, si tratta
di una risposta lenta. Le cellule possono regolare la sensibilit\`a ad un sengale e regolare la capacit\`a di un recettore di legarsi e regolare la sua risposta. La cellula pu\`o rendere
meno sensibile un recettore se la segnalazione persiste per evitare un'iperrisposta al messaggio continuo. La sensibilit\`a del recettore pu\`o avvenire in diversi modi: da un lato ci 
pu\`o essere il sequestramento del recettore tramite endocitosi e viene portato da membrana a interno della cellula tramite endocitosi e la cellula non presenta pi\`u il recettore sulla
membrana questo non comporta la degradazione del recettore, si tratta di un'inibinizionte temporanea. L'altro meccanismo, la downlregalazione meidata da lisoosmi in cui il recettore 
internalizzato viene degradato da lisosomi. Un altro meccanismo \`e quello di lasciare il recettore in membrana e inattivarlo agendo a livello post traduzionale su amminoacidi 
critici nella porizone citoplasmatica che porta la presenza del recettore che inattivato \`e incapace di trasdurre il segnale. Un altro meccanismo si ha un attivazione indiretta in cui 
si attiva la molecola che viene attivata dal recettore, il segnale si lega al recettore che va ad attivare una seconda molecola che \`e stata inattivata. L'altro meccanismo consiste di 
attivare una proteine inibitoria che ha un ruolo di feedback negativo che comporta un'inibizione della trasduzione del segnale inibiendo l'interazione e attivaizone di un'effettore da 
parte del recettore. Questi meccanismi consentono alla cellula di modulare il emssaggio alterando la sensibilit\`a o presenza di recettori sulla propria membrana. Trasduzioni molto
rapide alla base degli organi di senso avvengono grazie alla produzione rapida di messaggeri secondari e stanno alla base di fenomeni rapide e vengono prodotti e rimossi rapidamente, 
questi sono calcio, cAMP, cGMP, inositolo 3 fosfato e altri. La trasduzione del segnale in molti casi va a convergere su proteine con un ruolo nel nucleo e va a attivare molecole nel
nucleo che possono trascrivere per geni specifici o va a modificare proteine citoplasmatici per una risposta pi\`u rapida. I recettori di membrana sono essere distinti in tre categorie:
recettori di superficie legati a canali ionici come recettori nicotinici per l'acetilcolina, recettori di superficie legati a proteine G, normalmente sono recettori di tipo muscarinico 
con diversi domini che attraversano le membrana e vanno ad attivare delle proteine G e i recettori di superficie collegati ad enzimi. La trasduzione del segnael che parte dell'interazione
tra recettore e molecola sengale porta all'attivazione da molecole che fungono da trasduttori intermedi detti interruttori molecolari e possono essere molecole attivate a seguito di 
fosforilazione dopo l'arrivo del segnale che va ad attivare altri fattori, si tratta di una modifica post traduzionale e la molecola viene attivata gruaze alla fosfatasi che rimuove il 
gruppo fosfato e inattiva la moleocla, il altri casi il segnale viene trassdotto da molecole che legano GTP, il segnale arriva, le proteone legano GTP e vengono attivate attivando 
molecole a valle e vengono disattivate quando si ha l'idrolisi in GDP e le molecole vengono disattivate, si ha molecola, recetttore, interruttori molecolari e messaggeri secondari. Vi
sono afttori che facilitano attivazione o disattivaizone di molecole che elgano GTP rendendo il segnale pi\`u rapido o mantenendolo pi\`u a lungo, le GEF (GDP exchage factor), fattori 
che facilitano lo scambio rapido da GDP a GTP attivando le molecole che legano GTP in maniera pi\`u rapida e i GAP che facilitano l'idrolisi e aumentano la velocit\`a di inattivazione 
permettendo l'idorlisi del GTP in GDP e questi fattori interagiscono con gli interruttori molecolari favorendo l'attivazione o disattivazione rendendo pi\`u rapida la trasduzione del 
segnale o viceversa. UNa proteine che lega il GTP sono Rho Rac, Cdc42 che si trovano nello stato attivo quando legano GTP e tornano nello stato inattivo quando viene idrolizzato in GDP
e possono essere attivate o inattivate graze a GAP e GEF, sono fattori numerosi. La funzione di queste proteine che legano il GTP come le Rho si tratta di proteine associate con il 
citoscheletro e regolano lungezza, organizzazione e comportamento dinamico degli elementi del citoscheletro, modifica di migrazione della cellula, risposta durante la mitosi, il 
citoscheletro importante durante la fase di divisione cellulare, coinvolti nella separazione di cromosomi, attivit\`a importante per la funzionalit\`a e risposta a emessaggi che hanno
iengali del citoscheletro, segnali cellulari possono attivare recettori di superficie che convergono sul gruppo di proteine. Diversi recettori possono essere attivati da diversi tipi id
segnali che attivano castate di segnali che agiscono su diversi gruppi di proteine. I canali di tipo metabotropico sono i canali che comprendono recettori di superfici legati a 
proteine G. Questi recettori colelgati a proteine G sono recettori transmebmbrana con domini 7 domini di membrana e poriozni idorfobiche del recettore che attraversano la membrana, 
porzione N terminale rivolta all'esterno nello spazio extracellulare dove avviene l'interazione con la molecola messaggio e una poriozne citoplasmatica con residui che vanno incontro 
a modifiche post trascrizionali come fosforilazione e sumoilazione e importante in quanto \`e quella che interagisce con le porteine G trimerica, una proteina G trimerica il recettore una
volta che riceve il segnaale attiva la porteinea G trimerica di tre subunit\`a $\alpha$ che lega il GTP o GDP e legandosi a GDP \`e inattiva metnre viene attivata quando si ha lo scambio 
da GDP a GDP, la molecola segnale attiva interagisce con il recettore che cambia conformazione nella porzione citoplasmatica che facilita il reclutamento con la proteine G trimerica
inattiva che interagisce con la membrana graizie a un dominio in $\alpha$ e $\gamma$, metrne la $\beta$ non interagisce direttametne, l'interazoine con il recettore favorisce lo scambio
da GDP a GTP che attiva la subunit\`a della proetinea G trimerica che si dissocia in subunit\`a $\alpha$ e complesso attivato $\beta\gamma$. La subunit\`a $\alpha$ pu\`o a sua volta
interagire e attivare una proteina target successiva, una volta che \`e stata attivata da parte di $\alpha$ questa viene attivata attraverso l'idorlidi del GTP e comporta la sua 
disattivazione e il riassemblamento della proteina G trimerica inattiva dal momemtno che lega il GDP. Se lo stimolo persiste il recettore viene inattivato grazie alla presenza della 
chinasi GRK che fosforila la porzione citoplasmatica del recettore che deermina il reclutamento dell'arrestina che blocca l'attivit\`a del recettore pur in presenza del ligando e 
questo avviene quando si ha l'attivazione continua e persistente. L'arrestina pu\`o anche indurre l'endocitosi del recettore dove pu\`o andare incontro a degradazione, alcune tossine
batteriche impedisce la disattivazione delle proteine G che rimangono attivate in maniera costitutiva come quella del colera che lega la subinit\`a $\alpha$ che comporta un aumento 
dei livelli di cAMP che provoca il rialscio di ioni sodio e acqua nell'intestino con conseguente diarrea e squilibrio degli elettroliti, queste tossine bloccano in uno stato 
costitutamente attivo la subunit\`a $\alpha$ la cui attivazione determina degli aumenti di cAMP che rimangono sempre alti, i messaggeri secondari devono ritornare a concentrazioni 
basse. L'adenilato ciclasi vieen attivata dalle proteine G, una proteina di membrana che produce un messaggero secondario il cAMP che viene prodotto partendo da ATP tramite il rilascio
di pirofosfato si crea il messaggero secondario che viene modificato e la sua attivit\`a cessa quando interviene la cAMP fosfodiesterasi che rompe l'anello formando il $5'$ AMP che non
\`e un messsaggero secondario. L'adenilato viene attivata dalla subunit\`a alpha della proteina Gs stimolatoria una volta che sono aumetnati il cAMP va ad attivare la chinasi PKA 
(proteina chinasi A) tra le tante e la PKA \`e un enzima di quattro subunit\`a, due regolatorie e due catalite, in assenza di cAMP le due subinit\`a catalitiche sono legate a quelle
subunit\`a regolatorie, quando cAMP aumenta quattro molecole di cAMP si legano le subunit\`a regoaltorei liberando e attivando le subunit\`a catalitiche che entrano nel nucleo dove vanno
ad attivare la protiena CREB (cAMP responsive element binding protein) che si lega ad una sequenza specifica nel DNA e la cui attivit\`a dipende dai livelli di cAMP. L'attivazione di
CREB si lega insieme ad altri fattori su sequenze specifiche in zone promotrici e legandosi va ad attvare la trascrizione di geni. I livelli di cAMP in diversi tessuti presenta diverse
risposte: nel muscolo l'adrenalina determina la produzuone di glicogeno che rilascia il glucosio del sangue aumentando la capacit\`a muscolare, il glucagone nel fegato comprta la rottura
del glicogeno e il rilascio nel sangue di zucchero. 


Ci sono tre famiglie di recettori ionotropici come il recettore per la nicotina che legandosi ad un ligando come l'acetilcolina cambia conformazione, si apre e determina un flisso di 
uioni che porta delle modifiche fisiche all'interno della membrana e una risposta da parte della cellula. Un recettore \`e pertanto canale. La seconda sono i recettori elgati a proteine
G trimeriche come il recettore muscarinico per l'acetilcolina con 7 domini transmembrana, una extracellulare che interagisce con la molecola e una citoplasmatica che va incontro a 
cambi conformazionali e amminoacidi che vanno incontro a modifiche post traduzionail  inibendolo e la parte citoplasmatica importante perch\`e attivava una proteina G trimerica fomrata
da tre subunit\`a e al legame del ligando che determina il reclutamento e attivazione della protein a G trimerica, $\alpha$ che lega GTP e una $\beta$ e una $\gamma$ l'attvazione della
proteina porta all'attivazione del complesso adenilato ciclasi che \`e un'enzima che catalizza la formazione del cAMP partendo dall'ATP, un messaggero secondario che aumenta notevolmente
quando l'adenilato ciclasi viene attivata dalla proteina G trimerica. Ci sono altri tipi di messaggeri secondari, come vengono prodotti e come la trasduzione del segnale viene amplificata
e da origine a una risposta come nei fotorecettori e quelli dell'olfatto. I messaggeri secondari si trovano a bassa concentrazione in condizioni normali, aumentano rapidamente con il 
segnale e rimossi rapidamente quando smette. Messaggi portati da diversi ormoni agendo su difersi tessuti possono originare aumenti di cAMP che si traducevano in una risposta specifica. 
Un componente importante \`e la proteina chinasi A costituita da due subunit\`a regolatorie che legano cAMP e due catalitiche che in assenza di cAMP rimangono inattive. Sono necessarie 
4 molecole per determinare la scissione delle subunit\`a catalitiche che vanno a fosforilare diversi effettori come il CREB, un fattore trascrizionale attivato quanto PKA viene attivata 
e si lega in regioni specifiche a monti di geni particolari detti sequene CRE (ciclic AMP response element), riconosciute quando CREB vien e attivata dall'aumento di cAMP. La serie di 
cascate porta a CREB. Un altro esempio di attivazione della PKA la cui funzione \`e diversa la si trova nella via attivata dal glucagone, un ruolo opposto dal ruolo dell'insulina (che
\`e un ormone deputato alla regolazione dei livelli di glucosio, rimuovendolo dal sangue favorendo la formazione di glicogeno che rappresenta la riserva di zuccheri come polimero del 
glucosio, quando l'organismo necessita il glucosio viene scisso) Il glucagone fa l'opposto e determina un rilascio di glucosio nel sangue: il glucagone si lega a un recettore che attiva
una proteina G che aumenta livelli di cAMP dopo aver attivato l'adenilato ciclasi e la PKA fosforila che fosforila la glicogeno sintetasi e la fosforilasi chinasi. A seguito della loro 
fosforilazione, della glicogeno sintetasi determina il blocco della sintesi del glicogeno e si fosforila l'enzima che causa la degradaizone del glicogeno formando glucosio. Lo stesso
segnale che comporta l'attivazione di un effettore pu\`o avere conseguenze differenti in base al tipo cellulare e alle molecole che vengono attivate, diverso per CREB e metabolismo del
glucosio, l'aumento dei livelli di cAMP e in tessuti diversi pu\`o portare all'attivazione di elementi che hanno un ruolo diverso. Oltre al cAMP vi sono altri messaggeri secondari e 
questi vengono prodotti a seguito dell'attivazione da parte di altri molecole segnale in diversi tessuti a seguito di una serie di passaggi enzimatici come l'inositolo tre fosfato e il 
diacilglicerolo prodotti dall'attivazione della fosfolipasi C e producono aumenti di inositolo tre fosfato e di diacilglicerolo che derivano da un fosfolipide di membrana 
fosfatidilinositoolo. Un lipide importante \`e il fosfatidilinositolo con un gruppo polare (inositolo) e due catene di acidi grassi: una satura e una insatura. Il fosfatidilinositolo \`e
una molecola anfipatica con una porzione polare e una apolare. Da questo precursore a seguito della fosfatidilinositolo chinasi si forma un fosfatidili inositolo 4 fosfato, la chinasi 
aggiunge un gruppo fosfato in posizione 4 crendo il fosfatidilinositolo 4 fosfato e interviene la fosfatidilinositolo fosfato chinasi che usanso ATP aggiunge un gruppo fosfato in
posizione 5 creando il fosfatidil 4, 5, bifosfato. Questa forma viene tagliata dalla fosfolipasi C attivata a seguito di ormoni che taglia il fosfatitilinositolo 4, 5, bifosfato creando
il diacilglicerolo e l'inositolo 1, 4, 5 trifosfato (IP3), due messaggeri che funzionano come messaggeri secondari. Uno rimane associato alla membrana mentre l'altro viene rilasciato nel
citoplasma. Si forma un messaggero secondario di membrana e uno secondaria. L'inositolo 3 fosfato citoplasmatica diffonde rapidamente e va ad aprire i canali del calcio sulla membrana 
del reticolo endoplasmatico in quanto riserva di calcio presente a bassissime concentrazioni nel citoplasma ma accumulato nell'ER liscio e queste pompe devono fare un lavoro contro il 
gradiente per pompare il calcio dal citoplasma nell'ER, aprendo i canali che lasciano passare il calcio questo tende a uscire secondo il gradiente chimico ed elettrostatico. il IP3 
determina pertanto il calcio che funziona come messaggero secondario. L'inositolo 3 - fosfato viene inattivato mediante fosfatasi e il calcio viene riaccumulato o nell'Er o nei mitocondri
o portato fuori dalla cellula. L'inositolo 3 fosfato e il diacilglicerolo vengono prodotti dalla fosfolipasi C attivata da una proteina G trimerica atrivata dall'interazoine con un 
recettore e una molecola segnale. Il diacilglicerolo rimane in membrana e l'IP3 apre i canali di calcio che esce ed attiva altre proteine come la proteina chinasi C che va a forforilare 
altri fattori e viene attivata grazie al recluamento in membrana del gliacilglicerolo e poi fosforila proteine di membrana o prossime alla membrana. Sono importanti per le chinasi e per
l'attivazione in un punto particolare della chinasi. Si \`e aggiunto ai messaggeri secondari l'inositolo 3 fosfato, il diacilglicerolo e lo ione calcio. L'altro messaggero \`e lo ione
calcio, importante coinvolto nell'attivazoine di molti cammini nella cellula come l'attivazione di proteine che attivano le chinasi e svolge un ruolo importante durante il processo
di fecondazione dell'oocita da parte dello spermatozoo che porta all'ispessimento della membrana impedendo ad altri spermatozoi di fecondarla. Lo ione calcio \`e omportante per la 
segnalazione delle cellule nervose l'attivit\`a dei neuroni pu\`o essere visualizzata grazie alle fluttuazioni delle concentrazioni di ione calciio. Lo ione calcio \`e visibile 
attraverso marcatori, sostanze che legano lo ione calcio ed emettono fluorescenza in base alla sua concentrazione. Le gap junction permettono la creazione di onde di calcio in quanto
questo pu\`o passare attraverso loro. La presenza di trasportatori di ioni calcio nella membrana plasmatica, nell'ER e nei mitocondri facendo lavoro contro gradiente chimico ed elettrico
queste pompe utilizzano ATP importante per fare in modo che questi trasportatori accumulino ione calcio o mitocondri. Questo gradiente che si viene a creare pu\`o essere sfruttato da
altri trasportatori per portare altre molecole contro gradiente (simporti-antiporti), il calcio dopo che aumenta all'interno della cellula attiva la calmodulina in un caso, una proteina
con una forma simile ad un manubrio in grado di legare 4 ioni calcio e a quel punto cambia conformazione e possono interagire con altri effettori attivandoli. Uno degli effettori attivati
\`e la proteina chinasi II la cui attivit\`a dipende dallo ione calcio (CAMIIK) il cui RNA localizza nei dendriti delle cellule nervose e la sua ttivit\`a \`e dipendente dal ccalcio e 
viene attivata grazie all'attivazione della calmodulina. La calmodulina lega domini elicali. La fluorescenza di fret per la quantit\`a di calmodulina legata al calcio. Nel momento in cui 
il calcio si lega le due estremit\`a vanno in contatto tra di loro e si eccitano e si ha il fenomeno della fret. In base alla quantit\`a di fluorescenza si determina il passaggio di 
calcio. 
\section{Effetto dell'acetilcolina sul cuore}
L'acetilcolina ha un effetto diverso a seconda del tipo di recettore con cui interagisce. Pu\`o interagire con un recettore muscarinico con una proteina Gi inibitoria che inibisce 
l'adenilato ciclasi tramite la subunut\`a $\alpha$ diminuiendo i livelli di cAMP e si riducono se viene attivata la fosfodiesterasi che inibisce ulteriormente i livelli di cAMP, il 
complesso $\beta\gamma$ apre i canali del postassio di membrana del muscolo cardiaco con un'iperpolarizzazione e rende pi\`u difficile la partenza di potenziali d'azione trasducendo 
nell'effetto opposto alla contrazione. L'acetilcolina legandosi ad un recettore canale nel muscolo scheletrico invece attiva la contrazione.
\section{Recettori del sistema olfattivo}
Nel sistema olfattivo importante per la sopravvivenza della maggior parte delle specie animali in particolare la memoria \`e connessa con questo tipo di senso. Nei mammiferi il DNA 
contiene circa 1000 geni per diversi recettori per gli odori e nell'uomo sono attivi il quaranta percento in quanto l'olfatto \`e sceso in secondo piano rispetto al sistema visivo. 
La capacit\`a di sentire gli odori \`e presente fin dal momento in cui si nasce, un senso sviluppato durante la nascita. Il sistema olfattivo \`e caratterizzato dalla presenza di 
neuroni olfattori specifici per determinate molecole che contengono delle ciglia che sono immerse in un muco grazie al quale gli odori si dissolvono e possono interagire con i recettori
presenti sulle ciglia. Questi neuroni olfattori mandano le informazioni all'interno del sistema nervoso nei glormeruli olfattori la caratteristica che neuroni olfattori che esprimono 
lo stesso tipo di recettori convergono sullo stesso glomerulo olfattorio. L'epitelio olfattorio esprimono diversi recettori e quelle che esprimono lo stesso recettore convergono sullo
stesso glomerulo. L'informazione viene mandata nel sistema limbico e sulla corteccia celebrale. In molti organismi il sistema olfattorieo \`e estremamente sviluppato. Axel e Buck premio
nobel in quanto furono i primi a clonare i recettori olfattori. A livello di microscopia ottica ed elettronica la morfologia dei neuroni si trova il corpo cellulare un unico dendrita la
cui sommit\`a si trovano le ciglia. Nell'epitelio olfattorio neuroni che esprimono un unico tipo di recettore e tutti quelli che esprimono lo stesso tipo di recettore convergono nello
stesso glomerulo olfattorio. La presenza di una determinata molecola si trasduce i neuroni devono trasmettere l'inofrmazione da molecola a potenziale d'azione. La molecola si lega a 
un recettore muscarinico con diversi domini di membrana legata a una proteina Golf trimerica che attiva attva l'adenilato ciclasi che aumenta i livelli di cAMP che quando aumenta vanno
a cambiare la permeabilit\`a della membrana grazie all'apertura dei canali del sodio la cui apertutra causa l'entrata del sodio e della modifica delle propriet\`a bioelettriche della
membrana e la creazione di un potenziale elettrico portata dal sistema nervoso. Il meccanismo passa dall'attivaizone della proteina G che attiva l'adenilato ciclasi e dei livelli di cAMP
\section{Trasduzione visiva}
Non si ha la produzione di cAMP ma il messaggero secondario \`e il cGMP prodotta da una guanilato ciclasi che ciclizza il GTP il trasduzione \`e la ropsosta mediata da proteine G pi\`u 
veloci. La luce provoca una caduta dei livelli di cGMP. L'arrivo di un fotone causa una riduzione dei livelli di cGMP. I fotorecettori sono presenti nella porzione della retina e sno 
all'inizio di una lunga catena di iversi tipi cellulari la cui segnalazione converge sulle cellule gangliali che mandano l'informazione nel sistema nervoso centrale tramite il nervo
ottivo. La trasduzione del segnale viviso avviene tra coni e bastoncelli. Tra i modelli usati per lo studio del sistema visivo si trova il pollo con occhi molto grandi e epresentano un
sistema nervoso molto ben eveidente e facilmente manipolabile del tetto ottico che corrisponde alla corteccia visiva dove convergono e vengono elaborate le informaizoni date dal nervo 
ottico \`e facilmente manipolabile in quanto si prende l'uovo fecondato lo si lascia qualche ora e l'embrione ti trova nella parte alta. \`E possibile evidenziarlo con dell'inchiostro. 
ed \`e facilmente manipolabile. \`E possibile inserire, overesprimere regolare geni iniettandoli direttamente. Un altro organismo utilizzato per lo studio \`e lo zebrafish un modello 
pi\`u facile da manipolare e meno costoso rispetto al pollo e si ha che uno zebrafish da orogine a un elevato numero di embrioni. Anche lo zebrafish ha occhi molto grandi e facilmente
accessibili e si ha la presenza della struttura detta tetto ottico in cui convergono le infroamzioni dalla retina. Lo zebrafish \`e un buon modello per le problematiche connesse al 
sistema visivo e si sa se ci sono problemi nella trasduzione del sistema visivo nello zebrafish ci sono due procedure: quello pi\`u evidente \`e il fenotipo in quanto la pigmentazione 
alterata ha dei difetti della trasduzuone del segnale in quanto la pigmentaione riflette ci\`o che l'animale vede igmanetazione legata ai melanofori. Nel secondo screening si vede a 
livello ultrastrutturale se ci sono problemi nella struttura della retina che \`e costituita da una serie di strati. Vi sono mutanti in cui la struttura della retina \`e alterata o in 
cui pu\`o essere pi\`u o meno noermale macon un cristallino alterato o una retina alterata. I fotorecettori sono i coni e i bastoncelli con struttura peculiare ocn un nucleo e corpo 
cellulare, un segmento interno citoplasmatico caratterizzato da molti mitocondri e un sistema di membrane detti dischi con il pigmento e i recettori che raccolgono i fotoni. Infine si 
trova la terminazione sinaptica dove la presenza di un neurotrasmettitore che quando rilasciato eccita o inibisce i neuroni a valle. Il recettore \`e simile al recettore con domini 
transmembrana, una porzione citoplasmatica e una interna e trasduce il segnale in maniera simila ma viene attivato dalla lice. La rodopsina contiene un pigmento cromoforo detto 11-cis
retinale e questo che raccogliendo il fotone cambia conformazione e si ha il passaggio da una conformazione cis a una trans pi\`u lineare. La modifica retinaria a seguito da un fotone
mofifica la rodopsina che attiva la proteina Gt transducina, la subunit\`a $\alpha$ attiva la fosfodiesterasi che degrada il cGMP, si chiudono i canali del sodio che essendo permeabili
al calcio provoca una riduzione del rilascio del neurotrasmettitore e l'ttivazione della guanilato ciclasi che fa risalire i livelli di cGMP. La riduzione di calcio riduce il rilascio
del neurotrasmettitore a livello sinaptico e attiva la guanilato ciclasi che fa risalire i livelli di cGMP in quanto si dovr\`a poter trasdurre un nuovo segnale. In presenza di luce non
si trova il rilascio di neurotrasmettitore. In questo caso il fotorecettore reagisce bloccando il rilascio del neurotrasmettitore. Alla fine nel momento in cui arriva la luce si trova
una stimolazione il blocco \`e legato al fatto che il neurotrasmettitore rilasciato dalla cellula \`e un inibitore e la fine dell'inibizione produce un'eccitazione. 

\section{Lezione 16}
NOn ha registrato. IL tutto avviene a livello della giunzione neuromuscolare, ovvero il punto in cui la terminazione del motoneurone \`e adiacente e in prossimit\`a della cellula 
muscolare. La giunzione a livello post sinaptico \`e caratterizzata dalla presenza di diverse invaginazioni che sono importanti per aumentare la superficie per mantenere il maggior numero
possibile di recettori per l'acetilcolina, il pi\`u importante neurotrasmettitore in quanto permette la contrazione del muscolo. Viene rilasciata nello spazio intersinaptico e va a 
legarsi in recettori presenti a livello della membrana post sinaptica del mucsolo. Queste invaginazioni permettono un nunmero elevato di recettori ionotropici per l'accetilcolina il 
cui legame causa l'apertutra e l'entrata di ione sodio che determina una modifica delle propriet\`a biofisiche causando una depolarizzaiozne che si ripercuote all'interno della 
fibra dando origine alla contrazione. La giunzione muscolare \`e caratterizzata dalla terminaizone nervosa,da una membrana post sinaptica invaginata e delle cellule che circondano la
giuznioen che fanno parte a livello del sistema nervoso periferico di Schwann che proteggono e aiutano il funzionamento e l'attivit\`a della giunzione neuromuscolare o placca 
neuromuscolare la terminaizone nervosa \`e ricca in mitocondri e in bescicole che vontengono acetilcolina che si fondono con la membrana pre sinaptica rilasciandola nello spazio 
intersinaptico a livello post sinaptico \`e presente zone dense agli eletttroni con canali e proteine legate al di sotto della membrana con ruolo imporgtante nle mantenere i recettori in
loco e con ruolo strutturale importante. I puntini nel citoplasma del muscolo sono fibre di actina e miosina organizzate in modod regolare e precisa, l'organizzazione \`e importante 
per il funzionaamnto del muscolo per permettere la contrazione efficace. Il muscolo non \`e una cellula singola ma un sincizio, durante lo sviluppo le cellule mioblasti differenziano, si 
fondono tra di loro e formano una struttura polinucleata, cellule unite e formano una struttura polinucleata pi\`u elecata di organizzazione. La formazione e il differenziamento miscolare
e la formaizone della sinapsi si formano in contemporanea e richiedono diverse settimane. Quando si forma ilsincizio sono presenti recettori per l'acetilcolina che prima della 
formazine della giunzione si trvano sparsi. Quando si ha il contatto tra il motoneurone e la fibra muscolare i recettori che prima erano sparsi clusterizzano in prosssimit\`a al di sotto
dei punti di contatto che il motoneurone ha sviluppato formando le placche muscolari. Ogni motoneurone lega una singola fibra. Il modello \`e molto studiato e semplice in quanto anumali
con grande quantit\`a di muscoli e grande quantit\`a di recettori e la giunzione neuromuscoalre \`e un grado di rigenerarsi e possibile da ricreare in vitro. esperimenti che uanno 
portato a capire come avviene la contrazione. Il muscolo \`e caratterizzato dalla presenza di tante fibre muscolari, i nuclei si trvano alla periferia della fibra in quanto altrimenti 
comprometterebbero l'irganizzazione dei filamenti di actina e miosina impedento un'efffciente contrazione i nuclei si trovano alla periferia. Si dice stritata in quanto prendendo 
a sezione longitudinale con staining si nota che oltre ai nuclei compaiono striature trasversali che da nome al muscolo scheletrico striato tipico del muscolo cardiaco e quelli volontari.
La fibra muscolare \`e ricoperta dalla membrana endomisio, pi\`u fibre formano un fascio contenuto dal perimisio e l'epimisio circonda pi\`u fasci che costituiscono il muscolo. Qeuste
striature che caratterizzano la muscolatura sono delle striature legate alla presenza di due proteine: actina e miosina. Sono organizzate in modo da creare delle bande e linee, si 
identificano le linee Z e al loro interno a met\`a si trova la linea M le due linee Z delimitano l'unit\`a funizonale del muscolo detta sarcomero che in parallelo formano la fibra 
muscolare. Oltre alle linee si identificano le bande I , A e H. Durante la contrazione nel sarcomero va incontro a contrazione, pi\`u sarcomeri si contraggono e il muscolo si accorcia. 
Al sarcomero quando si contrae durante la contraizone le linee Z si avvicinano e la distanza tra le due linee Z diminuisce. La linea A rimane costante la linea I diventa pi\`u piccola
e la linea H sparisce. Quando il muscolo si contrae le bane A si accorciano, la linea M rimane e sparisce la banda H. Le linee Z sono i punti in cui i filamenti di actina si uniscono 
tra di loro nella terminazione, tra i filamenti di actina si trova un'altra proteina filamentosa la miosina. La linea M \`e il punto dove le code dei filamenti di miosina si uniscono 
tra di loro, durante la contraizone i filamenti di actina scivolano all'intero dei filamenti di miosina e speiga perch\`e le linee Z si avvicinano tra di loro e la linea M non cambia in
quanto miosona entrano nei filaemnti di actina. Le due componenti principali del miscolo sono la miosina e l'actina. Sono di fatto delle proteine che formano filamenti, l'
actina costituita da monomeri lineari che formano filamenti (simile al rosario). La miosina \`e caratterizzate da teste globulari. un collo mobile e una dominio retilinio che permett 
l'interazione tra le molecole di miosina. La presenza di filmaenti di actina e miosina che formano i sarcomeri, i nuclei si trovano in periferia e i mitocondri, tanti per produrre ATP. 
L'altra caratteristica del muscolo \`e quella di avere un reticolo sarcoplasmatico esteso che circonda la fibra muscolare che contiene lo ione calcio importante per la contraizone. Sono
presente un sistema a tubuuli T delle invaginaizoni della membrana che circonda la fibra muscolare e sono a stretto contatto con il reticolo sarcoplamsatico, importante per trasformare 
la variazione del potenziale di membrana in un messaggio di natura chimica che comporta il rilascio dello ione calcio e della contrazione muscolare. I tubuli T penetrano dalla membrana 
all'interno. Oltre alle due proteine actina e miosina vi sono altri due complessi: la tropomiosina, una proteina filamentosa che sovrasta i filamenti di actina e il complesso della 
troponina costituito da tre diverse molecole che interagisce da un lato con la tromonina e con i filamenti di actina. La miosina presenta una coda con doppia elica proteica, un collo
mobile e una testa globulare che interagisce con i monomeri di actina. La miosina \`e formata da un dimero con una catena leggera e da una pesante il dimero ha un estremit\`a on due
teste che interagiscono con i monomeri di actina e la coda che unita tra di loro formano la linea M quando si uniscono tra di loro. L'ATP permette il movimento. La troponina ha
tre subunit\`a troponina T, I, C, la subunit\`a T lega la troponina alla tropomiosina, la I lega la tromonina all'actina. Nel momento in cui I si lega all'actina impedisce alla miosina 
di legarsi della miosina con l'attina inibendo il legame tra le teste di miosina con i monomeri di actina. La terza compnente, la subunit\`a C che lega lo ione calcio, di fatto \`e un
sensore per il calcio e il legame porta ad una serie di modifiche conformaizonali della troponina che si muove e rimuove la tropomiosina dall'actina lasciando libera la testa di miosina
di interagire con l'actina. Il motoneurone rilascia acetilcolina che si lega ai recettori presenti nella giunzione neuromuscolare e l'apertura dei canali determina un onda di
depolarizzazione che penetra grazie ai tubuli T, quando arriva apre i canali voltaggio dipendenti e sono presenti sulla membrana del reticolo sarcoplasmatico. L'apertutra dei canali 
causa un riascio dello ione calcio che interagisce con la subunit\`a C della troponina. IL legame causa un cambio conforamzionale, la troponina rimuove la tropiomisionei dall'actina
che permette l'attacco della mioscina che permette la contrazione ad opera della miosina. La contrazione avviene in quanto la miosina ha due teste motrice indipendenti con un nucleo 
catalictico e un braccio di leva la coda la lega al filamento spesso e all'inizio la miosina contiena ADP e fosfato e ha affinit\`a debole per l'actina. Quando una delle teste si lega 
all'actina il fosfato \`e rilasciato. QUando l'inetrazione rimane stabile il fosfato \`e rilasciato e si forma un legame pi\`u stabile tra testa della miosina e actina. E causa un
cambio conformazionale della miosina che causa il suo avanzamento lingo il filaento di actina. ADP si dissocia e si associa ATP causando lo stacco della testa di miosina dall'acinta
e l'ATP \`e idrolizzato causando il ritorno nella confromazione originaria dell'actina. L'actina non ritorna indietro in quanto sono presenti altre miosine attaccate al filamento. Quando 
il livello di calcio diminiuisce le miosine si staccano e l'actina ritorna nella conformazione di risposo della miofibrilla. Il musoclo consuma tanto ATP per svolgere la propria funzione
una delle risorse che il muscolo ha una riserva di ATP prodotta dalla creatina che diventa creatina fosfato che pu\`o cedere un gruppo fosfato all'ADP creando ATP. Si crea una riserva di 
ATP al muscolo. La reazione avviene in assenza di ossigeno ma non produce acido lattico. Quando i livelli si abbassano e il muscolo si affatica si comincia a produrre ATP in condizione
anaerobica con la produzione di acido lattico che abbassa il pH che va a compromettere la contrazione muscolare in quanto il muscolo tende a ridurre l'ampiezza delle contrazione e va
incontro a affaticamento muscolare, l'abbassamento di pH blocca la reaiozne di stacco delle teste di miosina che rimangono legate ai filamenti di actina e si ha la contrattura da
fatica. Dopo di che il consimo di ossigeno rimane elevato per ripristinare i livelli di ATP e CP normali e per demolire l'acido lattico. Anche lo ione magnesio favorisce il distacco
delle teste di miosina e facilita la decontrazione del muscolo. Questo \`e un esempio di trasduzione del segnale con rilascio dello ione calcio. UN altro muscolo scheletrico che 
lavora in maniera analoga \`e quello del cuore. Ci sono i muscoli lisci senza questa contrazione. Sono i muscoli involontari come per esempio quelli della peristassi intestinale che
favorisce lo spostamento del cibo. Sono le stesse responsabili della vasocostrizione e della vasodilatazione. Nono formano sincizi e non hanno l'organizzazione dei filamenti di 
actina e miosina e a velocit\`a di contrazione \`e inferiore rispetto al muscolo scheletrico. 
Legato e connesso dell'apparato scheletrico responsabile dei movimenti volontari anche se alcune contrazioni sono involontarie e permettono ilmantenimento del tono muscolare, contrazioni
che permettono di far fronte alla gravit\`a e mantenere la postura che viene regolato a livello del cervellettoo. In generale il muscolo scheletrico \`e volontario. \`E chiamato 
striato in virt\`u dell'organizzazione precisa dei filamenti di actina e miosina alla base del funzionamento e permette la contrazione. Ci sono altri tipi di muscoli come quello caradiaco
con caratteristiche analoghe. La musculatura lisca presente cellule fusiformi mononucleare prive di striatura trasversale e il controllo della contrazione \`e indipendente dalla 
volont\`a. La caratteristiche principali che le contraddistigue dal muscolo striato. La musculatura liscia si trova nei vasi per vasocrostriione, nella pelle e nelle visceri, per la 
peristalsi intestinaele, onda di contrazione che permette il movimento del cibo nell'apparato gastrointestinale, \`e involontario e l'idrolisi di ATP \`e importante per la contrazione
che \`e pi\`u lenta e prolungata (fase di contrazione pi\`u lunga) e richiede una minor quantit\`a di ATP nel muscolo liscio \`e distinguibile dalla musculatura scheletrica in quanto non
ci sono striature il nucleo presente nel citoplasma e la quantit\`a di mitocondri \`e minore. Inoltre a differenza del muscolo scheletrico le contrazioni della cellula possono essere
parziali i filamenti coinvolti sono l'actina la miosina che non hanno organizzazione precisa. Anche in questo caso la trasduzione del segnale \`e mediata dallo ione calcio il cui rilascio
va ad attivare i meccanismi di contrazione leggermente diversi nel caso della fibra muscolare scheletrica. 
\section{Recettori ad attivit\`a enzimatica intrinseca}
I recettori ad attivit\`a enzimatica intrinseca sono proteine di membrana monopasso che posseggono un dominio catalitico ovvero tirosine chinasi in grado di fosforilare l'amminoacido 
tirosina. Sono loro che a seguito del legame con la molecola esterna si attivano o vanno ad attivare e sono collegati in maniera funzionale con degli enzimi. Questi enzimi sono 
chinasi, proteine che attivate  fosforilano altre molecole. Le caratteristiche di questi recettori sono di membrana monopasso con un solo dominio transmembrana, possono avere un
dominio catalitico e la famiglia pi\`u grandi \`e costituita da tirosine chinasi che fosforilano residui di tirosina. Ci sono diversi recettori come fattori di crescita mitogeni ed 
ormoni come PDGF (platelet derived growth factor per le piastrine), NGF (fattore di crescita per le cellule nervose), EGF (fattore di crescita dell'epidermide). Questi recettori possono
avere essi stessi attivit\`a chinasica o associati a proteine chinasi che fosforilano altre molecole in tirosina, serina o treonina, associati a istidina chinasi (fosforila a livello di 
istidina), tirosine fosfatasi. 
\subsection{recettori associati a tirosina chinasi}
Per antigene, interleuchine, integrine e citochine. In particolare il recettore per le citochine, molecole che vanno a stimolare l'attivit\`a mitotica delle cellule. Questi recettori 
che si legano alle citochine comprendono una famiglia di recettori e in particolare vi sono i recettori associate alle JAK che vanno ad attivare una via di segnalazione Jak-Stat, prendono
il nome da Janus kinase e prende questo nome perch\`e hanno la propriet\`a di autofosforilarsi di cross fosforilazione e di fosforilare altre proteine. Questo tipo di recettore per le 
citochine \`e associato ad una proteina con attivit\`a chinasica. Si ha il recettore un dimero che in assenza della citochina sono separati. La citochina determina la dimerizzazione del
recettore, i due monomeri vengono a contatto e favorisce l'attivazione della JAK che attivata fosforila la JAK preseente sull'altra subunit\`a e quando si sono autoattivate fosforilano 
il recettore. La citochina promuove la dimerizzazione del recettore, le due JAK si attivano a vicenda e il recettore viene fosforilato in modo che tale fosforilazoine C termianle 
determina il reclutamento delle STAT che vengono reclutate al recettore quando \`e stato fosforilato. Sono normamlente citoplasmatiche STAT1, 2 l'itnreazione tramite il dominio SH2 che
permette l'interaizone con il recettore fosforilato e tale presenze presente la loro attivaizone da paret delle proteine JAK. Questa fosforilazone \`e importante in quanto normalmente 
sono monomeri che fosforilati formano dimeri che sno in grado di etrare nel nucleo e di attivare la trascrzione di geni specifici attivati quando il recettore interagisce con le 
citochine. Le JAK sono 4: 1, 2, 3, e tyk 2 e quattro ja coinvolte a diversi livelli che riconoscono specifiche citochine, loro mutazioni compromettono la trasduzione del segnale in 
presenza della molecola. Un altro esempio di segnalazione simile con attivazione di recettore con attivit\`a serina/treonina chinasica sono il transforming growth factor TGF$\beta$ e 
BMP bone morphogenetic proteins il rilascio di questi fattori e il loro legame con un loro recettore attiva una via di segnalazione che coinvolge le SMAD, simile a quella precedente
ma che ha delle differenze: il recettore non \`e costituito da due monomeri ma due subunit\`a diverse: per il TGF$\beta$ si lega al recettore costituito da un recettore di tipo 1 e di 
tipo 2 che ha attivit\`a di serina / treonina chinasi e determina la dimerizzazione del recettore che va ad attivare il sito chinasico del recettore di tipo 2 che fosforila l'altra 
subunit\`a del recettore di tipo 1. La subunit\`a catalitica fosforila al ivello di serina e treonina l'altra subuniit\`a. La fosforilazone recluta SMAD2, 3 fattori citoplasmatici ed
\`e importante in quanto permette la loro fosforilazione che porta alla formazione di dimeri con SMAD4 e la formazione di questi dimeri permette la loro entrata nel nucleo e il dimero di 
SMAD va ad attivare la trascrizione di geni specifici per la segnalazione attivata da TGF$\beta$. Il meccanismo \`e simile ma ha un meccanismo diverso. Questa via di segnalazione viene
bloccata attraverso un meccanismo di feedback: quando il dimero inizia a trascrivere geni specifici si rova un meccanismo di trascrizione che trascrive SMAD6 e SMAD7 con funzione 
inibitoria. UN feedback negativo che permette di bloccare la via di segnalazione. Le SMAD6 e 7 si legano al recettore e competono con le SMAD2 e 3 in modo che la trasmisisone del 
segnale sia bloccata. Si \`e visto che anche in un altro meccanismo di inibizione comporta l'attivazione di ubiquitina smurf: SMAD ubuquitylation regulatory factor una molecola che 
va ad ubiquitinare il recettore e ne promuove la degradazione. Un altro meccanismo prevede il reclutamneto di na fosfatasi che rimuove il gruppo fosfato sul recettore di tipo 1 
disattivandolo. Ci sono tre vie di inibizione attivate dalla via di segnalazione che blocca la trasmissione del segnale. 
Un esempio di trasduzione del segnale che porta alla regolazione dell'attivit\`a del flagello dei batteri che si muovono grazie alla presenza di questa truttura con un dominio iniziale
rigido e una coda flessivile e la rotaizone permette al batteriodi muoversi in una direzione verso una sostanza chemoattrattiva o allontanandosi a una sostanza chemorepellente. QUesto
avviene tramite un recettore per la sostanza, e una via di segnalazione che va a regolare l'attivit\`a e la rotazione del flagello, costiuito da una serie di strutture: un rotore, 
uno stator che ancora il flagello alla parete batterica, un uncino rigido e un filamento flagellare. La via di trasduzione coinvolge almeno due proteine regolatrici CheA e CheY il 
tutto \`e mediato dalla sostanza con il recettore che attiva una proteina di legame tra il recettore e CheA che viene fosforilatta. CheA viene fosforilata e attivata dall'interazione
della sostanza con il recettore mediata da CheW. L'attivazoine di CheA porta all'attivazione grazie a fosofrilazione di CheY che si dissocia dal recettore, diffoonde nel citoplasma e 
si lega al motore flagellare facendolo girare in una determinata direzione che pu\`o avvicinare o allontanare il batterio dalla sostanza. CheY viene inattivata grazie alla presenza di 
CheZ che la defosforila. Struttura del filamento e suo metodo di movimento. CheA istidina chinasi la quale trasmette il segnale che regola il senso di rotazione del flagello. Non \`e 
chiaro il meccanismo con cui il rotore funziona, \`e importante lo ione \ce{H^+}, due ioni idrogeno entrano permettendo un primo scatto di rotazione. Il gradiente \`e importante in 
quanto permette la motilit\`a e la produzione di ATP che avviene sfruttando il gradiente di ioni. ATP sintesi. Il gradiente quando si inverte idrolizza ATP producendo ADP e fosfato il 
gradiente viene mantenuto tale grazie ad una pompa che butta fuori gli ioni, quando smette di funzionare il gradiente si dissipa e i meccanismi di produzioone di ATP si interrompono. 
