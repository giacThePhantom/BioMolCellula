\chapter{Segnalazione cellulare: trasduzione del segnale}

\section{Panoramica}
Le cellule devono essere in grado di comunicare tra di loro in modo da presentare una risposta coordinata ad eventi esterni.
Devono inoltre poter trasportare all'interno questi segnali in modo da produrre la risposta desiderata.
L'insieme di eventi che porta una cellula a generare una risposta in base a un segnale extracellulare viene detto trasduzione del segnale.

	\subsection{Tipologie di risposte}

		\subsubsection{Risposta lenta}
		La risposta lenta avviene quando il segnale provoca sintesi proteica o modifiche di trascrizione.

		\subsubsection{Risposta veloce}
		La risposta \`e veloce quando il segnale provoca modifiche post-traduzionali a proteine gi\`a sintetizzate.

\section{Modelli di comunicazione delle cellule}
Le cellule possono comunicare tra di loro in molti modi.

	\subsection{Comunicazione contatto dipendente}
	La comunicazione contatto dipendente avviene tra due cellule che si trovano in contatto fisico.
	Questo contatto \`e mediato da recettori di membrana con un meccanismo a chiave-serratura.

	\subsection{Comunicazione paracrina}
	La comunicazione paracrina permette la comunicazione tra cellule lontane: una cellula diffonde un segnale all'esterno.
	La molecola segnale rilasciata nell'ambiente viene riconosciuta da cellule con recettori specifici per il segnale.
	La distanza che la molecola pu\`o raggiungere dipende dalla permissivit\`a dell'ambiente, ma rimane minore rispetto alla comunicazione endocrina e maggiore rispetto alla sinaptica.

	\subsection{Comunicazione sinaptica}
	La comunicazione sinaptica \`e specifica per i neuroni.
	Il contatto avviene tra due cellule nervose o tra una cellula nervosa e una ghiandola o un muscolo.
	Il meccanismo \`e veloce in quanto le cellule si trovano vicine.
	Gli assoni permettono alle cellule nervose di entrare in contatto con cellule a grandi distanze.

	\subsection{Comunicazione endocrina}
	La comunicazione endocrina avviene grazie a ormoni immessi nel sistema circolatorio.
	Questi sono rilasciati in piccole quantit\`a in quanto i recettori sono molto sensibili.

		\subsubsection{Confronto con comunicazione sinaptica}
		\begin{table}[H]
			\centering
			\caption{Confronto tra comunicazione sinaptica ed endocrina}
			\begin{tabular}{|c|c|}
				\hline
				Comunicazione sinaptica & Comunicazione endocrina\\
				\hline
				\makecell{Pi\`u veloce grazie alla\\distanza ridotta tra le cellule} & \makecell{Pi\`u lenta in quanto le ghiandole possono\\trovarsi a distanza notevoli}\\
				\hline
				\makecell{Il neurotrasmettitore \`e presente\\in concentrazioni pi\`u elevate} & \makecell{Gli ormoni si trovano in basse concentrazioni}\\
				\hline
				\makecell{Causa affaticamento a causa della grande\\produzione di neurotrasmettitori} & \makecell{Non va incontro ad affaticamento}\\
				\hline
				\makecell{I recettori sono poco affini alla molecola\\compensando la grande produzione\\di neurotrasmettitore} & \makecell{I recettori sono molto specifici e sensibili,\\permettendo una bassa produzione\\dell'ormone}\\
				\hline
			\end{tabular}
		\end{table}

	\subsection{Comunicazione tramite gap-junction}
	Le gap-junction sono giunzioni che mettono in comunicazione due cellule vicine in modo diretto.
	Ci sono proteine che creano un ponte tra le cellule che permette la condivisione di elementi citoplasmatici e piccole molecole come correnti ioniche.
	I gap sono delimitati da connessine.

		\subsubsection{Connessine}
		Le connessine sono strutture monomeriche che si associano in $6$ formando un connessone.

			\paragraph{Canali}
			Due connessoni formano un canale che pu\`o essere omotipico se formato dalle stesse connessine o eterotipico, se formato da connessine eterogenee.

		\subsubsection{Chiusura del canale}
		Il canale pu\`o essere chiuso quando una delle due cellule va incontro ad apoptosi, andando cos\`i a impedire al fattore che ha scatenato l'evento di passare all'altra cellula.
		Altri fattori di chiusura del canale possono essere un abbassamento del $pH$ o un aumento degli ioni calcio.

\section{Trasmissione sinaptica}
La trasmissione sinaptica \`e una delle trasmissioni che produce una risposta pi\`u velocemente.
\`E legata al rilascio di neurotrasmettitori a livello della terminazione sinaptica.
Nello spazio intersinaptico il neurotrasmettitore si lega a un recettore che attiva una risposta cellulare.
Nella fibra sinaptica si formano vescicole contenenti il neurotrasmettitore.
La vescicola si fonde alla membrana a causa ad un aumento dei livelli di calcio causati da un potenziale elettrico causando il rilascio del neurotrasmettitore nello spazio intersinaptico.
La membrana post-sinaptica contiene dei recettori per tale neurotrasmettitore che andranno ad attivare un pathway di trasduzione.

	\subsection{Acetilcolina}
	L'acetilcolina \`e un neurotrasmettitore alla base di segnali che causano contrazione muscolare o il rilascio di sostanze da ghiandole.

		\subsubsection{Processo di segnalazione}
		\begin{multicols}{2}
			\begin{enumerate}
				\item L'enzima \emph{acetill CoA} aggiunge un gruppo acetile alla colina producendo aetilcolina.
					La colina viene introdotta nella cellula grazie ad un co-trasporto guidato dal gradiente di sodio.
				\item L'acetilcolina viene pompata all'interno di vescicole.
				\item Le vescicole vengono rilasciate dal neurone nell'ambiente extracellulare.
				\item L'acetilcolina si lega a un recettore dando inizio alla trasduzione del segnale.
				\item L'acetilcolina viene scissa in colina attraverso actilcolinesterasi che viene reimportata nella cellula nervosa.
			\end{enumerate}
		\end{multicols}

		\subsubsection{Miastenia gravis}
		La miastenia gravis \`e una malattia autoimmune in cui il sistema immunitario produce anticorpi contro i recettori dell'acetilcolina.
		Questo causa una contrazione debole dei muscoli o una sua assenza.
		Farmaci che contrastano la malattia vanno a bloccare l'acetilcolinesterasi.
		Questi bloccano il riciclo dell'acetilcolina che pu\`o agire per pi\`u tempo.

\section{Recettori}
I  recettori sono molecole presenti sulla membrana delle cellule, tipicamente proteine, responsabili della ricezione dei segnali extracellulari.
Quando ricevono il segnale vengono attivati e attivano il processo di trasduzione del segnale.

	\subsection{Canali ionici}
	I canali ionici o ionotropici sono canali di membrana che mediano risposte molto rapide.
	All'arrivo del segnale questo si lega al recettore che cambia conformazione aprendo il canale.
	Il canale permette l'entrata di ioni che possono cambiare il potenziale di membrana.

		\subsubsection{Esempi}

			\paragraph{Cervello}
			\begin{multicols}{2}
				\begin{itemize}
					\item Recettore per il glutammato.
					\item \emph{CNS}.
					\item \emph{GABA}.
				\end{itemize}
			\end{multicols}

			\paragraph{Muscoli}
			\begin{multicols}{2}
				\begin{itemize}
					\item Recettore nicotinico.
					\item Recettori per l'acetilcolina.
				\end{itemize}
			\end{multicols}
		
		\subsubsection{Recettore nicotinico}
		Il recettore nicotinico \`e un recettore canale che si apre o chiude permettendo un passaggio dello ione sodio.
		Spesso la nicotina \`e la molecola target.

	\subsection{Recettori di superficie collegati a proteina \emph{G}}
	I recettori di superficie collegati a proteine $G$ sono recettori di tipo muscarinico o legati a proteine $G$ trimeriche.

		\subsubsection{Recettore muscarinico}
		Il recettore muscarinico non \`e un canale ma un recettore composto da $7$ domini di membrana e una porzione citoplasmatica.
		Il segnale d\`a origine a una serie di messaggi per la trasduzione di membrana.
		Attiva i messaggeri metabotropici come \emph{cAMP} e \emph{IP3} e le proteine $G$ trimeriche.

	\subsection{Recettori di superficie collegati ad enzimi}
	I recettori di superficie collegati ad enzimi sono recettori che quando legano la molecola specifica vanno ad attivare la loro componente enzimatica dando il via al pathway di segnalazione intracellulare.

\section{Messaggeri}
Si intende per messaggeri le molecole di segnalazione extracellulare. 

	\subsection{Ossido di azoto}
	L'ossido di azoto \`e una molecola segnale che si trova nel sistema nervoso, muscolare e in altre tipologie di cellule.

		\subsubsection{Effetto sulla muscolatura}
		L'ossido di azoto va ad agire sulla muscolatura che riveste le cellule endoteliali che delimitano i vasi sanguigni causandone la dilatazione o contrazione.

			\paragraph{Processo}
			\begin{multicols}{2}
				\begin{enumerate}
					\item La cellula rilascia acetilcolina.
					\item L'acetilcolina si lega alla membrana della cellula endoteliale e d\`a origine a una serie di cascate di segnale che sintetizzano \emph{NO}.
					\item L'ossido di azoto prodotto diffonde e agisce sulle cellule muscolari.
					\item \emph{NO} nella cellula muscolare attiva un enzima che aumenta il \emph{CGMP}.
					\item L'aumento di \emph{cGMP} determina il miorilassamento e la vasodilatazione favorendo l'afflusso di sangue al cuore.
				\end{enumerate}
			\end{multicols}

			\paragraph{Viagra}
			Il Viagra inibisce l'enzima che produce \emph{GTP} aumentando i valori di \emph{cGMP} causando una vasodilatazione favorendo l'afflusso di sangue.
			Inibisce l'enzima che degrada \emph{cGMP}.

	\subsection{Ormoni steroidei}
	Gli ormoni steroidei vengono prodotti a partire dal colesterolo.
	L'interazione dell'ormone con il recettore causa la trascrizione e traduzione di fattori specifici.

	\subsection{Messaggeri secondari}
	Si dicono messaggeri secondari le molecole di segnalazione intracellulari.
	Sono prodotti in seguito all'attivazione di una certa cascata di segnale e trasmettono i segnali ricevuti dai recettori di superficie all'interno della cellula.
	Vengono attivati in maniera rapida in seguito di una trasduzione del segnale.
	Vengono rimossi rapidamente quando il segnale viene interrotto.

		\subsubsection{Esempi}
		\begin{multicols}{2}
			\begin{itemize}
				\item \emph{cAMP}.
				\item \emph{cGMP}.
				\item \emph{IP3}, inositolo-3-fosfato prodotto da un peptide di membrana.
			\end{itemize}
		\end{multicols}

\section{Molecular switches}
I molecular switches sono molecole che vengono attivate o disattivate durante la trasduzione del segnale.

	\subsection{Tipologie}

		\subsubsection{Molecole sensibili alle fosforilazione}
		Le molecole sensibili alla fosforilazione vengono attivate tramite fosforilazione andando a trasdurre e attivare altre componenti.
		La disattivazione avviene attraverso defosforilazione.
		Le chinasi aggiungono il gruppo fosfato mentre le fosfatasi lo rimuovono.
		La cascata di segnale attiva una delle due, determinando il destino della molecola.

		\subsubsection{Molecole legate all'idrolisi di \emph{GTP}}
		Le molecole legate all'idrolisi di \emph{GTP} vengono dette \emph{GTP binding proteins}.
		Quando sono attive presentano legato \emph{GTP}, mentre quando disattive legano \emph{GDP}.

			\paragraph{Processo di attivazione}
			\begin{multicols}{2}
				\begin{itemize}
					\item All'arrivo del segnale la molecola lega \emph{GTP} e passa ad uno stato attivo.
					\item La molecola si disattiva quando idrolizza il \emph{GTP} in \emph{GDP}.
				\end{itemize}
			\end{multicols}
			Si nota come l'idrolisi del \emph{GTP} \`e molto rapida, pertanto la disattivazione \`e un processo molto veloce.

			\paragraph{Fattori coinvolti}
			In molti casi cofattori o proteine regolatrici regolano il legame con \emph{GTP} e la sua idrolisi.
			I cofattori vengono pertanto attivati o disattivati durante una cascata di segnale, favorendo il rispettivo processo.
				
				\subparagraph{\emph{GEF}}
				Le proteine \emph{GEF} o fattori di separazione del nucleotide guaninico promuovono il distacco del \emph{GDP} e il legame con \emph{GTP}.
				Velocizzano l'attivazione.

				\subparagraph{\emph{GAP}}
				Le proteine \emph{GAP} o proteine di attivazione di \emph{GTPasi} promuovono l'idrolisi di \emph{GTP}.
				Velocizzano l'inattivazione.

			\paragraph{Esempi}
			Le molecole che legano \emph{GTP} sono coinvolte in molti pathway chiave come mitosi o motilit\`a.
			Possono determinare inoltre il grado di dinamicit\`a della cellula e la direzione del suo spostamento.

				\subparagraph{Famiglia \emph{Rho}}
				Le proteine della famiglia \emph{Rho} sono \emph{GTPasi} monomeriche.
				Sono associate con il citoscheletro di cui regolano lunghezza e organizzazione, determinando cos\`i il movimento in una direzione specifica.
				I segnali extracellulari governano la loro attivit\`a determinando la direzione del movimento.
				I segnali si legano a diversi recettori che convergono sulle proteine \emph{Rho} all'interno della cellula.
				Possiedono diversi domini di interazione per la trasduzione del segnale:
				\begin{multicols}{2}
					\begin{itemize}
						\item \emph{SH1}: domini di omologia \emph{Src2}, steroid receptor coactivator.
						\item Domini di legame alla fosfotirosina.
						\item \emph{SH3}, domini di omologia a \emph{Scr3}.
						\item Domini $PH$ di omologia alla pleckstrina.
					\end{itemize}
				\end{multicols}

				\subparagraph{Famiglia \emph{Ras}}
				Le proteine della famiglia \emph{Ras} sono regolate da \emph{GTP}.
				Il legame con \emph{GTP} causa un cambio conformazionale che permette l'interazione con un'altra proteina.
				Le molecole \emph{Ras-GAP} promuovono l'idrolisi del \emph{GTP} portando le molecole in una forma inattiva.

	\subsection{Proteine \emph{G} trimeriche}
	Le proteine \emph{G} trimeriche sono un esempio di proteine che legano \emph{GTP}.

		\subsubsection{Recettori}
		I recettori associati a proteine \emph{G} trimeriche sono i recettori di superficie pi\`u presenti.
		Mediano la maggior parte dei segnali proveniente dall'ambiente extracellulare.
		Mediano per esempio i segnali del sistema olfattivo e visivo.

			\paragraph{Struttura}
			Sono formati da una singola catena polipeptidica che attraversa il doppio strato lipidico $7$ volte formando una struttura cilindrica.
			Questa struttura contiene un sito di legame centrale per il ligando.
			La proteina $G$ trasduce il segnale, che contiene una parte $N$-terminale extracellulare che viene legata dalla molecola di segnale e una parte $C$-terminale che subisce modifiche in base alla presenza di tale molecola determinando la cascata di segnale.

			\paragraph{Recettore muscarinico}
			Il recettore muscarinico possiede un dominio citoplasmatico capace di attivare le proteine $G$ trimeriche a cui \`e associato.
			Mutazioni in questi siti determinano un blocco della trasduzione del segnale o una sua alterazione.

		\subsubsection{Struttura}
		Le proteine $G$ trimeriche possono essere associate al recettore stabilmente o essere reclutate ad esso quando questo viene attivato.
		Ci sono varie tipologie specifiche per i recettori e le proteine bersaglio, ma tutte presentano una struttura simile.

			\paragraph{Subunit\`a}
			\begin{multicols}{2}
				\begin{itemize}
					\item Subunit\`a $\alpha$: lega \emph{GTP} e \emph{GDP}, associata alla membrana.
					\item Subunit\`a $\gamma$: associata alla membrana.
					\item Subunit\`a $\beta$: non associata alla membrana.
				\end{itemize}
			\end{multicols}

				\subparagraph{Stato non stimolato}
				Quando la proteina non \`e attivata la subunit\`a $\alpha$ lega \emph{GDP} e la proteina \`e inattiva.
				Interagisce con la membrana attraverso le subunit\`a $\alpha$ e $\gamma$.

				\subparagraph{Stato stimolato}
				Il legame di $\alpha$ con \emph{GTP} determina un cambio conformazionale di $\alpha$ che determina il rilascio della proteina $G$ dal recettore.
				Il rilascio promuove la dissociazione di $\alpha$ da $\gamma$ e $\beta$.

		\subsubsection{Attivazione}
		Nello stato inattivo iniziale si trova il recettore legato alla proteina $G$ che lega il \emph{GDP}.
		A questo punto l'acetilcolina viene rilasciata e lega il recettore.
		Questo determina un cambio conformazionale della porzione citoplasmatica del recettore che permette il reclutamento e l'interazione del recettore con la proteina $G$ trimetica.
		Dopo l'interazione viene favorito lo scambio di \emph{GDP} con \emph{GTP}.
		La proteina $G$ cambia conformazione e avviene il distacco di $\beta$ e $\gamma$ da $\alpha$, permettendo l'interazione con altre molecole.
		La subunit\`a $\alpha$ pu\`o interagire con la proteina target effettore attivandola.

		\subsubsection{Disattivazione}
		La disattivazione avviene con l'idrolisi di \emph{GTP} in \emph{GDP}.
		La subunit\`a $\alpha$ non \`e pi\`u in grado di interagire con l'effettore.
		Questo favorisce l'interazione tra $\alpha$, $\beta$ e $\gamma$ e la ricostituzione della proteina $G$ trimerica inattiva.

		\subsubsection{Effettori}
		Si intende per effettore la molecola che viene attivata dalla proteina $G$ trimerica.
		Sono tipicamente enzimi che producono \emph{cAMP} o \emph{cGMP}, rispettivamente adenilatociclasi e guanilatociclasi.
		In particolare \emph{GRK} fosforila il recettore che legato da arrestina viene disattivato favorendo la sua endocitosi.

			\paragraph{\emph{AMP} ciclico \emph{cAMP}}
			La cascata di segnalazione di alcune proteine $G$ coinvolge la produzione di \emph{cAMP}.
			Questo messaggero secondario viene prodotto e degradato molto velocemente, permettendo variazioni rapide nella risposta.

				\subparagraph{Sintesi}
				La sintesi avviene attraverso la proteina di membrana adenilato ciclasi, che trasforma \emph{ATP} in \emph{cAMP}.
				In particolare tutti i recettori che agiscono tramite \emph{cAMP} sono legati da una proteina $G$ stimolatrice \emph{Gs}.
				
				\subparagraph{Degradazione}
				Tutti i segnali che riducono i livelli di \emph{cAMP} lo fanno attivando una proteina $G$ inibitrice \emph{Gi} che inibisce l'adenilato ciclasi.
				La degradazione avviene attraverso l'adenilato ciclasi fosfodiesterasi che idrolizza il \emph{cAMP} in adenosina $5'$ fosfato \emph{$5'$-AMP}.

				\subparagraph{\emph{Gs} e \emph{Gi} sono bersaglio di tossine batteriche}
				Alcune tossine batteriche possono inibire il normale spegnimento della proteina $G$ andando a bloccare la sua attivit\`a \emph{GTPasica}.
				La subunit\`a $\alpha$ rimane pertanto costantemente attivata, mantenendo pertanto attive le vie a valle anche in assenza del segnale esterno iper-attivando cos\`i fattori.
				
				\subparagraph{Colera}
				Nelle cellule epiteliali intestinali la tossina del colera blocca la proteina \emph{G$\alpha$} legata a \emph{GTP}, andando a bloccare la sua idrolisi.
				Le cellule continuano cos\`i ad avere proteine $G$ attive e a produrre grandi quantit\`a di \emph{cAMP}.
				Questo causa uno squilibrio osmotico portando a il rilascio di ioni \emph{$Na^+$} e di acqua nell'intestino.
				L'organismo perde acqua e va incontro a disidratazione con squilibrio degli elettroliti.

		\subsubsection{Modulazione di un effettore}
		Quando un recettore \`e iper-stimolato vengono attivati meccanismi che permettono di bloccare la traduzione del segnale attraverso una proteina come \emph{CRK}.
		Questa fosforila la pate citoplasmatica del recettore inattivandolo.
		Dopo la fosforilazione il recettore viene riconosciuto da una proteina \emph{Resti} che si lega ad esso e impedisce la sua interazione con la proteina $G$ trimerica che pertanto non viene attivata e non trasduce il segnale.

\section{Regolazione di un recettore}
Le cellule possono regolare la loro sensibilit\`a al segnale in base a diversi meccanismi:
\begin{multicols}{2}
	\begin{itemize}
		\item Sequestro del recettore mediante endocitosi, l'assenza di degradazione permette un ripristino rapido della sensibilit\`a, inoltre un recettore endocitato pu\`o continuare a inviare il segnale.
		\item Down-regulation: questo processo avviene attraverso degradazione.
		\item Inattivazione del recettore diretta.
		\item Inattivazione indiretta del recettore.
		\item Produzione di proteine inibitorie che agiscono sugli effettori.
	\end{itemize}
\end{multicols}
L'adattamento ai segnali delle cellule permette ad esse di rispondere a variazioni di concentrazioni di un ligando in un ambito molto vasto di sue concentrazioni.

	\subsection{Esempio}
	L'ormone tiroideo stimolante, l'ormone reinocortico tropico, l'adrenalina e il glucagone determina un'attivazione della proteina $G$.

		\subsubsection{\emph{PKA}}
		\emph{cAMP} si lega e attiva la chinasi \emph{PKA}, composta da $2$ subunit\`a regolatorie e due catalitiche \`e una chinasi dipendente da \emph{cAMP}.
		Regola l'attivit\`a di proteine bersaglio, segnali o effettori.

			\paragraph{Stato inattivo}
			Nello stato inattivo \emph{PKA} \`e formata da due subunit\`a catalitiche e due regolatrici.

			\paragraph{Stato attivo}
			Affinch\`e \emph{PKA} venga attivata ciascuna subunit\`a regolatoria lega due \emph{cAMP}.
			Questo determina un loro cambio conformazionale e il distacco delle subunit\`a catalitiche che, attive possono fosforilare proteine nel citoplasma o andare nel nucleo fosforilando fattori trascrizionali per la trascrizione di geni specifici.

		\subsubsection{\emph{CREB}}
		\emph{CREB} \`e una proteina legante \emph{CRE}, legata da \emph{cAMP}.
		Il messaggero lega la sequenza \emph{CRE} o \emph{cAMP} responsive element.
		Quando fosforilata \emph{CREB} viene legata da \emph{CBP} (\emph{CREB} binding protein) formando un complesso trascrizionalmente attivo.
		Il complesso lega agli elementi \emph{CRE}, una sequenza promotrice attivata da \emph{cAMP}.
		Si nota come \`e un fattore trascrizionale.

	\subsection{Vie attivate da \emph{cAMP}}
	Una delle vie attivate da \emph{cAMP} \`e quella della mobilitazione di glucosio nel sangue.
	Lo zucchero viene accumulato sotto forma di glicogeno e quando necessario viene scisso nel sangue.

		\subsubsection{Ormoni coinvolti}
		\begin{multicols}{2}
			\begin{itemize}
				\item Insulina: rimuove glucosio e favorisce la produzione di glicogeno.
				\item Glucagone: scinde il glicogeno rilasciando il glucosio nel sangue.
			\end{itemize}
		\end{multicols}

		\subsubsection{Via di segnalazione}
		Il glucone e l'adrenalina portano all'attivazione della proteina $G$ che determina aumenti nei livelli di \emph{cAMP} e l'attivazione di \emph{PKA} nel citoplasma che una volta attivata:
		\begin{multicols}{2}
			\begin{itemize}
				\item Fosforila la glicogeno sintestasi che blocca la sintesi di glicogeno quando fosforilata.
				\item Fosforila la fosforilasi chinasi che attiva la glicogeno fosforilasi, l'enzima che degrada il glicogeno liberando glucosio.
			\end{itemize}
		\end{multicols}

\section{Inositolo 3 fosfato}

	\subsection{Sintesi}
	L'inositolo $3$ fosfato \emph{IP3} viene prodotto a partire dal fosfatidil inositolo, un lipide di membrana che subisce due fosforilazioni:
	\begin{multicols}{2}
		\begin{itemize}
			\item Fosfatidil inositolo chinasi $P$ che usa \emph{ATP} producendo \emph{Inositolo 4 fosfato}.
			\item Fosfatidil inositolo fosfato chinasi \emph{PIP} che usa \emph{ATP} producendo \emph{inositolo 4, 5 difosfato}.
		\end{itemize}
	\end{multicols}
	Una volta fosforilato viene tagliato ad opera della fosfolipasi $C$, che viene attivata a seguito di stimolazioni ormonali.
	Vengono pertanto prodotti \emph{inositolo 1, 4, 5 trifosfato} e diaglicerolo.
	Vengono pertanto rilasciati due messaggeri secondari:
	\begin{multicols}{2}
		\begin{itemize}
			\item \emph{IP3}: viene liberato nel citoplasma e agisce a livello del reticolo endoplasmatico.
			\item Diaglicerolo: rimane nella membrana e pu\`o reclutare fattori dal citoplasma della membrana attivandoli, attiva la proteina chinasi $C$ \emph{PKC}, ecicosanoidi e \emph{$Ca^{2+}$}.
		\end{itemize}
	\end{multicols}

	\subsection{Caratteristiche}
	Il \emph{IP3} \`e una molecola solubile in acqua e quando lascia la membrana plasmatica si diffonde rapidamente nel citosol.
	Raggiunge la membrana del reticolo endoplasmatico e apre i canali del \emph{$Ca^{2+}$}.
	L'aumento della concentrazione di calcio causa una propagazione del segnale che viene rapidamente inattivato mediante defosforilazione ad opera di fosfatasi.
	Il calcio \`e rapidamente riassorbito dal citoplasma.

	\subsection{Diaglicerolo}
	Il diaglicerolo rimane immerso nella membrana plasmatica con funzione di segnalazione.
	Attiva la proteina chinasi $C$ \emph{PKC} calcio dipendente.
	L'aumento di calcio nel citosol causa la traslocazione di \emph{PKC} dal citosol alla membrana dove viene attivata e fosforila la proteina bersaglio.
	Il diaglicerolo pu\`o anche essere usato nella sintesi di eicosanoidi: molecole di segnalazione lipidiche che partecipano a risposte infiammatorie.

	\subsection{Calcio}
	Il calcio \`e uno dei messaggeri secondari presenti in minor quantit\`a nella cellula $10^{-7}M$.
	Questo avviene in quanto \`e un mediatore di segnalazione molto efficace e viene accumulato nel reticolo endoplasmatico liscio, nei mitocondri e nell'ambiente extracellulare con concentrazione maggiore.


		\subsubsection{Mantenimento dei bassi livelli di calcio}
		La concentrazione degli ioni calcio viene mantenuta bassa attraverso sistemi sia passivi che attivi.
		Il calcio viene spostato verso l'esterno della cellula o in compartimenti intracellulari attraverso:
		\begin{multicols}{2}
			\begin{itemize}
				\item Antiporto \emph{$Na^+/Ca^{2+}$}: nelle cellule nervose e muscolari, dove si accoppia l'afflusso di calcio all'ingresso di sodio.
				\item Pompa del \emph{$Ca^{2+}$}: canali \emph{ATP}-dipendenti che pompano ioni calcio fuori dal citosol.
			\end{itemize}
		\end{multicols}

		\subsubsection{Meccanismo di compartimentazione}
		La cellula utilizza delle pompe per pompare il calcio nel reticolo endoplasmatico contro un gradiente molto elevato.
		Se si accumula tanto ione nel compartimento, quando lo si apre esce molto rapidamente, permettendo un aumento rapido di concentrazioni e un rapido sequestramento.

		\subsubsection{Utilizzi}

			\paragraph{Fecondazione}
			Quando lo spermatozoo raggiunge una cellula uovo si ha la fusione tra la sua membrana e la membrana della cellula uovo.
			Per prevenire l'ingresso di altri spermatozoi, la membrana della cellula si ispessisce in modo da impedire l'ingresso di altri spermatozoi.
			Questo avivene grazie al calcio: viene rilasciata un'onda di calcio che percorre l'uovo fino al sito di ingresso dello spermatozoo.

			\paragraph{Sistema nervoso}
			La cellula neuronale sequestra e secerne ioni calcio in maniera molto rapida e coordinata.

		\subsubsection{Calmodulina}
		La calmodulina \`e una proteina che si lega al calcio per trasmettere il segnale: \`e un recettore intracellualare di \emph{$Ca^{2+}$}.
		Si trova in tutte le cellule eucariote.
		Una volta attivata agisce legando altre proteine attivandole come la \emph{CaMkII} o chinasi calcio dipendente.

			\paragraph{Struttura}
			La calmodulina \`e una catena polipeptidica con $4$ siti di legame ad alta affinit\`a per lo ione calcio.
			Viene attivata dal legame con esso subendo un cambiamento conformazionale.
			La conformazione attiva viene assunta dopo il legame di due ioni.

			\paragraph{Legame con il calcio}
			Quando la calmodulina si lega con il calcio la molecola attivata si lega alla proteina bersaglio cambiando conformazione.
			Le due estremit\`a si avvicinano entrando in contatto.
			Le due subunit\`a adiacenti sono in grado di autofosforilarsi prolungando l'attivit\`a della proteina oltre il segnale del calcio.
			L'enzima mantiene la sua attivit\`a fino all'intervento di una fosfatasi.

	
	\subsection{Recettori del sistema olfattivo}
	I neuroni olfattori con un unico dendrita presentano ciglia immerse in un muco.
	Gli odori si dissolvono in esso e glomeruli olfattori li percepiscono.
	La trasduzione del segnale viene attivata grazie alla presenza di una molecola che si lega ai recettori olfattivi di tipo muscarinico sulla membrane epiteliale del naso.

		\subsubsection{Funzionamento dei recettori}
		I recettori agiscono tramite \emph{cAMP}.
		\begin{multicols}{2}
			\begin{itemize}
				\item Quando stimolati dagli odori i recettori legano una proteina $G$ specifica.
				\item Il cambio di conformazione dei recettori attiva la proteina $G$.
				\item La proteina $G$ attiva l'adenilato ciclasi.
				\item L'aumento dei livelli di \emph{cAMP} causa un apertura dei canali sodio.
				\item Il sodio entra causando un aumento del potenziale di membrana.
				\item La depolarizzazione della cellula causa una propagazione dell'informazione al sistema nervoso.
			\end{itemize}
		\end{multicols}

	\subsection{Recettori del sistema visivo}
	Il sistema visivo viene usato per studiare i meccanismi dell'accrescimento delle cellule nervose e di come riescano a raggiungere il loro target.
	Studiato negli anfibi e nei bassi vertebrati che presentano occhi molto grandi, un sistema nervoso pi\`u semplice e la capacit\`a di rigenerare le cellule della retina.
	
		\subsubsection{Funzionamento dell'occhio}
		L'occhio funziona come una camera oscura.
		Davanti si trova una lente detta cristallino.
		Quando un'immagine attraversa il cristallino viene capovolta e il cervello la elabora attraverso un sistema di proiezioni topografiche.
		
			\paragraph{Retina}
			La retina pu\`o essere divisa in una sezione nasale, una temporale, una ventrale e una dorsale.
			L'immagine si proietta lungo la retina nelle quattro sezioni.
			Ogni posizione spaziale viene vista da una diversa posizione della retina, connessa mediante il nervo ottico ad una diversa posizione del tetto ottico.
			Le cellule della retina che si trovano nella parte ventrale inferiore si legano con la parte superiore del tetto o corteccia visiva, mentre quelle della parte superiore le trasferisce nella parte inferiore.
			In questo modo l'immagine viene riproiettata in maniera corretta a livello del sistema nervoso.

			\paragraph{Visione stereoscopica dei mammiferi}
			Nei mammiferi l'occhio sinistro invia informazioni sia alla parte sinistra che alla parte destra del cervello.
		
			\paragraph{Embrione di pollo}
			Nell'embrione del pollo l'occhio \`e una struttura molto grande e facilmente manipolabile.
			Il tetto ottico \`e la parte deputata a ricevere informazioni in campo visivo.
			\`E possibile evidenziare l'embrione a diversi stadi dello sviluppo e osservare il sistema nervoso.
		
			\paragraph{Zebrafish}
			Lo zebrafish \`e il modello utilizzato per lo studio della visione.
			\`E analogo agli anfibi: cellule della retina creano sinapsi con la parte controlaterale del sistema nervoso.
			Per studiare cosa un pesce \`e in grado di vedere si utilizza un approccio istologico attraverso una sezione dell'occhio o un approccio di valutazione della sua pigmentazione: i melanofori sono collegati alla vista e permette al pesce di mimetizzarsi.
			La presenza di una morfologia dei melanofori alterata \`e indice di problemi a livello visivo.
	
			\paragraph{Mutanti della retina}
			I mutanti della retina possono presentare:
			\begin{multicols}{2}
				\begin{itemize}
					\item Mancato sviluppo del cristallino.
					\item Degradazione del cristallino.
					\item Tumori del cristallino.
					\item Mancanza di retina o sua traslocazione.
					\item Degenerazione di alcuni sottotipi di cellule della retina.
				\end{itemize}
			\end{multicols}

		\subsubsection{Trasduzione del segnale visivo}
		I sistemi di rivelazione dei segnali visivi sono sensibili ed elaborati.
		Sono coinvolti canali ionici regolati da nucleotidi ciclici come \emph{cGMP}.
		La trasduzione \`e una risposta mediata da proteine $G$.
		Il segnale colpisce la retina e i fotorecettori e la luce stimola la loro attivit\`a che causano una diminuizione di livelli di \emph{cGMP}.

			\paragraph{Fotorecettori}

				\subparagraph{Bastoncelli}
				I bastoncelli sono deputati alla visione della luce monocromatica.
				Sono formati da un corpo cellulare, un segmento interno e una terminazione sinaptica.
				\`E presente un cromoforo legato alla rodopsina che cambia conformazione all'arrivo del fotone.
				Il segmento esterno dei bastoncelli contiene dischi con la rodopsina, un recettore muscarinico caratterizzato da diversi domini di membrana.

				\subparagraph{Coni}
				I coni sono deputati alla visione dei colori.

				\subparagraph{Cromoforo}
				Il cromoforo, contenuto nel recettore o $11$-cis retinale.

				\subparagraph{Attivazione}
				\begin{multicols}{2}
					\begin{enumerate}
						\item All'arrivo del fotone viene colpito il $11$-cis retinale che passa a una conformazione trans.
						\item Si altera la rodopsina che attiva la trasducina, una proteina $G_t$.
						\item L'attivazione della proteina $G$ causa l'attivazione della fosfodiesterasi da parte di $\alpha$.
						\item Viene degradato il \emph{cGMP} ciclico che chiude i canali sodio.
						\item Si nota una riduzione di rilascio del neurotrasmettitore.
						\item Attivazione della guanilato ciclasi per far risalire i livelli di \emph{cGMP}, che se rimanessero bassi causerebbero un blocco del processo.
					\end{enumerate}
				\end{multicols}

				\subparagraph{Condizioni di buio}
				In condizioni di buio si notano:
				\begin{multicols}{2}
					\begin{itemize}
						\item Livelli di \emph{cGMP} alti.
						\item Canali di sodio aperti che permettono il rilascio di neurotrasmettitori.
						\item Corrente al buio: il rilascio di un neurotrasmettitore avviene quando non ci sono stimoli luminori.
					\end{itemize}
				\end{multicols}

				\subparagraph{Condizioni di luce}
				In condizione di luce si nota:
				\begin{multicols}{2}
					\begin{itemize}
						\item Diminuzione dei livelli di \emph{cGMP}.
						\item Canali del sodio chiusi.
						\item Terminazione del rilascio del neurotrasmettitore.
					\end{itemize}
				\end{multicols}

			\paragraph{Tipologie di neurotrasmettitore}
			Un neurotrasmettitore pu\`o essere:
			\begin{multicols}{2}
				\begin{itemize}
					\item Eccitatorio: bloccandone il rilascio non si ha pi\`u eccitazione.
					\item Inibitorio: bloccandone il rilascio si blocca l'inibizione e si ha una risposta eccitatoria della retina all'arrivo del fotone.
				\end{itemize}
			\end{multicols}

\section{Recettori ad attivit\`a enzimatica intrinseca}
I recettori ad attivit\`a enzimatica intrinseca sono recettori di superficie collegati ad enzimi.
Sono tipicamente proteine di membrana monopasso con un dominio catalitico.
Sono tipicamente tirosine-chinasi e si legano tipicamente a fattori di crescita e ormoni.

	\subsection{Classificazione}
	I recettori ad attivit\`a enzimatica intrinseca possono essere divisi in classi:
	\begin{multicols}{2}
		\begin{itemize}
			\item Recettori con attivit\`a tirosina chinasi.
			\item Recettori associati a tirosina chinasi per:
				\begin{itemize}
					\item Antigene.
					\item Interleuchine.
					\item Integrine.
					\item Citochine.
				\end{itemize}
			\item Recettori con attivit\`a serina/treonina chinasica.
			\item Recettori guanilinico ciclasi.
			\item Tirosina fosfatasi simili a recettori.
		\end{itemize}
	\end{multicols}

	\subsection{Recettori tirosina chinasi \emph{RTK}}
	I recettori tirosina chinasi mediano molte proteine di segnalazione extracellulari.
	Il legame con la proteina segnale attiva il dominio tirosina chinasi che fosforila le catene laterali di tirosine creando un sito di attacco per proteine di segnale intracellulari.

	\subsection{Recettori per le citochine}
	Le citochine sono importanti nell'attivit\`a del sistema immunitario e sono agenti che promuovono la mitosi.
	Si legano ai recettori di membrana associati a tirosine chinasi specifiche dette \emph{JAK1,2,3} e \emph{TIC2}.

		\subsubsection{\emph{JAK-STAT}}
		Il signal transducer and activators of transcription \emph{JAK-STAT} presenta delle proteine a doppia faccia in grado di fosforilarsi e fosforilare quando vengono attivate.
		Normalmente il recettore \`e un monomero che dimerizza nel momento in cui si lega alla citohina.
		I due monomeri si uniscono facendo entrare in contatto le due \emph{JAK} che si autofosforilano a vicenda e fosforilano a loro volta un recettore.

			\paragraph{Fosforilaizone del recettore}
			La fosforilazione del recettore recluta e attiva membri della famiglia \emph{STAT1,2}, regolatori trascrizionali a livello del recettore.
			La via di segnalazione fornisce una delle vie pi\`u dirette per la modifica trascrizionale: la fosforilazione del recettore causa fosforilazioni e reclutamento di \emph{STAT1} e \emph{STAT2} dalle \emph{JAK}.
			Questi passaggi permettono l'attivazione sequenziale di molecole e proteine per una trasduzione coretta del sengale.

			\paragraph{Processo di attivazione trascrizionale}
			\begin{multicols}{2}
				\begin{enumerate}
					\item La fosforilazione determina la possibilit\`a dei monomeri \emph{STAT1,2} di dimerizzare.
					\item Il dimero entra nel nucleo e si lega ad altre proteine regolatorie.
					\item Il complesso si lega nella zona promotrice di geni stimolandone la trascriizone.
					\item Il processo converge in una regolazione trascrizionale specifica per geni che vengono regolati dalle citochine.
				\end{enumerate}
			\end{multicols}
	
	\subsection{Recettori con attivit\`a chinasica}
	I recettori con attivit\`a chinasica sono recettori per fattori morfogenici o trasformanti.
	La trasduzione del segnale viene regolata da un fattore tumorale \emph{TGF$\beta$} o fattore trasformante di crescita $\beta$.

		\subsubsection{Superfamiglia del fattore \emph{TFG$\mathbf{\beta}$}}
		La superfamiglia del fattore \emph{TFG$\beta$} consiste di un insieme di proteine che agiscono da mediatori per la regolazione di vari funzioni biologiche.
		Comprende le attivine \emph{TGF$\beta$} e le e la famiglia \emph{BMP} (proteine morfogeniche dell'osso).

			\paragraph{Recettori}
			Queste proteine agiscono tramite recettori serina-treonina chinasi.
			Di questi ne esistono due classi di omodimeri simili.
			I membri di \emph{TGF$\beta$} legano i recettori inducendo l'avvicinamento delle due classi di recettori e la formazione di un complesso recettoriale attivo tetramerico.
			Questo contiene due recettori di tipo $I$ e due di tipo $II$ con attivit\`a di serina-treonina chinasica.

			\paragraph{Trasmissione del segnale al nucleo}
			\begin{multicols}{2}
				\begin{enumerate}
					\item Il recettore di tipo $II$ fosforila il recettore di tipo $I$.
					\item La fosforilaizone attiva la porzione catalitica del recettore $I$ che diventa in grado di fosforilare.
					\item Il recettore attivo si lega e fosforila i membri della famiglia \emph{Smad1,3}, dei regolatori trascrizionali.
					\item La fosforilazione di \emph{Smad1,3} permette la formazione di un eterodimero con \emph{Smad4}.
					\item L'eterodimero entra nel nucleo e si associa ad altri regolatori trascrizionali e attiva la trascrizione di geni per \emph{TGF$\beta$}.
					\item I recettori attivati per \emph{TGF$\beta$} possono determinare l'attivazione o l'inibizione della trasduzione del segnale.
				\end{enumerate}
			\end{multicols}

			\paragraph{Inibizione della trasduzione del segnale}
			L'inibizione della trasduzione del segnale viene svolta da \emph{Smad6,7} inibitorie.
			Tra i geni bersaglio attivati dai complessi \emph{Smad} si trovano geni che codificano per le \emph{Smad} inibitrici.
			Il processo \`e pertanto di feedback negativo.
			
				\paragraph{Meccanismi di inibizione}
				Le \emph{Smad} inibitorie:
				\begin{multicols}{2}
					\begin{itemize}
						\item Competono con altre \emph{Smad} interrompendo la trasduzione del segnale a livello del recettore.
						\item Competono con le altre impedendo la formaizone dei dimeri.
					\end{itemize}
				\end{multicols}

					\subparagraph{\emph{SMURF}}
					\emph{SMURF} \`e un'ubiquitina che viene reclutata a livello del recettore grazie alle \emph{Smad} inibitorie.
					L'ubiquitina si lega al recettore causando la sua degradazione.
					La trasduzione del segnale viene bloccata per molto tempo.

					\subparagraph{Fosfatasi}
					Un altro modo per disattivare il processo \`e attraverso fosfatasi che rimuovono il gruppo fosfato dal recettore di tipo $I$ disattivandolo.
	
\section{Muscolo}
La struttura fondamentale per la trasduzione del segnale per la contrazione del muscolo \`e la placca o giunzione neuromuscolare.
Questa viene protetta dalla cellula di Shwann.

	\subsection{Placca neuromuscolare}

		\subsubsection{Terminazione nervosa}
		La terminazione nervosa \`e ricca di mitocondri e vescicole che contengono acetilcolina.
		Le vescicole si fondono con la membrana e rilasciano il neurotrasmettitore nello spazio tra il motoneurone e il muscolo.

		\subsubsection{Membrana post-sinaptica}
		La membrana post-sinaptica contiene proteine che impediscono ai recettori di diffondersi nello spazio.
		Questo spazio presenta un gran numero di invaginazioni determinando un aumento della superficie.
		In quanto la comunicazione sinaptica \`e veloce ma meno specifica richiede un maggior numero di recettori.

	\subsection{Organizzazione del muscolo}
	Le fibre principali del tessuto muscolare sono fatte da fibre di actina e miosina.
	Le fibre muscolari si organizzano in fasci che formano il muscolo.
	
		\subsubsection{Sincizio}
		Il muscolo scheletrico \`e un sincizio: una fusione di cellule che si fondono tra di loro perdendo la propria individualit\`a ma mantenendo i nuclei.
		Si forma un'unica struttura polinucleata.

		\subsubsection{Taglio sagittale}
		I nuclei non si trovano all'interno delle fibre ma in periferia.
		Questo \`e dovuto alla funzione del nucleo di contrarsi.

		\subsubsection{Taglio longitudinale}
		Il taglio longitudinale del muscolo presenta molte striature trasversali.
		Le striature sono date dalla presenza delle due proteine principali del muscolo: actina e miosina.
		La maggior parte del citoplasma \`e occupato da miofibrille formate da fasci di filamenti spessi di miosina$2$ e filamenti sottili di actina.
		La fibra muscolare viene ricoperta da endomisio.
		Nel perimisio si trovano pi\`u fibre e pi\`u fasci costituiscono il muscolo circondato dall'epimisio.

		\subsubsection{Componenti del citoscheletro}
		Actina e miosina sono le componenti del citoscheletro con una struttura precisa.
		Sono in grado di interagire tra di loro in modo specifico.

			\paragraph{Actina}
			L'actina \`e una molecola filamentosa caratterizzata da una testa, un corpo mobile e un corpo.

			\paragraph{Miosina 2}
			La miosina 2 \`e una proteina filamentosa caratterizzata da una testa, un collo mobile e un corpo.
			\begin{multicols}{2}
				\begin{itemize}
					\item Testa: proteina globulare che interagisce con l'actina generando la contrazione.
					\item Corpi: i corpi interagiscono tra di loro dando stabilit\`a alla molecola.
					\item Coda: la coda \`e una doppia elica proteica.
				\end{itemize}
			\end{multicols}

			\paragraph{Interazione}
			L'interazione specifica tra actina e miosina $2$ \`e responsabile della contrazione muscolare e di altri movimenti nelle cellule non muscolari.
	
	\subsection{Contrazione muscolare}

		\subsubsection{Sarcomeri}
		Ogni miofibrilla \`e costituita da una serie di sarcomeri.
		I sarcomeri sono l'unit\`a contrattili e funzionali.

			\paragraph{Linee \emph{Z}}
			Le linee \emph{Z} si trovano all'estremit\`a del sarcomero.
			Sono formate dalla sovrapposizione di actina.

			\paragraph{Bande}
			All'interno del sarcomero si trova un'alternanza di bande scure dove si trova la miosina e chiare dove non c'\`e.

			\paragraph{Zona \emph{H}}
			La zona \emph{H} \` e la zona centrale contenente unicamente i filamenti di miosina.

			\paragraph{Linea \emph{M}}
			La linea \emph{M} nella parte mediana del sarcomero \`e il punto in cui sono ancorati i filamenti di miosina.
			Viene determinata dai punti di giunzione delle miosine.

			\paragraph{Banda \emph{A}}
			La banda \emph{A} si trova tra due linee $Z$ ed \`e centrale.
			\`E data dalla sovrapposizione dei filamenti di actina e miosina.

			\paragraph{Movimento dei sarcomeri}
			Quando i sarcomeri si muovano cambia la distanza tra le linee \emph{Z}, la linea \emph{A} rimane costante, la \emph{I} diventa pi\`u piccola e la \emph{H} sparisce.

			\paragraph{Contrazione}
			Durante la contrazione le linee \emph{Z} si avvicinano e il sarcomero si accorcia.

			\paragraph{Rilassamento}
			Durante il rilassamento le linee \emph{Z} si allontanano e il sarcomero si allunga.

		\subsubsection{Polarit\`a dei filamenti di actina}
		La polarit\`a dei filamenti di actina si inverte alla linea \emph{M} in modo che l'orientamento di actina e miosina risulti identico ad entrambi i lati del sarcomero.
		L'attivit\`a motrice della miosina sposta le due teste verso l'actina.
		In questo modo i filamenti di actina scorrono verso la linea \emph{M} causando un accorciamento del sarcomero e una contrazione.

		\subsubsection{Organizzazione del muscolo a livello cellulare}
		La membrana plasmatica forma invaginazioni che mettono in comunicazione la membrana con i tubuli \emph{T} del reticolo endoplasmatico liscio o sarcoplasmatico.
		Questo conteine calcio.

			\paragraph{Miosina \emph{2}}
			La miosina 2 \`e una proteina di grandi dimensioni costituita da una coppia di catene pesanti, una testa globulare, una coda ad $\alpha$-elica e due catene leggere.
			Le code si avvolgono a doppia elica e le catene leggere si associano alla testa formando la molecola completa.
			I filamenti muscolari sono formati da molecole di miosina associate tra di loro a livello della coda e all'actina attraverso le teste.
			L'orientamento si inverte alla linea \emph{M}.

	\subsection{Ciclo di contrazione}
	\begin{multicols}{2}
		\begin{enumerate}
			\item Il motoneurone rilascia acetilcolina che si lega ai recettori presenti nella giunzione.
			\item Questo apre i canali di calcio che porta a un'onda di polimerizzazione.
			\item L'onda di depolimerizzazione si trasmette attraverso i tubuli T.
			\item Arriva all'interno e apre i canali di \emph{$Ca^{2+}$} sul reticolo endoplasmatico.
			\item Il calcio interagisce con la troponina causandone un cambio formazionale che libera la tropomiosina.
			\item Ora miosina e actina sono in grado di interagire.
		\end{enumerate}
	\end{multicols}

		\subsubsection{Ciclo della miosina}
		\begin{multicols}{2}
			\begin{enumerate}
				\item La miosina \`e strettamente legata all'actina e dissociata da \emph{ATP}.
				\item La miosina lega \emph{ATP} e si dissocia dall'actina.
				\item L'idrolisi di \emph{ATP} da parte della testa libera l'energia necessaria allo scorrimento dei filamenti.
				\item Il cambio conformazionale della miosina permette lo scorrimento delle teste lungo i filamenti di actina.
			\end{enumerate}
		\end{multicols}

		\subsubsection{Rilascio di calcio}
		La contrazione del muscolo scheletrico \`e attivata da impulsi nervosi che stimolano il rilascio di \emph{$Ca^{2+}$} dal reticolo sarcoplasmatico.
		Questo rilascio nel citosol ne determina un aumento della concentrazione da $10^{-7}$ a $10^{-5}M$.
		L'aumento di concentrazione promuove la contrazione attraverso l'azione di due proteine: tropomiosina e troponina legate ai filamenti di actina.
		La troponina si lega alla tropomiosina e inibisce l'interazione miosina e actina.
		Il legame del calcio porta a una serie di cambiamenti conformaizonali della troponina che muovono la tropomiosoina liberando il sito del legame della miosina sull'actina.

		\subsubsection{Tropomiosina}
		La tropomiosina \`e una molecola filamentosa che sovrasta l'actina.
		Si adagia su di essa impedendo alle teste di miosina di interagire con l'actina in condizioni di riposo.

			\paragraph{Troponina}
			La troponina \`e una molecola globulare formata da:
			\begin{multicols}{2}
				\begin{itemize}
					\item Una subunit\`a sensibile al calcio.
					\item Una che interagisce con la tropomiosina.
					\item Una che interagisce con l'actina.
				\end{itemize}
			\end{multicols}
			
				\subparagraph{Funzione}
				La troponina rimuove il blocco della tropomiosina quando viene rilasciato calcio permettendo la contrazione.
				\begin{multicols}{2}
					\begin{enumerate}
						\item Il calcio viene rilasciato quando l'acetilcolina si lega al recettore attraverso un canale ionotropico.
						\item Il canale si apre creando un'onda di depolimerizzazione che arriva ai tubuli T.
						\item I tubuli sono in contatto con il reticolo endoplasmatico su cui si trovano canali del calcio sensibili a variazioni del potenziale di membrana.
						\item Il canale del calcio percepisce la depolimerizzazione e si apre.
						\item Il calcio esce e si lega alla troponina causandone un cambio conformazionale legato all'idrolisi di \emph{ATP}.
					\end{enumerate}
				\end{multicols}
				Il cambio conformazionale della troponina causa un interruzione dell'interazione tra miosina e actina.

		\subsubsection{Energia della contrazione}
		\emph{ATP} viene idrolizzaato durante ogni ciclo di attacco, spostamento e stacco dei ponti trasversali.
		\`E fondamentale per determinare il cambio conformazionale per lo stacco della miosina dall'actina.
		Alimenta inoltre la pompa del calcio sulla membrana.

			\paragraph{Presenza di ossigeno}
			In presenza di ossigeno si produce \emph{ATP} spostando un gruppo fosfato dalla creatina fosfato a \emph{ADP}.

			\paragraph{Carenza di ossigeno}
			In carenza di ossigeno si produce \emph{ATP} in anaerobiosi con produzione di acido lattico.
			L'acido lattico abbassa il \emph{pH} e il muscolo rimane contratto.

			\paragraph{Carenza di \emph{ATP}}
			In carenza di \emph{ATP} il muscolo riduce l'ampiezza delle contrazioni in quanto meno teste di miosina interagiscono con l'actina.
			La fatica \`e dovuta a un consumo di \emph{ATP} tale da non riuscire ad essere soddisfatto dal metabolismo.

		\subsubsection{Tetano muscolare}
		Il tetano muscolare \`e una malattia in cui un muscolo viene continuamente stimolato e non si rilassa causando tensione.
		Questo \`e dovuto al rilascio di calcio che rimane ad alte concentrazioni.
		Un alta frequenza di contrazioni il muscolo non si rilassa.

		\subsubsection{Rapporto forza velocit\`a}
		Il rapporto forza-velocit\`a della contrazione muscolare \`e determinato da:
		\begin{multicols}{2}
			\begin{itemize}
				\item Numero di ponti tra actina e miosina.
				\item Quantit\`a di calcio rilasciata dal reticolo sarcoplasmatico.
				\item Attivit\`a della \emph{ATPasi}.
				\item Numero di fibre reclutate.
			\end{itemize}
		\end{multicols}

			\paragraph{Fibra ipertrofizzata}
			Una fibra ipertrofizzata presenta pi\`u sarcomeri in parallelo, aumenta la forza generata ma non cambia la velocit\`a di accorciamento.

			\paragraph{Fibra allungata}
			Una fibra allungata ha pi\`u sarcomeri in serie, non cambia il rapporto generato ma aumenta la velocit\`a della contrazione.


	\subsection{Muscolo cardiaco}
	In una cellula muscolare cardiaca permangono le striature dovute ai sarcomeri regolari, si trovano mitocondri pi\`u numerosi e un disco intercalante.
	Il muscolo non \`e un sincizio, ma composto da cellule mononucleate, unite da desmosomi che comunicano tramite gap-junctions.
	Nel muscolo cardiaco queste giunzioni si dicono dischi intercalari e permettono un maggiore contatto tra le cellule.
	Le prtoeine contrattili sono allineate in modo regolare.
	La contrazione non \`e volontaria e omogenea.

		\subsubsection{Fibre miocardiche}
		Le fibre miocardiche non sono innervate direttamente ma si notano delle cellule pacemaker.
		Queste determinano il ritmo della contrazione e possono eccitare in maniera cadenzata grazie a particolari canali.
		Le fibre miocardiche si eccitano in relazione a potenziali d'azione che si propagano tra di loro grazie a sinapsi elettriche.

			\paragraph{Comunicazione}
			\begin{multicols}{2}
				\begin{itemize}
					\item Quando una cellula eccitata \`e in fase di pontenziale d'azione tutte le altre si eccitano simultaneamente.
					\item Si ottiene cos\`i una contrazione simultanea e ordinata che permette il corretto pompaggio del sangue.
					\item Il periodo di refrattariet\`a dura quanto la contrazione, perci\`o il cuore non pu\`o andare incontro a contrazioni di tipo tetanico.
				\end{itemize}
			\end{multicols}

		\subsubsection{Gap junctions}
		Le gap junctions permettono la trasmissione simultanea dello stimolo alle cellule, permettendo una contrazione simultanea del muscolo.
		
			\paragraph{Ione sodio}
			Nel momento in cui entra lo ione sodio questo importa cariche positive, una grande quantit\`a modifica il potenziale di membrana, che passa a valori positive.

			\paragraph{Meccanismo tutto o nulla}
			La frequenza determina l'informazione: maggiore frequenza maggiore informazione.
			La propagazione avviene nel corpo cellulare attraverso l'assone fino alle terminazioni.
			I canali si aprono e chiudono e una volta aperti, l'onda di depolarizzazione si propaga causando l'apertura dei canali successivi.
			I canali precedenti si trovano in uno stato di refrattariet\`a assoluta e non possono essere apreti.
			La modulazione della frequenza \`e pertanto importante per la modulazione del messaggio.
			Il messaggio \`e di tipo polarizzato e avviene da una direzione verso la periferia.

	\subsection{Muscolo liscio}
	I muscoli lisci subiscono una contrazione non volontaria.
	Le sue cellule sono fusiformi e mononucelate.
	Non presentano una striatura trasversale.
	Il controllo della contrazione \`e indipendente dalla volont\`a e possiedono meno mitocondri.
	La mescolatura liscia presenta una velocit\`a di contrazione inferiore e si trova nei vasi, pelle e visceri.
	
		\subsubsection{Intestino}
		Nell'intestino la muscolatura liscia determina una contrazione seriale di porzioni dell'intestino per favorire il passaggio del cibo.

		\subsubsection{Vasi sanguigni}
		Nelle arterie la contrazione e decontrazione determina il diametro del vaso e il flusso del sangue.
		Le vene ne sono prive e pi\`u soggette a problemi di vascoralit\`a.

		\subsubsection{Contrazione}
		La cellula di un muscolo liscio pu\`o solo contrarsi in parte.
		L'idrolisi di \emph{ATP} \`e pi\`u lenta e genera contrazioni pi\`u lente, prolungate ed efficienti.

		\subsubsection{Cellule coinvolte}

			\paragraph{Smooth muscle cells}
			Le smooth muscle cells \emph{SMC} sono unit\`a singole che si muovono in sincrono.
			L'attivit\`a \`e miogenica e spontanea.
			Sono innervate dal \emph{ANS}: sistema nervoso vegetativo.
			Sono attivate da impulsi nervosi del \emph{ANS}.
			Presentano uno stato di tensione costante che pu\`o essere modulato da ormoni e fattori di crescita.

	\subsection{Sviluppo della giunzione muscolare}
	La giunzione muscolare forma connessioni specializzate tra motoneuroni e cellule muscolari scheletriche.
	Il sincizio presenta diversi nuclei che trascrivono in maniera diversa a seconda della distanza della giunzione neuromuscolare.
	Minore la distanza maggiore la produzione di RNA recettori per l'acetilcolina.

		\subsubsection{Localizzazione degli RNA del recettore}
		Gli RNA vengono trasportati e tradotti in corrispondenza della placca neuromuscolare.
		Alla formazione del sincizio il motoneurone non \`e ancora stato attaccato e i recettori sono sparsi.
		Durante l'innervamento l'agrina porta un segnale che d\`a origine a una serie di cascate di eventi della trasduzione del segnale che porta i recettori ad avvicinarsi e clusterizzare in prossimit\`a della giunzione neuromuscolare.
		La chinasi \emph{MuSK}  viene attivata permettendo questo processo.

\section{Esempi}

	\subsection{\emph{GLI3}}
	La proteina \emph{GLI3} \`e una proteina zinc finger e la sua mutazione causa la sindrome della cefalopolisindattia di Greig.
	\`E un attivatore o un repressore di bersagli a valle della via di segnalazione mediata da \emph{Shh} Sonic hedgehog, mutante di Drosophila che presenta spine durante lo sviluppo.
	La sua presenza legata a un recettore causa una serie di segnalazioni che confluiscono sul \emph{GLI} o glioblastoma.
	A seguito dell'interazione questa pu\`o essere attivata o disattivata: se \emph{Shh} \`e presente attiva \emph{GLI} che promuove la trascrizione nel nucleo.
	Se non presente alcuni membri del complesso vengono tagliati producendo una forma tronca che entra nel nucleo e inibisce la trascrizione.

