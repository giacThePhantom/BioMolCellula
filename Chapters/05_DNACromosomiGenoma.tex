\chapter{DNA, cromosomi e genomi}
La vita dipende dall'abilit\`a della cellula di conservare, recuperare e tradurre le istruzioni genetiche richieste per creare e mantenere un organismo vivente. Queste informazioni
ereditarie sono passate da una cellula a una sua figlia durante la divisione cellulare e da una generazione di organismi all'altra attraverso le cellule riproduttive, sono salvate come
geni. Le informazioni genetiche consistono principalmente su istruzioni per la costruzione delle proteine, macromolecole versatili che svolgono la maggior parte delle funzioni della 
cellula. Le informazioni genetiche sono trasportate su cromosomi, strutture a filo nel nucleo che diventano visibili durante la divisione e composti di DNA (acido desossiribonucleico) e 
proteine in egual misura. La determinazione della struttura a doppia elica del DNA ha risolto il problema di come le informazioni nel DNA sono replicate e come una molecola di DNA 
utilizza la sequenza dei suoi monomeri per produrre le proteine. 
\section{La struttura e la funzione del DNA}
Negli anni $50$ la struttura del DNA \`e stata determinata grazie a diffrazione a raggi X, che indicarono la composizione a due fili avvolti in un'elica fornendo un grande indizio a
Watson-Crick e al loro modello.
\subsection{Una molecola di DNA consiste di due catene di nucleotidi complementari}
Una molecola di acido desossiribonucleico casiste di due catene polinucleotide lunghe note come strand e composte da quattro tipi di subunit\`a. Le catene sono anti-parallele tra di loro
e tra le basi dei nucleotidi si formano legami a idrogeno che le tengono unite. I nucleotidi sono composti da zuccheri a 5 atomi di carbonio a cui sono attaccati dei gruppi fosfati e 
una base contenente dell'azoto. Nel caso dei nucleotidi del DNA lo zucchero \`e il desossiribosio attaccato a un singolo gruppo fosfato (da cui il nome) e la base pu\`o essere adenina
(A), citosina (C), guanina (G) o timina (T). I nucleotidi sono legati covalentemente in una catena attraverso gli zuccheri e i fosfati che formano un backbone di legami 
zucchero-fosfato-zucchero-fosfato alternati. I legami dei nucleotidi danno al filo del DNA una polarit\`a chimica. Il gruppo $5'$ fosfato si lega con il gruppo $3'$ idrossile di un
altro monomero e pertanto tutte le subunit\`a hanno lo stesso orientamento. Inoltre le due terminazioni dello strand sono facilmente distinguibili e ci si riferisce ad esse come alla
terminazione $3'$ e $5'$, indicando la polarit\`a. Grazie alla direzionalit\`a e linearit\`a del filo di DNA pu\`o essere letto facilmente. La doppia elica si genera dalla struttura 
chimica delle catene polinucleotidiche in quanto sono mantenute insieme da legami a idrogeno tra le basi sui fili diversi tutte le basi si trovano all'interno della doppia elica con i 
backbones verso l'esterno. In ogni caso una base a due anelli (una purina) \`e sempre legata con una ad anello singolo (una pirimidina): A si accoppia con T e G con C. Questo 
accoppiamento complementare delle basi permette a coppie di basi di essere ammassate nell'ordinamento pi\`u energeticamente favorevole all'interno della doppia elica in quanto ogni
coppia possiede la stessa lunghezza mantenendo i backbones a distanza costante. Per massimizzare l'efficienza spaziale i backbones si avvolgono formando una doppia elica destra con
un giro completo ogni $10$ paia. I membri di ogni base possono adattarsi insieme solo se i fili sono anti-paralleli. Ogni strand della molecola contiene una sequenza complementare a
quella dell'altro strand. 
\subsection{La struttura del DNA mette a disposizione un meccanismo per l'ereditariet\`a}
La struttura a polimero lineare composto da quattro monomeri permette al DNA di trasportare informazioni in forma chimica, mentre la natura a doppio filamento permette la sua 
duplicazione. Essendo ogni filamento complementare all'altro pu\`o agire come stampo per la sintesi di un nuovo filamento complementare. Nominando i filamenti $S$ e $S'$, $S$ agisce
da stampo per la formazione di $S'$ e $S'$ per $S$, permettendo l'accurata copia dell'informazione nel DNA. Questa capacit\`a permette alla cellula di replicare il proprio genoma
e passarlo ai discendenti. Il DNA permette la codifica di proteine (le cui funzioni sono determinate dalla forma e pertanto in ultimo dalla sequenza di amminoacidi) attraverso una
corrispondenza esatta tra i quattro nucleotidi e i venti amminoacidi. L'intero gruppo di informazioni mantenute dal DNA di un organismo \`e detto genoma e specifica tutte le molecole
di RNA e proteine che l'organismo sintetizzer\`a. 
\subsection{Negli eucarioti il DNA \`e racchiuso nel nucleo cellulare}
Circa tutti il DNA della cellula si trova isolato nel nucleo, che occupa circa il $10\%$ del volume totale, delimitato da un involucro nucleare formato da due bistrati lipidi 
concentrici. Queste membrane sono puntuate a intervalli da pori nucleari in cui le molecole si trasferiscono dal nucleo al citosol. Tale involucro \`e connesso a un sistema di 
membrane intracellulari detto reticolo endoplasmatico che si estende nel citoplasma ed \`e meccanicamente supportato da una rete di filamenti intermedi detti lamine nucleari. L'involucro
permette alle proteine che agiscono sul DNA di essere concentrate dove necessario e mantiene gli enzimi nucleari e citosolici separati. 
\section{DNA cromosomico e il suo confezionamento nella fibra cromatina}
La funzione pi\`u importante del DNA \`e quella di trasportare i geni, informazioni che specificano tutte le molecole di RNA e le proteine che compongono l'organismo. Il DNA nucleare 
degli eucarioti \`e diviso in cromosomi. 
\subsection{Il DNA eucariotico \`e confezionato in un insieme di cromosomi}
Ogni cromosoma consiste di una singola molecola lineare di DNA estremamente lunga associata con proteine che piegano e impacchettano il filamento in una struttura compatta. Oltre a 
queste proteine si trovano altre proteine e molecole di RNA che si occupano dell'espressione genica e della riparazione e replicazione del DNA. Il complesso formato dal DNA e le proteine
strettamente legate prende il nome di cromatina. Nei batteri d'altra parte il DNA si trova in forma circolare. Con l'eccezione dei gameti, alcune cellule specializzate che non possono
moltiplicare o a cui manca il DNA e altre che replicano il loro DNA senza completare la divisione cellulare ogni nucleo cellulare umano contiene due coppie di ogni cromosoma, uno 
ereditato dal padre e uno dalla madre. I cromosomi di tale coppia sono detti omologhi, l'unica coppia non omologa nei maschi \`e quella dei cromosomi Y (dal padre) e X (dalla madre).
Ogni uomo contiene 46 cromosomi: 22 paia comuni tra i sessi e una coppia detta dei cromosomi sessuali che possono essere distinti attraverso colorazione basata sull'ibridazione del DNA
in cui un piccolo filamento marcato con un colore fluorescente agisce da sonda che si lega alla sequenza complementare illuminando il cromosoma obiettivo. L'insieme dei 46 cromosomi
durante la mitosi \`e detto cariotipo.
\subsection{I cromosomi contengono lunghe stringhe di geni}
I cromosomi trasportano geni, definiti come segmenti di DNA che contengono le istruzioni per sintetizzare una particolare proteina (o una famiglia) o molecole di RNA significativamente 
funzionali. Oltre ai geni il genoma di organismi multicellulari e di molti altri eucarioti contiene grandi segmenti sparsi la cui funzione non \`e compresa. Alcuni di questi segmenti
sono fondamentali per la regolazione dell'espressione genica. Questi segmenti generano grandi diversit\`a nella dimensione genomica quando si comparano le specie e creano sequenze molto
diverse per organismi strettamente imparentati, anche se contengono gli stessi geni. La divisione del genoma in cromosomi \`e unico per la specie, con numero e dimensione dei cromosomi
diversa. 
\subsection{La sequenza nucleotidica del genoma umano mostra come i geni sono ordinati}
Si nota come solo una piccola percentuale del genoma umano codifica le proteine e circa met\`a \`e composta da pezzi mobili che si sono inseriti nei cromosomi durante l'evoluzione. La
dimensione media di un gene \`e circa $27000$ coppie di basi. Siccome solo circa $1300$ paia sono richieste per codificare una proteina di dimensione media ($430$ amminoacidi negli 
umani) il resto della sequenza consiste di lunghezze di DNA non codificante che interrompe quello codificante. Le sequenze codificanti sono dette esoni, mentre quelle non codificanti
introni. Il DNA umano consiste di una lunga stringa di esoni e introni (che sono la maggioranza) che si alternano. Oltre a introni e esoni ogni gene \`e associato con sequenze di DNA
regolatorio responsabile che il gene sia espresso al momento e a livello giusto. Negli esseri umani occupano decine di migliaia di coppie di basi. Oltre ai geni codificanti le proteine
ne esistono altri che codificano molecole di RNA con funzione propria.
\subsection{Ogni molecola di DNA che forma un cromosoma lineare deve contenere un centromero, due telomeri e un'origine di replicazione}
Per formare un cromosoma funzionale una molecola di DNA deve poter essere abile di replicarsi e i replicati devono essere separati e partizionati in cellule figlie. Questo processo 
accade attraverso una serie ordinata di passaggi detti ciclo cellulare che fornisce la separazione temporale tra la duplicazione dei cromosomi e la segregazione in due cellule figlie. 
Durante una lunga interfase i geni sono espressi e i cromosomi sono replicati rimanendo in un paio di cromatidi sorelle. In questo momento i cromosomi sono estesi in modo che la maggior
parte della cromatina esista come lunghi fili e i cromosomi individuali non siano distinguibili. In un momento successivo si condensano in modo da separare le cromatidi sorelle e 
distribuiti alle cellule figlie. I cromosomi altamente condensati sono dette cromosomi mitotici. Ogni cromosoma opera come un'unit\`a strutturale indipendente. Le operazioni di 
replicazione sono controllate da tre tipi di sequenze nucleotidiche nel DNA, ognuna delle quali si lega a proteine specifiche che guidano la macchina che replica e segrega i cromosomi.
Un tipo di sequenza nucleotidica agisce come origine della replicazione del DNA, ovvero il luogo dove la duplicazione inizia, ne esistono diverse per assicurarsi che l'intero cromosoma
venga replicato rapidamente. Dopo la replicazione del DNA le due cromatidi sorelle rimangono attaccate a una fine mentre vengono condensate producendo cromosomi mitotici. Un'ulteriore
sequenza specializzata detta centromero permette a una copia di ogni cromosoma duplicato e condensato di essere portata nella cellula figlia quando si divide. Un complesso proteico detto
cinetocoro si forma al centromero e attacca i cromosomi al mandrino mitotico, permettendo la loro separazione. La terza sequenza di DNA specializzata forma i telomeri, le terminazioni 
dei cromosomi. Permettono una replicazione efficiente e formano strutture che proteggono la fine del cromosoma da essere confusa con una molecola di DNA in necessit\`a di riparazione.
\subsection{Le molecole di DNA sono altamente conservate nei cromosomi}
Tutti gli organismi eucariotici possiedono metodi per impacchettare il DNA in cromosomi. Questa compressione viene svolta da proteine che avvolgono e piegano il DNA in livelli sempre 
pi\`u alti di organizzazione. Nonostante molto meno condensati rispetto a cromosomi mitotici sono comunque compressi. La struttura dei cromosomi \`e dinamica: regioni specifiche del
interfase dei cromosomi decondensa per permettere l'accesso di sequenze specifiche del DNA per l'espressione genica, riparazione e replicazione ricomdensandosi dopo. La condensazione 
avviene pertanto in modo da permettere accesso rapido e localizzato al DNA su richiesta.
\subsection{I nucleosomi sono unit\`a base della struttura del cromosoma eucariote}
Le proteine che si legano al DNA per formare i cromosomi sono divise in istoni e proteine non-istoniche che contribuiscono con la stessa massa ad un cromosoma. Il complesso di entrambe 
le classi di proteine \`e noto come cromatina. Gli istoni sono responsabili per il primo e basico livello di condensazione cromosomica detto nucleosoma, un complesso DNA-proteina. Si nota
come il nucleosoma sia formato da una stringa di DNA con parti di esso avvolte intorno a istone. Ogni nucleo istonico del nucleosoma consiste di un complesso di $8$ proteine istoniche
e DNA a doppio filamento lungo $147$ coppie di basi. Questo ottamero forma un nucleo proteico lungo il quale si lega. La regione di DNA linker che separa ogni nucleo di nucleosoma
pu\`o essere lunga fino a $80$ basi. In media i nucleosomi si ripetono ogni circa $200$ coppie di basi. 
\subsection{La struttura del nucleo del nucleosoma rivela come il DNA \`e condensato}
La forma a disco del nucleo istonico in cui il DNA \`e avvolto in una bobina sinistra di $1.7$ giri \`e composta da quattro istoni, proteine che condividono una piega istonica formata da
tre $\alpha$-eliche connesse da due anelli. Durante l'assemblaggio del nucleosoma le pieghe istoniche si legano prima tra di loro formando dimeri, poi tetrameri e infine ottameri. 
L'interfaccia tra il DNA e l'istone forma $142$ legami a idrogeno in ogni nucleosoma. Circa met\`a di questi legami derivano dal backbone amminoacido e quello del DNA, le interazioni
idrofobiche e saline mantengono il DNA e la proteina insieme nel nucleosoma. Un quinto del nucleo istonico \`e formato da lisina o da arginina (amminoacidi basici) e le loro cariche 
positive possono neutralizzare il backbone del DNA. Queste interazioni spiegano perch\`e il legame con l'istone \`e indipendente dalla sequenza delle basi. Oltre alla piega istonica
ogni nucleo istonico possiede una coda \ce{N}-terminale che si estende all'esterno del nucleo DNA-istone che subisce cambi covalenti che controllano aspetti della struttura cromatina
e della funzione. Gli istoni sono proteine altamente conservate. 
\subsection{I nucleosomi hanno strutture dinamiche e sono soggetti a cambi catalizzati da complessi rimodellanti a cromatina ATP-dipendenti}
Il DNA in un nucleosoma isolato si svolge quattro volte al secondo, rimanendo esposto per intervalli di tempo, rendendolo disponibile per il legame con altre proteine. Dalla
cromatina nella cellula \`e richiesto un ulteriore allentamento \`e necessario in quanto le cellule contengono una variet\`a di complessi di rimodellamento a cromatina ATP-dipendenti.
Questi complessi includono una subunit\`a che idrolizza ATP che lega il nucleo proteico del nucleosoma e il DNA legato a esso. Utilizzando l'energia fornita dall'ATP muove il DNA
cambiando la struttura del nucleosoma temporaneamente allentando il legame tra il DNA e il nucleo istonico. Attraverso ripetuti cicli di idrolisi dell'ATP il complesso pu\`o catalizzare
uno scorrimento del nucleosoma in modo da riposizionarlo esponendo specifiche regioni di DNA rendendole disponibili ad altre proteine. Inoltre cooperando con altre altre proteine che si
legano agli istoni e servono come accompagnatrici di istoni questi complessi sono capaci di rimuovere parte o tutto il nucleo del nucleosoma. Un tipico nucleosoma \`e sostituito ogni
ora o due all'interno della cellula. Questi complessi esistono in molte varianti con ruoli diversi. La maggior parte sono proteine con $10$ o pi\`u subunit\`a altri creano specifici
cambi sugli istoni. Quando i geni sono attivati o disattivati questi complessi vengono portati a regioni specifiche del DNA dove agiscono sulla struttura cromatinica. La maggior 
influenza sul posizionamento del nucleosoma sembra essere altre proteine strettamente legate al DNA, che possono favorire la presenza di un nucleosoma adiacente a loro o forzare il 
suo allontanamento. 
\subsection{I nucleosomi sono tipicamente condensati in una fibra cromatinica}
I nucleosomi sono tipicamente condensati uno sull'altro, generando vettori in cui il DNA \`e altamente denso, pertanto la cromatina forma una fibra. Questa vicinanza \`e causata da 
legami tra nucleosomi che coinvolgono la coda degli istoni e un istone addizionale che svolge la funzione di linker (H1) che si lega a ogni nucleosoma e cambia il percorso del DNA 
quando esce dal nucleosoma. 
\section{La struttura e la funzione della cromatina}
I meccanismi che creano le strutture cromatiniche in diverse regioni della cellula hanno diverse funzioni e possono essere ereditate attraverso ereditariet\`a epigenetica, ovvero 
ereditariet\`a superimposta sull'ereditariet\`a genetica del DNA. 
\subsection{L'eterocromatina \`e altamente organizzata e limita l'espressione genica}
Esistono due tipi di cromatina nell'interfase dei nuclei di molte cellule eucariotiche: una forma altamente condensata detta eterocromatina e una meno condensata detta eucromatina. 
L'eterocromatina \`e altamente concentrata in certe regioni specializzate: i centromeri e nei telomeri e presente in luoghi che variano in base allo stato fisiologico della cellula. 
Contiene tipicamente pochi geni e quando parti eucromatiche vengono convertite in eterocromatiche i rispettivi geni vengono disattivati. Descrive domini di cromatina compatti che sono
resistenti all'espressione genica.
\subsection{Lo stato eterocromatico \`e auto-propagante}
Attraverso la rottura e la riunione dei cromosomi una parte eucromatica pu\`o essere trasformata in una eterocromatina in un effetto detto di posizione. Riflette l'allargamento dello 
stato eterocromatico ed \`e il modo in cui l'eterocromatina \`e creata e mantenuta. Quando una condizione eterocromatica si stabilisce in una zona viene ereditata dalle cellule figlie
nella variegazione dell'effetto di posizione. Si nota pertanto come l'eterocromatina genera s\`e stessa in un feedback positivo che si espande spazialmente e temporalmente. Esistono 
geni che aumentano o diminuiscono questo processo che codificano proteine cromosomali non istoni che interagiscono con gli istoni coinvolti nella modifica o mantenimento della struttura
cromatinica.
\subsection{I nuclei istonici sono modificati covalentemente a molti siti diversi}
Le catene laterali degli amminoacidi dei quattro istoni del nucleosoma sono soggette a molti cambi covalenti, come l'acetilazione della lisina, la mono-, di- e triametilazione delle 
lisine e la fosforilazione delle serine. Un gran numero di queste modifiche avvengono sulle code istoniche \ce{N-} terminali che protrudono dal nucleosoma. Sono tutte reversibili e 
catalizzate da enzimi altamente specifici. Ogni enzima \`e reclutato su siti specifici della cromatina in tempi determinati in base alle proteine regolatorie della trascrizione o 
fattori di trascrizione che riconoscono e si legano a specifiche sequenze di DNA. Sono prodotte in tempi e luoghi diversi, determinando dove e quando tali enzimi agiscono. Il DNA 
determina pertanto come gli istoni sono modificati, ma in alcuni casi le modifiche covalenti possono persistere creando una memoria della storia di sviluppo della cellula che pu\`o 
essere trasmessa ereditariamente. Differenti gruppi di nucleosomi presentano diverse modifiche, che vengono controllate e hanno importanti conseguenze. L'acetilazione delle lisine 
allenta la struttura cromatinica in quanto il gruppo acetile rimuove la carica positiva della lisina, riducendone l'affinit\`a con le code dei nucleosomi adiacenti, ma gli effetti pi\`u 
importanti sono generati dall'abilit\`a di reclutare altre proteine in queste lunghezze di cromatina alterata. La trimetilazione di una specifica lisina sulla coda H3 attrae la proteina
HP1 eterocromatina-specifica contribuendo alla stabilizzazione ed espansione della cromatina. Le proteine reclutate agiscono con gli istoni modificati determinando dove e quando i geni
saranno espressi. Si nota pertanto come la struttura di ogni dominio cromatinico governa la lettura delle informazioni genetiche e infine struttura e funzioni della cellula.
\subsection{La cromatina acquisisce variet\`a addizionale attraverso inserzioni specifiche al sito di un insieme di varianti di istoni}
Nella cellula sono contenuti alcune varianti di istoni che si possono assemblare in nucleosomi. Gli istoni principali sono sintetizzati primariamente durante la fase S del ciclo 
cellulare e si assemblano in nucleosomi sulle eliche di DNA figlie dietro la forcella di replicazione, mentre le varianti sono sintetizzate nell'interfase e sono inseriti in cromatine
gi\`a formate con un processo di cambio istonico catalizzato dal complesso di rimodellazione della cromatina ATP-dipendente che contiene subunit\`a che causano il suo legame a siti 
specifici della cromatina e a accompagnatori di istoni che trasportano una particolare variante, inserendoli pertanto in luoghi con alta specificit\`a.
\subsection{Le modifiche covalenti e le varianti istoniche agiscono insieme per controllare le funzioni dei cromosomi}
Le modifiche degli istoni avvengono in insiemi ordinati. Alcune combinazioni determinano come e dove il DNA condensato possa essere acceduto e manipolato, creando un insieme di marcature
che determinano se la lunghezza di cromatina \`e stata appena replicata, o \`e danneggiata o come e dove l'espressione genica deve accadere. Molte proteine regolatorie contengono piccoli
domini che si legano a marcature specifiche come la lisina trimetilata sull'istone H3. Questi domini sono legati insieme in una grande proteina o complesso che riconosce una combinazione
specifica di modifiche all'istone. Si crea pertanto un complesso lettore che permette a particolari combinazioni di marcature di attrarre proteine per eseguire le funzioni biologiche
nel momento corretto. Queste marcature sono dinamiche e si creano sulle code istoniche che si trovano sull'esterno del nucleo del nucleosoma.
\subsection{Un complesso di proteine lettrici e scrittrici possono espandere modifiche alla cromatina lungo un cromosoma}
Il fenomeno dell'effetto di posizione richiede che forme modificate di cromatina abbiano l'abilit\`a di espandersi per distanze sostanziale lungo il cromosoma. Si nota come gli enzimi 
che aggiungono o rimuovono modifiche agli istoni fanno parte di complessi che possono essere trasportati dai regolatori di trascrizione su lunghezze specifiche di DNA, ma dopo che
l'enzima marca i nucleosomi pu\`o avvenire una reazione a catena in cui lavora insieme a una proteina di lettura nello stesso complesso che riconosce la marcatura e si lega a tale 
nucleosoma, attivando l'enzima di scrittura e posizionandolo su un nucleosoma vicino. Attraverso i cicli di lettura-scrittura la proteina pu\`o trasportare l'enzima di scrittura lungo
il DNA espandendo la marcatura lungo il cromosoma. Un processo simile avviene per rimuovere le modifiche agli istoni. 
\subsection{Barriere nella sequenza di DNA bloccano l'espansione dei complessi di lettura-scrittura separando domini di cromatina vicini}
Nella cellula esistono sequenze di DNA che marcano i confini di domini di cromatina e li separano. Questa sequenza di barriera contiene un insieme di siti di legame per gli enzimi
istoni acetilasi. Siccome la lisina acetilata \`e incompatibile con la metilazione richiesta per l'espansione dell'eterocromatina si blocca la sua espansione. Esistono comunque molte
altre modifiche alla cromatina che raggiungono questo scopo.
\subsection{La cromatina nei centromeri rivela come le varianti degli istoni possono creare strutture speciali}
Le varianti di istoni che trasportano nucleosomi possono produrre marcature nella cromatina persistenti. Un esempio di questo \`e la formazione e ereditariet\`a di una struttura di 
cromatina specializzata al centromero, la regione necessaria per l'attacco al mandrino mitotico e alla segregazione ordinata delle copie del genoma durante la divisione cellula. Ogni
centromero \`e contenuto da una cromatina centromerica che persiste durante l'interfase che cotniene una variante dell'istone H3 nota come CENP-A (centromere protein-A) e altre 
proteine che condensano il nucelosoma in ordinamenti particolarmente densi e formano il cinetocoro, la struttura richiesta per l'attacco al mandrino mitotico. I centromeri consistono 
di sequenze di DNA corte e ripetute dette DNA alfa satellite. La presenza di queste sequenze in altre parti di DNA suggerisce come da sole non siano sufficienti per la formazione 
del centromero. Altri centromeri sono stati osservati su cromosomi frammentati e inizialmente eucromatici. I centromeri sono pertanto definiti da un insieme di proteine, piuttosto che
da una sequenza di DNA specifica. 
\subsection{Alcune strutture cromatiniche possono essere ereditate direttamente}
I cambi nell'attivit\`a del centromero devono essere trasmessi alle generazioni successive. Si nota come la formazione di un centromero de novo richiede un inizio evento di 
inseminazione con strutture specifiche al DNA che contengono nucleosomi con CENP-A, che avviene pi\`u rapidamente lungo il DNA alfa satellite. Essendo che i tetrameri di H3-H4 sono
ereditati si nota come un nucleosoma contenente CENP-A \`e stato assemblato su una lunghezza di DNA \`e banale come un nuovo centromero possa essere generato nello stesso luogo, 
assumendo che la presenza di un istone CENP-A recluti selettivamente pi\`u istoni CENP-A nei suoi vicini. Si nota inoltre come la generazione del centromero siaun processo altamente 
cooperativo.
\subsection{L'attivazione e la repressione di strutture di cromatina pu\`o essere ereditato epigeneticamente}
L'ereditariet\`a epigenetica \`e centrale nella creazione di organismi multicellulari in quanto permettono la creazione di tessuti diversi. Si nota come le strutture specifiche della
cromatina tendono a persistere e a essere trasmesse nei cicli di divisione.
\subsection{Le strutture di cromatina sono importanti per le funzioni del cromosoma eucariotico}
La condensazione del DNA in nucleosomi \`e fondamentale per l'evoluzione di organismi multicellulari in quanto cellule si devono specializzare cambiando l'accessibilit\`a e l'attivit\`a
di centinaia di geni, processo che richiede sulla memoria cellulare che si trova in parte nella struttura cromatinica che crea strutture che silenziano geni temporaneamente o in 
maniera persistente.
\section{La struttura globale dei cromosomi}
Come una fibra a $30nm$ il nucleo avrebbe una lunghezza di $1mm$ e sarebbe troppo grande per essere contenuto nel nucleo. Si rende pertanto necessario un secondo livello di piegamento
che coinvolge il piegamento della cromatina in una serie di anelli e bobine in maniera dinamica.
\subsection{I cromosomi sono piegati in grandi anelli di cromatina}
I cromosomi, accoppiati in preparazione per la meiosi si presentano in una serie di grandi anelli di cromatina che si emanano da un asse cromosomico lineare. In queste circostanze un
anello contiene sempre la stessa sequenza di DNA che rimane estesa nello steso modo. Questi cromosomi producono grandi quantit\`a di RNA e la maggior parte dei geni negli anelli sono
espressi. La maggior parte del DNA si trova comunque condensata nell'asse dove i geni non sono espressi. 
\subsection{Cromosomi politene sono utili per visualizzare le strutture di cromatina}
Alcune tipi di cellule diventano anormalmente grandi attraverso multipli cicli di sintesi del DNA senza divisione cellulare. Tali cellule sono dette poliploidi e tutte le copie di ogni 
cromosoma sono allineate creando giganti cromosomi politene organizzati in bande e interbande, dove il DNA \`e pi\`u concentrato nelle prime. Queste strutture permettono di esaminare 
l'organizzazione della cromatina su larga scala. Si nota come insiemi specifici di proteine non istoni si assemblano sui nucleosomi influenzando le funzioni biologiche. Il reclutamento
di queste proteine pu\`o avvenire su grandi lunghezze di DNA, generando strutture cromatiniche simili a grandi tratti del genoma, separate dai domini vicini da proteine di barriera. Il
cromosoma interfase pu\`o essere pertanto considerato come un mosaico di strutture cromatiniche contenenti modifiche del nucleosoma associate con un particolare insieme di proteine non
istoni. 
\subsection{Esistono multiple forme di cromatina}
L'analisi della locazione degli istoni e delle proteine non istone nella cromatina pu\`o essere mappata lungo tutta la sequenza di DNA di un organismo. Tre tipi di cromatina repressiva
dominano nell'organismo della Drosophila e due tipi di cromatina su geni attivamente trascritti. Ognuno di questi tipi \`e associato con un complesso diverso di proteine non istone. 
L'eterocromatina classica contiene pi\`u di $6$ proteine come la proteina eterocromatina $1$ (HP1), mentre le forme a polycomb contengono un simile numero di proteine PcG. Oltre ai 
cinque principali sono presenti forme minori ognuna delle quali diversamente regolata e con ruoli distinti. L'insieme delle proteine legate ad un luogo varia in base al tipo della 
cellula e al suo stato di sviluppo.
\subsection{Gli anelli di cromatina decondensano quando i geni al loro interno sono espressi}
Quando la cellula si sviluppa distinti puffs cromosomici si formano e i vecchi recedono in cromosomi politene quando nuovi geni sono espressi e i vecchi sono disattivati. Dall'ispezione
di un puf si nota come la maggior parte si generino dalla decondensazione di una singola banda cromosomica. Quando i geni nell'anello non sono espressi questo assume una struttura 
spessa, estendendosi durante la loro espressione. Si nota come la cromatina a lato del loop decondensato appaia pi\`u compatta in quanto un anello costituisce un dominio funzionale
distinto della struttura cromatinica. Anelli di cromatina altamente condensata si espandono quando il gene \`e espresso. Ognuno dei $46$ cromosomi dell'interfase tendono a occupare il 
proprio territorio discreto nel nucleo e non sono impigliati tra di loro estensivamente. Le regioni eterocromatiniche sono associate con la lamina nucleare. La maggior parte dei 
cromosomi umani sono piegati in una conformazione dette globulo frattale: un ordinamento senza nodi che facilita un impacchettamento denso che preserva l'abilit\`a della cromatina di
spiegarsi e ripiegarsi.
\subsection{La cromatina pu\`o muoversi a siti specifici nel nucleo per alterare l'espressione dei geni}
La posizione dei geni nel nucleo cambia quando diventa altamente espresso, pertanto una regione che diventa attivamente trascritta si espande fuori dal territorio del suo cromosoma in
un anello esteso in quanto l'assemblaggio delle pi\`u di $100$ proteine richieste per iniziare l'espressione genica viene facilitato in regioni del nucleo ricche di queste. Si nota come
il nucleo \`e eterogeneo con regioni con funzioni diverse in cui porzioni del cromosoma si possono muovere quando sono soggette a processi biochimici.
\subsection{Reti di macromolecole formano un insieme di ambienti biochimici distinti nel nucleo}
Uno dei sottocompartimenti pi\`u grossi del nucleo \`e il nucleolo, una struttura dove si forma la subunit\`a del ribosoma e altre reazioni specializzate. Consiste di una rete di RNA e 
proteine concentrate intorno a geni di RNA ribosomiale che sono attivamente trascritti, presenti in multiple copie nel genoma che sono raggruppate in un singolo nucleolo nonostante si 
trovino su cromosomi diversi. Sono presenti anche diversi organelli come corpi di Cajal e granuli intercromatinici, composti da proteine e RNA che si legano creando reti altamente
permeabili a altre proteine e molecole di RNA nel nucleoplasma circostante. Queste strutture creano ambienti biochimici distinti immobilizzando gruppi di macromolecole distinte 
permettendo a altre molecole in essi di essere efficientemente processate attraverso cammini di reazione complessi impartendo molti dei vantaggi cinetici della compartimentalizzazione 
pur essendo incapaci di di concentrare o escludere piccole molecole in quanto non possiedono il bistrato lipidico. Questi sottocompartimenti si formano solo quando necessario creando 
alte concentrazioni locali di enzimi e molecole di RNA necessarie per un processo particolare. In maniera analoga quando il DNA \`e danneggiato da radiazioni l'insieme di enzimi 
necessario alla riparazione del DNA si unisce in foci discrete nel nucleo creando fabbriche di riparazione che sintetizzano DNA o RNA. Queste entit\`a si creano attraverso 
lunghe catene polipeptidiche o molecole di RNA non codificanti intervallati con siti di legame specifici che concentrano multiple proteine e altre molecole necessarie.  Esiste anche una
struttura analoga al citoscheletro in cui i cromosomi e gli altri componenti del nucleo sono organizzati detta matrice nucleare.
\subsection{I cromosomi mitotici sono altamente condensati}
I cromosomi di quasi tutte le cellule eucariotiche diventano visibili durante la mitosi, quando si arrotolano per formare una struttura altamente condensata con una lunghezza di un
decimo rispetto a un cromosoma interfase. Le due molecole di DNA prodotte durante la replicazione del DNA durante la replicazione sono piegate separatamente producendo due cromatidi 
sorelle unite al centromero, coperti da grandi quantit\`a di complessi RNA-proteine e altre molecole. Una volta che si toglie questa copertura ogni cromatide si organizza in anelli di 
cromatina che si estendono da un'impalcatura centrale. Le caratteristiche visibili del cromosoma mitotico sono nello stesso ordine ripetto alla molecola di DNA. La condensazione di 
cromosomi mitotici \`e il livello finale di impacchettamento dei cromosomi. Questo processo di compattazione \`e dinamico e altamente organizzato che permette lo sgarbugliamento delle
cromatidi sorelle in modo da permettere la loro separazione e protegge le molecole di DNA fragili dalla rottura quando sono tirate nelle cellule figlie. La condensazione comincia nella
fase $M$ ed \`e connessa con la progressione del ciclo cellulare. Durante questa fase si blocca l'espressione dei geni e sono fatte specifiche modifiche agli istoni in modo da 
riorganizzare la cromatina mentre si compatta. 
\section{Come si evolvono i genomi}
Geni omologhi (con uno stesso antenato) possono essere riconosciuti in lunghe distanze filogenetiche. Il riconoscimento di questa similitudine \`e utile per inferire la funzione di 
proteine e geni. La sequenza dei geni individuali \`e pi\`u conservata rispetto alla struttura genomica. 
\subsection{Il confronto tra i genomi rivela sequenze di DNA attraverso la loro evoluzione del throughput di evoluzione}
Le regioni di progenie che codificano le sequenze di amminoacidi delle proteine (esoni) si trovano in segmenti corti ($145$ paia di nucleotidi), dispersi in area in cui la sequenza 
nucleotidica ha poche conseguenze. Questo ordinamento rende difficile identificare tutti gli esoni e inizio e fine di un gene. Un approccio \`e cercare per sequenze di DNA con una 
funzione simile e sono pi\`u probabilmente conservate rispetto a quelle senza funzione. Tali regioni con sequenze simili sono dette regioni conservate. Oltre a rivelare le sequenze che
codificano esoni importanti e molecole di RNA contengono sequenze di DNA regolatorio con funzione sconosciuta. Le regioni non conservate riflettono il DNA la cui sequenza \`e molto 
meno probabilmente critica per la funzione. Si possono ottenere risultati pi\`u importanti includendo diverse specifiche. Circa il $5\%$ del genoma consiste di sequenze conservate 
multispecie, di cui solo un terzo codifica proteine, mentre il resto \`e regolatorio e cluster di siti di legami per proteine e per la produzione di molecole di RNA con altro scopo. 
\subsection{L'alterazione dei genomi \`e causata da fallimento del normale meccanismo per la copia e il mantenimento del DNA e da elementi del DNA trasponibili}
L'evoluzione dipende da incidenti e errori seguiti da sopravvivenza non casuale. La maggior parte dei cambi genetici sono il risultato di fallimenti nel meccanismo in cui i genomi
sono copiati e riparati quando danneggiati insieme al movimento di parti del DNA trasponibili. Il meccanismo di replicazione compie un errore in un nucleotide su mille ogni milione di
anni. Nonostante questo in una popolazione di $10^.000$ diploidi individuali in un milione di anni ogni possibile sostituzione nucleotidica sar\`a stata provata circa $20$ volte. Gli
errori nella replicazione, ricombinazione o riparazione del DNA possono portare a cambi locali nella sequenza dette mutazioni puntuali o a riordinamenti a larga scala come eliminazioni, 
duplicazioni, inversioni e traslocazioni. Oltre a questi fallimenti il genoma contiene elementi di DNA mobile che sono fonte di cambi genomici, questi trasposoni sono sequenze 
parassitiche che si diffondono nel genoma che colonizzano, distruggendo la funzione o alterando la regolazione dei geni. Questo processo ha modificato profondamente i genomi.
\subsection{La sequenza genomica di due specie differisce in proporzione al tempo in cui si sono evolute separatamente}
Le differenze tra i genomi delle specie si sono accumulate lungo $3$ miliardi di anni. Il processo di evoluzione genomica si pu\`o ricostruire tra confronti dettagliati dei genomi degli
organismi contemporanei. La struttura di base per questo lavoro \`e l'albero filogenetico, costruito secondo il confronto dei geni o sequenze proteiche. Per organismi distanti la 
conservazione delle sequenze \`e dovuta a una soluzione purificativa in cui vengono eliminate le mutazioni che interferiscono con le funzioni genetiche importanti. 
\subsection{L'albero filogenetico costruito dal confronto delle sequenze di DNA traccia le relazioni di tutti gli organismi}
Gli alberi filogenetici basati sulle sequenze molecolari possono essere confrontati con i record di fossili in modo da avere un'idea ancora pi\`u chiara del processo evolutivo in quanto
il secondo fornisce date assolute, nonostante tempi di divergenza precisi siano difficili da stabilire. Gli alberi costruiti con questo aiuto suggeriscono che i cambi nella sequenza
di particolari geni o proteine avvengono ad un tasso costante che d\`a un orologio molecolare per l'evoluzione unico per differenti categorie di sequenze di DNA. Questo orologio scorre 
pi\`u rapidamente per sequenze non soggette a selezione purificativa: porzioni di introni che non hanno splicing o segnali regolatori, la terza posizione in codoni sinonimi e geni
che sono stati inattivati irreversibilmente da mutazioni (pseudogeni) e pi\`u lentamente per sequenze con vincoli funzionali critici. Occasionalmente cambi rapidi possono accadere in
sequenze altamente conservate e si pensa riflettano periodi di selezione positiva per mutazioni vantaggiose. La velocit\`a di questo orologio molecolare \`e determinata dal tasso di 
selezione purificante e dal tasso di mutazione. Si nota come il tasso di mutazione in mitocondri animali sia estremamente alto. Le categorie per cui l'orologio \`e veloce sono le pi\`u 
informative per eventi evolutivi recenti. Questi orologi presentano inoltre una risoluzione temporale maggiore rispetto ai record fossili e sono una guida affidabile. 
\subsection{Un confronto tra cromosomi umani e dei topi mostra come la struttura dei genomi diverge}
I lignaggi umani e dei topi si sono evoluti separatamente per circa $80$ milioni di anni e i secondi hanno orologi molecolari veloci, divergendo pi\`u rapidamente. Se il genoma \`e 
organizzato in cromosomi quasi identicamente la sua organizzazione diverge significativamente. Continuano ad esistere blocchi di DNA in cui l'ordine dei geni \`e lo stesso, dette regioni
di sintenia. Da queste osservazioni si \`e notato come piccole parti di DNA sono eliminate e aggiunte a velocit\`a elevate, anche se guadagni di sequenza da duplicazione e moltiplicazione
di trasposoni ha compensato per queste eliminazioni, pertanto la dimensione del genoma \`e rimasta invariata dall'ultimo comune antenato di topi e umani per gli umani, mentre \`e 
diminuita per i topi. Prove di questo possono essere ottenute dal confronto dettagliato delle regioni di sinteina. Il DNA \`e aggiunto ai genomi attraverso la duplicazione spontanea
di segmenti cromosomici e dall'inserzione di nuove coppie di trasposoni attivi essendo gli eventi di trasposizione duplicativi. 
\subsection{La dimensione del genoma dei vertebrati riflette i tassi relativi di addizione e perdita del DNA in un lignaggio}
In vertebrati imparentati lontanamente la dimensione del genoma pu\`o cambiare consideramente senza effetto sull'organismo o sul numero dei geni. Questo avviene in quanto tutti i 
vertebrati sono soggetti di un continuo processo di addizione e perdita del DNA e pertanto la dimensione del genoma dipende dall'equilibrio di questi processi. Il numero di geni \`e 
invece mantenuto attraverso selezione purificante.
\subsection{Si pu\`o inferire la sequenza di alcuni genomi antichi}
I genomi di organismi ancestrali possono essere inferiti ma non osservati direttamente. La selezione deve aver mantenuto propriet\`a funzionali critiche, pertanto il confronto tra 
organismi odierni mostra che la frazione del genoma soggetta a selezione purificante \`e piccola e pertanto la sequenza di DNA varia grandemente. Per antenati con un gran numero di 
discendenti si pu\`o inferire la loro sequenza confrontando la sequenza di questi ultimi. 
\subsection{Il confronto tra sequenze multispecie identifica sequenze di DNA conservato di funzione sconosciuta}
La massa di sequenze di DNA nei database mette a disposizione una risorsa utile a determinare i cammini evolutivi e come le cellule e gli organismi funzionano. Si \`e notata una grande
quantit\`a di sequenza di DNA che non codifica proteine conservata durante l'evoluzione dei mammiferi, chiaramente rivelata analizzando blocchi di DNA sintetici da specie diverse, 
determinando le sequenze conservate multispecie. La maggior parte di queste sequenze non codificanti sono corte per la maggior parte e la loro conservazione implica una funzione 
importante che \`e stata mantenuta dalla selezione purificante. Molte di queste sequenze producono molecole di RNA non tradotte come i lunghi RNA non codificanti lncRNA che si pensa
abbiano un'importante funzione nella regolazione dell'espressione genica. 
\subsection{Cambi in sequenze precedentemente conservate possono aiutare a decifrare passi critici dell'evoluzione}
L'analisi delle differenze tra le specie viene effettuata attraverso le sequenze conservate multispecie che rappresentano le sequenze di DNA probabilmente funzionalmente importanti. 
Queste sequenze non sono conservate perfettamente e si trovano delle differenze che in una piccola percentuale rappresentano segni di scatti evolutivi. Le regioni accelerate umane
(HARs) si pensa riflettano funzioni importanti che ci differiscono dalle altre specie. Un quarto dei $50$ siti individuali si trovano vicino a geni associati con lo sviluppo neuronale. 
Un altro approccio consiste nel ricercare mutazioni importanti attraverso le sequenze di DNA conservate e concentrarsi sui siti cromosomici con eliminazioni. Solo una di queste 
eliminazioni trovate negli umani si trovano in regioni codificanti. 
\subsection{Mutazioni nella sequenza di DNA che controlla l'espressione genica ha guidato molti dei cambi evolutivi nei vertebrati}
Si pu\`o tentare di tracciare le origini degli elementi regolatori del DNA che hanno avuto ruoli critici nell'evoluzione dei vertebrati. Si inizia identificando i $3$ milioni di sequenze
non codificanti che sono state conservate in evoluzioni recenti dei vertebrati che rappresenta un'innovazione caratteristica di un ramo dell'albero dei vertebrati e consistono di DNA
regolatorio che regola il gene vicino. Si possono identificare i geni pi\`u vicini a esse e si pu\`o stimare quando ogni elemento regolatorio \`e arrivato in esistenza. Gli elementi 
regolatori conservati appaiono associati con geni che codificano proteine per la regolazione della transizione per proteine per lo sviluppo embrionale, successivamente si trovano le
regolazioni per geni codificanti i recettori per segnali extracellulari e l'ultima innovazione sembra riguardante le proteine che modificano altre proteine dopo la traduzione. 
\subsection{La duplicazione genica fornisce una fonte di novit\`a genica durante l'evoluzione}
L'evoluzione dipende dalla creazione di un nuovo gene e dalla modifica di quelli preesistenti. I geni senza omologhi sono scarsi e si trovano spesso famiglie di geni con differenti 
membri in specie diverse per la cui creazione \`e stato necessario la duplicazione ripetuta di geni le cui copie diversero per svolgere funzioni diverse. La  duplicazione dei geni 
avviene ad alti tassi in tutti i lignaggi, contribuendo al processo di addizione del DNA. 
\subsection{Geni duplicati divergono}
Inizialmente non \`e presente alcuna selezione per mantenere lo stato duplicato, pertanto molti eventi di duplicazione possono essere seguiti da mutazioni che causano la perdita di 
funzione sui geni. Questo ciclo eliminerebbe l'effetto della duplicazione. Passando il tempo la somiglianza tra tali pseudogeni disattivati e l'origine della duplicazione 
viene erosa dalle mutazioni con la relazione omologa non pi\`u identificabile. Un'altra possiblit\`a \`e che entrambe le copie rimangano funzionali mentre divergono, prendendo ruoli 
diversi e spiega la presenza delle famiglie geniche e ha un ruolo critico nell'evoluzione di complessit\`a biologica crescente. Una duplicazione dell'intero genoma pu\`o avvenire
nel caso in cui avvenga la replicazione del DNA senza divisione cellulare. Il risultato con il passare del tempo crea altro materiale genico con possibilmente funzioni diverse. 
\subsection{L'evoluzione del gene globina mostra come la duplicazione del DNA contribuisce all'evoluzione degli organismi}
La somiglianza delle proteine globina odierne in struttura e sequenza amminoacida indica che devono derivare da un gene antenato comune. Si possono ricostruire eventi che hanno prodotto
i vari tipi di molecole di emoglobina considerando le diverse forme che assume nell'albero della vita. Questa molecola ha permesso agli organismi di crescere in dimensione e molecole 
che legano l'ossigeno si possono trovare in piante, funghi e batteri. Negli animali la proteina pi\`u primitiva \`e una catena polipeptidica di circa $150$ amminoacidi, ma si presenta in
forma pi\`u complessa in vertebrati complessi, composta da due catene di globina. Durante una serie di mutazioni e duplicazioni hanno stabilito un gene globina per nel genoma in cui 
ognuno codificante due catene $\alpha$- e $\beta$- che si associano per formare una molecola di emoglobina che consiste delle quattro catene. I quattro siti che si legano all'ossigeno
nella molecola $\alpha_2\beta_2$ interagiscono, permettendo un cambio allosterico cooperativi quando lega e rilascia l'ossigeno che permette uno scambio pi\`u efficiente. Durante 
l'evoluzione dei mammiferi il gene per la catena $\beta$ si \`e duplicato ed \`e stato modificato dando origine a una catena simile alla $\beta$ presente nel feto, con una molecola con 
un affinit\`a pi\`u alta rispetto a quella presente negli adulti. Il nuovo gene si \`e duplicato di nuovo e mutato producendo due nuovi geni $\varepsilon$ e $\gamma$, con la prima 
prodotta nelle prime fasi dello sviluppo per formare $\alpha_2\varepsilon_2$ rispetto alla catena fetale $\gamma$ che produce $\alpha_2\gamma_2$. Durante l'evoluzione dei primati una 
nuova duplicazione del gene $\beta$ ha prodotto un gene per la $\delta$-globina producendo una forma minore di emoglobina $\alpha_2\delta_2$ presente unicamente nei primati adulti.
Ognuno di questi geni duplicati \`e stato modificato da mutazioni puntiformi che influiscono sulle propriet\`a della molecola di emoglobina finale e cambi nelle regioni regolatorie che
determinano la temporizzazione e il livello di espressione del gene e pertanto ogni globina \`e creata in diverse quantit\`a e a tempi diversi dello sviluppo dell'uomo. Nel genoma umano 
i geni generati dal $\beta$ gene originale sono ordinati in una serie di sequenze di DNA omologhe che si trovano a $50^.000$ basi di distanza su un singolo cromosoma. Un singolo cluster
di geni $\alpha$-globina si trova su un cromosoma separato. Esistono duplicazioni di sequenze di pseudo geni globina disattivati da mutazioni che impediscono la loro espressione come
proteine funzionali.
\subsection{I geni che codificano nuove proteine possono essere creati ricombinando gli esoni}
La duplicazione del DNA pu\`o agire su una scala pi\`u piccola unendo segmenti di DNA corti duplicati. Le proteine codificate da geni creati in questa maniera possono essere riconosciute
dalla presenza di domini simili legati covalentemente l'uno all'altro in serie. Le immunoglobuline e le proteine fibrose sono un esempio di questo. Ogni esone separato codifica 
un'unit\`a di proteina pieghevole o dominio. Si crede che l'organizzazione del DNA come una serie di esoni separati da introni ha facilitato l'evoluzione di nuove proteine: la 
duplicazione necessaria per formare un singolo gene codificante per una proteina con domini ripetuti pu\`o avvenire rompendo e riunendo il DNA lungo l'introne a un lato di un esone e 
pertanto gli introni aumentano la probabilit\`a di un evento di duplicazione favorevole. Vari parti dei geni sono servite come elementi modulari che si sono duplicati e spostati lungo il
genoma per formare una grande diversit\`a delle cose viventi. 
\subsection{Mutazioni neutrali si diffondono diventando fissate in una popolazione, con probabilit\`a dipendente dalla dimensione della popolazione}
Confrontando due specie che si sono separate fa poca differenza quali individui sono confrontati. Nonostante questo ogni diversit\`a fissa comincia come una nuova mutazione in un 
individui singolo. Se la dimensione della popolazione riproducente in cui \`e avvenuta la popolazione \`e $N$, la frequenza dell'allele iniziale per una nuova mutazione \`e 
$\frac{1}{2N}$ per un organismo diploide. La fissazione nella popolazione dipende dalle conseguenze funzionali della popolazione: una mutazione con un evento deleterio viene eliminata
dalla selezione purificante e non si fissa, mentre le mutazioni che danno un vantaggio riproduttivo su individui si possono diffondere rapidamente. La mutazione selezionata lungo una 
quantit\`a di sequenze vicine sar\`a una parte di un grande mosaico. Le mutazioni neutrali possono diffondersi e fissarsi nella popolazione e contribuiscono al cambio evolutivo dei 
genomi, creando la maggiore differenza tra scimmie e esseri umani. La diffusione delle mutazioni neutrali dipende su una variazione casuale del numero di progenie che porta la mutazione
prodotta da ogni individuo che porta la mutazione, causando cambi nella relativa frequenza dell'allele mutante nella popolazione. Questo processo di cammino casuale pu\`o causare 
l'estinzione della mutazione o la sua fissazione. Assumendo una popolazione di dimensione costante e con accoppiamenti casuali e selezione neutrale per la popolazione, quando una nuova
mutazione avviene in una popolazione di dimensione $N$, la probabilit\`a che essa si fissi \`e circa $\frac{1}{2N}$ in quanto ci sono $2N$ copie di geni nella popolazione diploide e 
ognuna di esse ha una probabilit\`a uguale di diventare la versione dominante. Per le mutazioni che si fissano il tempo di fissazione \`e circa $4N$ generazioni. 
\subsection{Molto pu\`o essere imparato dall'analisi della variazione tra gli umani}
Esiste un grande numero di varianti che si incontrano. Da un confronto dettagliato di sequenze di DNA tra un gran numero di umani in tutto il globo \`e possibile determinare quante 
generazioni sono passate dall'origine di una mutazione neutrale. Da questi dati \`e possibile determinare le migrazioni degli esseri umani primitivi. Un altra fonte di variazione si 
trova nella duplicazione e eliminazione di grandi blocchi di DNA. Confrontando il genoma di individuo con il modello si possono trovare circa $100$ differenze in blocchi di sequenza. 
Alcune di queste copy number variations (CNVs) sono molto comuni, mentre altre sono presenti in poche minoranze. Le variazioni intraspece sono caratterizzate come polimorfismi a 
singolo nucleotide (SNPs) che sono punti in cui la sequenza genomica in una grande frazione della popolazione possiede un nucleotide, mentre un'altra un secondo. Per qualificarsi come 
un polimorfismo la variante deve essere comune abbastanza che due individui casuali siano diversi sia almeno dell' $1\%$. Queste variazioni possono utili per analisi della mappatura
genetica in cui si tenta di associare tratti specifici con sequenze specifiche, pur non avendo effetto sul fitness umano. Si trovano anche sequenze con tassi di mutazioni estremi come
le ripetizioni CA, onnipresenti nel genoma umano e degli altri eucarioti. Se sequenze con il motif $(CA)_n$ sono replicate con bassa fedelt\`a a causa di un errore tra lo stampo e 
un nuovo filamento di DNA appena sintetizzato, pertanto il valore $n$ varia  grandemente tra un genoma e il prossimo. Questa ripetizione crea un ottimo marcatore. Nonostante si trasmetta
abbastanza fedelmente tra i figli (una lunghezza per la madre e una per il padre) i cambi sono abbastanza frequenti da permettere alti livelli di eterozigosit\`a nella popolazione. 
Un sottoinsieme di questi mutazioni sono responsabili per gli aspetti ereditabili degli individui.
