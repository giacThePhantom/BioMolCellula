\chapter{DNA, cormosomi e genomi}
La vita dipende dall'abilit\`a della cellula di conservare, recuperare e tradurre le istruzioni genetiche richieste per creare e mantenere un organismo vivente. Queste informazioni
ereditarie sono passate da una cellula a una sua figlia durante la divisione cellulare e da una generazione di organismi all'altra attraverso le cellule riproduttive, sono salvate come
geni. Le informazioni genetiche consistono principalmente su istruzioni per la costruzione delle proteine, macromolecole versatili che svolgono la maggior parte delle funzioni della 
cellula. Le informazioni genetiche sono trasportate su cromosomi, strutture a filo nel nucleo che diventano visibili durante la divisione e composti di DNA (acido desossiribonucleico) e 
proteine in egual misura. La determinazione della struttura a doppia elica del DNA ha risolto il problema di come le informazioni nel DNA sono replicate e come una molecola di DNA 
utilizza la sequenza dei suoi monomeri per produrre le proteine. 
\section{La struttura e la funzione del DNA}
Negli anni $50$ la struttura del DNA \`e stata determinata grazie a diffrazione a raggi X, che indicarono la composizione a due fili avvolti in un'elica fornendo un grande indizio a
Watson-Crick e al loro modello.
\subsection{Una molecola di DNA consiste di due catene di nucleotidi complementari}
Una molecola di acido desossiribonucleico cosiste di due catene polinucleotide lunghe note come strand e composte da quattro tipi di subunit\`a. Le catene sono antiparallele tra di loro
e tra le basi dei nucleotidi si formano legami a idrogeno che le tengono unite. I nucleotidi sono composti da zuccheri a 5 atomi di carbonio a cui sono attaccadi dei gruppi fosfati e 
una base contenente dell'azoto. Nel caso dei nucleotidi del DNA lo zucchero \`e il desossiribosio attaccato a un singolo gruppo fosfato (da cui il nome) e la base pu\`o essere adenina
(A), citosina (C), guanina (G) o timina (T). I nucleotidi sno legati covalentemente in unca catena attraverso gli zuccheri e i fosfati che formano un backbone di legami 
zucchero-fosfato-zucchero-fosfato alternati. I legami dei nucleotidi danno al filo del DNA una polarit\`a chimica. Il gruppo $5'$ fosfato si lega con il gruppo $3'$ idrossile di un
altro monomero e pertanto tutte le subunit\`a hanno lo stesso orientamento. Inoltre le due terminazioni dello strand sono facilmente distinguibili e ci si riferisce ad esse come alla
terminazione $3'$ e $5'$, indicando la polarit\`a. Grazie alla direzionalit\`a e linearit\`a del filo di DNA pu\`o essere letto facilmente. La doppia elica si genera dalla struttura 
chimica delle catene polinucleotidiche in quanto sono mantenute insieme da legami a idrogeno tra le basi sui fili diversi tutte le basi si trovano all'interno della doppia elica con i 
backbones verso l'esterno. In ogni caso una base a due anelli (una purina) \`e sempre legata con una ad anello singolo (una pirimidina): A si accoppia con T e G con C. Questo 
accoppiamento complementare delle basi permette a coppie di basi di essere ammassate nell'ordinamento pi\`u energeticamente favorevole all'interno della doppia elica in quanto ogni
coppia possiede la stessa lunghezza mantenendo i backbones a distanza costante. Per massimizzare l'efficienza spaziale i backbones si avvolgono formando una doppia elica destra con
un giro completo ogni $10$ paia. I membri di ogni base possono adattarsi insieme solo se i fili sono antiparalleli. Ogni strand della molecola contiene una sequenza complementare a
quella dell'altro strand. 
\subsection{La struttura del DNA mette a disposizione un meccanismo per l'ereditariet\`a}
La struttura a polimero lineare composto da quattro monomeri permette al DNA di trasportare informazioni in forma chimica, mentre la natura a doppio filamento permette la sua 
duplicazione. Essendo ogni filamento complementare all'altro pu\`o agire come stampo per la sintesi di un nuovo filamento complementare. Nominando i filamenti $S$ e $S'$, $S$ agisce
da stampo per la formazione di $S'$ e $S'$ per $S$, permettendo l'accurata copia dell'informazione nel DNA. Questa capacit\`a permette alla cellula di replicare il proprio genoma
e passarlo ai discendenti. Il DNA permette la codifica di proteine (le cui funzioni sono determinate dalla forma e pertanto in ultimo dalla sequenza di amminoacidi) attraverso una
corrispondenza esatta tra i quattro nucleotidi e i venti amminoacidi. L'intero gruppo di informazioni mantenute dal DNA di un organismo \`e detto genoma e specifica tutte le molecole
di RNA e proteine che l'organismo sintetizzer\`a. 
\subsection{Negli eucarioti il DNA \`e racchiuso nel nucleo cellulare}
Circa tutti il DNA della cellula si trova isolato nel nucleo, che occupa circa il $10\%$ del volume totale, delimitato da un involucro nucleare formato da due bistrati lipidi 
concentrici. Queste membrane sono puntuate a intervalli da pori nucleari in cui le molecole si trasferiscono dal nucleo al citosol. Tale involucro \`e conneso a un sistema di 
membrane intracellulari detto reticolo endoplasmatico che si estende nel citoplasma ed \`e meccanicamente supportato da una rete di filamenti intermedi detti lamine nucleari. L'involucro
permette alle proteine che agiscono sul DNA di essere concentrate dove necessario e mantiene gli enzimi nucleari e citosolici separati. 
\section{DNA cromosomico e il suo confezionamento nella fibra cromatina}
