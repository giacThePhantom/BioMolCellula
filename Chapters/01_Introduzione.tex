\chapter{Introduzione}
\section{Microscopia}
I microscopi possono essere divisi in due categorie principali: i microscopi ottici (composti, semplici o a fluorescenza) e i microscopi elettronici. Mentre
i primi utilizzano la luce e lenti ingrandenti i secondi utilizzano fasci di elettroni direzionati attraverso campi magnetici. Diverse tecniche di 
microscopia elettronica includono trascrizione, sezione, freeze fracture e scansione. I microscopi elettronici generano immagini unicamente in bianco e nero
e si rende pertanto necessario utilizzare dei falsi colori in un secondo momento, ma hanno come vantaggio una maggiore risoluzione delle immagini. Per un
microscopio ottico l'ingrandimento dell'immagine \`e dato dall'ingrandimento dell'obiettivo moltiplicato per l'ingrandimento degli eye-pieces: 
$$M_{microscope} = M_{objective}\cdot M_{eyepieces}$$
\subsection{Risoluzione}
La risoluzione di un immagine \`e una misura del dettaglio che essa contiene e se attraverso tecniche digitali l'ingrandimento \`e illimitato la risoluzione
no. Si indica con risoluzione la minor distanza tra due punti che possono essere distinti come separati. La risoluzione dipende da parametri dell'utente 
finale e da parametri fisici. 
\subsubsection{Parametri fisici}
I parametri fisici che determinano la risoluzione sono:
\begin{itemize}
\item Il corretto allineamento del sistema ottico del microscopio.
\item La lunghezza d'onda della luce ($\lambda$): maggiore la lunghezza d'onda minore la risoluzione.
\item L'apertura numerica (NA) dell'obiettivo e del condensatore. Questo parametro indica la capacit\`a di un obiettivo di raccogliere luce e risolvere 
dettagli ad una distanza fissata dall'oggetto. Dipende dall'ingrandimento e dall'indice di rifrazione del medium tra microscopio e oggetto (aria, acqua, 
olio). Maggiore l'indice di rifrazione maggiore il numero di apertura e maggiore la risoluzione.
\end{itemize}
\subsection{Tecniche di microscopia ottica}
\subsubsection{Microscopia a cambio di fase}
Nella microscopia a cambio di fase viene sfruttato lo shift di fase della luce quando attraversa il corpo che si vuole osservare. L'interpretazione dello 
shift d\`a origine a diverse possibili considerazioni sull'oggetto.
\subsubsection{Microscopia a fluorescenza}
La microscopia a fluorescenza sfrutta il fenomeno della fluorescenza: certe molecole, quando colpite da certe lunghezze d'onda, si eccitano. Successivamente
quando la molecola ritorna dallo stato eccitato a quello basale emette a sua volta un'onda luminosa di lunghezza d'onda maggiore di quella con cui era
stata colpita (legge di Stokes). Per questa tecnica vengono tipicamente utilizzate delle proteine come GFP (green fluorescent protein), YFP, CFP tipicamente
estratte da organismi (nel caso della GFP da una medusa) che permettono pertanto la loro codifica nel DNA della cellula. Queste proteine vengono aggiunte ad
una proteina in modo da riuscire ad osservarne il comportamento. Si deve prestare attenzione al fatto che per massa o struttura questa aggiunta potrebbe 
creare una variazione in comportamento della proteina oggetto di studio. I fattori essenziali per la microscopia a fluorescenza sono:
\begin{itemize}
\item Eccitazione di alta intensit\`a.
\item Filtri di eccitazione e emissione appropriati.
\item Auto-fluorescenza minima nell'oggetto di studio.
\item Utilizzo di un olio di immersione non fluorescente. 
\item Antifade reagents.
\end{itemize}
Questa tecnica permette pertanto di osservare proteine in cellule vive a differenza dell'antibody staining. Un'importante applicazione \`e 
l'immunoistochimica.
\paragraph{FRET}
Le tecnica FRET si utilizza per determinare la prossimit\`a di due diverse proteine: si attaccano ad esse due molecole fluorescenti tali che l'emissione 
della prima eccita la seconda, che a sua volta emette luce. Se questo accade si \`e dimostrato che le due proteine sono vicine.
\paragraph{Photobleach}
Questa tecnica viene utilizzata per mostrare la velocit\`a di movimento di una proteina della cellula: si trova una cellula di controllo e quella oggetto di 
studio entrambe con la proteina fluorescente. La seconda viene sottoposta ad un raggio ad alta potenza (photoactivation) che distrugge la parte fluorescente
della proteina. Si osserva ora la velocit\`a con cui la luminescenza torna nella zona colpita, confrontandola con la cellula di controllo e si crea il grafo
di velocit\`a di movimento della proteina.
\section{Storia}
Mendel \`e considerato il padre della genetica moderna. Fu un monaco benedettino che visse nell'$800$ che introdusse la quantificazione in biologia. I suoi esperimenti riguardanti piante
di piselli hanno osservato che prendendo due linee pure con un fenotipo diverso e incrociandole nella prima generazione si ottenevano piante con un solo fenotipo. Successivamente 
incrociando tra di loro piante della prima generazione osserv\`o una ricomparsa dell'altro fenotipo in rapporto di $3$ a $1$. Mendel associ\`o questa scoperta alla presenza di due 
alleli, uno dominante e uno recessivo che determinavano il fenotipo. Successivamente determin\`o anche che due caratteri vengono trasmessi in modo indipendente. \\
Nel $1871$ viene scoperto il DNA grazie al lavoro di Miescher che chiama nucleina in quanto si trova ne nucleo della cellula. Il suo lavoro riguardava l'estrazione delle cellule da
bende impregnate di pus, ricche pertanto di materiale cellulare. Dall'estrazione della nucleina, ricca di fosforo e acida,  not\`o che la sua concentrazione era lineare con il numero di 
cellule e le cellule che si dividevano molto ne possedevano di pi\`u.\\
Dal $1909$ gli esperimenti di Morgan gli valsero il premio Nobel per le sue scoperte riguardanti il ruolo dei cromosomi nell'ereditariet\`a. Riconobbe infatti che la sede 
dell'informazione genetica sono i cromosomi, isolati dalla Drosophiila, utilizzata a causa dei suoi grandi cromosomi. Riusc\`i pertanto a confermare che i geni sono depositati nei 
cromosomi nel nucleo della cellula, che sono organizzati in una lunga riga nei cromosomi, che tratti dipendenti l'uno dall'altro corrispondono a geni che sono in siti vicini sui 
cromosomi e scopr\`i il fenomeno del crossover.\\
Nel $1928$ Griffith studi\`o lo streptococcus Pneumoniae, un patogeno presente in due varianti con diversa patogenicit\`a: Rough (R) e Smooth (S). Se la variante S viene iniettata in un
topo questo muore, mentre con R non muore. Dopo aver ucciso il patogeno di tipo S e averlo iniettato in un topo sano, questo non si ammalava, ma unendo a S morto un agente R vivo il 
topo, iniettato con la combinazione, moriva. Viene pertanto ipotizzato il passaggio orizzontale di materiale genetico.\\
Nel $1943$ Avery conferma che il DNA \`e la molecola responsabile del trasferimento genetico orizzontale utilizzando lo stesso agente di Griffith riusc\`i a separare in coltura 
l'estratto cellulare nelle varie componenti molecolari: proteine, DNA, lipidi e polisaccaridi. Trattando le cavie con queste molecole separatamente scopr\`i che il responsabile della
trasmissione genica orizzontale era il DNA.\\
Nel $1952$ Hershey e Chase marcarono DNA con $P^{32}$ e le proteine con $S^{35}$ dei virus batteriofagi (che replicano il proprio DNA attraverso batteri fino a farli esplodere). Durante
l'infezione di questi virus sull'escherichia coli, separati tramite frullazione e con il loro materiale estratto attraverso lisi si nota come la maggior parte conteneva $P^{32}$, 
determinando il DNA come la molecola responsabile dell'infezione.\\
Nel $1953$ Watson e Crick riuscirono a determinare la struttura del DNA e a formulare il dogma centrale della biologia molecolare:
\begin{figure}[h]
\begin{equation*}
DNA\xrightarrow[\leftarrow]{}mRNA\rightarrow proteina
\end{equation*}
\caption{Il dogma centrale della biologia molecolare}
\end{figure}
Il passaggio da DNA a mRNA pu\`o avvenire anche in senso contrario nel caso dei retrovirus.\\
Nel $1996$ Mello e Fire evidenziarono l'importanza dell'RNA e del microRNA, vincendo il premio Noble per la scoperta dell'interferenza dell'RNA e la propriet\`a dell'RNA a doppio
filamento di interferire e spegnere l'espressione genica. Il microRNA \`e un RNA con 20-22 nucleotidi a singolo filamento, scoperto prima nelle piante che lo usano per difendersi dalle
infezioni virali. L'RNA a doppio filamento fa in modo che il gene codificante la proteina corrispondente venga silenziato in modo che non esprima la proteina.
