\chapter{Introduzione}
\section{Microscopia}
\subsection{Tipi di microscopi}
\subsubsection{Microscopi ottici}
I microscopi ottici utilizzano la luce e lenti ingrandenti per permettere la visualizzazione del campione. Ne esistono vari tipi:
\begin{itemize}
	\item In base al numero di lenti ingrandenti: 
		\begin{itemize}
			\item Semplici: una sola lente.
			\item Composti: pi\`u lenti, in tal caso la capacit\`a di ingrandimento totale \`e data dal prodotto delle capacit\`a di ingrandimento. 
		\end{itemize}
	\item In base alla luce che utilizzano:
		\begin{itemize}
			\item In campo chiaro: la luce attraversa direttamente il campione.
			\item A contrasto di fase: permette di distinguere caratteristiche del campione senza colorarlo basandosi sui diversi indici di rifrazione tra il campione e il 
				mezzo circostante: la luce deviata dal campione e quella no viene fatta convergere su una superficie in modo che abbia lunghezza d'onda e frequenza
				diverse rispetto alla luce incidente mettendo in evidenza i bordi.
			\item A contrasto di interferenza differenziale: a differenza di quello a contrasto di fase il campione viene illuminato di lato, eliminando l'alone di 
				diffrazione luminoso.
			\item In campo scuro: simile a quella a contrasto di fase, ma solo la luce che ha attraversato il campione viene raccolta dall'obiettivo che appare pertanto 
				chiaro in campo scuro. 
			\item A fluorescenza: si compone di:
				\begin{itemize}
					\item Lampada ad arco a vapori di mercurio: lampada che genera luce ultravioletta.
					\item Filtro di eccitazione: seleziona la lunghezza d'onda ultravioletta che si vuole generare.
					\item Condensatore a campo oscuro: aumenta il contrasto delle strutture legate dalle molecole fluorescenti in modo che lo sfondo sia nero. 
					\item Filtro di sbarramento: evita che residui di luce ultravioletta raggiungano gli occhi dell'operatore. 
				\end{itemize}
				Si dividono in upright scope con l'obiettivo sopra il campione, luce bianca dal basso e fluorescente dall'alto e inverted scope con la luce dal basso e
				micromanipolatori in quanto sopra \`e libero. Vengono utilizzate molecole particolari fluorescenti come GFP , YFP e CFP (green, yellow e cyan fluorescent 
				protein). Se la fosforescenza \`e un fenomeno duraturo nel tempo la fluorescenza avviene temporaneamente, solo mentre la molecola viene eccitata da una 
				lunghezza d'onda che causa alla molecola di produrne una a lunghezza maggiore (energia minore), con uno spostamento determinato dallo spostamento di 
				Stockes.
		\end{itemize}
\end{itemize}
\paragraph{Risoluzione}
Si intende per risoluzione una misura del dettaglio che un'immagine contiene, ovvero la minor distanza tra due punti che possono essere distinti come separati. Dipende da parametri 
dell'utente e fisici. I parametri fisici sono il corretto allineamento del sistema ottico del microscopio, la lunghezza d'onda della luce ($\lambda$, inversamente proporzionale alla
risoluzione) e l'apertura numerica ($NA$, direttamente proporzionale all'indice di rifrazione e alla risoluzione), ovvero la capacit\`a di un obiettivo di raccogliere luce e risolvere 
dettagli ad una distanza fissata dall'oggetto. Dipende dall'ingrandimento e dall'indice di rifrazione del medium tra microscopio e oggetto. 
\subsubsection{Microscopi elettronici}
I microscopi elettronici utilizzano fasci di elettroni direzionati da campi magnetici per superare la capacit\`a di ingrandimento di un microscopio ottico. Non sono per\`o in grado di
compiere osservazioni in vivo. Si distinguono in:
\begin{itemize}
	\item SEM, scanning electron microscopy: un fascio di elettroni colpisce il campione che si vuole osservare che emette diverse particelle come gli elettroni secondari che vengono
		rilevati da un rivelatore e convertiti in impulsi elettrici. Il fascio scansiona una zona rettangolare riga per riga in sequenza. 
	\item TEM, transmission electron microscopy: gli elettroni che costituiscono il fascio attraversano una sezione dove \`e stato creato il vuoto per poi passare completamente
		attraverso il campione. Richiede che il campione sia preparato in una ``thin section" o attraverso freeze fracture (il campione viene velocemente congelato, poi rotto, 
		replicato in modo che sia la replica ad essere osservata al microscopio). 
\end{itemize}
\subsection{Tecniche di microscopia}
\subsubsection{Microscopia in campo chiaro}
\subsubsection{Immunoistochimica}
\`E una tecnica in cui si individuano specifiche molecole o strutture del compartimento intra ed extracellulare in base al principio di coniugazione antigene-anticorpo con sistemi di 
rivelazione enzimatici o fluorescenza (poste sugli anticorpi) che rendono visibile l'avvenuta reazione al microscopio. Avviene con cellule fissate e quindi morte.
\subsubsection{Fret}
\`E una tecnica utilizzata per verificare le interazioni tra proteine in una cellula: ad entrambe viene fatta esprimere una porzione a fluorescenza in modo che la lunghezza d'onda 
espressa dalla prima ecciti la seconda. In questo modo la seconda emetter\`a luce se e solo se la prima, eccitata dal microscopio, si trova in sua prossimit\`a. Si noti come sono i 
fluorofori e non le proteine a dover essere vicine. 
\subsubsection{Biomolecular fluorescent complementation}
\`E una tecnica in cui ad una proteina $A$ viene fatta esprimere la parte N-terminale del fluoroforo e alla proteina $B$ la parte C-terminale. Se le proteine interagiscono produrranno
una lunghezza d'onda. 
\subsubsection{Photobleach}
\`E una tecnica in cui con un laser si altera un fluoroforo in modo che non sia pi\`u capace di produrre flourescenza. In questo modo si pu\`o osservare quanto tempo impiega la 
fluorescenza a tornare al livello precedente misurando pertanto la velocit\`a di diffusione o trasporto dell'elemento che si vuole osservare. 
\subsubsection{Fotoattivazione}
\`E una tecnica che permette di osservare la cinetica di una molecola: a questa si fa esprimere la GFP che viene eccitata solo localmente. La molecola continuer\`a ad emettere per un
certo periodo, permettendo di osservare i suoi spostamenti. 
\section{Tecniche di laboratorio}
\subsection{Reazione a catena della polimerasi PCR}
Questa tecnica consente l'amplificazione di frammenti di acidi nucleici dei quali si conoscono le sequenze neuclotidiche iniziali e terminali. Ricostruisce la sintesi di un segmento di 
DNA a doppia elica a partire da un filamento a singola elica. Quello mancante viene ricostituito a partire da una serie di nucleotidi che vengono disposti nella corretta sequenza. 
Devono essere disponibili i nucleotidi da polimerizzare sotto forma di desossiribonucleosidi trifosfati (dNTP). Il DNA deve essere denaturato. Il segmento da ricostruire pu\`o essere 
solo prolungato e devono esserci opportune condizioni di temperatura e pH. Sono necessari alla reazione una quantit\`a del segmento che si vuole riprodurre, nucleotidi liberi, primer 
a brevi sequenze di DNA complementari alle estremit\`a $5'$ e $3'$ dei due filamenti da riprodurre, una DNA polimerasi termo-resistente (TAQ polimerasi), un tampone per stabilizzare il 
pH e elementi di supporto. Si trovano pertanto tre fasi: 
\begin{enumerate}
	\item Fase di denaturazione: la soluzione di DNA da replicare, desissiribonucleotidi trifosfati, ioni magnesio, primer e TAQ polimerasi vengono portate a una temperatura tra i 
		$94$ e i $99$ gradi in modo che la doppia elica si scinda e i due filamenti siano liberi in soluzione.
	\item Fase di annealing: la temperatura viene abbassata tra i $40$ e i $55$ gradi in modo da permettere il legame dei primer alle regioni loro complementari dei filamenti di DNA 
		denaturati.
	\item Fase di prolungamento: la temperatura viene alzata tra i $65$ e i $72$ gradi in modo da massimizzare l'azione della TAQ polimerasi in modo da allungare i primer legati 
		utilizzando come stampo il filamento singolo di DNA.
\end{enumerate}
Il ciclo viene ripetuto tra le $30$ e $40$ volte. La lunghezza dei primer si aggira tra le $20$ e $30$ paia di basi in modo da poter mantenere la temperatura di annealing ragionevole
(che dipende anche dalla composizione delle basi) e  impedire la formazione di strutture secondarie.
\subsection{Elettroforesi}
Questa tecnica analitica e separativa si basa sul movimento di particelle elettricamente cariche immerse in un fluido per effetto di un campo elettrico applicato mediante una
coppia di elettrodi al fluido stesso (catodo negativo e anodo positivo). Le particelle si muovono verso quello che assume carica opposta alla propria. La mobilit\`a dipende dalla 
dimensione della molecola, dalla carica, natura e concentrazione del mezzo elettroforetico, dalla concentrazione delle molecole  e dalla tensione applicata. Essendo che le molecole di
DNA possono presentare forme diverse migrano nel gel in maniera diversa, pertanto si applica un campo alternato in cui le molecole di DNA sono sono sottoposte alternativamente a due 
campi elettrici perpendicolari in modo che le molecole pi\`u piccole si riorientino pi\`u velocemente e abbiano maggiore mobilit\`a. 
\subsection{Blots}
\subsubsection{Southern blot}
Questa tecnica viene usata per rilevare la presenza di sequenze di DNA specifiche in una miscela complessa. Un campione eterogeneo di DNA genomico viene trattato con enzimi di 
restrizione (classe di idrolasi che catalizzano il taglio endonucleolitico del DNA per dare frammenti a doppia elica specifici con fosfati terminali al $5'$) e sottoposto ad 
elettroforesi su gel d'agarosio o di poliacrilammide (polimeri). Nel gel si ossever\`a uno smear, una striscia continua e non bande nette in quanto il DNA digerito dall'enzima di 
restrizione possieder\`a tantissimi punti di taglio e i diversi frammenti migrano con velocit\`a diverse in base al peso molecolare. Il gel viene immerso in una soluzione alcalina per
denaturare il DNA. Il gel viene coperto da un foglio di nitrocellulosa o nylon a carica positiva con sopra una pila di fogli assorbenti. Per capillarit\`a la soluzione tende ad 
attraversare il gel fino ai fogli assorbenti. I sali trascinano i segmenti di DNA in verticale depositandoli sullo strato di nitrocellulosa in cui instaurano legami elettrostatici 
(carica negativa dei gruppi fosfato). Il foglio viene separato dal gel e vengono saturate le cariche positive con eterologo (DNA da salmone, processo di preibridazione). Il foglio viene
immerso in una soluzione marcata che ibrida con sequenze di DNA complementari identificandole. Il lavaggio della nitrocellulosa elimina sonde non ibridate e si fa una lastra 
fotografica che mette in evidenza dove la sonda ha legato il DNA genomico. 
\subsubsection{Western blot}
Assolutamente analogo al Southern blot ma viene utilizzato per visualizzare le proteine e il passaggio dal gel alla membrana di nitrocellulosa avviene grazie a una corrente elettrica e 
le proteine sono riconosciute attraverso anticorpi. 
\subsubsection{Northern blot}
Assolutamente analogo al Southern blot ma viene utilizzato per visualizzare l'RNA che viene denaturato a $70$ gradi e poi raffreddato in un bagno di ghiaccio con una piccola quantit\`a 
di agente denaturante per mantenerlo privo di strutture secondarie.  
\subsection{Tecniche di manipolazione del DNA}
Si intende per tecniche di manipolazione del DNA tecniche che permettono l'inserimento di geni o frammenti di geni all'interno di cellule. Viene normalmente fatto per esprimere o 
eliminare la produzione di una proteina.
\subsubsection{Sequenziamento del DNA}
Si intende per sequenziamento la determinazione della sequenza genomica per capire la proteina coinvolta in un meccanismo o rimuoverne una \`e necessario clonare il DNA, identificare le
zone promotrici che controllano la trascrizione di un gene e possibili mutazioni che possono avvenire.
\subsubsection{Clonaggio del DNA}
Si intende per clonaggio del DNA un insieme di metodi sperimentali che descrive l'assemblaggio di molecole ricombinanti e una serie di tecniche per ottenere pi\`u copie di una sequenza 
nucleotidica. Dopo averlo clonato si usa cDNA per produrre proteine di interesse di studio.
\subsubsection{Transfezione}
Si intende per transfezione l'introduzione di DNA esogeno in cellule eucaritiche, pu\`o essere transiente o stabile a seconda del tempo in cui il DNA transfettato rimane nel citoplasma
della cellula obiettivo. 
\paragraph{Vettori di clonaggio}
I vettori di clonazione sono elementi genetici di DNA che possono essere isolati e che si replicano in maniera autonoma rispetto al cromosoma batterico. Contengono un marcatore 
selezionabile, un gene che consente la selezione delle cellule che hanno effettivamente inglobato il vettore. Il DNA di interesse pu\`o essere clonato nel vettore e replicato in cellule
ospiti se \`e stato ben caratterizzato. 
\paragraph{Plasmidi}
I plasmidi sono i vettori di clonaggio maggiormente utilizzati, sono piccoli filamenti circolari di DNA a doppio filamento in grado di duplicarsi in maniera indipendente rispetto al 
genoma batterico che li ospita e in grado di spostarsi tra le cellule influendo sulla loro variabilit\`a genetica. Il cDNA viene normalmente clonato all'interno dei plasmidi specifici 
per l'espressione in cellule eucariote. Le componenti principali di un plasmide sono il promotore a monte specifico per la cellula eucariote, il segnale di poliadenilazione a valle, la
sequenza che contiene i siti di riconoscimento per le endonucleasi di restrizione e i geni marcatori selezionabili, sequenze necessarie per selezionare le cellule che hanno incorporato 
il plasmide. La trasduzione pu\`o essere transiente (il cDNA non viene integrato nel patrimonio genetico dell'ospite) e stabile (il cDNA si integra). In molti casi non \`e un aspetto
controllabile. La tecnica del CRISPR Cas9 utilizza il gene editing per permettere di modificare il gene endogeno (gi\`a presente). Si basa su metodi fisici (elettroporazione,
micropipette, microinieizione, biolistic, liposomi) o biologici (virus). 
\paragraph{Elettroporazione}
Nell'elettroporazione viene applicata una differenza di potenziale per un lasso di tempo per introdurre DNA o RNA nelle cellule. Le condizioni di impulso sono diverse per ogni tipo di 
cellula. Si formano pori all'interno della membrana che permettono il passaggio e che si richiudono quando la cellula riceve il DNA. \`E una tecnica utilizzata in vivo.
\paragraph{Microprecipitati}
Si combina il DNA con il calcio cloruro e una soluzione di fosfato in modo che si trovi come micro-precipitati contenenti DNA. Questi vengono messi a contatto con le cellule. La 
grandezza dipende dalla qualit\`a del DNA, dal pH della soluzione. I precipitati devono essere abbastanza piccoli in modo che possano essere endocitati e l'efficienza dipende dalla 
grandezza. \`E un metodo economico che permette il trasferimento di grandi quantit\`a di DNA, ma non tutte le cellule vengono trasfettate in maniera efficiente e il calcio fosfato 
potrebbe distruggere la membrana cellulare. 
\paragraph{Microiniezione}
Il DNA viene iniettato attraverso un apparato con capillari supersottili per bucare la membrana. \`E ottimo in cellule grandi.
\paragraph{Biolistics}
Viene utilizzato in cellule vegetali o organuli in quanto hanno parete spesse. Il DNA viene fissato su particelle d'oro e sparato sulla cellula. La camera di scoppio \`e un vetrino con
microparticelle d'oro su cui viene fissato il DNA. 
\paragraph{Liposomi}
\`E il metodo pi\`u usato: vescicole formate da lipidi contenenti il DNA si fondono con la membrana rilasciando il DNA presente al loro interno. Viene usata sia per la trasduzione 
stabile che transiente, ha alta efficienza e bassa tossicit\`a ma presenta alti costi.
\paragraph{Virus}
I virus vengono utilizzati come cavalli di Troia contenenti genoma modificato e inseriscono il DNA che si vuole inserire nella cellula. Il DNA da inserire deve essere piccolo. In questo
metodo influiscono la qualit\`a del DNA, qualit\`a dei reagenti, vitalit\`a delle cellule, pulizia e strumentazione, livelli di \ce{CO2} e umidit\`a, stabilit\`a e o tossicit\`a della
proteina ricombinante in quanto potrebbe diventare tossica per la cellula. 
\subsubsection{Immunoprecipitazione}
In questo metodo si usa un anticorpo capace di far precipitare una proteina specifica attraverso TAG sequence o sostanze che vi si legano specificatamente. \`E un fenomeno in cui in
una soluzione sono presenti antigeni e anticorpi diretti contro di essi: si possono formare aggregati  multipli che se abbastanza concentrati precipitano dalla soluzione. Svolge un ruolo
fondamentale la temperatura e il pH dell'ambiente e l'affinit\`a tra antigene e anticorpo. 
\subsection{Cristallografia a raggi X}
Si tratta di una tecnica che permette la visualizzazione della composizione e struttura degli atomi all'interno di un elemento che subisce un processo cristallizzazione. Attraverso il 
cristallo viene fatto passare un fascio di raggi X che interagiscono con gli atomi e vanno a colpire una lastra fotografica su cui si evidenziano una serie di ombre che formano una
figura che permette la ricostruzione della struttura tridimensionale attraverso diversi calcoli. Utilizzato per determinare la struttura del DNA da Watson e Crick. 
\subsection{Altre che non so????}

\section{Figure importanti e loro lavori}
\subsection{Miescher}
Miescher studia a Tubinger i meccanismi per il trasferimento delle informazioni ereditarie. Per la ricerca usa cellule di pus in quanto economiche e facilmente reperibili. Queste cellule
erano costituite per lo pi\`u da nuclei cellulari e reticolo endoplamsatico. Riesce pertanto a mettere a punto un metodo per estrarre e purificare le cellule senza danneggiarle e nota 
che riesce a ricavare pi\`u materiale durante la fase di divisione cellulare. Purifica infine una sostanza che chiama nucleina, molto acida e ricca in fosfato presente in grande 
quantit\`a. Successivamente utilizza lo sperma di salmone, ricco di DNA e mitocondri in quanto deve essere veloce e deve trasmettere i geni. Intuisce pertanto che la nucleina debba 
essere importante durante la divisione cellulare ma non la collega alla trasmissione ereditaria. 
\subsection{Mendel}
Mendel \`e considerato il padre della genetica. Comincia a fare ricerca sui piselli isolando linee pure (caratteristiche che permangono nel tempo) e le incrocia. Nota come alcuni 
caratteri non si manifestino nella progenie, ma che incrociando questa prima generazione ricompaiono in un rapporto di $3:1$ e li chiama rispettivamente alleli recessivi e dominanti. 
I primi compaiono solo se sono presenti entrambi gli alleli recessivi (si studiano attraverso i quadrati di Punnet). 
\subsection{Morgan}
Morgan riesce a isolare i cromosomi dalla Drosophila utilizzata a causa dei grandi cromosomi. Riesce pertanto a confermare che i geni sono depositati nei cromosomi del nucleo della
cellula, che sono organizzati in una lunga riga nei cromosomi e che tratti dipendenti tra di loro dipendono da geni in siti vicini. Determina inoltre il fenomeno del crossover. 
\subsection{Griffith e Avery}
Griffith studia lo \emph{sreptococcus pneumonia}, un batterio presente in due ceppi uno \emph{S} (smooth) patogeno e mortale per i topi e uno R (rough) non patogeno. Dopo aver 
inattivato il ceppo \emph{S} con il calore e iniettatolo in un topo questo non muore. Se invece al ceppo \emph{S} inattivato si mette a contatto il ceppo \emph{R} vivo e lo si inetta nel
topo questo muore. Successivamente Avery conferma che la molecola responsabile del trasferimento genetico orizzontale \`e il DNA isolando prima le diverse componenti molecolari e poi 
iniettandole singolarmente. 
\subsection{Hershey e Chase}
Hershey e Chase fecero esperimenti con i virus batteriofagi, marcando i fagi con sonde radioattive per osservare dove avviene la trasmissione genica. Marcarono un primo terreno con 
sonde a fosforo \textbf{\ce{^{32}P}} per il DNA e il secondo con zolfo \textbf{\ce{^{35}Z}} per le proteine. In base a quale elemento viene trasmesso lo si ritrover\`a nelle generazioni 
successive. Dopo aver inoculato \emph{escherichia coli} con i batteriofagi li si frullano e si staccano i virus dai batteri per evitare contaminazioni. Notarono come la maggior parte 
della marcatura era data dal fosforo radioattivo e pertanto il DNA doveva essere il responsabile della trasmissione genetica. 
\subsection{Watson e Crick}
Watson e Crick riuscirono a determinare la struttura del DNA attraverso la cristallografia a raggi X e a determinare il dogma centrale della biologia molecolare: l'informazione passa 
dal DNA all'RNA e arriva alle proteine. 
\subsection{Mello e Fire}
Mello e Fire furono in grado di evidenziare l'importanza dell'RNA e del microRNA scoprendo l'interferenza dell'RNA e la propriet\`a dell'RNA a doppio filamento di interferire e spegnere
l'espressione genica. Si dice microRNA un RNA tra i $20$ e i $22$ nucleotidi a singolo filamento. 

\section{Cellula e genomi}
La biologia molecolare \`e lo studio della cellula e dei suoi componenti in vivo. Si definisce la vita come un sistema chimico auto-sostenibile capace di subire un'evoluzione di tipo 
Darwiniano. Le cellule e gli organismi sono in grado di adattarsi a una grande variet\`a di ambienti: i batteri che vivono a temperature elevate come i solfobatteri che utilizzano lo 
zolfo come elemento per sopravvivere (esprimono TAQ polimerasi) e animali in grado di vivere a temperature molto basse come il pesce ghiaccio, privo di pigmentazione con branchie molto
grandi. Il freddo \`e un problema per quanto riguarda la circolazione e questi pesci sono privi di globuli rossi e hanno adottato un sistema di trasporto dell'ossigeno diverso. I globuli
rossi sono ricchi di emoglobina, una molecola costituita da due dimeri di $\alpha$ e $\beta$ globulina con un gruppo eme. La sua funzione \`e quella di scambiare \textbf{\ce{CO2}} con 
\textbf{\ce{O2}}. Le modifiche subite di queste animali sono dovute in quanto i fluidi corporei aumentano la viscosit\`a a basse temperature. I pesci riducono pertanto il numero di 
globuli rossi e possiedono una serie di cambiamenti che permettono al sangue di rimanere fluido a basse temperature modificando proteine che agiscono da antigelo. Si \`e assistito anche 
a modifiche nella sequenza dei microtubuli in quanto negli altri organismi diventano instabili e si polarizzano a temperature inferiori ai $10$ gradi cambiando la forma cellulare e 
eliminando l'adesione alla superficie. La modifica della composizione amminoacidica dei microtubuli li rende pi\`u resistenti alle basse temperature. Per facilitare gli scambi i 
capillari e le branchie si sono ingranditi e non sono presenti scaglie. 
\subsection{Caratteristiche universali delle cellule}
\subsubsection{Dinamicit\`a}
\`E il primo aspetto che determina lo stato della cellula che contiene diversi organelli dinamici. Quelli pi\`u trasportati sono i mitocondri, le fabbriche di ATP che si dividono e 
fondono che tendono a viaggiare molto velocemente attraverso proteine motrici. Il movimento avviene lungo i microtubuli. Esiste il trasporto di vescicole e membrane. La caratteristica
principale tra una cellula sana e una morta \`e pertanto la dinamicit\`a. 
\subsubsection{Informazione}
Le cellule contengono informazione sotto forma di DNA ($3$ miliardi di nucleotidi per gli esseri umani). Si pu\`o legare la quantit\`a di informazione con la superiorit\`a dal punto di 
vista evolutivo. In realt\`a il concetto di perfezione e quantit\`a di informazione non vanno di pari passo: l'ameba contiene molta pi\`u informazione di una cellula umana. 
\subsubsection{Riproduzione}
La riproduzione caratterizza la maggior parte delle cellule e permette alle cellule di trasmettere le informazioni contenute all'interno di DNA. Nel momento in cui il DNA si duplica il 
processo non \`e perfetto e gli errori stanno alla base delle mutazioni che stanno alla base dell'evoluzione. Il concetto di mutazione non \`e prettamente negativo in quanto ci sono 
mutazioni che tendono a migliorare l'efficienza dell'organismo in determinate condizioni.
\subsubsection{Struttura}
Tutte le cellule sono isolate dall'esterno da una membrana cellulare e a volte da una parete. Se per i procarioti non si trovano sottostrutture negli eucarioti sono presenti organelli
circondati anch'essi da membrane lipidiche che si differenziano per funzione.
\paragraph{Membrana}
La membrana plasmatica \`e una barriera selettiva che permette alla cellula di concentrare i nutrienti conservando i prodotti delle sintesi ed espellere i materiali di scarto. Le 
propriet\`a della membrana sono dovute alla composizione di fosfolipidi, molecole anfipatiche con testa polare e code apolari che in ambiente acquoso tendono a disporsi come un doppio
strato creando una barriera.
\paragraph{Struttura della cellula eucariote}
Definita da una membrana plasmatica che la isola dall'esterno, contiene un ambiente citoplasmatico in cui si trovano i suoi organelli.
\subparagraph{Nucleo}
Delimitato da due strati di membrane nucleari e da una lamina interna, contiene l'informazione genica della cellula. La membrana nucleare contiene innumerevoli porine, 
canali che permettono il passaggio di proteine ed RNA da e verso il nucleo. Nel nucleo si pu\`o trovare un nucleolo la cui forma e posizione dipende dallo stato del ciclo cellulare, 
\`e la zona dove avviene l'assemblaggio e la trascrizione degli rRNA.
\subparagraph{Reticolo endoplasmatico}
In contatto con la membrana nucleare \`e un sistema di cisterne a membrana che si divide in liscio e rugoso. L'ER liscio \`e una riserva di ioni calcio mentre quello rugoso possiede
dei ribosomi che si occupano della sintesi delle proteine di membrana e quelle secrete.
\subparagraph{Apparato di Golgi}
In continuit\`a funzionale ma non fisica con l'ER si occupa di ricevere le vescicole che gli invia e di modificare le proteine in esse contenute (glicosilazione) per poi inviarle alla
destinazione finale. 
\subparagraph{Mitocondri}
Possiedono un proprio DNA e apparato di traduzione con pochi geni: molte delle proteine necessarie sono codificate nel nucleo e trasportate successivamente. Producono efficientemente
ATP in presenza di ossigeno. Possono dividersi o fondersi tra di loro. Sono circondati da due membrane invaginate in cui avviene sintesi e trasporto di ATP. Nelle cellule vegetali sono
affiancati dai cloroplasti per la produzione di glucosio.
\subparagraph{Lisosomi} 
Sede della degradazione delle proteine.
\subparagraph{Citoscheletro} 
Formato da microtubuli, filamenti di actina e intermedi conferisce motilit\`a alla cellula permettendole di esperire l'ambiente e dandole al contempo una certa rigidit\`a strutturale. 
\subsubsection{Funzione}
Tutte le cellule conservano le loro informazioni come molecole a doppio filamento di DNA, catene polimeriche accoppiate senza ramificazioni formate da quattro monomeri. La catena
codifica l'informazione genica. L'ordine di lettura \`e determinato dalla polarit\`a della catena e i monomeri formano legami specifici con il complementare formando un doppio filamento
tenuto insieme da legami a idrogeno. La doppia catena permette la replicazione semi-conservativa e il passaggio dell'informazione nelle cellule figlie. La cellula esprime le informazioni
contenute nel DNA producendo da esso RNA e proteine. Durante la trascrizione il DNA viene letto producendo catene singole di RNA che possono assumere strutture secondarie. Processi
successivi alla trascrizione aumentano il grado di variabilit\`a. Alcuni mRNA vengono poi tradotti in proteine, catene polimeriche lineari di amminoacidi che assumono complesse strutture
spaziali che conferiscono loro la funzione di catalizzatori specifici delle reazioni biochimiche necessarie alla vita della cellula. Le proteine assumono inoltre funzione strutturale e 
di segnalazione. Si noti come tutte le cellule si possono assimilare a fabbriche biochimiche che impiegano le stesse strutture molecolare di base. 
\subsection{Differenze principali tra le forme di vita}
La classificazione delle cose viventi dipende da somiglianze comuni che suggeriscono un antenato comune pi\`u recente. L'albero della vita si pu\`o dividere in tre rami principali: 
procarioti, classificati in termini della loro biochimica e necessit\`a nutrizionali, archea ed eucarioti. La differenza principale tra procarioti ed eucarioti \`e che i secondi 
contengono un nucleo mentre nei primi il DNA si trova nel citoplasma attaccato alla parete cellulare (non presente negli eucarioti). Gli archea si trovano a met\`a strada tra eucarioti
e procarioti. 
\subsubsection{Mutazioni}
L'informazione genica pu\`o cambiare a seguito della replicazione o durante la vita nella cellula. Queste mutazioni possono garantirle un vantaggio rispetto alle sue simili, uno 
svantaggio o non cambiare nulla. Sono queste mutazioni ad aver causato la differenziazione delle forme viventi e la loro proliferazione in ambienti cos\`i diversi tra di loro. Non tutti 
i geni mutano con la stessa frequenza: tipicamente quelli fondamentali alla vita tendono ad essere altamente conservati. Questo in quanto l'ambiente causa alla cellula una pressione 
selettiva andando ad eliminare rapidamente le mutazioni svantaggiose. Si intende per gene una sequenza, un segmento di DNA che corrisponde a una proteina o a una serie di varianti di 
trascritti con funzione regolatoria o strutturale. Molte mutazioni possono cambiare il fenotipo dell'organismo. Le mutazioni di geni che mantengono comunque struttura o funzione simile
vengono poi raggruppate in famiglie. 
\paragraph{Geni ortologhi}
Supponendo di avere un gene $G$ di un organismo ancestrale sue mutazioni possono dare origine a due geni distinti $G_a$ e $G_b$ nelle due specie diverse che gli succedono.
\paragraph{Geni paraloghi}
Si supponga di avere un gene $G$ di un organismo ancestrale: questo pu\`o andare incontro a duplicazione e i due possono divergere generando cos\`i due geni $G_1$ e $G_2$ paraloghi.
