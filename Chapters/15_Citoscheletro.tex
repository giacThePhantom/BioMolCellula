\chapter{Il citoscheletro}
In modo che la cellula funzioni correttamente deve organizzarsi nello spazio e interagire meccanicamente con le altre nell'ambiente: devono avere la forma corrette, essere robuste e 
possedere una propria struttura interna. Le funzioni spaziali e meccaniche che forniscono queste propriet\`a alla cellula (che \`e capace di modificare) dipendono dal citoscheletro, un 
sistema di filamenti. Le sue funzioni dipendono da tre famiglie di filamenti proteici: i filamenti di actina, i microtubuli e i filamenti intermedi. Ogni tipo ha propriet\`a meccaniche, 
dinamica e ruoli biologici distinti ma condividono delle caratteristiche fondamentali. Normalmente agiscono insieme in modo da dare alla cellula le propriet\`a precedenti. 
\section{Funzione e origine del citoscheletro}
I tre filamenti del citoscheletro sono responsabili per diversi aspetti dell'organizzazione spaziale della cellula e delle sue propriet\`a meccaniche. I filamenti di actina determinano 
la forma della superficie cellulare e sono necessari per la sua locomozione oltre a guidare la sua separazione durante la divisione cellulare. I microtubuli determinano la posizione 
degli organelli, direzionano il trasporto intracellulare e formano il mandrino mitotico che segrega i cromosomi durante la divisione cellulare. I filamenti intermedi forniscono forza
meccanica. Tutti questi filamenti interagiscono con centinaia di proteine accessorio che regolano e uniscono i filamenti con altre componenti cellulari e tra di loro. Sono essenziali 
per l'assemblaggio controllato dei filamenti in posti particolari e includono le proteine motrici che convertono l'energia dell'ATP in forza meccanica che muove gli organelli o i 
filamenti. 
\subsection{I filamenti del citoscheletro si adattano per formare strutture dinamiche o stabili}
I sistemi del citoscheletro sono dinamici e adattabili, in modo che possano cambiare o persistere in base alle necessit\`a. Le componenti macromolecolari sono in un continuo stato di 
flusso in modo che un riordinamento richieda poca energia in pi\`u al cambiare delle condizioni. La regolazione del comportamento dinamico e assemblaggio dei filamenti permette la
costruzione di strutture varie. I filamenti di actina sottostanno alla membrana plasmatica, fornendo forza e forma al bistrato lipidico oltre a formare proiezioni sulla superficie
cellulare come strutture dinamiche come lamellipodia e filopodia che la cellula usa per esplorare il territorio e muoversi. Strutture pi\`u stabili permettono alla cellula di unirsi
a un substrato e permette alle cellule muscolari di contrarsi. I microtubuli possono riordinarsi per formare un mandrino mitotico bipolare durante la divisione cellulare e cilia, che
funzionano come fruste motili o dispositivi sensoriale sulla superficie cellulare, oltre a percorsi per il trasporto di materiali. I filamenti intermedi formano una gabbia protettiva per
il DNA cellulare e nel citosol sono avvolti in cavi in modo da formare forti appendici. Durante la divisione cellulare avviene una drammatica e rapida riorganizzazione del citoscheletro:
dopo che i cromosomi si sono replicati il vettore di microtubili interfase che si diffonde attraverso il citoplasma \`e riconfigurato nel mandrino mitotico che segrega le due coppie di 
ogni cromosoma nei nuclei delle figlie. Allo stesso tempo strutture di actina specializzate si riordinano in modo che la cellula smette di muoversi e assume una dimensione pi\`u sferica.
L'actina e la miosina formano una cintura intorno al centro della cellula (anello contrattile) che la costringe fino a che si separa. Quando la divisione \`e completa il citoscheletro 
delle due cellule figlie si riordina nelle strutture interfase. 
\subsection{Il citoscheletro determina l'organizzazione cellulare e la sua polarit\`a}
Nelle cellule con una morfologia stabile e differenziata gli elementi dinamici del citoscheletro devono anche fornire strutture stabili per l'organizzazione cellulare: su cellule 
epiteliali negli intestini e nei poloni protrusioni basate sul citoscheletro come microvilli e cilia vengono mantenute durante tutta la vita della cellula. Oltre a questo il 
citoscheletro \`e responsabile della polarit\`a cellulare in modo che le cellule siano capaci di distinguere le diverse terminazioni, cosa mantenuta per l'intera vita della cellula. 
\subsection{I filamenti si assemblano da subunit\`a proteiche che impartiscono propriet\`a fisiche e dinamiche specifiche}
I filamenti possono allungarsi lungo tutta la dimensione della cellula a causa del fatto che sono costruiti assemblando grandi numeri di piccole subunit\`a che possono diffondersi
rapidamente nel citosol in modo che le reorganizzazioni strutturali siano rapide. I filamenti di actina e i microtubuli sono costruiti da subunit\`a compatte e globulari (rispettivamente
subunit\`a di actina e di tubulina) mentre quelli intermedi da subunit\`a pi\`u piccole allungate e fibrose. Tutti formano come assemblaggi elicali di queste subunit\`a che si associano
da sole usando una combinazione di contatti tra terminazioni e tra lati. La diversit\`a e forza delle interazioni fornisce diversa stabilit\`a e diverse propriet\`a meccaniche per ogni
filamento. Le subunit\`a sono tenute insieme da interazioni non covalenti che ne permettono un rapido assemblaggio e disassemblaggio. Le subunit\`a di filamenti di actina e dei 
microtubuli sono asimmetriche e si legano tra di loro testa a coda con lo stesso orientamento, polarit\`a che causa diversi comportamenti in ogni terminazione. Le subunit\`a sono inoltre
enzimi che catalizzano l'idrolisi di un nucleoside trisosfato, rispettivamente ATP e GTP la cui energia permette rapidi rimodellamenti. Controllandone l'assemblaggio la cellula controlla
le propriet\`a dinamiche e polari dei filamenti per generare forza in direzione specifica. Le subunit\`a dei filamenti intermedi sono simmetriche e non catalizzano l'idrolisi dei 
nucleotidi
