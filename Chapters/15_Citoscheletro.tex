\chapter{Il citoscheletro}
In modo che la cellula funzioni correttamente deve organizzarsi nello spazio e interagire meccanicamente con le altre nell'ambiente: devono avere la forma corrette, essere robuste e 
possedere una propria struttura interna. Le funzioni spaziali e meccaniche che forniscono queste propriet\`a alla cellula (che \`e capace di modificare) dipendono dal citoscheletro, un 
sistema di filamenti. Le sue funzioni dipendono da tre famiglie di filamenti proteici: i filamenti di actina, i microtubuli e i filamenti intermedi. Ogni tipo ha propriet\`a meccaniche, 
dinamica e ruoli biologici distinti ma condividono delle caratteristiche fondamentali. Normalmente agiscono insieme in modo da dare alla cellula le propriet\`a precedenti. 
\section{Funzione e origine del citoscheletro}
