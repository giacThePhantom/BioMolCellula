\chapter{Il citoscheletro}
In modo che la cellula funzioni correttamente deve organizzarsi nello spazio e interagire meccanicamente con le altre nell'ambiente: devono avere la forma corrette, essere robuste e 
possedere una propria struttura interna. Le funzioni spaziali e meccaniche che forniscono queste propriet\`a alla cellula (che \`e capace di modificare) dipendono dal citoscheletro, un 
sistema di filamenti. Le sue funzioni dipendono da tre famiglie di filamenti proteici: i filamenti di actina, i microtubuli e i filamenti intermedi. Ogni tipo ha propriet\`a meccaniche, 
dinamica e ruoli biologici distinti ma condividono delle caratteristiche fondamentali. Normalmente agiscono insieme in modo da dare alla cellula le propriet\`a precedenti. 
\section{Funzione e origine del citoscheletro}
I tre filamenti del citoscheletro sono responsabili per diversi aspetti dell'organizzazione spaziale della cellula e delle sue propriet\`a meccaniche. I filamenti di actina determinano 
la forma della superficie cellulare e sono necessari per la sua locomozione oltre a guidare la sua separazione durante la divisione cellulare. I microtubuli determinano la posizione 
degli organelli, direzionano il trasporto intracellulare e formano il mandrino mitotico che segrega i cromosomi durante la divisione cellulare. I filamenti intermedi forniscono forza
meccanica. Tutti questi filamenti interagiscono con centinaia di proteine accessorio che regolano e uniscono i filamenti con altre componenti cellulari e tra di loro. Sono essenziali 
per l'assemblaggio controllato dei filamenti in posti particolari e includono le proteine motrici che convertono l'energia dell'ATP in forza meccanica che muove gli organelli o i 
filamenti. 
\subsection{I filamenti del citoscheletro si adattano per formare strutture dinamiche o stabili}
I sistemi del citoscheletro sono dinamici e adattabili, in modo che possano cambiare o persistere in base alle necessit\`a. Le componenti macromolecolari sono in un continuo stato di 
flusso in modo che un riordinamento richieda poca energia in pi\`u al cambiare delle condizioni. La regolazione del comportamento dinamico e assemblaggio dei filamenti permette la
costruzione di strutture varie. I filamenti di actina sottostanno alla membrana plasmatica, fornendo forza e forma al bistrato lipidico oltre a formare proiezioni sulla superficie
cellulare come strutture dinamiche come lamellipodia e filopodia che la cellula usa per esplorare il territorio e muoversi. Strutture pi\`u stabili permettono alla cellula di unirsi
a un substrato e permette alle cellule muscolari di contrarsi. I microtubuli possono riordinarsi per formare un mandrino mitotico bipolare durante la divisione cellulare e cilia, che
funzionano come fruste motili o dispositivi sensoriale sulla superficie cellulare, oltre a percorsi per il trasporto di materiali. I filamenti intermedi formano una gabbia protettiva per
il DNA cellulare e nel citosol sono avvolti in cavi in modo da formare forti appendici. Durante la divisione cellulare avviene una drammatica e rapida riorganizzazione del citoscheletro:
dopo che i cromosomi si sono replicati il vettore di microtubili interfase che si diffonde attraverso il citoplasma \`e riconfigurato nel mandrino mitotico che segrega le due coppie di 
ogni cromosoma nei nuclei delle figlie. Allo stesso tempo strutture di actina specializzate si riordinano in modo che la cellula smette di muoversi e assume una dimensione pi\`u sferica.
L'actina e la miosina formano una cintura intorno al centro della cellula (anello contrattile) che la costringe fino a che si separa. Quando la divisione \`e completa il citoscheletro 
delle due cellule figlie si riordina nelle strutture interfase. 
\subsection{Il citoscheletro determina l'organizzazione cellulare e la sua polarit\`a}
Nelle cellule con una morfologia stabile e differenziata gli elementi dinamici del citoscheletro devono anche fornire strutture stabili per l'organizzazione cellulare: su cellule 
epiteliali negli intestini e nei poloni protrusioni basate sul citoscheletro come microvilli e cilia vengono mantenute durante tutta la vita della cellula. Oltre a questo il 
citoscheletro \`e responsabile della polarit\`a cellulare in modo che le cellule siano capaci di distinguere le diverse terminazioni, cosa mantenuta per l'intera vita della cellula. 
\subsection{I filamenti si assemblano da subunit\`a proteiche che impartiscono propriet\`a fisiche e dinamiche specifiche}
I filamenti possono allungarsi lungo tutta la dimensione della cellula a causa del fatto che sono costruiti assemblando grandi numeri di piccole subunit\`a che possono diffondersi
rapidamente nel citosol in modo che le reorganizzazioni strutturali siano rapide. I filamenti di actina e i microtubuli sono costruiti da subunit\`a compatte e globulari (rispettivamente
subunit\`a di actina e di tubulina) mentre quelli intermedi da subunit\`a pi\`u piccole allungate e fibrose. Tutti formano come assemblaggi elicali di queste subunit\`a che si associano
da sole usando una combinazione di contatti tra terminazioni e tra lati. La diversit\`a e forza delle interazioni fornisce diversa stabilit\`a e diverse propriet\`a meccaniche per ogni
filamento. Le subunit\`a sono tenute insieme da interazioni non covalenti che ne permettono un rapido assemblaggio e disassemblaggio. Le subunit\`a di filamenti di actina e dei 
microtubuli sono asimmetriche e si legano tra di loro testa a coda con lo stesso orientamento, polarit\`a che causa diversi comportamenti in ogni terminazione. Le subunit\`a sono inoltre
enzimi che catalizzano l'idrolisi di un nucleoside trisosfato, rispettivamente ATP e GTP la cui energia permette rapidi rimodellamenti. Controllandone l'assemblaggio la cellula controlla
le propriet\`a dinamiche e polari dei filamenti per generare forza in direzione specifica. Le subunit\`a dei filamenti intermedi sono simmetriche e non catalizzano l'idrolisi dei 
nucleotidi pur riuscendo ad avere una rapida dissociazione. Per fornire forza e adattibilit\`a i microtubuli si assemblano da 13 protofilamenti, stringhe lineari che si associano tra di 
loro formando un cilindro vuoto, molto pi\`u resistente a rotture centrali e pi\`u adatto alla perdita e al guadagno di subunit\`a alle terminazioni. I filamenti sono tenuti insieme
da legami non covalenti e interazioni idrofobiche il cui tipo e luogo dipende dal tipo di filamento: quelli intermedi si assemblano formando forti contatti laterali tra bobine di 
$\alpha$-eliche. 
\subsection{Proteine accessorie e motrici regolano i filamenti del citoscheletro}
La cellula regola la lunghezza, stabilit\`a, forma e numero dei filamenti regolando come si attaccano tra di loro e con le altre componenti della cellula in modo che i filamenti possano
formare una grande variet\`a di strutture di alto livello. Modifiche covalenti dirette dei filamenti regolano alcune delle propriet\`a, ma la maggior parte della regolazione \`e svolta
da centinaia di proteine accessorie che determinano la distribuzione spaziale e il comportamento dinamico del filamenti, convertendo segnali in azioni citosiliche. Queste si legano ai
filamenti per determinare i siti di assemblaggio di nuovi filamenti, per regolare il partizionamento dei polimeri, per cambiare la cinetica dell'assemblaggio, per recuperare energia e 
per unire i filamenti tra di loro e ad altri compartimenti. Questi processi avvengono sotto il controllo di segnali extra e intracellulari. Tra le proteine che si associano al 
citoscheletro si trovano le proteine motrici che si legano a un filamento polarizzato e utilizzano ripetuti cicli di idrolisi dell'ATP per muoversi lungo il filamento. Molte di queste
trasportano organelli, mentre altre causano tensione tra i filamenti generando la forza necessaria per il movimento. Sono simili alle polimerasi o elicasi nel senso che si muovono lungo
una pista lineare. 
\subsection{La divisione e organizzazione di cellule batteriche dipende da omologhi di proteine citoscheletrali eucariotiche}
\section{Actina e proteine leganti all'actina}
Il citoscheletro actina svolge diverse funzioni e ogni subunit\`a (G-actina o actina globulare) \`e formata da un polipeptide lungo 375 amminoacidi associato con una molecola di ATP o 
ADP. Nei vertebrati si trovano tre isoformi dell'actina: $\alpha$, $\beta$ e $\gamma$ con distinte funzioni: la prima nelle cellule muscolari, le altre in tutte le altre.
\subsection{Le subunit\`a di actina si assemblano testa a coda per creare filamenti flessibili e polari}
Le subunit\'a di actina si assemblano testa a coda per formare una elica a destra e una struttura di $8 nm$ detta actina filamentosa o F-actina, filamenti polari con diverse 
terminazioni: una che cresce lentamente o terminazione meno e una pi\`u velocemente o pi\`u. La prima si dice anche a punta e l'altra spinata. Le subunit\`a sono posizionate con la 
fessura legante i nucleotidi verso la terminazione meno. I filamenti individuali sono flessibili, cosa ridotta da proteine che uniscono e legano incrociandoli creando strutture pi\`u
grandi e rigide.
\subsection{La nucleazione \`e il passo limitante nella formazione dei filamenti di actina}
Il regolamento della formazione dei filamenti di actina \`e importante per controllare la forma e il movimento cellulare. Piccoli oligomeri di actina si possono assemblare 
spontaneamente, ma sono instabili e si disassemblano prontamente. Per creare strutture durature le subunit\`a si devono assemblare in un aggregato iniziale o nucleo stabilizzato da
contatti tra subunit\`a multipli che pu\`o allungarsi rapidamente attraverso l'aggiunta di altre subunit\`a nel processo di nucleazione del filamento. L'instabilit\`a dell'actina 
piccola crea una barriera cinetica alla polimerizzazione che causa una fase di ritardo in cui pochi aggregati riescono a formare la transizione ad una forma pi\`u stabile che poi porta
alla fase di allungamento rapida in cui le subunit\`a sono aggiunte velocemente. Quando la concentrazione di monomeri diminuisce il sistema raggiunge uno stato di equilibrio in cui 
il tasso di aggiunta \`e uguale a quello di dissociazione. La concentrazione di subunit\`a libere a questo punto \`e detta concentrazione critica $C_C$.
\subsection{Filamenti di actina hanno due terminazioni distinte che crescono a tassi diversi}
A causa delle differenze alle terminazioni i filamenti hanno comportamenti diversi alle terminazioni: alla terminazione pi\`u sono molto maggiori i tassi di associazione e dissociazione.
Hanno comunque la stessa affinit\`a per le subunit\`a. La dinamica e polarit\`a dei filamenti viene sfruttata per generare lavoro meccanico: essendo che l'allungamento genera energia
quando la concentrazione di subunit\`a eccede la concentrazione critica pu\`o essere accoppiato con un movimento di un carico attaccato. 
\subsection{L'idrolisi dell'ATP tra filamenti di actina porta a routine in uno stato di equilibrio}
L'actina pu\`o idrolizzare ATP, cosa che avviene pi\`u rapidamente quando si trova in filamenti. Poco dopo l'idrolisi il gruppo fosfato \`e rilasciato e l'ADP rimane nella subunit\`a e 
possono pertanto esistere forme di actina con legato ATP (forme T) e con legato ADP (forme D). L'energia liberata dall'idrolisi viene conservata nel polimero aumentando l'energia libera
per la degradazione di esso alla stessa concentrazione rispetto alla forma T. La concentrazione di subunit\`a in forma T nel filamento aumenta nel tempo. Nel caso in cui si trovano 
concentrazioni intermedie \`e possibile che l'addizione di subunit\`a sia pi\`u veloce dell'idrolisi alla terminazione pi\`u e pi\`u lenta alla terminazione meno in modo che 
il filamento subisca una netta aggiunta di subunit\`a alla terminazione pi\`u, mentre perde subunit\`a dalla terminazione meno. Alla concentrazione di equilibrio l'aggiunta e la 
rimozione si eguagliano e le subunit\`a ciclano rapidamente tra lo stato libero e quello filamentoso che richiede un consumo costante di energia attraverso idrolisi di ATP. L'azione di 
proteine che alterano la stabilit\`a della polimerizzazione del filamento mostrano come la funzione di esso dipende dall'equilibrio dinamico tra filamento e monomero. 
\subsection{Proteine leganti all'actina influenzano la dinamica e l'organizzazione del filamento}
In una cellula il comportamento dell'actina \`e regolato da proteine accessorio che possono legarsi al monomero o al filamento. Questo avviene grazie al controllo spaziale e temporale
di disponibilit\`a di monomeri, allungamento del filamento, nucleazione e depolimerizzazione. 
\subsection{La disponibilit\`a dei monomeri controlla l'assemblaggio dei filamenti di actina}
Si nota come nelle cellule non muscolari solo una piccola parte dell'actina polimerizza a causa di proteine che si legano al monomero e rendono il processo meno favorevole. La pi\`u 
abbondante \`e la timosina che si lega ai monomeri portandoli ad uno stato bloccato in cui non si possono associare con le terminazioni del filamento o idrolizzare o scambiare il 
nucleotide legato. Il reclutamento di questi monomeri avviene grazie alla profilina che si lega al monomero di actina dal lato opposto della fessura legante all'ATP bloccando il lato
del monomero che si associa alla terminazione meno lasciando esposto il legame per la terminazione pi\`u. Quando polimerizza la profilina si stacca, lasciandola libera di competere con
la timosina per la prossima subunit\`a. Diversi meccanismi regolano la profilina come la fosforilazione e il legame al fosfolipide inositolo in modo da definire i siti di azione di 
questa proteina. 
\subsection{Fattori di nucleazione dell'actina accelerano la polimerizzazione e generano filamenti dritti o ramificati}
Un secondo prerequisito per la polimerizzazione \`e la nucleazione dei filamenti. Le proteine che contengono motivi di legame per i monomeri di actina si uniscono in tandem per
mediare i meccanismi di nucleazione. Tali proteine uniscono diverse subunit\`a per formare un seme. La nucleazione \`e catalizzata da il complesso Arp 2/3 o formina. Il primo \`e un
complesso formato da due proteine imparentate a actina o ARP, che nucleano i filamenti dalla terminazione meno permettendo una rapida crescita alla terminazione pi\`u. Il complesso 
pu\`o anche unirsi al lato di un altro filamento creando una struttura ramificata. Le formine sono proteine dimeriche che nucleano la crescita di filamenti lunghi e non ramificati che 
possono essere uniti da altre proteine per formare unioni parallele. Ogni subunit\`a di formina ha un sito di legame per l'actina monomerica e il dimero nuclea la polimerizzazione 
catturando due monomeri. Mentre il filamento cresce il dimero formina rimane associato alla terminazione pi\`u permettendo l'addizione di nuove subunit\`a. Tale crescita \`e fortemente
aumentata da monomeri associati con profilina. La nucleazione avviene principalmente alla membrana plasmatica dove si forma la corteccia cellulare e i filamenti di actina in questa 
regione determinano forma e movimento della superficie cellulare. 
\subsection{Proteine che si legano a filamenti di actina alterano la dinamica del filamento}
Il comportamento dei filamenti \`e regolato da due classi di proteine di legame: quelle che si legano ai lati di un filamento e quelle che si legano alle terminazioni. Le proteine 
leganti ai lati includono la tropomiosina, una proteina che lega sei o sette subunit\`a adiacenti lungo due scanalature del filamento elicale. Oltre a stabilizzare e irrigidire il 
filamento tale legame pu\`o impedire l'interazione con altre proteine. Un filamento di actina che smette di crescere e non \`e stabilizzato nella cellula depolimerizza rapidamente una
volta che le molecole di actina idrolizzano il loro ATP. Il legame con proteine di incappucciamento o proteine CapZ stabilizzano il filamento alla terminazione pi\`u rendendola inattiva
e riducendo il tasso di crescita e depolimerizzazione. Alla terminazione meno un filamento di actina pu\`o essere incappucciato dal complesso Arp 2/3. La tropomodulina si lega alla
terminazione meno di filamenti incapsulati e stabilizzati da tropomiosina. PU\`o incappucciare filamenti puri transientemente. Proteine che si legano al lato dei filamenti devono 
incapsulare l'intero filamento e pertanto essere presenti ad alte concentrazioni. 
\subsection{Proteine di separazione regolano la depolimerizzazione del filamento di actina}
Un altro meccanismo di regolazione dipende da proteine capaci di rompere un filamento in filamenti pi\`u piccoli generando un grande numero di nuove terminazioni. Il destino di queste
dipende dalla presenza di altre proteine accessorio. Sotto alcune condizioni le terminazioni nucleano l'allungamento, sotto altre la separazione promuove la depolimerizzazione. Oltre
a questo la separazione rende il citoplasma pi\`u fluido. Una classe di tali proteine \`e la superfamiglia di gelsoline che sono attivate da alti livelli di \ce{Ca^{2+}} citoslico. 
Le gelsoline interagiscono con il lato del filamento e contengono sottodominiche si legano a un sito esposto sulla superficie e uno nascosto tra subunit\`a adiacenti. Secondo un
modello la gelsolina si lega al lato del filamento fino a che una fluttuazione termica crea un piccolo spazio tra subunit\`a vicine in cui la gelsolina si inserisce per rompere 
il filamento. Dopo la rottura la gelsolina rimane attaccata al filamento e incappuccia la nova terminazione. In altra proteina destabilizzatrice \`e la cofilina che si lega lungo 
il filamento forzando un suo avvolgimento causando stress meccanico in modo che movimenti termici lo rompano pi\`u facilmente. La cofilina si lega principalmente a filamenti contenenti
ADP in modo che i nuovi filamenti siano resistenti ad essa. Il filamento pu\`o essere protetto da essa grazie al legame con tropomiosina. 
\subsection{Insiemi di filamenti di actina di alto livello influenzano le propriet\`a meccaniche e la segnalazione della cellula}
I filamenti di actina nelle cellule animali sono organizzati in diversi tipi di insiemi: reti dendriditche, fasci e reti simili a ragnatele. Le strutture sono iniziate dall'azione
di distinte proteine nucleatrici: i filamenti delle reti dendritiche sono nucleate dal complesso Arp 2/3, mentre i fasci da formine. L'organizzazione strutturale delle reti dipende
da proteine accessorio specializzate: strutture sono assemblate dalle proteine infascianti che uniscono i filamenti in insiemi paralleli e proteine formanti gel che uniscono due 
filamenti insieme ad un grande angolo creando una mesh pi\`u allentata. Entrambe le classi hanno siti di legame per l'actina simili formate da catene singole o dimeri. La separazione
e ordinamento di questi domini determina il tipo di struttura di actina che la proteina forma. Ogni tipo di proteine infasciante determina quali altre molecole possono interagire con
i filamenti di actina uniti. La miosina II \`e una proteina motrice che permette a fibre di stress e ad altri insiemi contrattili di contrarsi. L'impacchettamento di filamenti causato
da fimbrina esclude la miosina rendendo il fascio non contrattile. La $\alpha$-actinina unisce filamenti polarizzati in fasci allentati permettendo il legame di miosina. Queste due 
proteine sono mutualmente esclusive. Le proteine che legano filamenti possono avere connessioni flessibili o piegate tra i domini di legame, permettendo la formazione di reti o gel. Un
esempio di queste \`e la spectrina. 
\section{Miosina e actina}
