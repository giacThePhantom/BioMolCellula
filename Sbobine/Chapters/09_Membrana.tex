\chapter{Membrana}

Non ha registrato La membrana \`e estermamente fluida, cosa che dipende dalla composiozne dei lipidi, in grado di cambiare le caratteristiche biofisiche in grado ai lipidi di cui 
\`e costituita, una membrana di una sola classe passa da una fase semisolida ad una pi\`u solida in uno stato di gel a una determinata temperatura modificata in base alla composizione
dei fosfolipidi. Una membrana costituita da catene idrocarburica insature ha una transizone di fase ad una temperatura pi\`u basse rispetto a fosfoilipidi con catene idrocarburiche
sature, importante in quanto organismi tendono ad andare incontro a variazioni della temperatura esterna che influiscono sulla transizione di fase modificando la composizione delle 
membrane. Il colesterolo pu\`o modificare a questo tipo di propriet\`a rendendo le membrane molto meno fluide e fornisce aumenta la selettivit\`a della membrana nei confronti di 
determinate molecole. Le specie lipidiche nelle membrane plasmatiche sono colesterolo, fosfatidiletanolammina, fosfatidilserina, fosfatidilcolina, sfingomielina, glicolipidi e altri. 
L'altra componente importante dei lipidi di membrana sono i glicolipidi con uno zucchero associato al lipide. Si pensa che la membrana fosse omogenea dal punto di vista di caratteristiche
biofisiche, ma i lipidi tendon a formare domini con caratteristiche particolare dette zattere lipidiche all'interno delle quali sembrano avere un ruolo funzionale mportante in quanto 
permettono il reclutamento e aggregazione di proteine creando domini specifici all'interno della membrana stessa. Le zattere lipidiche sono zone di disomogeneit\`a della composizione 
lipidica possono per esempio pi\`u ricchi di colesterolo che favorisce il mantenimento in loco di proteine associate alla membrana che formano domini con funzioni specifiche di membrane.
Questi domini esistono e hanno un ruolo funzionale importante. Il doppio strato lipidico presenta delle differenze in termini di tipo di lipidi presenti: il movimento da un monolayer 
all'altro \`e accentuato grazie all'opera di molecole dette flippasi che favoriscono lo scambio lipidico. Si trova asimmetria di composizione lipidica tra i monolayer, in particolare
i glicolipidi, lipidi con zuccheri si trovano esclusivamente nella parte extracellulare e vengono prodotti in parte nell"ER e poi nell'apparato del Golgi e conferiscono deelle cariche
a un ambiente all'esterno della cellula che pu\`o essere riconosciuto da molecole che si legano agli zuccheri e mediano le interazioni tra una cellula e l'altra hanno ruolo di 
riconsciumento e interazione cellula cellula e contribuiscono un ambiente ionico diverso rispetto a quello interacellulare. La condizione asimmetrica varia in condizioni particolari:
ina cellula che va incntro ad apoptosi normalmente presenta la fosfatidilserina localizzata nel monolayer che guarda il citosol, quando va incontro ad apoptosi la fosfatidilserina viene
portata nel monolayer che guarda l'ambiente extracellulare che viene riconosciuta dai macrofagi, le cellule del sistema immunitario deputate alla fagocitosi e la presenza della fosfatitil
serina permette il reclutamento dei macrofagi. I glicolipidi si trovano esclusivamente nel monostrato verso l'esterno e costituiscno il $5\%$ delle molecole del monolayer esterno, si 
trovano tutte le membrane esterno, si trovano in alcuni casi nelle membrane intracellulare nell'ER e nll'apparato del golgi. Le pi\`u complessi sono i glangliosidi che contengono 
oligosaccaridi con uno o pi\`u redisdui di acido sialico. \`E una famiglia numerosa con pi\`u di 40 membri, sono abbondanti nella membrana plasmatica delle cellule nervose. Alcuni 
glicolipidi vengon utilizzati come porta d'ingresso per tossine batteriche e di virus. Come per la tossina del colera che aumento di cAMP e richiamo di acqua con squilibrio ionico e 
conseguente diarrea che porta in molti casi alla morte. La tossina del colera entra interagendo con il glanglioside Gm1 e viene portata all'interno della cellula dove va a bloccare 
l'attivit\`a dell'adenilato ciclasi. Le proteine di membrana vi sono alcune che di fatto sono presenti dei domini che attraversano la membrana plasmatica, normalmente ad $\alpha$ elica
e possono essere un unico o pi\`u dominio, mono e multipasso. Come nei recettori muscarinici multipasso con 7 domini di membrana. In altri casi ci sono proteine con domini di membrana
a struttura $\beta$ che formano cesti di strutture a barile come le pore forming protein nei batteri e poi delle proteine presenti nella membrana ma non la attraversano e associate
in maniera meno forte con essa e presentano un dominio idrofobico che ne permette interazioni, ci sono prorteine legate a membrana con legami specifici con i lipidi di membrana, 
categorie attraverso legame GPI con il glicosilfosfatidilinositolo, proteine extracellulari, inizialmente di membrana poi tagliate e si forma il legame con esso e pu\`o essere spezzato
da delle fosfolipasi e rilasciare la proteina nell'ambente extracitoplasmatico, ci sono proetine che interagiscono con la membrana attraverso proteine di membrna nell'ambiente
citoplasmatico o extracitoplasmatico. Per capire se una proteina ha un dominio idrofobico per capire se pu\`o interagire con la membrana plasmatica con uno strumento che identifica 
amminoacidi con una certa affinit\`a con il bistrato ed \`e possibile identificare i domini. Questo si misura con l'indice di idropatia, negativo parte citosolica, positivo domini di 
membrana. QUesto pu\`o essere utilizzato per identificare le regioni potenzialmente pi\`u antigeniche rispetto ad altre: per produrre un anticorpo contro una proteina si sceglie la
regione non inserita nella membrana. Si vedono regioni con scarsa interazioni con la membrana in quanto sono le parti pi\`u esposte e ottime per indurre la capacit\`a di formazione di 
anticorpi. L'aquaporina \`e una proteina di membrana che forma tramite strutture idrofobiche ad $\alpha$ elica delimita un canale per il passaggio delle molecole d'acqua che possono 
comunque passare in maniera diffusiva ma con permeabilit\`a estremamemnte bassa. La luce crea un gradiente di ioni H+ importante per la produzione di ATP grazie alla rodopsina, un 
meccanismo simile alla trasduzione del segnale visivo. I detergenti vengono utilizzati per dissociare proteine ancorate alla emmbrana e dissolbono la componente lipidica permettendo 
di isolare proteine con domini di membrana, come il sodio dodecil solfato, il triton X 100 e cos\`i via. Le proteine possono diffondere lungo tutta la cellula ma non avviene tramite 
a un traffickin intracellulare specifico che porta le proteine negli epiteli dalla parte apicale e quella basolaterale e non diffondono in quanto tra una cellula ci sono delle giunzioni
strette che impediscono la diffusione laterale. Sono giunzioni particolari che coinvolgono elementi del citoscheletro che creano una barriera che impedisce la diffusione delle proteine 
che permette alla cellula di acquisire una polarit\`a apicale con proteine con funzione particolare con altre proteine nella parte basolaterale. 
