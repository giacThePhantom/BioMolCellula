\chapter{Traduzione}
Si vede come l'informazione trascritta da DNA a RNA viene interpretata dall'apparato dedicato alla produzione di proteine. Innanzitutto i ribosomi sono le fabbriche traduttive, 
una componente importante nel processo traduzionale e sono quelle che svolgono il ruolo primario nella formazione del ruolo peptidico, l'efficienza \`e legata ad altri fattori che
garantiscono un efficiente traduzione. GLi attori principali sono i ribosomi. Sono costituiti da due subunit\`a, una maggiore L e minore S sia nei procarioti che nei procarioti. 
Differiscono per il peso molecolare. l'S \`e il coefficiente di sedimentazione \`e il coefficiente che d\`a attraverso un lisato e si fa un gradiente di densit\`a di saccarosio maggiore
sul fondo della provetta meno e si carica l'estrattto cellulare e si centrifuga le particelle tendono a sedimentare in maniera dipendente dalla loro massa e forma. Particelle molto grandi
e pesanti tendono ad andare verso il fondo. La velocit\`a di sedimentazione dipende dalla massa e dalla forma. Coeffciente S maggiore per il peso molecolare maggiore. S \`e una stima 
approssimativa del peso molecolare e non aumenta proporzionalmente alla massa in quanto la forma pu\`o dipendere dalla massa. Il completo per i procarioti \`e di 70S e le due subunit\`a
sono di 50S e di 30S. Gli eucarioti L \`e 60S S 40S e completo 80S. I ribosomi sono dei complessi riboproteici perch\`e contengono proteine e RNA sottoforma di RNA ribosomali in 
particolare la subunit\`a 50S contiene nei procarioti due specie di RNA il 5S e il 23S il tutto insieme e 34 proteine formano la subuniit\`a maggiore nei ribosomi, una delle strutture
pi\`u difficili da caratterizzare dal punto di vista strutturale. La subunit\`a minore nei procarioti contiene un RNA 16S e 21 proteine. Negli eucarioti ci sono tre RNA nella 60S il 5S, 
il 28S e il 5.8S con 49 proteine, mentre nella subunut\`a minore si trova la 40S con il 18S e 33 proteine tutti trascritti dalla polimerasi I tranne il 5s dalla III. Conengono un numero
variabile di ribosomi. Si tratta di componenti molto frequenti all'interno delle cellule. Uno degli RNA pi\`u presenti nelle cellule \`e l'rRNA. La maturazione e la produzione dei 
ribosomi aivene all'intenro del nucleolo nel nucleo il cui numero dipende dal tipo cellulare e dallo stato fisiologico. In attiva mitosi hanno forma e dimensione di nucleoli diverse. I
nucleoli sono strutture prive di membrana e sono costituiti da diversi centri; uno fibrillre F e da una zona pi\`u densa e una zona granulare. Il nucleolo contiene 500 proteine asssociate
ad esso eil $30\%$ ha funzione sconosciuta. La funzione in generale del nuclelo si pensa abbia sede di produzione dei ribosomi ma invece avviene il contatto tra mRNA e alcune RNA binding
proteins con diverse funzioni. Vi sono dei marcatori che vanno a marcare e localizzare nella componente fibrillare altre nella componente granulare. Gli rRNA vengono processati ad
eccezione del 5s vengono trascritti dall'RNA polimerasi I per generare un unico e lungo precursore come un gene policistronico che poi va incontro a processamento in cui viene tagliato
e si formano gli rRNA maturi 18S, 5.8S e 28S. GLi RNA ribosomiali hanno una struttura secondaria complessa importnta per la loro attivit\`a di catalizzare la formazione del legame 
peptidico. Le loro sequenze sono molto conservate e le basi che li costituiscono possono andare incontro a modifiche come i tRNA e altre classi di RNA. Questa struttura secondaria 
facilita la loro interaizone con le proteine ribosomiali. Alcune delle modifiche delle basi post trascrizionali come l'aggiunta di uno zolfo per la tio-uridina, metilazione. Le prorteine
ribosomiali sono caratterizzate come L o S in base alla subunit\`a cui appartengono. Alcune interagiscono direttamente con l'RNA e altre con altre proteine. Il core \`e costituito da RNA
e quelle che interagiscono con l'RNA e si trovano pi\`u all'interno mentre quelle con le proteine pi\`u all'esterno. Sono stati caratterizzato modificando la forza ionica e vedendo quali
proteine si staccavano, legate debolmente tra di loro e non interagiscono con RNA tendono a dissociarsi pi\`u velocemente. Si trtta di una struttura complessa. Non si trova un unico 
ribosoma sull'mRNA ma pi\`u ribosomi si attaccano formando i polisomi. \`E possibile caratterizzarli e isolare l'RNA a loro associato attraverso centrifugazione che mostra le frazioni
e le cellule vengono lisate e centrifugate su un gradiente e si isolano delle frazioni dall'alto verso il basso e si nota analizzando l'assorbanza dei picchi che corrispondono a 
subunit\`a questo \`e determinato attraverso un anticorpo, altri picchi corrispondono alla frfazione polisomale con ribosomi e mRNA in fase traduzionale. Questo permette di 
caratterizzare due linee cellulare in base a quali e quanti RNA vengono tradotti isolando l'RNA dalle frazioni polisomiali e capire quali e quanti sono presenti e fare paragoni con 
linee cellulari analoghe e valutare le differenze. I ribosomi si trovano liberi nel citoplasma che traducono gli mRNA che danno origine proteine ed enzimi la cui attivit\`a si svolge
all'interno della cellula e ci sono ribosomi associati all'ER rugoso delimitato da una membrana e si trovano ribosomi per la traduzione di proteine secrete o che vanno ancorate alla 
membrana. Oltre a questo si trovano ribosomi attorno ai mitocondri. La struttura dei ribosomi \`e una strututra complessa e particolare ed \`e tale che la subunit\`a maggiore e minore
delimitano tre camere, siti E, P e A. Il sito E per exit, P per peptidico e A per acceptor. Il motivo per le tre camere A \`e dove viene accettato il tRNA con l'amminoacido, P \`e il
sito dove avviene la formazione del legame peptidico e E \`e il punto dove il tRNA che ha donato l'amminoacido rimane vuoto e se ne va. 
\section{Biogenesi dei ribosomi}
La subunit\`a maggiore e minore si assemblano nel nucleolo prima di essere trasportate nel citoplasma. \`E un processo altamente coordinato che richiede la sintesi e modifica degli rRNA, 
il loro assemblamento con proteine, interazione transitoria con fattori non ribosomiali che sono sia RNA come gli snoRNA e proteine non ribosomiali che aiutano l'assemblamento dei 
ribosomi. L'assemblaggio intervengono delgi snoRNA che modificano gli rRNA e altre proteine e fattori non ribosomali che facilitano l'assemblamento dei pre ribosomi. All'inizio si 
forma un precursore 90S caratterizzato da proteine e RNA ribosomali e fattori non ribosomali e snoRNA e a questo punto il precursore si dissocia e si forma la subunit\`a pre 40S e la
subunit\`a pre 60S, la subunit\`a pre 40S lascia il nucleo come la pre 60S, vanno nel citoplasma dove i fattori non ribosomali si dissociano e si hanno le subunit\`a 40S e 60S mature. 
La sintesi dei ribosomi nei batteri \`e ingrado di produrre 20 amminoacidi al secondo mentre negli eucarioti solo 2. Nei procarioti si nota come un certo numero di proteine e RNA che
rendono la macchina complessa si sottolineano come la subunit\`a minore nei procarioti che negli eucarioti contiene degli RNA 16-18S che riconoscono sequenze specifiche all'interno 
dell'mRNA e permettono un inizio della sintesi corretto. Si \`e poi visto come vengono prodotti gli RNA ribosomali come un unico trascritto che va incontro a modifica dalla polimerasi
I trane il 5s che viene prodotto dalla III la composizione degli rRNA \`e molto conservata ed \`e importante per l'attivit\`a della formazione delle subunit\`a e per la formazione del
legame peptdico. La struttura del ribosoma \`e complessa e va a delineare quando unite le subunit\`a tre siti: sito A dove vengono accettati ed entrano i tRNA con l'amminoacido, il sito
P dove avviene il legame pepetidico tra la catena nascente e l'amminoacido portato dall'RNa e il sito E che \`e il sito di uscita. Contiene quattro siti di legame per l'RNA, uno per 
l'mRNA e tre per i tRNA. La struttura di un tRNA sono ipotizzati da Crick come molecole adattatrici costituiscono il 10-15 percento degli RNA un tRNA per ciascun amminoacido. Ci sono
20 tipi di tRNA si tratta di molecole dai 70 ai 95 nucleotidi e hanno una struttura a trifoglio legata alla presenza di sequenze compleementari che permettono alla molecola di andare 
incontro a folding. Vengono trascritti come precursori che vanno incontro a splicing particolari. La caratteriscita dei tRNA \`e che vanno incontro a modifiche post trascrizionali con 
l'aggiunta di gruppi specifici importanti per la sua funzione. Ci sono tre domini detti bracci, c'\`e quello dell'anticodone che forma una struttura alla paret derminale che porta la 
tripletta che riconosce la parte terminel, un braccio D e un braccio T$\psi$C per la presenza di pseudouridina, la D dalla diidrouridina. L'altro braccio con funzione importante \`e il 
braccio accettore che porta su cui avviene il legame con l'amminoacido corrispondente. Il braccio accettore \`e caratterizzato da tre nucleotidi CCA su cui avviene il legame specifico
su cui avviene il legame con l'amminoacido $tRNA^{leu}$ specifico per la leucina $LeutRNA^{leu}$ specifico per il leu con l'amminoacido legato. L'accoppiamento tra un tRNA specifico e 
l'amminoacido corrispondente avviene ad opera dell'amminacil tRNA sintetasi, un enzima che sintetizza il legame tra il tRNA e l'amminoacido. Contiene una serie di cavit\`a, una per il 
tRNA, uno per l'amminoacido e uno per l'ATP e un altro per il controllo del corretto accoppiamento. Il legame accoppiamento avviene tramite due step: nel primo l'amminoacido va incontro
e si lega all'ATP formando l'amminoacido adenilato con la formazione di pirofosfato. L'amminoacido adenilato poi nel secondo step catalitico comporta la rottura dell'amminoacido adenilato
e la fomrazione del legame con il $3'$ OH con la prima base del tRNA formando un legame ad alta energia con la liberazione di AMP. QUeste due fasi vengono catalizzate dall'amminoacil tRNA
sintetasi. L'accoppiamento corretto tra un tRNA e l'amminoacido, un meccanismo regolato deve esserci una perfetta corrispondenza e questo avviene in due modi: l'amminoacil tRNA 
sintetasi ha un sito per un amminoacido che se troppo grande non trova spazio. Per quanto riguarda amminoacidi con le stesse dimensioni esiste un altro sito detto sito di editing
in cui pu\`o trovare alloggiametno con dimensioni simili ma pi\`u piccolo, se trova alloggiamento nel sito avviene la scissione e rimosso il legame tra amminoacido e tRNA e l'errore viene
rimosso. Questo meccanismo permette il controllo qualitativo e che il determinato tRNA \`e stato unito all'amminoacido corretto. Questo effetto di correzzione causa un errore ogni 
10000 sintesi o caricamento di tRNA. Nei procarioti la prima differenza \`e che la sitensi proteica \`e quasi contemportanea alla trascrizione, negli eucarioti \`e successiva in quanto 
gli RNA vanno incontro a modifica. In entrambi i casi si distinguono tre fasi: una fase di inizio lenta in cui la macchina ribosomiale interagisce con l'RNA, il riconoscimento del 
corretto inizio della traduzione, poi una fase di allungamento che procede come la duplicazione in maniera processiva in cui il riboosma scorre e ingloba i tRNA dando origine alla
catena peptidica e infine si trova una fase di terminazione. Nei batteri il primo amminoacido che viene inserito \`e la metionina che viene modificata, formilata in cui un gruppo 
aldeidico \`e stato aggiunto al residuo amminico tramite alla metionil tRNA transformilasi. Il riconoscimento della tripletta AUG di inizio avviene grazie alla presenza dell'RNA 16S nella
subunit\`a minore che riconosce e si accoppia con la sequenz adi Shine-Dalgarno, Kozak negli eucarioti e si trova a monte della parte codificante. Il ribosoma si attacca, scannerizza il 
$5'$ fino a quando trova la sequenza di Shine-Dalgardno. A quel punto si ferma, si assembla la subnunit\`a maggiore e inizia la traduzione dell'RNA messaggero. La presenza di queste
due sequenze a monte del sito di inizio \`e estremamente importante. QUesto \`e stato visto utilizando un esperimento in cui lasciando ribosomi interagire con l'RNA, degradato quello
libero e quello protetto dal ribosoma e sequenziato si \`e trovata la sequenza conservata. I fattori di iniziazione del processo trduzionale per i procarioti sono IF e il processo di 
sintesi coinvolgi diversi fattori di inizio, coinvolti nel processo di elungazione e di fine, diversi fattori e tutti i quali vengono reclutati favorendo un processo traduzionale 
efficiente o per il controllo. Alcuni per i procarioti. Il complesso di inizio 30S \`e il complesso che si forma con il mRNA la subunit\`a minore, il tRNA con la metionina formilata
e GTP e IF2 pi\`u altri due complessi IF1 e IF3, \`e il complesso che si forma all'inizo del processo di traduzione ed \`e uno step critico in quanto i due fattori permettono alla
subunit\`a di non associarsi dalla subunit\`a maggiore e di accettare il tRNA che porta la metionina formilata. Il fattore IF1 e IF3 si dissociano e permette il reclutamento della
subunit\`a maggiore che reclutata causa l'idrolisi del GTP da parte di IF2 che viene rilasciata e si forma il complesso di inizio 70S pronto a dare inizio al processo traduzionale. Il 
processo di traduzione negli eucarioti inizia similmente anche se con step aggiuntivi e in molti casi sono legati al fatto che l'mRNA ha subito modifcihe con l'aggiunta di un capping
e la poliadenilazione, il capping legato e protetto da fattori che legano il cap rimossi al citoplasma a cui si aggiungono altri fattori coinvolti nel regoolare l'inizio della traduzione.
I fattori di inizio eucariotici o eIF anche negli eucarioti si forma un complesso di inizio 43S che \`e costituito dall'RNA, dalla subunit\`a minore, dal fattore eIF2 legato al GTP, 
il tRNA con la metionina e il fattore che si lega al cap, il eIF4G, ci sono diversi fattori che si legano al cap e fanno parte del complesso di inizio. Il meccanismo \`e simile e nel
caso degli eucarioti \`e reso complesso dalla presenza di fattori che si legano al cap e che vengono reclutati nel complesso iniziale. Una volta che si \`e formato questo inizia la 
scansione dell'RNA messaggero fino a quando trova la sequenza di Kozak in cui il complesso si ferma, si ha la rimozione dei fattori che impedivano l'unione delle subunit\`a maggiore e 
si forma il ribosoma completo e alla rimozione di altri fattori inizia la traduzione. La subunit\`a ribosomale maggiore e minore formano i siti E, P e A, il legame peptidico avivene
grazie al trasfermimento dell'amminoacido al tRNA proveniente dal sito accettore da quello presente nel sito P che porta la catena in fase di crescita. Questo \`e importante in quanto
il tRNA nel sito A non pu\`o uscire. Pertanto il tRNA in P \`e vuoto e a seguito del cambio conformazione della subunit\`a maggiore il tRNA si trova in E e pu\`o uscire. La
formazione del elgame peptidico \`e mediata dall'rRNA e comporta un cambio conformazionale che fa scorrere la subunit\`a maggiore di tre nucleotidi. Scorrendo la subunit\`a maggiore il 
tRNA che era in P si trova in E e quello che si trovava in A si trova in P. INterviene un altro fattore detto fattore di elungazione G che cambia e fa in modo che la subunit\`a minore
si sposti di tre nucleotidi andando a riformare i siti E P ed A. A questo punto il tRNA in E esce, il sito A \`e libero e pu\`o accettare un ulteriore tRNA con il corrispettivo 
ammioacido. Questo processo avivene grazie al reclutamento di fattori di allugamento EF come EF-Tu che \`e accoppiato ai tRAN, sono delle GTPasi che legano il GTP e lo idrolizzano, questo
fattore ha il ruolo di controllare che il tRNA entrato sia quello corretto, ovvero che l'anticodone trovi il corrisponente codone sull'mRNA. Se questo non avviene allora viene rimosso. 
Se invece il tRNA \`e corretto si ha la stabilizzazione del legame, l'idrolisi del GTP in GDP, il rilascio del fattore di elongazione, il tRNA rimane in loco e pu\`o essere utilizzato. 
A questo punto interviene EF=G che permette la traslocazione della subunit\`a minore. Grazie a questo fattore si forma il ribosoma completo e il ciclo continua fino a quando il ribosoma
trova una sequenza corrispondente a uno stop. (UAA, UAG e UGA) per cui non c'\`e un tRNA ma viene riconosciuta da una proteina detta fattore di rilascio i quali riconoscono queste 
triplette di stop, simile ad un RNA come struttura, caso di mimetismo molecolaer: una proteina che mima l'attivit\`a di un tRNA. UNa volta che si lega alla tripletta di stop catalizza 
l'idrolisi del legame tra tRNA e proteina in P. Si trova un canale nel ribosoma che impdeisce la formazione di strutture secondarie all'interno del ribosoma. All'interno del citoplasma
pu\`o cominciare ad assumere strutture secondarie e terziarie in maniera spontanee o guidata da chaperonine. Si pu\`o sfruttare un;IRES, internal ribosomal entry site: nel caso degli
eucarioti il riconoscimento al livello del $5'$ \`e un acondizione importante per iniziare in maniera corretta la traduzione. Ci sono sequenze IRES che quando introdotte permettono
la traduzione senza l'utilizzo kozak consenso o shine dalgarno. Cosa utilizzata dai virus per utilizzare l'apparato traduzionale della cellula a favore dei propri mRNA. Questa sequwnza
pu\`o essere utilizzata a scopi pratici per esprimere due RNA contemporaneamente basta inserire una sequenza IRES tra i due mRNA, uno tradotto dall'approccio canonico, l'altro grazie
al reclutamento della sequenza IRES. Le modifiche post traduzionali sono importante in quanto poliadenilazione e capping contribuiscono a rendere pi\`u efficienti la traduzione. La
regolazione a livello traduzionale pu\`o avvenire a diversi livelli: stabilit\`a dell'RNA, post-trascrizionale o si agisce sulla traduzione, sui fattori coinvolti nel processo 
traduzionale mediante fosforilazioni o modifiche che rendono il processo pi\`u o meno efficiente. 
