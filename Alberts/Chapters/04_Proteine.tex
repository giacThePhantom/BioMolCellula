\chapter{Proteine}
Le proteine costituiscono la maggior parte della massa secca della cellula. Svolgono le principali funzioni strutturali e la maggior parte delle funzioni
della cellula: gli enzimi mettono a disposizione le superfici che catalizzano le reazioni chimiche. Quelle incorporate nella membrana plasmatica formano 
canali e pompe per il controllo del passaggio di piccole molecole nella cellula, hanno funzione di messaggeri intra e inter cellulari, si occupano del
movimento della cellula, sono anticorpi, tossine, ormoni e hanno mille altre funzioni. 
\section{Forma e struttura delle proteine}
\subsection{La forma di una proteina \`e specificata dalla sua sequenza degli amminoacidi}
Esistono $20$ amminoacidi diversi che vengono codificati dal DNA di un organismo, ognuno con propriet\`a chimiche diverse. Una molecola proteica \`e
costituita da una lunga catena senza ramificazioni di questi amminoacidi, ognuno legato ai suoi vicini attraverso un legame peptidico covalente. Sono
pertanto dette polipeptidi. Ogni tipo di proteina possiede una sequenza di amminoacidi unica e ne esistono migliaia di tipi. La sequenza ripetuta di atomi 
lungo il nucleo della catena polipeptidica \`e detto il polypeptide backbone, al quale si attaccano quelle porzioni di amminoacidi non coinvolte nel 
legame peptidico e che conferiscono all'amminoacido le sue propriet\`a uniche. Le $20$ catene laterali differenti degli amminoacidi differiscono nelle
propriet\`a: alcune sono apolari e idrofobiche, altre negativamente o positivamente cariche, altre formano legami covalenti. Le proteine formano una 
catena flessibile che pu\`o ripiegarsi in infiniti modi. Tale struttura pu\`o essere determinata da molti legami non covalenti che si formano tra una 
parte della catena e l'altra e sono: legami idrogeno, attrazioni elettrostatiche e forze di Van der Waals che in parallelo possono mantenere insieme due 
regioni della catena. La forza di questo gran numero di legami non covalenti determina la stabilit\`a di ogni forma ripiegata. Pu\`o entrare anche in gioco
una forza apparente di attrazione idrofobica. La forma viene pertanto influenzata fortemente dalla distribuzione degli amminoacidi polari e non polari. I 
non polari tendono a trovarsi all'interno della molecola in modo da evitare il contatto con l'acqua, mentre i gruppo polari verso l'esterno. Amminoacidi
polari all'interno della proteina sono tipicamente legati con altri amminoacidi polari o al backbone polipeptidico attraverso legami a idrogeno.
\subsection{Le proteine si ripiegano nella conformazione a minore energia}
La maggior parte delle proteine possiedono una particolare struttura tridimensionale. Tale conformazione \`e quella che minimizza la sua energia libera. 
La sequenza di amminoacidi contiene tutte le informazioni necessarie alla struttura della proteina. La maggior parte si piegano in una singola 
conformazione stabile, che pu\`o variare leggermente quando interagiscono con altre molecole. Nelle cellule una proteina detta molecular chaperone assiste
il processo di piegatura legandosi a catene polipeptidiche parzialmente piegate e aiutandole verso il cammino pi\`u favorevole. Sono richiesti per 
prevenire la formazione di aggregati proteici attraverso regioni idrofobiche temporaneamente esposte. Rendono pertanto il processo pi\`u affidabile. La
maggior parte delle proteine si trovano ad una lunghezza da $50$ a $2000$ amminoacidi. Proteine grandi consistono di domini proteici, unit\`a strutturali 
che si piegano indipendentemente. 
\subsection{L'$\alpha$-elica e il $\beta$-foglietto sono pattern di piegamento comuni}
Analizzando la struttura tridimensionale delle proteine diventa chiaro come esistano due pattern di piegamento regolari. L'$\alpha$-elica e il 
$\beta$-foglietto e sono causati dal legame a idrogeno tra i gruppi \ce{N-H} e \ce{C=O} nel backbone polipeptidico, senza coinvolgere elementi della 
catene secondarie degli amminoacidi. Pertanto, nonostante siano incompatibili rispetto ad alcuni amminoacidi molte sequenze le possono formare. In 
entrambi i casi la proteina adotta una conformazione regolare e ripetuta. La parte centrale di molte proteine contiene regioni estese di 
$\beta$-foglietti che possono fermare segmenti vicini di backbone polipeptidico con lo stesso orientamento (catene parallele) o che si piega su s\`e 
stessa (catene anti-parallele). Un $\alpha$-elica viene generata quanto una singola catena polipeptidica si torce su s\`e stessa per formare un cilindro
rigido. Un legame a idrogeno si forma ogni quarto legame peptidico, legando il \ce{C=O} di un peptide con il \ce{N-H} di un altro. L'elica pertanto
compie un giro completo ogni $3.6$ amminoacidi. Questa conformazione \`e abbondante nelle proteine di membrana come quelle di trasporto e i recettori, 
specialmente nella parte che attraversa la stessa e composta da amminoacidi non polari. In altre proteine le $\alpha$-eliche si avvolgono su s\`e stesse
formando una bobina arrotolata, che si forma quando due o quattro $\alpha$-eliche hanno la maggior parte delle catene laterali non polari da una parte, in
modo che possano avvolgersi tra di loro. 
\subsection{I domini proteici sono unit\`a modulari da cui le proteine pi\`u grandi sono costruite}
Vengono distinti quattro livelli di organizzazione nella struttura di una proteina: la struttura primaria \`e la sequenza di amminoacidi, lunghezze di
catena polipeptidica che formano $\alpha$-eliche e $\beta$-foglietti sono la struttura secondaria, la completa organizzazione tridimensionale viene
detta struttura terziaria e se la proteina \`e costituita da un complesso di pi\`u catene polipeptidiche la struttura \`e detta quaternaria. Si intende
per dominio proteico una sotto-struttura prodotta da una qualsiasi parte contigua di catena polipeptidica che pu\`o piegarsi indipendentemente rispetto 
alle altre. Contengono solitamente tra i $40$ e i $350$ amminoacidi e sono le unit\`a modulari che costituiscono proteine pi\`u grandi. Diversi domini di
una proteina sono solitamente associati con diverse funzioni. Le proteine pi\`u piccole contengono un singolo dominio, mentre le pi\`u grandi anche a
dozzine, spesso connessi da corte e non strutturate lunghezze di catena polipeptidica che formano cardini flessibili tra i domini.
\subsection{Poche delle catene polipeptidiche possibili saranno utili alla cellula} 
Le possibili combinazioni di una catena polipeptidica di lunghezza $n$ sono $20^n$ e pertanto solo una piccola frazione di questo insieme crea una conformazione tridimensionale stabile, 
circa una in un miliardo. La selezione naturale ha portato la cellula ha scegliere quelle proteine che, oltre a possedere tale conformazione, possiedono propriet\`a chimiche finemente
regolate in modo da permettere alla proteina di catalizzare una particolare reazione o per svolgere la funzione strutturale richiesta. 
\subsection{Le proteine possono essere classificate in molte famiglie}
Una volta che la proteina si \`e evoluta per formare una conformazione stabile con un'utilit\`a pu\`o essere modificata attraverso meccanismi genetici in modo da creare nuove proteine
con diverse funzioni. Questo processo porta alla nascita di famiglie di proteine con sequenza e conformazione simili ma con funzione distinte. La struttura di diversi membri di una 
famiglia di proteine \`e conservata maggiormente rispetto alla sequenza di amminoacidi. Molti cambi di amminoacidi sono neutri, senza effetto sulla struttura e funzione della proteina.
Le proteine che subiscono cambi maligni vengono scartate durante il processo evolutivo. 
\subsection{Alcuni domini proteici si trovano in molte proteine diverse}
Le proteine formate da multipli domini si sono formate dall'unione accidentale di sequenze di DNA che codificano ogni dominio. Nel processo evolutivo del mescolamento del dominio, 
molte larghe proteine si sono evolute attraverso l'unione di domini preesistenti in nuove combinazione. Come risultato si sono create nuove superfici leganti alla giustapposizione dei
domini, dove si trovano molti dei siti funzionali della proteina. Un sottoinsieme di domini \`e stato molto mobile durante l'evoluzione, con strutture versatili dette moduli proteici. 
Alcuni domini possiedono un nucleo stabile formato da $\beta$-foglietti con anelli sporgenti di catena polipeptidica. Gli anelli sono situati per formare siti di legame per altre 
molecole. Il loro successo evolutivo \`e dovuto al fatto che mettono a disposizione una base per la generazione di siti di legame, richiedendo unicamente piccoli cambi agli anelli 
esterni. Si possono inoltre integrare facilmente all'interno di altre proteine in quanto possiedono alle terminazioni \ce{-N} e \ce{-C}. Quando il DNA codifica tale dominio svolge una
duplicazione a tandem. I domini duplicati con questo ordinamento in linea possono essere collegati per formare strutture estese con s\`e stessi o altri domini. Queste strutture estese
rigide sono comuni in matrici di molecole extracellulari e nella porzione extracellulare di proteine recettrici sulla superficie della cellula. Altri tipi di domini sono detti plug-in
con i legami \ce{-N} e \ce{-C} vicini. Dopo il riordinamento genetico sono messi come inserimenti in una regione ad anello di una sequenza proteina. La frequenza di utilizzo dei domini
differisce tra tipi di organismi. Il complesso maggiore di istocompatibilit\`a (MHC) che possiede un dominio di riconoscimento degli antigeni \`e presente unicamente negli umani, con 
funzioni specializzate e sono stati selezionati fortemente durante evoluzioni recenti. Molte coppie di domini si trovano insieme in molte proteine: la maggior parte delle proteine che
contengono coppie di due domini si sono sviluppate relativamente tardi durante l'evoluzione.
\subsection{Il genoma umano codifica un complesso insieme di proteine, rivelando che molto rimane sconosciuto}
Il sequenziamento del genoma umano ha rivelato che contiene circa $21000$ geni che codificano proteine e che i vertebrati hanno ereditato la maggior parte delle proteine dagli 
invertebrati, nonostante in media ogni proteina sia pi\`u complessa. Il mescolamento dei domini durante l'evoluzione ha portato alla creazione di molte nuove combinazioni di domini. La
maggior variet\`a delle proteine permette pi\`u possible interazioni proteina-proteina.
\subsection{Proteine pi\`u grandi contengono pi\`u di una catena polipeptidica}
Gli stessi legami non covalenti che permettono il piegamento della proteina le permettono di legarsi con altre proteine per formare strutture pi\`u grandi nella cellula. Ogni regione di
una superficie di una proteina che pu\`o interagire con altre molecole si dice sito di legame. Una proteina ne pu\`o contenere diversi. Se tale sito riconosce la superficie di una 
seconda proteina il legame tra le due catene polipeptidiche crea una proteina pi\`u larga con una geometria definita. Ogni catena polipeptidica in tale proteina \`e detta subunit\`a 
proteica. Nel caso pi\`u semplice due catene polipeptidiche con la stessa conformazione possono legarsi testa-a-testa, formando un complesso simmetrico di due subunit\`a mantenuto 
dall'interazione tra due siti di legame identici. Molte proteine contengono due o pi\`u tipi di catene polipeptidiche.
\subsection{Alcune proteine globulari formano lunghi filamenti elicoidali}
Una proteina si dice globulare se la catena polipeptidica si ripiega su s\`e stessa formando una forma compatta simile ad una palla con una superficie irregolare. Alcune di queste 
proteine si possono combinare formando lunghi filamenti se ogni molecola possiede un sito di legame complementare ad un'altra regione sulla superficie della tessa molecola. Essendo che
ogni subunit\`a si lega alle altre allo stesso modo e il legame non \`e mai una retta la struttura complessiva assumer\`a una forma ad elica. 
\subsection{Molte proteine hanno forme allungate e fibrose}
Gli enzimi tendono ad essere proteine globulari: nonostante molte sono larghe e complicate con multiple subunit\`a, la maggior parte hanno una forma arrotondata. Ci sono funzioni che
richiedono che ogni molecola proteica occupi una lunga distanza. Queste proteine possiedono generalmente una struttura semplice allungata e sono dette proteine fibrose. Il citoscheletro
\`e formato da forme chiamate filamenti intermedi simili a corde. Sono abbondanti all'esterno della cellula, dove formano la maggior parte della struttura del gel della matrice
extracellulare che aiuta  a legare collezioni di cellule insieme per formare tessuti. Le proteine di matrice sono secrete dalle cellule e si assemblano in fibrilli lunghi.
\subsection{Le proteine contengono una grande quantit\`a di catene polipeptidiche intrinsecamente disordinate}
Un'altra molecola abbondante nella matrice proteica \`e l'elastina, un polipeptide altamente disordinato, il cui disordine \`e fondamentale per la sua funzione di produrre una mesh che
pu\`o essere spostata da una conformazione all'altra. Hanno funzioni importanti nei siti di legame, prendendo una forma solo quando incontrano la molecola che legano. Una funzione 
predominante di queste parti \`e per l'appunto formare siti di legami con altre proteine ad alta specificit\`a ma alterati da fosforilazione o defosforilazione o modifiche iniziate da
eventi di segnale. Vengono utilizzate come legame per mantenere due domini proteici in prossimit\`a in modo da permettere al substrato di muoversi tra i siti attivi in un complesso 
multi-enzima. Creano micro-regioni con una consistenza simile a gel che limita la diffusione. 
\subsection{Legami incrociati covalenti stabilizzano proteine extracellulari}
Molte proteine si trovano all'esterno della membrana plasmatica della cellula o sono secrete come matrice extracellulare. Tutte queste sono esposte alle condizioni dell'ambiente. Per 
mantenere la loro struttura la catena polipeptidica \`e stabilizzata da legami covalenti incrociati che possono legare due amminoacidi nella stessa proteina o legare differenti catene
polipeptidiche. Il legame pi\`u comune \`e quello zolfo-zolfo o legami disolfuro che si formano quando la cella prepara le proteine per l'esportazione. La loro formazione \`e catalizzata
nel reticolo endoplasmatico da un enzima che lega due paia di gruppi \ce{-SH} di cisteina adiacenti nella proteina piegata. Non cambiano la conformazione della proteina ma si limitano
a rafforzarne la struttura. Tali legami non riescono a formarsi nel citosol a causa del gran numero di agenti riduttori.
\subsection{Le molecole proteiche spesso svolgono la funzione di subunit\`a per l'assemblaggio di grandi strutture}
Lo stesso principio che permette alle proteine di associarsi tra di loro formando anelli o filamenti operano per generare larghe strutture formate da un insieme di diverse macromolecole
come complessi enzimatici, ribosomi, virus e membrane che non sono composti da una singola molecola formata da legami covalenti, ma dall'assemblaggio di molte molecole attraverso legami
non covalenti che formano le subunit\`a della struttura. Questo processo presenta dei vantaggi:
\begin{itemize}
	\item Una grande struttura costruita da subunit\`a ripetute richiede meno informazione genetica.
	\item Sia l'assemblaggio che il deassemblamento  possono essere controllati e resi reversibili in quanto i legami di associazione hanno poca energia.
	\item Gli errori nella sintesi della struttura sono evitati pi\`u facilmente. 
\end{itemize}
Alcune subunit\`a proteiche si assemblano in larghi fogli piatti in cui altre subunit\`a sono ordinate in pattern esagonali. In alcuni casi proteine di membrana si originano come un
bistrato lipidico. Il foglio esagonale pu\`o essere facilmente trasformato in un tubo o una sfera cava che si lega a molecole di DNA e RNA nell'involucro dei virus. La formazione
di strutture chiuse come anelli, tubi o sfere crea maggiore stabilit\`a in quanto aumenta il numero di legami tra le subunit\`a ed essendo creata da legami mutualmente dipendenti e 
cooperativi pu\`o essere assemblata o disassemblata da piccoli cambi di una subunit\`a. Molte strutture nella cellula sono capaci di auto-assemblaggio. L'informazione per la creazione di
molte delle macromolecole \`e contenuta nelle subunit\`a stesse. 
\subsection{I fattori di assemblaggio aiutano la formazione di complesse strutture biologiche}
Nei casi in cui le molecole non si possano formare spontaneamente da una soluzione dei componenti sono necessarie informazioni supplementari fornite da enzimi e altre proteine che 
svolgono la funzione di stampi, ovvero fattori di assemblaggio che guidano la costruzione ma non fanno parte della struttura finale. 
\subsection{Fibrille amiloidi possono formarsi da molte proteine}
Una classe di strutture proteiche, utilizzata per delle funzioni della cellula pu\`o contribuire a certe malattie quando non controllata. Sono $\beta$-foglietti auto-riproducenti chiamati
fibrille amiloide. Si costruiscono da una serie di catene polipeptidiche identiche che si stratificano una sull'altra creando uno stack di $\beta$-foglietti orientati perpendicolarmente
all'asse formando un filamento cross-beta. Una porzione della proteina possiede la capacit\`a di formare tali strutture in quanto possiede differenti sequenze e seguono diversi cammini. 
In condizioni normali il meccanismo che controlla le proteine si degrada con l'et\`a, permettendo occasionalmente alle proteine di formare aggregati patologici che possono essere 
rilasciati dalla cellula morta e aggregarsi all'esterno come fibrille amiloide nella matrice extracellulare che in casi estremi possono uccidere altre cellule e danneggiare i tessuti.
Queste molecole possono anche essere utilizzate dalla cellula per concentrare in granuli secretori molecole da espellere. 
\subsection{Molte proteine contengono domini a bassa complessit\`a che possono formare amiloidi reversibili}
Un grande insieme di domini a bassa complessit\`a possono formare fibre amiloidi che hanno ruoli funzionali nel nucleo e nel citoplasma, sono normalmente senza struttura e consistono
di lunghezze di sequenze di amminoacidi lunghe centinaia di monomeri con un sottoinsieme dei $20$ amminoacidi. Queste strutture formate sono unite da legami non covalenti deboli e si
dissociano facilmente in presenza dei segnali corretti. Molte proteine con tale dominio contengono un altro insieme di domini che si lega specificamente ad altre proteine o al RNA, 
pertanto la loro aggregazione controllata pu\`o formare un idrogel che unisce queste e altre molecole in strutture puntate chiamate corpi intracellulari o granuli. mRNA specifico pu\`o
essere inviato in questi granuli dove \`e salvato fino a che reso disponibile da un disassemblaggio controllato del nucleo formato dalla struttura amiloide. 
\section{Funzione delle proteine}
Le proteine spesso possiedono parti mobili la cui azione meccanica \`e accoppiata ad un evento chimico che le dona le capacit\`a che sottostanno i processi dinamici della cellula. 
\subsection{Tutte le proteine si legano ad altre molecole}
Le interazioni fisiche di una proteina con altre molecole determinano le sue propriet\`a biologiche. Tutte le proteine si legano con altre molecole con legami che possono essere stretti
o deboli e temporanei. Il legame ha un alto livello di specificit\`a con una molecola detta ligando. Tale abilit\`a dipende dalla capacit\`a della proteina di formare un insieme di 
legami non covalenti deboli e interazioni idrofobiche che essendo deboli sono effettivi unicamente quando si formano simultaneamente e pertanto possibili solo se la superficie del 
ligando si adatta alla proteina. Il sito che si associa con il ligando, detto sito di legame consiste di una cavit\`a nella superficie costituito da un particolare ordinamento di 
amminoacidi che appartengono a diverse parti della catena che vengono avvicinate dal piegamento. Regioni separate forniscono siti di legame per diversi ligandi permettendo la regolazione
dell'attivit\`a della proteina. Gli atomi che si trovano all'interno della proteina contribuiscono alla forma della superficie e alle propriet\`a fisico-chimiche. 
\subsection{La conformazione superficiale di una proteina determina la sua chimica}
Le capacit\`a chimiche delle proteine richiedono che i gruppi chimici nella superficie interagiscono in modo che aumentino la reattivit\`a chimica di una o pi\`u catene laterali di 
amminoacidi. Queste interazioni sono divisibili in due categorie. L'interazione con diverse parti vicine della catena possono impedire l'accesso di molecole d'acqua al sito di legame in
quanto possono formare legami a idrogeno che possono competere con i ligandi per i siti. Questo avviene in quanto l'acqua tende a formare legami con altre molecole d'acqua pertanto 
risulta sfavorevole che una singola molecola si separi dalla rete creata. Il raggrupparsi di catene laterali polari pu\`o alterare la loro reattivit\`a: se un gran numero di catene
laterale cariche negativamente vengono forzate insieme la loro affinit\`a per le cariche positive aumenta di molto e quando amminoacidi interagiscono tra di loro gruppi solitamente 
non reattivi come \ce{-CH2OH} possono diventare reattivi permettendo la creazione o rottura dei legami covalenti selezionati. La reattivit\`a della superficie proteica non dipende 
pertanto unicamente dalla sequenza di amminoacidi ma anche dal loro orientamento relativo. 
\subsection{La comparazione di sotto-sequenze tra membri di una famiglia proteica evidenzia i siti di legame cruciali}
Il sequenziamento genomico permette di raggruppare molti domini proteici in famiglie che mostrano la loro evoluzione da un antenato comune. I membri di una famiglia sono simili. Si 
pu\`o utilizzare il tracciamento evolutivo per identificare i siti in un dominio proteico che sono fondamentali alla sua funzione. I siti che legano altre molecole sono i pi\`u 
probabili ad essere mantenuti invariati e pertanto gli amminoacidi che non sono cambiati in tutti membri della famiglia sono mappati su un modello della struttura tridimensionale di 
un membro della famiglia e le posizioni invarianti formano dei cluster sulla superficie della proteina. 
\subsection{Le proteine si legano con altre proteine attraverso molti tipi di interfacce}
Le proteine possono legarsi tra di loro in molti modi. In molti casi una porzione della superficie di una proteina entra in contatto con un anello esteso di una catena polipeptidica con 
una seconda proteina (interazione superficie-stringa). Un secondo tipo di interfaccia si forma quando due $\alpha$-eliche da due proteine si accoppiano formando una coiled-coil, che
si trova spesso in famiglie regolatrici dei geni. Il modo pi\`u comune di interazione \`e attraverso la corrispondenza precisa tra due superfici con un'interazione molto stretta data dal
gran numero d legami che si formano e molto specifica. 
\subsection{I siti di legame degli anticorpi sono estremamente versatili}
Tutte le proteine si devono legare a particolari ligandi per compiere la propria funzione. Gli anticorpi o immunoglobuline sono proteine prodotte dal sistema immunitario in risposta
a molecole esterne. Ogni anticorpo si lega strettamente con una molecola obiettivo particolare disattivandola o marcandola per la distruzione. L'anticorpo deve riconoscere l'obiettivo
o antigene con grande specificit\`a. Sono molecole a forma di Y con due siti di legame identici complementari a una piccola parte della superficie dei una molecola antigenica. Sono
formati da molti anelli di catena polipeptidica che protrudono dalla fine di un paio di domini proteici giustapposti. Solo cambiando la lunghezza e gli amminoacidi di questi anelli sono
in grado di generare diversi siti di legame per diversi antigeni senza cambiare la struttura di base. 
\subsection{La costante di equilibrio misura la forza di legame}
Le molecole che collidono con superfici che si accoppiano male formano pochi legami deboli e si dissociano rapidamente mentre se ne formano troppi l'associazione pu\`o persistere
nel tempo. Associazioni forti accadono quando una funzione biologica richiede che le molecole rimangano associate per molto tempo. Eventualmente ogni complesso di ligandi e proteine
arriva ad un equilibrio in cui i tassi di distruzione e associazione delle molecole si eguaglia. Si pu\`o pertanto calcolare la costante di equilibrio $K$ per misurare la forza 
dell'associazione ed \`e anche una misura diretta della differenza di energia libera tra lo stato legato e libero. 
\subsection{Gli enzimi sono catalizzatori potenti e molto specifici}
Molte proteine possono svolgere la loro funzione semplicemente legandosi con altre molecole, ma per alcune questo \`e solo un necessario primo passo, come per gli enzimi. Gli enzimi sono
molecole che causano le trasformazioni chimiche che creano e rompono legami covalenti nelle cellule. Legano dei ligandi detti substrati e li convertono in prodotti con grande rapidit\`a.
Gli enzimi velocizzano le reazioni di molti ordini di grandezza senza modificarsi: agiscono da catalizzatori. \`E infatti la catalisi di un insieme di reazioni chimiche da parte degli
enzimi che permette alla cellula di rimanere in vita. Gli enzimi si possono raggruppare in classi funzionali che possono svolgere reazioni chimiche simili. Ogni tipo di enzima 
all'interno della classe \`e altamente specifico e catalizza un singolo tipo di reazione. Lavorano in gruppo, con il prodotto di un enzima che diventa il substrato per un altro in una
rete elaborata di cammini che fornisce alla cellula energia e materiali.
\subsection{Il legame con il substrato \`e il primo passo nella catalisi enzimatica}
Per un enzima il legame di ogni substrato alla proteina \`e un inizio necessario. Denotando un enzima $E$, il substrato $S$ e il prodotto $P$ la reazione pi\`u semplice \`e:
$$\ce{E + S -> ES -> EP -> E + P}$$
C'\`e un limite a quanto substrato un enzima pi\`o processare nel tempo, nonostante un aumento della concentrazione di enzima questa raggiunge un massimo in cui la molecola di enzima 
\`e saturata con il substrato e il tasso della reazione dipende su quanto rapidamente l'enzima pu\`o processarlo. Questo tasso massimo \`e detto numero di turnover, solitamente di 
$1000$ molecole di substrato al secondo per molecola di enzima. L'altro parametro cinetico \`e $K_m$, la concentrazione del substrato che permette la reazione di procedere a met\`a 
del tasso massimo. Un valore di $K_m$ basso vuol dire che l'enzima raggiunge il tasso catalitico massimo ad una bassa concentrazione di substrato e che si legano molto strettamente. 
\subsection{Gli enzimi velocizzano la reazione stabilizzando selettivamente stati transitivi}
Gli enzimi raggiungono tassi di reazioni elevatissimi grazie a molte ragioni. In primo luogo quando due molecole devono reagire l'enzima aumenta la concentrazione locale di queste 
molecole al sito catalitico, mantenendole nell'orientazione corretta. Oltre a questo dell'energia di legame contribuisce alla catalisi. Le molecole di substrato devono passare degli 
stati intermedi di geometria e distribuzione di elettroni alterata prima di formare il prodotto finale. L'energia libera necessaria per ottenere lo stato pi\`u instabile o stato di
transizione \`e detta energia di attivazione ed \`e il determinante maggiore per il tasso di reazione. Gli enzimi possiedono un maggiore grado di affinit\`a per lo stato di transizione
del substrato rispetto alla forma stabile. Siccome il legame stretto riduce l'energia dello stato di transizione, l'enzima accelera la reazione abbassando l'energia di attivazione 
richiesta.
\subsection{Gli enzimi possono usare simultaneamente catalizzatori acidi e basici}
Gli enzimi non solo si legano strettamente ad uno stato di transizione ma contengono atomi posizionati in modo da alterare la distribuzione degli elettroni che partecipano nella 
creazione e rottura del legame covalente. I legami polipeptidici possono essere idrolizzati nell'assenza di un enzima esponendoli a un acido o base forte. Sono gli unici a poter usare 
catalisi acide o basiche simultaneamente in quanto la struttura rigida impedisce che si combinino. L'adattamento stretto tra enzima e substrato deve essere preciso: un piccolo cambio, 
anche di $1\AA$ pu\`o ridurre di l'attivit\`a dell'enzima di un migliaio di volte.
\subsection{Il lisozima illustra come un enzima funziona}
Per dimostrare come gli enzimi catalizzano le reazioni chimiche si analizza un enzima antibiotico che si trova in secrezioni come saliva e lacrime. Il lisozima catalizza il taglio di 
catene polisaccaridi nella parete cellulare dei batteri. Il lisozima catalizza un idrolisi: aggiunge una molecola d'acqua ad un singolo legame nella catena causandone la rottura. La 
rottura \`e energeticamente favorevole in quanto l'energia richiesta per la rottura del legame \`e minore di quella della catena intatta, ma vi si trova una barriera energetica e la
molecola d'acqua che collide solo se il polisaccaride \`e distorto nello stato di transizione, stato irraggiungibile in condizioni normali. Quando il polisaccaride si lega al lisozima 
il sito attivo dell'enzima \`e una lunga fessura che lega sei zuccheri legati alla volta. Appena il polisaccaride si lega forma un complesso e l'enzima lo taglia aggiungendo una 
molecola d'acqua lungo uno dei legami zucchero-zucchero. Le catene prodotto sono rilasciate, lasciando libero l'enzima per una successiva interazione. Il tasso di idrolisi aumenta 
perch\`e le condizioni causate nel sito attivo riducono l'energia di attivazione necessaria distorcendo uno degli zuccheri. Il legame da rompere \`e tenuto vicino da due amminoacidi
acidi (acido glutammico e aspartico) che partecipano nella reazione. Nelle reazioni che coinvolgono due o pi\`u reagenti il sito attivo agisce come stampo che unisce i substrati nella
corretta orientazione. Passi successivi nella reazione portano le catene laterali nello stato originale permettendo il riutilizzo dell'enzima.
\subsection{Molecole strettamente legate aggiungono altre funzioni alle proteine}
Ci sono molte istanze in cui \`e necessario aggiungere altre molecole alle proteine affinch\`e svolgano la loro funzione. Tali molecole si trovano nei siti attivi e sono dette 
coenzimi che sono spesso vitamine o loro derivati. In alcuni casi possono formare legami covalenti che le rendono parte della proteina stessa.
\subsection{Complessi multi-enzima aiutano ad aumentare il tasso del metabolismo cellulare}
L'efficienza degli enzimi \`e cruciale per il mantenimento della vita in quanto il tasso di reazioni desiderabili deve essere maggiore del tasso delle reazioni avverse che avvengono 
naturalmente. Si pu\`o misurare il tasso del metabolismo attraverso il consumo di ATP. Una cellula mammifera consuma la propria capacit\`a di ATP ogni paio di minuti, ovvero circa 
$10^7$ molecole al secondo. Tale velocit\`a \`e dovuta all'efficienza degli enzimi, alcuni hanno raggiunto la massima velocit\`a possibile e sono limitati unicamente dalla frequenza
delle collisioni (reazione limitata dalla diffusione). La quantit\`a del prodotto di un enzima dipende dalla sua concentrazione e quella del substrato. Se una sequenza deve avvenire 
rapidamente deve essere presente ogni intermedio metabolico ed enzima in grande concentrazione. Ci sono limiti alla concentrazione massima. La maggior parte dei metaboliti sono presenti
in concentrazioni micromolari e gli enzimi con concentrazioni ancora pi\`u basse. Il tasso metabolico viene mantenuto pertanto grazie all'organizzazione spaziale della cellula che
pu\`o aumentare il tasso della reazione spostando enzimi formando grandi proteine dette complessi multi-enzima. Essendo organizzato in modo che il prodotto di un enzima sia passato 
direttamente all'enzima successivo la concentrazione del substrato non deve essere limitante. La maggior parte degli enzimi inoltre hanno evoluto siti di legame con particolari regioni
della cellula. Anche i sistemi di membrane intracellulari sono utilizzati a questo scopo in quanto possono segregare particolari substrati e i relativi enzimi. 
\subsection{La cellula regola le attivit\`a catalitiche dei propri enzimi}
Una cellula vivente contiene migliaia di enzimi, molti dei quali operano contemporaneamente e vicini tra di loro, generando una complessa rete di cammini metabolici composti da catene
di reazioni chimiche. In questi cammini ci sono molti punti in cui diversi enzimi competono per lo stesso substrato. Sono pertanto necessari controlli per regolare quando e quanto 
rapidamente ogni reazione accade. Questa regolazione avviene a molti livelli: viene controllato il numero di enzimi prodotti regolando l'espressione genica. Gli enzimi vengono inoltre
confinati in compartimenti subcellulari come membrane intracellulari o scaffold proteici (o concentrandoli su essi). Gli enzimi possono anche essere modificati covalente mente per 
disattivarli. Il tasso della distruzione delle proteine \`e un altro metodo di regolazione, ma il principale \`e il cambio di attivit\`a derivato dal legame con una molecola particolare.
Quest'ultimo accade quando un enzima si lega con una molecola (non substrato) a un sito regolatorio speciale, alterando il tasso con cui l'enzima converte il substrato in prodotto. Nel
processo di feedback inhibition un prodotto successivo del cammino chimico inibisce un enzima precedente, pertanto quando la reazione produce grandi quantit\`a di prodotto rallenta da 
sola. Quando i cammini si intersecano o si ramificano ci sono diversi punti di controllo alla fine del cammino che controllano la propria sintesi. Quando la concentrazione di prodotto
diminuisce il processo ritorna a velocit\`a normali. Un altro tipo di regolazione \`e positiva: il prodotto stimola l'azione dell'enzima, questo accade principalmente quando un prodotto
di un ramo della rete chimica stimola l'attivit\`a di un enzima in un altro cammino. 
\subsection{Gli enzimi allosterici possiedono due o pi\`u siti di legame che interagiscono}
Si nota nelle regolazioni a feedback che la molecola regolatoria possiede una forma diversa rispetto al substrato o all'enzima. Per questo l'effetto sull'enzima \`e detto allosteria. 
Gli enzimi che partecipano in questo processo possiedono due siti di legame sulla loro superficie: uno attivo per il legame con il substrato e uno regolatorio per il legame con la
molecola regolatoria. Questi due siti comunicano in modo da influenzare gli eventi catalitici. L'interazione \`e detta cambio conformazionale della proteina: il legame in uno dei siti
attivi cambia leggermente la forma della proteina stessa: durante l'inibizione quando la molecola regolatoria si lega cambia la forma del sito attivo in capacitandolo. La maggior parte
delle proteine sono allosteriche e possono adottare due o pi\`u conformazioni differenti, cambiando tra esse in base a quale ligando si lega. Oltre agli enzimi sono allosterici i 
recettori, le proteine strutturali e quelle motrici. 
\subsection{Due ligandi i cui siti di legame sono accoppiati devono avere effetto reciprocamente sul legame dell'altro}
Gli effetti di un legame del ligando segue un principio chimico detto linkage: se una proteina che lega il glucosio lega un'altra molecola su un sito attivo distante, se tale sito
attivo cambia forma a causa del cambio di conformazione causato dal glucosio i due siti sono detti accoppiati. Quando due ligandi preferiscono il legame con la stessa conformazione di
una proteina allosterica ogni ligando deve aumentare l'affinit\`a dell'altro per la proteina. In maniera inversa il linkage opera negativamente quando i due ligandi si vogliono legare
a conformazioni diverse della molecola e devono competere per il legame. 
\subsection{Insiemi di proteine simmetrici producono transizioni allosteriche cooperative}
Un singolo feedback negativo non basta per l'ottimale regolazione della cellula e pertanto la maggior parte degli enzimi che partecipano nel feedback sono insiemi simmetrici di 
subunit\`a identiche. Con questo ordinamento il legame di una molecola ad un ligando su un sito su una subunit\`a promuove un cambio allosterico nell'intero insieme (i legami con i 
ligandi delle altre subunit\`a avviene pi\`u facilmente). Avviene pertanto una transizione allosterica cooperativa che permette un cambio nella concentrazione dei ligandi per cambiare 
l'insieme da uno stato attivo a uno inattivo. Questo principio \`e comune a proteine che non sono enzimi. 
\subsection{Molti cambi nelle proteine sono guidati dalla fosforilazione proteica}
Le proteine sono regolate anche dall'addizione di un gruppo fosfato. Un evento di fosforilazione pu\`o influenzare la proteina: pu\`o cambiare la conformazione attraendo un insieme
di amminoacidi carichi positivamente e cambiando la forma del sito attivo. Quando un secondo enzima rimuove il gruppo fosfato la proteina ritorna alla conformazione originale. In 
secondo luogo il gruppo fosfato pu\`o formare parte di una struttura che altre proteine riconoscono. Infine la sua addizione pu\`o mascherare un sito di legame che tiene unite 
due proteine, rompendo il legame. La fosforilazione proteica reversibile controlla l'attivit\`a, struttura, e localizzazione di enzimi e altre proteine. 
\subsection{Una cellula eucariotica contiene un gran numero di proteina chinasi e proteina fosfatasi}
La fosforilazione di una proteina coinvolge il trasferimento catalizzato dalla proteina chinasi del gruppo fosfato terminale di una molecola di ATP al gruppo idrossile di un amminoacido 
serina, treonina o tirosina. Tale reazione \`e tipicamente unidirezionale a causa della grande quantit\`a di energia libera rilasciata quando viene rotto il legame fosfato-fosfato 
nell'ATP. La proteina fosfatasi catalizza l'azione inversa di rimozione del gruppo fosfato o defosforilazione. Esistono un gran numero di questi enzimi, ognuno responsabile di 
poche proteine o specifici substrati  da subunit\`a regolatori. Lo stato di fosforilazione dipende dall'attivit\`a relativa della proteina chinasi e fosfatasi che la modificano.
Questi enzimi fanno parte di una grande famiglia di enzimi che condivide una sequenza catalitica di circa $290$ amminoacidi. Le differenze permettono la specificit\`a. 
\subsection{La regolazione della proteina chinasi Src rivela come una proteina pu\`o funzionare come un microprocessore}
Le migliaia di diverse proteine chinasi sono organizzate in una rete complessa di cammini segnalatori che aiutano a coordinare le attivit\`a della cellula, guidare il ciclo cellulare e 
ritrasmettere segnali nella cellula dall'esterno. Proteine chinasi individuali svolgono il ruolo di dispositivi di input/output nel processo di integrazione. Una parte importante 
dell'input a queste proteine di processamento dei segnali viene dal controllo eseguito dai fosfati aggiunti e rimossi da essi dalle proteine chinasi e fosfatasi. La famiglia Src delle
proteine chinasi hanno questo comportamento e contengono una corta regione \ce{-N} terminale che si lega covalentemente a un acido grasso fortemente idrofobico che la ancora alla
faccia citoplasmatica della membrana plasmatica. Dopo il gruppo terminale si trovano due domini \ce{SH}3 e \ce{SH}2 seguiti da il dominio catalitico della chinasi. Normalmente esistono 
nella versione inattiva, ma transizionano in quella attiva quando viene rimosso il fosfato \ce{-C} terminale al legame del dominio \ce{SH}3 da una proteina di attivazione. Quando accade
l'attivazione segnala il completamento di un insieme di eventi. 
\subsection{Le proteine che si legano e regolano il GTP sono onnipresenti regolatori}
Il controllo dell'attivit\`a di una cellula attraverso addizione e rimozione di un gruppo fosfato avviene anche quando non viene direttamente attaccato a essa ma \`e parte del nucleotide
guanina GTP che si legame molto strettamente con un gruppo di proteine dette GTP-leganti. Queste proteine si trovano nella conformazione attiva quando formano il legame con il GTP. La
perdita del gruppo fosfato avviene quando il legame GTP \`e idrolizzato in GDP da una reazione catalizzata dalla proteina stessa. Tali proteine, dette anche GTPasi costituiscono una 
grande famiglia di proteine con variazioni dello stesso dominio globulare dove avviene il legame con il GTP. Quando il GTP legato si idrolizza il dominio cambia conformazione e disattiva
la proteina.
\subsection{Le proteine regolatorie GAP e GEF controllano l'ativit\`a delle proteine GTP-leganti determinando se \`e legato GTP o GDP}
Le proteine GTP-leganti sono controllate da proteine regolatorie che determinano se a esse \`e legato GTP o GDP. Sono disattivate da una proteina GTPasi-attivante che si lega a essa e 
idrolizza il GTP legato in GDP che rimane legato e fosfato inorganico che viene rilasciato. La proteina rimane in forma inattiva fino a che non incontra un fattore di scambio di 
nucleotide guanina (GEF) che si lega alla proteina con il GDP e causa il suo rilascio. Il sito \`e immediatamente riempito da una molecola di GDP, presenti in eccesso. Il GEF attiva la
proteina aggiungendo il gruppo fosfato indirettamente. 
\subsection{Le proteine possono essere regolate dall'aggiunta covalente di altre proteine}
Le cellule contengono una famiglia speciale di piccole proteine i cui membri sono attaccati a molte altre proteine in modo da determinarne l'attivit\`a. In ogni caso la loro estremit\`a
carbossilica si lega all'amminoacido lisina di una proteina obiettivo attraverso un legame isopeptidico. Questa addizione porta ad un cambio di conformazione reversibile (attraverso
reazioni catalizzate da enzimi). La proteina pi\`u comune di questo tipo \`e la ubiquitina che si lega alle proteine in vari modi, ognuno con un significato diverso. Pu\`o formare
catene di poliubiquitina con diverse modifiche alle funzioni della proteina che legano.
\subsection{Un elaborato sistema di coniugazione di ubiquitina \`e utilizzato per marcare le proteine}
Per selezionare le proteine target per l'addizione di ubiquitina si deve attivare la terminazione carbossilica di quest'ultima attraversi un enzima ubiquitina-attivante che utilizza 
l'idrolisi dell'ATP per ricavare l'energia necessaria ad attaccare l'ubiquitina con s\`e stessa attraverso un legame covalente ad alta energia. La proteina passa a enzimi di coniugazione
di ubiquitina che la uniscono con le ubiquitina ligasi che si legano a speciali segnali di degradazione detti degrons nei substrati aiutando il secondo insieme di enzimi a formare una
catena di poliubiquitina collegata ad una lisina. La catena sulla proteina obiettivo viene riconosciuta da un particolare recettore nel proteosoma, causano la distruzione di essa. 
\subsection{I complessi proteici con parti interscambiabili fanno un efficiente uso dell'informazione genica}
Il SCF ubiquitina ligasi \`e un complesso proteico che lega proteine obiettivo aggiungendogli una catena di poliubiquitina. Ha una struttura a forma di C composta da cinque subunit\`a
proteiche, la pi\`u grande serve come impalcatura per le altre. Ad un'estremit\`a di trova il secondo enzima del processo di cungiunzione di ubiquitina, all'altra parte un braccio
che lega il substrato o proteina F-box. Quando il complesso \`e attivato F-box lega il sito attivo alla proteina obiettivo e la sposta nel centro in modo che sue lisine entrino in 
contatto con la poliubiquitina. Catalizza poi una ripetuta addizione di polipeptidi ubiquitina alle lisine marcando la proteina per distruzione rapida nel proteosoma.
Specifiche proteine sono marcate per distruzione in risposta a segnali specifici, aiutando a guidare il ciclo della cellula. Il tempismo della distruzione richiede la creazione di 
specifici pattern di fosforilazione della proteina richiesto per il riconoscimento dalla subunit\`a F-box e l'attivazione dell'ubiquitina ligasi SCF. Molti dei bracci che legano il 
substrato sono interscambiabili nel complesso proteico e richiedono pi\`u di $70$ geni. Esistono molti tipi di proteine F-box e una famiglia di SCF-like ubiquitina ligasi. Una proteina
con parti interscambiabili fa uso economico dell'informazione genetica nella cellula.
\subsection{Una proteina GTP-legante mostra come grandi movimenti proteici siano possibili}
La proteina EF-Tu fornisce un buon esempio di come cambi allosterici nella conformazione della proteina producono movimenti amplificando un cambio conformazionale locale. Tale proteina
\`e un fattore di allungamento nella sintesi proteica, caricando ogni amminoacil-tRNA sul ribosoma in quanto quest'ultimo forma un forte complesso con la forma legata al GTP. Questa 
molecola di tRNA pu\`o trasferire il suo amminoacido alla catena polipeptidica crescente solo dopo che il GTP \`e idrolizzato. Essendo che tale reazione \`e fatta partire da un giusto
adattamento alla molecola di mRNA nel ribosoma EF-Tu serve come fattore discriminante tra coppie errate di mRNA e tRNA. Confrontando la conformazione della proteina nelle forme legate
a GTP e GDP so osserva come il riposizionamento del tRNA avviene. La dissociazione del gruppo fosfato causa un cambio nel sito che lega il GTP che causa una propagazione di cambio di 
movimento lungo un $\alpha$-elica detta switch helix che serve come aggancio che aderisce a un sito specifico in un altro dominio della molecola, mantenendola in una conformazione 
chiusa. Il cambio conformazionale causa la separazione della switch helix permettendo a domini separati di aprirsi rilasciando la molecola di tRNA permettendo l'utilizzo 
dell'amminoacido. 
\subsection{Le proteine motrici producono grandi movimenti nelle cellule}
Lo scopo principale delle proteine motrici \`e muovere altre molecole. Sono responsabili della contrazione muscolare e dei movimenti delle cellule, oltre a permettere movimenti 
intracellulari come il movimento dei cromosomi in parte opposte della cellula durante la mitosi, per muovere gli organelli e gli enzimi con filamenti di DNA durante la sintesi del DNA. 
Senza nessuna guida questi movimenti sono reversibili e la proteina si sposterebbe casualmente lungo un filamento. Per rendere i cambi conformazionali, e pertanto i movimenti, 
unidirezionali si deve accoppiare il movimento con l'idrolisi di molecole di ATP legate alla proteina. Essendo che l'idrolisi dell'ATP rilascia molta energia \`e molto improbabile che
avvenga il movimento inverso. Il legame di ATP sposta la proteina in una conformazione e l'idrolizzazione cambia la conformazione e il rilascio dell'ADP la cambia in una terza. L'energia
rilasciata dall'idrolisi rende i cambi conformazionali irreversibili e il ciclo va in una direzione, spostando sempre la proteina lungo la stessa direzione. 
\subsection{I trasportatori di membrana imbrigliano energia per pompare le molecole attraverso le membrane}
Le proteine allosteriche possono imbrigliare l'energia derivata dall'idrolisi dell'ATP, dal gradiente ionico o dal trasporto di elettroni. I trasportatori ACO costituiscono una classe
di proteine pompe legate alla membrana. La loro funzione principale \`e trasportare molecole idrofobiche dal citoplasma, rimuovendo molecole tossiche. Tali trasportatori contengono un
paio di subunit\`a che attraversano la membrana legate a una coppia di subunit\`a che legano ATP sotto la membrana plasmatica. L'idrolisi dell'ATP guida un cambio conformazionale
trasmettendo una forza che causa il movimento da parte delle subunit\`a legate alla membrana della molecola legata attraverso il doppio strato lipidico. Alcune di queste proteine 
sono pompe rotanti che accoppiano l'idrolisi dell'ATP con il trasporto di ioni \ce{H+} e sono utilizzate per acidificare l'interno del lisozoma. Possono funzionare al contrario 
catalizzando la reazione di sintesi dell'ATP se il gradiente \`e abbastanza ripido. 
\subsection{Le proteine formano grandi complessi che funzionano come macchinari proteici}
Grandi proteine possono svolgere funzioni elaborate. La maggior parte delle reazioni sono catalizzate da un insieme altamente coordinato e collegato di queste macchine proteiche che
utilizzano una reazione energeticamente favorevole come l'idrolizzazione dell'ATP per causare una serie di cambi conformazionali che causano il movimento coordinato dell'insieme. 
\subsection{Impalcature concentrano insiemi di proteine che interagiscono}
Le proteine sono localizzate in specifici siti della cellula e sono assemblate e attivate solo quando necessario. Questo meccanismo coinvolge impalcature proteiche, proteine con 
multipli siti di legame con altre proteine che servono per unire proteine interagenti e per posizionarle in parti specifiche della cellula. 
\subsection{Molte proteine sono controllate da modifiche covalenti che le dirigono verso siti specifici all'interno della cellula}
Un gran numero di proteine sono modificate su pi\`u amminoacidi con eventi regolatori producendo pattern diversi. Queste modifiche covalenti possono essere considerate come un codice
regolatorio combinatorio: gruppi modificanti specifici sono aggiunti o rimossi in risposta a segnali cambiando l'attivit\`a e la stabilit\`a della proteina, pertanto la molecola \`e in
grado di rispondere velocemente e con grande versatilit\`a a cambi nella sua condizione o ambiente.
\subsection{Una complessa rete di interazioni proteiche sottosta alle funzioni della cellula}
Per comprendere le funzioni di ogni complesso proteico si deve ricostituirlo dalle parti proteiche purificate in modo da poter studiare dettagliatamente i modi di operazioni in condizioni
controllate. Esistono un gran numero di interazioni proteina-proteina e possono essere studiate attraverso mappe di iterazione tra proteine, utili per identificare la funzione probabile 
di proteine non caratterizzate in base alla posizione relativa alle altre. Queste reti devono essere analizzate con cura perch\`e la stessa proteina pu\`o essere utilizzata in complessi
diversi e avere diverse funzioni. Nei confronti incrociati le proteine con pattern di interazione simile probabilmente hanno la tessa funzione della cellula. 
